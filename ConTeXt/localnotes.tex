\startitemgroup[noteitems]
\item\subnoteref{x0}\NoteKeywordAgamaHead{「無所攀緣,亦無所住(\ccchref{SA.1267}{https://agama.buddhason.org/SA/dm.php?keyword=1267});無可攀挽、無安足處(GA)」},南傳作\NoteKeywordNikaya{「無住立、無用力地」}(Appatiṭṭhaṃ anāyūhaṃ),菩提比丘長老英譯為\NoteKeywordBhikkhuBodhi{「以不停止行進,與以不使勁」}(By not halting…and by not straining)。
\stopitemgroup

\startitemgroup[noteitems]
\item\subnoteref{x1}\NoteKeywordNikayaHead{「解脫、已解脫、遠離」}(nimokkhaṃ pamokkhaṃ vivekanti),菩提比丘長老英譯為\NoteKeywordBhikkhuBodhi{「脫離、解開、隔離」}(emancipation, release, seclusion)。按:「已解脫」(pamokkha),水野弘元《巴利語辭典》解為「能被解脫的」(pamuñcati的未來被動分詞),PTS英巴辭典解為「解脫」(release, deliverance),但《顯揚真義》說,眾生以道從污染束縛分離,因此道被稱為眾生的「解脫」(nimokkhoti),而在果剎那時(Phalakkhaṇe),他們從污染束縛解脫,因此果被稱為眾生的「已解脫」(pamokkhaṃ),到達涅槃後眾生遠離一切苦,因此涅槃被稱為眾生的「遠離」(vivekoti),今準此譯。
\stopitemgroup

\startitemgroup[noteitems]
\item\subnoteref{x2}\NoteKeywordNikayaHead{「以有之歡喜的遍盡」}(Nandībhavaparikkhayā),菩提比丘長老英譯為\NoteKeywordBhikkhuBodhi{「以對存在歡樂的被完全破壞」}(By the utter destruction of delight in existence),並解說,這是滅盡對三(界)有的渴愛者。
\stopitemgroup

\startitemgroup[noteitems]
\item\subnoteref{x3}\NoteKeywordNikayaHead{「種種年齡」}(vayoguṇā,逐字譯為「年齡+種類」),菩提比丘長老英譯為\NoteKeywordBhikkhuBodhi{「生命的階段」}(The stages of life)。《顯揚真義》以「初中後年齡的種類」(paṭhamamajjhimapacchimavayānaṃ guṇā)解說。「次第地拋棄」則以「它們次第地拋棄人(anupaṭipāṭiyā puggalaṃ jahanti),因為在立於中間年齡時,初年齡拋棄[人](Majjhimavaye ṭhitaṃ hi paṭhamavayo jahati),在立於後年齡時,中初二者拋棄[人],死亡剎那,三種年齡拋棄[人]」解說。
\stopitemgroup

\startitemgroup[noteitems]
\item\subnoteref{x4}\NoteKeywordAgamaHead{「斷除五捨五(\ccchref{SA.1002}{https://agama.buddhason.org/SA/dm.php?keyword=1002});斷五捨於五(\ccchref{SA.1312}{https://agama.buddhason.org/SA/dm.php?keyword=1312})」},南傳作\NoteKeywordNikaya{「該切斷五,應該捨斷五」}(Pañca chinde pañca jahe),菩提比丘長老英譯為\NoteKeywordBhikkhuBodhi{「一個人必須切斷五,捨斷五」}(One must cut off five, abandon five)。按:《顯揚真義》說,切斷五指應該切斷五下分結(身見、疑、戒禁取、貪、瞋);捨斷五指應該捨斷五上分結(色貪、無色貪、慢、掉舉、無明)。
\stopitemgroup

\startitemgroup[noteitems]
\item\subnoteref{x5}\NoteKeywordAgamaHead{「增修於五根(\ccchref{SA.1002}{https://agama.buddhason.org/SA/dm.php?keyword=1002});五法上增修(\ccchref{SA.1312}{https://agama.buddhason.org/SA/dm.php?keyword=1312});增上修五根(GA)」},南傳作\NoteKeywordNikaya{「且更應該修習五」}(pañca cuttari bhāvaye),菩提比丘長老英譯為\NoteKeywordBhikkhuBodhi{「且必須更進一步開發五」}(And must develop a further five)。按:《顯揚真義》說,應該更修習五指應該修習[信、活力(精進)、念、定、慧]五根。
\stopitemgroup

\startitemgroup[noteitems]
\item\subnoteref{x6}\NoteKeywordAgamaHead{「超越五和合(\ccchref{SA.1002}{https://agama.buddhason.org/SA/dm.php?keyword=1002});超五種積聚(\ccchref{SA.1312}{https://agama.buddhason.org/SA/dm.php?keyword=1312});成就五分法/成就五分身(GA)」},南傳作\NoteKeywordNikaya{「超越五染著」}(Pañca saṅgātigo,另譯為「打勝五染著」),菩提比丘長老英譯為\NoteKeywordBhikkhuBodhi{「已克服五羈絆」}(has surmounted five ties)。按:《顯揚真義》說,五染著指貪染著、瞋染著、癡染著、慢染著(mānasaṅgo)、見染著(diṭṭhisaṅgoti)。
\stopitemgroup

\startitemgroup[noteitems]
\item\subnoteref{x7}\NoteKeywordAgamaHead{「五人(\ccchref{SA.1003}{https://agama.buddhason.org/SA/dm.php?keyword=1003});誰(GA.141)」},南傳作\NoteKeywordNikaya{「五」},《顯揚真義》解說:「當信等五根醒時,名為五蓋是睡的,為什麼呢?因為具備這些的人,任何時候,即使坐、立、黎明起來,放逸、具備不善是睡的。這樣,當五蓋睡時,五根是醒的,為什麼呢?因為具備這些的人,任何時候,即使躺下睡覺,不放逸、具備善是醒的。五蓋拿起、握持、執取雜染塵垢,欲的意欲等前者以後者為緣,被五根淨化。」
\stopitemgroup

\startitemgroup[noteitems]
\item\subnoteref{x8}\suttaref{SN.4.8}作「魔波旬」。
\stopitemgroup

\startitemgroup[noteitems]
\item\subnoteref{x9}\NoteKeywordAgamaHead{「比丘覆惡覺(GA.174)」},南傳作\NoteKeywordNikaya{「比丘在意之尋中收存著」}(samodahaṃ bhikkhu manovitakke),菩提比丘長老英譯為\NoteKeywordBhikkhuBodhi{「縮在心的心思中」}(Drawing in the mind's thoughts)。按:《顯揚真義》以「從意生起的諸尋」(manamhi uppannavitakke)解說「意之尋」。
\stopitemgroup

\startitemgroup[noteitems]
\item\subnoteref{x10}\NoteKeywordAgamaHead{「其心無所依(\ccchref{SA.600}{https://agama.buddhason.org/SA/dm.php?keyword=600});無所依(GA)」},南傳作\NoteKeywordNikaya{「不依止的」}(Anissito),菩提比丘長老英譯為\NoteKeywordBhikkhuBodhi{「獨立的」}(Independent)。按:《顯揚真義》說,由渴愛、[邪]見的依止成為不依止的(Anissitoti taṇhādiṭṭhinissayehi anissito hutvā.)。
\stopitemgroup

\startitemgroup[noteitems]
\item\subnoteref{x11}\NoteKeywordNikayaHead{「時間我不知道」}(Kālaṃ vohaṃ na jānāmi),菩提比丘長老英譯為\NoteKeywordBhikkhuBodhi{「我不知道時間會是什麼」}(I do not know what the time might be)。按:《顯揚真義》以「降下」(nipātamattaṃ)解說這裡的vo,即「時間落下」,「被隱藏的時間」指死亡的時間(maraṇakālaṃ-死時),「不要時間離開我」指作沙門法的事之時間(samaṇadhammakaraṇakālaṃ)。
\stopitemgroup

\startitemgroup[noteitems]
\item\subnoteref{x12}\NoteKeywordAgamaHead{「非時樂(\ccchref{SA.1078}{https://agama.buddhason.org/SA/dm.php?keyword=1078})」},南傳作\NoteKeywordNikaya{「時間的」}(kālikaṃ),菩提比丘長老英譯為\NoteKeywordBhikkhuBodhi{「那些費時間的」}(what takes time),坦尼沙羅比丘長老英譯為「屬於時間的」(what's subject to time),Sujato比丘長老英譯為「隨時間生效的」(what takes effect over time)。《顯揚真義》說,多數未見真理、未離貪、不知他人心的天神(aparacittavidūniyo devatā)以為修遍純淨梵行十年、二十年……六十年的比丘們是天界欲上的希求者,「因此,在直接可見的人間欲中而作在天時間的後,說這個(Tasmā mānusake kāme sandiṭṭhike, dibbe ca kālike katvā evamāha)。」
\stopitemgroup

\startitemgroup[noteitems]
\item\subnoteref{x13}\NoteKeywordNikayaHead{「能被講述的有想者眾生」}(Akkheyyasaññino sattā),菩提比丘長老英譯為\NoteKeywordBhikkhuBodhi{「覺知所有能被表示的眾生」}(Beings who perceive what can be expressed)。按:《顯揚真義》以五蘊被稱為能被講述的。長老解說,語言所及的客觀範圍,當一般人認知五蘊,其想法受常、樂、我的影響,稱為顛倒,這些扭曲的想法接著挑起污穢。
\stopitemgroup

\startitemgroup[noteitems]
\item\subnoteref{x14}\NoteKeywordAgamaHead{「則為死方便(\ccchref{SA.1078}{https://agama.buddhason.org/SA/dm.php?keyword=1078});是名屬死徑(GA)」},南傳作\NoteKeywordNikaya{「被死神束縛」}(yogamāyanti maccuno),菩提比丘長老英譯為\NoteKeywordBhikkhuBodhi{「他們進入死亡之軛(枷鎖)下」}(They come under the yoke of Death),並解說,這是指在生死中反覆,因此保持在被輪迴的時間網捕捉中。
\stopitemgroup

\startitemgroup[noteitems]
\item\subnoteref{x15}\NoteKeywordAgamaHead{「慢(\ccchref{SA.1078}{https://agama.buddhason.org/SA/dm.php?keyword=1078});三種慢(GA.17)」},南傳作\NoteKeywordNikaya{「勝慢」}(vimānam),菩提比丘長老英譯為\NoteKeywordBhikkhuBodhi{「自大」}(conceit)。按:《顯揚真義》以實與不實之勝、等、劣等「九類三種慢」(navabhedaṃ tividhamānaṃ),或「母親子宮之住處」(Nivāsaṭṭhena vā mātukucchi)解說,但長老認為,vimānam應為ca mānam(慢)之訛誤。
\stopitemgroup

\startitemgroup[noteitems]
\item\subnoteref{x16}\NoteKeywordAgamaHead{「無觸不報觸(\ccchref{SA.1275}{https://agama.buddhason.org/SA/dm.php?keyword=1275});不觸者勿觸(GA)」},南傳作\NoteKeywordNikaya{「不接觸沒接觸者」}(Nāphusantaṃ phusati),菩提比丘長老英譯為\NoteKeywordBhikkhuBodhi{「它不接觸沒接觸者」}(It does not touch one who does not touch)。《顯揚真義》解說,這裡的「接觸」指「業果報的接觸」或「業的接觸(業被做)」(大意)。
\stopitemgroup

\startitemgroup[noteitems]
\item\subnoteref{x17}\NoteKeywordAgamaHead{「不瞋不招瞋(\ccchref{SA.1275}{https://agama.buddhason.org/SA/dm.php?keyword=1275});不應妄有觸(GA)」},南傳作\NoteKeywordNikaya{「[對]無犯錯者的有過失者」}(appaduṭṭhapadosinanti,逐字譯為「無犯錯者+有過失者」),菩提比丘長老英譯為\NoteKeywordBhikkhuBodhi{「冤屈一個無辜者的人」}(The one who wrongs an innocent man)。
\stopitemgroup

\startitemgroup[noteitems]
\item\subnoteref{x18}\NoteKeywordAgamaHead{「外纏結(\ccchref{SA.599}{https://agama.buddhason.org/SA/dm.php?keyword=599});外結髮(GA)」},南傳作\NoteKeywordNikaya{「外結縛」}(bahi jaṭā),菩提比丘長老英譯為\NoteKeywordBhikkhuBodhi{「外在的纏結」}(a tangle outside)。按:「結縛」,《顯揚真義》以「渴愛網(taṇhāya jāliniyā)的同義語」解說, 「外」指自身以外的。
\stopitemgroup

\startitemgroup[noteitems]
\item\subnoteref{x19}\NoteKeywordAgamaHead{「內心修智慧(\ccchref{SA.599}{https://agama.buddhason.org/SA/dm.php?keyword=599});修心及智慧(GA)」},南傳作\NoteKeywordNikaya{「心與慧的修習者」}(cittaṃ paññañca bhāvayaṃ),菩提比丘長老英譯為\NoteKeywordBhikkhuBodhi{「開發心與慧」}(Developing the mind and wisdom)。按《顯揚真義》以「定與毘婆舍那修習者」(samādhiñceva vipassanañca bhāvayamāno)解說。長老說,這裡的「心」相當於「定」,這裡經文所說是「戒、定、慧」三學。
\stopitemgroup

\startitemgroup[noteitems]
\item\subnoteref{x20}\NoteKeywordNikayaHead{「有對與色想」}(Paṭighaṃ rūpasaññā),菩提比丘長老英譯為\NoteKeywordBhikkhuBodhi{「以及色的撞擊與認知」}(And also impingement and perception of form)。按:《顯揚真義》說,捉取者因有對想而為欲有,因色想而為色有,在這兩種捉取者上以有的集積(bhavasaṅkhepenāti)捉取者為無色有。長老說,其譯法乃依據註疏偈誦押韻的看法,並解說「色的撞擊」是指五根的所緣對五根的撞擊,「撞擊想」則引《清淨道論》第十章的內容(按:即色想、聲想、氣味想、味道想、所觸想的有對想),「色想」的範圍很廣,包括禪定中的地遍處等,結論還是回到《顯揚真義》的看法,意指三界眾生。
\stopitemgroup

\startitemgroup[noteitems]
\item\subnoteref{x21}\NoteKeywordAgamaHead{「平等假名說(\ccchref{SA.581}{https://agama.buddhason.org/SA/dm.php?keyword=581});隨順世俗故(GA)」},南傳作\NoteKeywordNikaya{「他會只以慣用語的程度說」}(Vohāramattena so vohareyyā),菩提比丘長老英譯為\NoteKeywordBhikkhuBodhi{「他使用這樣的詞僅當作表達」}(He uses such terms as mere expressions)。參看\ccchref{MN.74}{https://agama.buddhason.org/MN/dm.php?keyword=74}「以及凡在世間中說的以那個無執取地說」,\ccchref{DN.9}{https://agama.buddhason.org/DN/dm.php?keyword=9}「如來以那些無執取地說」。
\stopitemgroup

\startitemgroup[noteitems]
\item\subnoteref{x22}\NoteKeywordAgamaHead{「賴吒槃提國(\ccchref{SA.589}{https://agama.buddhason.org/SA/dm.php?keyword=589});羅吒國(GA)」},南傳作\NoteKeywordNikaya{「擁有國家的」}(raṭṭhavantopi,逐字譯為「國+有的」),菩提比丘長老英譯為\NoteKeywordBhikkhuBodhi{「統治國家者」}(who rule the country)。按:「賴吒槃提國;羅吒國」,應該是「擁有國家的」(raṭṭhavanto)音譯。
\stopitemgroup

\startitemgroup[noteitems]
\item\subnoteref{x23}\NoteKeywordAgamaHead{「如是競勝心(\ccchref{SA.589}{https://agama.buddhason.org/SA/dm.php?keyword=589});為財產鬥諍(GA)」},南傳作\NoteKeywordNikaya{「在那些生起貪欲者中」}(Tesu ussukkajātesu),菩提比丘長老英譯為\NoteKeywordBhikkhuBodhi{「在那些成為那樣貪心者中」}(Among these who have become so avid),並解說ussukka有正面與負面的意思,正面的意思他譯為「熱心的;熱中的;狂熱的」(zealous),負面的意思他就譯為「貪心的」(avid),水野弘元《巴利語辭典》譯為「熱心、努力」(正面意思),形容詞ussuka譯為「熱心的、貪欲的、渴求的」(正負面意思)。
\stopitemgroup

\startitemgroup[noteitems]
\item\subnoteref{x24}\NoteKeywordNikayaHead{「在有之流隨行者中」}(bhavasotānusārisu),菩提比丘長老英譯為\NoteKeywordBhikkhuBodhi{「在存在之流中隨著流動著」}(Flowing along in the stream of existence)。按:「有之流」(bhavasota)的「有」(bhava),即十二緣起中的「有支」,《顯揚真義》以「輪迴之流」(vaṭṭasotaṃ)解說。
\stopitemgroup

\startitemgroup[noteitems]
\item\subnoteref{x25}\NoteKeywordAgamaHead{「伊尼耶鹿𨄔(\ccchref{SA.602}{https://agama.buddhason.org/SA/dm.php?keyword=602});伊尼延(GA)」},南傳作\NoteKeywordNikaya{「如鹿小腿」}(Eṇijaṅghaṃ,三十二相之一),菩提比丘長老英譯為\NoteKeywordBhikkhuBodhi{「具麋鹿小腿」}(with antelope calves)。按:「鹿」(eṇi,另譯為「麋鹿、羚羊」),「伊尼耶;伊尼延」應為音譯,「小腿」(jaṅgha)即𨄔。
\stopitemgroup

\startitemgroup[noteitems]
\item\subnoteref{x26}\NoteKeywordAgamaHead{「心法說第六(\ccchref{SA.602}{https://agama.buddhason.org/SA/dm.php?keyword=602});及說第六意(\ccchref{SA.1329}{https://agama.buddhason.org/SA/dm.php?keyword=1329});意第六顯現(GA)」},南傳作\NoteKeywordNikaya{「意被宣說為第六」}(manochaṭṭhā paveditā),菩提比丘長老英譯為\NoteKeywordBhikkhuBodhi{「以心被宣說是第六」}(With mind declared to be the sixth)。按:《顯揚真義》說,以五種欲為色,以意為名(manena nāmaṃ),以兩者為五蘊之意。另參看《小部/經集1品9(\ccchref{SA.1329}{https://agama.buddhason.org/SA/dm.php?keyword=1329})。
\stopitemgroup

\startitemgroup[noteitems]
\item\subnoteref{x27}\NoteKeywordAgamaHead{「乃能行其惠(\ccchref{SA.1288}{https://agama.buddhason.org/SA/dm.php?keyword=1288})」},南傳作\NoteKeywordNikaya{「是施物的了知者」}(deyyaṃ hoti vijānatā”ti),菩提比丘長老英譯為\NoteKeywordBhikkhuBodhi{「當然應該給與贈與」}(Should surely give a gift)。按:《顯揚真義》說,「有布施的果報」就應該被知道者給與(atthi dānassa phalanti jānantena dātabbamevāti)。
\stopitemgroup

\startitemgroup[noteitems]
\item\subnoteref{x28}\NoteKeywordAgamaHead{「死則不隨死(\ccchref{SA.1288}{https://agama.buddhason.org/SA/dm.php?keyword=1288})」},南傳作\NoteKeywordNikaya{「他們在死者中不死」}(Te matesu na mīyanti),菩提比丘長老英譯為\NoteKeywordBhikkhuBodhi{「他們在死者中不死」}(They do not die among the dead)。按:《顯揚真義》說,布施德行(dānasīlā)為在死者中不死之意。
\stopitemgroup

\startitemgroup[noteitems]
\item\subnoteref{x29}\NoteKeywordAgamaHead{「百千邪盛會(\ccchref{SA.1288}{https://agama.buddhason.org/SA/dm.php?keyword=1288});設百千大祀(GA)」},南傳作\NoteKeywordNikaya{「百千個的供千犧牲者」}(Sataṃ sahassānaṃ sahassayāginaṃ),菩提比丘長老英譯為\NoteKeywordBhikkhuBodhi{「則獻祭一千者的一個十萬的奉獻」}(Then a hundred thousand offerings Of those who sacrifice a thousand)。按:百千即十萬。
\stopitemgroup

\startitemgroup[noteitems]
\item\subnoteref{x30}\NoteKeywordAgamaHead{「十六不及一(\ccchref{SA.1288}{https://agama.buddhason.org/SA/dm.php?keyword=1288});十六分中一(GA)」},南傳作\NoteKeywordNikaya{「十六分之一」}(Kalampi,另譯為「一小部分」),菩提比丘長老英譯為\NoteKeywordBhikkhuBodhi{「一小部分」}(a fraction)。按:《顯揚真義》以「十六分之一,或百分之一,或千分之一,這裡取百分之一(Idha satabhāgo gahito)」解說。
\stopitemgroup

\startitemgroup[noteitems]
\item\subnoteref{x31}\NoteKeywordNikayaHead{「是施物的了知者」}同\suttaref{SN.1.32}。
\stopitemgroup

\startitemgroup[noteitems]
\item\subnoteref{x32}\NoteKeywordNikayaHead{「簡別後的布施」}(viceyya dānampi,另譯為「審察施」),菩提比丘長老英譯為\NoteKeywordBhikkhuBodhi{「區別地給予」}(Giving discriminately)。按:《顯揚真義》以「簡別後被給與的布施」解說,並說,有兩種選擇:(i)施物的選擇(dakkhiṇāvicinanaṃ):排除低劣的,選擇勝妙的;(ii)應該被供養者的選擇(dakkhiṇeyyavicinanañca):捨棄品德壞失者,施與在教說中出家之戒等德性具足者(sīlādiguṇasampannānaṃ)。
\stopitemgroup

\startitemgroup[noteitems]
\item\subnoteref{x33}\NoteKeywordNikayaHead{「美的」}(citrāni),菩提比丘長老英譯為\NoteKeywordBhikkhuBodhi{「漂亮的東西」}(the pretty things, SN/AN)。按:《顯揚真義》以「美的所緣」(ārammaṇacittāni, \suttaref{SN.1.34})解說,《滿足希求》以「美的種種所緣」(citravicitrārammaṇāni, \ccchref{AN.6.63}{https://agama.buddhason.org/AN/an.php?keyword=6.63})解說。
\stopitemgroup

\startitemgroup[noteitems]
\item\subnoteref{x34}\NoteKeywordNikayaHead{「貪的意向」}(Saṅkapparāgo),菩提比丘長老英譯為\NoteKeywordBhikkhuBodhi{「慾望的意向;好色的意向」}(the intention of lust/lustful intention, SN/AN)。按:《顯揚真義》以「已意圖的貪」(saṅkappitarāgo, \suttaref{SN.1.34})解說,《滿足希求》以「因意向而生起的貪」(saṅkappavasena uppannarāgo, \ccchref{AN.6.63}{https://agama.buddhason.org/AN/an.php?keyword=6.63})解說。又,此句這裡依\suttaref{SN.1.34}區隔在偈誦外,這似乎比原標在偈誦內的五句好。
\stopitemgroup

\startitemgroup[noteitems]
\item\subnoteref{x35}\NoteKeywordNikayaHead{「無所有者」}(akiñcanaṃ),菩提比丘長老英譯為\NoteKeywordBhikkhuBodhi{「什麼都沒有者」}(one who has nothing),並解說,「無所有者」通常指阿羅漢,《顯揚真義》說,這是指沒有貪、瞋、癡。
\stopitemgroup

\startitemgroup[noteitems]
\item\subnoteref{x36}\NoteKeywordNikayaHead{「挑毛病」}(ujjhānasaññikā,另譯為「有不滿的想法的」),菩提比丘長老英譯為\NoteKeywordBhikkhuBodhi{「找毛病者」}(Faultfinders)。按:《顯揚真義》說,沒有名叫挑毛病的單獨天界,這位天神因為不滿如來四種必需品的使用,祂心想,沙門喬達摩對比丘的糞掃衣、團食、樹下住處、腐尿藥喜悅、極稱讚,自己卻穿黃麻布、亞麻布等勝妙衣……。因此被法的結集上座們取其名為挑毛病者(ujjhānasaññikā)。
\stopitemgroup

\startitemgroup[noteitems]
\item\subnoteref{x37}\NoteKeywordNikayaHead{「無所有者」}(無貪瞋癡者),參看\suttaref{SN.1.34}。
\stopitemgroup

\startitemgroup[noteitems]
\item\subnoteref{x38}\NoteKeywordAgamaHead{「金鎗(\ccchref{SA.1289}{https://agama.buddhason.org/SA/dm.php?keyword=1289});佉陀羅(GA)」},南傳作\NoteKeywordNikaya{「碎石片」}(sakalika),菩提比丘長老英譯為\NoteKeywordBhikkhuBodhi{「石頭碎片」}(a stone splinter),並解說,世尊邪惡的堂弟提婆達多在靈鳩山上推大石頭試圖謀殺他,石頭投偏了,但那塊石頭的碎石割傷世尊的腳而流血。另參看Mi21.〈8.腳被碎石片所傷的問題〉。
\stopitemgroup

\startitemgroup[noteitems]
\item\subnoteref{x39}\NoteKeywordAgamaHead{「叫喚[地獄]」(roruvaṃ),菩提比丘長老英譯為「羅盧瓦地獄」(Roruva Hell)。按:《顯揚真義》說,有兩個叫喚:煙叫喚(dhūmaroruvo)與「焰叫喚」(jālaroruvo),前者是另一個(visuṃ),後者名為無間大地獄(avīcimahānirayassevetaṃ),在那裡,眾生被火燒時,他們一再地吼叫(punappunaṃ ravaṃ ravanti),與\ccchref{AA.42.2}{https://agama.buddhason.org/AA/dm.php?keyword=42.2}的「涕哭地獄、大涕哭地獄」}相當。
\stopitemgroup

\startitemgroup[noteitems]
\item\subnoteref{x40}《顯揚真義》給了這樣的解說:一根本=無明,二漩渦=常見、斷見,三垢穢=貪等(貪、瞋、癡),五岩石=五種欲,深淵=渴愛(以無立足點意-Apatiṭṭhaṭṭhena),大海=渴愛(以不滿足意-apūraṇīyaṭṭhena),十二個漩渦=六內外處。
\stopitemgroup

\startitemgroup[noteitems]
\item\subnoteref{x41}\NoteKeywordAgamaHead{「如毘舍脂眾(\ccchref{SA.587}{https://agama.buddhason.org/SA/dm.php?keyword=587});毘舍闍充滿(GA)」},南傳作\NoteKeywordNikaya{「是跟隨的惡鬼眾」}(pisācagaṇasevitaṃ),菩提比丘長老英譯為\NoteKeywordBhikkhuBodhi{「被一大群魔鬼糾纏」}(Haunted a host of demons)。《顯揚真義》說,這位天子[生前]在大師的教誡中出家五年後,取喜歡的業處進入林野,因害怕放逸而日夜不食不臥,只作意業處,以致體內生起刀風絕命,死後隨即往生在三十三天的大宮殿門旁,如睡醒般。他看見宮殿內成千天女說宮殿的主人天子來了,取樂器圍繞他。天子不知道已死,以為自己還是出家人而羞愧,所以會視那些天女眾為勾引誘惑他的惡鬼眾,而跑來向佛陀求援(大意)。按:「惡鬼」(pisāca),另譯為「食人鬼」、「吸血鬼」,「毘舍脂」或「毘舍闍」應為其音譯。
\stopitemgroup

\startitemgroup[noteitems]
\item\subnoteref{x42}\NoteKeywordAgamaHead{「如法戒具足(\ccchref{SA.997}{https://agama.buddhason.org/SA/dm.php?keyword=997});如法而持戒/正法淨持戒(GA)」},南傳作\NoteKeywordNikaya{「法住立者、戒具足者」}(Dhammaṭṭhā sīlasampannā),菩提比丘長老英譯為\NoteKeywordBhikkhuBodhi{「建立法,具有德行」}(Established in Dhamma, endowed with virtue)。
\stopitemgroup

\startitemgroup[noteitems]
\item\subnoteref{x43}「閻摩世界」(yamalokaṃ),菩提比丘長老解說為「餓鬼界」(pettivisaya)。《顯揚真義》在後面的「並且來世為惡趣」中以此解說(Samparāye ca duggatīti ‘‘yamalokaṃ upapajjare’’ti vutte samparāye ca duggati.)。 
  「
\stopitemgroup

\startitemgroup[noteitems]
\item\subnoteref{x44}\NoteKeywordNikayaHead{「極難覺醒的」}(sudubbudhaṃ),以sudubbuddhaṃ解讀。
\stopitemgroup

\startitemgroup[noteitems]
\item\subnoteref{x45}\NoteKeywordNikayaHead{「所依」}(vatthu,另譯為「土地,宅地,基礎,事物」),菩提比丘長老英譯為\NoteKeywordBhikkhuBodhi{「支持;支撐」}(support)。按:《顯揚真義》說,當老的時候,以兒子的持續照料(paṭijagganaṭṭhena)為依所(patiṭṭhā, 立足處)。
\stopitemgroup

\startitemgroup[noteitems]
\item\subnoteref{x46}\NoteKeywordAgamaHead{「業者可依怙(\ccchref{SA.1017}{https://agama.buddhason.org/SA/dm.php?keyword=1017});苦為大怖畏(GA)」},南傳作\NoteKeywordNikaya{「業是他的所趣處」}(kammaṃ tassa parāyananti),菩提比丘長老英譯為\NoteKeywordBhikkhuBodhi{「業決定他的命運」}(Kamma determines his destiny)。按:《顯揚真義》以「必定完成」(nipphatti avassayo)解說「所趣處」。
\stopitemgroup

\startitemgroup[noteitems]
\item\subnoteref{x47}\NoteKeywordAgamaHead{「女則累世間(\ccchref{SA.1019}{https://agama.buddhason.org/SA/dm.php?keyword=1019});亦惱害世間(GA)」},南傳作\NoteKeywordNikaya{「在這裡這個被世代黏著」}(etthāyaṃ sajjate pajā),菩提比丘長老英譯為\NoteKeywordBhikkhuBodhi{「在這裡男人們被絆住」}(Here menfolk are enmeshed)。按:\ccchref{AN.1.1}{https://agama.buddhason.org/AN/an.php?keyword=1.1}說:「女子形色持續遍取男子的心」;\ccchref{AN.1.6}{https://agama.buddhason.org/AN/an.php?keyword=1.6}說:「男子形色持續遍取女子的心」。「世代」(pajā),水野弘元《巴利語辭典》作「人人」。
\stopitemgroup

\startitemgroup[noteitems]
\item\subnoteref{x48}\NoteKeywordAgamaHead{「第二(\ccchref{SA.1014}{https://agama.buddhason.org/SA/dm.php?keyword=1014});二伴(GA)」},南傳作\NoteKeywordNikaya{「伴侶」}(dutiyā,另譯為「第二的;第二者;同伴」),菩提比丘長老英譯為\NoteKeywordBhikkhuBodhi{「合夥人;伙伴」}(partner)。按:《顯揚真義》以「善趣與涅槃」(sugatiñceva nibbānañca)解說。
\stopitemgroup

\startitemgroup[noteitems]
\item\subnoteref{x49}\NoteKeywordAgamaHead{「欲者是偈因(\ccchref{SA.1021}{https://agama.buddhason.org/SA/dm.php?keyword=1021});偈以欲為初(GA)」},南傳作\NoteKeywordNikaya{「韻律是偈頌的起源」}(Chando nidānaṃ gāthānaṃ),菩提比丘長老英譯為\NoteKeywordBhikkhuBodhi{「韻律是偈頌的骨架」}(Metre is the scaffolding of verses)。按:「Chanda」的意思是「欲」,「Chando」的意思是「韻律」,此處北傳漢譯為「欲」,應屬不當。
\stopitemgroup

\startitemgroup[noteitems]
\item\subnoteref{x50}\NoteKeywordAgamaHead{「名者偈所依(\ccchref{SA.1021}{https://agama.buddhason.org/SA/dm.php?keyword=1021});偈依止於名(GA)」},南傳作\NoteKeywordNikaya{「名字是偈頌依止的」}(Nāmasannissitā gāthā),菩提比丘長老英譯為\NoteKeywordBhikkhuBodhi{「詩安置在名字的基礎上」}(Verses rest on a base of names)。按:《顯揚真義》說,偈頌是依於海洋、大地等一些名字開始的。
\stopitemgroup

\startitemgroup[noteitems]
\item\subnoteref{x51}\NoteKeywordAgamaHead{「名者世無上(\ccchref{SA.1020}{https://agama.buddhason.org/SA/dm.php?keyword=1020});四陰名最勝GA)」},南傳作\NoteKeywordNikaya{「名征服一切」}(Nāmaṃ sabbaṃ addhabhavi),菩提比丘長老英譯為\NoteKeywordBhikkhuBodhi{「名壓下一切事物」}(Name has weighed down everything)。按:「名」即「四無色陰;四陰」(\ccchref{SA.298}{https://agama.buddhason.org/SA/dm.php?keyword=298}),《顯揚真義》說,化生的或人造的,脫離名的眾生不存在,即使不知名的樹木或岩石,仍名為無名者(anāmakotveva)。
\stopitemgroup

\startitemgroup[noteitems]
\item\subnoteref{x52}\NoteKeywordAgamaHead{「拘牽(\ccchref{SA.1009}{https://agama.buddhason.org/SA/dm.php?keyword=1009})」},南傳作\NoteKeywordNikaya{「被牽引」}(parikassati,原意為「拉著繞轉」),菩提比丘長老英譯為\NoteKeywordBhikkhuBodhi{「被到處拖」}(is…dragged here and there, \suttaref{SN.1.62}),或「被拖繞」(is…dragged around, \ccchref{AN.4.186}{https://agama.buddhason.org/AN/an.php?keyword=4.186})。按:《顯揚真義》以「拉入;引誘」(parikaḍḍhati, \suttaref{SN.1.62})解說,《滿足希求》以「被牽引」(ākaḍḍhiyati, \ccchref{AN.4.186}{https://agama.buddhason.org/AN/an.php?keyword=4.186})解說,今準此譯。
\stopitemgroup

\startitemgroup[noteitems]
\item\subnoteref{x53}參看\suttaref{SN.1.62}。
\stopitemgroup

\startitemgroup[noteitems]
\item\subnoteref{x54}\NoteKeywordNikayaHead{「尋是它的腳」}(vitakkassa vicāraṇaṃ),菩提比丘長老英譯為\NoteKeywordBhikkhuBodhi{「思考是它四處移動的方法」}(Thought is its means of travelling about),並解說,「腳」(vicāraṇa),與其同源的「伺」(vicāra),通常與「尋」(vitakka)合起來描述心思的過程(the thought process),如初禪[的「有尋有伺」],這裡似乎是表示:「尋」不需要身體的運動就能四處遊歷。按:《顯揚真義》以「尋是其腳」(vitakko tassa pādā)解說,亦即將vitakkassa拆為vitakka+assa。assa這裡採ayaṃ/tad的sg. gen./dat.,譯為「它的」。
\stopitemgroup

\startitemgroup[noteitems]
\item\subnoteref{x55}參看\suttaref{SN.1.64}。
\stopitemgroup

\startitemgroup[noteitems]
\item\subnoteref{x56}\NoteKeywordNikayaHead{「被欲求薰」}(icchādhūpāyito),菩提比丘長老英譯為\NoteKeywordBhikkhuBodhi{「燃燒著慾望」}(burning with desire)。按:《顯揚真義》以「被欲求燃燒」(icchāya āditto)解說,今依此格式譯。又,原文icchādhūpāyito sadā”ti應譯作「欲求經常被薰」,前對應句kissa dhūpāyito sadā”ti應譯作「什麼的經常被薰」,這樣兩句在「格」上沒有對應。New Concise Pali English Dictionary作dhūpāyita=dhūmāyita(被冒煙)。
\stopitemgroup

\startitemgroup[noteitems]
\item\subnoteref{x57}\NoteKeywordAgamaHead{「六法(\ccchref{SA.1008}{https://agama.buddhason.org/SA/dm.php?keyword=1008})」},南傳作\NoteKeywordNikaya{「六」},《顯揚真義》說:在六內處生起中這被名為生起,在六外處中作親密交往,對六內處執取後,在六外處之中被惱害(Chasu hi ajjhattikāyatanesu uppannesu ayaṃ uppanno nāma hoti, chasu bāhiresu santhavaṃ karoti, channaṃ ajjhattikānaṃ upādāya chasu bāhiresu vihaññatīti)。
\stopitemgroup

\startitemgroup[noteitems]
\item\subnoteref{x58}\NoteKeywordNikayaHead{「好幾種處地」}(canekāyatana),菩提比丘長老英譯為\NoteKeywordBhikkhuBodhi{「許多基礎(遍處;業處)」}(many bases)。按:《顯揚真義》以「因38個所緣而被好幾種方式地說」(aṭṭhatiṃsārammaṇavasena anekehi kāraṇehi kathito)解說。
\stopitemgroup

\startitemgroup[noteitems]
\item\subnoteref{x59}\NoteKeywordNikayaHead{「{財產}[心]」}(vittaṃ, cittaṃ),菩提比丘長老英譯為\NoteKeywordBhikkhuBodhi{「心」}(the mind)。按:《顯揚真義》沒解說,其他版本(sī. syā. kaṃ. pī)均作cittaṃ,今準此譯。
\stopitemgroup

\startitemgroup[noteitems]
\item\subnoteref{x60}\NoteKeywordNikayaHead{「女人是物品中最上的」}(itthī bhaṇḍānamuttamaṃ),菩提比丘長老英譯為\NoteKeywordBhikkhuBodhi{「女人列為最好的物品」}(A woman ranks as the best of goods)。按:《顯揚真義》說,因為女人是「不應該丟棄的品物」(avissajjanīyabhaṇḍattā),一切菩薩與轉輪王就從母親的子宮出生。
\stopitemgroup

\startitemgroup[noteitems]
\item\subnoteref{x61}\ccchref{SA.1292}{https://agama.buddhason.org/SA/dm.php?keyword=1292}作「賊劫奪則遮,沙門奪歡喜(SA);賊劫戒遮殺,沙門劫生喜(GA)」。
\stopitemgroup

\startitemgroup[noteitems]
\item\subnoteref{x62}\NoteKeywordNikayaHead{「人不應該布施自己」}(Attānaṃ na dade poso),《顯揚真義》說,除了一切菩薩外,不應該將自己施與作其他人的奴隸(parassa dāsaṃ katvā attānaṃ na dadeyya ṭhapetvā sabbabodhisatteti vuttaṃ)。
\stopitemgroup

\startitemgroup[noteitems]
\item\subnoteref{x63}\NoteKeywordAgamaHead{「信者持資糧(\ccchref{SA.1292}{https://agama.buddhason.org/SA/dm.php?keyword=1292});信為遠資糧(GA)」},南傳作作「信維繫旅程的資糧」(Saddhā bandhati pātheyyaṃ)。按:《顯揚真義》說,生起信後給與布施,守護戒(德行),作布薩行為,因為那樣它被說(saddhaṃ uppādetvā dānaṃ deti, sīlaṃ rakkhati, uposathakammaṃ karoti, tenetaṃ vuttaṃ)。
\stopitemgroup

\startitemgroup[noteitems]
\item\subnoteref{x64}\NoteKeywordNikayaHead{「行動範圍」}(iriyāpatho),菩提比丘長老英譯為\NoteKeywordBhikkhuBodhi{「行動過程」}(course of movement),智髻比丘長老英譯為「姿態」(posture, \ccchref{MN.20}{https://agama.buddhason.org/MN/dm.php?keyword=20})。按:iriyāpatho古譯為「威儀路」,「威儀」,今譯為「行動;舉止」,「路」,還有一個意思為「範圍」,比腳符合這裏的意思。《顯揚真義》以「活命生計;活命模式」(jīvitavutti)解說。
\stopitemgroup

\startitemgroup[noteitems]
\item\subnoteref{x65}\NoteKeywordNikayaHead{「心的到達」}(hadayassānupattiṃ),菩提比丘長老英譯為\NoteKeywordBhikkhuBodhi{「心的到達,熱中於他的利益」}(the heart's attainment, Bent on that as his advantage)。按:《顯揚真義》說,這是指阿羅漢狀態(arahattaṃ),而「那個效益」指阿羅漢狀態的效益(arahattānisaṃso)。
\stopitemgroup

\startitemgroup[noteitems]
\item\subnoteref{x66}\NoteKeywordNikayaHead{「阿修羅的征服者」}(vatrabhū),另譯為「帝釋天;天帝釋」。按:「摩伽」(Māgha),為天帝釋的別名,參看\ccchref{SA.1105}{https://agama.buddhason.org/SA/dm.php?keyword=1105}、\ccchref{SA.1106}{https://agama.buddhason.org/SA/dm.php?keyword=1106}。
\stopitemgroup

\startitemgroup[noteitems]
\item\subnoteref{x67}\NoteKeywordAgamaHead{「為婆羅門事(\ccchref{SA.1311}{https://agama.buddhason.org/SA/dm.php?keyword=1311})」},南傳作\NoteKeywordNikaya{「這應該被婆羅門做」}(Karaṇīyametaṃ brāhmaṇena),菩提比丘長老英譯為\NoteKeywordBhikkhuBodhi{「這應該被婆羅門做」}(This should be done by the brahmin),並說,這是三藏中佛陀對生起活力(the arousing of energy)的僅有批評。按:《顯揚真義》將這裡的「婆羅門」以「阿羅漢」解說。
\stopitemgroup

\startitemgroup[noteitems]
\item\subnoteref{x68}\NoteKeywordNikayaHead{「流」}(sotaṃ),菩提比丘長老英譯為\NoteKeywordBhikkhuBodhi{「流」}(stream)。按:《顯揚真義》以「渴愛之流」(Sotanti taṇhāsotaṃ)解說。
\stopitemgroup

\startitemgroup[noteitems]
\item\subnoteref{x69}\NoteKeywordNikayaHead{「不往生單一性」}(nekattamupapajjati),菩提比丘長老英譯為\NoteKeywordBhikkhuBodhi{「不到達單一(統一)」}(does not reach unity)。按:「往生」,《顯揚真義》以「得到」(paṭilabhatīti)解說,「單一性」以「禪那」(jhānaṃ)解說。
\stopitemgroup

\startitemgroup[noteitems]
\item\subnoteref{x70}\NoteKeywordNikayaHead{「如果以應該做的他做」}(Kayirā ce kayirāthenaṃ),菩提比丘長老英譯為\NoteKeywordBhikkhuBodhi{「如果一個人做應該被做的」}(If one would do what should be done)。按:《顯揚真義》以「如果他做活力,他應該做,那個活力不應該退縮」(yadi vīriyaṃ kareyya, kareyyātha, taṃ vīriyaṃ na osakkeyya)解說。
\stopitemgroup

\startitemgroup[noteitems]
\item\subnoteref{x71}\NoteKeywordNikayaHead{「我的人」}(pajaṃ mamaṃ),菩提比丘長老英譯為\NoteKeywordBhikkhuBodhi{「我的孩子」}(my child)。按:《顯揚真義》說,在大集會經[\ccchref{DN.20}{https://agama.buddhason.org/DN/dm.php?keyword=20}]的談論那天,日月兩位天子證得須陀洹果,因此世尊說「我的人(pajaṃ mamaṃ)」,這是我的兒子(putto mama )的意思。
\stopitemgroup

\startitemgroup[noteitems]
\item\subnoteref{x72}\NoteKeywordNikayaHead{「捨棄過失的」}(raṇañjahāti),菩提比丘長老英譯為\NoteKeywordBhikkhuBodhi{「以拋棄了瑕疵」}(with flaws discarded)。按:raṇa水野弘元《巴利語辭典》作「爭論,諍亂」,Buddhadatta《簡明巴英辭典》作「戰爭,罪,過失」,而《顯揚真義》以污染(kilesa)解說。
\stopitemgroup

\startitemgroup[noteitems]
\item\subnoteref{x73}\NoteKeywordNikayaHead{「心的到達」}(hadayassānupattiṃ),菩提比丘長老英譯為\NoteKeywordBhikkhuBodhi{「心的到達,熱中於他的利益」}(the heart's attainment, Bent on that as his advantage)。按:《顯揚真義》以「阿羅漢狀態」(arahattaṃ)解說。
\stopitemgroup

\startitemgroup[noteitems]
\item\subnoteref{x74}\NoteKeywordAgamaHead{「盡於歡喜有(\ccchref{SA.1269}{https://agama.buddhason.org/SA/dm.php?keyword=1269})」},南傳作\NoteKeywordNikaya{「貪之歡喜已遍滅盡者」}(Nandīrāgaparikkhīṇo),菩提比丘長老依錫蘭本(Nandībhavaparikkhīṇo)英譯為「有之歡喜的破壞者」(Who has destroyed delight in existence)。
\stopitemgroup

\startitemgroup[noteitems]
\item\subnoteref{x75}\NoteKeywordAgamaHead{「恐怖(\ccchref{SA.596}{https://agama.buddhason.org/SA/dm.php?keyword=596});驚懼(GA)」},南傳作\NoteKeywordNikaya{「驚嚇」}(utrasta),菩提比丘長老英譯為\NoteKeywordBhikkhuBodhi{「受驚嚇」}(frightened)。《顯揚真義》說,傳說[善梵]天子隨天女眾去歡喜園嬉戲,坐在晝度樹下,而五百位隨行天女突然命終,出生在阿鼻地獄受大苦。天子看見她們出生在地獄中,也看見自己七天後也將命終如她們出生到地獄,所以來到世尊的面前說偈誦(大意)。長老說,經末這句「[世尊]說這個……」的刪節號,表示善梵天子一聽完佛陀的教說,就迫不及待地趕回去修習了。
\stopitemgroup

\startitemgroup[noteitems]
\item\subnoteref{x76}\NoteKeywordNikayaHead{「苦行」}(tapasā,另譯為「鍛鍊」),菩提比丘長老英譯為\NoteKeywordBhikkhuBodhi{「嚴格的生活」}(austerity)。按:《顯揚真義》說,這是接受業處後進入林野修習,連同毘婆舍那修習覺支。
\stopitemgroup

\startitemgroup[noteitems]
\item\subnoteref{x77}\NoteKeywordNikayaHead{「無煩天」}(Aviha),即第三果(阿那含)聖者往生的「五不還天」(色界的最高天,又稱為「淨居天(suddhāvāsa)」)之一(其他四天為「無熱天(Atappa)」、「善現天;易被見天(Sudassa)」、「善見天(Sudassi)」、「色究竟天;阿迦膩吒天(Akaniṭṭha)」)。按:《顯揚真義》說,那七位比丘一往生無煩天就立即解脫了。
\stopitemgroup

\startitemgroup[noteitems]
\item\subnoteref{x78}\NoteKeywordAgamaHead{「煩惱軛/諸天軛(\ccchref{SA.595}{https://agama.buddhason.org/SA/dm.php?keyword=595});天境界(GA)」},南傳作\NoteKeywordNikaya{「天軛」}(dibbayogaṃ),菩提比丘長老英譯為\NoteKeywordBhikkhuBodhi{「天的束縛」}(the celestial bond)。按:《顯揚真義》以「五上分結」解說,而「捨斷人的身體後」則是斷除「五下分結」。
\stopitemgroup

\startitemgroup[noteitems]
\item\subnoteref{x79}\NoteKeywordAgamaHead{「弗迦羅娑梨(\ccchref{SA.595}{https://agama.buddhason.org/SA/dm.php?keyword=595});佛羯羅(GA.189);弗迦邏娑利(\ccchref{MA.162}{https://agama.buddhason.org/MA/dm.php?keyword=162})」},南傳作\NoteKeywordNikaya{「補估沙地」}(pukkusāti)。按:《破斥猶豫》說,他原是[健陀羅]德迦尸羅城的統治王(takkasīlanagare pukkusāti rājā rajjaṃ kāresi),王舍城頻毘沙羅王與他有邊境貿易來往,他因頻毘沙羅王之緣而嚮往佛法,遂自行剃髮、穿袈裟出走到王舍城,想跟隨世尊出家修學(大意),而後證得三果,成為經中所說往生無煩天的七位比丘之一,參看\ccchref{MA.162}{https://agama.buddhason.org/MA/dm.php?keyword=162}/\ccchref{MN.140}{https://agama.buddhason.org/MN/dm.php?keyword=140}。
\stopitemgroup

\startitemgroup[noteitems]
\item\subnoteref{x80}\NoteKeywordAgamaHead{「汝亦彼良友(\ccchref{SA.595}{https://agama.buddhason.org/SA/dm.php?keyword=595})」},南傳作\NoteKeywordNikaya{「是你在過去的同伴」}。按:依\ccchref{MN.81}{https://agama.buddhason.org/MN/dm.php?keyword=81},在迦葉佛時代,這位陶匠(kumbhakāra,直譯為「製甕者」)名叫「額低葛勒」(ghaṭikāra),\ccchref{MA.63}{https://agama.buddhason.org/MA/dm.php?keyword=63}譯作「難提波羅」,由於他,釋迦菩薩才隨迦葉佛出家。
\stopitemgroup

\startitemgroup[noteitems]
\item\subnoteref{x81}\NoteKeywordAgamaHead{「無常心乞食(\ccchref{SA.1343}{https://agama.buddhason.org/SA/dm.php?keyword=1343});乞食及住時,常思於無常(GA)」},南傳作\NoteKeywordNikaya{「無欲求的食物尋求者」}(Anicchā piṇḍamesanā),菩提比丘長老英譯為\NoteKeywordBhikkhuBodhi{「沒有願望,他們尋求他們的施捨」}(Without wishes they sought their alms)。按:「無欲」(Anicchā, 梵文同),《顯揚真義》以「成為離渴愛的」(nittaṇhā hutvā)解說。這與「無常的」(Anicca, 梵Anitya)巴利語極似,但梵文差異明顯,不知傳誦中發生了什麼事。
\stopitemgroup

\startitemgroup[noteitems]
\item\subnoteref{x82}\NoteKeywordAgamaHead{「一尋(\ccchref{SA.1307}{https://agama.buddhason.org/SA/dm.php?keyword=1307})」},南傳作\NoteKeywordNikaya{「一噚之長」}(byāmamatte ),菩提比丘長老英譯為\NoteKeywordBhikkhuBodhi{「一噚之高」}(fathom-high)。按:噚=6英呎,約183公分。
\stopitemgroup

\startitemgroup[noteitems]
\item\subnoteref{x83}\NoteKeywordNikayaHead{「不希求」}(nāsīsati),菩提比丘長老英譯為\NoteKeywordBhikkhuBodhi{「不渴望」}(does not desire)。按:《顯揚真義》以「不欲求」(na pattheti)解說。
\stopitemgroup

\startitemgroup[noteitems]
\item\subnoteref{x84}\NoteKeywordAgamaHead{「四輪九門(\ccchref{SA.588}{https://agama.buddhason.org/SA/dm.php?keyword=588});九門四輪轉(GA)」},南傳作\NoteKeywordNikaya{「四輪與九門」}(Catucakkaṃ navadvāraṃ),菩提比丘長老英譯為\NoteKeywordBhikkhuBodhi{「有四輪與九門」}(Having four wheels and nine doors)。按:《顯揚真義》以四舉止行為路(catuiriyāpathaṃ,行住坐臥)解說四輪,以九個傷口(navahi vaṇamukhehi)解說九門,就是身上的眼耳鼻嘴尿肛九個開口。
\stopitemgroup

\startitemgroup[noteitems]
\item\subnoteref{x85}\NoteKeywordAgamaHead{「深溺烏泥中(\ccchref{SA.588}{https://agama.buddhason.org/SA/dm.php?keyword=588});深淤泥之中(GA)」},南傳作\NoteKeywordNikaya{「污泥所生」}(Paṅkajātaṃ),菩提比丘長老英譯為\NoteKeywordBhikkhuBodhi{「從泥沼生」}(Born from a bog)。按:註疏以不淨的母親子宮(Mātukucchisaṅkhāte asucipaṅke)解說污泥。
\stopitemgroup

\startitemgroup[noteitems]
\item\subnoteref{x86}\NoteKeywordAgamaHead{「出(\ccchref{SA.588}{https://agama.buddhason.org/SA/dm.php?keyword=588});去(GA)」},南傳作\NoteKeywordNikaya{「逃離」}(yātrā),菩提比丘長老英譯為\NoteKeywordBhikkhuBodhi{「逃脫」}(escape)。按:yātrā原意為「生活;生存」,《顯揚真義》以「離開(niggamanaṃ);解脫、完全脫離、超越(niggamanaṃ mutti parimutti samatikkamo)」解說,今準此譯。
\stopitemgroup

\startitemgroup[noteitems]
\item\subnoteref{x87}\NoteKeywordAgamaHead{「斷愛喜長縻(\ccchref{SA.588}{https://agama.buddhason.org/SA/dm.php?keyword=588});斷於喜愛結(GA)」},南傳作\NoteKeywordNikaya{「切斷皮帶與細繩後」}(Chetvā naddhiṃ varattañca),菩提比丘長老英譯為\NoteKeywordBhikkhuBodhi{「切斷皮帶與皮繩後」}(Having cut the thong and the strap)。按:《顯揚真義》以之前的憤怒與之後的怨恨轉起的強烈憤怒(balavakodhanti)解說皮帶,而細繩指除此之外的殘留的污染(kilese ṭhapetvā avasesā)。
\stopitemgroup

\startitemgroup[noteitems]
\item\subnoteref{x88}\NoteKeywordAgamaHead{「身光增明(\ccchref{SA.1306}{https://agama.buddhason.org/SA/dm.php?keyword=1306});其身光曜,倍更殊常;顏貌威光轉熾盛(GA)」},南傳作\NoteKeywordNikaya{「出現種種輝耀的容色」}(uccāvacā vaṇṇanibhā upadaṃseti),菩提比丘長老英譯為\NoteKeywordBhikkhuBodhi{「表現種種光輝的顏色」}(displayed diverse lustrous colours)。按:《顯揚真義》說,出藍光的[天子]更藍(nīlaṭṭhānaṃ atinīlaṃ);黃的更黃;紅的更紅;白的更白(odātaṭṭhānaṃ accodātanti)。
\stopitemgroup

\startitemgroup[noteitems]
\item\subnoteref{x89}\NoteKeywordNikayaHead{「等待死時」}(kālaṃ kaṅkhati),菩提比丘長老英譯為\NoteKeywordBhikkhuBodhi{「他等待時間(到)」}(he awaits the time)。按:《顯揚真義》說,他希求(pattheti)般涅槃的時間,因為諸漏已滅盡者既不喜歡死亡,也不希求生命,他希求[般涅槃的]時間,就像站著領取日薪的人(divasasaṅkhepaṃ vetanaṃ gahetvā ṭhitapuriso)那樣。另參看Thag.654(\ccchref{Mi.11}{https://agama.buddhason.org/Mi/Mi11.htm}, 4.結生感受)。
\stopitemgroup

\startitemgroup[noteitems]
\item\subnoteref{x90}\NoteKeywordNikayaHead{「四禁戒善防護者」}(cātuyāmasusaṃvuto),菩提比丘長老英譯為\NoteKeywordBhikkhuBodhi{「以四種控制之好的自制」}(Well restrained by the four controls)。按:《顯揚真義》以「被所有的防止防止,被所有的防止軛制,被所有的防止除去,被所有的防止遍滿」(sabbavārivārito ca hoti sabbavāriyutto ca sabbavāridhuto ca sabbavāriphuṭo cāti)解說「四禁戒」,另參看\ccchref{DN.2}{https://agama.buddhason.org/DN/dm.php?keyword=2}。
\stopitemgroup

\startitemgroup[noteitems]
\item\subnoteref{x91}\NoteKeywordNikayaHead{「破裂法」}(bhedanadhammo),如陶器的破裂,參看\suttaref{SN.3.22}。
\stopitemgroup

\startitemgroup[noteitems]
\item\subnoteref{x92}\NoteKeywordNikayaHead{「現在賢面將以法庭被看到」}(bhadramukho dāni aḍḍakaraṇena paññāyissatī’”ti),菩提比丘長英譯為:「現在是好面將被知道他的審判。」(Now it is Good Face who will be known by his judgements.)。按:《顯揚真義》說:當國王坐在法庭中時知道大臣(amaccā)收賄(lañjaṃ),而想:「毘琉璃(viṭaṭūbho, Viḍūḍabho)將軍現在將主持自己的統治。」依此,「賢面」是國王的兒子毘琉璃,但菩提比丘長老依〈本生〉No.465的因緣(序)說前將軍Bandhula的可能性比毘琉璃大。
\stopitemgroup

\startitemgroup[noteitems]
\item\subnoteref{x93}\NoteKeywordNikayaHead{「往下拉的」}(ohārinaṃ),菩提比丘長老英譯為\NoteKeywordBhikkhuBodhi{「降低品格;墮落的」}(Degrading)。按:《顯揚真義》說,拉到四苦界(catūsu apāyesu)。
\stopitemgroup

\startitemgroup[noteitems]
\item\subnoteref{x94}\NoteKeywordAgamaHead{「闍祇羅(\ccchref{SA.1148}{https://agama.buddhason.org/SA/dm.php?keyword=1148});長髮梵志(GA)」},南傳作\NoteKeywordNikaya{「結髮者」}(jaṭilā),菩提比丘長老解說,這是「結髮(編髮)沙門」。
\stopitemgroup

\startitemgroup[noteitems]
\item\subnoteref{x95}\NoteKeywordAgamaHead{「一舍羅(\ccchref{SA.1148}{https://agama.buddhason.org/SA/dm.php?keyword=1148});一衣外道(GA)」},南傳作\NoteKeywordNikaya{「一衣者」}(ekasāṭakā),菩提比丘長老英譯為\NoteKeywordBhikkhuBodhi{「一件長袍的禁欲修道者(沙門)」}(one-robed ascetics)。
\stopitemgroup

\startitemgroup[noteitems]
\item\subnoteref{x96}\NoteKeywordNikayaHead{「因為以善制御的特徵」}(Susaññatānañhi viyañjanena),菩提比丘長老英譯為\NoteKeywordBhikkhuBodhi{「因為在控制得很好的喬裝外貌下」}(For in the guise of the well controlled)。按:《顯揚真義》以「必需品道具」(parikkhārabhaṇḍakena)解說「特徵」。
\stopitemgroup

\startitemgroup[noteitems]
\item\subnoteref{x97}\NoteKeywordNikayaHead{「被附屬物包覆者們在世間中行」}(Caranti loke parivārachannā),菩提比丘長老英譯為\NoteKeywordBhikkhuBodhi{「某些以偽裝到處移動」}(Some move about in disguise)。
\stopitemgroup

\startitemgroup[noteitems]
\item\subnoteref{x98}\NoteKeywordNikayaHead{「迦哈玻那」}(kahāpaṇa)為當時的貨幣單位。
\stopitemgroup

\startitemgroup[noteitems]
\item\subnoteref{x99}\NoteKeywordNikayaHead{「最多一那利的飯量」}(nāḷikodanaparamatāya),「那利」(nāḷikā,另譯為「莖,管,筒,藥袋」)與「桶」(doṇa),都是容量的單位。
\stopitemgroup

\startitemgroup[noteitems]
\item\subnoteref{x100}\NoteKeywordAgamaHead{「臥覺寂靜樂(\ccchref{SA.1236}{https://agama.buddhason.org/SA/dm.php?keyword=1236});寂滅安睡眠(GA)」},南傳作\NoteKeywordNikaya{「寂靜者睡得安樂」}(Upasanto sukhaṃ seti,逐字譯為「寂靜者樂臥」,「臥」為動詞),菩提比丘長老英譯為\NoteKeywordBhikkhuBodhi{「平和者睡得安樂」}(The peaceful one sleeps at ease)。按:依巴利語經文與別譯雜阿含經推斷,「臥覺寂靜樂」應為「寂靜臥覺樂」,\ccchref{SA.1081}{https://agama.buddhason.org/SA/dm.php?keyword=1081}就譯為「臥覺常安樂」。
\stopitemgroup

\startitemgroup[noteitems]
\item\subnoteref{x101}\NoteKeywordNikayaHead{「好妻子的」}(subhagiyā),菩提比丘長老英譯為\NoteKeywordBhikkhuBodhi{「幸福的女人」}(blessed woman)。按:《顯揚真義》以「好妻子的」(subhariyāya)解說,今準此譯。
\stopitemgroup

\startitemgroup[noteitems]
\item\subnoteref{x102}\NoteKeywordAgamaHead{「不放逸」(Appamādo),菩提比丘長老英譯為「勤奮」(Diligence)。按:《顯揚真義》以「發令者的不放逸」(kārāpakaappamādo)解說,註疏以「三種福德行為的基礎轉起之不放逸」(Kārāpakaappamādoti tiṇṇaṃ puññakiriyavatthūnaṃ pavattakaappamādo)解說「發令者的不放逸」,長老附註為「布施、德行(戒)、禪修」(giving, virtue, and meditation),這與\ccchref{SA.264}{https://agama.buddhason.org/SA/dm.php?keyword=264}說的「一者、布施,二者、調伏,三者、修道」},\ccchref{MA.61}{https://agama.buddhason.org/MA/dm.php?keyword=61}等說的「一者、布施,二者、調御,三者、守護」,\ccchref{AA.24.4}{https://agama.buddhason.org/AA/dm.php?keyword=24.4}說的「惠施、慈仁、自守」都相應。有問:為何這裡的不放逸只說福德行為呢?拙作:因為波斯匿王問的是「當生與後世的利益」,並非問苦的止息。
\stopitemgroup

\startitemgroup[noteitems]
\item\subnoteref{x103}\NoteKeywordAgamaHead{「大叫喚地獄(GA-\ccchref{SA.1233}{https://agama.buddhason.org/SA/dm.php?keyword=1233});涕哭大地獄(GA)」},南傳作\NoteKeywordNikaya{「大叫喚地獄」}(mahāroruvaṃ nirayaṃ),菩提比丘長老英譯為\NoteKeywordBhikkhuBodhi{「大羅盧瓦地獄」}(the Great Roruva Hell),並解說有兩個大叫喚地獄:「煙叫喚」(dhūmaroruva)與「焰叫喚」(jālaroruva),前者是另一個地獄(separate hell),而後者是大無間地獄(the great hell Avīci,大阿鼻地獄)的另名,被稱為「叫喚」,是因為那裡的眾生被烤時,「他們一再地哭吼嚎叫」(punappunaṃ ravaṃ ravanti, they cry out again and again,\suttaref{SN.1.39} note.93)。
\stopitemgroup

\startitemgroup[noteitems]
\item\subnoteref{x104}\NoteKeywordNikayaHead{「耐性、溫雅」}(khantisoraccaṃ),菩提比丘長老英譯為\NoteKeywordBhikkhuBodhi{「耐心與文雅之德行」}(The virtues of patience and gentleness)。按:《顯揚真義》說,「耐性」指「忍耐與信忍」(adhivāsanakhanti),「溫雅」指「阿羅漢性;阿羅漢性」(arahattaṃ),「諸法」(dhammā)指這二法(耐性、溫雅)。
\stopitemgroup

\startitemgroup[noteitems]
\item\subnoteref{x105}\NoteKeywordNikayaHead{「閃電盤繞」}(vijjumālī,直譯為「閃電花環」),菩提比丘長老英譯為\NoteKeywordBhikkhuBodhi{「閃電盤繞」}(Wreathed in lightning)。
\stopitemgroup

\startitemgroup[noteitems]
\item\subnoteref{x106}\NoteKeywordNikayaHead{「百峰般的雲」}(satakkaku,另譯為「雲;百峰」),菩提比丘長老英譯為\NoteKeywordBhikkhuBodhi{「具有百冠(浪峰)」}(with a hundred crests)。按:《顯揚真義》以「百峰(雲)、各種尖頂」(satasikharo, anekakūṭoti)解說。
\stopitemgroup

\startitemgroup[noteitems]
\item\subnoteref{x107}\NoteKeywordAgamaHead{「日日(\ccchref{SA.1147}{https://agama.buddhason.org/SA/dm.php?keyword=1147});日中(GA)」},南傳作\NoteKeywordNikaya{「中午」}(divā divassā),菩提比丘長老英譯為\NoteKeywordBhikkhuBodhi{「在中午」}(in the middle of the day)。按:《顯揚真義》以「在白天日中的」(divasassa divā)解說。
\stopitemgroup

\startitemgroup[noteitems]
\item\subnoteref{x108}\NoteKeywordNikayaHead{「沒有去處,沒有對象」}(natthi gati natthi visayo),菩提比丘長老英譯為\NoteKeywordBhikkhuBodhi{「沒有地方…沒領域」}(there is no place…, no scope)。「對象」(visayo),另譯為「境域;範圍」,《顯揚真義》以「機會、可能性」(okāso, samatthabhāvo vā)解說,而「去處」則以「結果」(nipphatti)解說。
\stopitemgroup

\startitemgroup[noteitems]
\item\subnoteref{x109}\NoteKeywordNikayaHead{「隨從」}(baddhagū,直譯「已達被繫縛的」),菩提比丘長老依羅馬拼音1998年版(paddhagū)英譯為「忠實追隨者」(henchmen)。按:《顯揚真義》以「隨從」(baddhacarā)解說,今準此譯。
\stopitemgroup

\startitemgroup[noteitems]
\item\subnoteref{x110}\NoteKeywordAgamaHead{「一一而去(\ccchref{SA.1096}{https://agama.buddhason.org/SA/dm.php?keyword=1096});各{二}[一一?]人(\ccchref{DA.1}{https://agama.buddhason.org/DA/dm.php?keyword=1});各各二人…莫獨去也(毘尼母經)」},南傳作\NoteKeywordNikaya{「不要兩個同一地走」}(mā ekena dve agamittha),菩提比丘長老英譯為\NoteKeywordBhikkhuBodhi{「不要兩個走同樣的道路」}(Let not two go the same way),Maurice Walshe先生英譯為「不要兩個走在一起」(Do not go two together, \ccchref{DN.14}{https://agama.buddhason.org/DN/dm.php?keyword=14})。
\stopitemgroup

\startitemgroup[noteitems]
\item\subnoteref{x111}\NoteKeywordNikayaHead{「為了睡眠」}(seyyo),菩提比丘長老英譯為\NoteKeywordBhikkhuBodhi{「」(for lodging)。按:seyyo(意為「更好的」}),但PTS版作seyyā,《顯揚真義》亦同,並以「為了睡眠」解說(Seyyāti seyyatthāya.),今準此譯。
\stopitemgroup

\startitemgroup[noteitems]
\item\subnoteref{x112}\NoteKeywordNikayaHead{「欲纏、縛著」}(jālinī visattikā),菩提比丘長老英譯為\NoteKeywordBhikkhuBodhi{「糾纏與綑綁」}(Entangling and binding)。按:「欲纏、縛著」也都可以譯為渴愛,長老說,渴愛(十八種渴愛思潮,參考\ccchref{AN.4.199}{https://agama.buddhason.org/AN/an.php?keyword=4.199}被說成「欲纏」,是因為它像網一樣撒向三界眾生;被說成「縛著」,是因為它拴住像色這樣的感官目標,它引導到「任何地方」(三界眾生)。
\stopitemgroup

\startitemgroup[noteitems]
\item\subnoteref{x113}\suttaref{SN.1.12}作「天子」。
\stopitemgroup

\startitemgroup[noteitems]
\item\subnoteref{x114}\NoteKeywordNikayaHead{「如陶醉於乳的(嬰兒)般地」}(khīramattova),菩提比丘長老英譯為\NoteKeywordBhikkhuBodhi{「像吸奶的嬰兒」}(like a milk-sucking baby)。按:《顯揚真義》以「朝上臥的(uttānaseyyako)幼童飲乳後,在黃麻布枕上躺臥,如無想者般地(asaññī viya)睡覺,善人這樣不思慮(na cinteti)『任何長短壽命』(kassaci āyuṃ appaṃ vā dīghaṃ vāti)」解說。
\stopitemgroup

\startitemgroup[noteitems]
\item\subnoteref{x115}\NoteKeywordAgamaHead{「猶如車輪轉(\ccchref{SA.1085}{https://agama.buddhason.org/SA/dm.php?keyword=1085});如輪軸轉(GA)」},南傳作\NoteKeywordNikaya{「如車輪對車軸(柱)」}(nemīva rathakubbaran),菩提比丘長老英譯為\NoteKeywordBhikkhuBodhi{「如車的輞繞轂轉」}(Like the chariot's felly round the hub),並解說這似乎暗指著《大林間奧義書》(Bṛhadāraṇyaka Upaniṣhad 11.5.15)的譬喻:「如所有車輪輻條包含於車軸和輪輞,所有眾生與所有那些(地、水等)自我,都包含於大我中。」
\stopitemgroup

\startitemgroup[noteitems]
\item\subnoteref{x116}\NoteKeywordNikayaHead{「我也不害怕睡覺」}(napi bhemi sottuṃ),菩提比丘長老英譯為\NoteKeywordBhikkhuBodhi{「我也不害怕睡覺」}(Nor am I afraid to sleep)。按:《顯揚真義》以「如某些人就害怕在獅子路徑等[處]睡覺,這樣我也不害怕睡覺」(Napi bhemi sottunti yathā ekacco sīhapathādīsuyeva supituṃ bhāyati, evaṃ ahaṃ supitumpi na bhāyāmi),今準此譯。
\stopitemgroup

\startitemgroup[noteitems]
\item\subnoteref{x117}\NoteKeywordAgamaHead{「相違不相違(\ccchref{SA.1097}{https://agama.buddhason.org/SA/dm.php?keyword=1097})」},南傳作\NoteKeywordNikaya{「在順從與排斥上」}(Anurodhavirodhesu),菩提比丘長老英譯為\NoteKeywordBhikkhuBodhi{「在吸引和排斥」}(In attraction and repulsion)。按:《顯揚真義》說,這是貪與瞋恚(rāgapaṭighesu),當講述法的談論時,因有一類人讚歎(sādhukāraṃ dadanti,給與作善哉」)他在那裡生起貪,有一類人不恭敬聽聞,他在那裡生起嫌惡(paṭigho,反感),名為說法者在順從與排斥上執著(sajjati,黏著)。
\stopitemgroup

\startitemgroup[noteitems]
\item\subnoteref{x118}\NoteKeywordNikayaHead{「陷阱」}(pāso,另譯為「罠;捕鳥獸的網,繩套」),菩提比丘長老英譯為\NoteKeywordBhikkhuBodhi{「陷阱」}(snare)。按:《顯揚真義》說,這是貪的陷阱(rāgapāsa,貪網, \suttaref{SN.4.5}),綁住在空中行進的[神足神通]者。長老解說,「在空中行進」(antalikkhacaro),更像是指「貪」能遠距離驅動心的特性。
\stopitemgroup

\startitemgroup[noteitems]
\item\subnoteref{x119}\NoteKeywordNikayaHead{「當於一切處探求時」}(Anvesaṃ sabbaṭṭhānesu),菩提比丘長老英譯為\NoteKeywordBhikkhuBodhi{「雖然他們到處找他」}(Though they seek him everywhere)。按:《顯揚真義》說,「探求」指在「所察量有胎趣處與眾生存續住所」(bhavayonigatiṭhitisattāvāsasaṅkhātesu)這一切中遍求(pariyesamānā)。長老說,這似乎包含了活著的阿羅漢與般涅槃的阿羅漢兩者。
\stopitemgroup

\startitemgroup[noteitems]
\item\subnoteref{x120}\NoteKeywordNikayaHead{「世間被迷昏頭」}(loko vimucchito),菩提比丘長老英譯為\NoteKeywordBhikkhuBodhi{「世間被迷戀」}(the world is infatuated)。按:「被迷昏頭」(vimucchito, vi-mucchito),原意為「被迷昏;被變成在夢中;被迷戀」,《顯揚真義》以「世間在這六種所緣中沉迷」(etesu chasu ārammaṇesu loko adhimucchito)解說。
\stopitemgroup

\startitemgroup[noteitems]
\item\subnoteref{x121}\NoteKeywordNikayaHead{「無花果樹枝的拐杖」}(udumbaradaṇḍaṃ),菩提比丘長老英譯為\NoteKeywordBhikkhuBodhi{「udumbara樹之杖」}(a staff of udumbara wood)按:《顯揚真義》說,拿彎曲的無花果樹枝拐杖棒為宣示少欲的狀態之意(appicchabhāvappakāsanatthaṃ),為高齡的出家婆羅門所拿。長老說,在吠陀祭典中,無花果木用於祭典中的各種儀式。
\stopitemgroup

\startitemgroup[noteitems]
\item\subnoteref{x122}\NoteKeywordNikayaHead{「吐舌」}(jivhaṃ nillāletvā),菩提比丘長老英譯為\NoteKeywordBhikkhuBodhi{「垂伸出他的舌」}(lolled his tongue)。按:nillāletvā《顯揚真義》以nīharitvā(取出)解說,今依此譯。
\stopitemgroup

\startitemgroup[noteitems]
\item\subnoteref{x123}\NoteKeywordNikayaHead{「三反」},南傳作\NoteKeywordNikaya{「三條溝」}(tivisākhaṃ,原意為「三條枝」),菩提比丘長老英譯為\NoteKeywordBhikkhuBodhi{「三條溝」}(three furrows)。按:《顯揚真義》以「三條溝」(tisākhaṃ)解說,今準此譯。
\stopitemgroup

\startitemgroup[noteitems]
\item\subnoteref{x124}\NoteKeywordNikayaHead{「起不愉快的神情後」}(nalāṭikaṃ vuṭṭhāpetvā),菩提比丘長老英譯在此處缺,智髻比丘長老英譯為「皺眉頭」(raised his eyebrows, \ccchref{MN.18}{https://agama.buddhason.org/MN/dm.php?keyword=18})。按:《顯揚真義》以「不愉快的神情(皺眉頭),在前額起皺紋的意思」(bhakuṭiṃ, nalāṭe uṭṭhitaṃ valittayanti attho)解說。
\stopitemgroup

\startitemgroup[noteitems]
\item\subnoteref{x125}\NoteKeywordNikayaHead{「我的念與慧已覺」}(Sati paññā ca me buddhā),菩提比丘長老依《長老偈46》(Sati paññā ca me vuḍḍha)英譯為「我的深切注意與智慧是圓熟的」(My mindfulness and wisdom are mature)。按:《顯揚真義》以「我的念、慧、智」(mayā sati ca paññā ca ñātā)解說。
\stopitemgroup

\startitemgroup[noteitems]
\item\subnoteref{x126}\NoteKeywordAgamaHead{「時受意解脫身作證(\ccchref{SA.1091}{https://agama.buddhason.org/SA/dm.php?keyword=1091});得時解脫自身作證(GA)」},南傳作\NoteKeywordNikaya{「觸達暫時的心解脫」}(sāmayikaṃ cetovimuttiṃ phusi),菩提比丘長老英譯為\NoteKeywordBhikkhuBodhi{「他達到暫時性心的釋放」}(he reached temporary liberation of mind)。按:《顯揚真義》說,「當障礙法每次突破的剎那解脫,以及在所緣上勝解」的世間等至名為暫時的心解脫(appitappitakkhaṇe paccanīkadhammehi vimuccati, ārammaṇe ca adhimuccatīti lokiyasamāpatti sāmayikā cetovimutti nāma),長老說「世間等至」也就是禪定與無色等至,所以被這麼稱呼,是因為在心安止(absorption the mind)的那一刻,心從障礙法(the opposing states)解脫,並且在其所緣上勝解,但他因病(慢性風病、膽病、痰病)而從這樣的心解脫退失,不能使定的狀態圓滿,每次一進入等至,很快地就又從那個狀態退失。
\stopitemgroup

\startitemgroup[noteitems]
\item\subnoteref{x127}\NoteKeywordNikayaHead{「肩膀轉回」}(vivattakkhandhaṃ, vivaṭṭakkhandhaṃ),菩提比丘長老英譯為\NoteKeywordBhikkhuBodhi{「帶著他已轉動的肩膀」}(with his shoulder turned)。按:《顯揚真義》以「回轉的肩膀」(parivattakkhandhaṃ)解說,並說,上座習慣向右躺臥(dakkhiṇena passena paricitasayanattā),所以就保持頭向右轉。又,對上段取刀[自殺]解說,上座心想:「以這種活命對我何用(kiṃ mayhaṃ)?」躺臥後,以刀切脖子,苦受生起,上座鎮伏[苦]受後,就探索那個受(taṃyeva vedanaṃ pariggahetvā),念生起,當把握(sammasanto)根本業處時,到達阿羅漢狀態,成為「命之齊頭者」(jīvitasamasīsī)般涅槃。又,有三種齊頭者:威儀路(舉止行為)之齊頭者(iriyāpathasamasīsī)、疾病之齊頭者(rogasamasīsī)、命之齊頭者(即「命終」與「得解脫」同時,餘三種類推)。
\stopitemgroup

\startitemgroup[noteitems]
\item\subnoteref{x128}\NoteKeywordNikayaHead{「期待機會」}(otārāpekkho),菩提比丘長老英譯為\NoteKeywordBhikkhuBodhi{「尋求得到接近他」}(seeking to gain access to him)。按:《顯揚真義》說,[魔心想:]「如果我看見沙門喬達摩的身門等(kāyadvārādīsu)有任不適當的,我要呵責。」
\stopitemgroup

\startitemgroup[noteitems]
\item\subnoteref{x129}\NoteKeywordNikayaHead{「為何」}(kenaci),PTS版作kena ci,依此譯。
\stopitemgroup

\startitemgroup[noteitems]
\item\subnoteref{x130}\NoteKeywordNikayaHead{「有貪的熱望」}(bhavalobhajappaṃ),菩提比丘長老英譯為\NoteKeywordBhikkhuBodhi{「對存在之貪心的驅策」}(greedy urge for existence)。按:「有」(bhava),即「十二緣起」支中的「有」支,《顯揚真義》說,有貪被稱為渴愛(bhavalobhasaṅkhātaṃ taṇhaṃ)。
\stopitemgroup

\startitemgroup[noteitems]
\item\subnoteref{x131}\NoteKeywordAgamaHead{「最受第一樂(\ccchref{SA.1092}{https://agama.buddhason.org/SA/dm.php?keyword=1092})」},南傳作\NoteKeywordNikaya{「隨覺樂」}(sukhamanubodhiṃ),菩提比丘長老英譯為\NoteKeywordBhikkhuBodhi{「我發覺幸福」}(I discovered bliss)。按:《顯揚真義》說,這是「隨覺阿羅漢樂」。
\stopitemgroup

\startitemgroup[noteitems]
\item\subnoteref{x132}\NoteKeywordAgamaHead{「五欲流(\ccchref{SA.1092}{https://agama.buddhason.org/SA/dm.php?keyword=1092});五駛流(GA)」},南傳作\NoteKeywordNikaya{「已渡過五暴流」}(pañcoghatiṇṇo),菩提比丘長老英譯為\NoteKeywordBhikkhuBodhi{「渡過五暴流」}(five floods crossed)。按:《顯揚真義》說,五暴流指五門(根)污染之暴流,第六指意門(根)污染之暴流,或者,五流指五下分結,第六暴流指五上分結。
\stopitemgroup

\startitemgroup[noteitems]
\item\subnoteref{x133}\NoteKeywordNikayaHead{「不流動」}(na sarati),菩提比丘長老英譯為\NoteKeywordBhikkhuBodhi{「不飄游」}(does not drift)。按:《顯揚真義》說,不以瞋而發怒,不以貪而流動,不以癡而惛沈。
\stopitemgroup

\startitemgroup[noteitems]
\item\subnoteref{x134}\NoteKeywordNikayaHead{「切斷渴愛後」}(Acchejja taṇhaṃ),菩提比丘長老依錫蘭本(Acchejji taṇhaṃ)英譯為「他已切斷渴愛」(He has cut off craving)。按:《顯揚真義》以「斷裂後」(acchinditvā)解說Acchejja。
\stopitemgroup

\startitemgroup[noteitems]
\item\subnoteref{x135}\NoteKeywordAgamaHead{「如風飄其綿(\ccchref{SA.1092}{https://agama.buddhason.org/SA/dm.php?keyword=1092});如風吹兜羅(GA)」},南傳作\NoteKeywordNikaya{「如風對落下的棉花」}(tūlaṃ bhaṭṭhaṃva mālutoti),菩提比丘長老英譯為\NoteKeywordBhikkhuBodhi{「如風,掉落的棉簇」}(As the wind, a fallen cotton tuft)。按:「兜羅」應為「棉花」(tūlaṃ)的音譯。
\stopitemgroup

\startitemgroup[noteitems]
\item\subnoteref{x136}\NoteKeywordAgamaHead{「二指智(\ccchref{SA.1199}{https://agama.buddhason.org/SA/dm.php?keyword=1199});鄙穢智(GA)」},南傳作\NoteKeywordNikaya{「以二指慧」}(dvaṅgulapaññāya),菩提比丘長老英譯為\NoteKeywordBhikkhuBodhi{「以她兩根手指的智慧」}(With her two-fingered wisdom)。按:《顯揚真義》說,這是指「微少之慧」(parittapaññāya),因為以二指取綿芯捻線,因此女人被稱為「二指慧」。
\stopitemgroup

\startitemgroup[noteitems]
\item\subnoteref{x137}\NoteKeywordNikayaHead{「在智轉起中時」}(Ñāṇamhi vattamānamhi),菩提比丘長老英譯為\NoteKeywordBhikkhuBodhi{「當理解穩定流動時」}(When knowledge flows on steadily)。按:《顯揚真義》以「在發生到達果智時」(phalasamāpattiñāṇe pavattamāne)解說。
\stopitemgroup

\startitemgroup[noteitems]
\item\subnoteref{x138}\NoteKeywordAgamaHead{「無邊際諸子(\ccchref{SA.1200}{https://agama.buddhason.org/SA/dm.php?keyword=1200});無欲無子想(GA)」},南傳作\NoteKeywordNikaya{「我是兒子死亡的終結者」}(Accantaṃ mataputtāmhi),菩提比丘長老英譯為\NoteKeywordBhikkhuBodhi{「我已終結兒子死亡」}(I've gotten past the death of sons)。按:「終結者(的)」(Accantaṃ),另譯為「究竟的;最終的;畢竟的;極邊的」,北傳的「無邊際」應與此相當。
\stopitemgroup

\startitemgroup[noteitems]
\item\subnoteref{x139}\NoteKeywordAgamaHead{「此則男子邊(\ccchref{SA.1200}{https://agama.buddhason.org/SA/dm.php?keyword=1200});我斷恩愛已(GA)」},南傳作\NoteKeywordNikaya{「為這個之男人們的結束者」}(purisā etadantikā),菩提比丘長老英譯為\NoteKeywordBhikkhuBodhi{「對這個,男人的尋求已結束」}(With this, the search for men has ended.)。按:PTS巴英詞典對此句解說為「男人們為那個(對我)是結束的,即:為了生育兒子目的之男人們對我來說不再存在。」(men are (to me) at the end for that, i. e. men do not exist any more for me, for the purpose of begetting sons.)今準此譯。
\stopitemgroup

\startitemgroup[noteitems]
\item\subnoteref{x140}\NoteKeywordNikayaHead{「以五種樂器」}(Pañcaṅgikena turiyena),菩提比丘長老英譯為\NoteKeywordBhikkhuBodhi{「以五種樂器合奏」}(With the music of a fivefold ensemble)。按:《顯揚真義》說,五種樂器為「單面鼓(ātataṃ)、雙面鼓(vitataṃ,依註疏)、弦樂器(ātatavitataṃ,依註疏)、打擊樂器(ghanaṃ)、吹奏樂器(susiranti)」。
\stopitemgroup

\startitemgroup[noteitems]
\item\subnoteref{x141}\NoteKeywordAgamaHead{「諸道(\ccchref{SA.1207}{https://agama.buddhason.org/SA/dm.php?keyword=1207});九十六種道;外道(GA)」},南傳作\NoteKeywordNikaya{「教條」}(pāsaṇḍaṃ,另譯為「教見(宗教之見);宗派;異學;異端;外道」),菩提比丘長老英譯為\NoteKeywordBhikkhuBodhi{「教義;信條」}(creed),並解說,這與邪見有關,譯為「creed」是不十分適當的。按:《顯揚真義》說,他們在眾生的心中投擲一個邪見圈套(diṭṭhipāsaṃ),教說(Sāsanaṃ)使之從圈套脫離,因此教條不被說,教條是[佛教]外部的。
\stopitemgroup

\startitemgroup[noteitems]
\item\subnoteref{x142}\NoteKeywordNikayaHead{「對他來說沒有前後」}(Na tassa pacchā na puratthamatthi),菩提比丘長老英譯為\NoteKeywordBhikkhuBodhi{「對他,沒有前後」}(For him there is nothing behind or in front)。按:《顯揚真義》以「對在過去未來諸蘊上意欲貪的擺脫(atītānāgatesu khandhesu chandarāgavirahitassa)」解說「沒有前後」。
\stopitemgroup

\startitemgroup[noteitems]
\item\subnoteref{x143}\NoteKeywordNikayaHead{「世尊![你說:]我是無邊之見者」}(Anantadassī bhagavāhamasmi),菩提比丘長老英譯為\NoteKeywordBhikkhuBodhi{「噢!幸福者![你說:]『我是無限視野者』」}(O Blessed One, [you say]: I am the one of infinite vision)。按:那個「我」如果是世尊而不是巴迦梵天,這樣上下文義比較合理,《顯揚真義》以「世尊!你說:『我是超越生(老死)等的無邊之見者。』」解說,今準此譯。
\stopitemgroup

\startitemgroup[noteitems]
\item\subnoteref{x144}\NoteKeywordNikayaHead{「禁戒與德行之行法」}(vatasīlavattaṃ),菩提比丘長老英譯為\NoteKeywordBhikkhuBodhi{「誓約與德行的實踐」}(practice of vow and virtue)。按:《顯揚真義》說,禁戒與德行就被稱為德行(sīlameva )。
\stopitemgroup

\startitemgroup[noteitems]
\item\subnoteref{x145}\NoteKeywordNikayaHead{「徒弟」}(baddhacaro,原意為「隨侍者;侍候者;奴隸」),菩提比丘長老英譯為\NoteKeywordBhikkhuBodhi{「徒弟」}(apprentice)。按:《顯揚真義》以「隨侍徒弟」(baddhacaro antevāsiko)出現,今準此譯。
\stopitemgroup

\startitemgroup[noteitems]
\item\subnoteref{x146}\NoteKeywordNikayaHead{「有正覺禁戒的」}(Sambuddhimantaṃ vatinaṃ),菩提比丘長老英譯為\NoteKeywordBhikkhuBodhi{「你想他是聰明的與虔誠的」}(You thought him intelligent and devout)。按:《顯揚真義》以「這位是有正覺、禁戒具足的」(‘‘sammā buddhimā vatasampanno aya’’nti)解說。
\stopitemgroup

\startitemgroup[noteitems]
\item\subnoteref{x147}\NoteKeywordAgamaHead{「阿羅漢(\ccchref{SA.1191}{https://agama.buddhason.org/SA/dm.php?keyword=1191})」},南傳作\NoteKeywordNikaya{「死亡的捨棄者」}(maccuhāyinaṃ),菩提比丘長老英譯為\NoteKeywordBhikkhuBodhi{「已丟下死亡離去」}(A thousand have left Death behind)。按:《顯揚真義》以「諸漏已滅盡的死亡永捨者」(maraṇapariccāginaṃ khīṇāsavānaṃ)解說。
\stopitemgroup

\startitemgroup[noteitems]
\item\subnoteref{x148}\NoteKeywordAgamaHead{「恐怖於妄說(\ccchref{SA.1191}{https://agama.buddhason.org/SA/dm.php?keyword=1191});畏懼不信敬(GA)」},南傳作\NoteKeywordNikaya{「對妄語的愧」}(musāvādassa ottapanti, musāvādassa ottappanti,另譯為「妄語的愧」),菩提比丘長老英譯為\NoteKeywordBhikkhuBodhi{「由於對不誠實地說話的畏懼」}(From dread of speaking falsely)。
\stopitemgroup

\startitemgroup[noteitems]
\item\subnoteref{x149}\NoteKeywordAgamaHead{「賓耆迦(\ccchref{SA.1152}{https://agama.buddhason.org/SA/dm.php?keyword=1152});卑嶷(GA)」},南傳作\NoteKeywordNikaya{「辱罵婆羅墮若」}(akkosakabhāradvājo)。按:《顯揚真義》說,他想:「哥哥(jeṭṭhakabhātaraṃ)被沙門喬答摩強奪(jāni katā)出家,伴黨已破碎(pakkho bhinno)。」他以五百偈偈頌罵如來,結集者對他取名「辱罵婆羅墮若」。
\stopitemgroup

\startitemgroup[noteitems]
\item\subnoteref{x150}\NoteKeywordAgamaHead{「阿修羅(\ccchref{SA.1151}{https://agama.buddhason.org/SA/dm.php?keyword=1151});阿脩羅鹽(GA)」},南傳作\NoteKeywordNikaya{「阿修羅王婆羅墮若」}(asurindakabhāradvājo)。按:《顯揚真義》說,他是惡罵婆羅墮若的弟弟(kaniṭṭho, 年紀較小的)。
\stopitemgroup

\startitemgroup[noteitems]
\item\subnoteref{x151}\NoteKeywordAgamaHead{「健罵婆羅豆婆遮婆(\ccchref{SA.1154}{https://agama.buddhason.org/SA/dm.php?keyword=1154});突邏闍(GA)」},這裡南傳作\NoteKeywordNikaya{「酸粥婆羅墮若」}(bilaṅgikabhāradvājo)。按:《顯揚真義》說,他想:「我的三位哥哥跟這位出家了。」(tayo me jeṭṭhakabhātaro iminā pabbājitā’’t)任何能說話者都不能保持沉默。他賣種種單純的與調味的(suddhañca sambhārayuttañca)酸粥致富,結集者對他取名「酸粥婆羅墮若」。依《顯揚真義》所說,四位婆羅墮若年紀從大到小依序為:婆羅墮若姓婆羅門、辱罵、阿修羅王、酸粥。
\stopitemgroup

\startitemgroup[noteitems]
\item\subnoteref{x152}\NoteKeywordAgamaHead{「不害(\ccchref{SA.1156}{https://agama.buddhason.org/SA/dm.php?keyword=1156});無害(GA)」},南傳作\NoteKeywordNikaya{「無害婆羅墮若」}(ahiṃsakabhāradvājo),菩提比丘長老英譯為\NoteKeywordBhikkhuBodhi{「無害」}(the Harmless)。按:《顯揚真義》說,或婆羅墮若問無害的問題,因此作結集時(saṅgītikārehi)取的名字,亦或他以無害為名,婆羅墮若為姓。
\stopitemgroup

\startitemgroup[noteitems]
\item\subnoteref{x153}\NoteKeywordAgamaHead{「明決定(\ccchref{SA.1161}{https://agama.buddhason.org/SA/dm.php?keyword=1161})」},南傳作\NoteKeywordNikaya{「已完成證智」}(abhiññāvosito),菩提比丘長老英譯為\NoteKeywordBhikkhuBodhi{「在直接的理解上成就達極點」}(consummate in direct knowledge)。
\stopitemgroup

\startitemgroup[noteitems]
\item\subnoteref{x154}\NoteKeywordNikayaHead{「殘餘供物」}(habyasesaṃ),菩提比丘長老英譯為\NoteKeywordBhikkhuBodhi{「獻祭的餅」}(sacrificial cake)。按:《顯揚真義》以「獻供的殘餘」(hutasesaṃ)解說。長老依奧義書(Upaniṣads)解說,這是指殘留在杓子、鍋子、容器的食物只能給婆羅門吃,不能再帶回家。
\stopitemgroup

\startitemgroup[noteitems]
\item\subnoteref{x155}\NoteKeywordNikayaHead{「通曉吠陀者」}(vedantagū,逐字譯為「吠陀+邊+達人」,另譯為「明智者;極智者;已達聖道者」),菩提比丘長老英譯為\NoteKeywordBhikkhuBodhi{「已達知識之極者」}(Who has reached the end of knowledge)。
\stopitemgroup

\startitemgroup[noteitems]
\item\subnoteref{x156}\NoteKeywordNikayaHead{「完全消化」}(sammā pariṇāmaṃ),菩提比丘長老英譯為\NoteKeywordBhikkhuBodhi{「適當地消化」}(properly digest)。按:《顯揚真義》說,食物拿近時,四方天神想:「大師將受用。」拿花果等、酥、生酥、油、蜜、糖,擠壓蜂巢取蜜,以神力產生滋養素(dibbānubhāvena nibbattitojameva)放入,因此到了精細情況,對以粗糙物(oḷārikaṃ vatthūti)的人來說不來到消化,又放入牛湯(Goyūse)、芝麻、熟的、粗糙混合的,對以精細物的天神來說也不來到消化,甚至乾觀諸漏已滅盡者(Sukkhavipassakakhīṇāsavassāpi)胃也不消化,只有得到八等至的諸漏已滅盡者能以等至力消化,而佛陀則能以其本能的業生火(pākatikeneva kammajatejena)消化。
\stopitemgroup

\startitemgroup[noteitems]
\item\subnoteref{x157}\NoteKeywordAgamaHead{「欲火常熾然(\ccchref{SA.1184}{https://agama.buddhason.org/SA/dm.php?keyword=1184});熾然不斷絕(GA)」},南傳作\NoteKeywordNikaya{「經常火的」}(Niccagginī),菩提比丘長老英譯為\NoteKeywordBhikkhuBodhi{「總是燃燒著」}(Always ablaze)。按:《顯揚真義》以「以被傾心吸引,以全知的智,常被燃燒的火」(āvajjanapaṭibaddhena sabbaññutaññāṇena niccaṃ pajjalitaggi)解說,今準此譯。又,「欲火」應解讀為「想要火……」。
\stopitemgroup

\startitemgroup[noteitems]
\item\subnoteref{x158}\NoteKeywordAgamaHead{「當善自調伏,消滅士夫火(\ccchref{SA.1184}{https://agama.buddhason.org/SA/dm.php?keyword=1184})」},南傳作\NoteKeywordNikaya{「善調御的自我是男子的火」}(attā sudanto purisassa joti),菩提比丘長老英譯為\NoteKeywordBhikkhuBodhi{「一個徹底馴服的自我是男子的光輝(燈火)」}(A well-tamed self is the light of a man)。按:《顯揚真義》說,「自我」即心(Attāti cittaṃ),這與北傳經文的「消滅士夫火」含意相左。
\stopitemgroup

\startitemgroup[noteitems]
\item\subnoteref{x159}\NoteKeywordAgamaHead{「淨戒為度濟(\ccchref{SA.1184}{https://agama.buddhason.org/SA/dm.php?keyword=1184});戒為津濟渡(GA)」},南傳作\NoteKeywordNikaya{「具有戒為渡場」}(sīlatittho,另譯為「戒的津岸」),菩提比丘長老英譯為\NoteKeywordBhikkhuBodhi{「具德行的淺灘」}(with fords of virtue)。按:《顯揚真義》以「對我的那個法湖來說,四遍清淨戒為渡場」(tassa pana me dhammarahadassa catupārisuddhisīlaṃ titthanti)解說。
\stopitemgroup

\startitemgroup[noteitems]
\item\subnoteref{x160}\NoteKeywordNikayaHead{「法之行者」}(dhammasārīti,另譯為「法之雲遊者」),菩提比丘長老英譯為\NoteKeywordBhikkhuBodhi{「被法推進者」}(one impelled by Dhamma)。
\stopitemgroup

\startitemgroup[noteitems]
\item\subnoteref{x161}\NoteKeywordAgamaHead{「一那羅(\ccchref{SA.98}{https://agama.buddhason.org/SA/dm.php?keyword=98})」},南傳作\NoteKeywordNikaya{「南山一蘆葦」}(dakkhiṇāgirismiṃ ekanāḷāyaṃ)。按:「一蘆葦」(ekanāḷa),顯然是「一那羅」的半義譯半音譯,《顯揚真義》說,這是該村落的名字。
\stopitemgroup

\startitemgroup[noteitems]
\item\subnoteref{x162}\NoteKeywordAgamaHead{「耕田婆羅豆婆遮(\ccchref{SA.98}{https://agama.buddhason.org/SA/dm.php?keyword=98}),耕作婆羅門名豆羅闍(GA)」},南傳作\NoteKeywordNikaya{「耕田婆羅墮若」}(kasibhāradvāja)。按:《顯揚真義》說,那位婆羅門依止耕田生活,「婆羅墮若」為姓。
\stopitemgroup

\startitemgroup[noteitems]
\item\subnoteref{x163}\NoteKeywordNikayaHead{「苦行」}(tapo),菩提比丘長老英譯為\NoteKeywordBhikkhuBodhi{「嚴格的生活」}(austerity)。按:《顯揚真義》說,而在這裡是根自制的意趣(idha pana indriyasaṃvaro adhippeto)。
\stopitemgroup

\startitemgroup[noteitems]
\item\subnoteref{x164}\NoteKeywordAgamaHead{「內藏(\ccchref{SA.98}{https://agama.buddhason.org/SA/dm.php?keyword=98})」},南傳作\NoteKeywordNikaya{「胃裡」}(udare,另譯為「在腹中」),菩提比丘長老英譯為\NoteKeywordBhikkhuBodhi{「在我的胃口」}(in my appetite)。
\stopitemgroup

\startitemgroup[noteitems]
\item\subnoteref{x165}\NoteKeywordNikayaHead{「柔和是我的脫離」}(soraccaṃ me pamocanaṃ),菩提比丘長老英譯為\NoteKeywordBhikkhuBodhi{「文雅為我的無上軛」}(gentleness as my unyoking)。按:《顯揚真義》以「而是阿羅漢果的意趣,因為那是在美妙涅槃中樂的狀態」(Arahattaphalaṃ pana adhippetaṃ. Taṃ hi sundare nibbāne ratattā)解說「柔和」,以「軛的捨離」(yogavissajjanaṃ)解說「脫離」。
\stopitemgroup

\startitemgroup[noteitems]
\item\subnoteref{x166}\NoteKeywordNikayaHead{「商量後」}(dārehi saṃpuccha),菩提比丘長老英譯為\NoteKeywordBhikkhuBodhi{「被他們的妻子煽動」}(instigated by their wives)。按:《顯揚真義》以「與自己的妻子一起商量後」(attano bhariyāhi saddhiṃ mantayitvā)解說,今準此譯。
\stopitemgroup

\startitemgroup[noteitems]
\item\subnoteref{x167}\NoteKeywordNikayaHead{「以兒子樣子」}(puttarūpena,直譯為「以兒子色」),菩提比丘長老英譯為\NoteKeywordBhikkhuBodhi{「以子之名」}(in the guise of son)。按:《顯揚真義》以「偽裝的兒子;表面上假裝的兒子」(puttavesena)解說。
\stopitemgroup

\startitemgroup[noteitems]
\item\subnoteref{x168}\NoteKeywordAgamaHead{「不時顧念(\ccchref{SA.92}{https://agama.buddhason.org/SA/dm.php?keyword=92});聊不顧視(GA)」},南傳作\NoteKeywordNikaya{「沒對他說話」}(taṃ nālapi),菩提比丘長老英譯為\NoteKeywordBhikkhuBodhi{「沒對他說話;沒理他」}(did not address him)。按:「不時顧念」就是「不顧念」;「聊不顧視」就是「一點也不看」。
\stopitemgroup

\startitemgroup[noteitems]
\item\subnoteref{x169}\NoteKeywordAgamaHead{「不善更增慢(\ccchref{SA.92}{https://agama.buddhason.org/SA/dm.php?keyword=92})」},南傳作\NoteKeywordNikaya{「慢是不好的」}(na mānaṃ brāhmaṇa sādhu),菩提比丘長老依錫蘭本(mānabrūhmaṇā)英譯為「助長自大一點也不好」(The fostering of conceit is never good)。按:錫蘭本mānabrūhmaṇā(增廣憍慢)與北傳經文的「增慢」相呼應,而南傳偈頌中的「不好」(na sādhu),古譯作「不善」,恰與北傳偈頌相同。
\stopitemgroup

\startitemgroup[noteitems]
\item\subnoteref{x170}\NoteKeywordAgamaHead{「枯摧(\ccchref{SA.1182}{https://agama.buddhason.org/SA/dm.php?keyword=1182})」},南傳作\NoteKeywordNikaya{「已枯萎」}(visūkaṃ, visukkhaṃ),菩提比丘長老依錫蘭本(visukkhaṃ)英譯為「被乾枯」(is dried up),今依錫蘭本譯。
\stopitemgroup

\startitemgroup[noteitems]
\item\subnoteref{x171}\NoteKeywordAgamaHead{「於林離林脫(\ccchref{SA.1182}{https://agama.buddhason.org/SA/dm.php?keyword=1182});於林而無林(GA)」},南傳作\NoteKeywordNikaya{「在林中無稠林的」}(vane nibbanatho,另譯為「在林中無欲的」),菩提比丘長老英譯為\NoteKeywordBhikkhuBodhi{「無林」}(Woodless),並解說這是為了保留「林」(vana)與「稠林;欲林;欲念」(vanatha, banatha)的雙關語,但後者通常指「雜染之林;污穢之林」(the woods of defilements)。按:《顯揚真義》以「無污染之林」(nikkilesavano)解說nibbanatho。
\stopitemgroup

\startitemgroup[noteitems]
\item\subnoteref{x172}\NoteKeywordNikayaHead{「世界主共住」}(lokādhipatisahabya),菩提比丘長老英譯為\NoteKeywordBhikkhuBodhi{「世界神聖主的夥伴」}(The company of the world's divine lord)。按:《顯揚真義》以「與世界主大梵天一起的情況」(lokādhipatimahābrahmunā sahabhāvaṃ)解說。
\stopitemgroup

\startitemgroup[noteitems]
\item\subnoteref{x173}\NoteKeywordNikayaHead{「無上的天界」}(tidivaṃ anuttaraṃ),菩提比丘長老英譯為\NoteKeywordBhikkhuBodhi{「無上的三重天」}(the supreme triple heaven)。按:《顯揚真義》以「關於梵天世界」(brahmalokameva sandhāyāha)解說。
\stopitemgroup

\startitemgroup[noteitems]
\item\subnoteref{x174}\NoteKeywordAgamaHead{「在家法(\ccchref{SA.97}{https://agama.buddhason.org/SA/dm.php?keyword=97}/GA.263)」},南傳作\NoteKeywordNikaya{「在家法」}(Vissaṃ dhammaṃ),菩提比丘長老英譯為\NoteKeywordBhikkhuBodhi{「熱心於家庭的實行」}(a domestic practice),並考證這裡的vissa即vesma(家;住處),正與北傳不謀而合,今準此譯。按:《顯揚真義》以「惡臭不善法」(duggandhaṃ akusaladhammaṃ)解說。而PTS巴英詞典以「像生肉的氣味」(a smell like raw flesh)解說Vissa。另,SA與GA雖都說「在家法」,但結論卻不同,GA也許從初果到第三果的在家聖者的角度稱「比丘法」。
\stopitemgroup

\startitemgroup[noteitems]
\item\subnoteref{x175}\NoteKeywordAgamaHead{「竟知何法(\ccchref{SA.1180}{https://agama.buddhason.org/SA/dm.php?keyword=1180})」},南傳作\NoteKeywordNikaya{「誰將知道集會所之法」}(ke ca sabhādhammaṃ jānissantīti),菩提比丘長老英譯為\NoteKeywordBhikkhuBodhi{「難道他們不知道會議規則嗎」}(Don't they know the rule of order),並解說他的翻譯沒緊按字面譯,而是帶出其中含意。按:世尊將此處的「法(規則)」(dhamma)與「正法」(dhamma);「集會所」(sabhā)與「善人」(santo)作相關語發揮。又,《顯揚真義》說,「集會所之法」指當[大眾]舒服坐好時(sukhanisinne)看見有進入者不移動,而非直接進入使大眾移動(mahājanaṃ cāletvā),但世尊卻直接進入,因此他們激動、輕蔑世尊。
\stopitemgroup

\startitemgroup[noteitems]
\item\subnoteref{x176}\NoteKeywordNikayaHead{「以及掛慮家的尋」}(gehasitañca vitakkaṃ),菩提比丘長老英譯為\NoteKeywordBhikkhuBodhi{「與家的心思」}(And household thoughts)。按:《顯揚真義》以「掛慮家的五種欲之惡尋」(pañcakāmaguṇagehanissitaṃ pāpavitakkañca)解說。
\stopitemgroup

\startitemgroup[noteitems]
\item\subnoteref{x177}\NoteKeywordNikayaHead{「有所覺者們」}(mutattā,另譯為「有所思者們」),菩提比丘長老英譯為\NoteKeywordBhikkhuBodhi{「賢能者們」}(The sages)。按:《顯揚真義》以「了知情況者們」(viññātattabhāvā)解說。
\stopitemgroup

\startitemgroup[noteitems]
\item\subnoteref{x178}\NoteKeywordNikayaHead{「(在)有對(上)」}(paṭighe),菩提比丘長老英譯為\NoteKeywordBhikkhuBodhi{「所感覺的」}(sensed)。按:《顯揚真義》說,在這裡,香味被把握為有對之語(ettha paṭighapadena gandharasā gahitā)。
\stopitemgroup

\startitemgroup[noteitems]
\item\subnoteref{x179}\NoteKeywordNikayaHead{「而依止六十的有尋」}(Atha saṭṭhinissitā savitakkā),菩提比丘長老英譯為\NoteKeywordBhikkhuBodhi{「而那些抓住六十者,被他們自己的心思引導」}(Then those caught in the sixty, Led by their own thoughts)。按:這裡說的六十所指不明,《顯揚真義》以「依止六所緣」(cha ārammaṇanissitā)模糊解說,長老說,可能指梵動經中說的六十二邪見。
\stopitemgroup

\startitemgroup[noteitems]
\item\subnoteref{x180}\NoteKeywordNikayaHead{「慢的道路」}(mānapathañca),菩提比丘長老英譯為\NoteKeywordBhikkhuBodhi{「自大之路」}(the pathway of conceit)。按:《顯揚真義》以「慢之所緣及與慢一起生起的法」(mānārammaṇañceva mānasahabhuno ca dhamme)解說,但長老說,它可能只是對被慢控制的一個隱喻表示(a metaphorical expression)。
\stopitemgroup

\startitemgroup[noteitems]
\item\subnoteref{x181}\NoteKeywordNikayaHead{「道的勝利者」}(maggajino),菩提比丘長老依Norman的意見英譯為「路的知道者」(A path-knower)。按:《顯揚真義》以「以道征服污染者」(maggena jitakileso)解說。
\stopitemgroup

\startitemgroup[noteitems]
\item\subnoteref{x182}\NoteKeywordAgamaHead{「修習於無相(\ccchref{SA.1214}{https://agama.buddhason.org/SA/dm.php?keyword=1214})」},南傳作\NoteKeywordNikaya{「請你修習無相」}(Animittañca bhāvehi),菩提比丘長老英譯為\NoteKeywordBhikkhuBodhi{「在無徵候上開發默想」}(Develop meditation on the signless)。按:《顯揚真義》說,常等相(niccādīnaṃ nimittānaṃ)的除去狀態名為無相毘婆舍那(vipassanā animittā)。
\stopitemgroup

\startitemgroup[noteitems]
\item\subnoteref{x183}\NoteKeywordAgamaHead{「有盡大仙人(\ccchref{SA.1212}{https://agama.buddhason.org/SA/dm.php?keyword=1212});盡於後有之大仙(GA);有盡仙(MA);無愛更不生(AA)」},南傳作\NoteKeywordNikaya{「再有已盡的仙人」}(khīṇapunabbhavā isī),菩提比丘長老英譯為\NoteKeywordBhikkhuBodhi{「已結束重新存在的先知」}(seers who have ended renewed existence)。
\stopitemgroup

\startitemgroup[noteitems]
\item\subnoteref{x184}\NoteKeywordAgamaHead{「無上商人主(\ccchref{MA.121}{https://agama.buddhason.org/MA/dm.php?keyword=121});無上商主(GA)」},南傳作\NoteKeywordNikaya{「無上的商隊領袖」}(satthavāhaṃ anuttaraṃ,另譯為「無上的商主」),菩提比丘長老英譯為\NoteKeywordBhikkhuBodhi{「無可凌駕的旅行商隊領袖」}(The unsurpassed caravan leader)。
\stopitemgroup

\startitemgroup[noteitems]
\item\subnoteref{x185}\NoteKeywordAgamaHead{「阿若拘鄰(\ccchref{SA.379}{https://agama.buddhason.org/SA/dm.php?keyword=379})」},南傳作\NoteKeywordNikaya{「阿若憍陳如」}(aññāsikoṇḍañña),「阿若」(aññā)為音譯,義譯為「已了知」,「拘鄰」(koṇḍañña)為「憍陳如」的另譯,菩提比丘長老英譯為\NoteKeywordBhikkhuBodhi{「已了知的憍陳如」}(koṇḍañña Who Has Understood)。
\stopitemgroup

\startitemgroup[noteitems]
\item\subnoteref{x186}\NoteKeywordAgamaHead{「聚落及家家(\ccchref{SA.1217}{https://agama.buddhason.org/SA/dm.php?keyword=1217});經歷諸城邑(GA)」},南傳作\NoteKeywordNikaya{「從村到村從城到城」}(gāmā gāmaṃ purā puraṃ)。《顯揚真義》說,尊者婆耆舍出家前是誦咒後以手指敲死者的頭就知道他出生在何處者,他從村到村等次第遊行,與約五百位學生一起舍衛城。當他遇到佛陀時,佛陀讓他看出生地獄、天界的[死者]頭[骨],他都正確地(tatheva)回答,然後讓他看諸漏已滅盡者的頭[骨],他一再誦咒後以手指敲,但沒看見出生處(nibbattaṭṭhānaṃ),佛陀說他知道,於是,他以學習佛陀的咒語而隨佛陀出家,大師(佛陀)為他講解三十二行相業處,當他順逆作意,毘婆舍那增大後,次第到達阿羅漢狀態(大意)。另,南傳《小部/法句經420偈註》也有關於尊者婆耆舍相同的記載。
\stopitemgroup

\startitemgroup[noteitems]
\item\subnoteref{x187}\NoteKeywordNikayaHead{「凡到達決定的看見者」}(ye niyāmagataddasā),菩提比丘長老英譯為\NoteKeywordBhikkhuBodhi{「他到達與看見固定進路」}(Who have reached and seen the fixed course),並解說「決定」(niyāma),指「正性決定」(the fixed course of rightness)。另外,本經只講到比丘、比丘尼,而《小部》〈長老偈〉相當的偈頌則說到「男人、女人、比丘、比丘尼」四眾弟子。
\stopitemgroup

\startitemgroup[noteitems]
\item\subnoteref{x188}\NoteKeywordNikayaHead{「佈滿塵土的」}(paṃsukunthito, paṃsukuṇṭhito, paṃsukuṇḍito),菩提比丘長老英譯為\NoteKeywordBhikkhuBodhi{「被土弄亂的;散落泥土的」}(littered with soil)。按:《顯揚真義》以「塵土塗抹的」(paṃsumakkhito)解說,今依此以paṃsukuṇḍito解讀。
\stopitemgroup

\startitemgroup[noteitems]
\item\subnoteref{x189}\NoteKeywordNikayaHead{「不依止的」}(asitaṃ,另譯為「獨立的」),菩提比丘長老英譯為\NoteKeywordBhikkhuBodhi{「分離的」}(detached)。按:《顯揚真義》以「不被渴愛、見之依止所依止」(taṇhādiṭṭhinissayena anissitaṃ)解說。
\stopitemgroup

\startitemgroup[noteitems]
\item\subnoteref{x190}\NoteKeywordAgamaHead{「起明斷無明(\ccchref{SA.1332}{https://agama.buddhason.org/SA/dm.php?keyword=1332})」},南傳作\NoteKeywordNikaya{「以明切斷無明後」}(Chetvā avijjaṃ vijjāya),菩提比丘長老依錫蘭本(Bhetvā avijjaṃ vijjāya)英譯為「以理解(明)突破無知(無明)」(by breaking ignorance with knowledge)。
\stopitemgroup

\startitemgroup[noteitems]
\item\subnoteref{x191}\NoteKeywordAgamaHead{「尺只(\ccchref{SA.1339}{https://agama.buddhason.org/SA/dm.php?keyword=1339});連迦(GA)」},南傳作\NoteKeywordNikaya{「獵人」}(chetaṃ,音譯近於「尺只」),菩提比丘長老英譯為\NoteKeywordBhikkhuBodhi{「獵人」}(hunter)。按:《顯揚真義》以獵鹿人(migaluddaka)解說。
\stopitemgroup

\startitemgroup[noteitems]
\item\subnoteref{x192}\NoteKeywordNikayaHead{「對在家人的撫慰」}(gihisaññatti),菩提比丘長老英譯為\NoteKeywordBhikkhuBodhi{「教導俗人」}(instructing lay people)。按:《顯揚真義》說,佛般涅槃(入滅)後,大迦葉上座對尊者阿難說,我們要在王舍城安居結集法,你去林野為得到上道目標(uparimaggattayatthāya)作努力。尊者阿難因此入一處林野住,隔天進入一個村落,人間上座(Manussā theraṃ)看見了,對他悲嘆著佛陀的入滅到黃昏,尊者阿難也持續安慰並教導他無常法等到黃昏,天神怕他不能參加法的結集,大師的教說不被收集(asaṅgahitapuppharāsi)而作督促(大意)。
\stopitemgroup

\startitemgroup[noteitems]
\item\subnoteref{x193}\NoteKeywordNikayaHead{「被天女欲求者」}(devakaññāhi patthitā”ti),菩提比丘長老依羅馬拼音一版devakaññābhisattikā英譯為「執著天女」(attached to celestial maidens)。按:《顯揚真義》沒有解說。
\stopitemgroup

\startitemgroup[noteitems]
\item\subnoteref{x194}\NoteKeywordNikayaHead{「到達深入的」}(ajjhogāḷhappatto),菩提比丘長老英譯為\NoteKeywordBhikkhuBodhi{「親密的」}(intimate)。按:《顯揚真義》說,他(比丘)在大師面前取得業處後進入叢林,在第二天白天,村子淨心者前來供給(abhikkantādīhi)飲食,某一家對他的舉止行為(iriyāpathe, 威儀路)得到淨信後五體投地禮敬、施與食物,聽聞食物[供養後]的感謝(Bhattānumodanaṃ)後更加得到淨信……。上座達成阿羅漢狀態後思惟(cintesi):這家對我有許多資助,去其他地方我將做什麼?就在那裏住,接受著證果之樂(Phalasamāpattisukhaṃ),天神不知道上座諸漏已滅盡狀態,思惟:這位上座不去其它村子、其它家,不在樹下住處等處坐,常進入這家坐,兩者親密交往(ogādhappattā paṭigādhappattā),什麼時候這位會污損這家,我要斥責(codessāmi),因此而說[偈頌]。
\stopitemgroup

\startitemgroup[noteitems]
\item\subnoteref{x195}\NoteKeywordNikayaHead{「輕心者」}(Lahucittoti),菩提比丘長老英譯為\NoteKeywordBhikkhuBodhi{「有浮躁心者」}(one with a fickle mind)。
\stopitemgroup

\startitemgroup[noteitems]
\item\subnoteref{x196}\NoteKeywordNikayaHead{「整夜之行」}(sabbaratticāro),菩提比丘長老英譯為\NoteKeywordBhikkhuBodhi{「整夜的慶典」}(an all-night festival)。按:《顯揚真義》說,宣佈迦刺底迦慶典(kattikanakkhattaṃ)後,整個城市被旗幟旗子等裝飾後,使整夜之行被持續轉起(pavattito sabbaratticāro)。
\stopitemgroup

\startitemgroup[noteitems]
\item\subnoteref{x197}\NoteKeywordAgamaHead{「覺觀所寢食(\ccchref{SA.1334}{https://agama.buddhason.org/SA/dm.php?keyword=1334});欲覺之所吞(GA)」},南傳作\NoteKeywordNikaya{「你被那個諸尋吃掉」}(so vitakkehi khajjasi),菩提比丘長老英譯為\NoteKeywordBhikkhuBodhi{「先生!你被你的心思吃了」}(You, sir, are eaten by your thoughts)。
\stopitemgroup

\startitemgroup[noteitems]
\item\subnoteref{x198}\NoteKeywordNikayaHead{「無欲地尋求食物」},參看\suttaref{SN.2.25}。
\stopitemgroup

\startitemgroup[noteitems]
\item\subnoteref{x199}\NoteKeywordAgamaHead{「如毛髮之惡(SA);有如毛髮惡(GA)」},南傳作\NoteKeywordNikaya{「對毛端大小的惡來說」}(Vālaggamattaṃ pāpassa),菩提比丘長老英譯為\NoteKeywordBhikkhuBodhi{「即使只是髮頂端的邪惡」}(Even a mere hair's tip of evil)。
\stopitemgroup

\startitemgroup[noteitems]
\item\subnoteref{x200}\NoteKeywordNikayaHead{「被黏著於洞穴(子宮)」}(sajjati gabbharasmin),菩提比丘長老英譯為\NoteKeywordBhikkhuBodhi{「在子宮被產生」}(is begotten in the womb)。按:《顯揚真義》說,因陀羅迦夜叉以為眾生在子宮突然生起(ekappahāreneva satto mātukucchismiṃ nibbattatī’’ti)。「於洞穴」(gabbharasmin),《顯揚真義》直接說是「於母親子宮」(mātukucchismiṃ),後文佛陀也這樣回答。
\stopitemgroup

\startitemgroup[noteitems]
\item\subnoteref{x201}\NoteKeywordAgamaHead{「迦羅邏(\ccchref{SA.1300}{https://agama.buddhason.org/SA/dm.php?keyword=1300});歌羅羅(GA)」},南傳作\NoteKeywordNikaya{「凝滑」}(kalalaṃ,古音譯為「羯羅藍」)。
\stopitemgroup

\startitemgroup[noteitems]
\item\subnoteref{x202}\NoteKeywordAgamaHead{「肉段(\ccchref{SA.1300}{https://agama.buddhason.org/SA/dm.php?keyword=1300});肉段(GA)」},南傳作\NoteKeywordNikaya{「肉片」}(pesi,古音譯為「閉尸」)。
\stopitemgroup

\startitemgroup[noteitems]
\item\subnoteref{x203}\NoteKeywordAgamaHead{「堅厚(\ccchref{SA.1300}{https://agama.buddhason.org/SA/dm.php?keyword=1300});堅䩕(GA)」},南傳作\NoteKeywordNikaya{「堅肉」}(ghano,古音譯為「犍南」)。
\stopitemgroup

\startitemgroup[noteitems]
\item\subnoteref{x204}\NoteKeywordAgamaHead{「肢節(\ccchref{SA.1300}{https://agama.buddhason.org/SA/dm.php?keyword=1300});五胞(GA)」},南傳作\NoteKeywordNikaya{「諸肢節」}(pasākhā,古音譯為「鉢羅奢佉」)。按:依現代醫學資料,第六週的胎兒手腳可辨。
\stopitemgroup

\startitemgroup[noteitems]
\item\subnoteref{x205}\NoteKeywordAgamaHead{「諸毛髮(\ccchref{SA.1300}{https://agama.buddhason.org/SA/dm.php?keyword=1300});髮爪(GA)」},南傳作\NoteKeywordNikaya{「以及頭髮體毛及指甲」}(kesā lomā nakhāpi ca,直譯為「頭髮、體毛及指甲」),菩提比丘長老英譯為\NoteKeywordBhikkhuBodhi{「頭髮、體毛與指甲」}(The head-hair, body-hair, and nails)。按:依現代醫學資料,第16週的胎兒長指甲,第19週的胎兒長頭髮。
\stopitemgroup

\startitemgroup[noteitems]
\item\subnoteref{x206}\NoteKeywordAgamaHead{「針毛鬼(\ccchref{SA.1324}{https://agama.buddhason.org/SA/dm.php?keyword=1324});箭毛夜叉(GA)」},南傳作\NoteKeywordNikaya{「針毛夜叉」}(sūcilomo yakkho),菩提比丘長老照錄原文,並解說夜叉的名字意思是「針毛」(Needle-hair),被這麼稱呼,是因為他的身體包覆了像針一樣的毛,《顯揚真義》說他是迦葉佛的比丘弟子,但沒有任何成就,這一生,出生在迦耶村入口的垃圾場,世尊見他有證入初果的潛力,所以進入他的領域教導他。他的領域「登居得床」(ṭaṅkitamañce),是一塊石板安在四塊石頭上的床(mañc),登居得(ṭaṅkita)為其名稱。
\stopitemgroup

\startitemgroup[noteitems]
\item\subnoteref{x207}\NoteKeywordAgamaHead{「覺想(\ccchref{SA.1314}{https://agama.buddhason.org/SA/dm.php?keyword=1314}/1324);意覺(GA)」},南傳作\NoteKeywordNikaya{「意尋」}(manovitakkā,另譯為「意覺;心之思惟」),菩提比丘長老英譯為\NoteKeywordBhikkhuBodhi{「心的心思[到處動搖一個人]」}(the mind's thoughts [Toss one around])。
\stopitemgroup

\startitemgroup[noteitems]
\item\subnoteref{x208}\NoteKeywordAgamaHead{「猶如鳩摩羅依倚於乳母(\ccchref{SA.1314}{https://agama.buddhason.org/SA/dm.php?keyword=1314})/猶如新生兒依倚於乳母(\ccchref{SA.1324}{https://agama.buddhason.org/SA/dm.php?keyword=1324});嬰孩捉母乳(GA.313)/嬰孩小兒云何生便知捉於乳(GA.323)」},南傳作\NoteKeywordNikaya{「如男童們放烏鴉」}(kumārakā dhaṅkamivossajantīti),菩提比丘長老英譯為\NoteKeywordBhikkhuBodhi{「如男孩拋擲一隻烏鴉」}(as boys toss up a crow),並說,此段梵文本作「如小男孩依靠奶媽」(kumārakā dhātrīm ivāśrayante, Ybhūś 11.1, Enomoto, CSCS, p. 59)。按:「鳩摩羅」應該就是「男童」(kumāraka)的音譯。《顯揚真義》說,男童們用繩子綁住烏鴉的腳,另一端綁在自己的手上,拋出烏鴉,烏鴉飛不遠就又跌回小男孩的腳下(大意)。
\stopitemgroup

\startitemgroup[noteitems]
\item\subnoteref{x209}\NoteKeywordAgamaHead{「如尼拘律樹處處隨所著(\ccchref{SA.1314}{https://agama.buddhason.org/SA/dm.php?keyword=1314})/如尼拘律樹展轉相拘引(\ccchref{SA.1324}{https://agama.buddhason.org/SA/dm.php?keyword=1324});如尼拘陀樹根鬚從土生然後入于地(GA.313)/如尼拘陀樹欲愛隨所著(GA.323)」},南傳作\NoteKeywordNikaya{「如從榕樹樹幹生的」}(nigrodhasseva khandhajā,另譯為「如從尼拘律樹的枝幹生起」),菩提比丘長老英譯為\NoteKeywordBhikkhuBodhi{「像榕樹的樹幹生出茁長」}(Like the trunk-born shoots of the banyan tree)。按:這是指榕樹從枝幹生出的氣根,一旦觸地後,就成長茁壯為另一株樹幹,所以老榕樹經常一大群的。
\stopitemgroup

\startitemgroup[noteitems]
\item\subnoteref{x210}\NoteKeywordAgamaHead{「如籐綿叢林(\ccchref{SA.1324}{https://agama.buddhason.org/SA/dm.php?keyword=1324});亦如摩樓多纏縛覆林樹/亦如摩樓多纏縛尼拘樹(GA)」},南傳作\NoteKeywordNikaya{「如葛藤在林中伸展的」}(māluvāva vitatā vane),菩提比丘長老英譯為\NoteKeywordBhikkhuBodhi{「像māluva攀爬物伸展越過樹林」}(Like a māluva creeper stretched across the woods)。按:此處的「摩樓多」,應該就是「葛藤;蔓草」(māluva)的音譯,這像「綠色殺手;綠癌」之稱的小葉蔓澤蘭爬藤類植物。
\stopitemgroup

\startitemgroup[noteitems]
\item\subnoteref{x211}\NoteKeywordNikayaHead{「明天是更好的」}(suve seyyo),菩提比丘長老英譯為\NoteKeywordBhikkhuBodhi{「每天更好」}(It is better each day)。按:《顯揚真義》以「每個明天是更好的,就是連續更好的」(suve suve seyyo, niccameva seyyoti)解說。
\stopitemgroup

\startitemgroup[noteitems]
\item\subnoteref{x212}\NoteKeywordNikayaHead{「鬼神所持」},南傳作\NoteKeywordNikaya{「被夜叉捉住」}(yakkhena gahito hoti),菩提比丘長老英譯為\NoteKeywordBhikkhuBodhi{「被一位夜叉佔有(附身)」}(had been possessed by a yakkha)。《顯揚真義》說,這位夜叉是「沙奴」前世的母親,同今世的母親一樣,都是要阻止沙奴放棄出家的。
\stopitemgroup

\startitemgroup[noteitems]
\item\subnoteref{x213}\NoteKeywordNikayaHead{「我們抱怨誰」}(ujjhāpayāmase),菩提比丘長老英譯為\NoteKeywordBhikkhuBodhi{「我們能向誰訴說我們的悲傷」}(To whom could we voice our grief)。按:放棄五欲出家了,又再歸來這裡想還俗,因而被夜叉捉住戲弄,我們抱怨誰?
\stopitemgroup

\startitemgroup[noteitems]
\item\subnoteref{x214}\NoteKeywordNikayaHead{「因為今日我已出脫了」}(ajjāhamhi samuggatā),菩提比丘長老英譯為\NoteKeywordBhikkhuBodhi{「今天我終於出來了」}(Today I have emerged at last)。按:「已出脫了」(samuggatā),原意為「生起, 來到存在」,《顯揚真義》將之與「已上昇」(uggatā)並列解說,並以四諦貫通(catusaccapaṭivedha)解說後句的「諸聖諦已被看見」,意指證得初果,與「已出脫了」相呼應。
\stopitemgroup

\startitemgroup[noteitems]
\item\subnoteref{x215}\NoteKeywordNikayaHead{「像喝了蜜酒那樣睡著了」}(madhupītāva seyare),菩提比丘長老英譯為\NoteKeywordBhikkhuBodhi{「他們如喝了蜂蜜酒那樣睡」}(They sleep as if they've been drinking mead),並解說,傳說喝了蜜酒會舉不起頭,只能無意識地躺著。按:《顯揚真義》以「像喝了香蜜酒那樣睡著了」(Madhupītāva seyareti gandhamadhupānaṃ pītā viya sayanti.)解說,今準此譯。
\stopitemgroup

\startitemgroup[noteitems]
\item\subnoteref{x216}\NoteKeywordNikayaHead{「應該被調御的」}(dammo),菩提比丘長老英譯為\NoteKeywordBhikkhuBodhi{「自我-控制」}(self-control)。按:「應該被調御的」,錫蘭本作「法」(dhammo),菩提比丘長老認為錫蘭本比較好,並解說本經偈頌用詞前後的對應關連:慧=法=應被調御;給予=施捨;盡責、奮起=堅固心=忍耐。
\stopitemgroup

\startitemgroup[noteitems]
\item\subnoteref{x217}\NoteKeywordNikayaHead{「帝釋!請你指示我那個殊勝的」}(taṃ me sakka varaṃ disā”ti),菩提比丘長老英譯為\NoteKeywordBhikkhuBodhi{「帝釋!授與我那個當作恩賜」}(Grant me that, Sakka, as a boon)。按:「請你指示」(disa),另一個意思是「請你賜與」,「殊勝的」(vara),另一個意思是「恩惠;恩寵」。《顯揚真義》對這句的解說為:帝釋!最上的天!請你為我告知,請你講述那個殊勝的、最上的處所、場所的方位(sakka devaseṭṭha, taṃ me varaṃ uttamaṃ ṭhānaṃ okāsaṃ disaṃ ācikkha kathehīti),今準此譯。
\stopitemgroup

\startitemgroup[noteitems]
\item\subnoteref{x218}\NoteKeywordNikayaHead{「不衰退」}(na jīvati,原文為「不活命;無生計」),菩提比丘長老依羅馬拼音版na jīyati英譯為「不衰退」(won't decline)。按:na jīvati與前後文意不通,今依na jīyati譯。
\stopitemgroup

\startitemgroup[noteitems]
\item\subnoteref{x219}\NoteKeywordNikayaHead{「你們應該使它輝耀」}(taṃ…sobhetha),菩提比丘長老英譯為\NoteKeywordBhikkhuBodhi{「超出多少會適合你們」}(how much more would it be fitting [here] for you),並解說這個動詞(一種心聲,第三人稱單數願望式)經常出現在佛陀述說一種比丘們應該能優於在家人行為類型的經文,因此,動詞指向一個人應該如何行動而使自己發光,也就是說,適合一個人身分的行為模式。
\stopitemgroup

\startitemgroup[noteitems]
\item\subnoteref{x220}\NoteKeywordAgamaHead{「以五繫縛(\ccchref{SA.1110}{https://agama.buddhason.org/SA/dm.php?keyword=1110});以五縛繫(GA);身為五繫(AA)」},南傳作\NoteKeywordNikaya{「以脖子為第五繫縛」}(kaṇṭhapañcamehi bandhanehi),菩提比丘長老英譯為\NoteKeywordBhikkhuBodhi{「綁他的四肢與頸部」}(bind…by his four limbs and neck)。
\stopitemgroup

\startitemgroup[noteitems]
\item\subnoteref{x221}\NoteKeywordNikayaHead{「弱者總是忍耐」}(niccaṃ khamati dubbalo),菩提比丘長老英譯為\NoteKeywordBhikkhuBodhi{「弱者經常必須忍耐」}(The weakling must be patient always),並解說,對弱者,忍耐是必須的而非德行,但強者的忍耐卻顯示其德行,本偈是難[理解]的。
\stopitemgroup

\startitemgroup[noteitems]
\item\subnoteref{x222}\NoteKeywordAgamaHead{「舊天(\ccchref{SA.1109}{https://agama.buddhason.org/SA/dm.php?keyword=1109})」},南傳作\NoteKeywordNikaya{「以前的天神」}(pubbadevā),菩提比丘長老英譯為\NoteKeywordBhikkhuBodhi{「前任天神」}(the senior deva)。按:《顯揚真義》說阿修羅是「久住於天世界者,以前的統治者」(devaloke ciranivāsino pubbasāmikā),註疏說阿修羅是「在帝釋為首的天眾世界前就生起的情況(uppannattā)」。
\stopitemgroup

\startitemgroup[noteitems]
\item\subnoteref{x223}\NoteKeywordNikayaHead{「以轅桿的前端」}(isāmukhena, īsāmukhena),菩提比丘長老英譯為\NoteKeywordBhikkhuBodhi{「以你的二輪戰車桿」}(with your chariot pole)。按:「轅桿」(īsā),為馬車、牛車等車前的兩根長直木,其前端搭架於軛上,用以拉駕馬牛,應該也可以控制車行進方向。
\stopitemgroup

\startitemgroup[noteitems]
\item\subnoteref{x224}\NoteKeywordNikayaHead{「已完成的目標是閃亮的」}(Nipphannasobhano attho),菩提比丘長老依錫蘭本(Nipphannasobhano atthā)英譯為「當已達成時而閃耀的目標」(Of goals that shine when achieved)。按:《顯揚真義》以「這些已完成的目標正輝耀」(ime atthā nāma nipphannāva sobhanti)解說。
\stopitemgroup

\startitemgroup[noteitems]
\item\subnoteref{x225}\NoteKeywordAgamaHead{「世間諸和合,及與第一義(\ccchref{SA.1119}{https://agama.buddhason.org/SA/dm.php?keyword=1119})」},南傳作\NoteKeywordNikaya{「然而一切生物的受用,結合是最高的」}(Saṃyogaparamā tveva, sambhogā sabbapāṇinaṃ,逐字譯為「結合+最高、然而,受用(共食;共事)、一切+生物」),菩提比丘長老英譯為\NoteKeywordBhikkhuBodhi{「但在享受中,對所有生物的結合是無上的」}(But for all creatures association Is supreme among enjoyments),但說這段話的意思模糊,而在《本生經》中也出現這段偈,意思與其他人結交,在AN.IV 57-58[\ccchref{AN.7.51}{https://agama.buddhason.org/AN/an.php?keyword=7.51}]中「結合」(Saṃyoga)的意思是男女的結合(有關性的,但不必然是性交),而長老的理解是「享受以結合為無上的」(enjoyments have association as supreme),而非「經由結合享受成為無上的」(through association enjoyments become supreme)。按:《顯揚真義》朝米飯等(odanādīni)食物共食的方向解說,也難理解整體經文含意。
\stopitemgroup

\startitemgroup[noteitems]
\item\subnoteref{x226}\NoteKeywordAgamaHead{「諸仙人(\ccchref{SA.1115}{https://agama.buddhason.org/SA/dm.php?keyword=1115});諸仙(GA)」},南傳作\NoteKeywordNikaya{「仙人」}(isayo),菩提比丘長老英譯為\NoteKeywordBhikkhuBodhi{「先知們」}(seers)。
\stopitemgroup

\startitemgroup[noteitems]
\item\subnoteref{x227}\NoteKeywordAgamaHead{「施無畏(\ccchref{SA.1115}{https://agama.buddhason.org/SA/dm.php?keyword=1115});施我等無畏(GA)」},南傳作\NoteKeywordNikaya{「無恐怖之施(無畏施)」}(abhayadakkhiṇaṃ),菩提比丘長老英譯為\NoteKeywordBhikkhuBodhi{「安全的保證」}(a guarantee of safety),並說,傳說阿修羅與天神的戰爭大多發生在大海,阿修羅常敗,逃走時經過仙人們隱居處,會大肆破壞住屋與道路,因為阿修羅相信仙人們獻策給天帝釋而擊敗他們。
\stopitemgroup

\startitemgroup[noteitems]
\item\subnoteref{x228}\NoteKeywordAgamaHead{「數數行施(\ccchref{SA.1106}{https://agama.buddhason.org/SA/dm.php?keyword=1106});數數施(GA)」},南傳作\NoteKeywordNikaya{「在城市施與布施」}(pure dānaṃ adāsi),菩提比丘長老依錫蘭本(pure pure dānaṃ adāsi)英譯為「他在一個城市又一個城市給贈與」(he gave gifts in city after city)。
\stopitemgroup

\startitemgroup[noteitems]
\item\subnoteref{x229}\NoteKeywordAgamaHead{「富蘭陀羅(\ccchref{SA.1106}{https://agama.buddhason.org/SA/dm.php?keyword=1106});富蘭但那(GA)」},南傳作\NoteKeywordNikaya{「城市施與者」}(purindado, pure-inda-da,音譯為「富蘭陀羅」,逐字譯為「城市-帝王(因陀羅)-施與」,水野弘元譯為「城的破者」),菩提比丘長老英譯為\NoteKeywordBhikkhuBodhi{「城市給予者」}(the Urban Giver)。按:「城市施與者」(purindad)為「在城市施與布施」(pure dānaṃ adāsi)之雙關語。
\stopitemgroup

\startitemgroup[noteitems]
\item\subnoteref{x230}\NoteKeywordAgamaHead{「娑婆婆(\ccchref{SA.1106}{https://agama.buddhason.org/SA/dm.php?keyword=1106});婆娑婆(GA)」},南傳作\NoteKeywordNikaya{「襪瑟哇」}(vāsavo,也可音譯為「婆娑婆」),菩提比丘長老英譯照錄原文。按:「襪瑟哇」(vāsavo)為「住處」(āvasathaṃ)之近音雙關語。
\stopitemgroup

\startitemgroup[noteitems]
\item\subnoteref{x231}\NoteKeywordAgamaHead{「舍脂鉢低(\ccchref{SA.1106}{https://agama.buddhason.org/SA/dm.php?keyword=1106});舍脂夫(GA)」},南傳作\NoteKeywordNikaya{「須闍之夫」}(sujampati, suja-pati),菩提比丘長老英譯為\NoteKeywordBhikkhuBodhi{「須闍之夫」}(sujā's husband)。按:「夫;丈夫」(pati),可以音譯為「鉢低」。「須闍」(suja),傳為阿修羅王之妹。
\stopitemgroup

\startitemgroup[noteitems]
\item\subnoteref{x232}\NoteKeywordAgamaHead{「瞋恚對治鬼(\ccchref{SA.1107}{https://agama.buddhason.org/SA/dm.php?keyword=1107});神妙之鬼(AA)」},南傳作\NoteKeywordNikaya{「食憤怒之夜叉」}(kodhabhakkho yakkho,逐字譯為「憤怒+食的-夜叉」),菩提比丘長老英譯為\NoteKeywordBhikkhuBodhi{「食憤怒之夜叉」}(the anger-eating yakkha)。按:《顯揚真義》說,他是一位色界的梵天眾,聽說天帝釋具備忍辱力,聽了後為考察而來(vīmaṃsanatthaṃ āgato),而且懷惡意夜叉(avaruddhakayakkhā)不可能進入像這樣有守護布置的地方。
\stopitemgroup

\startitemgroup[noteitems]
\item\subnoteref{x233}\NoteKeywordNikayaHead{「以轉起的」}(āvattena),菩提比丘長老英譯為\NoteKeywordBhikkhuBodhi{「憤怒之輪」}(anger's whirl)。按:《顯揚真義》以「以憤怒轉起的」(kodhāvattena)、「因憤怒控制而轉」(kodhavase vattetuṃ) 解說。
\stopitemgroup

\startitemgroup[noteitems]
\item\subnoteref{x234}\NoteKeywordAgamaHead{「善法(\ccchref{AA.45.5}{https://agama.buddhason.org/AA/dm.php?keyword=45.5})」},南傳作\NoteKeywordNikaya{「諸法」}(dhammāni),菩提比丘長老英譯為\NoteKeywordBhikkhuBodhi{「德行」}(virtues),並解說這裡的「dhammāni(一般譯為「法」)」是「指個人德行,非精神的教義(教導)(法)」,在《本生經》V172/221中就是這樣的用法。
\stopitemgroup

\startitemgroup[noteitems]
\item\subnoteref{x235}\NoteKeywordAgamaHead{「地獄(\ccchref{SA.1118}{https://agama.buddhason.org/SA/dm.php?keyword=1118});盧樓地獄(GA)」},南傳作\NoteKeywordNikaya{「恐怖地獄」}(nirayaṃ ghoraṃ,另譯為「可怕地獄」),菩提比丘長老英譯為\NoteKeywordBhikkhuBodhi{「可怕地獄」}(terrible hell)。按:「盧樓」,或為「恐怖;可怕」(ghora)的音譯。
\stopitemgroup

\startitemgroup[noteitems]
\item\subnoteref{x236}\NoteKeywordAgamaHead{「念(\ccchref{AA.49.5}{https://agama.buddhason.org/AA/dm.php?keyword=49.5})」},南傳作\NoteKeywordNikaya{「思」}(cetanā),菩提比丘長老英譯為\NoteKeywordBhikkhuBodhi{「意志」}(volition)。按:《顯揚真義》說,思、觸、作意應該被經驗為行蘊(saṅkhārakkhandho veditabbo)。
\stopitemgroup

\startitemgroup[noteitems]
\item\subnoteref{x237}\NoteKeywordAgamaHead{「若死;若遷(\ccchref{SA.285}{https://agama.buddhason.org/SA/dm.php?keyword=285})/自滅、自沒(\ccchref{SA.366}{https://agama.buddhason.org/SA/dm.php?keyword=366})」},南傳作\NoteKeywordNikaya{「死亡、死沒」}(mīyati ca cavati ca),菩提比丘長老英譯為\NoteKeywordBhikkhuBodhi{「以及死,它死去」}(and dies, it passes away)。
\stopitemgroup

\startitemgroup[noteitems]
\item\subnoteref{x238}北傳\ccchref{SA.285}{https://agama.buddhason.org/SA/dm.php?keyword=285},\ccchref{SA.287}{https://agama.buddhason.org/SA/dm.php?keyword=287},\ccchref{SA.366}{https://agama.buddhason.org/SA/dm.php?keyword=366}都沒有「集!集!」(‘samudayo, samudayo’ti)與「滅!滅!」(‘nirodho, nirodho’ti),這樣依「苦、集、滅、道」之「四聖諦」的述說層次就沒有南傳經文的明顯。
\stopitemgroup

\startitemgroup[noteitems]
\item\subnoteref{x239}\NoteKeywordAgamaHead{「若死;若遷(\ccchref{SA.285}{https://agama.buddhason.org/SA/dm.php?keyword=285})/自滅、自沒(\ccchref{SA.366}{https://agama.buddhason.org/SA/dm.php?keyword=366})」},南傳作\NoteKeywordNikaya{「死亡;死沒」}(mīyati ca cavati ca),菩提比丘長老英譯為\NoteKeywordBhikkhuBodhi{「以及死,它死去」}(and dies, it passes away)。
\stopitemgroup

\startitemgroup[noteitems]
\item\subnoteref{x240}\NoteKeywordAgamaHead{「頗求那(\ccchref{SA.372}{https://agama.buddhason.org/SA/dm.php?keyword=372});牟利破群㝹(\ccchref{MA.23}{https://agama.buddhason.org/MA/dm.php?keyword=23});牟犁破群那(\ccchref{MA.193}{https://agama.buddhason.org/MA/dm.php?keyword=193})」},南傳作\NoteKeywordNikaya{「摩利亞帕辜那」}(moḷiyaphagguno)。按:《顯揚真義》說,摩利被叫做髮髻,當他為在家者時有個大[髮髻]而起的稱呼,出了家仍以此名稱呼,他後來還俗了,參看\ccchref{MA.23}{https://agama.buddhason.org/MA/dm.php?keyword=23}。
\stopitemgroup

\startitemgroup[noteitems]
\item\subnoteref{x241}\NoteKeywordAgamaHead{「有食識者(\ccchref{SA.372}{https://agama.buddhason.org/SA/dm.php?keyword=372})」},南傳作\NoteKeywordNikaya{「某人食用」}(‘āhāretī’ti,原義為「他食用」),菩提比丘長老英譯為\NoteKeywordBhikkhuBodhi{「某人消耗(吃)」}(One consumes),並解說這裡的「某人」隱喻「真我」(self)。按:《顯揚真義》以「我不說『任何眾生或個人(補特伽羅)吃』」(ahaṃ koci satto vā puggalo vā āhāretīti na vadāmi)解說。
\stopitemgroup

\startitemgroup[noteitems]
\item\subnoteref{x242}\NoteKeywordAgamaHead{「為誰觸(\ccchref{SA.372}{https://agama.buddhason.org/SA/dm.php?keyword=372})」},南傳作\NoteKeywordNikaya{「誰觸呢」}(Ko nu kho……phusatīti),菩提比丘長老英譯為\NoteKeywordBhikkhuBodhi{「誰作接觸」}(who makes contact)。按:南傳經文的「觸」(phusati)為動詞,十二緣起支的「觸」(phassa)為名詞。
\stopitemgroup

\startitemgroup[noteitems]
\item\subnoteref{x243}\NoteKeywordAgamaHead{「阿支羅(\ccchref{SA.302}{https://agama.buddhason.org/SA/dm.php?keyword=302})」},南傳作\NoteKeywordNikaya{「裸行者」}(acelo),「阿支羅」顯然就是「裸行者」(acela)的音譯。
\stopitemgroup

\startitemgroup[noteitems]
\item\subnoteref{x244}\NoteKeywordAgamaHead{「有那個感受,那位感受」(sā vedanā, so vedayatī),菩提比丘長老英譯為「感受與感受它者是相同的」(The feeling and the one who feels it are the same)。按:此句為「他感受那個感受;感受即是感受者」之意,《顯揚真義》以常恆(sassataṃ)解說,參看\ccchref{SA.300}{https://agama.buddhason.org/SA/dm.php?keyword=300}「自作自覺」}。
\stopitemgroup

\startitemgroup[noteitems]
\item\subnoteref{x245}\NoteKeywordNikayaHead{「有另一個感受,另一位感受」}(aññā vedanā, añño vedayatī),菩提比丘長老英譯為\NoteKeywordBhikkhuBodhi{「感受是一個,感受它者是另一個」}(The feeling is one, the one who feels it is another)。按:此句為「他感受另一個感受;感受是一,感受者是另一」之意,《顯揚真義》以斷滅(ucchedaṃ)解說此句。
\stopitemgroup

\startitemgroup[noteitems]
\item\subnoteref{x246}\NoteKeywordAgamaHead{「內有此識身,外有名色(\ccchref{SA.294}{https://agama.buddhason.org/SA/dm.php?keyword=294})」},南傳作\NoteKeywordNikaya{「這個身與外名色」}(ayañceva kāyo bahiddhā ca nāmarūpaṃ),菩提比丘長老英譯為\NoteKeywordBhikkhuBodhi{「有這個身與外部的名-與-色」}(there is this body and external name-and-form)。按:《顯揚真義》說,這個身是自己的有識身(attano saviññāṇako kāyo),外名色是有識身外的對面(bahiddhā ca paresaṃ saviññāṇako kāyo,即被認識的對象)。另按:被認識的對象中,前五根所認識的「色、聲、氣味、味道、所觸」屬「色」,意根所認識的「受、想、行」(別法)屬「名」,參看印順法師《佛法概論》p.6。另外,印順法師認為「十二緣起支,是受此說影響的。」(《初期大乘佛教之起源與開展》p.239)另參看\ccchref{SA.198}{https://agama.buddhason.org/SA/dm.php?keyword=198}「內識身」(有識之身)。
\stopitemgroup

\startitemgroup[noteitems]
\item\subnoteref{x247}\NoteKeywordAgamaHead{「此二因緣生觸(\ccchref{SA.294}{https://agama.buddhason.org/SA/dm.php?keyword=294})」},南傳作\NoteKeywordNikaya{「緣於一對而有觸」}(dvayaṃ paṭicca phasso),菩提比丘長老英譯為\NoteKeywordBhikkhuBodhi{「依於一對而有接觸」}(Dependent on the dyad there is contact),長老認為,這裡說的一對,並非《顯揚真義》解說的「大的一對」(Mahādvayaṃ):六根對六境「生起觸」的「一對」,而是指「識身」與「外名色」。
\stopitemgroup

\startitemgroup[noteitems]
\item\subnoteref{x248}\NoteKeywordAgamaHead{「還復受身(\ccchref{SA.294}{https://agama.buddhason.org/SA/dm.php?keyword=294})」},南傳作\NoteKeywordNikaya{「是身體的轉生者」}(kāyūpago hoti),菩提比丘長老英譯為\NoteKeywordBhikkhuBodhi{「行至[另外的]身體」}(fares on to [another] body)。按:《顯揚真義》以「到達(upagantā hoti)另一個再生的身體者」解說。
\stopitemgroup

\startitemgroup[noteitems]
\item\subnoteref{x249}\NoteKeywordAgamaHead{「法界(\ccchref{SA.296}{https://agama.buddhason.org/SA/dm.php?keyword=296})」},南傳作\NoteKeywordNikaya{「那個界住立」}(ṭhitāva sā dhātu),菩提比丘長老英譯為\NoteKeywordBhikkhuBodhi{「那元素仍然持續」}(that element still persists),或「規律持續」(there persists that law, \ccchref{AN.3.137}{https://agama.buddhason.org/AN/an.php?keyword=3.137})。按:《顯揚真義》以「那個緣的自性是住立的,非偶爾生不是老死的緣」(ṭhitova so paccayasabhāvo, na kadāci jāti jarāmaraṇassa paccayo na hoti.)解說。
\stopitemgroup

\startitemgroup[noteitems]
\item\subnoteref{x250}\NoteKeywordAgamaHead{「懈怠苦住(\ccchref{SA.348}{https://agama.buddhason.org/SA/dm.php?keyword=348})」},南傳作\NoteKeywordNikaya{「怠惰者住於苦」}(Dukkhaṃ……kusīto viharati),菩提比丘長老英譯為\NoteKeywordBhikkhuBodhi{「怠惰的人住在苦中」}(the lazy person dwells in suffering)。
\stopitemgroup

\startitemgroup[noteitems]
\item\subnoteref{x251}\NoteKeywordNikayaHead{「身故思之因」}(kāyasañcetanāhetu),菩提比丘長老英譯為\NoteKeywordBhikkhuBodhi{「因為身體的意志」}(because of bodily volition)。
\stopitemgroup

\startitemgroup[noteitems]
\item\subnoteref{x252}\NoteKeywordNikayaHead{「田」}(khetta),菩提比丘長老英譯為\NoteKeywordBhikkhuBodhi{「土地;舞台」}(field)。按:《顯揚真義》等以「生長義」(viruhanaṭṭhena, \suttaref{SN.12.25}, \ccchref{AN.4.171}{https://agama.buddhason.org/AN/an.php?keyword=4.171})解說。
\stopitemgroup

\startitemgroup[noteitems]
\item\subnoteref{x253}\NoteKeywordNikayaHead{「地」}(vatthu),菩提比丘長老英譯為\NoteKeywordBhikkhuBodhi{「位置;地基」}(site)。按:《顯揚真義》等以「建立義」(patiṭṭhānaṭṭhena, \suttaref{SN.12.25}, \ccchref{AN.4.171}{https://agama.buddhason.org/AN/an.php?keyword=4.171})解說
\stopitemgroup

\startitemgroup[noteitems]
\item\subnoteref{x254}\NoteKeywordNikayaHead{「處」}(āyatana,另譯為「入處」),菩提比丘長老英譯為\NoteKeywordBhikkhuBodhi{「基地」}(the base)。按:《顯揚真義》等以「緣義」(paccayaṭṭhena, \suttaref{SN.12.25}, \ccchref{AN.4.171}{https://agama.buddhason.org/AN/an.php?keyword=4.171})解說。
\stopitemgroup

\startitemgroup[noteitems]
\item\subnoteref{x255}\NoteKeywordNikayaHead{「事件」}(adhikaraṇa,另有「問題、爭論、訴訟、控告」等義),菩提比丘長老英譯為\NoteKeywordBhikkhuBodhi{「基礎」}(foundation)。按:《顯揚真義》等以「原因義」(kāraṇaṭṭhena, \suttaref{SN.12.25}, \ccchref{AN.4.171}{https://agama.buddhason.org/AN/an.php?keyword=4.171})解說。
\stopitemgroup

\startitemgroup[noteitems]
\item\subnoteref{x256}\NoteKeywordAgamaHead{「超度(\ccchref{SA.354}{https://agama.buddhason.org/SA/dm.php?keyword=354})」},南傳作\NoteKeywordNikaya{「超越」}(samatikkamma,另譯作「克服」),菩提比丘長老英譯為\NoteKeywordBhikkhuBodhi{「已超越」}(having transcended)。
\stopitemgroup

\startitemgroup[noteitems]
\item\subnoteref{x257}\NoteKeywordAgamaHead{「波羅延耶阿逸多所問(\ccchref{SA.345}{https://agama.buddhason.org/SA/dm.php?keyword=345});在波羅延阿逸多所問中(\suttaref{SN.12.31})」}(pārāyane ajitapañhe,另譯為「在彼岸道-ajita的問題中」),〈波羅延〉(pārāyana,彼岸道)為十六位婆羅門學徒與佛陀的問答集,後來收在《小部》第五〈經集〉的最後一品(第五品),阿逸多(ajita)為十六位婆羅門學徒裡面發問的第一位。
\stopitemgroup

\startitemgroup[noteitems]
\item\subnoteref{x258}\NoteKeywordAgamaHead{「若得諸法數(\ccchref{SA.345}{https://agama.buddhason.org/SA/dm.php?keyword=345})」},南傳作\NoteKeywordNikaya{「凡法的察悟者們」}(Ye ca saṅkhātadhammāse),菩提比丘長老英譯為\NoteKeywordBhikkhuBodhi{「那些理解法者」}(Those who comprehended the Dhamma)。按:《小義釋》說,法的察悟者被稱為阿羅漢、諸漏已滅盡者(saṅkhātadhammā vuccanti arahanto khīṇāsavā)。長老說,這裡的法(dhamma),可以理解為世尊的教法,也可以理解為蘊處界等一切現象的法則,如無常、苦、非我等。
\stopitemgroup

\startitemgroup[noteitems]
\item\subnoteref{x259}\NoteKeywordAgamaHead{「若復種種學(\ccchref{SA.345}{https://agama.buddhason.org/SA/dm.php?keyword=345})」},南傳作\NoteKeywordNikaya{「凡個個有學們」}(ye ca sekkhā puthū),菩提比丘長老英譯為\NoteKeywordBhikkhuBodhi{「與種種訓練中者們」}(And the manifold trainees)。按:\ccchref{MA.127}{https://agama.buddhason.org/MA/dm.php?keyword=127}從「初果向」到「阿羅漢向」列有18階位,即「18學人(有學)」。
\stopitemgroup

\startitemgroup[noteitems]
\item\subnoteref{x260}\NoteKeywordAgamaHead{「真實(\ccchref{SA.345}{https://agama.buddhason.org/SA/dm.php?keyword=345});真說(\ccchref{MA.201}{https://agama.buddhason.org/MA/dm.php?keyword=201})」},南傳作\NoteKeywordNikaya{「『這是已生成的』」}(Bhūtamidanti),菩提比丘長老英譯為\NoteKeywordBhikkhuBodhi{「這已經生成」}(This has come to be)。按:「已生成的」(bhūta)為「有(存在)」(bhavati)的過去分詞語態,表示「已存在的;這是已生成的(有情)」,還可引伸出「真實(事實),眾生,元素(大種,如「四大」的「大」),鬼神,萬物(已經發生的事物)」,這裡如果譯為「真實」或「真說」,就難理解了。這裡《顯揚真義》以「已生成的五蘊」(nibbattaṃ khandhapañcakaṃ, \suttaref{SN.12.31})解說,《破斥猶豫》以「這是已生的、已出生的、已生成的五蘊」(idaṃ khandhapañcakaṃ jātaṃ bhūtaṃ nibbattaṃ, \ccchref{MN.38}{https://agama.buddhason.org/MN/dm.php?keyword=38})解說。
\stopitemgroup

\startitemgroup[noteitems]
\item\subnoteref{x261}\NoteKeywordAgamaHead{「生因盡已(\ccchref{MA.23}{https://agama.buddhason.org/MA/dm.php?keyword=23})」},南傳作\NoteKeywordNikaya{「我是已盡者」}(khīṇāmhīti,解讀為khīṇa amhi iti),菩提比丘長老英譯為\NoteKeywordBhikkhuBodhi{「[結果]被滅盡」}([the effect] is destroyed)。按:《顯揚真義》以「出生的諸緣、被稱為出生的果被滅盡。」(jātiyā paccaye jātisaṅkhātaṃ phalaṃ khīṇanti)解說。
\stopitemgroup

\startitemgroup[noteitems]
\item\subnoteref{x262}\NoteKeywordAgamaHead{「我自於內背而不向(\ccchref{MA.23}{https://agama.buddhason.org/MA/dm.php?keyword=23})」},南傳作\NoteKeywordNikaya{「從自身內的解脫」}(ajjhattaṃ vimokkhā),菩提比丘長老英譯為\NoteKeywordBhikkhuBodhi{「經由一個內在的釋放」}(through an internal deliverance)。按:《顯揚真義》說,掌握自身內的諸行(ajjhattasaṅkhāre)後,阿羅漢境界被達到。另參看《清淨道論》〈21.行道智見清淨解說〉中的行捨智段。「背而不向;背不向」可能是「背捨」的另譯,也就是南傳經文的「解脫」(vimokkhā),因為「八解脫」(aṭṭha vimokkhā)另譯為「八背捨」。又,《阿毘曇心論》:「解脫者,八解脫:未除色想不淨思惟一,除色想不淨思惟二,淨思惟三,四無色及滅盡定,境界背不向故說解脫。」
\stopitemgroup

\startitemgroup[noteitems]
\item\subnoteref{x263}\NoteKeywordNikayaHead{「導」}(neti,另譯為「達到……的結論」),菩提比丘長老英譯為\NoteKeywordBhikkhuBodhi{「應用」}(applies)。
\stopitemgroup

\startitemgroup[noteitems]
\item\subnoteref{x264}\NoteKeywordAgamaHead{「斷知智;斷智(\ccchref{SA.357}{https://agama.buddhason.org/SA/dm.php?keyword=357})」},南傳作\NoteKeywordNikaya{「智」}(ñāṇaṃ),菩提比丘長老英譯為\NoteKeywordBhikkhuBodhi{「理解」}(The knowledge)。按:《八犍度論》譯此為「斷智慧法;斷智慧;一等智」(T.26p.836),《發智論》譯為「遍知」(T.26p.969),《大毘婆沙論》亦同(T.27p.571),《舍利弗阿毘曇論》缺(T.28p.606)。從\ccchref{SA.112}{https://agama.buddhason.org/SA/dm.php?keyword=112}與\suttaref{SN.23.4}的比對知道「斷知」南傳作\NoteKeywordNikaya{「遍知」},與上引諸論合。長老說,在經典中,只有解脫阿羅漢才適合說「遍知」(\suttaref{SN.22.24}),而理解「此(法住)智」也屬於[會]破壞之智,《顯揚真義》稱之為「反觀之觀」(vipassanāpaṭivipassanā),即「就剛識知最初所緣消溶的洞察智行為洞察入消溶」(insight into the dissolution of the very act of insight knowledge that had just cognized the dissolution of the primary object, \suttaref{SN.12.34})。後來《般若經》的「空空」、「空空不可得」或源於此類思想。
\stopitemgroup

\startitemgroup[noteitems]
\item\subnoteref{x265}\NoteKeywordAgamaHead{「若思量(\ccchref{SA.253}{https://agama.buddhason.org/SA/dm.php?keyword=253})」},南傳作\NoteKeywordNikaya{「凡他意圖」}(Yañca……ceteti),菩提比丘長老英譯為\NoteKeywordBhikkhuBodhi{「人之所意圖」}(what one intends)。按:ceteti; cetayati; cinteti為動詞,一般譯為「想;思考;考慮」,菩提比丘長老英譯為動詞「打算;意圖」(intend),而將其相關的名詞cetanā(古譯為「思」,\suttaref{SN.12.38}的經名就用這個字)英譯為「意志;意志力;決心;意欲」(volition),另外以intend的名詞intention(意志;思惟;意圖)來譯另一個與ceteti無關的名詞saṅkappa(其動詞為saṅkappeti在巴利語中少用),原因是英文中缺乏對應volition的動詞之故,所以「思量」其實是動詞,有「意圖」的意思,而這裡《顯揚真義》以三界的善不善意志(tebhūmakakusalākusalacetanā)解說。
\stopitemgroup

\startitemgroup[noteitems]
\item\subnoteref{x266}\NoteKeywordNikayaHead{「這個識存續的所緣存在」}(ārammaṇametaṃ hoti viññāṇassa ṭhitiyā),菩提比丘長老英譯為\NoteKeywordBhikkhuBodhi{「這成為為了維持識的基礎」}(this becomes a basis for the maintenance of consciousness)。按:《顯揚真義》以業識( kammaviññāṇassa)的存續解說識存續,以其緣(tasmiṃ paccaye,其條件)解說所緣(有別於止觀的所緣)。
\stopitemgroup

\startitemgroup[noteitems]
\item\subnoteref{x267}\NoteKeywordNikayaHead{「凡他意圖」}、「這個識存續的所緣存在」,參看\ccchref{SA.350}{https://agama.buddhason.org/SA/dm.php?keyword=350}/\suttaref{SN.12.38}。
\stopitemgroup

\startitemgroup[noteitems]
\item\subnoteref{x268}\NoteKeywordNikayaHead{「順世派」}(lokāyatiko),菩提比丘長老英譯為\NoteKeywordBhikkhuBodhi{「一位宇宙學家」}(who was a cosmologist)。
\stopitemgroup

\startitemgroup[noteitems]
\item\subnoteref{x269}\NoteKeywordNikayaHead{「福行」}(puññaṃ ce saṅkhāraṃ),菩提比丘長了英譯為「有福的意志形成」(generates a meritorious volitional formation)。按:《顯揚真義》以「十三種思的福德作為」(terasacetanābhedaṃ puññābhisaṅkhāraṃ)解說,長老夾註為「即八欲界善心與五色界善心」(參看《阿毘達摩精解》第一章十三、十八節)。
\stopitemgroup

\startitemgroup[noteitems]
\item\subnoteref{x270}\NoteKeywordAgamaHead{「無所有行/非福不福行(\ccchref{SA.292}{https://agama.buddhason.org/SA/dm.php?keyword=292})」},南傳作\NoteKeywordNikaya{「不動行」}(āneñjaṃ saṅkhāraṃ),菩提比丘長老英譯為\NoteKeywordBhikkhuBodhi{「泰然自若之意志的形成」}(imperturbable volitional formation)。按:《顯揚真義》以「四種思、不動的造作」(catucetanābhedaṃ āneñjābhisaṅkhāraṃ)解說,長老夾註為「即四無色界的善心」(i.e., in the four wholesome cittas of the formless sphere)。
\stopitemgroup

\startitemgroup[noteitems]
\item\subnoteref{x271}\NoteKeywordNikayaHead{「當不造作、不思時」}(Anabhisaṅkharonto anabhisañcetayanto),菩提比丘長老英譯為\NoteKeywordBhikkhuBodhi{「因為他不產生或製造意志的形成」}(since he does not generate or fashion volitional formations)。
\stopitemgroup

\startitemgroup[noteitems]
\item\subnoteref{x272}\NoteKeywordNikayaHead{「陶瓷碎片會剩下」}(kapallāni avasisseyyuṃ),菩提比丘長老英譯為\NoteKeywordBhikkhuBodhi{「陶瓷碎片會留下」}(potsherds would be left),並解說「kapalla」通常指「壺」或「鉢」,但經文使用複數語態(kapallāni),則指破碎的「陶瓷碎片」。按:《顯揚真義》以「連同在甕邊緣有些不被結縛的陶瓷碎片」(saha mukhavaṭṭiyā ekābaddhāni kumbhakapallāni)解說。
\stopitemgroup

\startitemgroup[noteitems]
\item\subnoteref{x273}\NoteKeywordAgamaHead{「其心驅馳(\ccchref{SA.284}{https://agama.buddhason.org/SA/dm.php?keyword=284})」},南傳作\NoteKeywordNikaya{「有識的下生(識的下生存在)」}(viññāṇassa avakkanti hoti),菩提比丘長老英譯為\NoteKeywordBhikkhuBodhi{「有識的下降」}(there is a descent of consciousness)。
\stopitemgroup

\startitemgroup[noteitems]
\item\subnoteref{x274}\NoteKeywordAgamaHead{「極甚深、明亦甚深(\ccchref{MA.97}{https://agama.buddhason.org/MA/dm.php?keyword=97});甚深難解(\ccchref{DA.13}{https://agama.buddhason.org/DA/dm.php?keyword=13})」},南傳作\NoteKeywordNikaya{「甚深與顯現甚深」}(gambhīro…gambhīrāvabhāso),菩提比丘長老英譯為\NoteKeywordBhikkhuBodhi{「是深遠的與在意含上深遠的」}(is deep and deep in implications, SN),Maurice Walshe先生英譯為「是深奧的且顯現深奧的」(is profound and appears profound, DN)。按:「顯現」(avabhāsa),另譯為「光;光明;光照」,所以「明亦甚深」中的「明」,應為此字的對譯,《顯揚真義》、《吉祥悅意》都以「被看見」(dissatīti)解說。
\stopitemgroup

\startitemgroup[noteitems]
\item\subnoteref{x275}\NoteKeywordAgamaHead{「至淺至淺(\ccchref{MA.97}{https://agama.buddhason.org/MA/dm.php?keyword=97});無甚深之義(\ccchref{AA.49.5}{https://agama.buddhason.org/AA/dm.php?keyword=49.5})」},南傳作\NoteKeywordNikaya{「明顯明顯的」}(uttānakuttānako,另譯為「淺的淺的;明瞭的明瞭的」),菩提比丘長老英譯為\NoteKeywordBhikkhuBodhi{「再明白不過了」}(as clear as clear can be, SN),Maurice Walshe先生英譯為「再明白不過了」(as clear as clear, DN)。
\stopitemgroup

\startitemgroup[noteitems]
\item\subnoteref{x276}\NoteKeywordAgamaHead{「異生、異滅(\ccchref{SA.289}{https://agama.buddhason.org/SA/dm.php?keyword=289})」},南傳作\NoteKeywordNikaya{「都生起一個,另一個被滅」}(aññadeva uppajjati aññaṃ nirujjhati,逐字譯為「異生異滅」),菩提比丘長老英譯為\NoteKeywordBhikkhuBodhi{「依一物發生,並且依另一物終止」}(arises as one thing and ceases as another),並解說譬喻的重點不在於像猴子般的動個不停,而是心總離不開所緣。
\stopitemgroup

\startitemgroup[noteitems]
\item\subnoteref{x277}\NoteKeywordAgamaHead{「異生,異滅(\ccchref{SA.290}{https://agama.buddhason.org/SA/dm.php?keyword=290})」}(依一個生起,依另一個被滅),參看\ccchref{SA.289}{https://agama.buddhason.org/SA/dm.php?keyword=289}/\suttaref{SN.12.61}。
\stopitemgroup

\startitemgroup[noteitems]
\item\subnoteref{x278}\NoteKeywordAgamaHead{「斷知(\ccchref{SA.373}{https://agama.buddhason.org/SA/dm.php?keyword=373})」},南傳作\NoteKeywordNikaya{「被遍知」}(pariññāte, pariññāto),菩提比丘長老英譯為\NoteKeywordBhikkhuBodhi{「被完全地理解」}(is fully understood)。按:《顯揚真義》說,有這三種被遍知的遍知(tīhi pariññāhi pariññāte):i.知的遍知(ñātapariññā):了知四大之食物衝擊舌頭生起以觸為第五法的(phassapañcamakā dhammā)與四無色蘊(受想思識),在那些法中區別自味(自己作用)特相(sarasalakkhaṇato),當遍求其緣時看到逆、順緣起(anulomapaṭilomaṃ paṭiccasamuppādaṃ)。ii.衡量的遍知(tīraṇapariññā):就那個有緣的(sappaccaya)名色登上(āropetvā)無常、苦、無我後因而徹底知道(sammasati)七隨看(+厭、離貪、滅、斷念),就這範圍貫通三相的思惟智(觸知智)。 iii.捨斷的遍知(pahānapariññā):以抽回欲與貪,以不還道(anāgāmimaggena)遍知(大意)。
\stopitemgroup

\startitemgroup[noteitems]
\item\subnoteref{x279}\NoteKeywordAgamaHead{「於五欲功德貪愛則斷(\ccchref{SA.373}{https://agama.buddhason.org/SA/dm.php?keyword=373})」},南傳作\NoteKeywordNikaya{「五種欲的貪被遍知」}(pañcakāmaguṇiko rāgo pariññāto hoti),菩提比丘長老英譯為\NoteKeywordBhikkhuBodhi{「對五束感官快樂的慾望被完全地理解」}(lust for the five cords of sensual pleasure is fully understood)。按:《顯揚真義》說,在這裡有三種遍知:i.單一遍知(ekapariññā):遍知在舌門上單一味的渴愛(ekarasataṇhaṃ),依此五種欲的貪被遍知。ii.一切遍知(sabbapariññā):以一口入鉢的施食就得到五種欲的貪,像這樣在這食物上以念與正知把握後,以無欲貪受用食物為一切遍知。iii.根源遍知(mūlapariññā):物質食物是五種欲的貪之根源。長老說,因為飽食而慾望興。
\stopitemgroup

\startitemgroup[noteitems]
\item\subnoteref{x280}\NoteKeywordAgamaHead{「三受則斷(\ccchref{SA.373}{https://agama.buddhason.org/SA/dm.php?keyword=373})」},南傳作\NoteKeywordNikaya{「三受被遍知」}(tisso vedanā pariññātā honti),菩提比丘長老英譯為\NoteKeywordBhikkhuBodhi{「三種感受被完全地理解」}(the three kinds of feeling are fully understood)。按:緣觸受,受的根源是觸故說。
\stopitemgroup

\startitemgroup[noteitems]
\item\subnoteref{x281}參看\ccchref{SA.285}{https://agama.buddhason.org/SA/dm.php?keyword=285}「若死;若遷」,\suttaref{SN.12.4}「死亡;死沒」比對。
\stopitemgroup

\startitemgroup[noteitems]
\item\subnoteref{x282}\NoteKeywordAgamaHead{「齊識而還,不能過彼(SA)」},南傳作\NoteKeywordNikaya{「這個識從名色回轉,不更進一步走」}(paccudāvattati kho idaṃ viññāṇaṃ nāmarūpamhā na paraṃ gacchati),菩提比丘長老英譯為\NoteKeywordBhikkhuBodhi{「這識轉回,它不超越名與色」}(This consciousness turns back; it does not go further than name-and-form)。按:《顯揚真義》說,這個轉回的識為結生識,也是毘婆舍那識(Paṭisandhiviññāṇampi vipassanāviññāṇampi),結生識從其緣(paccayato)返回,毘婆舍那識從其所緣(ārammaṇato)返回,兩者都不征服名色,不超越名色。《瑜伽師地論》說:「識於現法中用名色為緣故,名色復於後法中用識為緣故。所以者何?以於母腹中有相續時說互為緣故,由識為緣;於母腹中諸精血色名所攝受,和合共成羯羅藍性,即此名色為緣,復令彼識於此得住。」《初期大乘佛教起源與開展》從「認識」與「胎生」的角度來解說亦同。又,長老說,可能菩薩一直在尋求奧義書模式的獨存識之真我(self),但卻發現識一直依存於名色,而知道那樣的尋求是徒勞的,因而顯示連感覺為真我之最細微的識都是條件所成的,因而標記為無常、苦、無我。
\stopitemgroup

\startitemgroup[noteitems]
\item\subnoteref{x283}\NoteKeywordAgamaHead{「就這個範圍」(Ettāvatā),菩提比丘長老英譯為「到這個範圍」(to this extent)。按:《顯揚真義》解說,識應為名色的緣;名色應為識的緣,兩者在互相為緣下(aññamaññapaccayesu),只這樣能被生或再生。此與\ccchref{SA.288}{https://agama.buddhason.org/SA/dm.php?keyword=288}所說的「然彼名色緣識生……然彼識緣名色生」}的意義相同。又說,未能見無明、行不能成為佛,不是嗎?真實!不能,但經由這有、取、渴愛而它們被看見(iminā pana te bhavaupādānataṇhāvasena diṭṭhāva),此為緣起的十支說,《大毘婆沙論》有相似的內容:「問:菩薩何故逆觀緣起唯至於識,心便轉還,為智力窮?為爾焰盡?……答,應作是說:非智力窮,非爾焰盡,但由菩薩於行、無明先已觀故,謂:先觀有即已觀行;先觀愛取已,觀無明。」而經文後面南傳出現「行」,北傳出現「行」與「無明」,顯然與這裡所說難相應,或為後來經文的增修。
\stopitemgroup

\startitemgroup[noteitems]
\item\subnoteref{x284}\NoteKeywordAgamaHead{「取(\ccchref{SA.291}{https://agama.buddhason.org/SA/dm.php?keyword=291})」},南傳作\NoteKeywordNikaya{「你們觸知」}(sammasatha, 動詞),菩提比丘長老英譯為\NoteKeywordBhikkhuBodhi{「從事於」}(engage in)。
\stopitemgroup

\startitemgroup[noteitems]
\item\subnoteref{x285}\NoteKeywordAgamaHead{「內觸法(\ccchref{SA.291}{https://agama.buddhason.org/SA/dm.php?keyword=291})」},南傳作\NoteKeywordNikaya{「內在的觸知」}(antaraṃ sammasanti),菩提比丘長老英譯為\NoteKeywordBhikkhuBodhi{「內在的探查」}(inward exploration)。按:「觸知」(sammasa,名詞),另譯為「把握;探查;徹底知道;思惟」,《清淨道論》中俗稱十六觀智的「思惟智」(sammasanañāṇa)就用此字(觸知智),《顯揚真義》以「內部緣的觸知」(abbhantaraṃ paccayasammasanaṃ)解說。
\stopitemgroup

\startitemgroup[noteitems]
\item\subnoteref{x286}\NoteKeywordNikayaHead{「鹹麵漿」}(bhaṭṭhaloṇikāya,逐字譯為「下+鹽的」),菩提比丘長老英譯為\NoteKeywordBhikkhuBodhi{「粥」}(porridge)。按:《顯揚真義》以「加鹽、麵粉、水」(saloṇena sattupānīyena)解說,今準此譯。
\stopitemgroup

\startitemgroup[noteitems]
\item\subnoteref{x287}\NoteKeywordNikayaHead{「鹹酸粥」}(loṇasovīraka),菩提比丘長老英譯為\NoteKeywordBhikkhuBodhi{「湯」}(soup)。按:《顯揚真義》以「放入一切穀物果實、頂芽等」(sabbadhaññaphalakaḷīrādīni pakkhipitvā )解說。
\stopitemgroup

\startitemgroup[noteitems]
\item\subnoteref{x288}\NoteKeywordAgamaHead{「有滅,寂滅涅槃(\ccchref{SA.351}{https://agama.buddhason.org/SA/dm.php?keyword=351})」},南傳作\NoteKeywordNikaya{「有之滅為涅槃」}(bhavanirodho nibbānanti),菩提比丘長老英譯為\NoteKeywordBhikkhuBodhi{「涅槃是存在的停止」}(Nibbāna is the cessation of existence)。按:這裡的「有」(bhava),即「十二緣起支」的「有」。
\stopitemgroup

\startitemgroup[noteitems]
\item\subnoteref{x289}\NoteKeywordAgamaHead{「而不觸身(\ccchref{SA.351}{https://agama.buddhason.org/SA/dm.php?keyword=351})」},南傳作\NoteKeywordNikaya{「但非以身觸達後而住」}(na ca kāyena phusitvā vihareyya),菩提比丘長老英譯為\NoteKeywordBhikkhuBodhi{「但他不能夠與它作身體地接觸」}(but he would not be able to make bodily contact with it)。按:《顯揚真義》說,看見井水如不還者看見涅槃,口渴的男子如不還者(ghammābhitattapuriso viya anāgāmī),水桶如阿羅漢道。
\stopitemgroup

\startitemgroup[noteitems]
\item\subnoteref{x290}\NoteKeywordNikayaHead{「高漲的」}(upayanto),菩提比丘長老英譯為\NoteKeywordBhikkhuBodhi{「遽增」}(surging)。按:《顯揚真義》以「水增加時,去到上方」(udakavaḍḍhanasamaye upari gacchanto)解說。反義詞apayanto為「消退的」。
\stopitemgroup

\startitemgroup[noteitems]
\item\subnoteref{x291}\NoteKeywordAgamaHead{「起色(\ccchref{SA.347}{https://agama.buddhason.org/SA/dm.php?keyword=347});過色(摩訶僧祇律)」},南傳作\NoteKeywordNikaya{「超越色」}(atikkamma rūpe),菩提比丘長老英譯為\NoteKeywordBhikkhuBodhi{「超越色」}(transcend form)。「正聞本」註解說「起色」的『「起」,疑「超」』是很合理的。
\stopitemgroup

\startitemgroup[noteitems]
\item\subnoteref{x292}\NoteKeywordAgamaHead{「後知涅槃(\ccchref{SA.347}{https://agama.buddhason.org/SA/dm.php?keyword=347});後比智(摩訶僧祇律)」},南傳作\NoteKeywordNikaya{「涅槃智在後」}(pacchā nibbāne ñāṇan”ti),菩提比丘長老英譯為\NoteKeywordBhikkhuBodhi{「之後涅槃智」}(afterwards knowledge of Nibbāna)。按:「知涅槃」顯為「涅槃智」的另譯,《顯揚真義》以「在毘婆舍那已實行後,道智被轉起,它在後生起(vipassanāya ciṇṇante pavattamaggañāṇaṃ, taṃ pacchā uppajjati),因此世尊說這個」解說,亦即以「道智」解說「涅槃智」。又說,即便沒有定,這樣也有智的生起、道理的看見(Vināpi samādhiṃ evaṃ ñāṇuppattidassanatthaṃ),[故]這被說:蘇尸摩!道或果不是定的等流(流出)、不是定的效益、不是定的成就(maggo vā phalaṃ vā na samādhinissando, na samādhiānisaṃso, na samādhissa nipphatti),又,這是毘婆舍那的等流、毘婆舍那的效益、毘婆舍那的成就(vipassanāya paneso nissando, vipassanāya ānisaṃso, vipassanāya nipphatti)。
\stopitemgroup

\startitemgroup[noteitems]
\item\subnoteref{x293}\NoteKeywordNikayaHead{「老師」}(satthā),菩提比丘長老英譯為\NoteKeywordBhikkhuBodhi{「老師」}(a teacher)。按:老師(satthar),一般譯作「大師」,專指「世尊」,《顯揚真義》說,這裡,satthā應為佛陀或弟子,凡依止他而得到道智者(maggañāṇaṃ labhati)名為satthā,今準此翻譯。
\stopitemgroup

\startitemgroup[noteitems]
\item\subnoteref{x294}\NoteKeywordNikayaHead{「不如實知見生者」}等2-11經(第一個\suttaref{SN.12.83}-93),從本品的攝頌來看,應該編在\suttaref{SN.12.82}中,不宜給予83-93的經號。
\stopitemgroup

\startitemgroup[noteitems]
\item\subnoteref{x295}\NoteKeywordNikayaHead{「獲得」}(lābha),菩提比丘長老英譯為\NoteKeywordBhikkhuBodhi{「獲得」}(gains)。按:《顯揚真義》說,遍求後獲得與渴愛俱的所緣(saha taṇhāya ārammaṇaṃ)被稱為色獲得(rūpalābho)……(依此類推法獲得),長老說,這樣就有帶著獲得所緣的(意)觸,以及經驗所緣的受,觸與受這一對就是這樣出現。
\stopitemgroup

\startitemgroup[noteitems]
\item\subnoteref{x296}\NoteKeywordAgamaHead{「緣出要界(\ccchref{SA.458}{https://agama.buddhason.org/SA/dm.php?keyword=458})」},南傳作\NoteKeywordNikaya{「緣於離欲界」}(Nekkhammadhātuṃ……paṭicca),菩提比丘長老英譯為\NoteKeywordBhikkhuBodhi{「依於放棄元素」}(In dependence on the renunciation element)。按:「離欲」(nekkhamma),另譯為「出離」,《顯揚真義》說,這裡,離欲界為離欲尋及一切善法(ettha nekkhammavitakkopi nekkhammadhātu sabbepi kusalā dhammā)。
\stopitemgroup

\startitemgroup[noteitems]
\item\subnoteref{x297}\NoteKeywordAgamaHead{「染著(\ccchref{SA.13}{https://agama.buddhason.org/SA/dm.php?keyword=13})」},南傳作\NoteKeywordNikaya{「貪著」}(sārajjeyyuṃ),菩提比丘長老英譯為\NoteKeywordBhikkhuBodhi{「成為迷戀(耽溺)」}(become enamoured)。
\stopitemgroup

\startitemgroup[noteitems]
\item\subnoteref{x298}\NoteKeywordNikayaHead{「生起、存續、生出、顯現、平息、滅、沒」},參看\suttaref{SN.22.30}。
\stopitemgroup

\startitemgroup[noteitems]
\item\subnoteref{x299}\NoteKeywordNikayaHead{「芥子」}(sāsapa,另譯為「罌粟的種子」),菩提比丘長老英譯為\NoteKeywordBhikkhuBodhi{「芥菜的種子」}(mustard seeds)。
\stopitemgroup

\startitemgroup[noteitems]
\item\subnoteref{x300}\NoteKeywordNikayaHead{「不安的」}(anassāsikā,另譯為「不安息的」),菩提比丘長老英譯為\NoteKeywordBhikkhuBodhi{「不可靠」}(unreliable)。
\stopitemgroup

\startitemgroup[noteitems]
\item\subnoteref{x301}\NoteKeywordAgamaHead{「月譬住(\ccchref{SA.1136}{https://agama.buddhason.org/SA/dm.php?keyword=1136});如月初生(GA)」},南傳作\NoteKeywordNikaya{「如月亮那樣」}(candūpama,另譯為「月亮譬喻」),菩提比丘長老英譯為\NoteKeywordBhikkhuBodhi{「像月亮」}(like the moon)。按:《顯揚真義》說,如月亮躍入空中不與任何人共作親密交往、親愛、愛著(阿賴耶),欲求、熱望、纏縛那樣,但非不為大眾的所愛的、合意的(na ca na hoti mahājanassa piyo manāpo)……又,如月亮碎破黑暗,遍佈光明那樣,這樣以碎破污穢的黑暗、遍佈智的光明成為月亮譬喻(evaṃ kilesandhakāravidhamanena ñāṇālokapharaṇena cāpi candūpamā hutvāti)。
\stopitemgroup

\startitemgroup[noteitems]
\item\subnoteref{x302}\NoteKeywordAgamaHead{「商那(GA.116)」},南傳作\NoteKeywordNikaya{「粗麻布」}(sāṇāni, sāṇa),菩提比丘長老英譯為\NoteKeywordBhikkhuBodhi{「破舊的大麻衣」}(worn-out hempen)。
\stopitemgroup

\startitemgroup[noteitems]
\item\subnoteref{x303}\NoteKeywordAgamaHead{「修行遠離(\ccchref{SA.1140}{https://agama.buddhason.org/SA/dm.php?keyword=1140});常樂空閑寂靜之處(GA)」},南傳作\NoteKeywordNikaya{「獨居者」}(pavivittā,另譯為「已遠離的;閑居的;獨居的;獨住的」),菩提比丘長老英譯為\NoteKeywordBhikkhuBodhi{「隔離,孤獨」}(secluded, solitude)。
\stopitemgroup

\startitemgroup[noteitems]
\item\subnoteref{x304}\NoteKeywordAgamaHead{「智慧薄少(\ccchref{SA.1143}{https://agama.buddhason.org/SA/dm.php?keyword=1143});嬰愚少智(GA)」},南傳作\NoteKeywordNikaya{「女人是愚癡的」}(bālo mātugāmo),菩提比丘長老英譯為\NoteKeywordBhikkhuBodhi{「女人們是愚蠢的」}(women are foolish),並解說這樣譯完全忠於巴利語原經文的語法,且看不出能以其他方式譯此句的可能。
\stopitemgroup

\startitemgroup[noteitems]
\item\subnoteref{x305}\NoteKeywordNikayaHead{「悅意俱行的身至念」}(sātasahagatā ca … kāyagatāsati),菩提比丘長老英譯為\NoteKeywordBhikkhuBodhi{「指向身體的深切注意與喜悅結合」}(mindfulness directed to the body associated with joy)。按:《顯揚真義》說,「悅意俱行的身至念」指在不淨觀與入出息念中因初禪而與樂相應的身至念(\suttaref{SN.16.11}),《吉祥悅意》說,「悅意俱行」指住立第四禪[外]的其它處有樂相應的喜悅俱行,「身至念」指安那般那、四舉止行為有念正知、三十二行相(身分)、四界分別、十不淨、棄屍處碎破作意、在頭髮等上之四色[界]禪(esādīsu cattāri rūpajjhānānīti),在這裡生起念(\ccchref{DN.34}{https://agama.buddhason.org/DN/dm.php?keyword=34})。
\stopitemgroup

\startitemgroup[noteitems]
\item\subnoteref{x306}\NoteKeywordAgamaHead{「相似像法(\ccchref{SA.906}{https://agama.buddhason.org/SA/dm.php?keyword=906});像法(GA)」},南傳作\NoteKeywordNikaya{「相似正法」}(saddhammappatirūpakaṃ),菩提比丘長老英譯為\NoteKeywordBhikkhuBodhi{「正法的假冒」}(a counterfeit of the true Dhamma)。
\stopitemgroup

\startitemgroup[noteitems]
\item\subnoteref{x307}\NoteKeywordNikayaHead{「以陷阱叉」}(papatāya,原意為「墮落處」),菩提比丘障老英譯為「以有線魚叉」(with a corded harpoon)。按:《顯揚真義》以長線綁著的鐵作的容器,有引入捕捉裝置,頂有箭、鐵刺,以速度落下容器中……」解說,今準此譯。
\stopitemgroup

\startitemgroup[noteitems]
\item\subnoteref{x308}\NoteKeywordNikayaHead{「以到達塗有毒的、塗上毒的箭」}(diddhagatena visallena sallena),菩提比丘障老英譯為「以塗了毒的標槍」(with a dart smeared in poison)。按: visallena應該是個錯字,光明寺譯本提出可能是visa-litta,《顯揚真義》以「以到達塗有毒的」(gatadiddhena)解說diddhagatena,以「以塗上毒藥的」(visamakkhitena)解說visallena,今準此譯。
\stopitemgroup

\startitemgroup[noteitems]
\item\subnoteref{x309}\NoteKeywordAgamaHead{「隨嵐風(\ccchref{AA.25.8}{https://agama.buddhason.org/AA/dm.php?keyword=25.8})」},南傳作\NoteKeywordNikaya{「迅猛風」}(verambhā…vātā),菩提比丘長老英譯為\NoteKeywordBhikkhuBodhi{「疾風;暴風」}(the gale winds)。「迅猛(風)」(verambhā),另譯為「毘嵐婆;季節風;颱風;旋風」,《顯揚真義》以「大風的情況」(mahāvātā)解說。
\stopitemgroup

\startitemgroup[noteitems]
\item\subnoteref{x310}\NoteKeywordAgamaHead{「鍼筩(\ccchref{AA.25.8}{https://agama.buddhason.org/AA/dm.php?keyword=25.8})」},南傳作\NoteKeywordNikaya{「針盒」}(sūcigharaṃ,另譯為「針筒」),菩提比丘長老英譯為\NoteKeywordBhikkhuBodhi{「針盒」}(needle case)。按:「鍼筩」即「針筒」,《一切經音義》:鍼筩(上,執林反。廣雅云:鍼,㓨也。顧野王云:綴衣鐵也。說文:鍼,所以縫衣也,從金,咸聲。或作針,俗字也。下,徒紅反,考聲云:竹筩也。說文云:斷竹也,從竹,甬聲,甬音。勇經作筒,亦通)。
\stopitemgroup

\startitemgroup[noteitems]
\item\subnoteref{x311}\NoteKeywordAgamaHead{「此是其限,此是其量(\ccchref{AA.9.1}{https://agama.buddhason.org/AA/dm.php?keyword=9.1})」},南傳作\NoteKeywordNikaya{「這是秤;這是衡量」}(Esā…tulā etaṃ pamāṇaṃ),菩提比丘長老英譯為\NoteKeywordBhikkhuBodhi{「這是標準與軌範」}(this is the standard and criterion)。
\stopitemgroup

\startitemgroup[noteitems]
\item\subnoteref{x312}\NoteKeywordNikayaHead{「賢女」}(ayye,原意為「高貴的;聖的,聖尼;大姊;貴婦人」),菩提比丘長老英譯為\NoteKeywordBhikkhuBodhi{「親愛的」}(Dear)。
\stopitemgroup

\startitemgroup[noteitems]
\item\subnoteref{x313}\NoteKeywordAgamaHead{「此是其限,此是其量(\ccchref{AA.9.2}{https://agama.buddhason.org/AA/dm.php?keyword=9.2})」},南傳作\NoteKeywordNikaya{「這是秤;這是衡量」}(Esā…tulā etaṃ pamāṇaṃ),菩提比丘長老英譯為\NoteKeywordBhikkhuBodhi{「這是標準與軌範」}(this is the standard and criterion)。
\stopitemgroup

\startitemgroup[noteitems]
\item\subnoteref{x314}\NoteKeywordNikayaHead{「在當生中的諸樂住處」}(diṭṭhadhammasukhavihārā),菩提比丘長老英譯為\NoteKeywordBhikkhuBodhi{「快樂住處」}(pleasant dwellings)。按:《顯揚真義》說,這是住於果等至(達到果位)之樂(phalasamāpattisukhavihārā),長老說,這通常指禪定(jhānas)。
\stopitemgroup

\startitemgroup[noteitems]
\item\subnoteref{x315}\NoteKeywordNikayaHead{「白法」}(sukko dhammo),菩提比丘長老英譯為\NoteKeywordBhikkhuBodhi{「光明的性質」}(bright nature)。
\stopitemgroup

\startitemgroup[noteitems]
\item\subnoteref{x316}\NoteKeywordAgamaHead{「……駏驉騾懷妊等亦復如是(\ccchref{SA.1064}{https://agama.buddhason.org/SA/dm.php?keyword=1064})」},南傳作\NoteKeywordNikaya{「……如胎對騾」}(…gabbho assatariṃ yathāti, \suttaref{SN.17.35}/\suttaref{SN.6.12}),\suttaref{SN.3.2}/\ccchref{It.50}{https://agama.buddhason.org/It/dm.php?keyword=50}的「從自己生起的它們殺害,如有自己果實的竹。」意趣亦同。又,\ccchref{AN.4.68}{https://agama.buddhason.org/AN/an.php?keyword=4.68}-提婆達多經與本經文全相同。
\stopitemgroup

\startitemgroup[noteitems]
\item\subnoteref{x317}\NoteKeywordNikayaHead{「灑」}(bhindeyyuṃ),《顯揚真義》以pakkhipeyyaṃ解說,註疏以osiñceyyuṃ解說,今依此譯。
\stopitemgroup

\startitemgroup[noteitems]
\item\subnoteref{x318}\NoteKeywordAgamaHead{「屠牛餘罪(\ccchref{SA.508}{https://agama.buddhason.org/SA/dm.php?keyword=508})」},南傳作\NoteKeywordNikaya{「就以該業的殘餘果報」}(tasseva kammassa vipākāvasesena),菩提比丘長老英譯為\NoteKeywordBhikkhuBodhi{「為那相同的業之剩餘結果」}(as a residual result of that same kamma)。按:《顯揚真義》說:當在地獄死時,有(出現)無肉的牛骨聚相,那被覆藏(paṭicchannampi)的業如對有智者顯現那樣而生為骷髏餓鬼。
\stopitemgroup

\startitemgroup[noteitems]
\item\subnoteref{x319}\NoteKeywordNikayaHead{「劍毛」}(asilomaṃ),菩提比丘長老英譯為\NoteKeywordBhikkhuBodhi{「像劍般的身毛」}(body-hairs of swords)。
\stopitemgroup

\startitemgroup[noteitems]
\item\subnoteref{x320}\NoteKeywordNikayaHead{「矛毛」}(sattilomaṃ),菩提比丘長老英譯為\NoteKeywordBhikkhuBodhi{「像矛般的身毛」}(body-hairs of spears)。「矛」(satti),另譯作「槍」。
\stopitemgroup

\startitemgroup[noteitems]
\item\subnoteref{x321}\NoteKeywordNikayaHead{「箭毛」}(usulomaṃ),菩提比丘長老英譯為\NoteKeywordBhikkhuBodhi{「像矛般的頭髮」}(Spear Hairs)。
\stopitemgroup

\startitemgroup[noteitems]
\item\subnoteref{x322}\NoteKeywordNikayaHead{「駕御者」}(sūto),菩提比丘長老英譯為\NoteKeywordBhikkhuBodhi{「馴馬師」}(a horse trainer)。按:《顯揚真義》說,駕御者為訓馬師」(sūtoti assadamako),他們也說就是「訓牛師」(Godamakotipi vadantiyeva),曾有他的帶針刺棒刺穿情況的相(Tassa patodasūciyā vijjhanabhāvoyeva nimittaṃ ahosi)。
\stopitemgroup

\startitemgroup[noteitems]
\item\subnoteref{x323}\NoteKeywordNikayaHead{「收賄裁判官」}(gāmakūṭako,原意為「村裡的騙子」),菩提比丘長老英譯為\NoteKeywordBhikkhuBodhi{「腐敗的法官」}(a corrupt magistrate)。按:《顯揚真義》以收賄(lañjaṃ gahetvā )的裁判官(vinicchayāmacco)解說,今準此譯。
\stopitemgroup

\startitemgroup[noteitems]
\item\subnoteref{x324}\NoteKeywordNikayaHead{「炎熱的、被燻黑的」}(okiliniṃ okiriniṃ),菩提比丘長老英譯為\NoteKeywordBhikkhuBodhi{「炎熱的、煤煙燻黑的」}(sweltering, sooty)。按:「okiliniṃ okiriniṃ」一般作「捨棄、驅逐」,《顯揚真義》說,她躺在炭火柴堆上被翻滾回轉地燒,因此被燒焦,以熱火烘身體,炎熱而身體劇痛(okilinī ca kilinnasarīrā),汗從身體滴下,被煙與炭火包圍(okirinī ca aṅgārasamparikiṇṇā),今準此譯。
\stopitemgroup

\startitemgroup[noteitems]
\item\subnoteref{x325}\NoteKeywordNikayaHead{「(被)小偷」}(kumbhatthenakehi,直譯為「(被)甕賊」),菩提比丘長老英譯沒譯。按:《顯揚真義》說,進入他人家後,以燈光檢視,然後想搬他人物品,他們放燈火於甕中進入,他們以那些甕的意義(tehi kumbhatthenakehi)名為小偷(甕賊)。
\stopitemgroup

\startitemgroup[noteitems]
\item\subnoteref{x326}\NoteKeywordNikayaHead{「百大口鍋[食物]」}(okkhāsataṃ),菩提比丘長老英譯為\NoteKeywordBhikkhuBodhi{「一百鍋煮好的食物」}(a hundred pots of food)。按:《顯揚真義》以「一百大口鍋的[食物]」(okkhāsatanti mahāmukhaukkhalīnaṃ sataṃ)解說,今依此譯。
\stopitemgroup

\startitemgroup[noteitems]
\item\subnoteref{x327}\NoteKeywordAgamaHead{「阿能訶(\ccchref{SA.1258}{https://agama.buddhason.org/SA/dm.php?keyword=1258})」},南傳作\NoteKeywordNikaya{「阿那葛」}(ānako,原意為「銅鼓;定音鼓;半球形銅鼓」),菩提比丘長老英譯為\NoteKeywordBhikkhuBodhi{「阿那葛」}(Summoner)。按:《顯揚真義》說,這是鼓(mudiṅgo)的名字,喜馬拉雅山大螃蟹(Himavante kira mahākuḷīradaho)的爪做的,聲音覆蓋12由旬城市。
\stopitemgroup

\startitemgroup[noteitems]
\item\subnoteref{x328}\NoteKeywordAgamaHead{「修多羅(\ccchref{SA.1258}{https://agama.buddhason.org/SA/dm.php?keyword=1258})」},南傳作\NoteKeywordNikaya{「經典」}(suttantā,另為「素呾纜(音譯);契經」),菩提比丘長老英譯為\NoteKeywordBhikkhuBodhi{「說教」}(discourses)。
\stopitemgroup

\startitemgroup[noteitems]
\item\subnoteref{x329}\NoteKeywordNikayaHead{「已滅沒了」}(antarahito),菩提比丘長老英譯為\NoteKeywordBhikkhuBodhi{「已過世」}(has passed away)。按:原文使用過去分詞,非假設語態,表示世尊入滅後,尊者舍利弗還在,這與尊者舍利弗入滅於世尊入滅之前的主流說法(如\suttaref{SN.47.13}/\ccchref{SA.638}{https://agama.buddhason.org/SA/dm.php?keyword=638}、\suttaref{SN.47.14}/\ccchref{SA.639}{https://agama.buddhason.org/SA/dm.php?keyword=639})相異。
\stopitemgroup

\startitemgroup[noteitems]
\item\subnoteref{x330}\NoteKeywordNikayaHead{「大龍」}(mahānāgā),菩提比丘長老英譯為\NoteKeywordBhikkhuBodhi{「大龍」}(great nāgas),並解說「龍」通常指\twnr{阿羅漢}{5.0}。「龍」(nāga),另譯為「象;龍象」。
\stopitemgroup

\startitemgroup[noteitems]
\item\subnoteref{x331}\NoteKeywordAgamaHead{「非下劣方便(\ccchref{SA.1070}{https://agama.buddhason.org/SA/dm.php?keyword=1070});懈怠(GA)」},南傳作\NoteKeywordNikaya{「這非鬆弛地精勤後」}(Nayidaṃ sithilamārabbha),菩提比丘長老英譯為\NoteKeywordBhikkhuBodhi{「非經由鬆弛的努力」}(Not by means of slack endeavour)。按:《顯揚真義》以「鬆弛的活力轉起後」(sithilavīriyaṃ pavattetvā)解說「鬆弛地精勤後」。
\stopitemgroup

\startitemgroup[noteitems]
\item\subnoteref{x332}\NoteKeywordAgamaHead{「侏儒拔提亞」(lakuṇḍakabhaddiyo, lakuṇṭakabhaddiyo,另譯為「侏儒婆提;羅婆那婆提」),\ccchref{AA.4.7}{https://agama.buddhason.org/AA/dm.php?keyword=4.7}說:「音響清徹,聲至梵天,所謂羅婆那婆提比丘是。」}\ccchref{AN.1.194}{https://agama.buddhason.org/AN/an.php?keyword=1.194}說:「妙音者,即:侏儒婆提」。
\stopitemgroup

\startitemgroup[noteitems]
\item\subnoteref{x333}\NoteKeywordAgamaHead{「難可觀視(\ccchref{SA.1063}{https://agama.buddhason.org/SA/dm.php?keyword=1063})」},南傳作\NoteKeywordNikaya{「難看(的)」}(duddasikaṃ),菩提比丘長老英譯為\NoteKeywordBhikkhuBodhi{「難看的;不體面的」}(unsightly)。
\stopitemgroup

\startitemgroup[noteitems]
\item\subnoteref{x334}\NoteKeywordAgamaHead{「無所依說(\ccchref{SA.1069}{https://agama.buddhason.org/SA/dm.php?keyword=1069})」},南傳作\NoteKeywordNikaya{「不依止的」}(anissitāya),菩提比丘長老英譯為\NoteKeywordBhikkhuBodhi{「離著的」}(with unattached, \suttaref{SN.21.7}),或「無阻礙的」([with] unhindered, \ccchref{AN.4.48}{https://agama.buddhason.org/AN/an.php?keyword=4.48})。按:《顯揚真義》、《滿足希求》同以「不依止輪轉」(vaṭṭaṃ anissitāya.)解說,即不重複繞說的意思。
\stopitemgroup

\startitemgroup[noteitems]
\item\subnoteref{x335}\NoteKeywordAgamaHead{「姨母子(\ccchref{SA.1067}{https://agama.buddhason.org/SA/dm.php?keyword=1067});姨母之子(GA)」},南傳作\NoteKeywordNikaya{「姨母之子」}(mātucchāputto),菩提比丘長老英譯為\NoteKeywordBhikkhuBodhi{「母方的表兄弟;姨表」}(maternal cousin)。因為尊者難陀是佛陀的父親與佛陀姨母(也是繼母;養母)的兒子之故。
\stopitemgroup

\startitemgroup[noteitems]
\item\subnoteref{x336}\NoteKeywordAgamaHead{「{抆}[文]飾兩目(\ccchref{AA.18.6}{https://agama.buddhason.org/AA/dm.php?keyword=18.6})」},南傳作\NoteKeywordNikaya{「畫眼妝」}(akkhīni añjetvā,直譯為「塗眼睛」),菩提比丘長老英譯為\NoteKeywordBhikkhuBodhi{「塗他的眼睛」}(painted his eyes)。按:《顯揚真義》以「以眼膏使之充滿」(añjanena pūretvā)解說。「眼膏」(añjana),另譯為「漆黑的;青黑色」,眼膏的顏色(añjanavaṇṇa)即是黑色。
\stopitemgroup

\startitemgroup[noteitems]
\item\subnoteref{x337}\NoteKeywordAgamaHead{「家家行乞食(\ccchref{SA.1067}{https://agama.buddhason.org/SA/dm.php?keyword=1067}),乞食(GA)」},南傳作\NoteKeywordNikaya{「以非有名者的殘食」}(Aññātuñchena),菩提比丘長老英譯為\NoteKeywordBhikkhuBodhi{「在陌生人的殘餘食物上」}(on the scraps of strangers)。按:《顯揚真義》說,在被標記為富人家有調味的資糧、極香的食物遍求的殘食名為有名者的殘食(ñātuñcho),但以站在順序家門口得到的混合食物名為非有名者的殘食(aññātuñcho, na ñāta uñcha)。
\stopitemgroup

\startitemgroup[noteitems]
\item\subnoteref{x338}\NoteKeywordNikayaHead{「一切的征服者」}(Sabbābhibhuṃ),菩提比丘長老英譯為\NoteKeywordBhikkhuBodhi{「全征服者」}(all-conqueror)。按:《顯揚真義》以「蘊處界與三有」解說「一切」。
\stopitemgroup

\startitemgroup[noteitems]
\item\subnoteref{x339}\NoteKeywordNikayaHead{「以威光」}(tejasāti),菩提比丘長老英譯為\NoteKeywordBhikkhuBodhi{「以容光;以光輪」}(with glory)。
\stopitemgroup

\startitemgroup[noteitems]
\item\subnoteref{x340}\NoteKeywordNikayaHead{「結交者」}(sametikā),菩提比丘長老英譯為\NoteKeywordBhikkhuBodhi{「已被結合」}(Have been united)。按:《顯揚真義》以「會合後、結交後意見住立者,傳說他們都曾一起走動五百生」(saṃsanditvā sametvā ṭhitaladdhino. Te kira pañcajātisatāni ekatova vicariṃsu)解說,今準此譯。
\stopitemgroup

\startitemgroup[noteitems]
\item\subnoteref{x341}\NoteKeywordAgamaHead{「取攝受(\ccchref{SA.107}{https://agama.buddhason.org/SA/dm.php?keyword=107})」},南傳作\NoteKeywordNikaya{「他是處在……的纏縛者」}(pariyuṭṭhaṭṭhāyī hoti),菩提比丘長老英譯為\NoteKeywordBhikkhuBodhi{「他活在被……纏住中」}(He lives obsessed by)。
\stopitemgroup

\startitemgroup[noteitems]
\item\subnoteref{x342}\NoteKeywordAgamaHead{「義品(\ccchref{SA.551}{https://agama.buddhason.org/SA/dm.php?keyword=551})」},南傳作\NoteKeywordNikaya{「八群[經]」}(aṭṭhakavaggiye),「八群[經]」應為「義品」之原始名稱,參看《原始佛教聖典之集成》p.796。
\stopitemgroup

\startitemgroup[noteitems]
\item\subnoteref{x343}\NoteKeywordNikayaHead{「無住處的行者」}(aniketasārī),菩提比丘長老英譯為\NoteKeywordBhikkhuBodhi{「無住所之漫遊者」}(one who roams about without abode)。按:「行者」(sārī),另譯為「雲遊者」,《顯揚真義》說,因所緣作的事而(ārammaṇakaraṇavasena)被稱為「有住處的行者」……為何五蘊被稱為「家」,六所緣被稱為「住處」,欲貪(Chandarāgassa)強或弱的狀態[故]……以常住被稱為家(niccanivāsanaṭṭhānagehameva vuccati),庭園等作指定理由(katasaṅketaṭṭhānaṃ)的住處為「住處」。
\stopitemgroup

\startitemgroup[noteitems]
\item\subnoteref{x344}\NoteKeywordNikayaHead{「不期盼者」}(apurakkharāno,原意為「不放置在前者;不重視者」),菩提比丘長老英譯為\NoteKeywordBhikkhuBodhi{「無期待」}(without expectations)。按:《顯揚真義》以「在已變成前不作者」(Apurakkharānoti vaṭṭaṃ purato akurumāno)解說,依此轉譯為「不期盼者」。
\stopitemgroup

\startitemgroup[noteitems]
\item\subnoteref{x345}\NoteKeywordNikayaHead{「從色相住處的擴散與繫縛」}(Rūpanimittaniketavisāravinibandhā),菩提比丘長老英譯為\NoteKeywordBhikkhuBodhi{「被擴散與限制在色徵候[組成]之住所」}(By diffusion and confinement in the abode [consisting in] the sign of forms)。按:《顯揚真義》說,以污染緣義,色是相(rūpameva kilesānaṃ paccayaṭṭhena nimittaṃ)。污染的擴展性與繫縛性(kilesānaṃ patthaṭabhāvo ca vinibandhanabhāvo)被稱為散逸與繫縛。
\stopitemgroup

\startitemgroup[noteitems]
\item\subnoteref{x346}\NoteKeywordNikayaHead{「來到結合」}(yogaṃ āpajjati),菩提比丘長老英譯為\NoteKeywordBhikkhuBodhi{「捲入」}(involves)。按:《顯揚真義》以「來到對他們應該做的義務與自己的結合」(upayogaṃ sayaṃ tesaṃ kiccānaṃ kattabbataṃ āpajjati)解說。
\stopitemgroup

\startitemgroup[noteitems]
\item\subnoteref{x347}\NoteKeywordNikayaHead{「得定的」}(samāhita,另音譯為「三摩呬多」,義譯為「等引」,另譯為「已定置;入定;入定者;等持者;得定者」),菩提比丘長老英譯為\NoteKeywordBhikkhuBodhi{「集中者」}(who is concentrated)。按:《顯揚真義》以「近行、安止」(upacārappanāhi)解說。
\stopitemgroup

\startitemgroup[noteitems]
\item\subnoteref{x348}\NoteKeywordAgamaHead{「取緣有(\ccchref{SA.67}{https://agama.buddhason.org/SA/dm.php?keyword=67})」},南傳作\NoteKeywordNikaya{「以那個的取為緣有有」}(tassupādānapaccayā bhavo),菩提比丘長老英譯為\NoteKeywordBhikkhuBodhi{「以取為條件,[結果是]存在」}(with one’s clinging as conditions, existence [comes to be])。按:「愛緣取,取緣有,有緣生,生緣老死、憂、悲、惱、苦」即「緣起五支說」,參看\ccchref{SA.283}{https://agama.buddhason.org/SA/dm.php?keyword=283}。
\stopitemgroup

\startitemgroup[noteitems]
\item\subnoteref{x349}\NoteKeywordAgamaHead{「心亦隨轉(\ccchref{SA.43}{https://agama.buddhason.org/SA/dm.php?keyword=43});識捫摸色(\ccchref{MA.164}{https://agama.buddhason.org/MA/dm.php?keyword=164})」},南傳作\NoteKeywordNikaya{「他的識成為隨色變易轉」}(Tassa…rūpavipariṇāmānuparivatti viññāṇaṃ hoti),菩提比丘長老英譯為\NoteKeywordBhikkhuBodhi{「他的識被改變的色所迷住」}(his consciousness becomes preoccupied with the change of form)。
\stopitemgroup

\startitemgroup[noteitems]
\item\subnoteref{x350}\NoteKeywordAgamaHead{「不顧(\ccchref{SA.8}{https://agama.buddhason.org/SA/dm.php?keyword=8}/\ccchref{SA.79}{https://agama.buddhason.org/SA/dm.php?keyword=79})」},南傳作\NoteKeywordNikaya{「無期待者」}(anapekkho,另譯為「不希望的」),菩提比丘長老英譯為\NoteKeywordBhikkhuBodhi{「不關心;冷淡」}(indifferent)。
\stopitemgroup

\startitemgroup[noteitems]
\item\subnoteref{x351}\NoteKeywordAgamaHead{「不欲(\ccchref{SA.8}{https://agama.buddhason.org/SA/dm.php?keyword=8});不欣(\ccchref{SA.79}{https://agama.buddhason.org/SA/dm.php?keyword=79})」},南傳作\NoteKeywordNikaya{「他不歡喜」}(nābhinandati,動詞),菩提比丘長老英譯為\NoteKeywordBhikkhuBodhi{「不求歡樂」}(not seek delight)。
\stopitemgroup

\startitemgroup[noteitems]
\item\subnoteref{x352}\NoteKeywordAgamaHead{「取擔(\ccchref{SA.73}{https://agama.buddhason.org/SA/dm.php?keyword=73});擔因緣(AA)」},南傳作\NoteKeywordNikaya{「負擔的拿起」}(bhārādāna),菩提比丘長老英譯為\NoteKeywordBhikkhuBodhi{「拿起負擔」}(the taking up of burden)。「取擔」就是「四聖諦」中的「集諦」,捨擔就是「四聖諦」中的「滅諦」。
\stopitemgroup

\startitemgroup[noteitems]
\item\subnoteref{x353}\NoteKeywordAgamaHead{「於色愛喜者,則於苦愛喜(\ccchref{SA.5}{https://agama.buddhason.org/SA/dm.php?keyword=5})」},南傳作\NoteKeywordNikaya{「歡喜色者,他歡喜苦」}(rūpaṃ abhinandati, dukkhaṃ so abhinandati),菩提比丘長老英譯為\NoteKeywordBhikkhuBodhi{「在色中求歡樂者在苦中求歡樂」}(one who seeks delight in form seeks delight in suffering)。按:此處的「歡喜」(abhinandati)為動詞。
\stopitemgroup

\startitemgroup[noteitems]
\item\subnoteref{x354}\NoteKeywordAgamaHead{「起(\ccchref{SA.78}{https://agama.buddhason.org/SA/dm.php?keyword=78})」},南傳作\NoteKeywordNikaya{「生起」}(uppāda),菩提比丘長老英譯為\NoteKeywordBhikkhuBodhi{「生起」}(the arising)。
\stopitemgroup

\startitemgroup[noteitems]
\item\subnoteref{x355}\NoteKeywordAgamaHead{「住(\ccchref{SA.78}{https://agama.buddhason.org/SA/dm.php?keyword=78})」},南傳作\NoteKeywordNikaya{「存續」}(ṭhiti,另譯作「住;住止」),菩提比丘長老英譯為\NoteKeywordBhikkhuBodhi{「持續;繼續」}(continuation)。
\stopitemgroup

\startitemgroup[noteitems]
\item\subnoteref{x356}\NoteKeywordAgamaHead{「出(\ccchref{SA.78}{https://agama.buddhason.org/SA/dm.php?keyword=78})」},南傳作\NoteKeywordNikaya{「生出、顯現」}(abhinibbatti pātubhāvo),菩提比丘長老英譯為\NoteKeywordBhikkhuBodhi{「生產與顯現」}(production and manifestation)。
\stopitemgroup

\startitemgroup[noteitems]
\item\subnoteref{x357}\NoteKeywordAgamaHead{「息(\ccchref{SA.78}{https://agama.buddhason.org/SA/dm.php?keyword=78})」},南傳作\NoteKeywordNikaya{「平息」}(vūpasama,另譯作「寂靜;寂滅;寂止;靜止」),菩提比丘長老英譯為\NoteKeywordBhikkhuBodhi{「沈澱;消去;平息」}(subsiding)。
\stopitemgroup

\startitemgroup[noteitems]
\item\subnoteref{x358}\NoteKeywordAgamaHead{「沒(\ccchref{SA.78}{https://agama.buddhason.org/SA/dm.php?keyword=78})」},南傳作\NoteKeywordNikaya{「滅沒」}(atthaṅgama),菩提比丘長老英譯為\NoteKeywordBhikkhuBodhi{「消失」}(passing away)。
\stopitemgroup

\startitemgroup[noteitems]
\item\subnoteref{x359}\NoteKeywordAgamaHead{「壞法(\ccchref{SA.51}{https://agama.buddhason.org/SA/dm.php?keyword=51})」},南傳作\NoteKeywordNikaya{「易壞的」}(pabhaṅgu,另譯為「壞法」),菩提比丘長老英譯為\NoteKeywordBhikkhuBodhi{「易碎的」}(the fragile)。按:《顯揚真義》以「被破壞自性」(pabhijjanasabhāvaṃ)解說。
\stopitemgroup

\startitemgroup[noteitems]
\item\subnoteref{x360}\NoteKeywordAgamaHead{「隨使死(\ccchref{SA.15}{https://agama.buddhason.org/SA/dm.php?keyword=15}/\ccchref{SA.16}{https://agama.buddhason.org/SA/dm.php?keyword=16})」},南傳作\NoteKeywordNikaya{「他被推量」}(anumīyati),菩提比丘長老英譯為\NoteKeywordBhikkhuBodhi{「依之而被測量」}(is measured in accordance with it)。按:《顯揚真義》說,『被推量者』,煩惱潛在趨勢的色以已死、以煩惱潛在趨勢為『隨死』(anumīyatīti taṃ anusayitaṃ rūpaṃ marantena anusayena anumarati.),正好與北傳的「隨使死」相應,但菩提比丘長老認為這是「滑稽」(ludicrous)的解讀。
\stopitemgroup

\startitemgroup[noteitems]
\item\subnoteref{x361}\NoteKeywordAgamaHead{「增諸數(\ccchref{SA.16}{https://agama.buddhason.org/SA/dm.php?keyword=16})」},南傳作\NoteKeywordNikaya{「他以那個走到稱呼」}(tena saṅkhaṃ gacchatī,另譯為「來到……之數」),菩提比丘長老英譯為\NoteKeywordBhikkhuBodhi{「被以該詞認定;從該角度被認定」}(is reckoned in terms of it)。按:《顯揚真義》說,如對色以欲貪等有煩惱潛在趨勢者,就被以該煩惱潛在趨勢來到安立:「染心的、憤怒的、愚癡的」(kāmarāgādīsu yena anusayena taṃ rūpaṃ anuseti, teneva anusayena ‘‘ratto duṭṭho mūḷho’’ti paṇṇattiṃ gacchati.)。長老進一步舉例作:對五蘊之某蘊顯得比較突出的,也能依此而得名,如傾向「受」者,被稱為「享樂主義者」(hedonist),傾向「想」者,被稱為「唯美主義者」(aesthete),傾向「行」者,被稱為「行動派者」(man of action)等等。
\stopitemgroup

\startitemgroup[noteitems]
\item\subnoteref{x362}\NoteKeywordAgamaHead{「當正觀察(\ccchref{SA.36}{https://agama.buddhason.org/SA/dm.php?keyword=36})」},南傳作\NoteKeywordNikaya{「它應該被如理考察」}(yoni upaparikkhitabbā),菩提比丘長老英譯為\NoteKeywordBhikkhuBodhi{「其基礎應該如這樣被調查」}(the basis itself should be investigated thus)。按,「如理」(yoni),原意為「胎;子宮」,引申為「根源;起源;原因」。
\stopitemgroup

\startitemgroup[noteitems]
\item\subnoteref{x363}\NoteKeywordNikayaHead{「什麼發生的」}(kiṃpahotikā),菩提比丘長老英譯為\NoteKeywordBhikkhuBodhi{「他們如何被製造」}(How are they produced)。按:《顯揚真義》以「從哪裡生起的」(kiṃpabhutikā, kuto pabhavantīti attho?)解說。
\stopitemgroup

\startitemgroup[noteitems]
\item\subnoteref{x364}\NoteKeywordAgamaHead{「名為涅槃(\ccchref{SA.36}{https://agama.buddhason.org/SA/dm.php?keyword=36})」},南傳作\NoteKeywordNikaya{「被稱為『那部分到達涅槃者』」}(‘tadaṅganibbuto’ti vuccati),菩提比丘長老英譯為\NoteKeywordBhikkhuBodhi{「被說成在那方面熄冷」}(is said to be quenched in that respect),並解說,nibbuto一般是用來形容解脫者的,但冠上tadaṅga表示尚未真正達涅槃,只是接近而已。水野弘元的《巴利語辭典》解說tadaṅga有兩個意思,一個是「確實」,另一個是「彼分(部分)」,前者就舉「tadaṅganibbuta」為例而解說為「確實寂止」,後者則舉「彼分空」(tadaṅgasuñña)為例。
\stopitemgroup

\startitemgroup[noteitems]
\item\subnoteref{x365}\NoteKeywordNikayaHead{「諸過去隨見」}(pubbantānudiṭṭhiyo),菩提比丘長老英譯為\NoteKeywordBhikkhuBodhi{「關於過去諸見解」}(views concerning the past)。按:《顯揚真義》以「過去隨行的18見」(pubbantaṃ anugatā aṭṭhārasa diṭṭhiyo)解說,長老說,這明確指梵網經(\ccchref{DN.1}{https://agama.buddhason.org/DN/dm.php?keyword=1})所說的62邪見中,關於過去的18種,而「諸未來隨見」則是其中關於未來的44種邪見。「隨見」(anudiṭṭhi),另譯為「邪見;見」。
\stopitemgroup

\startitemgroup[noteitems]
\item\subnoteref{x366}\NoteKeywordAgamaHead{「諸根增長(\ccchref{SA.45}{https://agama.buddhason.org/SA/dm.php?keyword=45});入於諸根(\ccchref{SA.63}{https://agama.buddhason.org/SA/dm.php?keyword=63})」},南傳作\NoteKeywordNikaya{「有五根的下生」}(pañcannaṃ indriyānaṃ avakkanti hoti),菩提比丘長老英譯為\NoteKeywordBhikkhuBodhi{「有五個器官機能下降承繼發生」}(there takes place a descent of the five faculties)。按:《顯揚真義》以「生起;出生」(nibbatti)解說「下生」。
\stopitemgroup

\startitemgroup[noteitems]
\item\subnoteref{x367}\NoteKeywordAgamaHead{「正觀(\ccchref{SA.1}{https://agama.buddhason.org/SA/dm.php?keyword=1})」},南傳作\NoteKeywordNikaya{「正見」}(sammādiṭṭhi),菩提比丘長老英譯為\NoteKeywordBhikkhuBodhi{「正確的見解」}(right view),也就是「八聖道」(或譯為「八正道;八支聖道」)中的「正見」。
\stopitemgroup

\startitemgroup[noteitems]
\item\subnoteref{x368}\NoteKeywordNikayaHead{「只是無常的」}(aniccaññeva, aniccaṃ eva),菩提比丘長老英譯為\NoteKeywordBhikkhuBodhi{「為不持久的」}(as impermanent)。
\stopitemgroup

\startitemgroup[noteitems]
\item\subnoteref{x369}\NoteKeywordNikayaHead{「正觀」},南傳作\NoteKeywordNikaya{「正見」}(sammādiṭṭhi),菩提比丘長老英譯為\NoteKeywordBhikkhuBodhi{「正確的見解」}(right view),也就是「八聖道」(或譯為「八正道;八支聖道」)中的「正見」。
\stopitemgroup

\startitemgroup[noteitems]
\item\subnoteref{x370}\NoteKeywordAgamaHead{「封滯者(\ccchref{SA.40}{https://agama.buddhason.org/SA/dm.php?keyword=40})」},南傳作\NoteKeywordNikaya{「攀住者」}(upayo,另譯為「接近;牽引」),菩提比丘長老英譯為\NoteKeywordBhikkhuBodhi{「被佔用者;與之發生關連者」}(one who engaged with)。按:《顯揚真義》以愛、慢、見接近(upagato)五蘊解說此字,另參看\ccchref{SA.39}{https://agama.buddhason.org/SA/dm.php?keyword=39}/\suttaref{SN.22.54}。
\stopitemgroup

\startitemgroup[noteitems]
\item\subnoteref{x371}\NoteKeywordAgamaHead{「根種子(\ccchref{SA.39}{https://agama.buddhason.org/SA/dm.php?keyword=39})」}(mūlabīja),菩提比丘長老英譯為\NoteKeywordBhikkhuBodhi{「根-種子」}(root-seeds),就是由根部繁衍新株者。
\stopitemgroup

\startitemgroup[noteitems]
\item\subnoteref{x372}\NoteKeywordAgamaHead{「莖種子(\ccchref{SA.39}{https://agama.buddhason.org/SA/dm.php?keyword=39})」}(khandhabīja),菩提比丘長老英譯為\NoteKeywordBhikkhuBodhi{「莖-種子」}(stem-seeds),就是由莖部繁衍新株者。
\stopitemgroup

\startitemgroup[noteitems]
\item\subnoteref{x373}\NoteKeywordAgamaHead{「自落種子(\ccchref{SA.39}{https://agama.buddhason.org/SA/dm.php?keyword=39})」},南傳作\NoteKeywordNikaya{「枝種子」}(aggabīja),菩提比丘長老英譯為\NoteKeywordBhikkhuBodhi{「切[枝]-種子」}(cutting-seeds),就是落地生根繁衍新株或接枝繁殖者。
\stopitemgroup

\startitemgroup[noteitems]
\item\subnoteref{x374}\NoteKeywordAgamaHead{「節種子(\ccchref{SA.39}{https://agama.buddhason.org/SA/dm.php?keyword=39})」}(phalubīja),菩提比丘長老英譯為\NoteKeywordBhikkhuBodhi{「莖節-種子」}(joint-seeds),就是由莖節繁衍新株者。
\stopitemgroup

\startitemgroup[noteitems]
\item\subnoteref{x375}\NoteKeywordAgamaHead{「實種子(\ccchref{SA.39}{https://agama.buddhason.org/SA/dm.php?keyword=39})」},南傳作\NoteKeywordNikaya{「種子種子」}(bījabījaññeva,直譯為「以種子為種子」),菩提比丘長老英譯為\NoteKeywordBhikkhuBodhi{「胚種-種子」}(germ-seeds),就是由種子繁衍新株者。
\stopitemgroup

\startitemgroup[noteitems]
\item\subnoteref{x376}\NoteKeywordAgamaHead{「不中風(\ccchref{SA.39}{https://agama.buddhason.org/SA/dm.php?keyword=39})」},南傳作\NoteKeywordNikaya{「無風吹日曬破壞的」}(avātātapahatāni,直譯為「無風無熱破壞」),菩提比丘長老英譯為\NoteKeywordBhikkhuBodhi{「不被風與陽光損害」}(undamaged by wind and sun)。
\stopitemgroup

\startitemgroup[noteitems]
\item\subnoteref{x377}\NoteKeywordNikayaHead{「四識住」}(catasso viññāṇaṭṭhitiyo),菩提比丘長老英譯為\NoteKeywordBhikkhuBodhi{「識的四個駐地」}(the four stations of consciousness),依經義就是指「色、受、想、行」四蘊。
\stopitemgroup

\startitemgroup[noteitems]
\item\subnoteref{x378}\NoteKeywordAgamaHead{「貪喜四取攀緣識住(\ccchref{SA.39}{https://agama.buddhason.org/SA/dm.php?keyword=39})」},意思是「貪喜著識所攀緣執取的那四個安住處」,也就是「貪喜著色、受、想、行」,南傳作\NoteKeywordNikaya{「歡喜、貪」}(nandirāga),菩提比丘長老英譯為\NoteKeywordBhikkhuBodhi{「歡樂與慾望」}(delight and lust)。
\stopitemgroup

\startitemgroup[noteitems]
\item\subnoteref{x379}\NoteKeywordAgamaHead{「取陰俱識(\ccchref{SA.39}{https://agama.buddhason.org/SA/dm.php?keyword=39})」},意思是「執取[其他四]蘊同在的識」,南傳作\NoteKeywordNikaya{「有食物的識」}(viññāṇaṃ sāhāraṃ),菩提比丘長老英譯為\NoteKeywordBhikkhuBodhi{「隨其滋養物一起的識」}(consciousness together with its nutriment)。識的食物(滋養物),依經義就是指「色、受、想、行」四蘊,但《顯揚真義》以「有緣的業識」(sappaccayaṃ kammaviññāṇaṃ)解說。
\stopitemgroup

\startitemgroup[noteitems]
\item\subnoteref{x380}\NoteKeywordAgamaHead{「攀緣(\ccchref{SA.39}{https://agama.buddhason.org/SA/dm.php?keyword=39})」},南傳經文分作「攀住」(upaya,另譯為「接近、牽引」,菩提比丘長老英譯為\NoteKeywordBhikkhuBodhi{「engagement, attachment」})與「所緣」(ārammaṇa,另譯為「緣境;對象」,英譯為「object」),另參看\ccchref{SA.359}{https://agama.buddhason.org/SA/dm.php?keyword=359}。
\stopitemgroup

\startitemgroup[noteitems]
\item\subnoteref{x381}\NoteKeywordAgamaHead{「封滯意生縛斷(\ccchref{SA.39}{https://agama.buddhason.org/SA/dm.php?keyword=39});意生縛斷(\ccchref{SA.64}{https://agama.buddhason.org/SA/dm.php?keyword=64})」},南傳作\NoteKeywordNikaya{「貪已被捨斷」}(rāgo pahīno hoti),菩提比丘長老英譯為\NoteKeywordBhikkhuBodhi{「捨斷了慾望」}(has abandoned lust)。按:「意」通「識」,「封滯」即「攀住」(\ccchref{SA.40}{https://agama.buddhason.org/SA/dm.php?keyword=40}),所以可以理解為「因識的攀住其他四蘊而有的束縛斷了」。
\stopitemgroup

\startitemgroup[noteitems]
\item\subnoteref{x382}\NoteKeywordNikayaHead{「那個無安住處的識」}(Tadappatiṭṭhitaṃ viññāṇaṃ),菩提比丘長老英譯為\NoteKeywordBhikkhuBodhi{「當那個識不被建立」}(When that consciousness is unestablished)。按:「無安住處」(appatiṭṭhitaṃ),另譯為「無住立的;不住立的」,可以是過去分詞,也可以作名詞。
\stopitemgroup

\startitemgroup[noteitems]
\item\subnoteref{x383}\NoteKeywordAgamaHead{「住(\ccchref{SA.39}{https://agama.buddhason.org/SA/dm.php?keyword=39})」},南傳作\NoteKeywordNikaya{「住止的」}(ṭhita,另譯為「住立」),菩提比丘長老英譯為\NoteKeywordBhikkhuBodhi{「堅固;不動搖」}(steady)。
\stopitemgroup

\startitemgroup[noteitems]
\item\subnoteref{x384}\NoteKeywordAgamaHead{「知足(\ccchref{SA.39}{https://agama.buddhason.org/SA/dm.php?keyword=39})」},南傳作\NoteKeywordNikaya{「滿足的」}(santusita,另譯為「善知足的」),菩提比丘長老英譯為\NoteKeywordBhikkhuBodhi{「滿足;心甘情願」}(content)。
\stopitemgroup

\startitemgroup[noteitems]
\item\subnoteref{x385}\NoteKeywordAgamaHead{「非當有(\ccchref{SA.64}{https://agama.buddhason.org/SA/dm.php?keyword=64})」},南傳作\NoteKeywordNikaya{「將消失」}(vibhavissanti,另譯為「將變成非有;當非有」,名詞vibhava「~的消失」),菩提比丘長老英譯為\NoteKeywordBhikkhuBodhi{「將被消滅」}(will be exterminated)。
\stopitemgroup

\startitemgroup[noteitems]
\item\subnoteref{x386}\NoteKeywordNikayaHead{「四輪」}(catuparivaṭṭaṃ),菩提比丘長老英譯為\NoteKeywordBhikkhuBodhi{「四方面」}(four phases)。按:《顯揚真義》說,對於每一蘊以四種輪轉(ekekasmiṃ khandhe catunnaṃ parivaṭṭanavasena),註疏說,在各個蘊上以以四聖諦輪轉(paccekakkhandhesu catunnaṃ ariyasaccānaṃ parivaṭṭanavasena)。又,本經名的paripavatta應該就是這裡的parivaṭṭa。
\stopitemgroup

\startitemgroup[noteitems]
\item\subnoteref{x387}\NoteKeywordAgamaHead{「七處善(\ccchref{SA.42}{https://agama.buddhason.org/SA/dm.php?keyword=42})」},南傳作\NoteKeywordNikaya{「七處熟練的」}(sattaṭṭhānakusalo),菩提比丘長老英譯為\NoteKeywordBhikkhuBodhi{「熟練於七件事」}(skilled in seven cases)。
\stopitemgroup

\startitemgroup[noteitems]
\item\subnoteref{x388}\NoteKeywordAgamaHead{「三種觀義(\ccchref{SA.42}{https://agama.buddhason.org/SA/dm.php?keyword=42})」},南傳作\NoteKeywordNikaya{「三種考察的」}(tividhūpaparikkhī),菩提比丘長老英譯為\NoteKeywordBhikkhuBodhi{「一位三重的調查者」}(a triple investigator)。
\stopitemgroup

\startitemgroup[noteitems]
\item\subnoteref{x389}\NoteKeywordAgamaHead{「愛喜是名色集(\ccchref{SA.42}{https://agama.buddhason.org/SA/dm.php?keyword=42})」},南傳作\NoteKeywordNikaya{「以食集而有色集」}(āhārasamudayā rūpasamudayo),菩提比丘長老英譯為\NoteKeywordBhikkhuBodhi{「由營養物的出現而有色的出現」}(with the arising of nutriment there is the arising of form)。兩者看似不同,但\ccchref{MN.9}{https://agama.buddhason.org/MN/dm.php?keyword=9}有「以渴愛集有食集,以渴愛滅有食滅。」的經文,\ccchref{SA.344}{https://agama.buddhason.org/SA/dm.php?keyword=344}有「云何食集如實知?謂:當來有愛、喜貪俱、彼彼樂著,是名食集,如是食集如實知。」的經文,故南北傳經文對「色集」之說法意趣是相同的。
\stopitemgroup

\startitemgroup[noteitems]
\item\subnoteref{x390}\NoteKeywordAgamaHead{「亦不得於色欲令如是、不令如是(\ccchref{SA.33}{https://agama.buddhason.org/SA/dm.php?keyword=33}-35);應得於眼欲令如是、不令如是(\ccchref{SA.316}{https://agama.buddhason.org/SA/dm.php?keyword=316}-318)」},南傳作\NoteKeywordNikaya{「以及在色上被得到:『令我的色是這樣;令我的色不是這樣。』」}(labbhetha ca rūpe– ‘evaṃ me rūpaṃ hotu, evaṃ me rūpaṃ mā ahosī’ti),菩提比丘長老英譯為\NoteKeywordBhikkhuBodhi{「那將可能有其色:『讓我的色是這樣,讓我的色不是這樣。』」}(it would be possible to have it of form: ‘Let my form be thus; let my form not be thus’)。此處南北傳經文文義似乎相反,但如果將北傳經文理解成從「他者」的角度來看,將南傳經文理解成「『我』自身」的角度來看,則含意是一樣的。
\stopitemgroup

\startitemgroup[noteitems]
\item\subnoteref{x391}\NoteKeywordNikayaHead{「悅意的」}(Attamanā,另譯為「適意的;滿意的」),菩提比丘長老英譯為\NoteKeywordBhikkhuBodhi{「得意洋洋的;興高采烈的」}(elated),或「滿意與喜悅」(was satisfied and delighted)。
\stopitemgroup

\startitemgroup[noteitems]
\item\subnoteref{x392}\NoteKeywordNikayaHead{「執取者」}(upādiyamāno),菩提比丘長老英譯為\NoteKeywordBhikkhuBodhi{「在執著中」}(in clinging)。按:《顯揚真義》以「因渴愛、慢、見而取(把持)者」(taṇhāmānadiṭṭhivasena gaṇhamāno)解說。
\stopitemgroup

\startitemgroup[noteitems]
\item\subnoteref{x393}\NoteKeywordNikayaHead{「歡喜者」}(abhinandamāna,現在分詞),菩提比丘長老英譯為\NoteKeywordBhikkhuBodhi{「在尋歡中」}(in seeking delight)。按:《顯揚真義》以「就以渴愛、慢、見之歡喜而歡喜著」(taṇhāmānadiṭṭhiabhinandanāhiyeva abhinandamāno)解說。
\stopitemgroup

\startitemgroup[noteitems]
\item\subnoteref{x394}\NoteKeywordNikayaHead{「會被染住立的」}(rajanīyasaṇṭhitaṃ),菩提比丘長老英譯為\NoteKeywordBhikkhuBodhi{「顯露渴望難熬的」}(appears tantalizing)。按:《顯揚真義》以「會被染的性質住立;會被貪的緣性住立」(rajanīyena ākārena saṇṭhitaṃ, rāgassa paccayabhāvena ṭhitanti)解說。
\stopitemgroup

\startitemgroup[noteitems]
\item\subnoteref{x395}\NoteKeywordAgamaHead{「境界七善法(\ccchref{MA.120}{https://agama.buddhason.org/MA/dm.php?keyword=120})」},南傳作\NoteKeywordNikaya{「在七善法的行境中」}(satta saddhammagocarā),菩提比丘長老英譯為\NoteKeywordBhikkhuBodhi{「在七個好的性質中來回」}(Ranging in the seven good qualities)。按:《顯揚真義》說,七善法為信、慚、愧、多聞(bāhusaccaṃ)、活力已被發動(āraddhavīriyatā)、念已現起(upaṭṭhitassatitā)、慧。
\stopitemgroup

\startitemgroup[noteitems]
\item\subnoteref{x396}\NoteKeywordAgamaHead{「七覺寶(\ccchref{MA.120}{https://agama.buddhason.org/MA/dm.php?keyword=120})」},南傳作\NoteKeywordNikaya{「七寶」}(Sattaratana),菩提比丘長老英譯為\NoteKeywordBhikkhuBodhi{「七寶石」}(the seven gems) 。按:\ccchref{SA.721}{https://agama.buddhason.org/SA/dm.php?keyword=721}等說,這是指七覺支。
\stopitemgroup

\startitemgroup[noteitems]
\item\subnoteref{x397}\NoteKeywordAgamaHead{「十支道(\ccchref{MA.120}{https://agama.buddhason.org/MA/dm.php?keyword=120})」},南傳作\NoteKeywordNikaya{「十支」}(Dasahaṅgehi),菩提比丘長老英譯為\NoteKeywordBhikkhuBodhi{「十要素」}(the ten factors) 。按:\ccchref{MA.189}{https://agama.buddhason.org/MA/dm.php?keyword=189}等說,此為八支聖道+正智+正解脫,阿羅漢具足此十支。
\stopitemgroup

\startitemgroup[noteitems]
\item\subnoteref{x398}\NoteKeywordAgamaHead{「梵行第一具(\ccchref{MA.120}{https://agama.buddhason.org/MA/dm.php?keyword=120})」},南傳作\NoteKeywordNikaya{「梵行的核心」}(sāro brahmacariyassa),菩提比丘長老英譯為\NoteKeywordBhikkhuBodhi{「對聖潔生活的核心」}(In regard to the core of the holy life)。按:《顯揚真義》說,核心名為果(sāro nāma phalaṃ)。
\stopitemgroup

\startitemgroup[noteitems]
\item\subnoteref{x399}\NoteKeywordAgamaHead{「眾事不移動(\ccchref{MA.120}{https://agama.buddhason.org/MA/dm.php?keyword=120})」},南傳作\NoteKeywordNikaya{「他們在慢類上不動搖」}(Vidhāsu na vikampanti),菩提比丘長老英譯為\NoteKeywordBhikkhuBodhi{「他們在差別待遇上不動搖」}(They do not waver in discrimination)。按:《顯揚真義》以「各種慢」(mānakoṭṭhāsesu)解說「慢類」(Vidhāsu),《滿足希求》則以等、勝、劣「三種慢」(tisso vidhā)解說。
\stopitemgroup

\startitemgroup[noteitems]
\item\subnoteref{x400}\NoteKeywordNikayaHead{「飛入」}(bhajanti,原意為「親近,服侍」),菩提比丘長老英譯為\NoteKeywordBhikkhuBodhi{「去」}(resort, \ccchref{AN.4.33}{https://agama.buddhason.org/AN/an.php?keyword=4.33}),或「飛入」(fly up into, \suttaref{SN.22.78})。
\stopitemgroup

\startitemgroup[noteitems]
\item\subnoteref{x401}\NoteKeywordAgamaHead{「自識種種宿命(\ccchref{SA.46}{https://agama.buddhason.org/SA/dm.php?keyword=46})」},南傳作\NoteKeywordNikaya{「回憶種種前世住處的」}(anekavihitaṃ pubbenivāsaṃ anussaramānā),菩提比丘長老英譯為\NoteKeywordBhikkhuBodhi{「回憶過去他們的種種住所」}(recollect their manifold past abodes)。按:《顯揚真義》說,這裡非因關於神通之力(abhiññāvasena)而說,這被說是關於沙門婆羅門因毘婆舍那之力(vipassanāvasena)回憶前世住處。長老說,《顯揚真義》似乎理解這段佛陀陳述的主旨為他們刻意地就五蘊來回憶過去,但他持不同觀點,認為從後文所述,這裡是指想像回憶過去永恆的我者,只回憶了過去的五蘊,\suttaref{SN.22.47}支持這樣的看法(大意)。
\stopitemgroup

\startitemgroup[noteitems]
\item\subnoteref{x402}\NoteKeywordAgamaHead{「閡(\ccchref{SA.46}{https://agama.buddhason.org/SA/dm.php?keyword=46})」},南傳作\NoteKeywordNikaya{「它變壞」}(ruppati,另譯為「被惱害;被壓迫;被改變),菩提比丘長老英譯為\NoteKeywordBhikkhuBodhi{「變形;不成形」}(it is deformed)。
\stopitemgroup

\startitemgroup[noteitems]
\item\subnoteref{x403}\NoteKeywordAgamaHead{「覺相(\ccchref{SA.46}{https://agama.buddhason.org/SA/dm.php?keyword=46})」},南傳作\NoteKeywordNikaya{「它感受」}(vedayati),菩提比丘長老英譯為\NoteKeywordBhikkhuBodhi{「感覺」}(it feels)。
\stopitemgroup

\startitemgroup[noteitems]
\item\subnoteref{x404}\NoteKeywordAgamaHead{「為作相(\ccchref{SA.46}{https://agama.buddhason.org/SA/dm.php?keyword=46})」},南傳作\NoteKeywordNikaya{「它們造作有為的」}(saṅkhatamabhisaṅkharonti),菩提比丘長老英譯為\NoteKeywordBhikkhuBodhi{「他們建造那條件所成的之物」}(they construct the conditioned)。「對色性它們造作有為的色」(rūpaṃ rūpattāya saṅkhatamabhisaṅkharonti),菩提比丘長老英譯為\NoteKeywordBhikkhuBodhi{「他們建造條件所成的色成為色」}(they construct conditioned form as form)。
\stopitemgroup

\startitemgroup[noteitems]
\item\subnoteref{x405}\NoteKeywordAgamaHead{「為色所食(\ccchref{SA.46}{https://agama.buddhason.org/SA/dm.php?keyword=46})」},南傳作\NoteKeywordNikaya{「我被色食」}(rūpena khajjāmi),菩提比丘長老英譯為\NoteKeywordBhikkhuBodhi{「被色吞噬」}(being devoured by form)。按:《顯揚真義》說:這裡不像狗撕拉肉吃那樣,而是如被穿上污染的衣服,以其因由有惱迫而說:「衣服吃我」(khādati maṃ vattha’’nti)一樣。
\stopitemgroup

\startitemgroup[noteitems]
\item\subnoteref{x406}\NoteKeywordAgamaHead{「滅而不增(\ccchref{SA.46}{https://agama.buddhason.org/SA/dm.php?keyword=46})」},南傳作\NoteKeywordNikaya{「拆解不堆積」}(apacināti, no ācināti),菩提比丘長老英譯為\NoteKeywordBhikkhuBodhi{「支解而不建立」}(dismentales and does not built up)。按:北傳經文若相對於「不增」來看,「正聞本」判「滅」應為「減」應屬合理,《顯揚真義》以「使輪迴滅亡,確實不累積」(vaṭṭaṃ vināseti, neva cināti)解說。
\stopitemgroup

\startitemgroup[noteitems]
\item\subnoteref{x407}\NoteKeywordAgamaHead{「捨而不取(\ccchref{SA.46}{https://agama.buddhason.org/SA/dm.php?keyword=46})」},南傳作\NoteKeywordNikaya{「捨斷不執取」}(pajahati, na upādiyati),菩提比丘長老英譯為\NoteKeywordBhikkhuBodhi{「捨棄而不固執」}(abandons and does not cling)。按:《顯揚真義》以「就捨離(釋放)那個,不取(把持)」(tadeva vissajjeti, na gaṇhāti)解說。
\stopitemgroup

\startitemgroup[noteitems]
\item\subnoteref{x408}\NoteKeywordAgamaHead{「退而不進(\ccchref{SA.46}{https://agama.buddhason.org/SA/dm.php?keyword=46})」},南傳作\NoteKeywordNikaya{「驅散不積聚」} (visineti, na ussineti,也有其它版本作viseneti, na usseneti),菩提比丘長老英譯為\NoteKeywordBhikkhuBodhi{「驅散而不積聚」}(scatters and does not amass)。按:《顯揚真義》以「分散不結合」(vikirati na sampiṇḍeti)解說。
\stopitemgroup

\startitemgroup[noteitems]
\item\subnoteref{x409}\NoteKeywordAgamaHead{「滅而不起(\ccchref{SA.46}{https://agama.buddhason.org/SA/dm.php?keyword=46})」},南傳作\NoteKeywordNikaya{「熄滅不點燃」} (vidhūpeti, na sandhūpeti,原意為「滅亡不燻煙」),菩提比丘長老英譯為\NoteKeywordBhikkhuBodhi{「熄滅而不點燃」}(extinguishes and does not kindle)。按:《顯揚真義》以「使熄滅不點燃」(nibbāpeti na jālāpeti)解說,今準此譯。
\stopitemgroup

\startitemgroup[noteitems]
\item\subnoteref{x410}\NoteKeywordAgamaHead{「滅而不增,寂滅而住(\ccchref{SA.46}{https://agama.buddhason.org/SA/dm.php?keyword=46})」},南傳作\NoteKeywordNikaya{「既不堆積也不拆解,拆解後為住立者」}(nevācināti na apacināti, apacinitvā ṭhito),菩提比丘長老英譯為\NoteKeywordBhikkhuBodhi{「他既不建立也不拆解,但他滯留於已拆解」}(who neither builds up nor dismantles, but who abides having dismantled),並解說,這是指已住於拆解輪迴的阿羅漢。
\stopitemgroup

\startitemgroup[noteitems]
\item\subnoteref{x411}\NoteKeywordAgamaHead{「不為失命(\ccchref{SA.272}{https://agama.buddhason.org/SA/dm.php?keyword=272})」},南傳作\NoteKeywordNikaya{「非生活所迫的」}(na ājīvikāpakatā),菩提比丘長老英譯為\NoteKeywordBhikkhuBodhi{「也不為掙得生計」}(nor to earn a livelihood)。
\stopitemgroup

\startitemgroup[noteitems]
\item\subnoteref{x412}\NoteKeywordAgamaHead{「焚尸火𣕊(栝)(\ccchref{SA.272}{https://agama.buddhason.org/SA/dm.php?keyword=272});燒人殘木(\ccchref{MA.140}{https://agama.buddhason.org/MA/dm.php?keyword=140})」},南傳作\NoteKeywordNikaya{「火葬場的燃燒木柴」}(chavālātaṃ),菩提比丘長老英譯為\NoteKeywordBhikkhuBodhi{「從火葬場燒剩的柴」}(a brand from a funeral pyre)。
\stopitemgroup

\startitemgroup[noteitems]
\item\subnoteref{x413}\NoteKeywordAgamaHead{「不善覺法(\ccchref{SA.272}{https://agama.buddhason.org/SA/dm.php?keyword=272})」},南傳作\NoteKeywordNikaya{「不善尋」}(akusalavitakkā),菩提比丘長老英譯為\NoteKeywordBhikkhuBodhi{「有害的心思」}(unwholesome thoughts)。
\stopitemgroup

\startitemgroup[noteitems]
\item\subnoteref{x414}\NoteKeywordAgamaHead{「欲住寂滅(\ccchref{SA.57}{https://agama.buddhason.org/SA/dm.php?keyword=57})」},南傳作\NoteKeywordNikaya{「就想單獨地住」}(ekova… viharitukāmo,另譯為「想住於單獨」),菩提比丘長老英譯為\NoteKeywordBhikkhuBodhi{「想獨自住」}(wishes to dwell alone)。按:依《顯揚真義》所說,這是在憍賞彌比丘們爭吵(kalahakāle, 即\ccchref{MN.48}{https://agama.buddhason.org/MN/dm.php?keyword=48}所說)後之事。又,\ccchref{Ud.35}{https://agama.buddhason.org/Ud/dm.php?keyword=35}將本經與\ccchref{AN.9.40}{https://agama.buddhason.org/AN/an.php?keyword=9.40}連起來。
\stopitemgroup

\startitemgroup[noteitems]
\item\subnoteref{x415}\NoteKeywordAgamaHead{「跋陀薩羅樹(\ccchref{SA.57}{https://agama.buddhason.org/SA/dm.php?keyword=57})」},南傳作\NoteKeywordNikaya{「吉祥(的)沙羅樹下」}(bhaddasālamūle),菩提比丘長老英譯為\NoteKeywordBhikkhuBodhi{「一棵吉祥的沙羅樹腳下」}(the foot of an auspicious sal tree)。按:「跋陀」顯然是「吉祥的」(bhadda)的音譯,「薩羅樹」一般都音譯為「沙羅樹」、「娑羅樹」(sāla)。
\stopitemgroup

\startitemgroup[noteitems]
\item\subnoteref{x416}\NoteKeywordNikayaHead{「深思」}(parivitakko,直譯為「遍尋」,另譯為「審慮;考慮」),菩提比丘長老英譯為\NoteKeywordBhikkhuBodhi{「深思」}(a reflection)。
\stopitemgroup

\startitemgroup[noteitems]
\item\subnoteref{x417}\NoteKeywordAgamaHead{「是名為行(\ccchref{SA.57}{https://agama.buddhason.org/SA/dm.php?keyword=57})」},南傳作\NoteKeywordNikaya{「那種認為,那是行」}(sā…… samanupassanā saṅkhāro so),菩提比丘長老英譯為\NoteKeywordBhikkhuBodhi{「那認為是形成物」}(That regarding is a formation)。按:這裡所說的「行」(saṅkhāra),菩提比丘長老解說為「由條件所開始的」(conditioned origination),而不是「意志形成的行為」(the action of volitional formation),即不是指五蘊中的「行蘊」,《顯揚真義》以「見之行」(diṭṭhisaṅkhāro)解說,「認為」也以「見之認為」(diṭṭhisamanupassanā)解說。
\stopitemgroup

\startitemgroup[noteitems]
\item\subnoteref{x418}\NoteKeywordNikayaHead{「常見」}(sassatadiṭṭhi),菩提比丘長老英譯為\NoteKeywordBhikkhuBodhi{「永恆論者的見解」}(eternalist view)。按:「常見」(sassatadiṭṭhi)的「常」,慣用「sassata」而不是用「nicca」,但「常的」(nicca)與「永恆的」(sassata),應屬於同義字。
\stopitemgroup

\startitemgroup[noteitems]
\item\subnoteref{x419}\NoteKeywordAgamaHead{「斷見、壞有見(\ccchref{SA.57}{https://agama.buddhason.org/SA/dm.php?keyword=57})」},南傳作\NoteKeywordNikaya{「斷滅見」}(ucchedadiṭṭhi),菩提比丘長老英譯為\NoteKeywordBhikkhuBodhi{「滅絕論」}(annihilationist)。
\stopitemgroup

\startitemgroup[noteitems]
\item\subnoteref{x420}\NoteKeywordAgamaHead{「不勤信而自慢惰(\ccchref{SA.57}{https://agama.buddhason.org/SA/dm.php?keyword=57})」},南傳作\NoteKeywordNikaya{「未達想要者」}(aniṭṭhaṅgato),菩提比丘長老英譯為\NoteKeywordBhikkhuBodhi{「不能下定決心的」}(indecisive)。
\stopitemgroup

\startitemgroup[noteitems]
\item\subnoteref{x421}\NoteKeywordAgamaHead{「先以解釋(\ccchref{SA.58}{https://agama.buddhason.org/SA/dm.php?keyword=58})」},南傳作\NoteKeywordNikaya{「(被)反問教導(調伏)」}(Paṭipucchāvinītā, \suttaref{SN.22.82}/Paṭivinītā, \ccchref{MN.109}{https://agama.buddhason.org/MN/dm.php?keyword=109}),普提比丘長老英譯為「已被訓練」(have been trained, SN/MN)。按:\ccchref{MN.109}{https://agama.buddhason.org/MN/dm.php?keyword=109}的Paṭivinītā一般解讀為「被除去,被驅逐;被征服」,這樣前後文難連貫,《破斥猶豫》未見解說,今解讀為「對-被教導」(Paṭi-vinītā)。其它版作「緣於被教導」(Paṭicca vinītā, sī. pī.);「被我反問教導」(paṭipucchāmi vinītā, syā. kaṃ.)。 
\stopitemgroup

\startitemgroup[noteitems]
\item\subnoteref{x422}\NoteKeywordAgamaHead{「生法計是我(\ccchref{SA.261}{https://agama.buddhason.org/SA/dm.php?keyword=261})」},南傳作\NoteKeywordNikaya{「執取後有『我是』[的觀念]」}(upādāya……asmīti hoti),菩提比丘長老英譯為\NoteKeywordBhikkhuBodhi{「依執著發生『我是』[的觀念]」}(It is by clinging……that [the notion] ‘I am’ occurs)。按:「執取」的原始意思是「取」,所以往下經文中所舉的「取鏡自照」譬喻,就用了這樣的雙關義。「我是」,《顯揚真義》說,這樣,渴愛、慢、見之虛妄情況被轉起(asmīti evaṃ pavattaṃ taṇhāmānadiṭṭhipapañcattayaṃ hoti)。
\stopitemgroup

\startitemgroup[noteitems]
\item\subnoteref{x423}\NoteKeywordAgamaHead{「更無所有(\ccchref{SA.104}{https://agama.buddhason.org/SA/dm.php?keyword=104})」},南傳作\NoteKeywordNikaya{「死後不存在」}(na hoti paraṃ maraṇāti),菩提比丘長老解說,這是邪見的一種:他認為當未證得阿羅漢前,眾生有生死輪迴的主體存在,這是「常見」,但,等到證得阿羅漢後,這個生死輪迴的主體就被斷滅了,這是「斷見」,所以這是一般人常有的選擇性、混合式的「常見與斷見」。
\stopitemgroup

\startitemgroup[noteitems]
\item\subnoteref{x424}\NoteKeywordAgamaHead{「異色有如來(\ccchref{SA.104}{https://agama.buddhason.org/SA/dm.php?keyword=104})」},南傳作\NoteKeywordNikaya{「除了色外有如來」}(aññatra rūpā tathāgatoti,另譯為「如來在色以外的其它處」),菩提比丘長老英譯為\NoteKeywordBhikkhuBodhi{「如來為除了色之外」}(the Tathāgata as apart from form)。按:「如來」一詞,可以是指「佛陀」(十號之一),也可以是一般眾生認為的「生死流轉的主體」(印順法師《如來藏之研究》p.12),這裡,個人認為後者的意思比較合經義,此處《顯揚真義》以「眾生」(Tathāgatoti satto)解說。
\stopitemgroup

\startitemgroup[noteitems]
\item\subnoteref{x425}\NoteKeywordAgamaHead{「如來見法真實如,住無所得,無所施設(\ccchref{SA.104}{https://agama.buddhason.org/SA/dm.php?keyword=104})」},南傳作\NoteKeywordNikaya{「當在此生中真實的、實際的如來未被你得到時」}(diṭṭheva-dhamme saccato thetato tathāgate anupalabbhiyamāne,逐字譯為「見-法-真實-如(堅固、永住)-如來-不-被得到」),菩提比丘長老英譯為\NoteKeywordBhikkhuBodhi{「這裡,就在這一生中,當如來不被你理解為真實的與實際的」}(When the Tathāgata is not apprehended by you as real and actual here in this very life)。按:從巴利語經文逐字譯的內容來看,本段北傳經文的譯文是否已完成,是很可疑的。
\stopitemgroup

\startitemgroup[noteitems]
\item\subnoteref{x426}\NoteKeywordAgamaHead{「為時說耶(\ccchref{SA.104}{https://agama.buddhason.org/SA/dm.php?keyword=104})」},南傳作\NoteKeywordNikaya{「你的那個記說是否是適當的呢」}(kallaṃ nu te taṃ veyyākaraṇaṃ),菩提比丘長老英譯為\NoteKeywordBhikkhuBodhi{「你適合宣稱嗎」}(is it fitting for you to declare)。
\stopitemgroup

\startitemgroup[noteitems]
\item\subnoteref{x427}\NoteKeywordAgamaHead{「從除了這四個地方外」,表示尚有其它「如來死後」的去處,這與佛陀對解脫者的教說不同,參看\ccchref{SA.39}{https://agama.buddhason.org/SA/dm.php?keyword=39}「識不至東、西、南、北,四維、上、下」}之說與\ccchref{SA.962}{https://agama.buddhason.org/SA/dm.php?keyword=962}「火則永滅」之譬喻。
\stopitemgroup

\startitemgroup[noteitems]
\item\subnoteref{x428}\NoteKeywordAgamaHead{「我只告知苦,連同苦的滅」,這段經文同樣出現在\ccchref{MN.22}{https://agama.buddhason.org/MN/dm.php?keyword=22},而與\ccchref{SA.301}{https://agama.buddhason.org/SA/dm.php?keyword=301}的「苦生而生,苦滅而滅」}(\ccchref{SA.262}{https://agama.buddhason.org/SA/dm.php?keyword=262}作「此苦生時生,滅時滅」)的「無我」含意相同,也與\ccchref{SA.335}{https://agama.buddhason.org/SA/dm.php?keyword=335}等的「生時無有來處;滅時無有去處」意趣相同。
\stopitemgroup

\startitemgroup[noteitems]
\item\subnoteref{x429}\NoteKeywordAgamaHead{「凡見法者他見我;凡見我者他見法」(Yo…dhammaṃ passati so maṃ passati; yo maṃ passati so dhammaṃ passati),菩提比丘長老英譯為「看見法者看見我;看見我者看見法」(One who sees the Dhamma sees me; one who sees me sees the Dhamma)。按:\ccchref{It.92}{https://agama.buddhason.org/It/dm.php?keyword=92}:「當看見法時,他看見我。」亦同。又,\ccchref{MA.30}{https://agama.buddhason.org/MA/dm.php?keyword=30}:「若見緣起便見法;若見法便見緣起。」}兩者連接,則可說「見緣起者則見我(佛)」。
\stopitemgroup

\startitemgroup[noteitems]
\item\subnoteref{x430}\NoteKeywordAgamaHead{「思惟解脫(\ccchref{SA.1265}{https://agama.buddhason.org/SA/dm.php?keyword=1265})」},南傳作\NoteKeywordNikaya{「意圖解脫」}(vimokkhāya cetetīti),菩提比丘長老英譯為\NoteKeywordBhikkhuBodhi{「意圖於釋放」}(is intent on deliverance),並解說,雖然vimokkha與vimutti的字根相同(vi + muc),但他們通常出現在不同的場合,為了避免混淆,將前者譯為deliverance,後者譯為liberation,而這裡,它們為同義詞。
\stopitemgroup

\startitemgroup[noteitems]
\item\subnoteref{x431}\NoteKeywordNikayaHead{「到達」}(adhigataṃ),菩提比丘長老認為這「可能是由來已久的訛誤」(probably an old corruption),而提議改讀成「未離」(avigataṃ, has not yet vanished),並解說這是「有學」與「阿羅漢」的根本差異。「有學」人消除了「身見」,不會再「認為」(identifies)五蘊是我,但還留有「餘慢」與「欲」之「我是」(I am)觀念的「無明」尚未根除。這種差異,那些尚未證果的上座比丘不明白,但尊者差摩當時已經是證得「初果」的聖者(有些論師認為已證得「不還果」,有些認為已證得「一來果」),對此相當清楚,所以會有經文往下的發展。按:《顯揚真義》以「『我是』指,這樣被轉起的渴愛、慢被到達」(asmīti evaṃ pavattā taṇhāmānā adhigatā)解說。
\stopitemgroup

\startitemgroup[noteitems]
\item\subnoteref{x432}\NoteKeywordNikayaHead{「容色的香」}(vaṇṇassa gandho),菩提比丘長老採用錫蘭手抄版(vaṇṭassa gandho)英譯為「屬於莖的香味」(the scent belongs to the stalk),正好與北傳經文的「莖香」相同。
\stopitemgroup

\startitemgroup[noteitems]
\item\subnoteref{x433}\NoteKeywordAgamaHead{「戶鉤(\ccchref{SA.262}{https://agama.buddhason.org/SA/dm.php?keyword=262})」},南傳作\NoteKeywordNikaya{「鑰匙」}(avāpuraṇaṃ),菩提比丘長老英譯為\NoteKeywordBhikkhuBodhi{「鑰匙」}(key)。
\stopitemgroup

\startitemgroup[noteitems]
\item\subnoteref{x434}\NoteKeywordAgamaHead{「得踊悅心(\ccchref{SA.262}{https://agama.buddhason.org/SA/dm.php?keyword=262})」},南傳作\NoteKeywordNikaya{「廣大喜、欣悅立刻生起」}(tāvatakeneva uḷāraṃ pītipāmojjaṃ uppajji),菩提比丘長老英譯為\NoteKeywordBhikkhuBodhi{「立刻出現高的狂喜與高興」}(at once a lofty rapture and gladness arose)。
\stopitemgroup

\startitemgroup[noteitems]
\item\subnoteref{x435}\NoteKeywordAgamaHead{「世間智者言有(\ccchref{SA.37}{https://agama.buddhason.org/SA/dm.php?keyword=37})」},南傳作\NoteKeywordNikaya{「被世間中賢智者們認同為存在」}(atthisammataṃ loke paṇḍitānaṃ),菩提比丘長老英譯為\NoteKeywordBhikkhuBodhi{「世間中的智者同意為存在的」}(the wise in the world agree upon as existing)。按:「認同」(sammataṃ),另譯為「同意的;指定的」。
\stopitemgroup

\startitemgroup[noteitems]
\item\subnoteref{x436}\NoteKeywordAgamaHead{「世間世間法(\ccchref{SA.37}{https://agama.buddhason.org/SA/dm.php?keyword=37})」},南傳作\NoteKeywordNikaya{「世間中的世間法」}(loke lokadhammo),菩提比丘長老英譯為\NoteKeywordBhikkhuBodhi{「世間的現象」}(a world-phenomenon in the world)。按:\ccchref{SA.231}{https://agama.buddhason.org/SA/dm.php?keyword=231}以「危脆敗壞」描述「世間」,《顯揚真義》則以「五蘊」(khandhapañcakaṃ)解說「世間法」。
\stopitemgroup

\startitemgroup[noteitems]
\item\subnoteref{x437}\NoteKeywordAgamaHead{「分別(\ccchref{SA.37}{https://agama.buddhason.org/SA/dm.php?keyword=37})」},南傳作\NoteKeywordNikaya{「解析」}(vibhajati,另譯為「分別;解釋」),菩提比丘長老英譯為\NoteKeywordBhikkhuBodhi{「分析它」}(analyses it)。
\stopitemgroup

\startitemgroup[noteitems]
\item\subnoteref{x438}\NoteKeywordNikayaHead{「一般人」}(puthujjanaṃ,另譯為「凡夫」),菩提比丘長老英譯為\NoteKeywordBhikkhuBodhi{「俗人」}(worldling)。
\stopitemgroup

\startitemgroup[noteitems]
\item\subnoteref{x439}\NoteKeywordNikayaHead{「蓮華」}的譬喻,另參看\ccchref{MA.92}{https://agama.buddhason.org/MA/dm.php?keyword=92}、\ccchref{SA.282}{https://agama.buddhason.org/SA/dm.php?keyword=282}、\ccchref{SA.913}{https://agama.buddhason.org/SA/dm.php?keyword=913}、\ccchref{SA.1183}{https://agama.buddhason.org/SA/dm.php?keyword=1183}。
\stopitemgroup

\startitemgroup[noteitems]
\item\subnoteref{x440}\NoteKeywordAgamaHead{「幻師(\ccchref{SA.265}{https://agama.buddhason.org/SA/dm.php?keyword=265})」},南傳作\NoteKeywordNikaya{「幻術師」}(māyākāro),菩提比丘長老英譯為\NoteKeywordBhikkhuBodhi{「使用魔法的人;魔術師」}(a magician)。
\stopitemgroup

\startitemgroup[noteitems]
\item\subnoteref{x441}\NoteKeywordAgamaHead{「壽暖及諸識(\ccchref{SA.265}{https://agama.buddhason.org/SA/dm.php?keyword=265})」},南傳作\NoteKeywordNikaya{「壽、暖與識」}(Āyu usmā ca viññāṇaṃ),菩提比丘長老英譯為\NoteKeywordBhikkhuBodhi{「生命力、熱、識」}(vitality, heat, and consciousness)。按:《顯揚真義》等以「命根/色命根」(jīvitindriyaṃ/rūpajīvitindriyaṃ, \suttaref{SN.22.95}/\ccchref{MN.43}{https://agama.buddhason.org/MN/dm.php?keyword=43})解說「壽」,以「業生火界」(kammajatejodhātu)解說「暖」。
\stopitemgroup

\startitemgroup[noteitems]
\item\subnoteref{x442}\NoteKeywordAgamaHead{「如木無識想(\ccchref{SA.265}{https://agama.buddhason.org/SA/dm.php?keyword=265})」},南傳作\NoteKeywordNikaya{「成為無思者、其牠者之食物」}(parabhattaṃ acetanaṃ),菩提比丘長老英譯為\NoteKeywordBhikkhuBodhi{「其牠者之食物,沒有意志力」}(Food for others, without volition)。
\stopitemgroup

\startitemgroup[noteitems]
\item\subnoteref{x443}\NoteKeywordAgamaHead{「清涼處(\ccchref{SA.265}{https://agama.buddhason.org/SA/dm.php?keyword=265})」},南傳作\NoteKeywordNikaya{「不死足跡的」}(accutaṃ padanti),菩提比丘長老英譯為\NoteKeywordBhikkhuBodhi{「不朽的狀態」}(the imperishable state)。按:《顯揚真義》以「涅槃」(nibbānaṃ)解說。
\stopitemgroup

\startitemgroup[noteitems]
\item\subnoteref{x444}\NoteKeywordNikayaHead{「俱胝」}(koṭi),義譯為「一千萬」,菩提比丘長老說,在平常的交談中,這是指「二十對衣服」(twenty pairs of cloth),但這裡是「十件衣服」(ten garments)的意思。
\stopitemgroup

\startitemgroup[noteitems]
\item\subnoteref{x445}\NoteKeywordAgamaHead{「似剎利女(\ccchref{SA.264}{https://agama.buddhason.org/SA/dm.php?keyword=264})」},南傳作\NoteKeywordNikaya{「偉拉米迦女」}(velāmika)。按:《顯揚真義》說,這是剎帝利與婆羅門、婆羅門與剎帝利腹中所生的(kucchismiṃ jātā)[女兒]。
\stopitemgroup

\startitemgroup[noteitems]
\item\subnoteref{x446}\NoteKeywordAgamaHead{「一切諸行(\ccchref{SA.264}{https://agama.buddhason.org/SA/dm.php?keyword=264})」},南傳作\NoteKeywordNikaya{「那一切諸行」}(sabbe te saṅkhārā),菩提比丘長老英譯為\NoteKeywordBhikkhuBodhi{「那一切形成物」}(all those formations)。按:這裡的「一切諸行」是指「一切有為法」,而不是指五蘊中的「行蘊」,「行蘊」一般是涵蓋所有「受、想、識」以外的心理作用。
\stopitemgroup

\startitemgroup[noteitems]
\item\subnoteref{x447}\NoteKeywordNikayaHead{「名為行的畫」}(caraṇaṃ nāma cittanti),菩提比丘長老英譯為\NoteKeywordBhikkhuBodhi{「叫做『輪迴』的圖畫」}(the picture called ‘Faring On’)。按:《顯揚真義》說,婆羅門宗派有這些名稱(Saṅkhā nāma brāhmaṇapāsaṇḍikā honti),他們在布上畫多種以善惡趣得到不幸的畫,使之認識:「做了這個業後得到這個,做了那個後有那個。」取該畫行走[宣教]。
\stopitemgroup

\startitemgroup[noteitems]
\item\subnoteref{x448}\NoteKeywordAgamaHead{「善治(\ccchref{SA.267}{https://agama.buddhason.org/SA/dm.php?keyword=267})」},南傳作\NoteKeywordNikaya{「在磨得很細緻的」}(suparimaṭṭhe,直譯為「善磨」),菩提比丘長老英譯為\NoteKeywordBhikkhuBodhi{「在完全磨光的」}(a well-polished)。
\stopitemgroup

\startitemgroup[noteitems]
\item\subnoteref{x449}\NoteKeywordNikayaHead{「邊」}(anta,另譯作「極端;極限;目的」),菩提比丘長老英譯為\NoteKeywordBhikkhuBodhi{「部分」}(portion)。
\stopitemgroup

\startitemgroup[noteitems]
\item\subnoteref{x450}\NoteKeywordNikayaHead{「說法者」}參看\suttaref{SN.22.116}。
\stopitemgroup

\startitemgroup[noteitems]
\item\subnoteref{x451}\NoteKeywordAgamaHead{「法師(\ccchref{SA.26}{https://agama.buddhason.org/SA/dm.php?keyword=26})」},南傳作\NoteKeywordNikaya{「說法者」}(dhammakathika),菩提比丘長老英譯為\NoteKeywordBhikkhuBodhi{「一位法的講述者」}(a speaker on the Dhamma)。
\stopitemgroup

\startitemgroup[noteitems]
\item\subnoteref{x452}\NoteKeywordAgamaHead{「內縛所縛(\ccchref{SA.74}{https://agama.buddhason.org/SA/dm.php?keyword=74})」},南傳作\NoteKeywordNikaya{「被內部外部之繫縛繫縛」}(santarabāhirabandhanabaddho),菩提比丘長老英譯為\NoteKeywordBhikkhuBodhi{「被內部與外部之繫縛繫縛的」}(who is bound by inner and outer bondage)。
\stopitemgroup

\startitemgroup[noteitems]
\item\subnoteref{x453}\NoteKeywordAgamaHead{「以縛生(\ccchref{SA.74}{https://agama.buddhason.org/SA/dm.php?keyword=74})」},南傳作\NoteKeywordNikaya{「被繫縛者衰老」}(baddho jīyati),菩提比丘長老英譯為\NoteKeywordBhikkhuBodhi{「在束縛中變老」}(who grows old in bondage)。按:錫蘭本與羅馬拼音版本作「被繫縛者被出生」(baddho jāyati),與北傳相同。
\stopitemgroup

\startitemgroup[noteitems]
\item\subnoteref{x454}\NoteKeywordAgamaHead{「以縛死(\ccchref{SA.74}{https://agama.buddhason.org/SA/dm.php?keyword=74})」},南傳作\NoteKeywordNikaya{「被繫縛者死亡」}(baddho mīyati),菩提比丘長老英譯為\NoteKeywordBhikkhuBodhi{「在束縛中死去」}(who dies in bondage)。
\stopitemgroup

\startitemgroup[noteitems]
\item\subnoteref{x455}\NoteKeywordAgamaHead{「劫波(\ccchref{SA.22}{https://agama.buddhason.org/SA/dm.php?keyword=22})」},南傳作\NoteKeywordNikaya{「葛波」}(kappa),應為同一人的不同音譯。
\stopitemgroup

\startitemgroup[noteitems]
\item\subnoteref{x456}\NoteKeywordAgamaHead{「多修厭(\ccchref{SA.48}{https://agama.buddhason.org/SA/dm.php?keyword=48})」},南傳作\NoteKeywordNikaya{「應該住於多厭」}(nibbidābahulo vihareyya),菩提比丘長老英譯為\NoteKeywordBhikkhuBodhi{「他應該住熱中厭惡」}(he should dwell engrossed in revulsion)。按:《顯揚真義》以「成為多渴望的後」解說(ukkaṇṭhanabahulo hutvā)。
\stopitemgroup

\startitemgroup[noteitems]
\item\subnoteref{x457}\NoteKeywordNikayaHead{「眾生」}(satta),另有「執著;固著」的意思,在巴利語中,這是雙關語,菩提比丘長老英譯為\NoteKeywordBhikkhuBodhi{「生命;生物」}(being)、「黏著」(stuck)。按:《顯揚真義》以「[關於]被黏著的問題」(laggapucchā)解說。
\stopitemgroup

\startitemgroup[noteitems]
\item\subnoteref{x458}\NoteKeywordAgamaHead{「死法(\ccchref{SA.121}{https://agama.buddhason.org/SA/dm.php?keyword=121})」},南傳作\NoteKeywordNikaya{「魔法」}(māradhammo,另譯為「死神法」),菩提比丘長老英譯為\NoteKeywordBhikkhuBodhi{「屬於魔者」}(subject to māra)。按:《顯揚真義》以「死法」(maraṇadhammo)解說。
\stopitemgroup

\startitemgroup[noteitems]
\item\subnoteref{x459}\NoteKeywordNikayaHead{「風沒吹……如直立不動的石柱」},這是生活派(ājivika,另譯為「邪命外道」) 的思想,可能與六師外道中浮陀‧迦旃延(pakudha kaccāyana)這一派的教義有關。
\stopitemgroup

\startitemgroup[noteitems]
\item\subnoteref{x460}\NoteKeywordAgamaHead{「地歸地(\ccchref{SA.156}{https://agama.buddhason.org/SA/dm.php?keyword=156})」},南傳作\NoteKeywordNikaya{「地沒入、隨行地身」}(pathavī pathavīkāyaṃ anupeti anupagacchati),菩提比丘長老英譯為\NoteKeywordBhikkhuBodhi{「地返回並與地-身合併」}(earth returns to and merges with the earth-body)。按:這是唯物論者阿夷多‧翅舍欽婆羅(ajita Kesakambalin)教派的教義,《顯揚真義》以「自身內的地界,連同凡外部的地界」(ajjhattikā pathavīdhātu bāhiraṃ pathavīdhātuṃ)解說「地」、「地身」。
\stopitemgroup

\startitemgroup[noteitems]
\item\subnoteref{x461}\NoteKeywordAgamaHead{「根隨空轉(\ccchref{SA.156}{https://agama.buddhason.org/SA/dm.php?keyword=156})」},南傳作\NoteKeywordNikaya{「諸根轉移到虛空」}(ākāsaṃ indriyāni saṅkamanti),菩提比丘長老英譯為\NoteKeywordBhikkhuBodhi{「器官機能被轉移到空中」}(the faculties are transferred to space)。
\stopitemgroup

\startitemgroup[noteitems]
\item\subnoteref{x462}\NoteKeywordAgamaHead{「乃至未燒可知(\ccchref{SA.156}{https://agama.buddhason.org/SA/dm.php?keyword=156})」},南傳作\NoteKeywordNikaya{「直到墓地為止[哀悼]諸句被知道」}(yāva āḷāhanā padāni paññāyanti),菩提比丘長老英譯為\NoteKeywordBhikkhuBodhi{「葬禮的致詞持續到埋葬場」}(The funeral orations last as far as the charnel ground)。
\stopitemgroup

\startitemgroup[noteitems]
\item\subnoteref{x463}\NoteKeywordAgamaHead{「骨白鴿色(\ccchref{SA.156}{https://agama.buddhason.org/SA/dm.php?keyword=156})」},南傳作\NoteKeywordNikaya{「骨頭成為灰白色」}(kāpotakāni aṭṭhīni bhavanti),菩提比丘長老英譯為\NoteKeywordBhikkhuBodhi{「骨頭變白」}(the bones whiten)。按:「灰白色」(kāpotaka),另譯為「鴿色;鳩色」。
\stopitemgroup

\startitemgroup[noteitems]
\item\subnoteref{x464}\NoteKeywordAgamaHead{「彼非惡因緣(\ccchref{SA.162}{https://agama.buddhason.org/SA/dm.php?keyword=162})」},南傳作\NoteKeywordNikaya{「從那個因由沒有惡的」}(natthi tatonidānaṃ pāpaṃ),菩提比丘長老英譯為\NoteKeywordBhikkhuBodhi{「因此而無惡的」}(because of this there would be no evil)。
\stopitemgroup

\startitemgroup[noteitems]
\item\subnoteref{x465}\NoteKeywordAgamaHead{「亦非招惡(\ccchref{SA.162}{https://agama.buddhason.org/SA/dm.php?keyword=162})」},南傳作\NoteKeywordNikaya{「沒有惡的傳來」}(natthi pāpassa āgamo),菩提比丘長老英譯為\NoteKeywordBhikkhuBodhi{「以及無惡的結果」}(and no outcome of evil)。
\stopitemgroup

\startitemgroup[noteitems]
\item\subnoteref{x466}\NoteKeywordAgamaHead{「護持(\ccchref{SA.162}{https://agama.buddhason.org/SA/dm.php?keyword=162})」},南傳作\NoteKeywordNikaya{「依抑制」}(saṃyamena),菩提比丘長老英譯為\NoteKeywordBhikkhuBodhi{「依自我控制;以自我控制」}(by self-control)。按:《顯揚真義》以「戒(德行)的抑制」(sīlasaṃyamena)解說。
\stopitemgroup

\startitemgroup[noteitems]
\item\subnoteref{x467}\NoteKeywordAgamaHead{「色是我(\ccchref{SA.166}{https://agama.buddhason.org/SA/dm.php?keyword=166})」},南傳作\NoteKeywordNikaya{「我是有色的」}(rūpī attā hoti),菩提比丘長老英譯為\NoteKeywordBhikkhuBodhi{「自我由色所組成」}(the self consists of form)。按:《顯揚真義》解說色、無色只以所緣作「我」之取見(ārammaṇameva ‘‘attā’’ti gahitadiṭṭhi);有色與無色只以所緣與禪定作「我」之取見;非有色非無色只以推理取見(takkamattena gahitadiṭṭhi),一向樂是回憶前生得到推論者生起的見:得禪定者當作意在過去的自體時生起這樣的見。另可參看\ccchref{SA.109}{https://agama.buddhason.org/SA/dm.php?keyword=109}。
\stopitemgroup

\startitemgroup[noteitems]
\item\subnoteref{x468}\NoteKeywordAgamaHead{「觀察忍(\ccchref{SA.892}{https://agama.buddhason.org/SA/dm.php?keyword=892})」},南傳作\NoteKeywordNikaya{「這樣信、勝解這些法」}(ime dhamme evaṃ saddahati adhimuccati),菩提比丘長老英譯為\NoteKeywordBhikkhuBodhi{「安放信心在這些教義上而對其下決心者」}(one who places faith in these teachings and resolves on them)。
\stopitemgroup

\startitemgroup[noteitems]
\item\subnoteref{x469}\NoteKeywordAgamaHead{「離凡夫地(\ccchref{SA.892}{https://agama.buddhason.org/SA/dm.php?keyword=892})」},南傳作\NoteKeywordNikaya{「已進入善士地、超越凡夫地」}(sappurisabhūmiṃ okkanto, vītivatto puthujjanabhūmiṃ),菩提比丘長老英譯為\NoteKeywordBhikkhuBodhi{「已進上等人的程度,超越俗人的程度」}(enter the plane of superior persons, transcended the plane of the worldlings)。按:《顯揚真義》以「已進入聖道」(paviṭṭho ariyamaggaṃ)解說「已進入善士地」。
\stopitemgroup

\startitemgroup[noteitems]
\item\subnoteref{x470}\NoteKeywordAgamaHead{「增上觀察忍(\ccchref{SA.892}{https://agama.buddhason.org/SA/dm.php?keyword=892})」},南傳作\NoteKeywordNikaya{「這些法以慧這樣足夠沉思地接受」}(ime dhammā evaṃ paññāya mattaso nijjhānaṃ khamanti),菩提比丘長老英譯為\NoteKeywordBhikkhuBodhi{「對一位這些法被以慧這樣足夠程度的沉思後接受者」}(one for whom these teachings are accepted thus after being pondered to a sufficient degree with wisdom)。按:「接受」(khamanti),古譯為「堪;忍」。「足夠地」(mattaso,另翻譯為「適量地」),《顯揚真義》以「量地;從量」(pamāṇato)解說。
\stopitemgroup

\startitemgroup[noteitems]
\item\subnoteref{x471}\NoteKeywordNikayaHead{「淨臉」}(sūcimukhī, sucimukhī),為女外道出家尼的名字,菩提比丘長老說,其意思是「淨臉」(Pure Face),按:「mukha」指「嘴」,也指「臉」。
\stopitemgroup

\startitemgroup[noteitems]
\item\subnoteref{x472}\NoteKeywordAgamaHead{「方口食(\ccchref{SA.500}{https://agama.buddhason.org/SA/dm.php?keyword=500})」},南傳作\NoteKeywordNikaya{「臉向四方吃」}(disāmukho bhuñjasī),菩提比丘長老英譯為\NoteKeywordBhikkhuBodhi{「朝[四]方位吃」}(eat facing the [four] quarters)。
\stopitemgroup

\startitemgroup[noteitems]
\item\subnoteref{x473}\NoteKeywordAgamaHead{「為他使命(\ccchref{SA.500}{https://agama.buddhason.org/SA/dm.php?keyword=500})」},南傳作\NoteKeywordNikaya{「遣使行走的實行」}(dūteyyapahiṇagamanānuyogāya),菩提比丘長老英譯為\NoteKeywordBhikkhuBodhi{「以承擔差事與帶信息」}(by undertaking to go on errands and run messages)。
\stopitemgroup

\startitemgroup[noteitems]
\item\subnoteref{x474}\NoteKeywordNikayaHead{「捨棄身體者」}(vossaṭṭhakāyā),菩提比丘長老英譯為\NoteKeywordBhikkhuBodhi{「死心於[關注]牠們的身體」}(relinquish [concern for] their bodies)。按:《顯揚真義》以「遍覺的捕蛇者捕捉後身體被捨棄(ahituṇḍikaparibuddhaṃ agaṇetvā vissaṭṭhakāyā)」解說,含意不明,註疏說,在那[布薩日]時,牠們以不考慮的心(nirapekkhacittatāya)堅持守戒(adhiṭṭhitasīlatāya)永捨身體(pariccattasarīrā)[而想]:「凡希求[我的]皮、血、骨者,讓他們全部拿走」解說比較明確。
\stopitemgroup

\startitemgroup[noteitems]
\item\subnoteref{x475}\NoteKeywordNikayaHead{「捨棄身體者」},參看\suttaref{SN.29.3}。
\stopitemgroup

\startitemgroup[noteitems]
\item\subnoteref{x476}\NoteKeywordNikayaHead{「在香樹根」}(mūlagandhe),菩提比丘長老英譯為\NoteKeywordBhikkhuBodhi{「在香樹根」}(in fragrant roots)。按:《顯揚真義》說,凡依止樹的根是香的而出生者[乾達婆天神],整顆樹都對祂們都適合[居住]。
\stopitemgroup

\startitemgroup[noteitems]
\item\subnoteref{x477}\NoteKeywordAgamaHead{「風雲天(\ccchref{SA.871}{https://agama.buddhason.org/SA/dm.php?keyword=871})」},南傳作\NoteKeywordNikaya{「風雲的天神」}(vātavalāhakā……devā),菩提比丘長老英譯為\NoteKeywordBhikkhuBodhi{「風-雲天」}(wind-cloud devas)。按:季節外的反常變化,如在夏天生起的寒冷;在寒冷時節生起的暑熱,《顯揚真義》說,那是被天神的威力產生的(devatānubhāvena nibbattaṃ)。
\stopitemgroup

\startitemgroup[noteitems]
\item\subnoteref{x478}\NoteKeywordNikayaHead{「生起、存續、生出、顯現、平息、滅、沒」},參看\ccchref{SA.78}{https://agama.buddhason.org/SA/dm.php?keyword=78}/\suttaref{SN.22.30},\ccchref{SA.298}{https://agama.buddhason.org/SA/dm.php?keyword=298}/\suttaref{SN.12.2}。
\stopitemgroup

\startitemgroup[noteitems]
\item\subnoteref{x479}\NoteKeywordNikayaHead{「一切」}(sabbaṃ),菩提比丘長老英譯為\NoteKeywordBhikkhuBodhi{「所有;全部」}(all)。按:《顯揚真義》說,一切有四類:i.一切的一切(sabbasabba):佛證知而指導(neyyaṃ)的一切。ii.一切處(āyatanasabba):如本經所說。iii.一切有身(sakkāyasabba):如\ccchref{MN.1}{https://agama.buddhason.org/MN/dm.php?keyword=1}所說。iv.指定(部分)的一切(padesasabba):對一切法初注意(根對境)生起心、意、心所…對應的意界等。
\stopitemgroup

\startitemgroup[noteitems]
\item\subnoteref{x480}\NoteKeywordNikayaHead{「成為相異的」}(Aññathābhāvī),菩提比丘長老英譯為\NoteKeywordBhikkhuBodhi{「成為其他的」}(becoming otherwise)。
\stopitemgroup

\startitemgroup[noteitems]
\item\subnoteref{x481}\NoteKeywordAgamaHead{「有一一住(\ccchref{SA.309}{https://agama.buddhason.org/SA/dm.php?keyword=309})」},南傳作\NoteKeywordNikaya{「獨住」}(ekavihārī,直譯為「一住」),菩提比丘長老英譯為\NoteKeywordBhikkhuBodhi{「獨自住者」}(a lone dweller)。
\stopitemgroup

\startitemgroup[noteitems]
\item\subnoteref{x482}\NoteKeywordAgamaHead{「有第二住(\ccchref{SA.309}{https://agama.buddhason.org/SA/dm.php?keyword=309})」},南傳作\NoteKeywordNikaya{「有伴同住」}(sadutiyavihārī,直譯為「有第二住」),菩提比丘長老英譯為\NoteKeywordBhikkhuBodhi{「與伴同住」}(dwelling with a partner)。
\stopitemgroup

\startitemgroup[noteitems]
\item\subnoteref{x483}\NoteKeywordNikayaHead{「人獨住的」}(manussarāhasseyyakāni, manussarāhaseyyakāni,另譯為「對人靜寂的」),菩提比丘長老英譯為\NoteKeywordBhikkhuBodhi{「從人群中隱藏;隱藏於讓人找不到的地方」}(hidden from people)。
\stopitemgroup

\startitemgroup[noteitems]
\item\subnoteref{x484}\NoteKeywordNikayaHead{「優波先那」}(upaseno),《顯揚真義》說,優波先那上座是法匠的最小弟弟(dhammasenāpatino kaniṭṭhabhātika upasenattherassa, \suttaref{SN.35.69}),《優陀那註釋》說,這位上座是尊者舍利弗的最小弟弟(Ayañhi thero āyasmato sāriputtassa kaniṭṭhabhātā, \ccchref{Ud.39}{https://agama.buddhason.org/Ud/dm.php?keyword=39}),佛陀說他是諸弟子中最端嚴者(\ccchref{AN.1.213}{https://agama.buddhason.org/AN/an.php?keyword=1.213})。
\stopitemgroup

\startitemgroup[noteitems]
\item\subnoteref{x485}\NoteKeywordAgamaHead{「世間空;空世間(\ccchref{SA.232}{https://agama.buddhason.org/SA/dm.php?keyword=232})」},南傳作\NoteKeywordNikaya{「世間是空」}(suñño loko,逐字譯為「空-世間」),菩提比丘長老英譯為\NoteKeywordBhikkhuBodhi{「世間是空」}(Empty is the world)。
\stopitemgroup

\startitemgroup[noteitems]
\item\subnoteref{x486}\NoteKeywordNikayaHead{「無應該被責備的」}(Anupavajjaṃ,另譯為「無罪的」,與sa-upavajja「有應該被責備的」相對),菩提比丘長老英譯為\NoteKeywordBhikkhuBodhi{「無可責難地;無罪地」}(blamelessly)。按:《顯揚真義》說,「無應該被責備的」指不流轉、不再生(appavattikaṃ appaṭisandhikaṃ)。尊者闡陀取刀切斷咽喉,就在那個剎那他出現死亡的恐懼,現起趣相(gatinimittaṃ),他知道自己仍是凡夫狀態(puthujjanabhāvaṃ)後,激起急迫感之心,建立毘婆舍那後,當緊把握行時(saṅkhāre pariggaṇhanto),得到阿羅漢性(境界),成為齊頭者(samasīsī-得解脫與入滅同時者)而般涅槃。《破斥猶豫》以「抵達的、不再生的」(anuppattikaṃ appaṭisandhikaṃ, \ccchref{MN.144}{https://agama.buddhason.org/MN/dm.php?keyword=144})解說。長老認為,從經文看,尊者闡陀也可能自殺前就已是阿羅漢了。北傳「[我]供養世尊事,於今畢矣」義同「無應該被責備的」,印證長老的看法。
\stopitemgroup

\startitemgroup[noteitems]
\item\subnoteref{x487}\NoteKeywordAgamaHead{「有所依者(\ccchref{SA.1266}{https://agama.buddhason.org/SA/dm.php?keyword=1266})」},南傳作\NoteKeywordNikaya{「對依止者來說」}(nissitassa),菩提比丘長老英譯為\NoteKeywordBhikkhuBodhi{「對一個依賴的人」}(For one who is dependent, SN),智髻比丘長老英譯為「在一個依賴的人」(in one who is dependent, MN)。按:《顯揚真義》等說,以渴愛、慢、見為依止者(taṇhāmānadiṭṭhīhi nissitassa, \suttaref{SN.35.87}/\ccchref{MN.144}{https://agama.buddhason.org/MN/dm.php?keyword=144})解說,《勝義燈》說,在色等之行上以渴愛、見為依止者(rūpādisaṅkhāre taṇhādiṭṭhīhi nissitassa, \ccchref{Ud.74}{https://agama.buddhason.org/Ud/dm.php?keyword=74})解說,長老說,因為尊者闡陀不能忍受病苦,所以有這段提醒(大意)。
\stopitemgroup

\startitemgroup[noteitems]
\item\subnoteref{x488}\NoteKeywordAgamaHead{「動搖(\ccchref{SA.1266}{https://agama.buddhason.org/SA/dm.php?keyword=1266})」},南傳作\NoteKeywordNikaya{「搖動」}(calitaṃ),菩提比丘長老英譯為\NoteKeywordBhikkhuBodhi{「搖擺不定」}(wavering, SN),智髻比丘長老英譯相同。按:《顯揚真義》等以「有掙扎(翻騰)」(vipphanditaṃ hoti, \suttaref{SN.35.87}/\ccchref{MN.144}{https://agama.buddhason.org/MN/dm.php?keyword=144})解說。
\stopitemgroup

\startitemgroup[noteitems]
\item\subnoteref{x489}\NoteKeywordAgamaHead{「趣向(\ccchref{SA.1266}{https://agama.buddhason.org/SA/dm.php?keyword=1266})」},南傳作\NoteKeywordNikaya{「傾斜」}(nati),菩提比丘長老英譯為\NoteKeywordBhikkhuBodhi{「傾斜」}(inclination, SN),智髻比丘長老英譯為「偏見」(bias, MN)。按:《顯揚真義》等以「渴愛的傾斜」(taṇhānatiyā)解說,而以「在有的目標之執著(阿賴耶)、欲求、纏上無念」(bhavatthāya ālayanikantipariyuṭṭhāne asati, \suttaref{SN.35.87}/\ccchref{MN.144}{https://agama.buddhason.org/MN/dm.php?keyword=144})解說「無傾斜」。
\stopitemgroup

\startitemgroup[noteitems]
\item\subnoteref{x490}\NoteKeywordAgamaHead{「二法(\ccchref{SA.213}{https://agama.buddhason.org/SA/dm.php?keyword=213});二(\ccchref{SA.214}{https://agama.buddhason.org/SA/dm.php?keyword=214})」},南傳作\NoteKeywordNikaya{「一對」}(dvayaṃ,另譯為「二種的、二者」),菩提比丘長老英譯為\NoteKeywordBhikkhuBodhi{「二數;一對」}(the dyad)。按:「眼」的識知對象為「色」,不是「聲音……」等其它外境,因此「眼」與「色」成為「一對」,其它諸根的情況亦同。
\stopitemgroup

\startitemgroup[noteitems]
\item\subnoteref{x491}\NoteKeywordAgamaHead{「二法(\ccchref{SA.213}{https://agama.buddhason.org/SA/dm.php?keyword=213});二(\ccchref{SA.214}{https://agama.buddhason.org/SA/dm.php?keyword=214})」},南傳作\NoteKeywordNikaya{「一對」}(dvayaṃ),參看\suttaref{SN.35.92}/\ccchref{SA.213}{https://agama.buddhason.org/SA/dm.php?keyword=213}。
\stopitemgroup

\startitemgroup[noteitems]
\item\subnoteref{x492}\NoteKeywordAgamaHead{「被接觸者意圖,被接觸者認知」,這樣的經文,也出現在\ccchref{SA.276}{https://agama.buddhason.org/SA/dm.php?keyword=276}:「觸緣想……觸緣思」}。「意圖」(ceteti),\ccchref{SA.359}{https://agama.buddhason.org/SA/dm.php?keyword=359}譯為「思量」。
\stopitemgroup

\startitemgroup[noteitems]
\item\subnoteref{x493}\NoteKeywordAgamaHead{「正信心不二(\ccchref{SA.279}{https://agama.buddhason.org/SA/dm.php?keyword=279})」},南傳作\NoteKeywordNikaya{「以信為伴侶」}(saddhādutiyā),菩提比丘長老英譯為\NoteKeywordBhikkhuBodhi{「以信為他們的伙伴」}(with faith their partner)。按:「伴侶」(dutiyā),另譯為「第二」,這樣,「正信心不二」豈非應作「正信為二」?
\stopitemgroup

\startitemgroup[noteitems]
\item\subnoteref{x494}\NoteKeywordAgamaHead{「平等捨苦樂(\ccchref{SA.279}{https://agama.buddhason.org/SA/dm.php?keyword=279})」},南傳作\NoteKeywordNikaya{「在苦樂兩觸上應該變成無關心,不被任何者喜樂、妨礙」}(phassadvayaṃ sukhadukkhe upekkhe, anānuruddho aviruddha kenaci),菩提比丘長老英譯為\NoteKeywordBhikkhuBodhi{「平等地看待快樂與痛苦,不被任何所吸引或厭惡」}(Look evenly on both the pleasant and painful, Not drawn or repelled by anything)。按:「應該變成無關心-平等地看待」(upekkhe),可能被解為「捨」(upekkhā)。
\stopitemgroup

\startitemgroup[noteitems]
\item\subnoteref{x495}\NoteKeywordAgamaHead{「見以見為量(\ccchref{SA.312}{https://agama.buddhason.org/SA/dm.php?keyword=312})」},南傳作\NoteKeywordNikaya{「在所見中將只有所見這麼多」}(diṭṭhe diṭṭhamattaṃ bhavissati),菩提比丘長老英譯為\NoteKeywordBhikkhuBodhi{「在所見中將只有所見」}(in the seen there will be merely the seen)。按:這句話一般簡為「看只是看」,而「只有……這麼多」(mattaṃ,另譯為「量、小量的、程度的」),應該就是「見以見為量」中的「為量」,《顯揚真義》說,在色處中以眼識而在所見中將只有所見這麼多,因為眼識在色上只看見色這麼多,沒有常等自性(na niccādisabhāvaṃ)……又或在所見中之看見(diṭṭhe diṭṭhaṃ);在色上色的識知(rūpe rūpavijānananti)名為眼識……我的心將只有眼識這麼多(cakkhuviññāṇamattameva me cittaṃ bhavissatīti),這是說,當色進入感官範圍時(āpāthagatarūpe ),眼識不被染、不憤怒、不變愚癡……。
\stopitemgroup

\startitemgroup[noteitems]
\item\subnoteref{x496}\NoteKeywordNikayaHead{「不以那個」}(na tena),菩提比丘長老英譯為\NoteKeywordBhikkhuBodhi{「將不『被那』」}(will not be 'by that.')。按:《顯揚真義》解說,此指「將不以那個成為被貪染著的,或被瞋憤怒的,或被癡變愚昧的。」(Na tenāti tena rāgena vā ratto, dosena vā duṭṭho, mohena vā mūḷho na bhavissati.)
\stopitemgroup

\startitemgroup[noteitems]
\item\subnoteref{x497}\NoteKeywordAgamaHead{「若汝非於彼(\ccchref{SA.312}{https://agama.buddhason.org/SA/dm.php?keyword=312})」},南傳作\NoteKeywordNikaya{「不在那裡」}(na tattha),菩提比丘長老英譯為\NoteKeywordBhikkhuBodhi{「不在那裡」}(will not be 'therein.')。按:《顯揚真義》說,此指「在那個所見的、所聽聞的、所覺的、所識知的中將不被束縛、黏著、住立。」(tadā tvaṃ na tattha tasmiṃ diṭṭhe vā sutamutaviññāte vā paṭibaddho allīno patiṭṭhito nāma bhavissasi.)
\stopitemgroup

\startitemgroup[noteitems]
\item\subnoteref{x498}\NoteKeywordNikayaHead{「接觸觸後」}(Phassaṃ phussa ),菩提比丘長老英譯為\NoteKeywordBhikkhuBodhi{「感受了觸後」}(Hving felt a contact)。按:「觸」(phassa),一般用在根、境、識和合生「觸」的情形,但這裡顯然是與身根的「所觸」(phoṭṭhabba)混用了,而「接觸」(phussa)為動詞「phusati」的「連續體」語態。
\stopitemgroup

\startitemgroup[noteitems]
\item\subnoteref{x499}\NoteKeywordNikayaHead{「他這樣進行念」}(evaṃ so caratī sato),菩提比丘長老英譯為\NoteKeywordBhikkhuBodhi{「一個人以這樣的方式留神地進行」}(One fares mindfully in such a way)。
\stopitemgroup

\startitemgroup[noteitems]
\item\subnoteref{x500}\NoteKeywordNikayaHead{「諸法變成明顯」}(dhammā pātubhavanti),菩提比丘長老英譯為\NoteKeywordBhikkhuBodhi{「事象成為明白的」}(phenomena become manifest)。按:《顯揚真義》以止觀諸法(samathavipassanādhammā)解說這裡的諸法,但長老認為是內外處(法)顯出無常、苦、非我,如\suttaref{SN.35.99}所說。
\stopitemgroup

\startitemgroup[noteitems]
\item\subnoteref{x501}\NoteKeywordAgamaHead{「如實知顯現(\ccchref{SA.206}{https://agama.buddhason.org/SA/dm.php?keyword=206});如實顯現(\ccchref{SA.207}{https://agama.buddhason.org/SA/dm.php?keyword=207},\ccchref{SA.367}{https://agama.buddhason.org/SA/dm.php?keyword=367},\ccchref{SA.368}{https://agama.buddhason.org/SA/dm.php?keyword=368})」},南傳作\NoteKeywordNikaya{「如實知道」}(yathābhūtaṃ pajānāti),菩提比丘長老英譯為\NoteKeywordBhikkhuBodhi{「如他們真實地理解事情」}(understands things as they really are),\suttaref{SN.35.160}也作「如實明瞭」(yathābhūtaṃ okkhāyati),菩提比丘長老英譯為\NoteKeywordBhikkhuBodhi{「如他們真實的對他變得明顯」}(become manifest to him as they really are)。
\stopitemgroup

\startitemgroup[noteitems]
\item\subnoteref{x502}\NoteKeywordNikayaHead{「解脫圓熟法」}(vimuttiparipācaniyā dhammā),菩提比丘長老英譯為\NoteKeywordBhikkhuBodhi{「釋放成熟的情況」}(the state that ripen in liberation)。按:《顯揚真義》等列舉使解脫圓熟的三類十五法:①信等五根②無常想、於無常苦想、於苦無我想、捨斷想、離貪想等五種與洞察有關的想(pañca nibbedhabhāgiyā saññā)③對昧西其長老的講述:善友等五法(參看\ccchref{AN.9.3}{https://agama.buddhason.org/AN/an.php?keyword=9.3}/\ccchref{Ud.31}{https://agama.buddhason.org/Ud/dm.php?keyword=31}/\ccchref{MA.56}{https://agama.buddhason.org/MA/dm.php?keyword=56})。
\stopitemgroup

\startitemgroup[noteitems]
\item\subnoteref{x503}\NoteKeywordAgamaHead{「優陀延那(\ccchref{SA.1165}{https://agama.buddhason.org/SA/dm.php?keyword=1165})」},南傳作\NoteKeywordNikaya{「優填那王」}(rājā udeno),菩提比丘長老解說他是「憍賞彌」(Kosambī,另譯為「憍賞彌」)國國王。
\stopitemgroup

\startitemgroup[noteitems]
\item\subnoteref{x504}\NoteKeywordNikayaHead{「身未修習」}(abhāvitakāyā),菩提比丘長老英譯為\NoteKeywordBhikkhuBodhi{「在身上未開發」}(undeveloped in body)。按:《顯揚真義》說,「身未修習」指「五門身未修習」(abhāvitapañcadvārikakāyā),長老說,也就是缺乏感官的自制(根自制;根律儀)。
\stopitemgroup

\startitemgroup[noteitems]
\item\subnoteref{x505}\NoteKeywordNikayaHead{「玩(作)種種跳戲」}(kānici kānici seleyyakāni(selissakāni) karonti),菩提比丘長老英譯為\NoteKeywordBhikkhuBodhi{「他們玩種種惡作劇」}(they played various pranks)。按:《顯揚真義》說,這是一個人跳上另一個背部往上堆積(cito)的行走遊戲(caṅkamanakīḷanāni)。
\stopitemgroup

\startitemgroup[noteitems]
\item\subnoteref{x506}\NoteKeywordNikayaHead{「在懦弱者與堅強者上違犯者」}(virajjamānā sataṇhātaṇhesu,原意為「在有渴愛無渴愛上離染」),菩提比丘長老依據錫蘭本(virajjhamānā tasathāvaresu)英譯為「他們折磨懦弱者與堅強者」(They molest both frail and firm)。按:《顯揚真義》以「在有渴愛離渴愛上」解說「在有渴愛無渴愛上」(Sataṇhātaṇhesūti sataṇhanittaṇhesu),無助於前後文意的貫通,今依錫蘭本譯。
\stopitemgroup

\startitemgroup[noteitems]
\item\subnoteref{x507}\NoteKeywordAgamaHead{「三浴誦三典(\ccchref{SA.255}{https://agama.buddhason.org/SA/dm.php?keyword=255})」},南傳作\NoteKeywordNikaya{「清早沐浴與三吠陀」}(pāto sinānañca tayo ca vedā),菩提比丘長老英譯為\NoteKeywordBhikkhuBodhi{「在黎明沐浴,(學習)三吠陀」}(Bathing at dawn, [study of] the three Vedas)。按:「三浴」指一天沐浴三次的戒規,「三典」就是指「梨具吠陀、夜柔吠陀、沙摩吠陀」等「三吠陀」。「吠陀」(veda),另音譯為「韋陀;毘陀論經」,另有「信受;宗教感情;知;智」的意思。
\stopitemgroup

\startitemgroup[noteitems]
\item\subnoteref{x508}\NoteKeywordNikayaHead{「些細的修習所做」}(katā kiñcikkhabhāvanā),菩提比丘長老依據《顯揚真義》的解說英譯為「被用來增加他們的世俗收益」(Are used to increase their worldly gains)。按:《顯揚真義》以「這只是背誦,或為了些細財物的增加所作之意」(Ayameva vā pāṭho, āmisakiñcikkhassa vaḍḍhanatthāya katanti attho)解說。
\stopitemgroup

\startitemgroup[noteitems]
\item\subnoteref{x509}\NoteKeywordNikayaHead{「少心的」}(parittacetaso),菩提比丘長老英譯為\NoteKeywordBhikkhuBodhi{「有限的心」}(a limited mind)。按:《顯揚真義》說:以念未現起的、以心污染的為少心的(anupaṭṭhitasatitāya saṃkilesacittena parittacitto),以念已現起的、以心無污染的為無量心的(upaṭṭhitasatitāya nikkilesacittena appamāṇacitto)。
\stopitemgroup

\startitemgroup[noteitems]
\item\subnoteref{x510}\NoteKeywordAgamaHead{「有六觸入處地獄(\ccchref{SA.210}{https://agama.buddhason.org/SA/dm.php?keyword=210})」},南傳作\NoteKeywordNikaya{「名叫六觸處的地獄」}(chaphassāyatanikā nāma nirayā),菩提比丘長老英譯為\NoteKeywordBhikkhuBodhi{「名叫『觸的六重基地』之地獄」}(the hell named  'contact's Sixfold Base.')。按:《顯揚真義》說,沒有個別名叫六觸處地獄,因為全安立於31大地獄(ekatiṃsamahānirayesu)中,這裡所說與無間大地獄(avīcimahānirayaṃ)有關。若依\suttaref{SN.56.43}所說,則與「大熱地獄」(mahāpariḷāha niraya)有關,參看\ccchref{SA.422}{https://agama.buddhason.org/SA/dm.php?keyword=422}。而無間地獄的眾生,受苦無間斷,直到業報盡而往生他處為止。
\stopitemgroup

\startitemgroup[noteitems]
\item\subnoteref{x511}\NoteKeywordAgamaHead{「有六觸入處天(\ccchref{SA.210}{https://agama.buddhason.org/SA/dm.php?keyword=210})」},南傳作\NoteKeywordNikaya{「名叫六觸處的天界」}(chaphassāyatanikā nāma saggā),菩提比丘長老英譯為\NoteKeywordBhikkhuBodhi{「名叫『觸的六重基地』之天」}(the heaven named  'contact's Sixfold Base.')。按:《顯揚真義》說,天界就是前生的三十三天(tāvatiṃsapurameva),地獄給與一向苦的狀態,天界給與一向樂的狀態,以一向享樂而生放逸,都不能住於梵行道的生活(maggabrahmacariyavāsaṃ),但人間樂與苦混合,只在這裡苦界與天界都被了知,這名為梵行道的業地(kammabhūmi)被你們得到,所以經文會說「由於你們得到梵行生活的機會」。
\stopitemgroup

\startitemgroup[noteitems]
\item\subnoteref{x512}\NoteKeywordNikayaHead{「從被有貪隨行者」}(bhavarāgānusārībhi),菩提比丘長老英譯為\NoteKeywordBhikkhuBodhi{「沿存在之流流動者」}(who flow along in the stream of existence)。「有」(bhava),即十二緣起的「有」。
\stopitemgroup

\startitemgroup[noteitems]
\item\subnoteref{x513}\NoteKeywordAgamaHead{「近住弟子(\ccchref{SA.235}{https://agama.buddhason.org/SA/dm.php?keyword=235})」},南傳作\NoteKeywordNikaya{「徒弟」}(antevāsika,直譯為「內住者」,另譯為「阿闍梨的弟子」,即「還需要跟隨阿闍梨住的弟子」),菩提比丘長老英譯為\NoteKeywordBhikkhuBodhi{「學生」}(students)。
\stopitemgroup

\startitemgroup[noteitems]
\item\subnoteref{x514}\NoteKeywordAgamaHead{「於此邊住者(\ccchref{SA.235}{https://agama.buddhason.org/SA/dm.php?keyword=235})」},南傳作\NoteKeywordNikaya{「它們住於他之內」}(tyāssa anto vasanti),菩提比丘長老英譯為\NoteKeywordBhikkhuBodhi{「它們住於他之內」}(They dwell within him)。按:徒弟(antevāsika),直譯為「內住者」,也譯為「師父(阿闍梨)的弟子」,經文說,如果生起「隨順結的惡不善法」,那「隨順結的惡不善法」猶如「徒弟」與他(師父)住在一起一樣,也就是「住於內」(anto vasanti),所以說他「有徒弟」(這時他的「徒弟」就是「隨順結的惡不善法」),這是「雙關語」。這裡北傳經文的「邊住」,可能即巴利經文的「住於內」(anto vasanti),因為「邊」(anta)經語尾變化可以是「anto」,而「anto」卻也是介系詞「內;內部」的意思,而從前後文義來看,後者之解讀較為適當。
\stopitemgroup

\startitemgroup[noteitems]
\item\subnoteref{x515}\NoteKeywordAgamaHead{「彼比丘行此法者(\ccchref{SA.235}{https://agama.buddhason.org/SA/dm.php?keyword=235})」},南傳作\NoteKeywordNikaya{「它們征服他」}(Te naṃ samudācaranti)。菩提比丘長老英譯為\NoteKeywordBhikkhuBodhi{「它們攻擊他」}(They assail him)。按:「征服」(samudācarati)這個動詞原意為「實行;熟習」,《顯揚真義》以征服(adhibhavanti)、覆蓋(ajjhottharanti)、使學習/教導(sikkhāpenti,教導醫術、差使事)解說,今準此譯。如果生起「隨順結的惡不善法」,那「隨順結的惡不善法」猶如「師父」監管徒弟一樣,所以說他「有師父」(這時他的「師父」就是「隨順結的惡不善法」),這是「雙關語」。而北傳經文顯然採取「行;實行」的意思,這與「師;阿闍梨」(ācariya)有「行」的意思,也同樣可以構成「雙關語」。
\stopitemgroup

\startitemgroup[noteitems]
\item\subnoteref{x516}\NoteKeywordAgamaHead{「我見(\ccchref{SA.202}{https://agama.buddhason.org/SA/dm.php?keyword=202})」},南傳作\NoteKeywordNikaya{「我隨見」}(attānudiṭṭhi),菩提比丘長老英譯為\NoteKeywordBhikkhuBodhi{「自我的見解」}(the view of self)。按:「隨見」(anudiṭṭhi),另譯為「邪見;見」。
\stopitemgroup

\startitemgroup[noteitems]
\item\subnoteref{x517}\NoteKeywordNikayaHead{「滅沒的他不再回來」}(atthaṅgato so na puneti),菩提比丘長老依錫蘭本(na pamāṇaṃ eti)英譯為「死後,他不能被衡量」(passed away, he cannot be measurred),並解說錫蘭本依偈頌的押韻來看才是對的。
\stopitemgroup

\startitemgroup[noteitems]
\item\subnoteref{x518}\NoteKeywordNikayaHead{「強烈的」}(adhimattānaṃ),菩提比丘長老英譯為\NoteKeywordBhikkhuBodhi{「顯眼的;顯著的」}(prominent)。按:《顯揚真義》說,想要的所緣(iṭṭhārammaṇaṃ)、會被能被染的對象佔據心,這裡說的是什麼?這裡,指尖大小的寶石真珠等會被能被染的對象為強烈的所緣。
\stopitemgroup

\startitemgroup[noteitems]
\item\subnoteref{x519}\NoteKeywordAgamaHead{「所害(\ccchref{SA.1172}{https://agama.buddhason.org/SA/dm.php?keyword=1172})」},南傳作\NoteKeywordNikaya{「被打」}(haññati,另譯為「被害;被殺」),菩提比丘長老英譯為\NoteKeywordBhikkhuBodhi{「被攻擊」}(is attacked)。
\stopitemgroup

\startitemgroup[noteitems]
\item\subnoteref{x520}\NoteKeywordAgamaHead{「不閡洲渚(\ccchref{SA.1174}{https://agama.buddhason.org/SA/dm.php?keyword=1174})」},南傳作\NoteKeywordNikaya{「不在高地堆積」}(na thale ussīdissati),菩提比丘長老英譯為\NoteKeywordBhikkhuBodhi{「沒沖上高地」}(does not get cast up on high ground)。按:《顯揚真義》以「(將)不登上高地」(thalaṃ nābhiruhissati)解說。「在高地」(thale),另譯為「在陸地;在乾地」。
\stopitemgroup

\startitemgroup[noteitems]
\item\subnoteref{x521}\NoteKeywordAgamaHead{「住處(\ccchref{SA.1173}{https://agama.buddhason.org/SA/dm.php?keyword=1173})」},南傳作\NoteKeywordNikaya{「住」}(vihāro,另譯為「住法;住處;生活模式),菩提比丘長老英譯為\NoteKeywordBhikkhuBodhi{「當他像這樣住」}(as he dwells thus)。
\stopitemgroup

\startitemgroup[noteitems]
\item\subnoteref{x522}\NoteKeywordAgamaHead{「緊獸(\ccchref{SA.1175}{https://agama.buddhason.org/SA/dm.php?keyword=1175})」},南傳作\NoteKeywordNikaya{「緊叔迦」}(kiṃsuko,逐字譯為「什麼-鸚鵡」,另譯為「赤花樹;肉色花;鸚鵡樹;樹森林的火焰;拙劣的柚木樹;紫鉚樹」,是蝶形花科紫礦屬下的植物樹種,又叫膠蟲樹),菩提比丘長老英譯照錄原文,但解說:依字面的意思為「它是什麼?」(what's it),可能起源於古印度的民間謎語,在〈本生〉中說到「緊叔迦」在發芽時,像燒過的殘株(charred stump);葉子變綠時,像榕樹;開花時,像肉片(按:呈鮮紅色,中間花瓣形狀像鸚鵡的嘴);結果時,像金合歡樹(acacia)。
\stopitemgroup

\startitemgroup[noteitems]
\item\subnoteref{x523}\NoteKeywordAgamaHead{「尸利沙(\ccchref{SA.1175}{https://agama.buddhason.org/SA/dm.php?keyword=1175})」},南傳作\NoteKeywordNikaya{「金合歡樹」}(sirīsoti),菩提比丘長老英譯為\NoteKeywordBhikkhuBodhi{「金合歡樹;洋槐;刺槐」}(an acacia tree)。
\stopitemgroup

\startitemgroup[noteitems]
\item\subnoteref{x524}\NoteKeywordAgamaHead{「尼拘婁陀樹(\ccchref{SA.1175}{https://agama.buddhason.org/SA/dm.php?keyword=1175})」},南傳作\NoteKeywordNikaya{「榕樹」}(nigrodhoti,另譯為「尼拘律」),菩提比丘長老英譯為\NoteKeywordBhikkhuBodhi{「榕樹」}(a banyan tree)。
\stopitemgroup

\startitemgroup[noteitems]
\item\subnoteref{x525}\NoteKeywordNikayaHead{「得到責備(說他人者)」}(labhati vattāraṃ),菩提比丘長老英譯為\NoteKeywordBhikkhuBodhi{「遇上某位這樣斥責他的人」}(meets someone who reproaches him thus)。按:《顯揚真義》以「得到指責(責備;呵責)」(Labhati vattāranti labhati codakaṃ)解說,今準此譯。
\stopitemgroup

\startitemgroup[noteitems]
\item\subnoteref{x526}\NoteKeywordNikayaHead{「是不淨的村落荊棘者」}(asucigāmakaṇṭakoti),菩提比丘長老英譯為\NoteKeywordBhikkhuBodhi{「是一位污穢的村落荊棘」}(is a foul village thorn)。按:《顯揚真義》說,「村落荊棘」指破壞、刺破的荊棘。註疏說,「村民的刺破」(Gāmavāsīnaṃ vijjhanaṭṭhenāti)指以接受他們無價值的行為(服務)壓迫[村民]。
\stopitemgroup

\startitemgroup[noteitems]
\item\subnoteref{x527}\NoteKeywordAgamaHead{「生其厭(\ccchref{SA.1171}{https://agama.buddhason.org/SA/dm.php?keyword=1171});不喜(\ccchref{AA.38.8}{https://agama.buddhason.org/AA/dm.php?keyword=38.8})」},南傳作\NoteKeywordNikaya{「排拒」}(byāpajjati,動詞,另譯為「反抗;惱害」),菩提比丘長老英譯為\NoteKeywordBhikkhuBodhi{「排斥」}(repelled),《光明寺経蔵》譯為「討厭」(嫌い)。按:《顯揚真義》以「因惡意而有腐敗(發臭)的心」(byāpādavasena pūticittaṃ hoti)解說。
\stopitemgroup

\startitemgroup[noteitems]
\item\subnoteref{x528}\NoteKeywordAgamaHead{「失收摩羅(\ccchref{SA.1171}{https://agama.buddhason.org/SA/dm.php?keyword=1171});鱣魚(\ccchref{AA.38.8}{https://agama.buddhason.org/AA/dm.php?keyword=38.8})」},南傳作\NoteKeywordNikaya{「鱷魚」}(Susumāraṃ, suṃsumāraṃ),菩提比丘長老英譯為\NoteKeywordBhikkhuBodhi{「鱷魚」}(crocodile)。按:「鱣」音義同「鱔」。
\stopitemgroup

\startitemgroup[noteitems]
\item\subnoteref{x529}\NoteKeywordNikayaHead{「順適」}(anurodhā,另譯為「順遂,滿悅;滿足」),菩提比丘長老英譯為\NoteKeywordBhikkhuBodhi{「吸引」}(attraction)。
\stopitemgroup

\startitemgroup[noteitems]
\item\subnoteref{x530}\NoteKeywordAgamaHead{「五受(\ccchref{SA.485}{https://agama.buddhason.org/SA/dm.php?keyword=485})」}:「樂根(身觸所生樂與喜悅的感受)、喜根(意觸所生樂與喜悅的感受)、苦根(身觸所生苦與不喜悅的感受)、憂根(意觸所生苦與不喜悅的感受)、捨根(身體的或心理的既非喜悅也非不喜悅的感受)」,亦即「身樂心喜,身苦心憂,身捨心捨」,參看\suttaref{SN.48.38}。
\stopitemgroup

\startitemgroup[noteitems]
\item\subnoteref{x531}\NoteKeywordAgamaHead{「百八受(\ccchref{SA.485}{https://agama.buddhason.org/SA/dm.php?keyword=485});一百零八受」},另參看\suttaref{SN.36.22}。
\stopitemgroup

\startitemgroup[noteitems]
\item\subnoteref{x532}\NoteKeywordNikayaHead{「欲樂」}(kāmasukhaṃ),菩提比丘長老英譯為\NoteKeywordBhikkhuBodhi{「感官快樂」}(sensual pleasure)。
\stopitemgroup

\startitemgroup[noteitems]
\item\subnoteref{x533}\NoteKeywordAgamaHead{「離欲樂,遠離樂,寂滅樂,菩提樂(\ccchref{SA.485}{https://agama.buddhason.org/SA/dm.php?keyword=485})」},南傳作\NoteKeywordNikaya{「不論什麽情況」}(yahiṃ yahiṃ),菩提比丘長老英譯為\NoteKeywordBhikkhuBodhi{「於任何方式」}(in whatever way)。按:《顯揚真義》等以「如來安立一切無苦的狀態就在樂中」(taṃ sabbaṃ tathāgato niddukkhabhāvaṃ sukhasmiṃyeva paññapetīti, \suttaref{SN.36.19}/\ccchref{MN.59}{https://agama.buddhason.org/MN/dm.php?keyword=59})解說。
\stopitemgroup

\startitemgroup[noteitems]
\item\subnoteref{x534}\NoteKeywordNikayaHead{「膽汁引起的」}(Pittasamuṭṭhānānipi),菩提比丘長老英譯為\NoteKeywordBhikkhuBodhi{「起源於膽汁失調」}(arise…originating from bile disorders)。「[三者]集合……也」(sannipātikānipi),菩提比丘長老英譯為\NoteKeywordBhikkhuBodhi{「源自[三者的]不平衡」}(originating from an imbalance [of the three]),並解說,古印度醫學中,「膽(汁)、痰、風」是三種身體疾病的要素(dosa)。
\stopitemgroup

\startitemgroup[noteitems]
\item\subnoteref{x535}\NoteKeywordNikayaHead{「不正注意生的」}(visamaparihārajānipi,另譯為「不等姿勢生的,險難襲來生的」),菩提比丘長老英譯為\NoteKeywordBhikkhuBodhi{「以不小心行為引起」}(produced by careless behaviour)。
\stopitemgroup

\startitemgroup[noteitems]
\item\subnoteref{x536}\NoteKeywordAgamaHead{「食念(\ccchref{SA.483}{https://agama.buddhason.org/SA/dm.php?keyword=483})」},南傳作\NoteKeywordNikaya{「肉體的喜」}(sāmisā pīti),菩提比丘長老英譯為\NoteKeywordBhikkhuBodhi{「肉體的狂喜」}(carnal rapture)。按:「食」即「肉體的」(sāmisa),另譯為「有味的;有食味的;物質的;塗滿食物的;有物質的」。
\stopitemgroup

\startitemgroup[noteitems]
\item\subnoteref{x537}\NoteKeywordAgamaHead{「無食念(\ccchref{SA.483}{https://agama.buddhason.org/SA/dm.php?keyword=483})」},南傳作\NoteKeywordNikaya{「精神的喜」}(nirāmisā pīti),菩提比丘長老英譯為\NoteKeywordBhikkhuBodhi{「精神的狂喜」}(spiritual rapture)。按:「無食」即「精神的」(nirāmisā),另譯為「無食味的;無染汚的;離財的;無肉的;無欲望的;無實質性的;離物質的;非物質的」。
\stopitemgroup

\startitemgroup[noteitems]
\item\subnoteref{x538}\NoteKeywordAgamaHead{「無食無食念(\ccchref{SA.483}{https://agama.buddhason.org/SA/dm.php?keyword=483})」},南傳作\NoteKeywordNikaya{「比精神更精神的喜」}(nirāmisā nirāmisatarā pīti),菩提比丘長老英譯為\NoteKeywordBhikkhuBodhi{「比精神更精神的狂喜」}(rapture more spiritual than the spiritual)。
\stopitemgroup

\startitemgroup[noteitems]
\item\subnoteref{x539}\NoteKeywordAgamaHead{「食捨(\ccchref{SA.483}{https://agama.buddhason.org/SA/dm.php?keyword=483})」},南傳作\NoteKeywordNikaya{「肉體的平靜」}(sāmisā upekkhā),菩提比丘長老英譯為\NoteKeywordBhikkhuBodhi{「肉體的平靜」}(carnal equanimity)。
\stopitemgroup

\startitemgroup[noteitems]
\item\subnoteref{x540}\NoteKeywordAgamaHead{「色俱行(\ccchref{SA.483}{https://agama.buddhason.org/SA/dm.php?keyword=483})」},南傳作\NoteKeywordNikaya{「色關聯的」}(Rūpappaṭisaṃyutto),菩提比丘長老英譯為\NoteKeywordBhikkhuBodhi{「與色領域連結」}(connected with the form sphere)。
\stopitemgroup

\startitemgroup[noteitems]
\item\subnoteref{x541}\NoteKeywordAgamaHead{「汝應常如此應(\ccchref{SA.569}{https://agama.buddhason.org/SA/dm.php?keyword=569})」},南傳作\NoteKeywordNikaya{「你就以這個方式應答」}(taññevettha paṭibhāseyyā),菩提比丘長老英譯為\NoteKeywordBhikkhuBodhi{「你應該澄清它」}(you should clear it up),並解說這是採用自由的譯法,原文為難譯的成語(I translate the awkward idiom freely in accordance with the natural sense)。
\stopitemgroup

\startitemgroup[noteitems]
\item\subnoteref{x542}\NoteKeywordAgamaHead{「在梵網中」(brahmajāle),指\ccchref{DN.1}{https://agama.buddhason.org/DN/dm.php?keyword=1}的「梵網經」,相當於\ccchref{DA.21}{https://agama.buddhason.org/DA/dm.php?keyword=21}的「梵動經」},可見這一部分的整理是比較晚的。
\stopitemgroup

\startitemgroup[noteitems]
\item\subnoteref{x543}\NoteKeywordNikayaHead{「像這樣就已離開」}(tathā pakkantova ahosi),菩提比丘長老英譯為\NoteKeywordBhikkhuBodhi{「他永久離開」}(he left for good),並解說他可能看到名利的危險,比較喜歡沒沒無名。
\stopitemgroup

\startitemgroup[noteitems]
\item\subnoteref{x544}\NoteKeywordAgamaHead{「戶鉤孔(\ccchref{SA.571}{https://agama.buddhason.org/SA/dm.php?keyword=571})」},南傳作\NoteKeywordNikaya{「鑰匙孔」}(tālacchiggaḷena),菩提比丘長老英譯為\NoteKeywordBhikkhuBodhi{「鎖孔」}(keyhole)。
\stopitemgroup

\startitemgroup[noteitems]
\item\subnoteref{x545}\NoteKeywordAgamaHead{「先已作方便心(\ccchref{SA.568}{https://agama.buddhason.org/SA/dm.php?keyword=568});本如是修習心(\ccchref{MA.210}{https://agama.buddhason.org/MA/dm.php?keyword=210})」},南傳作\NoteKeywordNikaya{「如之前他的心所修習的那樣」}(khvassa pubbeva tathā cittaṃ bhāvitaṃ),菩提比丘長老英譯為\NoteKeywordBhikkhuBodhi{「他的心之前就以這種方式開發」}(his mind has previously been developed in such a way)。按:《顯揚真義》等以「就在滅等至前的時限時間,我將有這麼多無心(無意識)時間,[這樣的]時限心被修習」(nirodhasamāpattito pubbe addhānaparicchedakāleyeva, ettakaṃ kālaṃ acittako bhavissāmīti addhānaparicchedacittaṃ bhāvitaṃ hoti, \suttaref{SN.41.6}/\ccchref{MN.44}{https://agama.buddhason.org/MN/dm.php?keyword=44})解說,亦即,入定者在進入想受滅等至前,就先在心中設定要進入多久,俟時間到就能出定。
\stopitemgroup

\startitemgroup[noteitems]
\item\subnoteref{x546}\NoteKeywordAgamaHead{「暖(\ccchref{SA.568}{https://agama.buddhason.org/SA/dm.php?keyword=568})」},南傳作\NoteKeywordNikaya{「熱」}(usmā,另譯為「煖」),菩提比丘長老英譯為\NoteKeywordBhikkhuBodhi{「肉體熱」}(physical heat)。
\stopitemgroup

\startitemgroup[noteitems]
\item\subnoteref{x547}\NoteKeywordAgamaHead{「觸不動(\ccchref{SA.568}{https://agama.buddhason.org/SA/dm.php?keyword=568});不移動觸(\ccchref{MA.211}{https://agama.buddhason.org/MA/dm.php?keyword=211})」},南傳作\NoteKeywordNikaya{「空觸」}(suññato phasso),菩提比丘長老英譯為\NoteKeywordBhikkhuBodhi{「空-接觸」}(emptiness-contact)。按:《顯揚真義》等說,應該以自己的性質(Saguṇena)與所緣(ārammaṇenāpi)解說,以自己的性質來說,果等至(phalasamāpatti-達到果位)被稱為空,所俱生的觸被稱為空觸;以所緣來說,涅槃因貪等空而被稱為空,貪相等不存在[而被稱為]無相,貪瞋癡願求不存在[而被稱為]無願求,空涅槃(suññataṃ nibbānaṃ)作為所緣後,所生起的果等至觸名為空[觸]。關於無相、無願,也是這個理趣(Animittappaṇihitesupi eseva nayo, \suttaref{SN.41.7}/\ccchref{MN.4}{https://agama.buddhason.org/MN/dm.php?keyword=4})。
\stopitemgroup

\startitemgroup[noteitems]
\item\subnoteref{x548}\NoteKeywordAgamaHead{「觸無所有(\ccchref{SA.568}{https://agama.buddhason.org/SA/dm.php?keyword=568});無所有觸(\ccchref{MA.211}{https://agama.buddhason.org/MA/dm.php?keyword=211})」},南傳作\NoteKeywordNikaya{「無願觸」}(appaṇihito phasso),菩提比丘長老英譯為\NoteKeywordBhikkhuBodhi{「不指向之接觸」}(undirected-contact, \suttaref{SN.41.6}),或「無欲之接觸」(desireless contact, \ccchref{MN.44}{https://agama.buddhason.org/MN/dm.php?keyword=44})。
\stopitemgroup

\startitemgroup[noteitems]
\item\subnoteref{x549}\NoteKeywordAgamaHead{「離(\ccchref{SA.568}{https://agama.buddhason.org/SA/dm.php?keyword=568})」},南傳作\NoteKeywordNikaya{「遠離」}(viveka),菩提比丘長老英譯為\NoteKeywordBhikkhuBodhi{「隔離;隱遁」}(seclusion)。按:《顯揚真義》等說,遠離即涅槃(nibbānaṃ viveko nāma, \suttaref{SN.41.6}/\ccchref{MN.44}{https://agama.buddhason.org/MN/dm.php?keyword=44})。
\stopitemgroup

\startitemgroup[noteitems]
\item\subnoteref{x550}\NoteKeywordAgamaHead{「貪有量(\ccchref{SA.567}{https://agama.buddhason.org/SA/dm.php?keyword=567})」},南傳作\NoteKeywordNikaya{「貪是衡量的作者」}(Rāgo…… pamāṇakaraṇo),智髻比丘長老英譯為「慾望是測量的製造者」(Lust is a maker of measurement)。按:《破斥猶豫》說,只要貪等不生起,則人(puggalaṃ)不能被認知,如看起來像須陀洹、斯陀含、阿那含的樣子,但每當貪等生起,貪染者、憤怒者、愚癡者(ratto duṭṭho mūḷhoti)被了知,像這樣,『這位是像這樣者。』這些人的(puggalassa)衡量如它們生起而使之看見,但菩提比丘長老認為,它們[貪瞋癡]對心靈的範圍和深度強加了限制(impose limitations, \ccchref{MN.43}{https://agama.buddhason.org/MN/dm.php?keyword=43})。
\stopitemgroup

\startitemgroup[noteitems]
\item\subnoteref{x551}\NoteKeywordAgamaHead{「貪者是所有(\ccchref{SA.567}{https://agama.buddhason.org/SA/dm.php?keyword=567})」},南傳作\NoteKeywordNikaya{「貪是件東西」}(Rāgo…… kiñcanaṃ/kiñcano),菩提比丘長老英譯為\NoteKeywordBhikkhuBodhi{「慾望是某物」}(Lust is a something)。按:《破斥猶豫》以「壓碎義即東西義」(maddanattho kiñcanatthoti, \ccchref{MN.43}{https://agama.buddhason.org/MN/dm.php?keyword=43})解說。
\stopitemgroup

\startitemgroup[noteitems]
\item\subnoteref{x552}\NoteKeywordAgamaHead{「貪者是有相(\ccchref{SA.567}{https://agama.buddhason.org/SA/dm.php?keyword=567})」},南傳作\NoteKeywordNikaya{「貪是相的作者」}(Rāgo…… nimittakaraṇo),菩提比丘長老英譯為\NoteKeywordBhikkhuBodhi{「慾望是形跡的製造者」}(Lust is a maker of signs)。按:《顯揚真義》等說,只要人的貪沒生起,就不能知道凡聖(ariyo vā puthujjano vāti),但當貪生起時,這個人就名為有貪者(sarāgo),當作了認出相(sañjānananimittaṃ)時,如他生出,因此被稱為「相的作者」(nimittakaraṇo’’ti vutto, \suttaref{SN.41.7}/\ccchref{MN.43}{https://agama.buddhason.org/MN/dm.php?keyword=43})。
\stopitemgroup

\startitemgroup[noteitems]
\item\subnoteref{x553}\NoteKeywordAgamaHead{「無覺、無觀三昧(\ccchref{SA.574}{https://agama.buddhason.org/SA/dm.php?keyword=574})」},南傳作\NoteKeywordNikaya{「無尋無伺定」}(avitakko avicāro samādhi),菩提比丘長老英譯為\NoteKeywordBhikkhuBodhi{「無心思與檢查之貫注集中」}(a concentration without thought and examination)。按:這是指第二禪。
\stopitemgroup

\startitemgroup[noteitems]
\item\subnoteref{x554}\NoteKeywordNikayaHead{「仰視」}(ulloketvā),菩提比丘長老英譯為\NoteKeywordBhikkhuBodhi{「驕傲地看著他自己的隨行人員」}(looked up proudly towards his own retinue)。按:《顯揚真義》說,他抬高身體,縮腹(kucchiṃ nīharitvā),脖子高舉觀看著所有方向,[然後]向上看(仰視)。
\stopitemgroup

\startitemgroup[noteitems]
\item\subnoteref{x555}\NoteKeywordAgamaHead{「一問一說一記論(\ccchref{SA.574}{https://agama.buddhason.org/SA/dm.php?keyword=574})」},南傳作\NoteKeywordNikaya{「一個問題、一個說示(概要)、一個解答」}(Eko pañho, eko uddeso, ekaṃ veyyākaraṇaṃ),菩提比丘長老英譯為\NoteKeywordBhikkhuBodhi{「一個問題、一個摘要、一個回答」}(One question, one synopsis, one answer)。按:註疏說,「問題」指以心審慮的問題,「說示」指義理的簡要語(atthassa saṃkhittavacanaṃ),「回答」指考察義理後的解釋談論。
\stopitemgroup

\startitemgroup[noteitems]
\item\subnoteref{x556}\NoteKeywordNikayaHead{「在家者也會有」}(Gihinopi siyā),菩提比丘長老依錫蘭本(kiṃhino siyā)英譯為「我怎麼不能」(How could I not)。
\stopitemgroup

\startitemgroup[noteitems]
\item\subnoteref{x557}\NoteKeywordNikayaHead{「以真真假假」}(saccālikena),菩提比丘長老英譯為\NoteKeywordBhikkhuBodhi{「以真實與謊話」}(by truth and lies)。
\stopitemgroup

\startitemgroup[noteitems]
\item\subnoteref{x558}\NoteKeywordNikayaHead{「名叫歡笑的地獄那裡」}(pahāso nāma nirayo tattha),菩提比丘長老英譯為\NoteKeywordBhikkhuBodhi{「在『笑的地獄』中」}(in the 'Hell of Laughter.')。按:《顯揚真義》說,沒有這種名稱的地獄,這是無間[地獄]的一部份(avīcisseva pana ekasmiṃ koṭṭhāse),他們被如唱歌跳舞的方式折磨。
\stopitemgroup

\startitemgroup[noteitems]
\item\subnoteref{x559}\NoteKeywordAgamaHead{「堪能、方便(\ccchref{SA.908}{https://agama.buddhason.org/SA/dm.php?keyword=908})」},南傳作\NoteKeywordNikaya{「竭力、努力」}(ussahati vāyamati,另譯為「能(敢)、勤(勵)」),菩提比丘長老英譯為\NoteKeywordBhikkhuBodhi{「奮鬥與努力」}(strives and exerts)。
\stopitemgroup

\startitemgroup[noteitems]
\item\subnoteref{x560}\NoteKeywordAgamaHead{「箭降伏天(\ccchref{SA.908}{https://agama.buddhason.org/SA/dm.php?keyword=908});箭莊嚴天(GA)」},南傳作\NoteKeywordNikaya{「被其他人征服天們的共住狀態」}(parajitānaṃ devānaṃ sahabyataṃ, sarañjitānaṃ devānaṃ sahavyataṃ, parijitānaṃ devānaṃ sahavyataṃ),菩提比丘長老英譯為\NoteKeywordBhikkhuBodhi{「與戰鬥被殺天在一起」}(the company of the the battle-slain devas),並解說「戰鬥被殺」(sarañjitānaṃ,「被箭征服」),是依「樂林阿難長老」(vanarata Ānanda Thera)建議的「自由翻譯」(free rendering)。
\stopitemgroup

\startitemgroup[noteitems]
\item\subnoteref{x561}\NoteKeywordNikayaHead{「他的那個心在之前已被惡作、惡意向捉住」}(tassa taṃ cittaṃ pubbe gahitaṃ dukkaṭaṃ duppaṇihitaṃ),菩提比丘長老英譯為\NoteKeywordBhikkhuBodhi{「他的心早已是低下的、墮落的,被心思誤導」}(his mind is already low, depraved, misdirected by the thought)。按:《顯揚真義》以「被邪惡地放置」(duṭṭhu ṭhapitaṃ)解說「惡意向」。
\stopitemgroup

\startitemgroup[noteitems]
\item\subnoteref{x562}\NoteKeywordNikayaHead{「名叫被其他人征服的地獄那裡」}(parajito nāma nirayo tattha),菩提比丘長老英譯為\NoteKeywordBhikkhuBodhi{「在『戰鬥被殺的地獄』中」}(in the 'Battle-Slain Hell.')。按:《顯揚真義》說,沒有這種名稱的地獄,這是無間[地獄]的一部份(avīcisseva pana ekasmiṃ koṭṭhāse),他們被如配備武器、手盾、駕御象馬車在戰場上戰鬥的方式折磨。
\stopitemgroup

\startitemgroup[noteitems]
\item\subnoteref{x563}\NoteKeywordNikayaHead{「戰鬥中竭力……」}以下同\suttaref{SN.42.3}。
\stopitemgroup

\startitemgroup[noteitems]
\item\subnoteref{x564}\NoteKeywordNikayaHead{「確實使已死的死者離開」}(mataṃ kālaṅkataṃ uyyāpenti nāma),菩提比丘長老英譯為\NoteKeywordBhikkhuBodhi{「被說成引導死人向上」}(are said to direct a dead person upwards)。按:《顯揚真義》以「使向上行去(存續;生存)」(pari yāpenti)解說「離開」,亦即下一句經文說的「進入天界(梵天世界)」。
\stopitemgroup

\startitemgroup[noteitems]
\item\subnoteref{x565}\NoteKeywordNikayaHead{「確實使知」}(saññāpenti nāma),菩提比丘長老英譯為\NoteKeywordBhikkhuBodhi{「引導他方向」}(to guide him along)。按:《顯揚真義》以「使正確地知道」(sammā ñāpenti)解說。
\stopitemgroup

\startitemgroup[noteitems]
\item\subnoteref{x566}\NoteKeywordAgamaHead{「蒺蔾論(\ccchref{SA.914}{https://agama.buddhason.org/SA/dm.php?keyword=914});二種論(GA)」},南傳作\NoteKeywordNikaya{「兩難」}(ubhatokoṭikaṃ,直譯為「兩方極端」),菩提比丘長老英譯為\NoteKeywordBhikkhuBodhi{「兩刀論法」}(dilemma)。
\stopitemgroup

\startitemgroup[noteitems]
\item\subnoteref{x567}\NoteKeywordNikayaHead{「沙門性生成的」}(sāmaññasambhūtāni),菩提比丘長老依錫蘭本(saññamasambhūtāni)英譯為「從自我控制」(from selfcontrol)。按:《顯揚真義》以「殘餘戒(德性)」(Sāmaññaṃ nāma sesasīlaṃ)解說。\ccchref{DN.33}{https://agama.buddhason.org/DN/dm.php?keyword=33}/\ccchref{AN.8.36}{https://agama.buddhason.org/AN/an.php?keyword=8.36}說,三種福德行為基礎:布施所成的福德行為基礎、戒所成的福德行為基礎、修習所成的福德行為基礎,與之相應。
\stopitemgroup

\startitemgroup[noteitems]
\item\subnoteref{x568}\NoteKeywordAgamaHead{「抵債不還(\ccchref{SA.914}{https://agama.buddhason.org/SA/dm.php?keyword=914});不解生業(GA)」},南傳作\NoteKeywordNikaya{「或劣企畫的諸工作失敗」}(duppayuttā vā kammantā vipajjanti, Duppayuttā vā kammantaṃ jahanti),菩提比丘長老英譯為\NoteKeywordBhikkhuBodhi{「管理不善的事業失敗」}(or mismanaged undertakings fail)。
\stopitemgroup

\startitemgroup[noteitems]
\item\subnoteref{x569}\NoteKeywordAgamaHead{「摩尼珠髻(人名, \ccchref{SA.911}{https://agama.buddhason.org/SA/dm.php?keyword=911});如意珠頂髮(GA)」},南傳作\NoteKeywordNikaya{「摩尼朱羅葛」}(maṇicūḷako,逐字義譯為「寶珠+髻」),菩提比丘長老照錄不譯。
\stopitemgroup

\startitemgroup[noteitems]
\item\subnoteref{x570}\NoteKeywordAgamaHead{「須作人索作人(\ccchref{SA.911}{https://agama.buddhason.org/SA/dm.php?keyword=911})」},南傳作\NoteKeywordNikaya{「男子(工人)能被有需要男子者遍求」}(puriso purisatthikena pariyesitabbo),菩提比丘長老英譯為\NoteKeywordBhikkhuBodhi{「工人可以被需要工人者尋求」}(a workman may be sought by one needing a workman)。
\stopitemgroup

\startitemgroup[noteitems]
\item\subnoteref{x571}\NoteKeywordAgamaHead{「力士人間(\ccchref{SA.913}{https://agama.buddhason.org/SA/dm.php?keyword=913});末牢(GA)」},南傳作\NoteKeywordNikaya{「在末羅」}(mallesu),「末羅」(malla),另有「力士;摔跤手」的意思。
\stopitemgroup

\startitemgroup[noteitems]
\item\subnoteref{x572}\NoteKeywordAgamaHead{「依父母(\ccchref{SA.913}{https://agama.buddhason.org/SA/dm.php?keyword=913});汝子未生,未依於母(GA)」},南傳作\NoteKeywordNikaya{「我有個男孩」}(Atthi me……kumāro,直譯為「有我的童子」,意即「我有個兒子」),菩提比丘長老英譯為\NoteKeywordBhikkhuBodhi{「我有個男孩」}(I have a boy)。按:「依父母」,應指需依靠父母親照顧的子女。
\stopitemgroup

\startitemgroup[noteitems]
\item\subnoteref{x573}\NoteKeywordNikayaHead{「受用諸欲者」}(kāmabhogino),菩提比丘長老英譯為\NoteKeywordBhikkhuBodhi{「享樂感官快樂者」}(these three persons who enjoy sensual pleasures existing),並解說這裡的三種指:(i) 財富如何被獲得,非法、合法或兩者兼有。(ii)是否為自己的利益使用。(iii)是否利益他人。按:這三種情況,以下經文共列有十種組合,另參看\ccchref{MA.126}{https://agama.buddhason.org/MA/dm.php?keyword=126}。
\stopitemgroup

\startitemgroup[noteitems]
\item\subnoteref{x574}\NoteKeywordAgamaHead{「伽彌尼(\ccchref{MA.20}{https://agama.buddhason.org/MA/dm.php?keyword=20})」},南傳作\NoteKeywordNikaya{「村長」}(gāmaṇi,古譯為「聚落主」),菩提比丘長老英譯為\NoteKeywordBhikkhuBodhi{「首領」}(headman)。按:「伽彌尼」應為「村長」(gāmaṇi)的音譯。
\stopitemgroup

\startitemgroup[noteitems]
\item\subnoteref{x575}\NoteKeywordAgamaHead{「幻(\ccchref{MA.20}{https://agama.buddhason.org/MA/dm.php?keyword=20})」},南傳作\NoteKeywordNikaya{「幻術」}(māyaṃ),菩提比丘長老英譯為\NoteKeywordBhikkhuBodhi{「魔法」}(magic)。「是幻;幻人」,南傳作\NoteKeywordNikaya{「幻術者」}(māyāvī),菩提比丘長老英譯為\NoteKeywordBhikkhuBodhi{「使魔法的人;魔法師」}(magician)。有關世尊知道幻術的訛傳,另參看\ccchref{MA.133}{https://agama.buddhason.org/MA/dm.php?keyword=133}。
\stopitemgroup

\startitemgroup[noteitems]
\item\subnoteref{x576}\NoteKeywordAgamaHead{「卒(\ccchref{MA.20}{https://agama.buddhason.org/MA/dm.php?keyword=20})」},南傳作\NoteKeywordNikaya{「髮髻下垂雇員」}(lambacūḷake bhaṭe),菩提比丘長老英譯為\NoteKeywordBhikkhuBodhi{「有著垂下頭飾的傭工」}(hirelings with drooping head-dresses),並引Rhys Davids在Buddhist India中的解說:「拘利國中央當局有特殊身體的傭工或警察(a special body of peon, or police),以一種特殊的頭飾,就像一種制服,他們得到他們的名稱,這些特殊的人因勒索和暴力聲名狼藉。」
\stopitemgroup

\startitemgroup[noteitems]
\item\subnoteref{x577}\NoteKeywordAgamaHead{「法之定/法定(\ccchref{MA.20}{https://agama.buddhason.org/MA/dm.php?keyword=20})」},南傳作\NoteKeywordNikaya{「法之定」}(dhammasamādhi,另譯為「法三昧」),菩提比丘長老英譯為\NoteKeywordBhikkhuBodhi{「法之集中」}(concentration of the Dhamma)。按:《顯揚真義》以「十善業道法」(dasakusalakammapathadhammā)解說,又說,欣悅、喜、寧靜、樂、定五法名為「法之定」。
\stopitemgroup

\startitemgroup[noteitems]
\item\subnoteref{x578}\NoteKeywordNikayaHead{「從色的名稱解脫的如來」}(Rūpasaṅkhāyavimutto kho…… tathāgato),菩提比丘長老英譯為\NoteKeywordBhikkhuBodhi{「如來被從物質色角度的認定中釋放」}(The Tathagata is liberated from reckoning in terms of material form)。按:「名稱」(saṅkhāya),\ccchref{MN.72}{https://agama.buddhason.org/MN/dm.php?keyword=72}以及暹羅本作「滅盡」(Saṅkhayā),《顯揚真義》以「未來色的不生,『以色、非色的部分而將名為這樣的色』(rūpārūpakoṭṭhāsenapi evarūpo nāma bhavissatīti):以名稱止息的狀態而已從色的安立解脫」解說。
\stopitemgroup

\startitemgroup[noteitems]
\item\subnoteref{x579}\NoteKeywordAgamaHead{「我只告知苦,連同苦的滅」,這段經文同樣出現在\ccchref{MN.22}{https://agama.buddhason.org/MN/dm.php?keyword=22},而與\ccchref{SA.301}{https://agama.buddhason.org/SA/dm.php?keyword=301}的「苦生而生,苦滅而滅」}(\ccchref{SA.262}{https://agama.buddhason.org/SA/dm.php?keyword=262}作「此苦生時生,滅時滅」)的「無我」含意相同。
\stopitemgroup

\startitemgroup[noteitems]
\item\subnoteref{x580}\NoteKeywordNikayaHead{「這是色之類的」}(rūpagatametaṃ),菩提比丘長老英譯為\NoteKeywordBhikkhuBodhi{「這是一種色的涉入」}(this is an involvement with form)。《顯揚真義》以「這僅是色(rūpamattametaṃ),在這裡,離開色沒有任何其他眾生可得(rūpato añño koci satto nāma na upalabbhati),但,當有色時,這僅是名字」解說,註疏說,這裡「僅」之語說區別之意(visesanivattiattho),什麼是那個區別呢?凡被外面定見的(bāhiraparikappito),這裡被稱為「如來」的真我(attā)。
\stopitemgroup

\startitemgroup[noteitems]
\item\subnoteref{x581}\NoteKeywordAgamaHead{「有餘(\ccchref{SA.957}{https://agama.buddhason.org/SA/dm.php?keyword=957});有取(GA)」},南傳作\NoteKeywordNikaya{「對有取著者」}(sa-upādānassa),菩提比丘長老英譯為\NoteKeywordBhikkhuBodhi{「對具有燃料者」}(for one with fuel)。按:「取著」(upādāna,另譯為「取;執著」)的另一個意思是「燃料」,在經文中顯然是雙關語。
\stopitemgroup

\startitemgroup[noteitems]
\item\subnoteref{x582}\NoteKeywordNikayaHead{「六處篇第四」}之「攝頌」,為35-44相應之名稱。按:《相應部》共分為五篇(vaggo,另譯為「品」),第一為「有偈篇」,第二為「因緣篇」,第三為「蘊篇;犍度篇」,第四為「六處篇」第五為「大篇」。
\stopitemgroup

\startitemgroup[noteitems]
\item\subnoteref{x583}\NoteKeywordAgamaHead{「婆羅門乘(\ccchref{SA.769}{https://agama.buddhason.org/SA/dm.php?keyword=769})」},南傳作\NoteKeywordNikaya{「梵乘」}(brahmayānaṃ),菩提比丘長老英譯為\NoteKeywordBhikkhuBodhi{「神的交通工具;神車」}(the divine vehicle),並解說這裡的「梵」(brahma)是指「最好的」(seṭṭha)的意思。
\stopitemgroup

\startitemgroup[noteitems]
\item\subnoteref{x584}\NoteKeywordAgamaHead{「正法律乘(\ccchref{SA.769}{https://agama.buddhason.org/SA/dm.php?keyword=769})」},南傳作\NoteKeywordNikaya{「法乘」}(dhammayānaṃ),菩提比丘長老英譯為\NoteKeywordBhikkhuBodhi{「正法的交通工具;正法之車」}(the vehicle of Dhamma)。
\stopitemgroup

\startitemgroup[noteitems]
\item\subnoteref{x585}\NoteKeywordAgamaHead{「法軛(\ccchref{SA.769}{https://agama.buddhason.org/SA/dm.php?keyword=769})」},南傳作\NoteKeywordNikaya{「成對的法」}(dhammā yuttā,逐字譯為「法軛」,另譯為「相應的法;結合的法」),菩提比丘長老英譯為\NoteKeywordBhikkhuBodhi{「其…被均勻地結合在一起的素養」}(Its qualities of…yoked evenly together)。
\stopitemgroup

\startitemgroup[noteitems]
\item\subnoteref{x586}\NoteKeywordAgamaHead{「長縻(\ccchref{SA.769}{https://agama.buddhason.org/SA/dm.php?keyword=769})」},南傳作\NoteKeywordNikaya{「繫繩」}(yottaṃ),菩提比丘長老英譯為\NoteKeywordBhikkhuBodhi{「軛-綁繩」}(yoke-tie)。
\stopitemgroup

\startitemgroup[noteitems]
\item\subnoteref{x587}\NoteKeywordAgamaHead{「捨三昧為轅(\ccchref{SA.769}{https://agama.buddhason.org/SA/dm.php?keyword=769})」},南傳作\NoteKeywordNikaya{「平靜為軛定」}(Upekkhā dhurasamādhi),菩提比丘長老英譯為\NoteKeywordBhikkhuBodhi{「平靜保持負擔平衡」}(Equanimity keeps the burden balanced)。按:「平靜」(Upekkhā),古譯為「捨」,「軛定」(dhurasamādhi),《顯揚真義》以「軛的定」(dhurassa samādhi)解說,「定」(samādhi),也譯為「三昧(音譯);等持」,這裡以「等持」,引申為「平衡」為適當。
\stopitemgroup

\startitemgroup[noteitems]
\item\subnoteref{x588}\NoteKeywordNikayaHead{「皮甲冑」}(cammasannāho),菩提比丘長老依錫蘭本(vammasannaho)英譯為「其盔甲和盾」(its armour and shield)。
\stopitemgroup

\startitemgroup[noteitems]
\item\subnoteref{x589}\NoteKeywordNikayaHead{「必定一一成為勝利者」}(aññadatthu jayaṃ jayan”ti),菩提比丘長老英譯為\NoteKeywordBhikkhuBodhi{「必然贏得勝利」}(Inevitably winning the victory)。
\stopitemgroup

\startitemgroup[noteitems]
\item\subnoteref{x590}\NoteKeywordNikayaHead{「精進」}(āyāmaṃ, vāyāmaṃ),菩提比丘長老英譯為\NoteKeywordBhikkhuBodhi{「努力」}(effort)。按:《顯揚真義》以「活力」(vīriyaṃ)解說,暹羅與錫蘭本均作「精進」(vāyāmaṃ)。又說,當意欲未被平息等,當三者未被平息時有八種與貪俱行相應心(aṭṭhalobhasahagatacittasampayuttā)的感受,意欲的平息為緣是初禪感受的,尋的平息為緣是第二禪,想為緣是六種等至的感受(chasamāpattivedanā, \suttaref{SN.45.12}),三者的平息為緣是非想非非想處的感受,未到達者為了到達是阿羅漢果的到達目的,當到達該處是他因活力努力(vīriyārambhassa)而為阿羅漢果的到達因素(kāraṇe anuppatte),有以其為緣的感受是阿羅漢狀態為緣(arahattassa ṭhānapaccayā)的感受,以這四道俱生(Etena catumaggasahajātā )而生起出世間的受。當意欲已被平息等就被稱為目標(vuttatthāneva)。
\stopitemgroup

\startitemgroup[noteitems]
\item\subnoteref{x591}\NoteKeywordNikayaHead{「也有以意欲的平息為緣感受的」}(chandavūpasamapaccayāpi vedayitaṃ),菩提比丘長老英譯為\NoteKeywordBhikkhuBodhi{「想要的消去為條件感受的」}(also feeling with the subsiding of desire as condition)。按:《顯揚真義》以「初禪的感受」(paṭhamajjhānavedanā)解說,尋的平息為緣是第二禪的感受,想為緣是六種等至的感受(chasamāpattivedanā),想的平息為緣是非想非非想處的感受。
\stopitemgroup

\startitemgroup[noteitems]
\item\subnoteref{x592}\NoteKeywordNikayaHead{「精進」}(āyāmaṃ, vāyāmaṃ)等,參看前經。
\stopitemgroup

\startitemgroup[noteitems]
\item\subnoteref{x593}\NoteKeywordNikayaHead{「聖道」}(ariyo aṭṭhaṅgiko maggo),原文有「八支」(aṭṭhaṅgiko),但依\suttaref{SN.46.18}、\suttaref{SN.47.33}、\suttaref{SN.51.2}(ariyo maggo)刪。
\stopitemgroup

\startitemgroup[noteitems]
\item\subnoteref{x594}\NoteKeywordAgamaHead{「希有諸人民,能度於彼岸(\ccchref{SA.771}{https://agama.buddhason.org/SA/dm.php?keyword=771})」},\ccchref{AN.10.117}{https://agama.buddhason.org/AN/an.php?keyword=10.117}、\suttaref{SN.45.34}作「在那些人中是少的:到彼岸的人」(Appakā te manussesu, ye janā pāragāmino)。
\stopitemgroup

\startitemgroup[noteitems]
\item\subnoteref{x595}\NoteKeywordNikayaHead{「渴望」}(tasinā),菩提比丘長老英譯為\NoteKeywordBhikkhuBodhi{「渴望」}(thirst),並解說taṇhā與tasinā皆等同梵文tṛṣṇā,錫蘭僧伽羅語本沒有這一經,而羅馬拼音本則沒有獨立經號。按:水野弘元《巴利語辭典》將此字寫作tasiṇā,並視為等同「渴愛」(taṇhā)。又,依本譯所依緬甸版後續相同組織的〈46相應/12.尋求品〉、〈47相應/9.尋求品〉、〈48相應/9.尋求品〉(略在「8.恒河中略品」下)、〈49相應/4.尋求品〉、〈50相應/9.尋求品〉、〈51相應/7.尋求品〉(略在「4.恒河中略品」下)、〈53相應/4.尋求品〉(略在「1.恒河中略品」下)等品都仍維持10經的編經數來看,推斷〈尋求品〉原來應該只有10經,\suttaref{SN.45.171}為後加入的,起先應該只附在\suttaref{SN.45.170}後,未編獨立的經號。
\stopitemgroup

\startitemgroup[noteitems]
\item\subnoteref{x596}\NoteKeywordNikayaHead{「[只有]這是真實之執持的身繫縛」}(idaṃsaccābhiniveso kāyagantho),菩提比丘長老英譯為\NoteKeywordBhikkhuBodhi{「對真實教理主張黏著的身體結」}(the bodily knot of adherence to dogmatic assertion of truth)。按:《顯揚真義》以「名身的繫縛,編織組合的雜染」(nāmakāyassa gantho ganthanaghaṭanakilesa)解說「身繫縛」。
\stopitemgroup

\startitemgroup[noteitems]
\item\subnoteref{x597}\NoteKeywordNikayaHead{「許多被作的第七品」},「攝頌」從這裡開始與實際內容不符,留下實際內容擴編的痕跡。
\stopitemgroup

\startitemgroup[noteitems]
\item\subnoteref{x598}\NoteKeywordAgamaHead{「眾多種種異道……(\ccchref{SA.281}{https://agama.buddhason.org/SA/dm.php?keyword=281})」},南傳作\NoteKeywordNikaya{「來往各團體者」}(parisāvacaro,逐字譯為「眾(集會處)+行境」),菩提比丘長老英譯為\NoteKeywordBhikkhuBodhi{「常與集合(團體)交際」}(frequents assemblies)。按:《顯揚真義》說:「他們訪問所謂愚癡的、賢智的團體,壓碎異論後能使自己的理論輝耀,這名為parisāvacaro」(Parisaṃ nāma bālāpi, paṇḍitāpi osaranti, yo pana parappavādaṃ madditvā attano vādaṃ dīpetuṃ sakkoti, ayaṃ parisāvacaro nāma)。另外,「遮羅迦」可能是「步行者」(caraka)的音譯。
\stopitemgroup

\startitemgroup[noteitems]
\item\subnoteref{x599}\NoteKeywordNikayaHead{「對欲貪處之法」}(Kāmarāgaṭṭhāniyānaṃ……dhammānaṃ),菩提比丘長老英譯為\NoteKeywordBhikkhuBodhi{「以感官的慾望為基礎之事」}(things that are a basis for sensual lust)。
\stopitemgroup

\startitemgroup[noteitems]
\item\subnoteref{x600}\NoteKeywordNikayaHead{「立起與彎下」}(ukkujjāvakujjaṃ),菩提比丘長老英譯為\NoteKeywordBhikkhuBodhi{「激增與下降」}(the surge and decline)。按:《顯揚真義》以「這裡,生起被稱為立起,彎下為消散,轉以生起衰滅(udayabbayavasena)把握」解說。
\stopitemgroup

\startitemgroup[noteitems]
\item\subnoteref{x601}\NoteKeywordNikayaHead{「到那樣的狀態」}(tathattāya,另譯為「如性;真如;涅槃的狀態」),菩提比丘長老英譯為\NoteKeywordBhikkhuBodhi{「到那樣的狀態」}(to such a state)。
\stopitemgroup

\startitemgroup[noteitems]
\item\subnoteref{x602}\NoteKeywordNikayaHead{「善法、善分、善黨」}(dhammā kusalā kusalabhāgiyā kusalapakkhikā),菩提比丘長老英譯為\NoteKeywordBhikkhuBodhi{「有益的狀態,參與有益的,關於有益的」}(states there are that are wholesome, partaking of the wholesome, pertaining to the wholesome)。
\stopitemgroup

\startitemgroup[noteitems]
\item\subnoteref{x603}\NoteKeywordNikayaHead{「善法、善分、善黨」}(有益的狀態,參與有益的,關於有益的),參看前經。
\stopitemgroup

\startitemgroup[noteitems]
\item\subnoteref{x604}\NoteKeywordAgamaHead{「阿濕波他樹(\ccchref{SA.708}{https://agama.buddhason.org/SA/dm.php?keyword=708})」},南傳作\NoteKeywordNikaya{「菩提樹」}(Assattho),菩提比丘長老英譯照錄原文。
\stopitemgroup

\startitemgroup[noteitems]
\item\subnoteref{x605}\NoteKeywordAgamaHead{「尼拘留他樹(\ccchref{SA.708}{https://agama.buddhason.org/SA/dm.php?keyword=708})」},南傳作\NoteKeywordNikaya{「榕樹」}(nigrodho,另音譯為「尼拘律樹」),菩提比丘長老英譯為\NoteKeywordBhikkhuBodhi{「榕樹」}(the banyan)。
\stopitemgroup

\startitemgroup[noteitems]
\item\subnoteref{x606}\NoteKeywordNikayaHead{「糙葉榕(依大馬比丘《巴利語辭典》)」}(pilakkho),菩提比丘長老英譯沒譯。
\stopitemgroup

\startitemgroup[noteitems]
\item\subnoteref{x607}\NoteKeywordAgamaHead{「優曇鉢羅樹(\ccchref{SA.708}{https://agama.buddhason.org/SA/dm.php?keyword=708})」},南傳作\NoteKeywordNikaya{「叢生榕(依大馬比丘《巴利語辭典》)」}(udumbaro),菩提比丘長老英譯照錄原文。
\stopitemgroup

\startitemgroup[noteitems]
\item\subnoteref{x608}\NoteKeywordAgamaHead{「揵遮耶樹(\ccchref{SA.708}{https://agama.buddhason.org/SA/dm.php?keyword=708})」},南傳作\NoteKeywordNikaya{「無花果樹」}(kacchako),菩提比丘長老英譯照錄原文。
\stopitemgroup

\startitemgroup[noteitems]
\item\subnoteref{x609}\NoteKeywordNikayaHead{「迦捭多羅樹」},南傳作\NoteKeywordNikaya{「山蘋果樹」}(kapitthano,依水野弘元《パ─リ語辞典》「山林檎」),菩提比丘長老英譯照錄原文。
\stopitemgroup

\startitemgroup[noteitems]
\item\subnoteref{x610}\NoteKeywordNikayaHead{「障、蓋」}(Āvaraṇā nīvaraṇā),在「攝頌」中,這是分開成二則的,但經文是合在一起的。錫蘭本分開為二則,但「攝頌」中的「二則如理」(即\suttaref{SN.46.35}、\suttaref{SN.46.36})是合成一則的。
\stopitemgroup

\startitemgroup[noteitems]
\item\subnoteref{x611}\NoteKeywordAgamaHead{「四念處(\ccchref{SA.743}{https://agama.buddhason.org/SA/dm.php?keyword=743})」}後接「心與慈俱」四無量之內容,標題與內容不相應。菩提比丘長老解說,本經將四梵住(四無量心)與七覺支連結是不常見的(unusual),就其動力(momentum)而言,四梵住導向往生梵天,而非涅槃,但當整合入佛陀的道(Buddha's path)結構中時,它們能產生足夠的定力(《顯揚真義》說三種禪,three jhānas)成為毘婆舍那的基礎而轉向正覺。
\stopitemgroup

\startitemgroup[noteitems]
\item\subnoteref{x612}\NoteKeywordAgamaHead{「於淨最勝(\ccchref{SA.743}{https://agama.buddhason.org/SA/dm.php?keyword=743})」},南傳作\NoteKeywordNikaya{「清淨是最高的」}(Subhaparamāhaṃ),菩提比丘長老英譯為\NoteKeywordBhikkhuBodhi{「以美的東西為其頂點」}(has the beautiful as its culmination),並解說,這是對以慈心修色界禪而無法證入解脫者來說的。尼柯耶中並無明確地說明[四]梵住相當於第幾禪,但有許多地方說到[四]梵住是往生梵天或色界的方法,因此《顯揚真義》被迫辛苦地解說本經令人費解的各個梵住(meditation subject)到達無色界之「上限」(upper limit)。摘要來說(i)住於慈心者容易將心運用在「淨色遍處」(a beautiful colour kasiṇa),迅速達到該清淨色遍處的「清淨解脫」。(ii)住於悲心者理解到色的過患,因而脫離色,修習空無邊處。(iii)住於利他喜悅者「理解眾生的喜悅識」(apprehends the joyful consciousness of beings),因而容易進入識無邊處。(iv)住於平靜(捨)者熟練於轉移苦與樂,因而容易轉移到缺乏任何具體的實體的無所有處。
\stopitemgroup

\startitemgroup[noteitems]
\item\subnoteref{x613}全品應如「相應部46相應/9.恒河中略品」(\suttaref{SN.45.91}-102)那樣使之被細說。
\stopitemgroup

\startitemgroup[noteitems]
\item\subnoteref{x614}\NoteKeywordNikayaHead{「成為專一的」}(ekodibhūtā),菩提比丘長老英譯為\NoteKeywordBhikkhuBodhi{「統一的」}(unified)。按:《顯揚真義》說,以剎那定(khaṇikasamādhinā)成為專一的。
\stopitemgroup

\startitemgroup[noteitems]
\item\subnoteref{x615}\NoteKeywordNikayaHead{「心明淨的」}(vippasannacittā),菩提比丘長老英譯為\NoteKeywordBhikkhuBodhi{「具備澄清的心」}(with limpid mind)。
\stopitemgroup

\startitemgroup[noteitems]
\item\subnoteref{x616}\NoteKeywordNikayaHead{「不善巧」}(abyatto),菩提比丘長老英譯為\NoteKeywordBhikkhuBodhi{「不熟練的」}(unskilful)。
\stopitemgroup

\startitemgroup[noteitems]
\item\subnoteref{x617}\NoteKeywordNikayaHead{「也以鹼的」}(khārikehipi),菩提比丘長老英譯為\NoteKeywordBhikkhuBodhi{「尖銳的」}(sharp)。按:這似乎相當於北傳經文的「酢(醋)」,後來的漢譯經典,有些似乎將此譯為「澀」。
\stopitemgroup

\startitemgroup[noteitems]
\item\subnoteref{x618}\NoteKeywordNikayaHead{「吃這種」}(imassa vā abhiharati,直譯為「持來這個」),菩提比丘長老英譯為\NoteKeywordBhikkhuBodhi{「他伸手拿這個」}(he reached for this one)。按:《顯揚真義》以「伸出手拿的意思」(gahaṇatthāya hatthaṃ pasāreti)解說abhiharati,長老回函說,該字在PTS英巴辭典(Margaret Cone, PTS, year 2000)列有「吃」(partakes of)的意思,或許譯為「吃」比「伸手拿」要好(Perhaps "partakes of" is better than "reaches for.")。
\stopitemgroup

\startitemgroup[noteitems]
\item\subnoteref{x619}\NoteKeywordAgamaHead{「當取自心相(\ccchref{SA.616}{https://agama.buddhason.org/SA/dm.php?keyword=616})」},南傳作\NoteKeywordNikaya{「掌握自己心的相」}(sakassa cittassa nimittaṃ uggaṇhātīti),菩提比丘長老英譯為\NoteKeywordBhikkhuBodhi{「拾取他自己心的形跡」}(picks up the sign of his own mind)。「相」,《顯揚真義》以「我的業處」(me kammaṭṭhānaṃ)解說。
\stopitemgroup

\startitemgroup[noteitems]
\item\subnoteref{x620}\NoteKeywordNikayaHead{「師傅留一手」}(ācariyamuṭṭhi,逐字譯為「師(阿闍梨)+拳;師的握拳」),菩提比丘長老英譯為\NoteKeywordBhikkhuBodhi{「老師的封拳」}(closed fist of a teacher, SN),Maurice Walshe先生英譯為「老師的拳頭」(teacher's fist, DN)。按:「師的握拳」指:老師教導時有所保留或弟子提問師未回答。
\stopitemgroup

\startitemgroup[noteitems]
\item\subnoteref{x621}\NoteKeywordAgamaHead{「方便修治(\ccchref{DA.2}{https://agama.buddhason.org/DA/dm.php?keyword=2})」},南傳作\NoteKeywordNikaya{「以包纏物複合的」}(veḷamissakena, vedhamissakena, veṭhamissakena, vekhamissakena, veghamissakena),菩提比丘長老依veṭhamissakena英譯為「以一個皮繩組合」(by a combination of straps, SN),Maurice Walshe先生依veghamissakena英譯為「以被皮繩綑住/以被皮繩維持在一起」(by being strapped up/by being held together with straps, DN)。按:《顯揚真義》、《吉祥悅意》都以「柄的繫縛、車輪的繫縛等修復」(bāhabandhacakkabandhādinā paṭisaṅkharaṇena)解說veṭhamissakena,今依veṭhamissakena譯。
\stopitemgroup

\startitemgroup[noteitems]
\item\subnoteref{x622}\NoteKeywordNikayaHead{「前後無昏昧(昧略)的」}(pacchāpure ‘asaṃkhittaṃ),菩提比丘長老英譯為\NoteKeywordBhikkhuBodhi{「它前後不被壓縮」}(It is unconstricted after and before),並依\suttaref{SN.51.32}修四神足「後如前那樣地,前如後那樣地」那樣,解說「前後」指「自始至終」不岔開。
\stopitemgroup

\startitemgroup[noteitems]
\item\subnoteref{x623}\NoteKeywordAgamaHead{「若起、若作(\ccchref{SA.638}{https://agama.buddhason.org/SA/dm.php?keyword=638});起法、作法(\ccchref{SA.639}{https://agama.buddhason.org/SA/dm.php?keyword=639})」},南傳作\NoteKeywordNikaya{「存在的」}(bhūtaṃ,另譯為「已變成;已出生」),菩提比丘長老英譯為\NoteKeywordBhikkhuBodhi{「生成」}(come to be)。
\stopitemgroup

\startitemgroup[noteitems]
\item\subnoteref{x624}\NoteKeywordNikayaHead{「四眾的」}(catunnañca parisānaṃ),菩提比丘長老英譯為\NoteKeywordBhikkhuBodhi{「四種集合」}(the four assemblies),並解說,這是指「比丘、比丘尼、優婆塞、優婆夷」。
\stopitemgroup

\startitemgroup[noteitems]
\item\subnoteref{x625}另一則不同的紀錄,參看\suttaref{SN.35.89}。
\stopitemgroup

\startitemgroup[noteitems]
\item\subnoteref{x626}\NoteKeywordNikayaHead{「正確方式」}(ñāyo,另譯為「理趣;正理;真理;方法」),菩提比丘長老英譯為\NoteKeywordBhikkhuBodhi{「方法」}(the method)。按:《顯揚真義》以「有根據的方法;有理由的方法」(so upāyo, taṃ kāraṇanti)解說。
\stopitemgroup

\startitemgroup[noteitems]
\item\subnoteref{x627}\NoteKeywordNikayaHead{「現起」}(upaṭṭhahanti,另譯為「出現;起反應」),菩提比丘長老英譯為\NoteKeywordBhikkhuBodhi{「他們保持呈現」}(they remain present)。
\stopitemgroup

\startitemgroup[noteitems]
\item\subnoteref{x628}即\suttaref{SN.45.172}-\suttaref{SN.45.181}。
\stopitemgroup

\startitemgroup[noteitems]
\item\subnoteref{x629}\NoteKeywordNikayaHead{「根的不同有果的不同」},參看\suttaref{SN.48.16}。
\stopitemgroup

\startitemgroup[noteitems]
\item\subnoteref{x630}\NoteKeywordAgamaHead{「根波羅蜜因緣知果波羅蜜(\ccchref{SA.653}{https://agama.buddhason.org/SA/dm.php?keyword=653})」},南傳作\NoteKeywordNikaya{「根的不同有果的不同」}(indriyavemattatā phalavemattatā hoti),菩提比丘長老英譯為\NoteKeywordBhikkhuBodhi{「由於在機能上不同而在結果上有不同」}(due to a difference in the faculties there is a difference in the fruits)。按:《顯揚真義》說,以種種根而有種種果;以種種果而有種種人(indriyanānattena phalanānattaṃ, phalanānattena puggalanānattanti)。
\stopitemgroup

\startitemgroup[noteitems]
\item\subnoteref{x631}\NoteKeywordNikayaHead{「三根」}(tīṇi indriyānī),菩提比丘長老英譯為\NoteKeywordBhikkhuBodhi{「三個機能」}(three faculties)。
\stopitemgroup

\startitemgroup[noteitems]
\item\subnoteref{x632}\NoteKeywordAgamaHead{「未知當知根(\ccchref{SA.642}{https://agama.buddhason.org/SA/dm.php?keyword=642});未知欲知根(\ccchref{DA.9}{https://agama.buddhason.org/DA/dm.php?keyword=9})」},南傳作\NoteKeywordNikaya{「『我將知未知的』根」}(Anaññātaññassāmītindriyaṃ),菩提比丘長老英譯為\NoteKeywordBhikkhuBodhi{「『我將知道過去未曾知道者』的機能」}(The faculty of 'I shall know the as-yet-unknown,')。按:《顯揚真義》以「[心想:]『我將了知在無始輪迴中,以前不知道的法。』行者在須陀洹道剎那(sotāpattimaggakkhaṇe)生起的根」解說,《吉祥悅意》以「這是須陀洹道智的同義語(Sotāpattimaggañāṇassetaṃ adhivacanaṃ)」解說。
\stopitemgroup

\startitemgroup[noteitems]
\item\subnoteref{x633}\NoteKeywordAgamaHead{「知根(\ccchref{SA.642}{https://agama.buddhason.org/SA/dm.php?keyword=642}/\ccchref{DN.9}{https://agama.buddhason.org/DN/dm.php?keyword=9})」},南傳作\NoteKeywordNikaya{「完全智根」}(aññindriyaṃ),菩提比丘長老英譯為\NoteKeywordBhikkhuBodhi{「最終理解的機能」}(the faculty of final knowledge)。按:《顯揚真義》以「就對那些已知法(ñātadhammānaṃ)以行相的了知(ājānanākārena),在入流果等之六種情況(chasu ṭhānesu)生起的根」解說,《吉祥悅意》以「成為完全智、成為了知之根(aññābhūtaṃ ājānanabhūtaṃ indriyaṃ)……這是智的同義語(ñāṇassetaṃ adhivacanaṃ)」解說。
\stopitemgroup

\startitemgroup[noteitems]
\item\subnoteref{x634}\NoteKeywordAgamaHead{「無知根(\ccchref{SA.642}{https://agama.buddhason.org/SA/dm.php?keyword=642});知已根(\ccchref{DA.9}{https://agama.buddhason.org/DA/dm.php?keyword=9})」},南傳作\NoteKeywordNikaya{「具知根」}(aññātāvindriyaṃ),菩提比丘長老英譯為\NoteKeywordBhikkhuBodhi{「具有最終理解者的機能」}(the faculty of one endowed with final knowledge)。按:《顯揚真義》以「關於已了知者(aññātāvīsu)的阿羅漢果法生起的根」解說,《吉祥悅意》以「關於已了知者識知應被作的最後目的達到情況(jānanakiccapariyosānappattesu dhammesu)的根,這是阿羅漢果智的同義語」解說。
\stopitemgroup

\startitemgroup[noteitems]
\item\subnoteref{x635}\NoteKeywordNikayaHead{「有行的」}(sasaṅkhāraṃ),菩提比丘長老英譯為\NoteKeywordBhikkhuBodhi{「有原因的形成」}(a causal formation),並解說,「有相的、有因的、有行的、有緣的」均為同義詞。
\stopitemgroup

\startitemgroup[noteitems]
\item\subnoteref{x636}\NoteKeywordNikayaHead{「為了那個目的」}(tadatthāya),菩提比丘長老依錫蘭本(tathattāya)英譯為「因此」(accordingly)。
\stopitemgroup

\startitemgroup[noteitems]
\item\subnoteref{x637}\NoteKeywordNikayaHead{「在末羅」}(mallesu),其他版本作「在末羅迦」(mallakesu),或「在末莉」(Mallikesu,水野弘元音譯為「末利迦」),後者與本經經名相符,菩提比丘長老英譯本該經之經名與內容均採用後者。
\stopitemgroup

\startitemgroup[noteitems]
\item\subnoteref{x638}\NoteKeywordAgamaHead{「精進定(\ccchref{SA.561}{https://agama.buddhason.org/SA/dm.php?keyword=561})」},南傳作\NoteKeywordNikaya{「活力定」}(vīriyasamādhi),菩提比丘長老英譯為\NoteKeywordBhikkhuBodhi{「基於活力的貫注集中」}(concentration due to energy)。
\stopitemgroup

\startitemgroup[noteitems]
\item\subnoteref{x639}\NoteKeywordNikayaHead{「心定」}(cittasamādhi),菩提比丘長老英譯為\NoteKeywordBhikkhuBodhi{「基於心的貫注集中」}(concentration due to mind)。
\stopitemgroup

\startitemgroup[noteitems]
\item\subnoteref{x640}\NoteKeywordAgamaHead{「方便(\ccchref{SA.561}{https://agama.buddhason.org/SA/dm.php?keyword=561})」},南傳作\NoteKeywordNikaya{「心」}(cittaṃ),菩提比丘長老英譯為\NoteKeywordBhikkhuBodhi{「決心」}(make up your mind, resolution)。
\stopitemgroup

\startitemgroup[noteitems]
\item\subnoteref{x641}\NoteKeywordAgamaHead{「籌量(\ccchref{SA.561}{https://agama.buddhason.org/SA/dm.php?keyword=561})」},南傳作\NoteKeywordNikaya{「考察」}(vīmaṃsā),菩提比丘長老英譯為\NoteKeywordBhikkhuBodhi{「作研究調查」}(make an investigation)。
\stopitemgroup

\startitemgroup[noteitems]
\item\subnoteref{x642}\NoteKeywordAgamaHead{「豈非無邊際(\ccchref{SA.561}{https://agama.buddhason.org/SA/dm.php?keyword=561})」},南傳作\NoteKeywordNikaya{「是有邊的,不是無邊的」}(santakaṃ hoti no asantakaṃ),菩提比丘長老依僧伽羅語手抄本(anantakaṃ hoti no santakaṃ)英譯為「情況是無限的,非可終止的」(the situation is interminable, not terminable),今依此譯。
\stopitemgroup

\startitemgroup[noteitems]
\item\subnoteref{x643}\NoteKeywordNikayaHead{「善決意」}(svādhiṭṭhitā,另譯為「善受持;善攝持;善確立」),菩提比丘長老英譯為\NoteKeywordBhikkhuBodhi{「很好地決心在;很好地決定在;很好地融入在」}(well resolved upon)。按:《顯揚真義》以同義詞「善決意(Suṭṭhu adhiṭṭhitā)、善保持(suṭṭhu ṭhapitā)」解說。
\stopitemgroup

\startitemgroup[noteitems]
\item\subnoteref{x644}\NoteKeywordNikayaHead{「能夠」}(omāti),菩提比丘長老英譯為\NoteKeywordBhikkhuBodhi{「能夠」}(is able)。按:《顯揚真義》以「能夠(pahoti)、能夠(sakkoti)」解說,並說,在三藏中,這是佛陀[專用]語(buddhavacane),不混合[使用]之語法(asambhinnapadaṃ)。
\stopitemgroup

\startitemgroup[noteitems]
\item\subnoteref{x645}\NoteKeywordNikayaHead{「收存」}(samodahati,另譯為「放一起;收妥」),菩提比丘長老英譯為\NoteKeywordBhikkhuBodhi{「沉浸」}(immerses)。
\stopitemgroup

\startitemgroup[noteitems]
\item\subnoteref{x646}\NoteKeywordNikayaHead{「以種種行相與差別」}(anekākāravokāraṃ),菩提比丘長老英譯為\NoteKeywordBhikkhuBodhi{「在它的許多相態與要素上」}(in its many aspects and factors)。
\stopitemgroup

\startitemgroup[noteitems]
\item\subnoteref{x647}\NoteKeywordNikayaHead{「捨斷五蓋後」}(pañca nīvaraṇe pahāya),菩提比丘長老英譯為\NoteKeywordBhikkhuBodhi{「捨棄了五個障礙」}(having abandoned the five hindrances),並解說,所有的有學已完全捨棄疑蓋;不還者加上根除惡意與後悔(以及其更多感官自制上的欲的意欲),有學只在禪定與毘婆舍那中暫時捨棄其它蓋,這裡的連續體pahāya(捨斷後)應該根據這些限制去理解。
\stopitemgroup

\startitemgroup[noteitems]
\item\subnoteref{x648}\NoteKeywordNikayaHead{「梵行立足處」}(brahmacariyogadhaṃ),菩提比丘長老英譯為\NoteKeywordBhikkhuBodhi{「立足於聖潔的生活」}(Grounded upon the holy life, \suttaref{SN.55.2}),或「精神生活中的頂點」(culmination of the spiritual life, \ccchref{AN.3.61}{https://agama.buddhason.org/AN/an.php?keyword=3.61})。按:《滿足希求》說,這是指「阿羅漢道(arahattamaggasaṅkhātassa)、涅槃(nibbānaṃ)」。
\stopitemgroup

\startitemgroup[noteitems]
\item\subnoteref{x649}\NoteKeywordAgamaHead{「自通之法(\ccchref{SA.1044}{https://agama.buddhason.org/SA/dm.php?keyword=1044})」},南傳作\NoteKeywordNikaya{「關係自己之法的法門」}(attupanāyiko dhammapariyāyo),菩提比丘長老英譯為\NoteKeywordBhikkhuBodhi{「能應用到自己之法的解說」}(a Dhamma exposition applicable to oneself)。按:《顯揚真義》以「應該被導引到自己」(attani upanetabbaṃ)解說。
\stopitemgroup

\startitemgroup[noteitems]
\item\subnoteref{x650}詳見\suttaref{SN.40.10},該經先說歸依三寶,後半段即同本經說四不壞淨。
\stopitemgroup

\startitemgroup[noteitems]
\item\subnoteref{x651}\NoteKeywordAgamaHead{「增上智慧審諦堪忍(\ccchref{SA.936}{https://agama.buddhason.org/SA/dm.php?keyword=936})」},南傳作\NoteKeywordNikaya{「那個如來宣說的諸法以慧足夠沉思地接受」}(Tathāgatappaveditā cassa dhammā paññāya mattaso nijjhānaṃ khamanti),菩提比丘長老英譯為\NoteKeywordBhikkhuBodhi{「在以智慧沉思到足夠程度後,他接受被如來宣告的教義」}(the teachings proclaimed by the Tathagata are accepted by him after being pondered to a sufficient degree with wisdom)。按:《顯揚真義》說,以衡量檢視接受(pamāṇena ca olokanaṃ khamanti),這是指在隨法行道的個人(dhammānusārimaggaṭṭhapuggalaṃ),即隨法行者。
\stopitemgroup

\startitemgroup[noteitems]
\item\subnoteref{x652}\NoteKeywordNikayaHead{「有來生死亡的害怕」}(hoti samparāyikaṃ maraṇabhayaṃ),菩提比丘長老英譯為\NoteKeywordBhikkhuBodhi{「有即將死亡的害怕」}(there is fear of imminent death)。按:《顯揚真義》以「有起因於來生死亡的害怕」(samparāyahetukaṃ maraṇabhayaṃ)解說,這是預期性的害怕。
\stopitemgroup

\startitemgroup[noteitems]
\item\subnoteref{x653}\NoteKeywordAgamaHead{「力;自在(\ccchref{SA.833}{https://agama.buddhason.org/SA/dm.php?keyword=833})」},南傳作\NoteKeywordNikaya{「有統治權」}(ādhipateyyena saṃyutto hoti,直譯為「加入統治權的」),菩提比丘長老英譯為\NoteKeywordBhikkhuBodhi{「成為具有(被賦予)統治權」}(becomes endowed with sovereignty)。
\stopitemgroup

\startitemgroup[noteitems]
\item\subnoteref{x654}\NoteKeywordAgamaHead{「比丘百歲壽命解脫涅槃(\ccchref{SA.1122}{https://agama.buddhason.org/SA/dm.php?keyword=1122})」},南傳作「與心從諸\twnr{漏}{188.0}解脫的比丘」(āsavā vimuttacittena bhikkhunā),菩提比丘長老依錫蘭本(vassasatavimuttacittena bhikkhunā)英譯為「一位心自由了(已解脫)一百年的比丘」(a bhikkhu who has been liberated in mind for a hundred years)。按:錫蘭本與北傳經文相同。又,本經是南北傳阿含經中唯一說到優婆塞證阿羅漢果(解脫涅槃)的經(其他都只說證阿那含果),但本經說的是頻死的情況,意味著在家人證阿羅漢果後入滅。另一個南北傳律典都記載在家人證阿羅漢果的例子,那是耶舍,但他證阿羅漢果後就立刻出家了。
\stopitemgroup

\startitemgroup[noteitems]
\item\subnoteref{x655}\NoteKeywordNikayaHead{「不放逸慧的狀態」}(Appamattapaññatā),菩提比丘長老依錫蘭本(Asāmantapaññatā)英譯為「無同等慧的狀態」(the state of unequalled wisdom)。按:\ccchref{AN.1.584}{https://agama.buddhason.org/AN/an.php?keyword=1.584}作「無與倫比慧的狀態」(asāmantapaññatāya)。
\stopitemgroup

\startitemgroup[noteitems]
\item\subnoteref{x656}\NoteKeywordNikayaHead{「三轉」}(tiparivaṭṭa),菩提比丘長老英譯為\NoteKeywordBhikkhuBodhi{「三階段」}(three phases),即對四聖諦第一轉的「知」,第二轉的「應遍知/應斷/應修/應證」以及第三轉的「已遍知/已斷/已修/已證」。後來「慧遠」(523~592AD)在其《大乘起信論義疏》中,分別以「示轉」、「勸轉」、「證轉」稱之,也很貼切。
\stopitemgroup

\startitemgroup[noteitems]
\item\subnoteref{x657}\NoteKeywordNikayaHead{「十二行;十二行相」}(dvādasākāra),菩提比丘長老英譯為\NoteKeywordBhikkhuBodhi{「十二種情況」}(twelve aspects),這就是四聖諦在每一轉時的內容。
\stopitemgroup

\startitemgroup[noteitems]
\item\subnoteref{x658}\NoteKeywordNikayaHead{「阿若拘鄰」},南傳作\NoteKeywordNikaya{「阿若憍陳如」}(aññāsikoṇḍañña),「阿若」(aññā)為音譯,義譯為「已知」,「拘鄰」(koṇḍañña)為「憍陳如」的另譯,菩提比丘長老英譯為\NoteKeywordBhikkhuBodhi{「已了解的憍陳如」}(koṇḍañña Who Has Understood)。
\stopitemgroup

\startitemgroup[noteitems]
\item\subnoteref{x659}\NoteKeywordNikayaHead{「閻浮洲中」}(jambudīpe),為「閻浮」(jambu,音譯為「贍部」)與「洲」(dīpa,義譯,另譯為「燈」)的組合,也有譯為「閻浮提、贍部洲」的,這是古印度人地理觀念中這個世界的四大洲之一,有說這是泛指人類居住的地方,有說是泛指恒河中、上游地區。
\stopitemgroup

\startitemgroup[noteitems]
\item\subnoteref{x660}\NoteKeywordAgamaHead{「前相(\ccchref{SA.394}{https://agama.buddhason.org/SA/dm.php?keyword=394})」},南傳作\NoteKeywordNikaya{「這是先導,這是前相」}( etaṃ pubbaṅgamaṃ etaṃ pubbanimittaṃ),菩提比丘長老英譯為\NoteKeywordBhikkhuBodhi{「這是前兆與前導」}(this is the forerunner and precursor)。
\stopitemgroup

\startitemgroup[noteitems]
\item\subnoteref{x661}\NoteKeywordAgamaHead{「明相(\ccchref{SA.394}{https://agama.buddhason.org/SA/dm.php?keyword=394})」},南傳作\NoteKeywordNikaya{「黎明」}(aruṇuggaṃ),菩提比丘長老英譯為\NoteKeywordBhikkhuBodhi{「黎明」}(the dawn)。以日出為正見的譬喻,另參看\ccchref{SA.748}{https://agama.buddhason.org/SA/dm.php?keyword=748}。
\stopitemgroup

\startitemgroup[noteitems]
\item\subnoteref{x662}\NoteKeywordAgamaHead{「常觀他面(\ccchref{SA.398}{https://agama.buddhason.org/SA/dm.php?keyword=398})」},南傳作\NoteKeywordNikaya{「仰視……的臉」}(mukhaṃ ullokenti),菩提比丘長老英譯為\NoteKeywordBhikkhuBodhi{「仰視……的臉」}(look up at the face of)。按:「仰視」(ullokenti),也有「期待」、「尋找」的意思。
\stopitemgroup

\startitemgroup[noteitems]
\item\subnoteref{x663}\NoteKeywordAgamaHead{「至諸論處(\ccchref{SA.399}{https://agama.buddhason.org/SA/dm.php?keyword=399})」},南傳作\NoteKeywordNikaya{「希求辯論、尋求辯論」}(vādatthiko vādagavesī),菩提比丘長老英譯為\NoteKeywordBhikkhuBodhi{「尋找辯論,搜尋辯論」}(seeking an argument, searching for an argument)。
\stopitemgroup

\startitemgroup[noteitems]
\item\subnoteref{x664}\NoteKeywordAgamaHead{「思惟世間思惟(\ccchref{SA.407}{https://agama.buddhason.org/SA/dm.php?keyword=407});思議世界(\ccchref{AA.29.6}{https://agama.buddhason.org/AA/dm.php?keyword=29.6})」},南傳作\NoteKeywordNikaya{「我將思惟世間的思惟」}(lokacintaṃ cintessāmī’ti),菩提比丘長老英譯為\NoteKeywordBhikkhuBodhi{「思考關於世界[之事]」}(reflecting about the world)。按:《顯揚真義》舉例:如日月被誰所作?大地?大海?眾生被誰所生?山?芒果、棕櫚、椰子?
\stopitemgroup

\startitemgroup[noteitems]
\item\subnoteref{x665}\NoteKeywordAgamaHead{「於生本諸行樂著(\ccchref{SA.421}{https://agama.buddhason.org/SA/dm.php?keyword=421})」},南傳作\NoteKeywordNikaya{「在導致(轉起)出生的諸行上尋歡」}(jātisaṃvattanikesu saṅkhāresu abhiramanti),菩提比丘長老英譯為\NoteKeywordBhikkhuBodhi{「他們在導致出生的意志形成上歡樂」}(they delight in volitional formations that lead to birth)。
\stopitemgroup

\startitemgroup[noteitems]
\item\subnoteref{x666}\NoteKeywordAgamaHead{「作是行(\ccchref{SA.421}{https://agama.buddhason.org/SA/dm.php?keyword=421})」},南傳作\NoteKeywordNikaya{「造作導致出生的諸行」}(jātisaṃvattanikepi saṅkhāre abhisaṅkharonti),菩提比丘長老英譯為\NoteKeywordBhikkhuBodhi{「他們產生導致出生的意志形成」}(they generate volitional formations that lead to birth)。「造作」(abhisaṅkharonti),另譯為「現行;為作」。
\stopitemgroup

\startitemgroup[noteitems]
\item\subnoteref{x667}\NoteKeywordAgamaHead{「箭箭(\ccchref{SA.405}{https://agama.buddhason.org/SA/dm.php?keyword=405})」},南傳作\NoteKeywordNikaya{「一箭接著一箭」}(poṅkhānupoṅkhaṃ),菩提比丘長老英譯為\NoteKeywordBhikkhuBodhi{「頭穿過尾」}(head through butt)。按:水野弘元《巴利語辭典》解說poṅkhānupoṅkha為「接連不斷,接二連三地」(矢つぎばやに, 引きもきらず),但《顯揚真義》說,一支箭被射出後(khipitvā),隨後[射出]名為接著之箭的第二支箭,其箭矢貫穿[前]箭,再隨後那支箭[亦然]。
\stopitemgroup

\startitemgroup[noteitems]
\item\subnoteref{x668}\NoteKeywordAgamaHead{「破一毛為百分,而射一毛分(\ccchref{SA.405}{https://agama.buddhason.org/SA/dm.php?keyword=405})」},南傳作\NoteKeywordNikaya{「能經分裂成七分的毛之一端貫通另一端」}(sattadhā bhinnassa vālassa koṭiyā koṭiṃ paṭivijjheyyā),菩提比丘長老英譯為\NoteKeywordBhikkhuBodhi{「以箭頭貫穿分割成七股的毛之端」}(to pierce with the arrowhead the tip of a hair split into seven strands)。按:「貫通」(paṭivijjhati),也作「洞察」。《顯揚真義》說,將毛分裂成七分後,取一分繫在茄子中間,另一分繫在箭頭上,站在140肘遠(usabhamatte,長老作200呎),將這兩分毛貫通。七分(sattadhā),錫蘭本作百分(satadhā)。
\stopitemgroup

\startitemgroup[noteitems]
\item\subnoteref{x669}\NoteKeywordNikayaHead{「照耀」}(dibbati, dippati),菩提比丘長老英譯為\NoteKeywordBhikkhuBodhi{「照耀」}(shines, SN/AN)。按:dibbati(博戱;賭博;自娛)與上下文無關,PTS版作dippati(照耀),今依此譯。
\stopitemgroup

\startitemgroup[noteitems]
\item\subnoteref{x670}\NoteKeywordAgamaHead{「中國(\ccchref{SA.442}{https://agama.buddhason.org/SA/dm.php?keyword=442})」},南傳作\NoteKeywordNikaya{「於中國地方」}(majjhimesu janapadesu),菩提比丘長老英譯為\NoteKeywordBhikkhuBodhi{「在中央國家中」}(in the middle countries)。按:此應指佛陀遊化區域。
\stopitemgroup


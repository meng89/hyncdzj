\pian{有偈篇}{1}{11}
\xiangying{1}{諸天相應}
\pin{蘆葦品}{1}{10}
\sutta{1}{1}{暴流之渡過經}{https://agama.buddhason.org/SN/sn.php?keyword=1.1}
  \twnr{被我這麼聽聞}{1.0}:

  \twnr{有一次}{2.0},\twnr{世尊}{12.0}住在舍衛城祇樹林給孤獨園。那時,在夜已深時,容色絕佳的某位天神使整個祇樹林發光後,去見世尊。抵達後,向世尊\twnr{問訊}{46.0}後,在一旁站立。在一旁站立的那位天神對世尊說這個:

  「\twnr{親愛的先生}{204.0}!你怎樣渡過暴流呢?」

  「\twnr{朋友}{201.0}!我無住立、\twnr{無用力地}{x0}渡過暴流。」

  「親愛的先生!那麼,如怎樣你無住立、無用力地渡過暴流呢?」

  「朋友!當我住立時,那時,我沈沒;朋友!當我用力時,那時,我被飄走,朋友!這樣,我無住立、無用力地渡過暴流。」

  「終於我確實看見,般涅槃的婆羅門:

   無住立、無用力地,已度脫世間中的執著。」

  那位天神說這個,\twnr{大師}{145.0}是認可者。那時,那位天神[想]:「大師是我的認可者。」向世尊問訊、\twnr{作右繞}{47.0}後,就在那裡消失。



\sutta{2}{2}{解脫經}{https://agama.buddhason.org/SN/sn.php?keyword=1.2}
  起源於舍衛城。

  那時,在夜已深時,容色絕佳的某位天神使整個祇樹林發光後,去見世尊。抵達後,向世尊\twnr{問訊}{46.0}後,在一旁站立。在一旁站立的那位天神對世尊說這個:

  「\twnr{親愛的先生}{204.0}!你知道眾生的解脫、已解脫、\twnr{遠離}{x1}嗎?」

  「\twnr{朋友}{201.0}!我知道眾生的解脫、已解脫、遠離。」

  「親愛的先生!那麼,如怎樣你知道眾生的解脫、已解脫、遠離?」

  「\twnr{以有之歡喜的遍盡}{x2}、以想與識的滅盡、以受的滅與寂靜,朋友!這樣,我知道眾生的解脫、已解脫、遠離。」



\sutta{3}{3}{被帶走經}{https://agama.buddhason.org/SN/sn.php?keyword=1.3}
  起源於舍衛城。

  在一旁站立的那位天神在\twnr{世尊}{12.0}面前說這\twnr{偈頌}{281.0}:

  「生命被帶走、壽命是少的,被帶到老年者的救護所不存在,

   觀看著這死亡的恐怖,\twnr{應該作帶來樂的福德}{893.1}。」

  「生命被帶走、壽命是少的,被帶到老年者的救護所不存在,

   觀看著這在死亡上的恐怖,\twnr{期待寂靜者}{893.2}應該捨棄\twnr{世間物質}{593.0}。」[\suttaref{SN.2.19}]



\sutta{4}{4}{流逝經}{https://agama.buddhason.org/SN/sn.php?keyword=1.4}
  起源於舍衛城。

  在一旁站立的那位天神在\twnr{世尊}{12.0}面前說這\twnr{偈頌}{281.0}:

  「時間流逝夜匆匆,\twnr{種種年齡}{x3}次第地拋棄[人],

   觀看著這死亡的恐怖,\twnr{應該作帶來樂的福德}{893.1}。」

  「時間流逝夜匆匆,種種年齡次第地拋棄,

   觀看著這死亡的恐怖,期待寂靜者應該捨棄\twnr{世間的誘惑物}{593.0}。」[\suttaref{SN.2.27}]



\sutta{5}{5}{切斷多少經}{https://agama.buddhason.org/SN/sn.php?keyword=1.5}
  起源於舍衛城。

  在一旁站立的那位天子在\twnr{世尊}{12.0}面前說這\twnr{偈頌}{281.0}:

  「應該切斷多少應該捨斷多少,且更應該\twnr{修習}{94.0}多少,

   \twnr{比丘}{31.0}超越多少染著,被稱為『已渡暴流者』?」

  「應該切斷五\twnr{應該捨斷五}{x4},\twnr{且更應該修習五}{x5},

   比丘\twnr{超越五染著}{x6},被稱為『已渡\twnr{暴流}{369.0}者』。」



\sutta{6}{6}{清醒經}{https://agama.buddhason.org/SN/sn.php?keyword=1.6}
  起源於舍衛城。

  在一旁站立的那位天神在\twnr{世尊}{12.0}面前說這\twnr{偈頌}{281.0}:

  「當清醒時多少已睡?在已睡時多少清醒?

   以多少他抓取了塵垢?以多少他被清淨?」

  「當清醒時\twnr{五}{x7}已睡,在已睡時五清醒,

   以五事他抓取了塵垢,以五事他被清淨。」



\sutta{7}{7}{未確知者經}{https://agama.buddhason.org/SN/sn.php?keyword=1.7}
  起源於舍衛城。

  在一旁站立的那位天神在\twnr{世尊}{12.0}面前說這\twnr{偈頌}{281.0}:

  「凡法未確知者,會被引導到異教,

   已熟睡他們未覺醒,是他們覺醒的時候。」

  「凡法已善確知者,不會被引導到異教,

   那些\twnr{正覺者}{185.1}\twnr{以究竟智}{191.0},\twnr{在不平順中平順地行}{882.0}。」



\sutta{8}{8}{極忘失者經}{https://agama.buddhason.org/SN/sn.php?keyword=1.8}
  起源於舍衛城。

  在一旁站立的那位天神在\twnr{世尊}{12.0}面前說這\twnr{偈頌}{281.0}:

  「凡法極忘失者,會被引導到異教,

   已熟睡他們未覺醒,是他們覺醒的時候。」

  「凡法不忘失者,不會被引導到異教,

   那些\twnr{正覺者}{185.1}\twnr{以究竟智}{191.0},\twnr{在不平順中平順地行}{882.0}。」



\sutta{9}{9}{慢之愛欲者經}{https://agama.buddhason.org/SN/sn.php?keyword=1.9}
  起源於舍衛城。

  在一旁站立的那位天神在\twnr{世尊}{12.0}面前說這\twnr{偈頌}{281.0}:

  「這裡慢之愛欲者沒有調御,不得定者沒有\twnr{牟那}{125.0},

   單獨放逸地住在\twnr{林野}{142.0},不能渡死亡領域的\twnr{彼岸}{226.0}。」

  「捨斷慢後為善得定狀態,善心者在一切處被釋放,

   單獨不放逸地住在林野,他能渡死亡領域的彼岸。」[\suttaref{SN.1.38}]



\sutta{10}{10}{林野經}{https://agama.buddhason.org/SN/sn.php?keyword=1.10}
  起源於舍衛城。

  在一旁站立的那位天神以\twnr{偈頌}{281.0}對\twnr{世尊}{12.0}說:

  「在\twnr{林野}{142.0}中居住者們的,寂靜的梵行者們的,

   食一餐者們的,容色為何明淨?」

  「他們不悲傷過去,不希求未來,

   他們以當面的維生,因此容色明淨。

   以希求未來的,以悲傷過去的,

   以這個愚者乾枯,如被割斷的綠蘆葦。」

  蘆葦品第一,其\twnr{攝頌}{35.0}:

  「暴流、解脫、被帶走,流逝、切斷多少,

   清醒、未確知者,極忘失者、慢之愛欲者,

   林野被說為第十,以那個被稱為品。」





\pin{歡喜園品}{11}{20}
\sutta{11}{11}{歡喜園經}{https://agama.buddhason.org/SN/sn.php?keyword=1.11}
  \twnr{被我這麼聽聞}{1.0}:

  \twnr{有一次}{2.0},\twnr{世尊}{12.0}住在舍衛城祇樹林給孤獨園。

  在那裡,世尊召喚\twnr{比丘}{31.0}們:「比丘們!」

  「\twnr{尊師}{480.0}!」那些比丘回答世尊。

  世尊說這個:

  「比丘們!從前,某位\twnr{三十三天眾}{280.0}的天神在\twnr{歡喜園}{704.0}中被天女眾圍繞,當賦有、擁有天的\twnr{五種欲}{187.0}自娛時,那時候,他說了這\twnr{偈頌}{281.0}:

  『他們不了知樂:凡沒見過\twnr{歡喜園}{704.0}者,

   是天人們的住所:三十三天的有名聲者。』

  比丘們!在這麼說時,某位天神以偈頌回應那位天神:

  『愚者!你不了知,關於\twnr{阿羅漢}{5.0}們的言語:

   一切行[確實]是無常的,是\twnr{生起與消散法的}{681.0},

   生起後被滅,它們的寂滅是樂。』[\suttaref{SN.6.15}, \suttaref{SN.9.6}, \suttaref{SN.15.20}, \ccchref{DN.16}{https://agama.buddhason.org/DN/dm.php?keyword=16}, \ccchref{DN.17}{https://agama.buddhason.org/DN/dm.php?keyword=17}]」



\sutta{12}{12}{歡喜經}{https://agama.buddhason.org/SN/sn.php?keyword=1.12}
  起源於舍衛城。

  在一旁站立的那位\twnr{天神}{x8}在\twnr{世尊}{12.0}面前說這\twnr{偈頌}{281.0}:

  「有孩子的以孩子他歡喜,就像那樣有牛的以牛他歡喜,

   \twnr{依著}{198.0}確實是人們的歡喜,凡無依著者他確實不歡喜。」

  「有孩子的以孩子他憂愁,就像那樣有牛的以牛他憂愁,

   依著確實是人們的憂愁,凡無依著者他確實不憂愁。」[\suttaref{SN.4.8}]



\sutta{13}{13}{沒有等同兒子者經}{https://agama.buddhason.org/SN/sn.php?keyword=1.13}
  起源於舍衛城。

  在一旁站立的那位天神在\twnr{世尊}{12.0}面前說這\twnr{偈頌}{281.0}:

  「沒有等同對兒子的愛,沒有等同牛之財,

   沒有等同太陽的光明,大海是池湖中最勝的。」

  「沒有等同對自我的愛,沒有等同穀物之財,

   沒有等同慧的光明,雨是池湖中最勝的。」



\sutta{14}{14}{剎帝利經}{https://agama.buddhason.org/SN/sn.php?keyword=1.14}
  「剎帝利是兩足中最上的,四足中則是公牛,

   處女是妻子中最上的,而凡兒子們中是生在前面的。」

  「正覺者是兩足中最上的,四足中則是駿馬,

   順從的是妻子中最上的,而凡兒子們中是孝順的。」



\sutta{15}{15}{出聲經}{https://agama.buddhason.org/SN/sn.php?keyword=1.15}
  「中午時分已住立時,在鳥兒們已安靜時,

   廣大的\twnr{林野}{142.0}就出聲,那個恐怖在我心中出現。」

  「中午時分已住立時,在鳥兒們已安靜時,

   廣大的林野就出聲,那個喜樂在我心中出現。」[\suttaref{SN.9.12}]



\sutta{16}{16}{睡眠與懶惰經}{https://agama.buddhason.org/SN/sn.php?keyword=1.16}
  「睡眠、懶惰、打哈欠,不喜樂、\twnr{餐後的睡意}{686.0},

   在這裡有生命者,以這個而聖道不顯現。」

  「睡眠、懶惰、打哈欠,不喜樂、餐後的睡意,

   以活力遣離它後,聖道變得清澈。」



\sutta{17}{17}{困難經}{https://agama.buddhason.org/SN/sn.php?keyword=1.17}
  「\twnr{沙門身分}{328.0}是困難的,以無能者是難忍耐的,

   因為多數在那裡有障礙,愚者在該處沈沒。」

  「能行多少沙門身分,如果不防護心?

   會在一步步中沈沒,隨順於意向的控制。」

  「如龜的肢體在自己的龜殼中,\twnr{比丘在意之尋中收存著}{x9},

   \twnr{不依止的}{x10}、不惱害其他人的,已般涅槃的、他不會辱罵任何人。」



\sutta{18}{18}{慚經}{https://agama.buddhason.org/SN/sn.php?keyword=1.18}
  「\twnr{慚}{250.0}之抑止的男子,在世間中存在嗎?

   凡不激起斥責,如良馬對鞭者?」

  「慚之抑止者是稀少的,他們總是具念地行,

   到達苦的終結後,\twnr{在不平順中平順地行}{882.0}。」



\sutta{19}{19}{小屋經}{https://agama.buddhason.org/SN/sn.php?keyword=1.19}
  「沒有你的小屋嗎?沒有巢嗎?

   沒有子孫嗎?你是解脫繫縛者嗎?」

  「確實沒有我的小屋,確實沒有巢,

   確實沒有子孫,我確實是解脫繫縛者。」

  「我說小屋究竟是什麼?我說你的巢是什麼?

   我說你的子孫是什麼?我說繫縛究竟是什麼?」

  「你說小屋是母親,你說巢是妻子,

   你說子孫是兒子,你說我的繫縛是渴愛。」

  「沒有你的小屋-\twnr{好}{44.0}!沒有巢-好!

   沒有子孫-好!你是解脫繫縛者-好!」



\sutta{20}{20}{三彌提經}{https://agama.buddhason.org/SN/sn.php?keyword=1.20}
  \twnr{被我這麼聽聞}{1.0}:

  \twnr{有一次}{2.0},\twnr{世尊}{12.0}住在王舍城溫泉園。

  那時,\twnr{尊者}{200.0}三彌提在破曉時起來後,前往溫泉灌洗身體。

  在溫泉灌洗身體後起來後,單衣站立弄乾著身體。

  那時,在夜已深時,容色絕佳的某位天神使整個溫泉園發光後,去見尊者三彌提。抵達後,成為空中站立者,以\twnr{偈頌}{281.0}對尊者三彌提說:

  「\twnr{比丘}{31.0}!不享受後你\twnr{乞食}{87.0},享受後你不乞食,

   比丘!享受後請你乞食,不要時間離開你。」

  「\twnr{時間我不知道}{x11},被隱藏的時間不被看見,

   因此不享受後我乞食,不要時間離開我。」

  那時,那位天神站到地上後對尊者三彌提說這個:

  「比丘!你年輕出家,黑髮的青年,具備青春的幸福,在人生初期,不在欲中娛樂,比丘!你要在人之欲中享受,不要捨棄直接可見的後追逐\twnr{時間的}{x12}。」

  「\twnr{朋友}{201.0}!我沒捨棄直接可見的後追逐時間的,朋友!但我捨棄時間的後追逐直接可見的,朋友!因為欲被世尊說是時間的、多苦的、多\twnr{絕望}{342.0}的,在這裡有更多的\twnr{過患}{293.0}。這個法是直接可見的、即時的、請你來看的、能引導的、應該被智者各自經驗的。」

  「比丘!但怎樣是欲被世尊說是時間的、多苦的、多絕望的,在這裡有更多的過患?怎樣是這個法是直接可見的、即時的、請你來看的、能引導的、應該被智者各自經驗的?」

  「朋友!我是新人、出家不久者、剛來這法律者,我不能詳細地講述它,那位世尊、\twnr{阿羅漢}{5.0}、遍正覺者住在這王舍城溫泉園,你去見世尊後問這個道理,你應該如世尊為你解說那樣\twnr{憶持}{57.0}它。」

  「比丘!去見那位世尊不易被我們做,因為被其它大影響力的諸天圍繞。比丘!如果你去見那位世尊後問這個道理,我們也會為了聽法而來到。」

  「是的,朋友!。」尊者三彌提回答那位天神後,去見世尊。抵達後,向世尊\twnr{問訊}{46.0}後,在一旁坐下。在一旁坐下的尊者三彌提對世尊說這個:

  「\twnr{大德}{45.0}!這裡,我在破曉時起來後,到溫泉灌洗身體。

  在溫泉灌洗身體後起來後,單衣站立弄乾著身體。

  那時,在夜已深時,容色絕佳的某位天神使整個溫泉園發光後,來見我。抵達後,站在空中以偈頌對我說:

  『比丘!不享受後你乞食,享受後你不乞食, 

   比丘!享受後請你乞食,不要時間離開你。』

  大德!在這麼說時,我以偈回應那位天神:

  『時間我不知道,被隱藏的時間不被看見,

   因此不享受後我乞食,不要時間離開我。』

  大德!那時,那位天神站到地上對我說這個:『比丘!你年輕出家,黑髮的青年,具備青春的幸福,在人生初期,不在欲中娛樂,比丘!你要在人之欲中享受,不要捨棄直接可見的後追逐時間的。』

  大德!在這麼說時,我這麼回答那位天神:『朋友!我沒捨棄直接可見的後追逐時間的,朋友!但我捨棄時間的後追逐直接可見的,朋友!因為欲被世尊說是時間的、多苦的、多絕望的,在這裡有更多的過患。這個法是直接可見的、即時的、請你來看的、能引導的、應該被智者各自經驗的。』

  大德!在這麼說時,那位天神對我說這個:『比丘!但怎樣是欲被世尊說是時間的、多苦的、多絕望的,在這裡有更多的過患?怎樣是這個法是直接可見的、即時的、請你來看的、能引導的、應該被智者各自經驗的?』

  大德!在這麼說時,我這麼回答那位天神:『朋友!我是新人、出家不久者、剛來這法律者,我不能詳細地講述它,那位世尊、阿羅漢、遍正覺者住在這王舍城溫泉園,你去見世尊後問這個道理,你應該如世尊為你解說那樣憶持它。』

  大德!在這麼說時,那位天神對我說這個:『比丘!去見那位世尊不易被我們做,因為被其它大影響力的諸天圍繞,比丘!如果你去見世尊後問這個道理,我們也會為了聽法而來到。』

  大德!如果那位天神的言語是真實的,那位天神就在這裡不遠處。」

  在這麼說時,那位天神對尊者三彌提說這個:

  「比丘!請你問,比丘!請你問,我已到達。」

  那時,世尊以偈頌對那位天神說:

  「\twnr{能被講述的有想者眾生}{x13},在能被講述的之上被建立,

   不\twnr{遍知}{154.0}能被講述的後,\twnr{被死神束縛}{x14}。

   但遍知能被講述的之後,不\twnr{思量}{963.0}講述者,[\ccchref{It.63}{https://agama.buddhason.org/It/dm.php?keyword=63}]

   因為對他『那個不存在』:以那個依能說他的對他不存在。

   如果你了知,請你說,\twnr{夜叉}{126.0}!」

  「大德!我對這個被世尊簡要地說的,不詳細地了知義理,大德!請世尊能為我說明,依之我能詳細地了知這被世尊簡要地說的義理,\twnr{那就好了}{44.0}!」

  「相同的或者殊勝的或卑劣的,凡他思量者因為那樣他會爭論,

   當在三種上不動搖時,『相同的殊勝的』對他來說不存在。

   如果你了知,請你說,夜叉!」

  「大德!我對這個被世尊簡要地說的,也不詳細地了知義理,請世尊能為我說明,依之我能詳細地了知這被世尊簡要地說的義理,那就好了!」

  「捨斷名稱他不到達\twnr{勝慢}{x15},在這裡切斷在名色上的渴愛,

   那位繫結已被切斷、無苦惱、無願望者,當遍求時他們沒得到:

   諸天、人們在此世或在後世,在天界或在一切住處。

   如果你了知,請你說,夜叉!」

  「大德!我對這個被世尊簡要地說的,這樣詳細地了知義理:

  『世間中任何事物,惡的不應該被身、語、意作,

   捨斷諸欲後有念正知,不應該實行苦的、伴隨無利益的。』」

  歡喜園品第二,其\twnr{攝頌}{35.0}:

  「歡喜園、歡喜,沒有等同兒子者,

   剎帝利與出聲,睡眠懶惰與困難,

   慚、小屋為第九,三彌提被說為第十則。」





\pin{矛品}{21}{30}
\sutta{21}{21}{矛經}{https://agama.buddhason.org/SN/sn.php?keyword=1.21}
  起源於舍衛城。

  在一旁站立的那位天神在\twnr{世尊}{12.0}面前說這\twnr{偈頌}{281.0}:

  「像被矛觸擊,如在頭處被燃燒著,

   為了欲貪的捨斷,\twnr{比丘}{31.0}應該具念地\twnr{遊行}{61.0}。」

  「像被矛觸擊,如在頭處被燃燒著,

   為了\twnr{有身見}{93.1}的捨斷,比丘應該具念地遊行。」



\sutta{22}{22}{接觸經}{https://agama.buddhason.org/SN/sn.php?keyword=1.22}
  「\twnr{不接觸沒接觸者}{x16},但之後會接觸接觸者,

   因此接觸接觸者:[對]\twnr{無犯錯者的有過失者}{x17}。」

  「凡冒犯無犯錯的人,純淨無穢的人,

   惡就回到那位愚者,如細塵被逆風地拋出。」[\suttaref{SN.7.4}]



\sutta{23}{23}{結縛經}{https://agama.buddhason.org/SN/sn.php?keyword=1.23}
  「內結縛\twnr{外結縛}{x18},\twnr{世代}{38.0}被結縛糾纏,

   \twnr{喬達摩}{80.0}!我問你這個:誰能解開這個結縛?」

  「有慧的人在戒上確立後,\twnr{心與慧的修習者}{x19},

   熱心明智的\twnr{比丘}{31.0},他能解開這個結縛。

   凡他們的貪與瞋,以及\twnr{無明}{207.0}已脫離者,

   諸漏已滅盡的\twnr{阿羅漢}{5.0},他們的結縛已被解開。

   於名與色處,被無餘地破壞,

   \twnr{有對與色想}{x20},那個結縛在這裡能被切斷。」[\suttaref{SN.7.6}]



\sutta{24}{24}{意的制止經}{https://agama.buddhason.org/SN/sn.php?keyword=1.24}
  「應該從每一處制止意,從每一該處苦不來到他,

   如果他從一切制止意,他從一切苦被釋放。[」]

  「不應該從一切制止意:非意的自我抑制已到達者,

   從惡的每一處,從每一該處應該制止意。」



\sutta{25}{25}{阿羅漢經}{https://agama.buddhason.org/SN/sn.php?keyword=1.25}
  「凡\twnr{比丘}{31.0}是\twnr{阿羅漢}{5.0}、已完成者,諸漏已滅盡者、持最後身者,

   他也會說:『我說』嗎?他也會說:『他們對我說』嗎?」

  「凡比丘是阿羅漢、已完成者,諸漏已滅盡者、持最後身者,

   他也會說:『我說』,他也會說:『他們對我說』,

   善知世間上的通稱後,\twnr{他會只以慣用語的程度說}{x21}。」

  「凡比丘是阿羅漢、已完成者,諸漏已滅盡者、持最後身者,

   那位比丘走入慢後,他也會說:『我說』嗎?他也會說:『他們對我說』嗎?」

  「捨斷慢者沒有繫結,慢繫結的一切都被破壞,

   那位極明智者已超越\twnr{思量}{963.0},他也會說:『我說』,

   他也會說:『他們對我說』,善知世間上的名稱後,他會只以慣用語的程度說。」



\sutta{26}{26}{燈火經}{https://agama.buddhason.org/SN/sn.php?keyword=1.26}
  「世間中有多少燈火,以它們世間變明亮?

   我們來問\twnr{世尊}{12.0},我們應該如何知道它?」

  「世間中有四種燈火,第五種在這裡沒被發現,

   太陽在白天照亮,月亮在晚上發亮,

   而火在白天與晚上,到處變明亮,

   正覺者是照亮者中最勝的,那是無上的光明。」[\suttaref{SN.2.4}]



\sutta{27}{27}{溪流經}{https://agama.buddhason.org/SN/sn.php?keyword=1.27}
  「\twnr{溪流}{664.0}從哪裡折回?輪轉在何處不轉?

   名與色在何處,被破壞無餘?」

  「於水地火風,不堅立之處,

   從那裡溪流折回,在這裡輪轉不轉,

   在這裡名與色,被破壞無餘。」



\sutta{28}{28}{大富者經}{https://agama.buddhason.org/SN/sn.php?keyword=1.28}
  「大富者與大財富者們,即使\twnr{擁有國家的}{x22}\twnr{剎帝利}{116.0},

   他們相互貪求,是在諸欲上作不滿足者。

   \twnr{在那些生起貪欲者中}{x23},\twnr{在有之流隨行者中}{x24},

   這裡誰捨斷了渴愛?誰是世間中無貪欲者?」

  「捨棄家後出家者,捨棄兒子、家畜、所愛的(PTS版)後,

   捨棄貪與瞋後,脫離\twnr{無明}{207.0}後,

   諸漏已滅盡的\twnr{阿羅漢}{5.0},他們是世間中無貪欲者。」



\sutta{29}{29}{四輪經}{https://agama.buddhason.org/SN/sn.php?keyword=1.29}
  「四輪與九門,被貪充滿、結縛,

   污泥所生,大英雄!要如何逃離?」

  「切斷皮帶與細繩,及惡欲求與貪後,

   連根拔除渴愛後,要這樣逃離。」[\suttaref{SN.2.28}]



\sutta{30}{30}{如鹿小腿經}{https://agama.buddhason.org/SN/sn.php?keyword=1.30}
  「\twnr{如鹿小腿}{x25}瘦的英雄,少食者、無動貪者,

   如獅子、\twnr{龍象}{131.0}的獨行者,在諸欲上不期待者,

   我們來詢問,如何從苦被釋放?」

  「世間上的\twnr{五種欲}{187.0},\twnr{意被宣說為第六}{x26},

   在這裡脫離意欲後,這樣從苦被釋放。」

  矛品第三,其\twnr{攝頌}{35.0}:

  「矛與接觸,結縛、意的制止,

   阿羅漢與燈火,溪流與大富者,

   以四輪為第九,與如鹿小腿它們為十。」





\pin{沙睹羅巴眾品}{31}{40}
\sutta{31}{31}{與善人們經}{https://agama.buddhason.org/SN/sn.php?keyword=1.31}
  \twnr{被我這麼聽聞}{1.0}:

  \twnr{有一次}{2.0},\twnr{世尊}{12.0}住在舍衛城祇樹林給孤獨園。

  那時,在夜已深時,容色絕佳的眾多屬於沙睹羅巴眾天神使整個祇樹林發光後,去見世尊。抵達後,向世尊\twnr{問訊}{46.0}後,在一旁站立。在一旁站立的一位天神在世尊的面前說這\twnr{偈頌}{281.0}:

  「只應該與善人們結交,應該與善人們作親密交往,

   了知善的正法後,成為更好的而非惡的。」

  那時,另一位天神在世尊的面前說這偈頌:

  「只應該與善人們結交,應該就與善人們作親密交往,

   了知善的正法後,慧被得到而非從其他者。」

  那時,另一位天神在世尊的面前說這偈頌:

  「只應該與善人們結交,應該就與善人們作親密交往,

   了知善的正法後,在憂愁中不憂愁。」

  那時,另一位天神在世尊的面前說這偈頌:

  「只應該與善人們結交,應該就與善人們作親密交往,

   了知善的正法後,在親族中輝耀。」

  那時,另一位天神在世尊的面前說這偈頌:

  「只應該與善人們結交,應該就與善人們作親密交往,

   了知善的正法後,眾生到\twnr{善趣}{112.0}。」

  那時,另一位天神在世尊的面前說這偈頌:

  「只應該與善人們結交,應該就與善人們作親密交往,

   了知善的正法後,眾生經常地住立。」

  那時,另一位天神對世尊說這個:

  「世尊!誰的被善說呢?」

  「一切你們的以\twnr{法門}{562.0}被善說,此外,請你們也聽我的:

  『應該就與善人們結交,應該就與善人們作親密交往,

   了知善的正法後,從一切苦被釋放。』」[\suttaref{SN.2.21}]

  世尊說這個,那些悅意的天神向世尊問訊、\twnr{作右繞}{47.0}後,就在那裡消失。



\sutta{32}{32}{慳吝經}{https://agama.buddhason.org/SN/sn.php?keyword=1.32}
  \twnr{有一次}{2.0},\twnr{世尊}{12.0}住在舍衛城祇樹林給孤獨園。

  那時,在夜已深時,容色絕佳的眾多屬於沙睹羅巴眾天神使整個祇樹林發光後,去見世尊。抵達後,向世尊\twnr{問訊}{46.0}後,在一旁站立。在一旁站立的一位天神在世尊的面前說這\twnr{偈頌}{281.0}:

  「以慳吝與放逸,這樣布施不被施與,

   希望福德者,\twnr{是施物的了知者}{x27}。」

  那時,另一位天神在世尊的面前說這些偈頌:

  「凡慳吝者不施與所畏懼的,就成為那位不施與者的害怕:

   飢餓與口渴,凡慳吝者害怕的,

   那位愚者就接觸,在這個世間與下一個中。

   因此調伏慳吝後,征服垢穢者應該施與布施,

   福德在來世中,是有生命者之所依。」

  那時,另一位天神在世尊的面前說這些偈頌:

  「\twnr{他們在死者中不死}{x28},如道路的同行者,

   凡他們在少許中施與者,這是古法。

   一些在少許中施與,一些在許多中不想要施與,

   在少許中被施與的\twnr{供養}{953.0},是等同千倍之量的。」

  那時,另一位天神在世尊的面前說這些偈頌:

  「對難施與的施與者們,難做事務(業)的作者們:

   不善者們不模仿,正法是難隨行的。

   因此善者不善者們,從這裡有種種\twnr{趣處}{679.0},

   不善者們去地獄,善者們有天界趣處。」

  那時,另一位天神在世尊的面前說這個:

  「世尊!誰的被善說呢?」

  「一切你們的以\twnr{法門}{562.0}被善說,此外,請你們也聽我的:

  『凡拾落穗者如果也行法,如果扶養妻子者行在少許中施與的,

   \twnr{百千個的供千犧牲者}{x29},他們也不值得像那樣的\twnr{十六分之一}{x30}。』」

  那時,另一位天神以偈頌對世尊說這個:

  「為何這廣大的牲祭,不等同地來到所施的價值呢?

   為何百千個的供千犧牲者,他們也不值得像那樣的十六分之一呢?」

  「因為一些施與被建立在不正的上,切斷、殺害及使之憂愁後,

   那些淚滿面的、有懲罰的供養,不等同地來到所施的價值。

   這樣百千個的供千犧牲者,他們也不值得像那樣的十六分之一。」



\sutta{33}{33}{好經}{https://agama.buddhason.org/SN/sn.php?keyword=1.33}
  起緣於舍衛城。

  那時,在夜已深時,容色絕佳的眾多屬於沙睹羅巴眾天神使整個祇樹林發光後,去見世尊。抵達後,向世尊\twnr{問訊}{46.0}後,在一旁站立。在一旁站立的一位天神在世尊的面前吟出這\twnr{優陀那}{184.0}:

  「\twnr{親愛的先生}{204.0}!布施是\twnr{好的}{44.0},

   以慳吝與放逸,這樣布施不被施與,

   希望福德者,\twnr{是施物的了知者}{x31}。」

  那時,另一位天神在世尊的面前吟出這優陀那:

  「親愛的先生!布施是好的,此外在少許中的布施也是好的,

   一些在少許中施與,一些在許多中不想要施與,

   在少許中被施與的\twnr{供養}{953.0},是等同千倍之量的。」

  那時,另一位天神在世尊的面前吟出這優陀那:

  「親愛的先生!布施是好的,在少許中的布施也是好的,

   此外以信的布施也是好的,

   他們說布施與戰爭是相同的,即使少許善的也打勝眾多的,

   如果有信者即使施與少許,就因為那樣他在來世是幸福者。」

  那時,另一位天神在世尊的面前吟出這優陀那:

  「親愛的先生!布施是好的,在少許中的布施也是好的,

   以信的布施也是好的,此外如法所得的布施也是好的,

   凡施與如法所得的,以活力的努力所獲得的布施之人,

   他超越閻摩王的[地獄]灰河後,死後來到天的住處。」

  那時,另一位天神在世尊的面前吟出這優陀那:

  「親愛的先生!布施是好的,在少許中的布施也是好的,

   以信的布施也是好的,如法所得的布施也是好的,

   此外\twnr{簡別後的布施}{x32}也是好的,

   簡別後的布施被\twnr{善逝}{8.0}稱讚,凡這裡在生命世界中應該被供養者,

   在這些上的布施有大果,如種子被播種在良田中。」

  那時,另一位天神在世尊的面前吟出這優陀那:

  「親愛的先生!布施是好的,在少許中的布施也是好的,

   以信的布施也是好的,如法所得的布施也是好的,

   簡別後的布施也是好的,此外在生命類上[自我]抑制也是好的,

   凡對\twnr{活的生命類}{470.0}不惱害的行者,以他人的斥責他們不作惡,

   在那裡他們讚賞膽小者而非英雄,因為善人以害怕而不作惡。」

  那時,另一位天神在世尊的面前說這個: 

  「世尊!誰的被善說呢?」 

  「一切你們的以\twnr{法門}{562.0}被善說,此外,請你們也聽我的:

  『確實(羅馬拼音版)布施被許多方式稱讚,但\twnr{法足}{944.0}甚且是比布施更好的,

   因為在以前與更早以前,有慧的善人甚且證得涅槃。」



\sutta{34}{34}{沒有經}{https://agama.buddhason.org/SN/sn.php?keyword=1.34}
  \twnr{有一次}{2.0},\twnr{世尊}{12.0}住在舍衛城祇樹林給孤獨園。

  那時,在夜已深時,容色絕佳的眾多屬於沙睹羅巴眾天神使整個祇樹林發光後,去見世尊。抵達後,向世尊\twnr{問訊}{46.0}後,在一旁站立。在一旁站立的一位天神在世尊的面前說這\twnr{偈頌}{281.0}:

  「在人們中欲沒有常的,在這裡有著能被欲求的而在那些上被繫縛,

   在那些上放逸者是不再來者:男子是從死神的領域不再來者。」

  「\twnr{痛苦}{752.0}是意欲生的、苦是意欲生的,

   以意欲的調伏有痛苦的調伏,以痛苦的調伏有苦的調伏。」

  「那些不是欲-凡世間中\twnr{美的}{x33},\twnr{貪的意向}{x34}是男子的欲,

   世間中美的只像那樣住立,而在這裡明智者們調伏意欲。

   應該捨棄憤怒應該放棄慢,應該超越一切結,

   他在名色上不執著,苦不降臨\twnr{無所有者}{x35}。

   他捨斷名稱不來到勝慢,這裡他在名色上切斷渴愛,

   那位繫結已被切斷、無苦惱、無願望者,當遍求時他們沒得到:

   諸天、人們在此世或在後世,在天界或在一切住處。」

  (像這樣\twnr{尊者}{200.0}摩加拉奢:)

  「如果他們沒看見那位像這樣的解脫者:諸天、人們在此世或在後世,

   最上的人、對人們利益的行者,凡禮敬他者他們應該被讚賞?」

  (世尊:「摩加拉奢!」)

  「那些\twnr{比丘}{31.0}也是應該被讚賞者,凡禮敬著像這樣的解脫者:

   了知法後捨斷疑後,那些比丘也是已超越執著者。」



\sutta{35}{35}{挑毛病經}{https://agama.buddhason.org/SN/sn.php?keyword=1.35}
  \twnr{有一次}{2.0},\twnr{世尊}{12.0}住在舍衛城祇樹林給孤獨園。

  那時,在夜已深時,容色絕佳的眾多\twnr{挑毛病}{x36}天神使整個祇樹林發光後,去見世尊。抵達後,站在空中。

  在空中站好後,一位挑毛病天神在世尊的面前說這\twnr{偈頌}{281.0}:

  「對自己是以一種方式,凡如果宣說以另一種方式者,

   如賭博者詐欺後,他所受用的是以他的偷盜。

   應該說凡你會作的,不應該說凡你不會作的,

   對說而不做者,賢智者們遍知。」

  「這非僅以說,或單以聽聞,

   能夠隨進入,凡這個堅固的道跡,

   以這個明智者、禪修者們,從魔的繫縛被釋放。

   確實明智者們不做:知道世間法門後,

   明智者們被\twnr{完全智}{489.0}冷卻,已度脫世間中執著。」

  那時,那些天神站到地上後,以頭落在世尊的腳上後對世尊說這個:

  「\twnr{大德}{45.0}!罪過征服如是愚的、如是愚昧的、如是不善的我們:凡我們認為世尊應該被攻擊。大德!為了未來的\twnr{自制}{217.0},請世尊接受那些我們的罪過為罪過。」

  那時,世尊顯露微笑。

  那時,那些天神成為更加挑毛病者,升上空中。

  一位挑毛病天神在世尊的面前說這偈頌:

  「對懺悔過錯者們,凡如果不接受者,

   內部憤怒、嚴重的瞋恚,他到達敵意。」

  「如果你不存在過錯,這裡如果沒有入歧途者,

   如果不平息敵意,這裡以誰會是善的?」

  「對誰來說過錯不存在?對誰來說不是入歧途者?

   誰不來到迷妄?而誰是經常具念的明智者?」

  「對如來、佛陀,一切生類的憐愍者,

   對他來說過錯不存在,對他來說不是入歧途者,

   他不走到迷妄,他就是經常具念的明智者。」

  「對懺悔過錯者們,凡如果不接受者,

   內部憤怒、嚴重的瞋恚,他到達敵意,

   那敵意我不歡喜,我接受你們的罪過。」



\sutta{36}{36}{信經}{https://agama.buddhason.org/SN/sn.php?keyword=1.36}
  \twnr{有一次}{2.0},\twnr{世尊}{12.0}住在舍衛城祇樹林給孤獨園。

  那時,在夜已深時,容色絕佳的眾多屬於沙睹羅巴眾天神使整個祇樹林發光後,去見世尊。抵達後,向世尊\twnr{問訊}{46.0}後,在一旁站立。在一旁站立的一位天神在世尊的面前說這偈頌:

  「信是男子的伴侶,如果無信則不住立,

   從那裡會有名聲與稱譽,捨棄遺骸後他到天界。」

  那時,另一位天神在世尊的面前說這偈頌:

  「應該捨棄憤怒應該放棄慢,應該超越一切結,

   他在名色上不執著,苦不降臨\twnr{無所有者}{x37}。」

  「愚者們、劣智慧的人們,從事放逸,

   而有智慧者對不放逸,如最上的財產般地保護。」

  「你們應該不要從事放逸,不要有欲之喜樂的親密交往,

   因為不放逸的修禪者,到達最高的[廣大的-\ccchref{MN.86}{https://agama.buddhason.org/MN/dm.php?keyword=86}]樂。」



\sutta{37}{37}{集會經}{https://agama.buddhason.org/SN/sn.php?keyword=1.37}
  \twnr{被我這麼聽聞}{1.0}:

  \twnr{有一次}{2.0},\twnr{世尊}{12.0}與約五百位\twnr{比丘}{31.0}全部都是\twnr{阿羅漢}{5.0}的大比丘\twnr{僧團}{375.0},共住在釋迦族人的迦毘羅衛城大林中,而從十個世間界的大部分天神,為了見世尊與比丘僧團已集合。

  那時,四位淨居天的天神想這個:

  「這位世尊與約五百位比丘全部都是阿羅漢的大比丘僧團,共住在釋迦族人的迦毘羅衛城大林中,而從十個世間界的大部分天神,為了見世尊與比丘僧團已集合,也讓我們去見世尊,抵達後,在世尊的面前各自說\twnr{偈頌}{281.0}。」

  那時,那些天神就猶如有力氣的男子伸直彎曲的手臂,或彎曲伸直的手臂,就像這樣在淨居天消失,出現在世尊的面前。

  那時,那些天神向世尊問訊後,在一旁站立。在一旁站立的一位天神在世尊的面前說這偈頌:

  「在叢林中的大集會,天群已集合,

   我們來到這個法的集會,見不敗的僧團。」

  那時,另一位天神在世尊的面前說這偈頌:

  「在那裡比丘們入定,端正自己的心,

   如御車手握持韁繩(引導物)後,賢智者們守護諸根。」

  那時,另一位天神在世尊的面前說這偈頌:

  「切斷標柱、切斷橫木後,移除\twnr{因陀羅柱}{631.0}後無\twnr{擾動}{965.0}的,

   他們清淨地離垢地行,小龍象被\twnr{有眼者}{629.0}善調御。」

  那時,另一位天神在世尊的面前說這偈頌:

  「凡任何已\twnr{歸依}{284.0}佛者,他們將不去\twnr{苦界}{109.0}之地,

   捨棄人身後,將使天身充滿。」[\ccchref{DN.20}{https://agama.buddhason.org/DN/dm.php?keyword=20}, 332段]



\sutta{38}{38}{碎石片經}{https://agama.buddhason.org/SN/sn.php?keyword=1.38}
  \twnr{被我這麼聽聞}{1.0}:

  \twnr{有一次}{2.0},\twnr{世尊}{12.0}住在王舍城嘛瘩姑七的鹿林。

  當時,世尊的腳被\twnr{碎石片}{x38}所傷,世尊的強烈感受轉起:苦的、激烈的、猛烈的、強烈的、不愉快的、不合意的身體的感受,世尊具念地、正知地忍受它,不被惱害著。那時,世尊摺大衣成四折後,[左]腳放在[右]腳上後,具念正知地\twnr{以右脅作獅子臥}{367.0}。

  那時,在夜已深時,容色絕佳的七百位屬於沙睹羅巴眾天神使整個嘛瘩姑七發光後,去見世尊。抵達後,向世尊\twnr{問訊}{46.0}後,在一旁站立。在一旁站立的一位天神在世尊的面前吟出這\twnr{優陀那}{184.0}:

  「\twnr{先生}{202.0}!\twnr{沙門}{29.0}\twnr{喬達摩}{80.0}確實是龍象,當生起苦的、激烈的、猛烈的、強烈的、不愉快的、不合意的身體的感受時,以如龍象的行為模式具念地、正知地忍受它,不被惱害著。」

  那時,另一位天神在世尊的面前吟出這優陀那:

  「先生!沙門喬達摩確實是獅子,當生起苦的、激烈的、猛烈的、強烈的、不愉快的、不合意的身體的感受時,以如獅子的行為模式具念地、正知地忍受它,不被惱害著。」

  那時,另一位天神在世尊的面前吟出這優陀那:

  「先生!沙門喬達摩確實是駿馬,當生起苦的、激烈的、猛烈的、強烈的、不愉快的、不合意的身體的感受時,以如駿馬的行為模式具念地、正知地忍受它,不被惱害著。」

  那時,另一位天神在世尊的面前吟出這優陀那:

  「先生!沙門喬達摩確實是牛王,當生起苦的、激烈的、猛烈的、強烈的、不愉快的、不合意的身體的感受時,以如牛王的行為模式具念地、正知地忍受它,不被惱害著。」

  那時,另一位天神在世尊的面前吟出這優陀那:

  「先生!沙門喬達摩確實是忍耐強的牛,當生起苦的、激烈的、猛烈的、強烈的、不愉快的、不合意的身體的感受時,以如忍耐強的牛的行為模式具念地、正知地忍受它,不被惱害著。」

  那時,另一位天神在世尊的面前吟出這優陀那:

  「先生!沙門喬達摩確實是已調御者,當生起苦的、激烈的、猛烈的、強烈的、不愉快的、不合意的身體的感受時,以如已調御者的行為模式具念地、正知地忍受它,不被惱害。」

  那時,另一位天神在世尊的面前吟出這優陀那:

  「看吧!定已善\twnr{修習}{94.0}者與心已\twnr{善解脫}{28.0}者:不彎曲、\twnr{不彎離}{276.0}、不\twnr{進入被有行折伏後妨礙狀態的}{591.0},凡像這樣的龍象之男子、獅子之男子、駿馬之男子、牛王之男子、忍耐強的牛之男子、已調御者之男子,如果認為能被超越,除了沒看見者之外還有什麼?」

  「婆羅門有五吠陀,苦行者行百年,

   而他們的心不被正確地解脫,那些下劣義形色者不到彼岸。

   陷入了渴愛、被誓願與戒束縛者,行粗苦行百年,

   而他們的心不被正確地解脫,那些下劣義形色者不到彼岸。

   這裡慢之愛欲者沒有調御,不得定者沒有\twnr{牟那}{125.0},

   單獨放逸地住在\twnr{林野}{142.0},不能渡死亡領域彼岸。」

  「捨斷慢後為善得定狀態,善心者在一切處被釋放,

   單獨不放逸地住在林野,他能渡死亡領域的彼岸。[\suttaref{SN.1.9}]」



\sutta{39}{39}{雨神的女兒經第一}{https://agama.buddhason.org/SN/sn.php?keyword=1.39}
  \twnr{被我這麼聽聞}{1.0}:

  \twnr{有一次}{2.0},\twnr{世尊}{12.0}住在毘舍離大林重閣講堂。

  那時,在夜已深時,容色絕佳的雨神女兒紅蓮使整個大林發光後,去見世尊。抵達後,向世尊\twnr{問訊}{46.0}後,在一旁站立。在一旁站立的那位雨神的女兒紅蓮女神在世尊的面前說這些\twnr{偈頌}{281.0}:

  「在毘舍離大林中居住者:眾生中最高的\twnr{正覺者}{185.1},

   我是紅蓮我禮拜:雨神的女兒紅蓮。

   在以前只被我聽聞:被\twnr{有眼者}{629.0}\twnr{領悟}{355.0}的法,

   現在我是證人我知道它:當牟尼、\twnr{善逝}{8.0}教導時。

   凡任何對聖法,行斥責的劣智慧者,

   他們到達恐怖的\twnr{叫喚[地獄]}{x39},經驗長久的苦。

   但凡在聖法上,以接受以寂靜到達者,

   捨棄人身後,將使天身充滿。」[\ccchref{DN.20}{https://agama.buddhason.org/DN/dm.php?keyword=20}, 332段]



\sutta{40}{40}{雨神的女兒經第二}{https://agama.buddhason.org/SN/sn.php?keyword=1.40}
  \twnr{被我這麼聽聞}{1.0}:

  \twnr{有一次}{2.0},\twnr{世尊}{12.0}住在毘舍離大林\twnr{重閣}{213.0}講堂。

  那時,在夜已深時,容色絕佳的雨神女兒小紅蓮使整個大林發光後,去見世尊。抵達後,向世尊\twnr{問訊}{46.0}後,在一旁站立。在一旁站立的那位雨神的女兒小紅蓮女神在世尊的面前說這些\twnr{偈頌}{281.0}:

  「這裡以閃電光輝美貌來,是雨神的女兒紅蓮,

   當禮敬佛與法時,我說這些有義理的偈頌。

   像那樣的法,我能以許多法門解析它,

   我將談談簡要的義理,以我意(心)所學得之所及。

   在一切世間中,任何惡不應該被身語意作,

   捨斷諸欲後有念正知,不應該實行苦的、伴隨無利益的。」

  沙睹羅巴眾品第四,其\twnr{攝頌}{35.0}:

  「與善人們、慳吝、好,沒有、挑毛病,

   信、集會、碎石片,兩則雨神的女兒。」





\pin{燃燒品}{41}{50}
\sutta{41}{41}{燃燒經}{https://agama.buddhason.org/SN/sn.php?keyword=1.41}
  \twnr{被我這麼聽聞}{1.0}:

  \twnr{有一次}{2.0},\twnr{世尊}{12.0}住在舍衛城祇樹林給孤獨園。

  那時,在夜已深時,容色絕佳的某位天神使整個祇樹林發光後,去見世尊。抵達後,向世尊\twnr{問訊}{46.0}後,在一旁站立。在一旁站立的那位天神在世尊的面前說這些\twnr{偈頌}{281.0}:

  「在家被燃燒時,凡取出器具,

   那是屬於他利益的,而非凡在那裡被燃燒[物]。

   這樣世間,被老與死燃燒,

   就應該以布施取出,被施與的是善取出的。[\ccchref{AN.3.53}{https://agama.buddhason.org/AN/an.php?keyword=3.53}]

   所施與的有樂果,未施與的它不是像那樣:

   盜賊國王們拿走,火燃燒、毀滅,

   而在最後捨棄,包含財產的遺體。

   有智慧者了知這樣後,應該受用及應該施與,

   如其能力地施與及受用後,無過失地到達天界處。」



\sutta{42}{42}{施與什麼經}{https://agama.buddhason.org/SN/sn.php?keyword=1.42}
  「施與什麼者是力氣施與者?施與什麼者是施與容色者?

   施與什麼者是安樂施與者?施與什麼者是施與眼睛者?

   誰是施與一切者?請你告知我所問的。」

  「施與食物者是施與力氣者,施與衣服者是施與容色者,

   施與車乘者是施與安樂者,施與燈火者是施與眼睛者。

   而他是施與一切者:凡施與住房者,

   但他是施與\twnr{不死}{123.0}者:凡教誡法者。」



\sutta{43}{43}{食物經}{https://agama.buddhason.org/SN/sn.php?keyword=1.43}
  「他們都歡喜食物:諸天與人們兩者,

   而還有什麼\twnr{夜叉}{126.0},不歡喜食物的?」

  「凡他們以信,以明淨心施與那個者,

   食物就服侍他,在這世間與下一個中。

   因此調伏慳吝後,征服垢穢者應該施與布施, 

   福德在來世中,是有生命者之所依。[\suttaref{SN.1.32}]」[\suttaref{SN.2.23}]



\sutta{44}{44}{一根本經}{https://agama.buddhason.org/SN/sn.php?keyword=1.44}
  「一根本、二漩渦,三垢穢、五岩石,

  大海\twnr{十二個漩渦}{x40}:仙人越過深淵。」



\sutta{45}{45}{最高經}{https://agama.buddhason.org/SN/sn.php?keyword=1.45}
  「完美名字者、微妙義的看見者,給與慧者、在欲之\twnr{阿賴耶}{391.0}上不執著者,

   請你們看那位已知一切的善智慧者:走入聖者之路的大仙。」



\sutta{46}{46}{天女經}{https://agama.buddhason.org/SN/sn.php?keyword=1.46}
  「響亮的天女眾,\twnr{是跟隨的惡鬼眾}{x41},

   那個樹林名為誘惑,要如何逃離?」

  「那個道路名為正直,那個方向名為無畏,

   馬車名為不嘎嘎響,被法輪連結的。

   慚是它的控制工具,念是它的帷幕,

   我說法是駕駛者,正見是先行者。

   凡有像這樣的車輛:女子的或男子的,

   他確實以這輛車,涅槃就在他的面前。」



\sutta{47}{47}{造林者經}{https://agama.buddhason.org/SN/sn.php?keyword=1.47}
  「誰的福德,日與夜經常地增長?

   誰是生天界之人,法住立者、\twnr{戒具足者}{x42}?」

  「園林種植者、樹林種植者,凡人們是造橋者,

   喝水處與水井,凡他們施與住房。

   他們的福德,日與夜經常地增長,

   他們是生天界之人,法住立者、戒具足者。」



\sutta{48}{48}{祇樹林經}{https://agama.buddhason.org/SN/sn.php?keyword=1.48}
  「這裡確實是那個祇樹林,仙人\twnr{僧團}{375.0}經常來往的,

   被法王居住,有我的喜之產生。

   行為、明與法,戒、最上的活命,

   \twnr{不免一死的人}{600.0}們以這些變純淨,非以種姓或以財產。

   因此賢智的人,自己利益的看見者,

   應該如理檢擇法,這樣在那裡變成清淨。

   如舍利弗以慧,以戒以寂靜,

   凡即使到\twnr{彼岸}{226.0}的\twnr{比丘}{31.0},最高者會是這樣程度的。」[\suttaref{SN.2.20}, \suttaref{SN.21.3}, \ccchref{MN.143}{https://agama.buddhason.org/MN/dm.php?keyword=143}]



\sutta{49}{49}{慳吝經}{https://agama.buddhason.org/SN/sn.php?keyword=1.49}
  「凡在這個世間中慳吝者,吝嗇者、惡口者,

   對其他施與者,製造障礙的人們。

   他們的果報是像什麼樣子的?以及來世是像什麼樣子的?

   我們來到後能問\twnr{世尊}{12.0},我們應該如何知道它?」

  「凡在這個世間中慳吝者,吝嗇者、惡口者,

   對其他施與者,製造障礙的人們。

   他們會往生,地獄、畜生界、\twnr{閻摩世界}{x43},

   如果來人的狀態,會被出生在貧困家,

   布料、食物、喜樂、娛樂,在那裡被困難地得到。

   愚者們從其他人希求的,那也不被他們得到,

   這是\twnr{在當生中}{42.0}的果報,且在來世為\twnr{惡趣}{110.0}。」

  「像這樣我們了知這個,\twnr{喬達摩}{80.0}!我們問另一個,

   凡在這裡已得人的狀態者,寬容者、離慳吝者,

   在佛與法上\twnr{淨信者}{340.0},在\twnr{僧團}{375.0}上極尊重者。

   他們的果報是像什麼樣子的?以及來世是像什麼樣子的?

   我們來到後能問世尊,我們應該如何知道它?」

  「凡在這裡已得人的狀態者,寬容者、離慳吝者,

   在佛與法上淨信者,在僧團上極尊重者。

   這些輝耀天界:於該處他們往生。

   如果來人的狀態,會被出生在富裕家,

   布料、食物、喜樂、娛樂,在那裡被不困難地得到。

    在他人聚集的財物上,能如自在天般地喜悅(享用),

   這是在當生中的果報,且在來世為\twnr{善趣}{112.0}。」



\sutta{50}{50}{額低葛勒經}{https://agama.buddhason.org/SN/sn.php?keyword=1.50}
  「往生無煩天,七位\twnr{比丘}{31.0}已解脫,

   貪瞋已滅盡,已度脫世間中的執著。」

  「而誰是那些度脫了泥沼,極難越過的死亡領域者?

   誰捨斷人的身體後,超越了天軛?」

  「優波迦與波羅揵荼,以及補估沙地他們三位,

   跋提雅與揵陀提婆,婆侯羅提與僧提雅,

   他們捨斷人的身體後,超越了天軛。」

  「你有善巧地說他們:魔網的捨斷者,

   他們了知誰的法後,切斷了有之繫縛?」

  「非從\twnr{世尊}{12.0}之外,非從你的教說之外,

   他們了知那位的法後,切斷了有之繫縛。

   於名與色處,被無餘地破壞,

   這裡他們了知這個法後,切斷了有之繫縛。」

  「你說了甚深的言語,難了知的、\twnr{極難覺醒的}{x44},

   你了知誰的法後,說像這樣的言語?」

  「在過去我是作陶器者,毘迦林加地方的陶匠,

   我是奉養父母者,迦葉佛的優婆塞。

   離婬欲法者,梵行無物質者,

   是你的同鄉,是你在過去的同伴。

   我是那位知道,這七位比丘已解脫,

   貪瞋已滅盡,已度脫世間中的執著。」

  「那時這是這樣,如你說,瑞祥兒!

   在過去你是作陶器者,毘迦林加地方的陶匠,

   你是奉養父母者,迦葉佛的優婆塞。

   離婬欲法者,梵行無物質者,

   是我的同鄉,是我在過去的同伴。」

  「這是這樣,是以前朋友的會合,

   \twnr{已自我修習}{658.0}的兩者,\twnr{持最後身者}{338.0}。」[\suttaref{SN.2.24}]

  燃燒品第五,其\twnr{攝頌}{35.0}:

  「燃燒、施與什麼、食物,一根本、最高,

   天女、造林者、祇樹,慳吝與額低葛勒。」





\pin{衰老品}{51}{60}
\sutta{51}{51}{衰老經}{https://agama.buddhason.org/SN/sn.php?keyword=1.51}
  「什麼是善的直到衰老?什麼是善的當已住立時?

   什麼是人人的寶物?什麼是難被盜賊拿走的?」

  「戒(德行)是善的直到衰老,信是善的當已住立時,

   慧是人人的寶物,福德是難被盜賊拿走的。」



\sutta{52}{52}{以不衰老經}{https://agama.buddhason.org/SN/sn.php?keyword=1.52}
  「什麼以不衰老是善的?當什麼已確立時是善的?

   什麼是人人的寶物?什麼是不容易被盜賊拿走的?」

  「戒(德行)以不衰老是善的,當信已確立時是善的,

   慧是人人的寶物,福德是不容易被盜賊拿走的。」



\sutta{53}{53}{朋友經}{https://agama.buddhason.org/SN/sn.php?keyword=1.53}
  「什麼是遠行者的朋友?什麼是自己家中的朋友?

   什麼是當需要生起時的朋友?什麼是來生的朋友?」

  「商隊是遠行者的朋友,母親是自己家中的朋友,

   同伴是當需要生起時,一再的朋友,

   自己所作的福德,那是來生的朋友。」



\sutta{54}{54}{所依經}{https://agama.buddhason.org/SN/sn.php?keyword=1.54}
  「什麼是人的\twnr{所依}{x45}?這裡什麼是最上的同伴?

   他們依什麼生活:凡依止大地的生類生物?」

  「兒子是人的所依,這裡妻子是最上的同伴,

   他們依雨生活:凡依止大地的生類生物[\suttaref{SN.1.80}]。」



\sutta{55}{55}{人經第一}{https://agama.buddhason.org/SN/sn.php?keyword=1.55}
  「什麼使人出生?他的什麼跑來跑去?

   什麼來到輪迴?什麼是他的大恐怖?」

  「渴愛使人出生,他的心跑來跑去,

   眾生來到輪迴,苦是他的大恐怖。」



\sutta{56}{56}{人經第二}{https://agama.buddhason.org/SN/sn.php?keyword=1.56}
  「什麼使人出生?他的什麼跑來跑去?

   什麼來到輪迴?不從什麼被釋放?」

  「渴愛使人出生,他的心跑來跑去,

   眾生來到輪迴,不從苦被釋放。」



\sutta{57}{57}{人經第三}{https://agama.buddhason.org/SN/sn.php?keyword=1.57}
  「什麼使人出生?他的什麼跑來跑去?

   什麼來到輪迴?什麼是他的所趣處?」

  「渴愛使人出生,他的心跑來跑去,

   眾生來到輪迴,\twnr{業是他的所趣處}{x46}。」



\sutta{58}{58}{邪道經}{https://agama.buddhason.org/SN/sn.php?keyword=1.58}
  「什麼被告知為邪道?什麼有日夜之消逝?

   什麼是\twnr{梵行}{381.0}的垢穢?什麼是無水的沐浴?[\suttaref{SN.1.76}]」

  「貪被告知為邪道,青春有日夜之消逝,

   女子是梵行的垢穢,\twnr{在這裡這個被世代黏著}{x47},

   苦行與梵行,那是無水的沐浴。」



\sutta{59}{59}{伴侶經}{https://agama.buddhason.org/SN/sn.php?keyword=1.59}
  「什麼是男子的\twnr{伴侶}{x48}?而什麼教導他?

   又\twnr{不免一死的人}{600.0}對什麼的喜樂,從一切苦被釋放?」

  「信是男子的伴侶,而慧教導他,

   又不免一死的人對涅槃的喜樂,從一切苦被釋放。」



\sutta{60}{60}{詩人經}{https://agama.buddhason.org/SN/sn.php?keyword=1.60}
  「什麼是\twnr{偈頌}{281.0}的起源?什麼是它們的相?

   什麼是偈頌依止的?什麼是偈頌的所依?」

  「\twnr{韻律是偈頌的起源}{x49},文字是它們的相,

   \twnr{名字是偈頌依止的}{x50},詩人是偈頌的所依。」

  衰老品第六,其\twnr{攝頌}{35.0}:

  「衰老、以不衰老、朋友,所依、人三則,

   邪道與伴侶,以詩人品被完成。」





\pin{征服品}{61}{70}
\sutta{61}{61}{名經}{https://agama.buddhason.org/SN/sn.php?keyword=1.61}
  「什麼征服一切?比什麼更多的不存在?

   什麼是\twnr{一法}{522.0},一切就隨其控制?」

  「\twnr{名征服一切}{x51},比名更多的不存在,

   名是一法,一切就隨其控制。」



\sutta{62}{62}{心經}{https://agama.buddhason.org/SN/sn.php?keyword=1.62}
  「世間被什麼引導?被什麼\twnr{牽引}{x52}?

   什麼是\twnr{一法}{522.0},一切就隨其控制?」

  「世間被心引導,被心牽引,

   心是一法,一切就隨其控制。」



\sutta{63}{63}{渴愛經}{https://agama.buddhason.org/SN/sn.php?keyword=1.63}
  「世間被什麼引導?被什麼\twnr{牽引}{x53}?

   什麼是\twnr{一法}{522.0},一切就隨其控制?」

  「世間被渴愛引導,被渴愛牽引,

   渴愛是一法,一切就隨其控制。」



\sutta{64}{64}{結經}{https://agama.buddhason.org/SN/sn.php?keyword=1.64}
  「什麼是世間的結?什麼是它的腳?

   以什麼的捨斷,像這樣被稱為涅槃?」

  「歡喜是世間的結,\twnr{尋是它的腳}{x54},

   以渴愛的捨斷,像這樣被稱為涅槃。」



\sutta{65}{65}{繫縛經}{https://agama.buddhason.org/SN/sn.php?keyword=1.65}
  「什麼是世間的連結?什麼是它的腳?

   以什麼的捨棄,切斷一切繫縛?」

  「歡喜是世間的連結,\twnr{尋是它的腳}{x55},

   以渴愛的捨斷,切斷一切繫縛。」



\sutta{66}{66}{被自己傷害(被折磨)經}{https://agama.buddhason.org/SN/sn.php?keyword=1.66}
  「世間被什麼折磨?被什麼包圍?

   被什麼箭刺入?經常被什麼薰?」

  「世間被死亡折磨,被老包圍,

   被渴愛之箭刺入,經常\twnr{被欲求薰}{x56}。」



\sutta{67}{67}{被綁住經}{https://agama.buddhason.org/SN/sn.php?keyword=1.67}
  「世間被什麼綁住(陷住)?被什麼包圍?

   世間被什麼覆蓋?世間在什麼上被建立?」

  「世間被渴愛綁住,被老包圍,

   世間被死亡覆蓋,世間在苦上被建立。」



\sutta{68}{68}{被覆蓋經}{https://agama.buddhason.org/SN/sn.php?keyword=1.68}
  「世間被什麼覆蓋?世間在什麼上被建立?

   世間被什麼綁住(陷住)?被什麼包圍?」

  「世間被死亡覆蓋,世間在苦上被建立,

   世間被渴愛綁住,被老包圍。」



\sutta{69}{69}{欲求經}{https://agama.buddhason.org/SN/sn.php?keyword=1.69}
  「世間被什麼繫縛?以什麼的調伏被解脫?

   以什麼的捨斷,切斷一切繫縛?」

  「世間被欲求繫縛,以欲求的調伏被解脫,

   以棄斷欲求,切斷一切繫縛。」



\sutta{70}{70}{世間經}{https://agama.buddhason.org/SN/sn.php?keyword=1.70}
  「世間在什麼之中被生起?在什麼之中作親密交往?

   對什麼執取後,世間在什麼之中被惱害?」

  「世間在\twnr{六}{x57}之中被生起,在六之中作親密交往,

   就對六執取後,世間在六之中被惱害。」

  征服品第七,其\twnr{攝頌}{35.0}:

  「名、心、渴愛,結、繫縛,

   被折磨、被綁住、被覆蓋,欲求與世間它們為十。」





\pin{切斷品}{71}{81}
\sutta{71}{71}{切斷後經}{https://agama.buddhason.org/SN/sn.php?keyword=1.71}
  起源於舍衛城。

  在一旁站立的那位天神以\twnr{偈頌}{281.0}對\twnr{世尊}{12.0}說:

  「切斷什麼後\twnr{睡得安樂}{532.0}?切斷什麼後不憂愁?

   對哪一法的殺害,\twnr{喬達摩}{80.0}同意?」

  「切斷憤怒後睡得安樂,切斷憤怒後不憂愁,

   天神!對端蜜,\twnr{而根毒之憤怒}{929.0}的殺害,

   聖者稱讚,因為切斷它後不憂愁。」[\suttaref{SN.2.3}]



\sutta{72}{72}{車經}{https://agama.buddhason.org/SN/sn.php?keyword=1.72}
  「什麼是車的表徵?什麼是火的表徵?

   什麼是國家的表徵?什麼是女人的表徵?」

  「旗幟是車的表徵,煙是火的表徵,

   國王是國家的表徵,丈夫是女人的表徵。」



\sutta{73}{73}{財產經}{https://agama.buddhason.org/SN/sn.php?keyword=1.73}
  「這裡什麼是男子的最上財產?善實行什麼者帶來樂?

   什麼確實是味道中最美味的?如何活著者是他們說最上的活命者?」

  「這裡信是男子的最上財產,善實行法者帶來樂,

   真理確實是味道中最美味的,以慧活著者是他們說最上的活命者。」



\sutta{74}{74}{雨經}{https://agama.buddhason.org/SN/sn.php?keyword=1.74}
  「什麼是最上的升起者?什麼是殊勝的落下者?

   對到處走者來說是什麼?什麼是殊勝的說話者?」

  「種子是最上的升起者,雨是殊勝的落下者,

   對到處走者來說是牛,兒子是殊勝的說話者。」

  「明是最上的升起者,無明是殊勝的落下者,

   對到處走者來說是\twnr{僧團}{375.0},佛陀是殊勝的說話者。」





\sutta{75}{75}{害怕經}{https://agama.buddhason.org/SN/sn.php?keyword=1.75}
  「這裡為何多數人是害怕的:當道被\twnr{好幾種處地}{x58}解說時?

   \twnr{喬達摩}{80.0}!我問你、廣慧者:當已站在什麼上面時不會害怕來世?」

  「正確地安置言語與意後,不以身作諸惡者,

   住在許多食物飲料家者,有信的、柔軟的、分享的、寬容的(親切的),

   當已站在這四法上時,當已站在法上時不會害怕來世。」



\sutta{76}{76}{不衰退經}{https://agama.buddhason.org/SN/sn.php?keyword=1.76}
  「什麼衰退、什麼不衰退?什麼被稱為『邪道』?

   什麼是法的障礙?什麼有日夜之消逝(滅盡)?

   什麼是\twnr{梵行}{381.0}的垢穢?什麼是無水的沐浴?

   世間中有多少孔隙,在那裡\twnr{{財產}[心]}{x59}不住立?

   我們來問\twnr{世尊}{12.0},我們應該如何知道它?」

  「\twnr{不免一死的人}{600.0}之色衰退、姓名不衰退,貪被稱為『邪道』,

   貪婪是法的障礙,青春有日夜之消逝。

   女子是梵行的垢穢,在這裡這個被世代黏著,

   苦行與梵行,那是無水的沐浴[\suttaref{SN.1.58}]。

   世間中有六個孔隙,在那裡心不住立:

   懶惰與放逸,不奮起不抑制,

   睡眠與倦怠,那些孔隙應該全部地回避它。」



\sutta{77}{77}{統治權經}{https://agama.buddhason.org/SN/sn.php?keyword=1.77}
  「什麼是世間中的統治權?什麼是物品中最上的?

   什麼是世間中的劍銹?什麼是世間中的\twnr{瘤}{885.0}?

   當誰拿走時他們阻止?而當誰拿走時是可愛的?

   當誰一再到來,賢智者們歡喜?」

  「權力是世間中的統治權,\twnr{女人是物品中最上的}{x60},

   憤怒是世間中的劍銹,盜賊是世間中的瘤。

   當盜賊拿走時他們阻止,而當\twnr{沙門}{29.0}拿走時\twnr{是可愛的}{x61},

   當沙門一再到來,賢智者們歡喜。」



\sutta{78}{78}{想要經}{https://agama.buddhason.org/SN/sn.php?keyword=1.78}
  「想要利益者不應該布施什麼?\twnr{不免一死的人}{600.0}不應該永捨什麼?

   應該釋出什麼善的?而不應該使惡的被釋出?」

  「\twnr{人不應該布施自己}{x62},不應該永捨自己,

   應該釋出善的言語,而不應該使惡的被釋出。」



\sutta{79}{79}{旅程的資糧經}{https://agama.buddhason.org/SN/sn.php?keyword=1.79}
  「什麼\twnr{維繫}{851.0}旅程的資糧?什麼是財富的棲息處?

   什麼拖著人人繞轉?什麼是世間中難放棄的?

   當個個眾生在什麼上被繫縛時,如鳥之於以網?」

  「\twnr{信維繫旅程的資糧}{x63},幸運是財富的棲息處,

   欲求拖著人人繞轉,欲求是世間中難放棄的,

   當個個眾生被欲求繫縛時,如鳥之於以網。」



\sutta{80}{80}{光明經}{https://agama.buddhason.org/SN/sn.php?keyword=1.80}
  「什麼是世間中的光明?什麼是世間中的清醒者?

   什麼是在工作上的同事?什麼是其\twnr{行動範圍}{x64}?

   什麼養育懶惰者與不懶惰者,如母對子?

   他們依什麼生活:凡依止大地的生類生物?」 

  「慧是世間中的光明,念是世間中的清醒者,

   牛是在工作上的同事,田埂是其行動範圍。

   雨養育懶惰者與不懶惰者,如母對子,

   他們依雨生活:凡依止大地的生類生物[\suttaref{SN.1.54}]。」 



\sutta{81}{81}{無諍者經}{https://agama.buddhason.org/SN/sn.php?keyword=1.81}
  「這裡誰是世間中的無諍者?對誰來說已完成的不消失?

   這裡誰\twnr{遍知}{154.0}欲?誰的自由經常有?

   誰是父母兄弟,禮拜他的已住立者?

   這裡誰是剎帝利們\twnr{問訊}{46.0},卑下的出生者?」

  「這裡\twnr{沙門}{29.0}是世間中的無諍者,對沙門來說已完成的不消失,

   這裡沙門遍知欲,沙門的自由經常有。

   沙門是父母兄弟,禮拜他的已住立者,

   這裡沙門是剎帝利們問訊,卑下的出生者。」

  切斷品第八,其\twnr{攝頌}{35.0}:

  「切斷後、車、財產,雨、害怕、不衰退,

   統治權、想要、旅程的資糧,光明與無諍者。」

  諸天相應完成。





\page

\xiangying{2}{天子相應}
\pin{第一品}{1}{10}
\sutta{1}{1}{迦葉經第一}{https://agama.buddhason.org/SN/sn.php?keyword=2.1}
  \twnr{被我這麼聽聞}{1.0}:

  \twnr{有一次}{2.0},\twnr{世尊}{12.0}住在舍衛城祇樹林給孤獨園。那時,在夜已深時,容色絕佳的迦葉\twnr{天子}{282.0}使整個祇樹林發光後,去見世尊。抵達後,向世尊\twnr{問訊}{46.0}後,在一旁站立。在一旁站立的迦葉天子對世尊說這個:

  「世尊!你說明了\twnr{比丘}{31.0},但無對比丘的教誡。」

  「迦葉!那樣的話,就這情況請你說明。」

  「他應該學習被善說的,與\twnr{沙門}{29.0}的訓練,

   單獨座位與孤獨處,以及心的寂靜。」

  迦葉天子說這個,\twnr{大師}{145.0}是認可者。那時,迦葉天子[想]:「大師是我的認可者。」向世尊問訊、\twnr{作右繞}{47.0}後,就在那裡消失。



\sutta{2}{2}{迦葉經第二}{https://agama.buddhason.org/SN/sn.php?keyword=2.2}
  起源於舍衛城。

  在一旁站立的迦葉\twnr{天子}{282.0}在\twnr{世尊}{12.0}面前說這\twnr{偈頌}{281.0}:

  「\twnr{比丘}{31.0}應該是禪修者、心解脫者,如果希望\twnr{心的到達}{x65},

   而知道世間的生起衰滅後,善心者、不依止者有那個效益。」



\sutta{3}{3}{摩伽經}{https://agama.buddhason.org/SN/sn.php?keyword=2.3}
  起源於舍衛城。

  那時,在夜已深時,容色絕佳的摩伽\twnr{天子}{282.0}使整個祇樹林發光後,去見世尊。抵達後,向世尊\twnr{問訊}{46.0}後,在一旁站立。在一旁站立的摩伽天子以\twnr{偈頌}{281.0}對世尊說:

  「切斷什麼後\twnr{睡得安樂}{532.0}?切斷什麼後不憂愁?

   對哪一法的殺害,\twnr{喬達摩}{80.0}同意?」

  「切斷憤怒後睡得安樂,切斷憤怒後不憂愁,

   \twnr{阿修羅的征服者}{x66}!對端蜜,\twnr{而根毒之憤怒}{929.0}的殺害,

   聖者稱讚,因為切斷它後不憂愁。」[\suttaref{SN.1.71}/\suttaref{SN.11.21}]



\sutta{4}{4}{摩伽大經}{https://agama.buddhason.org/SN/sn.php?keyword=2.4}
  起源於舍衛城。

  在一旁站立的摩伽大\twnr{天子}{282.0}以\twnr{偈頌}{281.0}對世尊說:

  「世間中有多少燈火,以它們世間變明亮?

   我們來問\twnr{世尊}{12.0},我們應該如何知道它?」

  「世間中有四種燈火,第五種在這裡沒被發現,

   太陽在白天照亮,月亮在晚上發亮,

   而火在白天與晚上,到處變明亮,

   正覺者是照亮者中最勝的,那是無上的光明。」[\suttaref{SN.1.26}]



\sutta{5}{5}{大嘛哩經}{https://agama.buddhason.org/SN/sn.php?keyword=2.5}
  起源於舍衛城。

  那時,在夜已深時,容色絕佳的大嘛哩\twnr{天子}{282.0}使整個祇樹林發光後,去見世尊。抵達後,向世尊\twnr{問訊}{46.0}後,在一旁站立。在一旁站立的大嘛哩天子在世尊的面前說這\twnr{偈頌}{281.0}:

  「\twnr{這應該被婆羅門做}{x67}:以不疲倦的勤奮,

   以對諸欲的捨斷,因此他不希望有。」

  (世尊:「大嘛哩!」)

  「沒有婆羅門應該作的,因為婆羅門應該被作的已作。

   只要在河中未得到立足處,人以全部的肢體努力,

   但凡得到立足處後在陸地上住立者,他不努力因為他就是已到達彼岸者。

   大嘛哩!這是對婆羅門,對諸漏已滅盡者、明智者、禪修者的比喻,

   到達生死的結束後,他不努力因為他就是已到達彼岸者。」



\sutta{6}{6}{葛麼大經}{https://agama.buddhason.org/SN/sn.php?keyword=2.6}
  起源於舍衛城。

  在一旁站立的葛麼大\twnr{天子}{282.0}對\twnr{世尊}{12.0}說這個:

  「世尊!難做的,世尊!極難做的。」

  (世尊:「葛麼大!」)

  「他們也做難作的:戒等持、自我住立的\twnr{有學}{193.0},

   對進入無家者,是滿足者、帶來樂者。」

  「世尊!難得到的,即:滿足者。」

  (世尊:「葛麼大!」)

  「他們也得到難得到的:樂於心寂靜者,

   日與夜,他們的意(心)樂於\twnr{修習}{94.0}者。」

  「世尊!難集中(定)的,即:心。」

  (世尊:「葛麼大!」)

  「他們也集中難集中的,樂於諸根寂靜者,

   他們切斷死神之網後,葛麼大!聖者們走。」

  「世尊!道路是難走的、不平坦的。」

  「在難走的、不平坦的上,葛麼大!聖者們也走,

   非聖者們在難走的、不平坦的道路上,頭向下地倒下,

   聖者們的道路是平坦的,因為聖者們在不平坦上是平坦的。」



\sutta{7}{7}{般闍羅健達經}{https://agama.buddhason.org/SN/sn.php?keyword=2.7}
  起源於舍衛城。

  在一旁站立的般闍羅健達\twnr{天子}{282.0}在世尊的面前說這\twnr{偈頌}{281.0}:

  「廣智慧者確實發現了,\twnr{在障礙中的空間}{680.0}:

   那位覺了禪的佛陀,\twnr{獨處的人牛王}{786.1}、牟尼。」

  (世尊:「般闍羅健達!」)

  「即使在障礙中,他們發現為了到達涅槃的法:

   那些得到念者,那些完全善得定者。」



\sutta{8}{8}{德亞那經}{https://agama.buddhason.org/SN/sn.php?keyword=2.8}
  起源於舍衛城。

  那時,在夜已深時,容色絕佳的德亞那\twnr{天子}{282.0},之前為宗派始祖,使整個祇樹林發光後,去見\twnr{世尊}{12.0}。抵達後,向世尊\twnr{問訊}{46.0}後,在一旁站立。在一旁站立的德亞那天子在世尊的面前說這些\twnr{偈頌}{281.0}:

  「努力後切斷\twnr{流}{x68},婆羅門!請你除去諸欲,

   未捨斷諸欲後,\twnr{牟尼}{125.0}\twnr{不往生單一性}{x69}。

   \twnr{如果以應該做的他做}{x70},應該堅固地努力,

   因為鬆弛的\twnr{遊行者}{79.0},散播更多的塵垢。

   惡作不被作是比較好的,惡作之後他苦惱,

   善作被作是比較好的,作那個後不後悔。

   如茅草被惡捉取,就割手,

   惡執取\twnr{沙門身分}{328.0}者,拉到地獄。

   凡任何鬆弛的行為,以及凡被污染的誓戒(禁制),

   可疑的\twnr{梵行}{381.0},那個沒有大果。」

  德亞那天子說這個,說這個後,德亞那天子向世尊\twnr{問訊}{46.0}、\twnr{作右繞}{47.0}後,就在那裡消失。

  那時,那夜過後,世尊召喚\twnr{比丘}{31.0}們: 

  「比丘們!這夜,在夜已深時,容色絕佳的德亞那天子,之前為宗派始祖,使整個祇樹林發光後,來見我。抵達後,向我問訊後,在一旁站立。比丘們!在一旁站立的德亞那天子在我面前說這些偈頌:

  『努力後切斷流,婆羅門!請你除去諸欲, 

   未捨斷諸欲後,牟尼不往生單一性。 

   如果以應該做的他做,應該堅固地努力, 

   因為鬆弛的遊行者,散播更多的塵垢。 

   惡作不被作是比較好的,惡作之後他苦惱, 

   善作被作是比較好的,作那個後不後悔。 

   如茅草被惡捉取,就割手, 

   惡執取沙門身分者,拉到地獄。 

   凡任何鬆弛的行為,以及凡被污染的誓戒, 

   可疑的梵行,那個沒有大果。』

  德亞那天子說這個,說這個後,德亞那天子向我問訊、作右繞後,就在那裡消失。比丘們!請你們學習德亞那天子的偈頌,比丘們!請你們學得德亞那天子的偈頌,比丘們!請你們\twnr{憶持}{57.0}德亞那天子的偈頌,比丘們!德亞那天子的偈頌是伴隨利益的\twnr{梵行基礎的}{446.0}。」 



\sutta{9}{9}{月經}{https://agama.buddhason.org/SN/sn.php?keyword=2.9}
  起源於舍衛城。

  當時,月\twnr{天子}{282.0}被羅侯阿修羅王抓住。

  那時,月天子回憶著\twnr{世尊}{12.0}時說這\twnr{偈頌}{281.0}:

  「願禮敬你,佛陀!英雄!你於一切處已解脫,

   我有行道的阻礙,為了那個願你成為我的歸依。」

  那時,世尊以偈頌對羅侯阿修羅王說關於月天子的事:

  「月已歸依,\twnr{如來}{4.0}、\twnr{阿羅漢}{5.0},

   羅侯!請你釋放月,諸佛是世間的憐愍者。」

  那時,羅侯阿修羅王釋放月天子後,形色匆匆地去見毘摩質多阿修羅王。抵達後,驚怖、生起\twnr{身毛豎立}{152.0}地在一旁站立。在一旁站立的毘摩質多阿修羅王以偈頌對羅侯阿修羅王說:

  「羅侯!為何你就匆忙地,釋放月呢?

   形色驚慌地到來後,你為何就驚恐地站著?」

  「我的頭會七片地破裂,活著時不會得到安樂,

   在佛陀誦偈頌時,如果沒釋放月。」



\sutta{10}{10}{日經}{https://agama.buddhason.org/SN/sn.php?keyword=2.10}
  起源於舍衛城。

  當時,日\twnr{天子}{282.0}被羅侯阿修羅王抓住。

  那時,日天子回憶著\twnr{世尊}{12.0}時說這\twnr{偈頌}{281.0}:

  「願禮敬你,佛陀!英雄!你於一切處已解脫,

   我有行道的阻礙,為了那個願你成為我的歸依。」

  「日已歸依,\twnr{如來}{4.0}、\twnr{阿羅漢}{5.0},

   羅侯!請你釋放日,諸佛是世間的憐愍者。

   凡在黑暗闇黑中帶來光者:偉大火燄圓輪的日(毘盧遮那),

   羅侯!在虛空行走者不要吞掉日,羅侯!請你釋放\twnr{我的人}{x71}。」

  那時,羅侯阿修羅王釋放日天子後,形色匆匆地去見毘摩質多阿修羅王。抵達後,驚怖、生起\twnr{身毛豎立}{152.0}地在一旁站立。在一旁站立的毘摩質多阿修羅王以偈頌對羅侯阿修羅王說:

  「羅侯!為何你就匆忙地,釋放日呢?

   形色驚慌地到來後,你為何就驚恐地站著?」

  「我的頭會七片地破裂,活著時不會得到安樂,

   在佛陀誦偈頌時,如果沒釋放日。」

  第一品,其\twnr{攝頌}{35.0}:

  「二則迦葉與摩伽,摩伽大、大嘛哩、葛麼大,

   般闍羅健達、德亞那,以月日它們為十。」





\pin{給孤獨品}{11}{20}
\sutta{11}{11}{鄭地麼瑟經}{https://agama.buddhason.org/SN/sn.php?keyword=2.11}
  起源於舍衛城。

  那時,在夜已深時,容色絕佳的鄭地麼瑟\twnr{天子}{282.0}使整個祇樹林發光後,去見世尊。抵達後,向世尊\twnr{問訊}{46.0}後,在一旁站立。在一旁站立的鄭地麼瑟天子在世尊的面前說這\twnr{偈頌}{281.0}:

  「他們確實將走到平安,如鹿在無蚊的沼澤,

   具足諸禪後,成為專一的、明智的、具念的。」

  「他們確實將走到彼岸,如魚對切斷網後,

   具足諸禪後,成為不放逸的、\twnr{捨棄過失的}{x72}。」



\sutta{12}{12}{毘紐經}{https://agama.buddhason.org/SN/sn.php?keyword=2.12}
  在一旁站立的毘紐\twnr{天子}{282.0}在\twnr{世尊}{12.0}面前說這\twnr{偈頌}{281.0}:

  「那些人正是快樂的:侍奉\twnr{善逝}{8.0}後,

   在\twnr{喬達摩}{80.0}的教說中努力後,不放逸地隨學。」

  (世尊:「毘紐!」)

  「凡在教誡之句被我宣說時,禪修者們\twnr{隨學}{398.0},

   那些總是不放逸者,不會是到達死亡控制者。」



\sutta{13}{13}{長杖經}{https://agama.buddhason.org/SN/sn.php?keyword=2.13}
  \twnr{被我這麼聽聞}{1.0}:

  \twnr{有一次}{2.0},\twnr{世尊}{12.0}住在王舍城栗鼠飼養處的竹林中。

  那時,在夜已深時,容色絕佳的長杖\twnr{天子}{282.0}使整個竹林發光後,去見世尊。抵達後,向世尊\twnr{問訊}{46.0}後,在一旁站立。在一旁站立的長杖天子對世尊說這偈誦:

  「\twnr{比丘}{31.0}應該是禪修者、\twnr{心解脫}{16.0}者,如果希望\twnr{心的到達}{x73}。

   而知道世間的生起衰滅後,善心者、不依止者有那個效益。」



\sutta{14}{14}{難陀經}{https://agama.buddhason.org/SN/sn.php?keyword=2.14}
  在一旁站立的難陀\twnr{天子}{282.0}以\twnr{偈頌}{281.0}對\twnr{世尊}{12.0}說:

  「\twnr{喬達摩}{80.0}!我問你、廣慧者:世尊的智見是不被覆蓋的,

   什麼類型者他們說持戒者?什麼類型者他們說有慧者?

   什麼類型者超越苦後行動?什麼類型者諸天敬奉?」

  「凡持戒者、有慧者、\twnr{已自我修習}{658.0}者,得定者、愛好禪定者、有念者,

   一切愁已消失者、已捨斷者,諸漏已滅盡者、\twnr{持最後身者}{338.0},

   像那種類型者他們說持戒者,像那種類型者他們說有慧者。

   像那種類型者超越苦後行動,像那種類型者諸天敬奉。」



\sutta{15}{15}{檀香經}{https://agama.buddhason.org/SN/sn.php?keyword=2.15}
  在一旁站立的檀香\twnr{天子}{282.0}以\twnr{偈頌}{281.0}對\twnr{世尊}{12.0}說:

  「{如何}[誰]渡\twnr{暴流}{369.0},日夜不倦怠的?

   在無住立、無支持下,誰在深處不沈沒?」

  「於一切時戒具足者,有慧者、善得定者,

   活力已發動者、自我努力者,渡難渡的暴流。

   已從欲想脫離者,已超越色結者,

   \twnr{貪之歡喜已遍滅盡者}{x74},在深處不沈沒。」



\sutta{16}{16}{瓦須達多經}{https://agama.buddhason.org/SN/sn.php?keyword=2.16}
  在一旁站立的瓦須達多\twnr{天子}{282.0}在\twnr{世尊}{12.0}面前說這\twnr{偈頌}{281.0}:

  「像被矛觸擊,像在頭處被燃燒著,

   為了欲貪的捨斷,\twnr{比丘}{31.0}應該具念地\twnr{遊行}{61.0}。」

  「像被矛觸擊,像在頭處被燃燒著,

   為了\twnr{有身見}{93.1}的捨斷,比丘應該具念地遊行。」



\sutta{17}{17}{善梵經}{https://agama.buddhason.org/SN/sn.php?keyword=2.17}
  在一旁站立的善梵\twnr{天子}{282.0}以\twnr{偈頌}{281.0}對\twnr{世尊}{12.0}說:

  「這個心常被\twnr{驚嚇}{x75},這個意常被震驚,

   在未生起的苦難上,還有在已發生的上,

   如果有不被驚嚇的,被詢問時請你告訴我。」

  「非從覺[支]、\twnr{苦行}{x76}之外,非從根的\twnr{自制}{217.0}之外,

   非從一切的放捨之外,我看見有生命者的平安。」

  [世尊]說這個……(中略)就在那裡消失。



\sutta{18}{18}{葛古踏經}{https://agama.buddhason.org/SN/sn.php?keyword=2.18}
  \twnr{被我這麼聽聞}{1.0}:

  \twnr{有一次}{2.0},\twnr{世尊}{12.0}住在娑雞多城漆黑林的鹿園。

  那時,在夜已深時,容色絕佳的葛古踏\twnr{天子}{282.0}使整個漆黑林發光後,去見世尊。抵達後,向世尊\twnr{問訊}{46.0}後,在一旁站立。在一旁站立的葛古踏天子對世尊說這個:

  「\twnr{沙門}{29.0}!你歡喜嗎?」

  「\twnr{朋友}{201.0}!得到什麼後呢?」

  「沙門!那樣的話,你憂愁嗎?」

  「朋友!失去什麼呢?」

  「沙門!那樣的話,你既不歡喜也不憂愁嗎?」

  「是的,朋友!」

  「比丘!希望你是無痛苦的,希望歡喜不被發現,

   希望當你單獨坐時,不使不喜樂破壞。」

  「\twnr{夜叉}{126.0}!我確是無痛苦的,其次歡喜不被發現,

   其次當我單獨坐時,不使不喜樂破壞。」

  「比丘!你如何是無痛苦的?如何歡喜不被發現?

   如何當你單獨坐時,不使不喜樂破壞呢?」

  「已生起痛苦者確實有歡喜,已生起歡喜者確實有痛苦,

   [我是]無歡喜、無痛苦的比丘,朋友!請你這麼了解。」

  「終於我確實看見,般涅槃的婆羅門:

   無歡喜、無痛苦的比丘,已度脫世間中的執著。」



\sutta{19}{19}{鬱多羅經}{https://agama.buddhason.org/SN/sn.php?keyword=2.19}
  起源於王舍城。

  在一旁站立的鬱多羅\twnr{天子}{282.0}以這\twnr{偈頌}{281.0}對\twnr{世尊}{12.0}說:

  「生命被帶走、壽命是短的,被帶到老年者的救護所不存在,

   觀看著這在死亡上的恐怖,\twnr{應該作帶來樂的福德}{893.1}。」

  「生命被帶走、壽命是短的,被帶到老年者的救護所不存在,

   觀看著這在死亡上的恐怖,\twnr{期待寂靜者}{893.2}應該捨斷\twnr{世間物質}{593.0}。」[\suttaref{SN.1.3}]



\sutta{20}{20}{給孤獨經}{https://agama.buddhason.org/SN/sn.php?keyword=2.20}
  在一旁站立的給孤獨\twnr{天子}{282.0}在世尊的面前說這些\twnr{偈頌}{281.0}:

  「這裡確實是那個祇樹林,仙人\twnr{僧團}{375.0}經常來往的,

   被法王居住,有我的喜之產生。

   行為、明與法,戒、最上的活命,

   \twnr{不免一死的人}{600.0}們以這些變純淨,非以種姓或以財產。

   因此賢智的人,自己利益的看見者,

   應該如理檢擇法,這樣在那裡變成清淨。

   如舍利弗以慧,以戒以寂靜,

   凡即使到\twnr{彼岸}{226.0}的\twnr{比丘}{31.0},最高者會是這樣程度的。」[\suttaref{SN.1.48}]

  給孤獨天子說這個,說這個後,向世尊\twnr{問訊}{46.0}、\twnr{作右繞}{47.0}後,就在那裡消失。

  那時,那夜過後,世尊召喚比丘們:

  「比丘們!這夜,在夜已深時,容色絕佳的某位天子使整個祇樹林發光後,來見我。抵達後,向我問訊後,在一旁站立。比丘們!在一旁站立的那位天子在我面前說這些偈頌:

  『這裡確實是那個祇樹林,仙人僧團經常來往的, 

   被法王居住,有我的喜之產生。 

   行為、明與法,戒、最上的活命, 

   不免一死的人們以這些變純淨,非以種姓或以財產。 

   因此賢智的人,自己利益的看見者, 

   應該如理檢擇法,這樣在那裡變成清淨。 

   舍利弗就以慧,以戒與以寂靜, 

   凡即使到彼岸的比丘,最高者會是這樣程度的。』

  那位天子說這個,說這個後,向我問訊、作右繞後,就在那裡消失。」

  在這麼說時,\twnr{尊者}{200.0}阿難對世尊說這個:

  「\twnr{大德}{45.0}!他必定是給孤獨天子,\twnr{屋主}{103.0}給孤獨對尊者舍利弗是\twnr{極淨信者}{340.1}。」

  「阿難!\twnr{好}{44.0}!好!凡推論之所及,阿難!你已到達,阿難!他確實是給孤獨天子。」

  給孤獨品第二,其\twnr{攝頌}{35.0}:

  「鄭地麼瑟與毘紐,長杖、難陀,

   檀香、瓦須達多,善梵與葛古踏,

   第九鬱多羅所說,第十給孤獨。」





\pin{種種外道品}{21}{30}
\sutta{21}{21}{濕婆經}{https://agama.buddhason.org/SN/sn.php?keyword=2.21}
  \twnr{被我這麼聽聞}{1.0}:

  \twnr{有一次}{2.0},\twnr{世尊}{12.0}住在舍衛城祇樹林給孤獨園。

  那時,在夜已深時,容色絕佳的濕婆\twnr{天子}{282.0}使整個祇樹林發光後,去見世尊。抵達後,向世尊\twnr{問訊}{46.0}後,在一旁站立。在一旁站立的濕婆天子在世尊的面前說這些\twnr{偈頌}{281.0}:

  「只應該與善人們結交,應該與善人們作親密交往,

   了知善的正法後,成為更好的而非惡的。

   只應該與善人們結交,應該與善人們作親密交往,

   了知善的正法後,慧被得到而非從其他的。

   只應該與善人們結交,應該與善人們作親密交往,

   了知善的正法後,在憂愁中不憂愁。

   只應該與善人們結交,應該與善人們作親密交往,

   了知善的正法後,在親族中輝耀。

   只應該與善人們結交,應該與善人們作親密交往,

   了知善的正法後,眾生到\twnr{善趣}{112.0}。

   只應該與善人們結交,應該與善人們作親密交往,

   了知善的正法後,眾生經常地住立。」

  那時,世尊以偈頌回答濕婆天子:

  「只應該與善人們結交,應該與善人們作親密交往,

   了知善的正法後,從一切苦被釋放。」[\suttaref{SN.1.31}]



\sutta{22}{22}{安穩經}{https://agama.buddhason.org/SN/sn.php?keyword=2.22}
  在一旁站立的安穩\twnr{天子}{282.0}在\twnr{世尊}{12.0}面前說這些\twnr{偈頌}{281.0}:

  「劣智慧的愚者們實行:如以自己為敵人,

   做著惡業:凡有辛辣之果者。

   那是不好的所作業:凡做了後他後悔者,

   對那個淚流滿面地哭泣著:他承受果報。

   但那是好的所作業:凡做了後他不後悔者,

   對那個有喜悅的快意的:他承受果報。[」]

  「他應該首先就做:凡知道對自己有利益的,

   不以車夫的思惟:考量後明智者應該努力。

   如車夫,捨去平整平坦的大道後,

   登上不平坦的道路後,就沉思斷裂的車軸。

   像這樣離開法後,順從非法後,

   愚鈍者已到達死亡之口,如沉思斷裂的車軸。」



\sutta{23}{23}{悉梨經}{https://agama.buddhason.org/SN/sn.php?keyword=2.23}
  在一旁站立的悉梨\twnr{天子}{282.0}以\twnr{偈頌}{281.0}對\twnr{世尊}{12.0}說:

  「他們都歡喜食物:諸天與人們兩者,

   而還有什麼\twnr{夜叉}{126.0},不歡喜食物的?」

  「凡他們以信,以明淨心施與那個者,

   食物就服侍他,在這世間與下一個中。

   因此調伏慳吝後,征服垢穢者應該施與布施, 

   福德在來世中,是有生命者之所依。[\suttaref{SN.1.32}]」[\suttaref{SN.1.43}]

  「\twnr{不可思議}{206.0}啊,\twnr{大德}{45.0}!未曾有啊,大德!

  大德!這被世尊多麼善說:

  『凡他們以信,以明淨心施與那個者,

   食物就服侍他,在這世間與下一個中。

   因此調伏慳吝後,征服垢穢者應該施與布施, 

   福德在來世中,是有生命者之所依。』

  大德!從前,我是名叫悉梨國王的施與者、施主、布施的稱讚者,大德!那個我的布施在四個門被施與\twnr{沙門}{29.0}、\twnr{婆羅門}{17.0}、貧民、旅人、流浪者、乞丐。

  大德!那時,後宮婦女們來見我後,說這個:『陛下的布施被施與,無我們的布施被施與,如果我們也依陛下的支持施與布施、作福德,\twnr{那就好了}{44.0}!』大德!我想這個:『我是施與者、施主、布施的稱讚者,當她們說:「我們要施與布施。」時,我能說什麼?』大德!那第一個門我給後宮婦女們,在那裡,後宮婦女們的布施被施與,我的布施退減。

  大德!那時,隨侍的\twnr{剎帝利}{116.0}們來見我後,說這個:『陛下的布施被施與,後宮婦女們的布施被施與,無我們的布施被施與,如果我們也依陛下的支持施與布施、作福德,那就好了!』大德!我想這個:『我是施與者、施主、布施的稱讚者,當他們說:「我們要施與布施。」時,我能說什麼?』大德!那第二個門我給隨侍的剎帝利們,在那裡,隨侍的剎帝利們的布施被施與,我的布施退減。

  大德!那時,軍人們來見我後,說這個:『陛下的布施被施與,後宮婦女們的布施被施與,隨侍的剎帝利們的布施被施與,無我們的布施被施與,如果我們也依陛下的支持施與布施、作福德,那就好了!』大德!我想這個:『我是施與者、施主、布施的稱讚者,當他們說:「我們要施與布施。」時,我能說什麼?』大德!那第三個門我給軍人們,在那裡,軍人們的布施被施與,我的布施退減。

  大德!那時,婆羅門、\twnr{屋主}{103.0}們來見我後,說這個:『陛下的布施被施與,後宮婦女們的布施被施與,隨侍的剎帝利們的布施被施與,軍人們的布施被施與,無我們的布施被施與,如果我們也依陛下的支持施與布施、作福德,那就好了!』大德!我想這個:『我是施與者、施主、布施的稱讚者,當他們說:「我們要施與布施。」時,我能說什麼?』大德!那第四個門我給婆羅門、屋主們,在那裡,婆羅門、屋主們的布施被施與,我的布施退減。

  大德!那時,男子們來見我後,說這個:『現在無陛下的任何布施被施與。』大德!在這麼說時,我對那些男子們說這個:『那樣的話,我說,凡收益在外國被生產者,請你們從那裡使一半進入內宮,一半就在那裡請你們對沙門、婆羅門、貧民、旅人、流浪者、乞丐施與布施。』

  大德!我不到達這樣\twnr{長久}{51.0}所作福德、這樣長久所作諸善法的盡頭[,說]:『有這麼多福德。』或『有這麼多福德果報。』或『能使之這麼久地被存續於天界。』

  不可思議啊,大德!未曾有啊,大德!

  大德!這被世尊多麼善說:

  『凡他們以信,以明淨心施與那個者,

   食物就服侍他,在這世間與下一個中。

   因此調伏慳吝後,征服垢穢者應該施與布施, 

   福德在來世中,是有生命者之所依。』」



\sutta{24}{24}{額低葛勒經}{https://agama.buddhason.org/SN/sn.php?keyword=2.24}
  在一旁站立的額低葛勒\twnr{天子}{282.0}在\twnr{世尊}{12.0}面前說這\twnr{偈頌}{281.0}:

  「往生\twnr{無煩天}{x77},七位\twnr{比丘}{31.0}已解脫,

   貪瞋已滅盡,已度脫世間中的執著。」

  「而誰是那些度脫了泥沼,極難越過的死亡領域者?

   誰捨斷人的身體後,超越了\twnr{天軛}{x78}?」

  「優波迦與波羅揵荼,以及\twnr{補估沙地}{x79}他們三位,

   跋提雅與揵陀提婆,婆侯羅提與僧提雅,

   他們捨斷人的身體後,超越了天軛。」

  「你有善巧地說他們:魔網的捨斷者,

   他們了知誰的法後,切斷了有之繫縛?」

  「非從\twnr{世尊}{12.0}之外,非從你的教說之外,

   他們了知那位的法後,切斷了有之繫縛。

   於名與色處,被無餘地破壞,

   這裡他們了知這個法後,切斷了有之繫縛。」

  「你說了甚深的言語,難了知的、\twnr{極難覺醒的}{x44},

   你了知誰的法後,說像這樣的言語?」

  「在過去我是作陶器者,毘迦林加地方的陶匠,

   我是奉養父母者,迦葉佛的優婆塞。

   離婬欲法者,梵行無物質者,

   是你的同鄉,\twnr{是你在過去的同伴}{x80}。

   我是那位知道,這七位比丘已解脫,

   貪瞋已滅盡,已度脫世間中的執著。」

  「那時這是這樣,如你說,瑞祥兒!

   在過去你是作陶器者,毘迦林加地方的陶匠,

   你是奉養父母者,迦葉佛的優婆塞。

   離婬欲法者,梵行無物質者,

   是我的同鄉,是我在過去的同伴。」

  「這是這樣,是以前朋友的會合,

   \twnr{已自我修習}{658.0}的兩者,\twnr{持最後身者}{338.0}。」[\suttaref{SN.1.50}]



\sutta{25}{25}{遮堵經}{https://agama.buddhason.org/SN/sn.php?keyword=2.25}
  \twnr{被我這麼聽聞}{1.0}:

  \twnr{有一次}{2.0},掉舉的、傲慢的、浮躁的、饒舌的、言語散亂的、\twnr{念已忘失的}{216.0}、不正知的、不得定的、散亂心的、根不控制的眾多\twnr{比丘}{31.0}住在憍薩羅國喜馬拉雅山山腹的\twnr{林野}{142.0}小屋中。

  那時,遮堵\twnr{天子}{282.0}在十五\twnr{那個布薩日}{222.0}去見那些比丘。抵達後,以偈頌對那些比丘說:

  「之前比丘們是快樂生活者:\twnr{喬達摩}{80.0}的弟子們,

   \twnr{無欲求的食物尋求者}{x81},無欲求的住所[尋求者],

   知道世間中的無常性後,他們得到了苦的結束。

   置(作)自己為難養者後,如村落中的村長,

   一再吃後他們躺臥,在他人家中被迷昏頭。

   對\twnr{僧團}{375.0}作\twnr{合掌}{377.0}後,這裡我說某些人,

   他們是被拋棄者、無庇護者,就如那些亡者那樣地。

   凡住於放逸者,關於他們被我說,

   凡住於不放逸者,我要對他們作禮敬。[\suttaref{SN.9.13}]」



\sutta{26}{26}{赤馬經}{https://agama.buddhason.org/SN/sn.php?keyword=2.26}
  起源於舍衛城。

  在一旁站立的赤馬\twnr{天子}{282.0}對\twnr{世尊}{12.0}說這個:

  「\twnr{大德}{45.0}!不被生、不衰老、不死亡、不死沒、不再生之處,大德!那個世界的邊(終結)能以行走知道,或看到,或到達嗎?」

  「\twnr{朋友}{201.0}!不被生、不衰老、不死亡、不死沒、不再生之處,我不說:『世界的邊以行走能被知道、能被看見、能被到達。』」

  「不可思議啊,大德!\twnr{未曾有}{206.0}啊,大德!大德!這被世尊多麼善說:『朋友!不被生、不衰老、不死亡、不死沒、不再生之處,我不說:『世界的邊以行走能被知道、能被看見、能被到達。」』

  大德!從前,我是名為赤馬的仙人,伯渣的兒子,有神通者,空中行走者。大德!那個我的速度是這樣:猶如善學、熟練、已作訓練、擅長弓術者以輕的箭就少困難地橫向射穿棕櫚樹影。

  大德!那個我的歩幅是這樣:猶如從東海到西海。

  大德!那個我的像這樣欲求樣貌生起:『我將以行走到達世界的邊。』

  大德!具備像這樣的速度與那樣的歩幅,那個我除了吃、喝、嚼、嘗,除了大小便事務,除了睡覺、疲勞之排除外,一百年的壽命;一百年活著的;走一百年後,沒到達世界的邊,在途中就死了。

  不可思議啊,大德!未曾有啊,大德!大德!這被世尊多麼善說:『朋友!不被生、不衰老、不死亡、不死沒、不再生之處,我不說:『世界的邊以行走能被知道、能被看見、能被到達。」』」

  「朋友!然而,我說,不到達世界的邊後,沒有苦的作終結。朋友!而是,就在這\twnr{一噚之長}{x82},有想、有心的身體上,我\twnr{安立}{143.0}世界、世界\twnr{集}{67.0}、世界\twnr{滅}{68.0}、導向世界\twnr{滅道跡}{69.0}。」

  「任何時候世界的邊,不能被行走到達,

   而不到達世界的邊後,沒有使從苦的解脫。

   因此確實\twnr{世間知者}{9.0}、善慧者,到達世界邊者\twnr{梵行}{381.0}已完成,

   寂靜者知道世界的邊後,對這個世間與其它的\twnr{不希求}{x83}。」[\ccchref{AN.4.45}{https://agama.buddhason.org/AN/an.php?keyword=4.45}]



\sutta{27}{27}{難陀經}{https://agama.buddhason.org/SN/sn.php?keyword=2.27}
  在一旁站立的難陀\twnr{天子}{282.0}在\twnr{世尊}{12.0}面前說這些\twnr{偈頌}{281.0}:

  「時間流逝夜匆匆,種種年齡次第地拋棄[人],

   觀看著這死亡的恐怖,\twnr{應該作帶來樂的福德}{893.1}。」

  「時間流逝夜匆匆,種種年齡次第地拋棄,

   觀看著這死亡的恐怖,期待寂靜者應該捨棄\twnr{世間的誘惑物}{593.0}。」[\suttaref{SN.1.4}]



\sutta{28}{28}{廣歡喜經}{https://agama.buddhason.org/SN/sn.php?keyword=2.28}
  在一旁站立的廣歡喜\twnr{天子}{282.0}在\twnr{世尊}{12.0}面前說這些\twnr{偈頌}{281.0}:

  「\twnr{四輪與九門}{x84},被貪充滿、結縛,

   \twnr{污泥所生}{x85},大英雄!要如何\twnr{逃離}{x86}?」

  「\twnr{切斷皮帶與細繩}{x87},及惡欲求與貪後,

   連根拔除渴愛後,要這樣逃離。」[\suttaref{SN.1.29}]



\sutta{29}{29}{蘇尸摩經}{https://agama.buddhason.org/SN/sn.php?keyword=2.29}
  起源於舍衛城。

  那時,\twnr{尊者}{200.0}阿難去見世尊。抵達後,向世尊\twnr{問訊}{46.0}後,在一旁坐下。世尊對在一旁坐下的尊者阿難說這個:「阿難!你也喜歡舍利弗嗎?」

  「\twnr{大德}{45.0}!確實,對非愚者、惡性者、愚昧者、心顛倒者,誰不會喜歡尊者舍利弗?大德!尊者舍利弗是賢智者;大德!尊者舍利弗是大慧者;大德!尊者舍利弗是博慧者;大德!尊者舍利弗是捷慧者;大德!尊者舍利弗是速慧者;大德!尊者舍利弗是\twnr{利慧}{978.0}者;大德!尊者舍利弗是\twnr{洞察慧}{566.0}者[\suttaref{SN.8.7}, \ccchref{MN.111}{https://agama.buddhason.org/MN/dm.php?keyword=111}];大德!尊者舍利弗是少欲者;大德!尊者舍利弗是知足者;大德!尊者舍利弗是獨居者;大德!尊者舍利弗是離群眾者;大德!尊者舍利弗是活力已發動者;大德!尊者舍利弗是告誡他人者;大德!尊者舍利弗是被告誡的能容忍者;大德!尊者舍利弗是舉罪者;大德!尊者舍利弗是惡的呵責者。大德!確實,對非愚者、惡性者、愚昧者、心顛倒者,誰不會喜歡尊者舍利弗?」

  「這是這樣,阿難!這是這樣,阿難!

  阿難!確實,對非愚者、惡性者、愚昧者、心顛倒者,誰不會喜歡尊者舍利弗?阿難!尊者舍利弗是賢智者;阿難!尊者舍利弗是大慧者;阿難!尊者舍利弗是博慧者;阿難!尊者舍利弗是捷慧者;阿難!尊者舍利弗是速慧者;阿難!尊者舍利弗是利慧者;阿難!尊者舍利弗是洞察慧者;阿難!尊者舍利弗是少欲者;阿難!尊者舍利弗是知足者;阿難!尊者舍利弗是獨居者;阿難!尊者舍利弗是離群眾者;阿難!尊者舍利弗是活力已發動者;阿難!尊者舍利弗是解說者;阿難!尊者舍利弗是言語的容忍者;阿難!尊者舍利弗是舉罪者;阿難!尊者舍利弗是惡的呵責者。阿難!確實,對非愚者、惡性者、愚昧者、心顛倒者,誰不會喜歡尊者舍利弗?」

  那時,當對尊者舍利弗的稱讚被說時,蘇尸摩天子被大天子眾圍繞,去見世尊。抵達後,向世尊問訊後,在一旁站立。在一旁站立的蘇尸摩天子對世尊說這個:

  「這是這樣,世尊!這是這樣,\twnr{善逝}{8.0}!大德!確實,對非愚者、惡性者、愚昧者、心顛倒者,誰不會喜歡尊者舍利弗?大德!尊者舍利弗是賢智者;大德!尊者舍利弗是大慧者;大德!尊者舍利弗是博慧者;大德!尊者舍利弗是捷慧者;大德!尊者舍利弗是速慧者;大德!尊者舍利弗是利慧者;大德!尊者舍利弗是洞察慧者;大德!尊者舍利弗是少欲者;大德!尊者舍利弗是知足者;大德!尊者舍利弗是獨居者;大德!尊者舍利弗是離群眾者;大德!尊者舍利弗是活力已發動者;大德!尊者舍利弗是告誡他人者;大德!尊者舍利弗是被告誡的能容忍者;大德!尊者舍利弗是舉罪者;大德!尊者舍利弗是惡的呵責者。大德!確實,對非愚者、惡性者、愚昧者、心顛倒者,誰不會喜歡尊者舍利弗?

  大德!又,不論我到哪個天子眾中,就多聽到這樣的聲音:『尊者舍利弗是賢智者;尊者舍利弗是大慧者;尊者舍利弗是博慧者;尊者舍利弗是捷慧者;尊者舍利弗是速慧者;尊者舍利弗是利慧者;尊者舍利弗是洞察慧者;尊者舍利弗是少欲者;尊者舍利弗是知足者;尊者舍利弗是獨居者;尊者舍利弗是離群眾者;尊者舍利弗是活力已發動者;尊者舍利弗是告誡他人者;尊者舍利弗是被告誡的能容忍者;尊者舍利弗是舉罪者;尊者舍利弗是惡的呵責者。』大德!確實,對非愚者、惡性者、愚昧者、心顛倒者,誰不會喜歡尊者舍利弗?」

  那時,當對尊者舍利弗的稱讚被說時,蘇尸摩天子的天子眾成為悅意的、喜悅的、生起\twnr{喜}{428.0}-喜悅的,\twnr{出現種種輝耀的容色}{x88}。

  猶如美麗的、純正的、八個切割面的、作工細緻的、放在黃毛布上的琉璃寶珠,輝耀、照亮、照耀。同樣的,當對尊者舍利弗的稱讚被說時,蘇尸摩天子的天子眾成為悅意的、喜悅的、生起喜-喜悅的,出現種種輝耀的容色。

  猶如熟練的鍛工子非常有技術地的在鍛冶爐打造的、放在黃毛布上的閻浮河產的金之金幣,輝耀、照亮、照耀。同樣的,當對尊者舍利弗的稱讚被說時,蘇尸摩天子的天子眾成為悅意的、喜悅的、生起喜-喜悅的,出現種種輝耀的容色。

  猶如在秋天季節晴朗無雲的天空,在破曉時太白星輝耀、照亮、照耀。同樣的,當對尊者舍利弗的稱讚被說時,蘇尸摩天子的天子眾成為悅意的、喜悅的、生起喜-喜悅的,出現種種輝耀的容色。

  猶如在秋季節天晴朗無雲的天空,天空當太陽上升時,擊破一切來到天空的黑闇後,輝耀、照亮、照耀。同樣的,當對尊者舍利弗的稱讚被說時,蘇尸摩天子的天子眾成為悅意的、喜悅的、生起喜-喜悅的,出現種種輝耀的容色。

  那時,蘇尸摩天子關於尊者舍利弗在世尊的面前說這個\twnr{偈頌}{281.0}:

  「有名的『賢智者』,舍利弗是不憤怒者,

   少欲者、溫雅者、已調御者,仙人是被\twnr{大師}{145.0}帶來稱讚者。」

  那時,世尊關於尊者舍利弗以偈頌回答蘇尸摩天子:

  「有名的『賢智者』,舍利弗是不憤怒者,

   少欲者、溫雅者、已調御者,已善調御者\twnr{等待死時}{x89}。」



\sutta{30}{30}{種種外道弟子經}{https://agama.buddhason.org/SN/sn.php?keyword=2.30}
  \twnr{被我這麼聽聞}{1.0}:

  \twnr{有一次}{2.0},\twnr{世尊}{12.0}住在王舍城栗鼠飼養處的竹林中。

  那時,在夜已深時,容色絕佳的眾多種種外道弟子的\twnr{天子}{282.0}:阿色麼、色喝哩、尼迦、阿勾得迦、威格玻哩、麻那哇格米雅使整個竹林發光後,去見世尊。抵達後,向世尊\twnr{問訊}{46.0}後,在一旁站立。在一旁站立的阿色麼天子關於富蘭那迦葉在世尊的面前說這\twnr{偈頌}{281.0}:

  「這裡在砍殺中,在殺戮強奪中,

   迦葉不認為是惡的,又或是自己的福德,

   他確實告知安心,那位大師值得尊敬。」

  那時,色喝哩天子關於末迦利瞿舍羅在世尊的面前說這偈頌:

  「以\twnr{苦行與嫌惡}{867.0}成為自己善防護者,捨斷與人諍論的話後,

   離有不實者、實語者,\twnr{像這樣者}{632.0}確實不作惡。」

  那時,尼迦天子關於尼乾陀若提子在世尊的面前說這偈頌:

  「\twnr{比丘}{31.0}是謹慎者、審慎者,\twnr{四禁戒善防護者}{x90},

   他告知所見與所聞,確實不會是罪行者。」

  那時,阿勾得迦天子關於種種外道在世尊的面前說這偈頌:

  「浮陀迦[旃延]、葛低亞那、尼乾陀,以及凡這末迦利、富蘭那,

   群眾的大師已到達\twnr{沙門身分}{328.0},他們確實離善人不遠。」

  那時,威格玻哩天子以偈頌回答阿勾得迦天子:

  「卑微的狐狼以一起行,狐狼任何時候都不是等同獅子者,

   裸體、妄語的群眾之大師,引起懷疑的所行者與善者是不相同的。」

  那時,魔\twnr{波旬}{49.0}佔有威格玻哩天子後,在世尊的面前說這偈頌:

  「熱心投入於苦行與嫌惡者,遠離之被守護者,

   以及凡在色上住立者,是在天的世界中有歡喜者,

   他們確實正確地教誡:對其他世界\twnr{不免一死的人}{600.0}。」

  那時,世尊知道:「這是魔波旬。」後,以偈頌回應魔:

  「凡任何色在這裡或在他世,以及凡在空中美麗容色者,

   惡魔!那一切被你稱讚的,就像為了殺害魚而投擲的餌。」

  那時,麻那哇格米雅天子關於世尊在世尊的面前說這偈頌:

  「毘富羅在王舍城中,被說是最勝的山,

   謝達是喜馬拉雅山中最勝的,天空行走者中是太陽。

   大海是海洋中最勝的,而星宿中是月亮,

   包括天的世間中,佛陀被說是第一的。」

  種種外道品第三,其\twnr{攝頌}{35.0}:

  「濕婆、安穩與悉梨,額低葛勒、遮堵與赤馬,

   難陀、廣歡喜,蘇尸摩與種種外道它們為十。」

  天子相應完成。





\page

\xiangying{3}{憍薩羅相應}
\pin{第一品}{1}{10}
\sutta{1}{1}{年輕經}{https://agama.buddhason.org/SN/sn.php?keyword=3.1}
  \twnr{被我這麼聽聞}{1.0}:

  \twnr{有一次}{2.0},\twnr{世尊}{12.0}住在舍衛城祇樹林給孤獨園。

  那時,憍薩羅國波斯匿王去見世尊。抵達後,與世尊一起互相問候。交換應該被互相問候的友好交談後,在一旁坐下。在一旁坐下的憍薩羅國波斯匿王對世尊說這個:

  「\twnr{喬達摩}{80.0}\twnr{尊師}{203.0}也自稱:『我\twnr{已現正覺}{75.0}\twnr{無上遍正覺}{37.0}』嗎?」

  「大王!凡當正確說它時,應該說『已現正覺無上遍正覺』,那就是我,當正確說時,應該說。大王!因為我已現正覺無上遍正覺。」

  「喬達摩尊師!即使那些有團體的、有群眾的、群眾的老師、有名聲的知名開宗祖師、被眾人認定善的\twnr{沙門}{29.0}、\twnr{婆羅門}{17.0},即:富蘭那迦葉、末迦利瞿舍羅、尼乾陀若提子、散惹耶毘羅梨子、浮陀迦旃延、阿夷多翅舍欽婆羅,當他們也被我質問:『請你們自稱:「我已現正覺無上遍正覺。」』時,他們不自稱:『我已現正覺無上遍正覺。』然而,為何是以僅年輕出生且以新出家的喬達摩\twnr{尊師}{203.0}?」

  「大王!這四種不應該被輕侮為『年輕』;不應該被輕蔑為『年輕』,哪四種?

  大王!\twnr{剎帝利}{116.0}不應該被輕侮為『年輕』;不應該被輕蔑為『年輕』。

  大王!蛇不應該被輕侮為『年輕』;不應該被輕蔑為『年輕』。

  大王!火不應該被輕侮為『年輕』;不應該被輕蔑為『年輕』。

  大王!\twnr{比丘}{31.0}不應該被輕侮為『年輕』;不應該被輕蔑為『年輕』。」

  世尊說這個,說這個後,\twnr{善逝}{8.0}、\twnr{大師}{145.0}又更進一步說這個:

  「剎帝利是出生具足者,有良好出身者、有名聲者,

   不能輕視為『年輕』,人們不能輕蔑他。

   因為他可能是人們的王,剎帝利得到王位後,

   當他發怒時,會以國王的處罰對他激烈地執行,

   因此應該避開他:守護著自己的生命。

   如果在村落或林野,在那裡看見蛇,

   不能輕視為『年輕』,人們不能輕蔑牠。

   以各種不同的容色,凶猛的蛇行走,

   牠靠近後會咬愚者,男人與女人同時(不論男人或女人),

   因此應該避開他,守護著自己的生命。

   對可吃很多的火焰,對火、對黑色足跡者,

   不能輕視為『年輕』,人們不能輕蔑它。

   因為它得到燃料後,成為大火後,

   它靠近後會燒愚者,男人與女人同時,

   因此應該避開它,守護著自己的生命。

   凡火燒森林:火、黑色足跡者,

   幼芽在那裡被出生:如果經過諸日夜。

   而凡戒具足的比丘,以[戒]火燃燒,

   他的兒子、家畜,繼承人、財產他們不擁有,

   無子孫、無繼承人,他們成為[如]無根的棕櫚樹。

   因此賢智的人,自己利益的看見者,

   對蛇與火,與對有名聲的剎帝利,

   以及對戒具足的比丘,應該正確地行動。」

  在這麼說時,憍薩羅國波斯匿王對世尊說這個:

  「\twnr{大德}{45.0}!太偉大了,大德!太偉大了,大德!猶如扶正顛倒的,或揭開隱藏的,或告知迷路者的道路,或在黑暗中持燈火:『有眼者們看見諸色。』同樣的,法被世尊以種種\twnr{法門}{562.0}說明。大德!這個我\twnr{歸依}{284.0}世尊、法、\twnr{比丘僧團}{65.0},大德!請世尊記得我為\twnr{優婆塞}{98.0},從今天起\twnr{已終生歸依}{64.0}。」



\sutta{2}{2}{男子經}{https://agama.buddhason.org/SN/sn.php?keyword=3.2}
  起源於舍衛城。

  那時,憍薩羅國波斯匿王去見世尊。抵達後,向世尊\twnr{問訊}{46.0}後,在一旁坐下。在一旁坐下的憍薩羅國波斯匿王對世尊說這個:

  「\twnr{大德}{45.0}!當男子自身內的幾法生起時,生起不利、苦、不\twnr{安樂住}{156.0}呢?」

  「大王!當男子自身內的三法生起時,生起不利、苦、不安樂住,哪三個?大王!當男子自身內的貪欲法生起時,生起不利、苦、不安樂住;大王!當男子自身內的瞋恚法生起時,生起不利、苦、不安樂住;大王!當男子自身內的愚癡法生起時,生起不利、苦、不安樂住,大王!當男子自身內的這三法生起時,生起不利、苦、不安樂住。」

  說這個……(中略):

  「貪欲瞋恚與愚癡以及,惡心意的人,

   從自己生起的它們殺害,如有自己果實的竹。[≃\suttaref{SN.17.35}]」[\suttaref{SN.3.23}]



\sutta{3}{3}{老死經}{https://agama.buddhason.org/SN/sn.php?keyword=3.3}
  起源於舍衛城。

  在一旁坐下的憍薩羅國波斯匿王對\twnr{世尊}{12.0}說這個:

  「\twnr{大德}{45.0}!已生者有除了老死之外嗎?」

  「大王!已生者沒有除了老死之外。大王!又,凡即使那些大財富的\twnr{剎帝利}{116.0}:富有的、大富的、大財富的、多金銀的、多財產資具的、多財穀的,已生的他們也沒有除了老死之外;大王!又,凡即使那些大財富的\twnr{婆羅門}{17.0}:……(中略)大財富的\twnr{屋主}{103.0}:富有的、大富的、大財富的、多金銀的、多財產資具的、多財穀的,已生的他們也沒有除了老死之外;大王!又,凡即使那些漏已滅盡的、已完成的、\twnr{應該被作的已作的}{20.0}、負擔已卸的、\twnr{自己的利益已達成的}{189.0}、\twnr{有之結已滅盡的}{190.0}、以\twnr{究竟智}{191.0}解脫的\twnr{阿羅漢}{5.0}\twnr{比丘}{31.0},他們的這身體也是\twnr{破裂法}{x91}、捨棄法。」

  世尊說這個……(中略)。

  「國王漂亮的車變老了,而身體也來到衰老,

   但正法不來到衰老,確實善人與善人們宣說。」



\sutta{4}{4}{可愛經}{https://agama.buddhason.org/SN/sn.php?keyword=3.4}
  起源於舍衛城。

  在一旁坐下的憍薩羅國波斯匿王對\twnr{世尊}{12.0}說這個:

  「\twnr{大德}{45.0}!這裡,當我獨處、\twnr{獨坐}{92.0}時,這樣心的深思生起:『對誰來說自己是可愛的呢?對誰來說自己是不可愛的呢?』

  大德!我想這個:『凡任何以身行惡行、以語行惡行、以意行惡行者,對他們來說自己是不可愛的,即使他們會說這個:「自己是可愛的。」但對他們來說自己是不可愛的,那是什麼原因?因為凡不可愛者會對不可愛者作的,那個他們就以自己對自己作了,因此,對他們來說自己是不可愛的。

  凡任何以身行善行、以語行善行、以意行善行者,對他們來說自己是可愛的,即使他們會說這個:「自己是不可愛的。」但對他們來說自己是可愛的,那是什麼原因?因為凡可愛者會對可愛者作的,那個他們就以自己對自己作了,因此,對他們來說自己是可愛的。』」

  「這是這樣,大王!這是這樣,大王!

  大王!凡任何以身行惡行、以語行惡行、以意行惡行者,對他們來說自己是不可愛的,即使他們會說這個:「自己是可愛的。」但對他們來說自己是不可愛的,那是什麼原因?因為凡不可愛者會對不可愛者作的,那個他們就以自己對自己作了,因此,對他們來說自己是不可愛的。

  大王!凡任何以身行善行、以語行善行、以意行善行者,對他們來說自己是可愛的,即使他們會說這個:「自己是不可愛的。」但對他們來說自己是可愛的,那是什麼原因?因為凡可愛者會對可愛者作作的,那個他們就以自己對自己作了,因此,對他們來說自己是可愛的。」

  世尊說這個……(中略):

  「如果他知道自己是可愛的,不應該與那個惡的連結,

   因為那個是不容易獲得的:以惡作之作者對樂。

   當被死神進入時,當離開人的存在時,

   什麼確實是他自己的呢?而他取了什麼後走呢?

   什麼是他之隨行的,如影不離的呢?

   福德與惡兩者,凡\twnr{不免一死的人}{600.0}在這裡做的。

   那個確實是他自己的,而他取了這個後走,

   那個是他之隨行的,如影不離的。

   因此應該作善的:後世的財富,

   福德在來世中,是有生命者之所依。[\suttaref{SN.3.20}]」



\sutta{5}{5}{已自己守護經}{https://agama.buddhason.org/SN/sn.php?keyword=3.5}
  起源於舍衛城。

  在一旁坐下的憍薩羅國波斯匿王對\twnr{世尊}{12.0}說這個:

  「\twnr{大德}{45.0}!這裡,當我獨處、\twnr{獨坐}{92.0}時,這樣心的深思生起:『對誰來說自己已守護呢?對誰來說自己未守護呢?』

  大德!我想這個:『凡任何以身行惡行、以語行惡行、以意行惡行者,對他們來說自己未守護,即使他們象軍會守護,或馬軍會守護,或車軍會守護,或步兵軍會守護,但對他們來說自己未守護,那是什麼原因?因為這個守護是外部的,這個守護不是自身內的,因此,對他們來說自己未守護。

  凡任何以身行善行、以語行善行、以意行善行者,他們已守護自己,即使他們象軍不會守護,或馬軍不會守護,或車軍不會守護,或步兵軍不會守護,但對他們來說自己已守護,那是什麼原因?因為這個守護是自身內的,這個守護不是外部的,因此,對他們來說自己已守護。』」

  「這是這樣,大王!這是這樣,大王!

  大王!凡任何以身行惡行……(中略)對他們來說自己未守護,那是什麼原因?大王!因為這個守護是外部的,這個守護不是自身內的,因此,對他們來說自己未守護。

  大王!凡任何以身行善行、以語行善行、以意行善行者,他們已守護自己,即使他們象軍不會守護,或馬軍不會守護,或車軍不會守護,或步兵軍不會守護,但對他們來說自己已守護,那是什麼原因?因為這個守護是自身內的,這個守護不是外部的,因此,對他們來說自己已守護。」

  世尊說這個……(中略):

  「以身\twnr{自制}{217.0},\twnr{好}{44.0}!以語自制,好!

   以意自制,好!到處自制,好!

   到處已自制的、有羞恥的,被稱為『已守護』。」



\sutta{6}{6}{少經}{https://agama.buddhason.org/SN/sn.php?keyword=3.6}
  起源於舍衛城。

  在一旁坐下的憍薩羅國波斯匿王對\twnr{世尊}{12.0}說這個:

  「\twnr{大德}{45.0}!這裡,當我獨處、\twnr{獨坐}{92.0}時,這樣心的深思生起:『世間中那些眾生是少的:凡得到優越、優越的財富後他們不沈醉,又不放逸,又不來到在欲上貪求,又不在眾生上侵犯,而世間中這些眾生正是更多的:凡得到優越、優越的財富後他們沈醉,又放逸,又來到在欲上貪求,又在眾生上侵犯。』」

  「這是這樣,大王!這是這樣,大王!

  大王!世間中那些眾生是少的:凡得到優越、優越的財富後他們不沈醉,又不放逸,又不來到在欲上貪求,又不在眾生上侵犯,而世間中這些眾生正是更多的:凡得到優越、優越的財富後他們沈醉,又放逸,又來到在欲上貪求,又在眾生上侵犯。」

  世尊說這個……(中略):

  「在受用諸欲上貪著者,在欲上貪求者、迷昏頭者,

   他們越界不覺醒,如鹿對佈置陷阱的詐欺者,

   後果是苦澀的,因為那個的果報是惡的。[\suttaref{SN.3.7}]」



\sutta{7}{7}{法庭經}{https://agama.buddhason.org/SN/sn.php?keyword=3.7}
  起源於舍衛城。

  在一旁坐下的憍薩羅國波斯匿王對\twnr{世尊}{12.0}說這個:

  「\twnr{大德}{45.0}!這裡,當坐在法庭時,我看見富有的、大富的、大財富的、多金銀的、多財產資具的、多財穀的\twnr{剎帝利}{116.0}富翁、婆羅門富翁、\twnr{屋主}{103.0}富翁,欲之因、欲之因由、因為欲故意虛妄地說,大德!那個我想這個:『現在,我受夠了法庭,\twnr{現在賢面將被法庭被看到}{x92}。』」

  「(這是這樣,大王!這是這樣,大王!)

  大王!凡那些富有的、大富的、大財富的、多金銀的、多財產資具的、多財穀的剎帝利富翁、婆羅門富翁、屋主富翁,欲之因、欲之因由、因為欲故意虛妄地說,那對他們將有長久的不利、苦。」

  世尊說這個……(中略):

  「在受用諸欲上貪著者,在欲上貪求者、迷昏頭者,

   他們越界不覺醒,如鹿對佈置陷阱的詐欺者,

   後果是苦澀的,因為那個的果報是惡的。[\suttaref{SN.3.6}]」



\sutta{8}{8}{茉莉經}{https://agama.buddhason.org/SN/sn.php?keyword=3.8}
  起源於舍衛城。 

  當時,憍薩羅國波斯匿王與茉莉皇后一起到殊勝高樓上層。

  那時,憍薩羅國波斯匿王對茉莉皇后說這個:

  「茉莉!對你來說,有任何其他人比自己更可愛的嗎?」

  「大王!對我來說,沒有任何其他人比自己更可愛的,大王!而,對你來說,有任何其他人比自己更可愛的嗎?」

  「茉莉!對我來說,也沒有任何其他人比自己更可愛的。」

  那時,憍薩羅國波斯匿王從高樓下來後,去見世尊。抵達後,向世尊\twnr{問訊}{46.0}後,在一旁坐下。在一旁坐下的憍薩羅國波斯匿王對世尊說這個:

  「\twnr{大德}{45.0}!這裡,我與茉莉皇后一起到殊勝高樓上層,對茉莉皇后說這個:『茉莉!對你來說,有任何其他人比自己更可愛的嗎?』大德!在這麼說時,茉莉皇后對我說這個:『大王!對我來說,沒有任何其他人比自己更可愛的,大王!而,對你來說,有任何其他人比自己更可愛的嗎?』大德!在這麼說時,我對茉莉皇后說這個:『茉莉!對我來說,也沒有任何其他人比自己更可愛的。』」

  那時,世尊知道這件事後,那時候說這\twnr{偈頌}{281.0}[優陀那-\ccchref{Ud.41}{https://agama.buddhason.org/Ud/dm.php?keyword=41}]:

  「以心遊歷一切方位後,從未證得任何比自己更可愛的,

   像這樣這對其他人來說個別的自己是可愛的,因此\twnr{愛惜自己}{943.0}者不應該害他人。」



\sutta{9}{9}{牲祭經}{https://agama.buddhason.org/SN/sn.php?keyword=3.9}
  起源於舍衛城。

  當時,有一個\twnr{大牲祭}{656.0}已為憍薩羅國波斯匿王準備好了:五百頭\twnr{公牛}{657.0}、五百頭小公牛、五百頭小母牛、五百頭山羊、五百頭公羊,為了牲祭被帶來祭壇的諸柱子,凡他們是他的「奴僕」或「僕人」或「工人」者,他們也被懲罰威脅地、被恐懼威脅地、淚滿面地哭泣著作準備工作。

  那時,眾多\twnr{比丘}{31.0}午前時穿衣、拿起衣鉢後,\twnr{為了托鉢}{87.0}進入舍衛城。

  在舍衛城為了托鉢行走後,\twnr{餐後已從施食返回}{512.0},去見世尊。抵達後,向世尊\twnr{問訊}{46.0}後,在一旁坐下。在一旁坐下的那些比丘對世尊說這個:

  「\twnr{大德}{45.0}!這裡,有一個大牲祭已為憍薩羅國波斯匿王準備好了:五百頭公牛、五百頭小公牛、五百頭小母牛、五百頭山羊、五百頭公羊,為了牲祭被帶來祭壇的諸柱子,凡他們是他的『奴僕』或『僕人』或『工人』者,他們也被懲罰威脅地、被恐懼威脅地、淚滿面地哭泣著作準備工作。」

  那時,世尊知道這件事後,那時候說這些\twnr{偈頌}{281.0}:

  「馬祭、人祭,擲棒祭、酒祭、\twnr{無遮祭}{784.0},

   大殺害的大牲祭,那些是無大果的。

   山羊綿羊與牛,種種在該處被殺,

   那個牲祭,正行的大仙們不接近。

   但凡無勞苦的牲祭,隨家庭經常地祭祀,

   山羊綿羊與牛,種種在這裡不被殺,

   而那個牲祭,正行的大仙們接近。

   有智慧者應該祭祀這個,這個種牲祭是有大果的, 

   因為牲祭這個者,有更善的、非惡的, 

   且牲祭是廣大的,以及天神們歡喜。[\ccchref{AN.4.39}{https://agama.buddhason.org/AN/an.php?keyword=4.39}]」



\sutta{10}{10}{繫縛經}{https://agama.buddhason.org/SN/sn.php?keyword=3.10}
  當時,大群人被憍薩羅國波斯匿王使之繫縛:一些被繩子,一些被腳鐐,另一些被鎖鏈。

  那時,眾多\twnr{比丘}{31.0}午前時穿衣、拿起衣鉢後,\twnr{為了托鉢}{87.0}進入舍衛城。

  在舍衛城為了托鉢行走後,\twnr{餐後已從施食返回}{512.0},去見世尊。抵達後,向世尊\twnr{問訊}{46.0}後,在一旁坐下。在一旁坐下的那些比丘對世尊說這個:

  「\twnr{大德}{45.0}!這裡,大群人被憍薩羅國波斯匿王使之繫縛:一些被繩子,一些被腳鐐,另一些被鎖鏈。」

  那時,世尊知道這件事後,那時候說這些\twnr{偈頌}{281.0}:

  「明智者說那不是堅固的繫縛:凡鐵製的、木製的與燈心草,

   在寶石耳環上著迷者染著者,以及凡在妻兒上期待者,

   明智者說這是堅固的繫縛:\twnr{往下拉的}{x93}、徐緩的、難解脫的。

   也切斷這個後他們遊行,捨斷欲樂後無期待的。」

  第一品,其\twnr{攝頌}{35.0}:

  「年輕、男子、老,可愛、已自己守護,

   少、法庭,茉莉、牲祭、繫縛。」





\pin{第二品}{11}{20}
\sutta{11}{11}{七位結髮者經}{https://agama.buddhason.org/SN/sn.php?keyword=3.11}
  \twnr{有一次}{2.0},\twnr{世尊}{12.0}住在舍衛城東園鹿母講堂。

  當時,世尊傍晚時,從\twnr{獨坐}{92.0}出來,坐在外面的門屋。

  那時,憍薩羅國波斯匿王去見世尊。抵達後,向世尊\twnr{問訊}{46.0}後,在一旁坐下。

  當時,有七位\twnr{結髮者}{x94}、七位尼乾陀、七位裸行者、七位\twnr{一衣者}{x95}、七位\twnr{遊行者}{79.0},長的腋毛、指甲、體毛,拿起\twnr{一佉梨重}{864.0}荷物的扁擔後,從世尊不遠處越過。

  那時,憍薩羅國波斯匿王從座位起來後,置(作)上衣到一邊肩膀、右膝觸地後,向那七位結髮者、七位尼乾陀、七位裸行者、七位一衣者、七位遊行者合掌鞠躬後,告知名字三次:

  「\twnr{大德}{45.0}!我是憍薩羅國波斯匿王……(中略)大德!我是憍薩羅國波斯匿王。」

  那時,在那七位結髮者、七位尼乾陀、七位裸行者、七位一衣者、七位遊行者離開不久,憍薩羅國波斯匿王去見世尊。抵達後,向世尊問訊後,在一旁坐下。在一旁坐下的憍薩羅國波斯匿王對世尊說這個:

  「大德!凡世間中\twnr{阿羅漢}{5.0},或進入阿羅漢道,這些是它們中之一。」

  「大王!以在家受用欲的,以居住兒子擁擠床的,以享用迦尸的檀香的,以持有花環、香料、塗油的,以受用金銀的你,這是難知的:『這些是阿羅漢,或這些是進入阿羅漢道者。』

  大王!以共住,戒能被感知,且那是以長時間的,非短暫的;以\twnr{作意}{43.1}者,非以不作意者;以有慧者,非以\twnr{劣慧者}{384.0}。

  大王!以對談,純淨能被感知,且那是以長時間的,非短暫的;以作意者,非以不作意者;以有慧者,非以劣慧者。

  大王!在災禍中,強力能被感知,且那是以長時間的,非短暫的;以作意者,非以不作意者;以有慧者,非以劣慧者。

  大王!以討論,慧能被感知,且那是以長時間的,非短暫的;以作意者,非以不作意者;以有慧者,非以劣慧者。[\ccchref{AN.4.192}{https://agama.buddhason.org/AN/an.php?keyword=4.192}]」

  「\twnr{不可思議}{206.0}啊,大德!\twnr{未曾有}{206.0}啊,大德!

  大德!這被世尊多麼善說:『大王!以在家受用欲的,以居住兒子擁擠床的,以享用迦尸的檀香的,以持有花環、香料、塗油的,以受用金銀的你,這是難知的:「這些是阿羅漢,或這些是進入阿羅漢道者。」

  大王!以共住,戒能被感知,且那是以長時間的,非短暫的;以作意者,非以不作意者;以有慧者,非以劣慧者。

  大王!以交談,純淨能被感知,且那是以長時間的,非短暫的;以作意者,非以不作意者;以有慧者,非以劣慧者。

  大王!在災禍中,強力能被感知,且那是以長時間的,非短暫的;以作意者,非以不作意者;以有慧者,非以劣慧者。

  大王!以討論,慧能被感知,且那是以長時間的,非短暫的;以作意者,非以不作意者;以有慧者,非以劣慧者。』

  大德!這些男子是我的間諜、臥底者,從它國偵察後回來。先被他們偵察後,我將處置。大德!現在,他們除去那塵垢污穢後,被善浴、善塗油,整理好髮鬚,穿上白衣,他們將賦有、擁有\twnr{五種欲}{187.0}自娛。」[\ccchref{Ud.52}{https://agama.buddhason.org/Ud/dm.php?keyword=52}]

  那時,世尊已知這件事後,那時候說這些\twnr{偈頌}{281.0}:

  「人非以容色形相而容易知道,也非以短暫的看見而應該信賴,

   \twnr{因為以善制御的特徵}{x96},無制御者們在這世間中行。

   如黏土作的假耳環,如銅的半分錢被黃金包覆,

   \twnr{被附屬物包覆者們在世間中行}{x97},內在不清淨、外在亮麗。」



\sutta{12}{12}{五位國王經}{https://agama.buddhason.org/SN/sn.php?keyword=3.12}
  起源於舍衛城。

  當時,當波斯匿王為首之五位國王賦有、擁有\twnr{五種欲}{187.0}自娛時,這個談論中出現:「什麼是諸欲中第一的呢?」

  在那裡,某些說這個:「色是諸欲中第一的。」

  某些說這個:「聲音是諸欲中第一的。」

  某些說這個:「氣味是諸欲中第一的。」

  某些說這個:「味道是諸欲中第一的。」

  某些說這個:「\twnr{所觸}{220.2}是諸欲中第一的。」

  由於那些國王不能夠互相說服,那時,憍薩羅國波斯匿王對那些國王說這個:

  「\twnr{親愛的先生}{204.0}們!我們走,我們將去見\twnr{世尊}{12.0}。抵達後,讓我們問世尊這件事,世尊將為我們解說,讓我們依那樣\twnr{憶持}{57.0}它。」

  「是的,親愛的先生!」那些國王回答憍薩羅國波斯匿王。

  那時,以波斯匿王為首的五位國王去見世尊。抵達後,向世尊\twnr{問訊}{46.0}後,在一旁坐下。在一旁坐下的憍薩羅國波斯匿王對世尊說這個:

  「\twnr{大德}{45.0}!這裡,當我們五位國王賦有、擁有五種欲自娛時,這個談論中出現:『什麼是諸欲中第一的呢?』某些說這個:『色是諸欲中第一的。』某些說這個:『聲音是諸欲中第一的。』某些說這個:『氣味是諸欲中第一的。』某些說這個:『味道是諸欲中第一的。』某些說這個:『所觸是諸欲中第一的。』大德!什麼是諸欲中第一的呢?」

  「大王!我說:『合意的範圍是五欲中第一的。』

  大王!同樣那個色,對某些人是合意的,同樣那個色,對某些人是不合意的。而凡他以那些色為悅意的、完全滿意的(意向被充滿)者,他不欲求其它比那些更上的或更勝妙的色,對他來說,那些色是最高的;對他來說,那些色是無上的。

  大王!同樣那個聲音,對某些人是合意的,同樣那個聲音,對某些人是不合意的。而凡他以那些聲音為悅意的、完全滿意的者,他不欲求其它比那些更上的或更勝妙的聲音,對他來說,那些聲音是最高的;對他來說,那些聲音是無上的。

  大王!同樣那個氣味,對某些人是合意的,同樣那個氣味,對某些人是不合意的。而凡他以那些氣味為悅意的、完全滿意的者,他不欲求其它比那些更上的或更勝妙的聲音,對他來說,那些氣味是最高的;對他來說,那些氣味是無上的。

  大王!同樣那個味道,對某些人是合意的,同樣那個味道,對某些人是不合意的。而凡他以那些味道為悅意的、完全滿意的者,他不欲求其它比那些更上的或更勝妙的味道,對他來說,那些味道是最高的;對他來說,那些味道是無上的。

  大王!同樣那個所觸,對某些人是合意的,同樣那個所觸,對某些人是不合意的。而凡他以那些所觸為悅意的、完全滿意的者,他不欲求其它比那些更上的或更勝妙的所觸,對他來說,那些所觸是最高的;對他來說,那些所觸是無上的。」

  當時,栴檀額利迦\twnr{優婆塞}{98.0}坐在那些群眾中。那時,栴檀額利迦優婆塞從座位起來後,置(作)上衣到一邊肩膀,向世尊合掌鞠躬後,對世尊說這個:

  「世尊!它在我心中出現;\twnr{善逝}{8.0}!它在我心中出現。」

  「栴檀額利迦!請你說明。」世尊說。

  那時,栴檀額利迦優婆塞在世尊的面前以相應那樣的\twnr{偈頌}{281.0}大讚:

  「如芬芳的紅蓮、紅睡蓮,會在清晨香氣不散地盛開,

   請看\twnr{放光者}{662.2}、照耀者,像太陽在空中照亮著。」

  那時,那五位國王以五件上衣使栴檀額利迦優婆塞裹上。

  那時,栴檀額利迦優婆塞以那五件上衣使世尊裹上。



\sutta{13}{13}{一桶煮好的經}{https://agama.buddhason.org/SN/sn.php?keyword=3.13}
  起源於舍衛城。

  當時,憍薩羅國波斯匿王吃一桶煮好的飯。

  那時,憍薩羅國波斯匿王是已食者、喘息者,他去見世尊。抵達後,向世尊\twnr{問訊}{46.0}後,在一旁坐下。

  那時,世尊知道憍薩羅國波斯匿王是已食者、喘息者後,那時候,他說了這\twnr{偈頌}{281.0}:

  「當人常有念:在所獲得的食物上知量時,

   他的諸苦痛(受)變薄,緩慢地老、守護著壽命。」

  當時,\twnr{學生婆羅門}{102.0}善見站在憍薩羅國波斯匿王的後面。

  那時,憍薩羅國波斯匿王召喚學生婆羅門善見:

  「來!親愛的善見!你在世尊的面前學得這偈頌後,在我的食物到達時說,而我將持續每天給你(對你轉起)一百\twnr{迦哈玻那}{x98}的食物。」

  「是的,陛下!」學生婆羅門善見回答憍薩羅國波斯匿王後,在世尊的面前學得這偈頌後,在憍薩羅國波斯匿王的食物到達時確實地說:

  「當人常有念:在所獲得的食物上知量時,

   他的諸苦痛變薄,緩慢地老、守護著壽命。」

  那時,憍薩羅國波斯匿王次第地保持\twnr{最多一那利的飯量}{x99}。

  那時,過些時候,憍薩羅國波斯匿王肢體瘦了(被善切片),以手拍打著肢體,在那時候吟出這\twnr{優陀那}{184.0}:

  「那位世尊確實以兩方面的利益憐憫我:以當生的與來世的利益。」



\sutta{14}{14}{戰鬥經第一}{https://agama.buddhason.org/SN/sn.php?keyword=3.14}
  起源於舍衛城。

  那時,摩揭陀國阿闍世王韋提希子部署四種軍後前往迦尸,進攻憍薩羅國波斯匿王。

  憍薩羅國波斯匿王聽聞:

  「聽說摩揭陀國阿闍世王韋提希子部署四種軍後前往迦尸,已對我進攻。」

  那時,憍薩羅國波斯匿王部署四種軍後前往迦尸,向摩揭陀國阿闍世王韋提希子前進。

  那時,摩揭陀國阿闍世王韋提希子與憍薩羅國波斯匿王戰鬥。

  又,在那次戰鬥中,摩揭陀國阿闍世王韋提希子打敗了憍薩羅國波斯匿王,而敗北的憍薩羅國波斯匿王向自己的首都舍衛城前進。

  那時,眾多\twnr{比丘}{31.0}午前時穿衣、拿起衣鉢後,\twnr{為了托鉢}{87.0}進入舍衛城。

  在舍衛城為了托鉢行走後,\twnr{餐後已從施食返回}{512.0},去見世尊。抵達後,向世尊\twnr{問訊}{46.0}後,在一旁坐下。在一旁坐下的那些比丘對世尊說這個:

  「\twnr{大德}{45.0}!這裡,摩揭陀國阿闍世王韋提希子部署四種軍後前往迦尸,進攻憍薩羅國波斯匿王。大德!憍薩羅國波斯匿王聽聞:『聽說摩揭陀國阿闍世王韋提希子部署四種軍後前往迦尸,已對我進攻。』大德!那時,憍薩羅國波斯匿王部署四種軍後前往迦尸,向摩揭陀國阿闍世王韋提希子前進。大德!那時,摩揭陀國阿闍世王韋提希子與憍薩羅國波斯匿王戰鬥,大德!又,在那次戰鬥中,摩揭陀國阿闍世王韋提希子打敗了憍薩羅國波斯匿王,大德!而敗北的憍薩羅國波斯匿王向自己的首都舍衛城前進。」

  「比丘們!摩揭陀國阿闍世王韋提希子有惡的朋友、惡的同伴、惡的親密朋友,而憍薩羅國波斯匿王有善的朋友、善的同伴、善的親密朋友。

  比丘們!今天,敗北的憍薩羅國波斯匿王這晚睡不好了。」

  世尊說這個……(中略):

  「勝利者招致怨恨,敗北者\twnr{睡不好}{874.0},

   \twnr{寂靜者睡得安樂}{x100}:捨斷勝敗後。」



\sutta{15}{15}{戰鬥經第二}{https://agama.buddhason.org/SN/sn.php?keyword=3.15}
  那時,摩揭陀國阿闍世王韋提希子部署四種軍後前往迦尸,進攻憍薩羅國波斯匿王。

  憍薩羅國波斯匿王聽聞:

  「聽說摩揭陀國阿闍世王韋提希子部署四種軍後前往迦尸,已對我進攻。」

  那時,憍薩羅國波斯匿王部署四種軍後前往迦尸,向摩揭陀國阿闍世王韋提希子前進。

  那時,摩揭陀國阿闍世王韋提希子與憍薩羅國波斯匿王戰鬥。

  又,在那次戰鬥中,憍薩羅國波斯匿王打敗了摩揭陀國阿闍世王韋提希子,並且活抓他。

  那時,憍薩羅國波斯匿王想這個:「即使當我沒違犯這位摩揭陀國阿闍世王韋提希子時,他違犯[我],但他仍是我的外甥,讓我佔據摩揭陀國阿闍世王韋提希子的所有象軍、佔據所有的馬軍、佔據所有的車軍、佔據所有的步軍後,只讓他能活著撤退。」

  那時,憍薩羅國波斯匿王佔據摩揭陀國阿闍世王韋提希子的所有象軍、佔據所有的馬軍、佔據所有的車軍、佔據所有的步軍後,只讓他活著撤退。

  那時,眾多\twnr{比丘}{31.0}午前時穿衣、拿起衣鉢後,\twnr{為了托鉢}{87.0}進入舍衛城。

  在舍衛城為了托鉢行走後,\twnr{餐後已從施食返回}{512.0},去見世尊。抵達後,向世尊\twnr{問訊}{46.0}後,在一旁坐下。在一旁坐下的那些比丘對世尊說這個:

  「\twnr{大德}{45.0}!這裡,摩揭陀國阿闍世王韋提希子部署四種軍後前往迦尸,進攻憍薩羅國波斯匿王。大德!憍薩羅國波斯匿王聽聞:『聽說摩揭陀國阿闍世王韋提希子部署四種軍後前往迦尸,已對我進攻。』大德!那時,憍薩羅國波斯匿王部署四種軍後前往迦尸,向摩揭陀國阿闍世王韋提希子前進。大德!那時,摩揭陀國阿闍世王韋提希子與憍薩羅國波斯匿王戰鬥。大德!又,在那次戰鬥中,憍薩羅國波斯匿王打敗了摩揭陀國阿闍世王韋提希子,並且活抓他。大德!那時,憍薩羅國波斯匿王想這個:『即使當我沒違犯這位摩揭陀國阿闍世王韋提希子時,他違犯[我],但他仍是我的外甥,讓我佔據摩揭陀國阿闍世王韋提希子的所有象軍、佔據所有的馬軍、佔據所有的車軍、佔據所有的步軍後,只讓他能活著撤退。』大德!那時,憍薩羅國波斯匿王佔據摩揭陀國阿闍世王韋提希子的所有象軍、佔據所有的馬軍、佔據所有的車軍、佔據所有的步軍後,只讓他活著撤退。」

  那時,世尊知道這件事後,那時候說這些\twnr{偈頌}{281.0}:

  「男子就一直掠奪,只要有利於他,

   而當其他人掠奪時,那位掠奪者被掠奪。

   愚者認為確實合理,只要未被惡的折磨,

   而當被惡的折磨時,那時他遭受苦。

   殺害者得到殺害,征服者得到征服,

   辱罵者對辱罵,以及激怒者對激怒,

   那時以業的輪迴,那位掠奪者被掠奪。」



\sutta{16}{16}{茉莉經}{https://agama.buddhason.org/SN/sn.php?keyword=3.16}
  起源於舍衛城。 

  那時,憍薩羅國波斯匿王去見\twnr{世尊}{12.0}。抵達後,向世尊\twnr{問訊}{46.0}後,在一旁坐下。

  那時,某位男子去見憍薩羅國波斯匿王。抵達後,在憍薩羅國波斯匿王的耳朵處通知:「陛下!茉莉皇后已生女兒。」

  在這麼說時,憍薩羅國波斯匿王成為不悅意的。

  那時,世尊知道憍薩羅國波斯匿王成為不悅意的後,那時候說這些\twnr{偈頌}{281.0}:

  「某些的女性確實,較男性好,王!

   有智慧的、持戒的,婆婆當天神、對丈夫貞淑。

   凡被她生的男性,成為英雄、諸方之主,

   像那樣\twnr{好妻子的}{x101}兒子,他教誡王國。」



\sutta{17}{17}{不放逸經}{https://agama.buddhason.org/SN/sn.php?keyword=3.17}
  起源於舍衛城。

  在一旁坐下。在一旁坐下的憍薩羅國波斯匿王對\twnr{世尊}{12.0}說這個:「\twnr{大德}{45.0}!有一法,凡達到後住立二者的利益:當生的連同後世的利益嗎?」「大王!有一法,凡達到後住立二者的利益:當生的連同後世的利益。」

  「大德!那麼,哪一法,凡達到後住立二者的利益:當生的連同後世的利益?」「大王!\twnr{不放逸}{x102}是一法,凡達到後住立二者的利益:當生的連同後世的利益。大王!猶如凡任何叢林生物的足跡類者,那些全都在象的足跡中走到容納,象的足跡被告知為它們中第一的,即:以大的狀態。同樣的,大王!不放逸是一法,達到後住立二者的利益:當生的連同後世的利益。」[≃\ccchref{AN.6.53}{https://agama.buddhason.org/AN/an.php?keyword=6.53}]

  世尊說這個……(中略):

  「對壽命、無病、美貌,對天界、高貴家系的狀態,

   對喜樂以欲求者:廣大的、更多更多的,

   賢智者稱讚,在福德行為上的不放逸。

   不放逸的賢智者,擁有二種利益:

   凡在當生中的利益,以及凡後世的利益,

   \twnr{現觀利益的堅固者}{698.0},被稱為『賢智者』。[\ccchref{AN.5.43}{https://agama.buddhason.org/AN/an.php?keyword=5.43}]」



\sutta{18}{18}{善友經}{https://agama.buddhason.org/SN/sn.php?keyword=3.18}
  起源於舍衛城。

  在一旁坐下的憍薩羅國波斯匿王對\twnr{世尊}{12.0}說這個:

  「\twnr{大德}{45.0}!這裡,當我獨處、\twnr{獨坐}{92.0}時,這樣心的深思生起:『法被世尊善說,而那是對有善的朋友、善的同伴、善的親密朋友者,不是對有惡的朋友、惡的同伴、惡的親密朋友者。』」

  「這是這樣,大王!這是這樣,大王!

  大王!法被我善說,而那是對有善的朋友、善的同伴、善的親密朋友者,不是對有惡的朋友、惡的同伴、惡的親密朋友者。

  大王!有這一次,我住在釋迦國,名叫那軋拉迦的釋迦族人城鎮,大王!那時,阿難\twnr{比丘}{31.0}來見我。抵達後,向我\twnr{問訊}{46.0}後,在一旁坐下,大王!在一旁坐下的阿難比丘對我說這個:『大德!\twnr{這是梵行的一半}{747.0},即:\twnr{善的朋友之誼}{321.0}、善的同伴之誼、善的親密朋友之誼。』

  大王!在這麼說時,我對阿難比丘說這個:『阿難!不要這樣[說-\ccchref{DN.15}{https://agama.buddhason.org/DN/dm.php?keyword=15}],阿難!不要這樣[說],阿難!這就是梵行的全部,即:善的朋友之誼、善的同伴之誼、善的親密朋友之誼。阿難!有善的朋友、善的同伴、善的親密朋友比丘的這個能被預期:他必將\twnr{修習}{94.0}\twnr{八支聖道}{525.0}、必將\twnr{多作}{95.0}八支聖道。

  阿難!而有善的朋友、善的同伴、善的親密朋友的比丘,如何修習八支聖道、多作八支聖道?阿難!這裡,比丘\twnr{依止遠離}{322.0}、依止離貪、依止滅、\twnr{捨棄的成熟}{221.0}修習正見;依止遠離……(中略)修習正志……(中略)修習正語……(中略)修習正業……(中略)修習正命……(中略)修習正精進……(中略)修習正念;依止遠離、依止離貪、依止滅、捨棄的成熟修習正定。阿難!有善的朋友、善的同伴、善的親密朋友的比丘這樣修習八支聖道、多作八支聖道。

  阿難!其次,以這個法門,這也能被知道:關於這就是梵行的全部,即:善的朋友之誼、善的同伴之誼、善的親密朋友之誼。阿難!由於善友的我,\twnr{生法}{587.0}的眾生從生被釋放;老法的眾生從老被釋放;死法的眾生從死被釋放;愁悲苦憂\twnr{絕望}{342.0}法的眾生從愁悲苦憂絕望被釋放。阿難!以這個法門,這也能被知道:關於這就是梵行的全部,即:善的朋友之誼、善的同伴之誼、善的親密朋友之誼。[\suttaref{SN.45.2}]』

  大王!因此,在這裡,應該被你這麼學:『我要成為有善的朋友、善的同伴、善的親密朋友者。』大王!應該被你這麼學。

  大王!當你有善的朋友、善的同伴、善的親密朋友時,這一法依止後應該被留下:在善法上的不放逸。

  大王!當你住於不放逸的時,依止不放逸的後,你的宮女隨從必將想這個:『國王住於不放逸,依止不放逸,來吧!讓我們也住於不放逸,依止不放逸。』

  大王!當你住於不放逸的時,依止不放逸的後,你的\twnr{剎帝利}{116.0}隨從必將想這個:『國王住於不放逸,依止不放逸,來吧!讓我們也住於不放逸,依止不放逸。』

  大王!當你住於不放逸的時,依止不放逸的後,你的軍隊必將想這個:『國王住於不放逸,依止不放逸,來吧!讓我們也住於不放逸,依止不放逸。』

  大王!當你住於不放逸的時,依止不放逸的後,你的城鎮與地方人們必將想這個:『國王住於不放逸,依止不放逸,來吧!讓我們也住於不放逸,依止不放逸。』

  大王!當你住於不放逸的時,依止不放逸的後,你自己必將被保護、被守護;宮女也必將被保護、被守護;寶庫與藏庫之家屋也必將被保護、被守護。」[\suttaref{SN.45.2}]

  世尊說這個……(中略):

  「對財富以欲求者:廣大的、更多更多的,

   賢智者稱讚,在福德行為上的不放逸。

   不放逸的賢智者,擁有二種利益:

   凡在當生中的利益,以及凡後世的利益,

   \twnr{現觀利益的堅固者}{698.0},被稱為『賢智者』。」



\sutta{19}{19}{無子者經第一}{https://agama.buddhason.org/SN/sn.php?keyword=3.19}
  起源於舍衛城。

  那時,憍薩羅國波斯匿王中午去見世尊。抵達後,向世尊\twnr{問訊}{46.0}後,在一旁坐下。世尊對在一旁坐下的憍薩羅國波斯匿王說這個:「那麼,大王!你中午從哪裡來呢?」

  「\twnr{大德}{45.0}!這裡,舍衛城中經營錢莊的\twnr{屋主}{103.0}死了,我搬走那位無子者的財產到國王後宮後而來,大德!黃金就有八百萬,更不用說白銀。大德!又,那位經營錢莊的屋主之食物受用是這樣的:吃碎米飯拌酸粥,衣服受用是這樣的:穿三片粗麻布的衣服,車乘受用是這樣的:乘以裝著樹葉傘的老舊小車。」

  「這是這樣,大王!這是這樣,大王!

  大王!非善人得到廣大的財富後,既不使自己快樂、喜悅,也不使父母快樂、喜悅,不使妻兒快樂、喜悅,不使奴僕、工人、傭人快樂、喜悅,不使朋友、同事快樂、喜悅,不在\twnr{沙門}{29.0}、\twnr{婆羅門}{17.0}們上使高的、導致生天的、安樂果報的、轉起天界的供養建立,當他的那些財富不這樣正當使用時,國王們拿走,或盜賊們拿走,或火燃燒,或水流走,或不可愛的繼承者們拿走,大王!那確實是這樣:當不正當使用時,那些財富走到遍盡,而非使用。

  大王!猶如在無人處有清澈水的、悅意水的、冷水的、透明水的、美麗堤岸的、能被喜樂的蓮花池,人們既不拿走它,也不喝飲,也不沐浴,也不如需要做,大王!那確實是這樣:當不正當使用時,那個水會走到遍盡,而非使用。同樣的,大王!當非善人得到廣大的財富後,既不使自己快樂、喜悅,也不使父母快樂、喜悅,不使妻兒快樂、喜悅,不使奴僕、工人、傭人快樂、喜悅,不使朋友、同事快樂、喜悅,不在沙門婆羅門們上使高的、導致生天的、安樂果報的、轉起天界的供養建立,當他的那些財富不這樣正當使用時,國王們拿走,或盜賊們拿走,或火燃燒,或水流走,或不可愛的繼承者們拿走,大王!那確實是這樣:當不正當使用時,那些財富走到遍盡,而非使用。

  大王!而善人得到廣大的財富後,使自己快樂、喜悅,也使父母快樂、喜悅,使妻兒快樂、喜悅,使奴僕、工人、傭人快樂、喜悅,使朋友、同事快樂、喜悅,在沙門婆羅門們上使高的、導致生天的、安樂果報的、轉起天界的供養建立,當他的那些財富這樣正當使用時,國王們不拿走,盜賊們不拿走,火不燃燒,水不流走,不可愛的繼承者們不拿走,大王!像這樣,當正當使用時,那些財富走到使用,而非遍盡。

  大王!猶如在村落或城鎮不遠處有清澈水的、悅意水的、冷水的、透明水的、美麗堤岸的、能被喜樂的蓮花池,而人們拿走那個水,也喝飲,也沐浴,也如需要做,大王!像這樣,當正當使用時,那個水會走到使用,而非遍盡。同樣的,大王!當善人得到廣大的財富後,使自己快樂、喜悅,也使父母快樂、喜悅,使妻兒快樂、喜悅,使奴僕、工人、傭人快樂、喜悅,使朋友、同事快樂、喜悅,在沙門婆羅門們上使高的、導致生天的、安樂果報的、轉起天界的供養建立,當他的那些財富這樣正當使用時,國王們不拿走,盜賊們不拿走,火不燃燒,水不流走,不可愛的繼承者們不拿走,大王!像這樣,當正當使用時,那些財富走到使用,而非遍盡。」

  「如冷水在無人處,當它不能被喝飲時走向乾涸,

   這樣邪惡人得到財物後,既不自己受用也不施與。

   明智者與智者得到財富後,他受用與成為盡義務者,

   那位\twnr{人牛王}{786.0}養育親屬眾後,不被責備地到達天界處。」



\sutta{20}{20}{無子者經第二}{https://agama.buddhason.org/SN/sn.php?keyword=3.20}
  那時,憍薩羅國波斯匿王中午去見\twnr{世尊}{12.0}[……(中略)]。

  世尊對在一旁坐下的憍薩羅國波斯匿王說這個:

  「那麼,大王!你中午從哪裡來呢?」

  「\twnr{大德}{45.0}!這裡,舍衛城中經營錢莊的\twnr{屋主}{103.0}死了,我搬走那位無子者的財產到國王後宮後而來,大德!黃金就有一千萬,更不用說白銀。大德!又,那位經營錢莊的屋主之食物受用是這樣的:吃碎米飯拌酸粥;衣服受用是這樣的:穿三片粗麻布的衣服,車乘受用是這樣的:乘以裝著樹葉傘的老舊小車。」

  「這是這樣,大王!這是這樣,大王!

  大王!從前,那位經營錢莊的屋主以\twnr{施食}{196.0}使名叫多迦羅尸棄的\twnr{辟支佛}{243.0}受取:說『請你們對\twnr{沙門}{29.0}施與食物』後,從座位起來後離開。

  但,給了後,後來成為後悔者:『這是恩惠,如果奴僕或工人吃施食。』而且他為財產的因素奪取兄弟獨子的生命。

  大王!凡那位經營錢莊的屋主以施食使名叫多迦羅尸棄的辟支佛受取,就以那個業的果報,他往生\twnr{善趣}{112.0}、天界七回,以正是那個業的殘餘果報,他就在這舍衛城作經營錢莊者七回。大王!凡那位經營錢莊的屋主施與後,後來他成為後悔者:『這是恩惠,如果奴僕或工人吃施食。』就以那個業的果報,他的心不傾向於上妙食物的受用,他的心不傾向於上妙衣服的受用,他的心不傾向於上妙車乘的受用,他的心不傾向於上妙\twnr{五種欲}{187.0}的受用。大王!而且凡那位經營錢莊的屋主為財產的因素奪取兄弟獨子的生命,就以那個業的果報,他在地獄受折磨好幾年、好幾百年、好幾千年、好幾十萬年,以正是那個業的殘餘果報,使這第七次無子者的財產進入國王的藏庫。

  大王!而那位經營錢莊的屋主以前的福德已滅盡,且沒累積新福德,大王!又,經營錢莊的屋主現在在\twnr{大叫喚地獄}{x103}受折磨。」

  「大德!像這樣,經營錢莊的屋主已往生大叫喚地獄了嗎?」

  「是的,大王!經營錢莊的屋主已往生大叫喚地獄了。」

  世尊說這個……(中略):

  「穀物、財物、銀、金,或甚至凡有任何所有物者,

   奴僕、工人、僕人,以及凡依他生活者,

   一切都沒拿後他將被[迫]走,一切都是導向捨棄的。

   而凡他以身作,以語或以心,

   那個確實是他自己的,而他拿這個後走,

   那個是他的隨行者,如影不離。

   因此應該作善的:後世的財富,

   福德在來世中,是有生命者之所依。[\suttaref{SN.3.4}]」

  第二品,其\twnr{攝頌}{35.0}:

  「結髮者、五位國王,以及一桶煮好的飯,

   戰鬥二說,茉莉與不放逸二則,

   無子者二說,以那個被稱為品。」





\pin{第三品}{21}{25}
\sutta{21}{21}{人經}{https://agama.buddhason.org/SN/sn.php?keyword=3.21}
  起源於舍衛城。

  那時,憍薩羅國波斯匿王去見\twnr{世尊}{12.0}。抵達後,向世尊\twnr{問訊}{46.0}後,在一旁坐下。世尊對在一旁坐下的憍薩羅國波斯匿王說這個:

  「大王!有這四種現在的世間中存在的人,哪四種?闇黑到闇黑者、闇黑到光明者、光明到闇黑者、光明到光明者。

  大王!而怎樣一個人是闇黑到闇黑者?大王!這裡,某人被再生在卑賤家:在\twnr{旃陀羅}{120.0}家,或在竹匠家,或在獵人家,或在車匠家,或在清垃圾者家,在貧窮處,在少食物、飲料、飲食處,在生活困難處,於該處衣食被困難地得到。他是醜陋者、醜惡者、矮小者、多種疾病者:單眼者,或手彎曲畸形者,或跛腳者,或癱瘓者,是食物、飲料、衣服、交通工具、花環、香料、塗油、臥床、房舍、燈燭的非利得者。他以身行惡行,以語行惡行,以意行惡行,他以身體的崩解,死後往生\twnr{苦界}{109.0}、\twnr{惡趣}{110.0}、\twnr{下界}{111.0}、地獄。

  大王!猶如男子從黑暗走到黑暗,或從闇黑走到闇黑,或從血垢走到血垢,大王!我說這個人是像這樣者。大王!這樣的人是闇黑到闇黑者。

  大王!而怎樣一個人是闇黑到光明者?大王!這裡,某人被再生在卑賤家:在旃陀羅家,或在竹匠家,或在獵人家,或在車匠家,或在清垃圾者家,在貧窮處,在少食物、飲料、飲食處,在生活困難處,於該處衣食被困難地得到。他是醜陋者、醜惡者、矮小者、多種疾病者:單眼者,或手彎曲畸形者,或跛腳者,或癱瘓者,是食物、飲料、衣服、交通工具、花環、香料、塗油、臥床、房舍、燈燭的非利得者。他以身行善行,以語行善行,以意行善行,他以身體的崩解,死後往生\twnr{善趣}{112.0}、天界。

  大王!猶如男子從地上登上轎子,或從轎子登上馬背,或從馬背登上象背,或從象背登上高樓,大王!我說這個人是像這樣者。大王!這樣的人是闇黑到光明者。

  大王!而怎樣一個人是光明到闇黑者?大王!這裡,某人被再生在高貴家:在大財富\twnr{剎帝利}{116.0}家,或在大財富\twnr{婆羅門}{17.0}家,或在大財富\twnr{屋主}{103.0}家,在富有處,在大富處,在大財富處,在多金銀處,在多財產資具處,在多財穀處。他是英俊者、好看者、端正者、具備最美的容色者,食物、飲料、衣服、交通工具、花環、香料、塗油、臥床、房舍、燈燭的利得者。他以身行惡行,以語行惡行,以意行惡行後,以身體的崩解,死後往生苦界、惡趣、下界、地獄。

  大王!猶如男子從高樓下到象背,或從象背下到馬背,或從馬背下到轎子,或從轎子下到地上,或從地上進入黑暗,大王!我說這個人是像這樣者。大王!這樣的人是光明到闇黑者。

  大王!而怎樣一個人是光明到光明者?大王!這裡,某人被再生在高貴家:在大財富剎帝利家,或在大財富婆羅門家,或在大財富屋主家,在富有處,在大富處,在大財富處,在多金銀處,在多財產資具處,在多財穀處。他是英俊者、好看者、端正者、具備最美的容色者,食物、飲料、衣服、交通工具、花環、香料、塗油、臥床、房舍、燈燭的利得者。他以身行善行,以語行善行,以意行善行後,以身體的崩解,死後往生善趣、天界。

  大王!猶如男子從轎子移到轎子,或從馬背移到馬背,或從象背移到象背,或從高樓移到高樓,大王!我說這個人是像這樣者。大王!這樣的人是光明到光明者。大王!這是現在的世間中存在的四種人。」[≃\ccchref{AN.4.85}{https://agama.buddhason.org/AN/an.php?keyword=4.85}]

  世尊說這個……(中略):

  「國王!男子是貧窮者,無信者、慳吝者,

   吝嗇者、惡意向者,邪見者、無敬意者。

   對\twnr{沙門}{29.0}、\twnr{婆羅門}{17.0},或即使對其他流浪者,

   他辱駡、誹謗,他是虛無論者、激怒者,

   當施與乞求者食物時,他制止。

   國王!像那樣的男子,人們的王!當死時,

   到恐怖的地獄:闇黑到闇黑者。

   國王!男子是貧窮者,有信者、不慳吝者,

   他以最上的意向施與,意不混亂的人。

   對沙門、婆羅門,或即使對其他流浪者,

   他起立後問訊,學習正行,

   當施與乞求者食物時,他不制止。

   國王!像那樣的男子,人們的王!當死時,

   到三十三天處:闇黑到光明者。

   國王!如果男子是富裕者,無信者、慳吝者,

   吝嗇者、惡意向者,邪見者、無敬意者。

   對沙門、婆羅門,或即使對其他流浪者,

   他辱駡、誹謗,他是虛無論者、激怒者,

   當施與乞求者食物時,他制止。

   國王!像那樣的男子,人們的王!當死時,

   到恐怖地獄:光明到闇黑者。

   國王!如果男子是富裕者,有信者、不慳吝者,

   他以最上的意向施與,意不混亂的人。

   對沙門、婆羅門,或即使對其他流浪者,

   他起立後問訊,學習正行,

   當施與乞求者食物時,他不制止。

   國王!像那樣的男子,人們的王!當死時,

   到三十三天處:光明到光明者。」



\sutta{22}{22}{祖母經}{https://agama.buddhason.org/SN/sn.php?keyword=3.22}
  起源於舍衛城。

  \twnr{世尊}{12.0}對在一旁坐下的憍薩羅國波斯匿王說這個:

  「那麼,大王!你中午從哪裡來呢?」

  「\twnr{大德}{45.0}!我衰老的、年老的、高齡的、老年的、到達老年的:從出生一百二十歲的祖母已死了,大德!而祖母對我來說是可愛的、合意的。

  大德!如果能以象寶得到『我的祖母不要死』,我會給與象寶[換]『我的祖母不要死』。大德!如果能以馬寶得到『我的祖母不要死』,我會給與馬寶[換]『我的祖母不要死』。大德!如果能以最上的村落得到『我的祖母不要死』,我會給與最上的村落[換]『我的祖母不要死』。大德!如果能以部分國土得到『我的祖母不要死』,我會給與部分國土[換]『我的祖母不要死』。」

  「大王!一切眾生都是\twnr{死法}{587.3}、死為終結的、未超越死的。」

  「\twnr{不可思議}{206.0}啊,大德!\twnr{未曾有}{206.0}啊,大德!

  大德!這被世尊多麼善說:『一切眾生都是死法、死為終結的、未超越死的。』」

  「這是這樣,大王!這是這樣,大王!一切眾生都是死法、死為終結的、未超越死的。大王!猶如凡任何陶匠所作的器具:生胚(生的)與燒好的(成熟的)都是破裂法、破裂為終結的、未超越破裂的。同樣的,大王!一切眾生都是死法、死為終結的、未超越死的。」

  世尊說這個……(中略):

  「一切眾生必將死,因為生命是死為終結的,

   他們依業而去,到達福德[或]惡的結果的:

   惡業者為地獄,而福德業者為\twnr{善趣}{112.0}。

   因此應該作善的:後世的財富,

   福德在來世中,是有生命者之所依。[\suttaref{SN.3.4}]」



\sutta{23}{23}{世間經}{https://agama.buddhason.org/SN/sn.php?keyword=3.23}
  起源於舍衛城。

  在一旁坐下的憍薩羅國波斯匿王對\twnr{世尊}{12.0}說這個:

  「\twnr{大德}{45.0}!當世間的幾法生起時,生起不利、苦、不\twnr{安樂住}{156.0}呢?」

  「大王!當世間的三法生起時,生起不利、苦、不安樂住,哪三個?大王!當世間的貪欲生起時,生起不利、苦、不安樂住;大王!當世間的瞋恚生起時,生起不利、苦、不安樂住;大王!當世間的愚癡生起時,生起不利、苦、不安樂住,大王!當世間的這三法生起時,生起不利、苦、不安樂住。」

  說這個……(中略):

  「貪欲瞋恚與愚癡以及,惡心意的人,

   從自己生起的它們殺害,如有自己果實的竹。」[\suttaref{SN.3.2}]



\sutta{24}{24}{弓術經}{https://agama.buddhason.org/SN/sn.php?keyword=3.24}
  起源於舍衛城。

  在一旁坐下的憍薩羅國波斯匿王對\twnr{世尊}{12.0}說這個:「\twnr{大德}{45.0}!布施應該被施與在何處呢?」「大王!在心淨信之處。」「大德!那麼,所施於何處有大果?」「大王!『布施應該被施與在何處』,這是一個[問題],而『所施於何處有大果』,這是另一個。大王!對持戒者的所施有大果,對破戒者不像那樣。大王!那樣的話,就在這件事上我將反問你,你就如對你能接受的那樣回答它。大王!你怎麼想它:這裡,如果有你的戰爭已出現,戰鬥已群集,那時,未學的、不熟練的、未作勤修的、無弓術善巧的、膽小的、僵硬的、恐懼的、逃跑的\twnr{剎帝利}{116.0}少年到來,你會雇用(養)那位男子,以及以像那樣的男子是你需要的?」「大德!我不雇用那位男子,以及以像那樣的男子非我需要的。」「那時,未學的……(中略)婆羅門少年到來……那時,未學的……(中略)\twnr{毘舍}{476.0}少年到來……那時,未學的……(中略)\twnr{首陀羅}{472.0}少年到來……」「……以及以像那樣的男子非我需要的。」

  「大王!你怎麼想它:這裡,如果有你的戰爭已出現,戰鬥已群集,那時,善學的、熟練的、作勤修的、弓術善巧的、不膽小的、不僵硬的、不恐懼的、不逃跑的剎帝利少年到來,你雇用那位男子,以及以像那樣的男子是你需要的?」「大德!我雇用那位男子,以及以像那樣的男子非我需要的。」「那時,……婆羅門少年到來……(中略)那時,……毘舍少年到來……(中略)那時,善學的、熟練的、作勤修的、弓術善巧的、不膽小的、不僵硬的、不恐懼的、不逃跑的首陀羅少年到來,你雇用那位男子,以及以像那樣的男子是你需要的?」「大德!我雇用那位男子,以及以像那樣的男子非我需要的。」

  「同樣的,大王!即使從凡什麼族姓,\twnr{從在家出家成為無家者}{48.0},而且他是捨斷五支者、具備五支者,所施於他有大果。哪五支被捨斷?\twnr{欲的意欲}{118.0}被捨斷,惡意被捨斷,惛沈睡眠被捨斷,掉舉後悔被捨斷,疑被捨斷,這五支被捨斷。哪五支被具備?\twnr{無學}{193.1}\twnr{戒蘊}{374.0}被具備,無學定蘊被具備,無學慧蘊被具備,無學解脫蘊被具備,無學解脫智見蘊被具備,這五支被具備。像這樣,所施於捨斷五支者、具備五支者有大果。」

  世尊說這個,[說這個後,\twnr{善逝}{8.0}、]\twnr{大師}{145.0}[更進一步說這個]:

  「弓術、力量與活力,凡在\twnr{學生婆羅門}{102.0}上如果被發現,

   戰爭需要的國王雇用他,而非\twnr{緣於}{252.0}出生的不勇敢者。

   就像那樣\twnr{耐性、溫雅}{x104},凡諸法於該位已住立者,

   神聖行為的有智慧者,即使出生低賤者[人們]也應該尊敬。

   應該使建造喜樂的草屋,在這裡使多聞者能居住,

   應該在荒地建造喝水處,以及在難行處的通道。

   食物、飲料、硬食,以及衣服、臥坐具,

   應該施於已成為正直者:以一顆明淨的心。

   因為如雨雲正在雷鳴,\twnr{閃電盤繞}{x105}、\twnr{百峰般的雲}{x106},

   使高地與低地充滿:當下雨到大地時。

   就像那樣地有信者有聞者,準備食物後,

   使乞食者滿足:賢智者以食物飲料。

   喜悅者散發[施物],說『施與!施與!』

   因為那是他的雷鳴,如天空下大雨,

   那廣大的福德陣雨,注入施與者。」



\sutta{25}{25}{如山經}{https://agama.buddhason.org/SN/sn.php?keyword=3.25}
  起源於舍衛城。

  \twnr{世尊}{12.0}對在一旁坐下的憍薩羅國波斯匿王說這個:

  「那麼,大王!你\twnr{中午}{x107}從哪裡來?」

  「\twnr{大德}{45.0}!凡那些成為統治權驕傲陶醉的、欲之貪求纏縛的、國土安定達到的、大土地範圍(圓周)征服後居住者之\twnr{剎帝利灌頂王}{151.0}國王們的應該被作的,今天,在那些上我已來到努力。」

  「大王!你怎麼想它:這裡,如果有值得信賴、能信賴的男子從東方對你來,他抵達後,如果對你說這個:『大王!請你應該知道,我從東方來,在那裡,看見與雲同[高]的大山,壓碎著一切生物而來,大王!請你做凡你應該做的。』

  那時,如果……第二位男子從西方來……(中略)。

  那時,如果……第三位男子從北方來……(中略)。

  那時,如果有值得信賴、可靠的第四位男子從南方來,他抵達後,如果對你說這個:『大王!請你應該知道,我從南方來,在那裡,看見與雲同[高]的大山,壓碎著一切生物而來,大王!請你做凡你應該做的。』

  大王!在像這樣大的大可怕的生起時,在恐怖的人類滅盡時,在人的狀態難得下,什麼是你應該作的?」

  「大德!在像這樣大的大可怕的生起時,在恐怖的人類滅盡時,在人的狀態難得下,除了以法行,除了以正行,除了以善的行為(所作),除了以福德行為外,什麼是我應該作的?」

  「大王!我告訴你;大王!我使你知道:大王!老死對你壓來。大王!如果在當老死對你壓來時,什麼是你應該作的?」

  「大德!在當老死對我壓來時,除了以法行,除了以正行,除了以善的行為,除了以福德行為外,什麼是我應該作的?

  大德!大德!凡那些統治權驕傲陶醉的、欲之貪求纏縛的、國土安定達到的、大土地範圍征服後居住者之剎帝利灌頂王國王們有戰鬥象兵,大德!在當老死壓來時,那些戰鬥象兵也沒有去處,\twnr{沒有對象}{x108}。

  大德!又,凡那些……居住者之剎帝利灌頂王國王們有戰鬥馬兵……(中略)……戰鬥車兵……(中略)戰鬥步兵,大德!在當老死壓來時,那些戰鬥步兵也沒有去處,沒有對象。

  大德!又,在這些王室中有參謀大臣,凡在怨敵已來時他們能以計謀使之分裂,大德!在當老死壓來時,那些戰鬥計謀也沒有去處,沒有對象。

  大德!又,在這些王室中地下的[地窖]與空中的[閣樓]存在很多黃金金幣,在怨敵已來時,我們能以那些財產謀合,大德!在當老死壓來時,那些戰鬥財產也沒有去處,沒有對象。

  大德!而在當老死對我壓來時,除了以法行,除了以正行,除了以善的行為,除了以福德行為外,什麼是我應該作的?」

  「這是這樣,大王!這是這樣,大王!在當老死壓來時,除了以法行,除了以正行,除了以善的行為,除了以福德行為外,什麼是應該作的?」

  世尊說這個,[說這個後,善逝、]\twnr{大師}{145.0}[更進一步說這個]:

  「廣大的岩石如,觸及雲霄的山,

   會普遍地到處走,壓碎著四方。

   這樣老與死,對有生命的壓來,

   剎帝利、\twnr{婆羅門}{17.0}、\twnr{毘舍}{476.0},\twnr{首陀羅}{472.0},\twnr{旃陀羅}{120.0}、清垃圾者,

   任何人都避不開,都壓碎一切。

   在那裡土地非諸象的,非諸車的非步兵們的,

   而且非以計謀戰,或以財產能打勝。

   因此賢智的人,自己利益的看見者,

   在佛與法與僧上,明智者應該使信安頓。

   凡以身行法,以語或者以心,

   就在這裡他們稱讚他,死後在天界喜悅。」

  第三品,其\twnr{攝頌}{35.0}:

  「人、祖母、世間,弓術、山的比喻,

   被最上的佛陀教導,這是憍薩羅五個的。」

  憍薩羅相應完成。





\page

\xiangying{4}{魔相應}
\pin{第一品}{1}{10}
\sutta{1}{1}{苦行經}{https://agama.buddhason.org/SN/sn.php?keyword=4.1}
  \twnr{被我這麼聽聞}{1.0}:

  \twnr{有一次}{2.0},初\twnr{現正覺}{75.0}的\twnr{世尊}{12.0}住在優樓頻螺,尼連禪河邊牧羊人的榕樹處。那時,當世尊獨處、\twnr{獨坐}{92.0}時,這樣心的深思生起:「我確實已從那個難行解脫,我確實已從那個不伴隨利益的難行解脫,\twnr{好}{44.0}!我確實到達了解脫的\twnr{菩提}{185.0},好!」

  那時,魔\twnr{波旬}{49.0}以心了知世尊心中的深思後,去見世尊。抵達後,以\twnr{偈頌}{281.0}對世尊說:

  「離開苦行後,以那個人們不淨化,

   不純淨的你認為純淨的,純淨之道已被違背。」

  那時,世尊知道:「這位是魔\twnr{波旬}{49.0}。」後,以偈頌對魔波旬說:

  「知道不伴隨利益的後:凡任何[求]不死的苦行,

   一切都是不帶來利益的,如在陸地上的槳與舵。

   戒、定與慧:修習著導向菩提之道,

   我已達到最高的純淨,死神!你已被擊敗。」

  那時,魔波旬:「世尊知道我,\twnr{善逝}{8.0}知道我。」痛苦地、不快意地就在那裡消失。



\sutta{2}{2}{象王的容色經}{https://agama.buddhason.org/SN/sn.php?keyword=4.2}
  \twnr{被我這麼聽聞}{1.0}:

  \twnr{有一次}{2.0},初\twnr{現正覺}{75.0}的\twnr{世尊}{12.0}住在優樓頻螺,尼連禪河邊牧羊人的榕樹處。當時,世尊在漆黑的夜晚坐在露天處,\twnr{而天空下著毛毛雨}{385.0}。

  那時,魔\twnr{波旬}{49.0}想要使世尊生出害怕、僵硬狀態、\twnr{身毛豎立的}{152.0},化作大象王的容色後,去見世尊,牠的頭猶如大木槵子寶石那樣,牠的牙猶如純銀那樣,牠的鼻猶如大犁柄那樣。

  那時,世尊知道:「這位是魔\twnr{波旬}{49.0}。」後,以\twnr{偈頌}{281.0}對魔波旬說:

  「長時間不斷地到來著:做清淨不清淨的容色後,

   波旬!夠了!以你的那個[可怕相貌],死神!你已被擊敗。」

  那時,魔波旬:「世尊知道我,\twnr{善逝}{8.0}知道我。」痛苦地、不快意地就在那裡消失。



\sutta{3}{3}{清淨經}{https://agama.buddhason.org/SN/sn.php?keyword=4.3}
  \twnr{被我這麼聽聞}{1.0}:

  \twnr{有一次}{2.0},初\twnr{現正覺}{75.0}的\twnr{世尊}{12.0}住在優樓頻螺,尼連禪河邊牧羊人的榕樹處。當時,世尊在漆黑的夜晚坐在露天處,\twnr{而天空下著毛毛雨}{385.0}。

  那時,魔\twnr{波旬}{49.0}想要使世尊生出害怕、僵硬狀態、\twnr{身毛豎立的}{152.0}而去見世尊。抵達後,在世尊的不遠處顯示各種不同輝耀的容色:清淨的與不清淨的。

  那時,世尊知道:「這位是魔波旬。」後,以\twnr{偈頌}{281.0}對魔波旬說:

  「長時間不斷地到來著:做清淨不清淨的容色後,

   波旬!夠了!以你的那個[可怕相貌],死神!你已被擊敗。

   凡以身與語,以及以意善防護者,

   他們不是魔控制的順從者,他們不是魔的\twnr{隨從}{x109}。」

  那時,魔……(中略)就在那裡消失。



\sutta{4}{4}{魔網經第一}{https://agama.buddhason.org/SN/sn.php?keyword=4.4}
  \twnr{被我這麼聽聞}{1.0}:

  \twnr{有一次}{2.0},\twnr{世尊}{12.0}住在波羅奈仙人墜落處的鹿林。

  在那裡,世尊召喚\twnr{比丘}{31.0}們:「比丘們!」

  「\twnr{尊師}{480.0}!」那些比丘回答世尊。

  世尊說這個:

  「比丘們!對我來說,以\twnr{如理作意}{114.0}、如理正勤已到達無上解脫,已作證無上解脫,比丘們!你們也應該以如理作意、如理正勤到達無上解脫,作證無上解脫。」

  那時,魔\twnr{波旬}{49.0}去見世尊。抵達後,以\twnr{偈頌}{281.0}對世尊說:

  「你被魔網束縛:那些諸天與那些人們[的],

   你被魔的繫縛束縛,\twnr{沙門}{29.0}!你將不解脫我的[繫縛]。」

  「我已經由魔網解脫:那些諸天與那些人們[的],

   我已解脫魔的繫縛,死神!你已被擊敗。」

  那時,魔波旬……(中略),就在那裡消失。



\sutta{5}{5}{魔網經第二}{https://agama.buddhason.org/SN/sn.php?keyword=4.5}
  \twnr{有一次}{2.0},\twnr{世尊}{12.0}住在波羅奈仙人墜落處的鹿林。

  在那裡,世尊召喚\twnr{比丘}{31.0}們:「比丘們!」

  「\twnr{尊師}{480.0}!」那些比丘回答世尊。

  世尊說這個:

  「比丘們!我已從一切網解脫:那些諸天與那些人們[的],比丘們!你們也已從一切網解脫:那些諸天與那些人們[的]。比丘們!為了眾人的利益,為了眾人的安樂,為了世間的憐愍,為了天-人們的需要、利益、安樂,請你們進行遊行,\twnr{不要兩個同一地走}{x110}。

  比丘們!你們要教導開頭是善的、中間是善的、結尾是善的;\twnr{有意義的}{81.0}、\twnr{有文字的}{82.0}法,你們要說明完全圓滿、\twnr{遍純淨的梵行}{483.0},有少塵垢之類的眾生以法的未聽聞情況而退失,他們將會是法的了知者。

  比丘們!我也為了法之教導將去優樓頻螺的謝那鎮。」

  那時,魔\twnr{波旬}{49.0}去見世尊。抵達後,以\twnr{偈頌}{281.0}對世尊說:

  「你被魔網束縛:那些諸天與那些人們[的],

   你被魔的繫縛束縛,\twnr{沙門}{29.0}!你將不解脫我的[繫縛]。」

  「我已經由魔網解脫:那些諸天與那些人們[的],

   我已解脫魔的繫縛,死神!你已被擊敗。」

  那時,魔波旬……(中略),就在那裡消失。



\sutta{6}{6}{蛇經}{https://agama.buddhason.org/SN/sn.php?keyword=4.6}
  \twnr{被我這麼聽聞}{1.0}:

  \twnr{有一次}{2.0},\twnr{世尊}{12.0}住在王舍城栗鼠飼養處的竹林中。

  當時,世尊在漆黑的夜晚坐在露天處,\twnr{而天空下著毛毛雨}{385.0}。

  那時,魔\twnr{波旬}{49.0}想要使世尊生出害怕、僵硬狀態、\twnr{身毛豎立的}{152.0},化作大蛇王的容色後去見世尊,身體猶如一整棵樹的大船一樣,頸部猶如酒屋的大墊一樣,眼睛猶如憍薩羅國的大銅鉢一樣,嘴巴伸出舌頭猶如空中打雷時出現的閃電一樣,呼吸聲猶如鐵匠吹火時的咕嚕咕嚕聲一樣。

  那時,世尊知道:「這位是魔波旬。」後,以\twnr{偈頌}{281.0}對魔波旬說:

  「那位\twnr{為了睡眠}{x111}使用諸空屋者,他是\twnr{牟尼}{125.0}、自我抑制者,

   捨棄後他應該在那裡過生活,對像那種者那確實是適當的。

   許多行走的、許多可怕的,還有許多虻、蛇,

   在那裡[一根]毛也不會動搖:來到空屋的大牟尼。

   即使天空破裂大地搖動,或者即使一切生物類都驚怖,

   即使刺箭被安排在胸部,覺者們也不在諸\twnr{依著}{198.0}中求(作)救護所。」

  那時,魔波旬:「世尊知道我,\twnr{善逝}{8.0}知道我。」痛苦地、不快意地就在那裡消失。



\sutta{7}{7}{睡覺經}{https://agama.buddhason.org/SN/sn.php?keyword=4.7}
  \twnr{被我這麼聽聞}{1.0}:

  \twnr{有一次}{2.0},\twnr{世尊}{12.0}住在王舍城栗鼠飼養處的竹林中。

  那時,世尊在露天處經行[整夜]後,在破曉時洗腳後,進入住處,[左]腳放在[右]腳上、\twnr{作意起來想後}{502.0},具念正知地\twnr{以右脅作獅子臥}{367.0}。

  那時,魔\twnr{波旬}{49.0}去見世尊。抵達後,以\twnr{偈頌}{281.0}對世尊說:

  「你怎麼睡覺、你為何睡覺呢?這是怎麼了你睡得像個不幸者?

   [想:]『屋舍是空的』你睡覺,這是怎麼了在太陽已上升時你[還]睡覺?」

  「對凡沒有\twnr{欲纏、縛著}{x112}的渴愛,能引導到任何地方者,

   遍滅盡一切\twnr{依著}{198.0}的覺者睡覺,魔!這個情形對你怎麼了嗎?」

  那時,魔波旬……(中略)就在那裡消失。



\sutta{8}{8}{歡喜經}{https://agama.buddhason.org/SN/sn.php?keyword=4.8}
  \twnr{被我這麼聽聞}{1.0}:

  \twnr{有一次}{2.0},\twnr{世尊}{12.0}住在舍衛城祇樹林給孤獨園。

  那時,\twnr{魔波旬}{x113}去見世尊。抵達後,魔\twnr{波旬}{49.0}在世尊的面前說這偈頌:

  「有孩子的以孩子他歡喜,就像那樣有牛的以牛他歡喜,

   \twnr{依著}{198.0}確實是人們的歡喜,凡無依著者他確實不歡喜。」

  「有孩子的以孩子他憂愁,就像那樣有牛的以牛他憂愁,

   依著確實是人們的憂愁,凡無依著者他確實不憂愁。」[\suttaref{SN.1.12}]

  那時,魔波旬:「世尊知道我,\twnr{善逝}{8.0}知道我。」痛苦地、不快意地就在那裡消失。



\sutta{9}{9}{壽命經第一}{https://agama.buddhason.org/SN/sn.php?keyword=4.9}
  \twnr{被我這麼聽聞}{1.0}:

  \twnr{有一次}{2.0},\twnr{世尊}{12.0}住在王舍城栗鼠飼養處的竹林中。

  在那裡,世尊召喚\twnr{比丘}{31.0}們:「比丘們!」

  「\twnr{尊師}{480.0}!」那些比丘回答世尊。

  世尊說這個:

  「比丘們!人的這壽命是短的,來世必將被走到,善的應該被作、\twnr{梵行}{381.0}應該被實行。生者無不死,比丘們!凡活得長者,他有百年加減(少或更多)。」

  那時,魔\twnr{波旬}{49.0}去見世尊。抵達後,以\twnr{偈頌}{281.0}對世尊說:

  「人的壽命是長的,善人不應該輕蔑它,

   應該\twnr{如陶醉於乳的(嬰兒)般地}{x114}過生活,沒有死亡的到來。」

  「人的壽命是短的,善人應該輕蔑它,

   應該如頭已燃燒般地過生活,沒有死亡的不到來。」

  那時,魔波旬……(中略)就在那裡消失。



\sutta{10}{10}{壽命經第二}{https://agama.buddhason.org/SN/sn.php?keyword=4.10}
  \twnr{被我這麼聽聞}{1.0}:

  \twnr{有一次}{2.0},\twnr{世尊}{12.0}住在王舍城栗鼠飼養處的竹林中。

  在那裡,世尊……(中略)說這個:

  「\twnr{比丘}{31.0}們!人的這壽命是短的,來世必將被走到,善的應該被作、\twnr{梵行}{381.0}應該被實行。生者無不死,比丘們!凡活得長者,他有百年加減(少或更多)。」

  那時,魔\twnr{波旬}{49.0}去見世尊。抵達後,以\twnr{偈頌}{281.0}對世尊說:

  「日夜不流逝,生命不被破壞,

   \twnr{不免一死的人}{600.0}壽命繞著走,\twnr{如車輪對車軸(柱)}{x115}。」

  「日夜流逝,生命被破壞,

   不免一死的人壽命被耗盡,如小河的水。」

  那時,魔波旬:「世尊知道我,\twnr{善逝}{8.0}知道我。」痛苦地、不快意地就在那裡消失。

  第一品,其\twnr{攝頌}{35.0}:

  「苦行與象,清淨與魔網二則,

   蛇、睡覺、歡喜,壽命二則在後。」





\pin{第二品}{11}{20}
\sutta{11}{11}{岩石經}{https://agama.buddhason.org/SN/sn.php?keyword=4.11}
  \twnr{被我這麼聽聞}{1.0}:

  \twnr{有一次}{2.0},\twnr{世尊}{12.0}住在王舍城\twnr{耆闍崛山}{258.0}。

  當時,世尊在漆黑的夜晚坐在露天處,\twnr{而天空下著毛毛雨}{385.0}。

  那時,魔\twnr{波旬}{49.0}想要使世尊生出害怕、僵硬狀態、\twnr{身毛豎立的}{152.0}而去見世尊。抵達後,在世尊的不遠處碎破諸大岩石。

  那時,世尊知道:「這位是魔波旬。」後,以\twnr{偈頌}{281.0}對魔波旬說:

  「即使那整座耆闍崛[山],你將使之搖動,

   正解脫者們、覺者們,確實不被動搖。」

  那時,魔波旬:「世尊知道我,\twnr{善逝}{8.0}知道我。」痛苦地、不快意地就在那裡消失。



\sutta{12}{12}{為何獅子經}{https://agama.buddhason.org/SN/sn.php?keyword=4.12}
  \twnr{有一次}{2.0},\twnr{世尊}{12.0}住在舍衛城祇樹林給孤獨園。

  當時,世尊被大眾圍繞,教導法。

  那時,魔\twnr{波旬}{49.0}想這個:

  「這位\twnr{沙門}{29.0}\twnr{喬達摩}{80.0}被大眾圍繞,教導法,讓我為了使之盲目去見沙門喬達摩。」

  那時,魔波旬去見世尊。抵達後,以\twnr{偈頌}{281.0}對世尊說:

  「你為何在群眾中,自信地如獅子地吼?

   因為有你的對手,你認為是勝利者嗎?」

  「大英雄們確實在群眾中,自信地吼,

   如來們已得到力量,已度脫世間中的執著。」

  那時,魔波旬:「世尊知道我,\twnr{善逝}{8.0}知道我。」痛苦地、不快意地就在那裡消失。



\sutta{13}{13}{碎石片經}{https://agama.buddhason.org/SN/sn.php?keyword=4.13}
  \twnr{被我這麼聽聞}{1.0}:

  \twnr{有一次}{2.0},\twnr{世尊}{12.0}住在王舍城嘛瘩姑七的鹿林。

  當時,世尊的腳被碎石片所傷,世尊的強烈感受轉起:苦的、激烈的、猛烈的、強烈的、不愉快的、不合意的身體的[感受]。世尊具念地、正知地忍受那些,不被惱害著。那時,世尊摺大衣成四折後,[左]腳放在[右]腳上後,具念正知地\twnr{以右脅作獅子臥}{367.0}。

  那時,魔\twnr{波旬}{49.0}去見世尊。抵達後,以\twnr{偈頌}{281.0}對世尊說:

  「你以懶惰躺臥或者陶醉於詩歌嗎?你的極多目標不存在嗎?

   獨自在遠離的坐臥所,為何你就以這張睡臉睡覺?」

  「我非以懶惰躺臥也非陶醉於詩歌,達到目標後我已離愁,

   獨自在遠離的坐臥所,我對一切生類有憐愍地躺臥。

   即使刺箭已進入他們的胸部,心臓急速地顫動著,

   這裡即使那些被箭射到者也得到睡眠,已離刺箭的我{因此}[為何?]不能睡?

   清醒非我擔心、\twnr{我也不害怕睡覺}{x116},白天夜晚對我都不苦惱,

   我不見[自己]在世間任何地方減損,憐愍一切生類因此我能睡。」

  那時,魔波旬:「世尊知道我,\twnr{善逝}{8.0}知道我。」痛苦地、不快意地就在那裡消失。



\sutta{14}{14}{適當經}{https://agama.buddhason.org/SN/sn.php?keyword=4.14}
  \twnr{有一次}{2.0},\twnr{世尊}{12.0}住在憍薩羅國一沙羅樹[地方]的\twnr{婆羅門}{17.0}村落。

  當時,世尊被在家大眾圍繞,教導法。

  那時,魔\twnr{波旬}{49.0}想這個:

  「這位\twnr{沙門}{29.0}\twnr{喬達摩}{80.0}被在家大眾圍繞,教導法,讓我為了使之盲目去見沙門喬達摩。」

  那時,魔波旬去見世尊。抵達後,以\twnr{偈頌}{281.0}對世尊說:

  「這不適合你:凡你教誡其他人,

   當那樣做時,不要\twnr{在順從與排斥上}{x117}執著。」

  「有益的憐愍者、\twnr{正覺者}{185.1},凡他教誡其他人,

   \twnr{如來}{4.0}已從,順從與排斥脫離。」

  那時,魔波旬:「世尊知道我,\twnr{善逝}{8.0}知道我。」痛苦地、不快意地就在那裡消失。



\sutta{15}{15}{意經}{https://agama.buddhason.org/SN/sn.php?keyword=4.15}
  \twnr{被我這麼聽聞}{1.0}:

  \twnr{有一次}{2.0},\twnr{世尊}{12.0}住在舍衛城祇樹林給孤獨園。

  那時,魔\twnr{波旬}{49.0}去見世尊。抵達後,以\twnr{偈頌}{281.0}對世尊說:

  「空中行走的\twnr{陷阱}{x118}:凡這個意行走,

   以那個我將誘捕你,\twnr{沙門}{29.0}!你將不被我釋放。」

  「色、聲音、氣味、味道,以及悅意的\twnr{所觸}{220.2},

   在這裡欲對我來說已消失,死神!你已被擊敗。」

  那時,魔波旬:「世尊知道我,\twnr{善逝}{8.0}知道我。」痛苦地、不快意地就在那裡消失。



\sutta{16}{16}{鉢經}{https://agama.buddhason.org/SN/sn.php?keyword=4.16}
  起源於舍衛城。

  當時,\twnr{世尊}{12.0}以關於五取蘊之法說對\twnr{比丘}{31.0}們開示、勸導、鼓勵、\twnr{使歡喜}{86.0},而那些比丘\twnr{作目標後}{316.0}、作意後、\twnr{全心注意後}{479.0}傾耳聽法。

  那時,魔\twnr{波旬}{49.0}想這個:

  「這位\twnr{沙門}{29.0}\twnr{喬達摩}{80.0}以關於五取蘊之法說對比丘們開示、勸導、鼓勵、使歡喜,而那些比丘作目標後、作意後、全心注意後傾耳聽法,讓我為了使之盲目,去見沙門喬達摩。」

  當時,許多鉢被放置在屋外。

  那時,魔波旬化作公牛的容色後,到那些鉢處。

  那時,某位比丘對另一位比丘說這個:

  「比丘!比丘!那隻公牛會打破眾鉢。」

  在這麼說時,世尊對那位比丘說這個:

  「比丘!那不是公牛,那是魔波旬,為了使你們盲目來的。」

  那時,世尊知道:「這位是魔波旬。」後,以\twnr{偈頌}{281.0}對魔波旬說:

  「色、被感受的、想,識以及凡被行作的,

   \twnr{我不是這個}{33.1}、\twnr{這不是我的}{32.1},這樣在這裡\twnr{離染}{558.0},

   這麼已離染已得安穩者,已超越一切結者,

   \twnr{當於一切處探求時}{x119},魔軍也沒得到。」

  那時,魔波旬……(中略)就在那裡消失。



\sutta{17}{17}{六觸處經}{https://agama.buddhason.org/SN/sn.php?keyword=4.17}
  \twnr{有一次}{2.0},\twnr{世尊}{12.0}住在毘舍離大林\twnr{重閣}{213.0}講堂。

  當時,世尊以關於\twnr{六觸處}{78.0}之法說對\twnr{比丘}{31.0}們開示、勸導、鼓勵、\twnr{使歡喜}{86.0},而那些比丘\twnr{作目標後}{316.0}、作意後、\twnr{全心注意後}{479.0}傾耳聽法。

  那時,魔\twnr{波旬}{49.0}想這個:

  「這位\twnr{沙門}{29.0}\twnr{喬達摩}{80.0}以關於六觸處之法說對比丘們開示、勸導、鼓勵、使歡喜,而那些比丘作目標後、作意後、全心注意後傾耳聽法,讓我為了使之盲目,去見沙門喬達摩。」

  那時,魔波旬去見世尊。抵達後,在世尊的不遠處作大恐怖與可怕聲,以至於看(聽)起來像大地被打裂一樣。

  那時,某位比丘對另一位比丘說這個:

  「比丘!比丘!這大地看起來像被打裂一樣。」

  在這麼說時,世尊對那位比丘說這個:

  「比丘!那不是大地被打裂,那是魔波旬,為了使你們盲目來的。」

  那時,世尊知道:

  「這位是魔波旬。」後,以\twnr{偈頌}{281.0}對魔波旬說:

  「色、聲、氣味、味道,\twnr{所觸}{220.2}與全部的法,

   這是可怕的\twnr{世間物質}{593.0},在這裡\twnr{世間被迷昏頭}{x120}。

   但超越這個後,具念的佛弟子,

   超越魔的領域後,如太陽輝耀。」

  那時,魔波旬……(中略)就在那裡消失。



\sutta{18}{18}{團食經}{https://agama.buddhason.org/SN/sn.php?keyword=4.18}
  \twnr{有一次}{2.0},\twnr{世尊}{12.0}住在摩揭陀國五沙羅樹[地方]的婆羅門村落。當時,五沙羅樹的婆羅門村落有少女饗宴客人節。

  那時,世尊午前時穿衣、拿起衣鉢後,\twnr{為了托鉢}{87.0}進入五沙羅樹的婆羅門村落。

  當時,五沙羅樹的婆羅門\twnr{屋主}{103.0}們已被魔\twnr{波旬}{49.0}所佔有:「不要\twnr{沙門}{29.0}\twnr{喬達摩}{80.0}得到團食。」

  那時,世尊以像洗淨了的鉢,如以洗淨了的鉢為了托鉢進入五沙羅樹的婆羅門村落一樣返回。

  那時,魔波旬去見世尊。抵達後,對世尊說這個:「沙門!是否你得到團食?」

  「波旬!你像那樣作,依之我不能得到團食嗎?」

  「\twnr{大德}{45.0}!那樣的話,請世尊第二次又為了托鉢進入五沙羅樹的婆羅門村落,我將像那樣作,依之世尊得到團食。」

  「攻擊那位如來後,魔產出非福,

   波旬!你認為:我的惡不報嗎?

   我們確實極安樂地生活,對凡沒有任何東西的我們,

   我們將是以喜為食者,如光音天神們。」

  那時,魔波旬:「世尊知道我,\twnr{善逝}{8.0}知道我。」痛苦地、不快意地就在那裡消失。



\sutta{19}{19}{農夫經}{https://agama.buddhason.org/SN/sn.php?keyword=4.19}
  起源於舍衛城。

  當時,\twnr{世尊}{12.0}以涅槃關聯的法說對\twnr{比丘}{31.0}們開示、勸導、鼓勵、\twnr{使歡喜}{86.0},而那些比丘\twnr{作目標後}{316.0}、作意後、\twnr{全心注意後}{479.0}傾耳聽法

  那時,魔\twnr{波旬}{49.0}想這個:

  「這位\twnr{沙門}{29.0}\twnr{喬達摩}{80.0}以涅槃關聯的法說……(中略)讓我為了使之盲目去見沙門喬達摩。」

  那時,魔波旬化作農夫的容色後,在肩上放置大犁後,拿起驅使牛的長刺棒後,頭髮散亂地,穿粗麻布衣地,泥土沾滿腳地去見世尊。抵達後,對世尊說這個:

  「沙門!是否看見公牛?」

  「波旬!那麼,你與公牛有什麼嗎?」

  「沙門!就眼是我的,色是我的,眼觸[相應的]識處是我的,沙門!走到哪裡後你將從我脫離?沙門!就耳是我的,聲音……(中略)沙門!就鼻是我的,氣味是我的……沙門!就舌是我的,味道是我的……沙門!身是我的,\twnr{所觸}{220.2}是我的……沙門!就意是我的,法是我的,意觸識處是我的,沙門!走到哪裡後你將從我脫離?」

  「波旬!就眼是你的,色是你的,眼觸識處是你的,但,波旬!於沒有眼、沒有色、沒有眼觸識處之處,波旬!那裡非你的去處。波旬!就耳是你的,聲音是你的,耳觸識處是你的,但,波旬!於沒有耳、沒有聲音、沒有耳觸識處之處,波旬!那裡非你的去處。波旬!就鼻是你的,氣味是你的,鼻觸識處是你的,但,波旬!於沒有鼻、沒有氣味、沒有鼻觸識處之處,波旬!那裡非你的去處。波旬!就舌是你的,味道是你的,舌觸識處是你的……(中略)波旬!就身是你的,所觸是你的,身觸識處是你的……(中略)波旬!就意是你的,法是你的,意觸識處是你的,但,波旬!於沒有意、沒有法、沒有意觸識處之處,波旬!那裡非你的去處。」

  「凡他們說『這是我的』,以及凡他們說『我的』,

   如果你的意(心)是在那裡,沙門!你沒從我脫離。」

  「凡他們說的那不屬於我,凡他們說的我非那些,

   波旬!你要這麼知道,你甚至沒看見我的道。」

  那時,魔波旬……(中略),就在那裡消失。



\sutta{20}{20}{統治經}{https://agama.buddhason.org/SN/sn.php?keyword=4.20}
  \twnr{有一次}{2.0},\twnr{世尊}{12.0}住在憍薩羅國喜馬拉雅山腳下的\twnr{林野}{142.0}小屋。那時,當世尊獨處、\twnr{獨坐}{92.0}時,這樣心的深思生起:「可能以法行使統治:無殺、無使之殺,無征服、無使之征服,無悲傷、無使之悲傷嗎?」

  那時,魔\twnr{波旬}{49.0}以心了知世尊心中的深思後,去見世尊。抵達後,對世尊說這個:

  「\twnr{大德}{45.0}!請世尊行使統治,請\twnr{善逝}{8.0}以法行使統治:無殺、無使之殺,無征服、無使之征服,無悲傷、無使之悲傷。」

  「波旬!那麼,你看見我什麼,而對我說這個:『大德!請世尊行使統治,請善逝以法行使統治!無殺、無使之殺,無征服、無使之征服,無悲傷、無使之悲傷。』呢?」

  「大德!四神足被世尊\twnr{修習}{94.0}、被\twnr{多作}{95.0}、被作為車輛、被作為基礎、被實行、被累積、\twnr{被善努力}{682.0},大德!而當希望時,如果世尊\twnr{勝解}{257.0}喜馬拉雅山山王就像黃金這樣,那麼它會成為黃金。」

  「對金山:對整座(全部)黃金,

   即便二座對一個人也不滿足,知道像這樣者能平等地行。

   凡從此因由看到苦者,那個人如何會彎向諸欲?

   知道\twnr{依著}{198.0}是世間中的『染著』後,人就應該為了它的調伏而學習。」

  那時,魔波旬:「世尊知道我,善逝知道我。」痛苦地、不快意地就在那裡消失。

  第二品,其\twnr{攝頌}{35.0}:

  「岩石、獅子、碎石片,適當、意,

   鉢、處、團食,農夫與統治它們為十。」





\pin{第三品}{21}{25}
\sutta{21}{21}{眾多經}{https://agama.buddhason.org/SN/sn.php?keyword=4.21}
  \twnr{被我這麼聽聞}{1.0}:

  \twnr{有一次}{2.0},\twnr{世尊}{12.0}住在釋迦族的尸羅哇地。

  當時,眾多\twnr{比丘}{31.0}在世尊的不遠處住於不放逸的、熱心的、自我努力的。

  那時,魔\twnr{波旬}{49.0}化作\twnr{婆羅門}{17.0}的容色後,以大的編織髮髻,穿羊皮衣的、年老的、像\twnr{椽木}{663.0}那樣彎曲的,呼嚕呼嚕的呼吸的,持\twnr{無花果樹枝的拐杖}{x121}後去見那些比丘。抵達後,對那些比丘說這個:

  「\twnr{尊師}{203.0}們!你們年輕出家,黑髮的青年,具備青春的幸福,在人生初期,不在欲中娛樂,尊師們!你們要在人之欲中享受,不要捨棄直接可見的後追逐時間的。[\suttaref{SN.1.20}]」

  「婆羅門!我們沒捨棄直接可見的後追逐時間的,婆羅門!但我們捨棄時間的後追逐直接可見的,婆羅門!因為欲被世尊說是時間的、多苦的、多\twnr{絕望}{342.0}的,在這裡有更多的\twnr{過患}{293.0}。這個法是直接可見的、即時的、請你來看的、能引導的、應該被智者各自經驗的。」

  在這麼說時,魔波旬搖頭、\twnr{吐舌}{x122}、額頭上[蹙出]\twnr{三條溝}{x123}\twnr{使不愉快的神情出現}{x124}、拄著拐杖後離開。

  那時,那些比丘去見世尊。抵達後,向世尊\twnr{問訊}{46.0}後,在一旁坐下。在一旁坐下的那些比丘對世尊說這個:

  「\twnr{大德}{45.0}!這裡,我們在世尊的不遠處住於不放逸的、熱心的、自我努力的,大德!那時,有\twnr{某位}{39.0}婆羅門以大的編織髮髻,穿羊皮衣的、年老的、像椽木那樣彎曲的,呼嚕呼嚕的呼吸的,持無花果樹枝的拐杖後來見我們。抵達後,對我們說這個:『尊師們!你們年輕出家,黑髮的青年,具備青春的幸福,在人生初期,不在欲中娛樂,尊師們!你們要在人之欲中享受,不要捨棄直接可見的後追逐時間的。』大德!在這麼說時,我們對那位婆羅門說這個:『婆羅門!我們沒捨棄直接可見的後追逐時間的,婆羅門!但我們捨棄時間的後追逐直接可見的,婆羅門!因為欲被世尊說是時間的、多苦的、多絕望的,在這裡有更多的過患。這個法是直接可見的、即時的、請你來看的、能引導的、應該被智者各自經驗的。』大德!在這麼說時,那位婆羅門搖頭後,使舌頭上下動後,使額頭上出現三條皺眉後,拄著拐杖後離開。」

  「比丘們!那位不是婆羅門,那位是魔波旬,為了使你們盲目來的。」

  那時,世尊知道這件事後,那時候說這\twnr{偈頌}{281.0}:

  「凡從此因由看到苦者,那個人如何會彎向諸欲?

   知道\twnr{依著}{198.0}是世間中的『染著』後,人就應該為了它的調伏而學習。」



\sutta{22}{22}{三彌提經}{https://agama.buddhason.org/SN/sn.php?keyword=4.22}
  \twnr{被我這麼聽聞}{1.0}:

  \twnr{有一次}{2.0},\twnr{世尊}{12.0}住在釋迦族的尸羅哇地。

  當時,\twnr{尊者}{200.0}三彌提在世尊的不遠處住於不放逸的、熱心的、自我努力的。那時,當尊者三彌提獨處、\twnr{獨坐}{92.0}時,這樣心的深思生起:

  「確實是我的利得,確實是我的善得的:我的大師是\twnr{阿羅漢}{5.0}、遍正覺者;確實是我的利得,確實是我的善得的:我在這麼善說的法律中出家;確實是我的利得,確實是我的善得的:我的\twnr{同梵行者}{138.0}們是持戒者、\twnr{善法者}{601.1}。」

  那時,魔\twnr{波旬}{49.0}以心了知尊者三彌提心中的深思後,去見尊者三彌提。抵達後,在尊者三彌提不遠處作大恐怖與可怕聲,以至於看(聽)起來像大地被打裂一樣。

  那時,尊者三彌提去見世尊。抵達後,向世尊\twnr{問訊}{46.0}後,在一旁坐下。在一旁坐下的尊者三彌提對世尊說這個:

  「\twnr{大德}{45.0}!這裡,我在世尊的不遠處住於不放逸的、熱心的、自我努力的,大德!獨處、獨坐的那個我這樣心的深思生起:『確實是我的利得,確實是我的善得的:我的大師是阿羅漢、遍正覺者;確實是我的利得,確實是我的善得的:我在這麼善說的法律中出家;確實是我的利得,確實是我的善得的:我的同梵行者們是持戒者、善法者。』大德!在那個我不遠處有大恐怖聲,以至於看起來像大地被打裂一樣。」

  「三彌提!那不是大地被打裂,那是魔波旬,為了使你盲目來的。三彌提!去吧!你就那裡住於不放逸的、熱心的、自我努力的。」

  「是的,大德!」尊者三彌提回答世尊後,從座位起來、向世尊問訊、\twnr{作右繞}{47.0}後,離開。

  第二次,尊者三彌提就那裡住於不放逸的、熱心的、自我努力的。

  第二次,尊者三彌提獨處、獨坐時……(中略)。

  第二次,魔波旬以心了知尊者三彌提心中的深思後……(中略)以至於看起來像大地被打裂一樣。

  那時,尊者三彌提以\twnr{偈頌}{281.0}對魔波旬說:

  「我\twnr{以信從在家出家成為無家者}{48.0},

   \twnr{我的念與慧已覺}{x125},且心善得定,

   任你作想要的諸形色吧!你確實將不使我動搖。」

  那時,魔波旬:「三彌提\twnr{比丘}{31.0}知道我。」痛苦地、不快意地就在那裡消失。



\sutta{23}{23}{瞿低迦經}{https://agama.buddhason.org/SN/sn.php?keyword=4.23}
  \twnr{被我這麼聽聞}{1.0}:

  \twnr{有一次}{2.0},\twnr{世尊}{12.0}住在王舍城栗鼠飼養處的竹林中。

  當時,\twnr{尊者}{200.0}瞿低迦住在仙吞山坡的黑岩處。那時,住於不放逸的、熱心的、自我努力的尊者瞿低迦\twnr{觸達暫時的心解脫}{x126},但,尊者瞿低迦又從那個暫時的心解脫退失。

  第二次,住於不放逸的、熱心的、自我努力的尊者瞿低迦觸達暫時的心解脫,第二次,尊者瞿低迦從那個暫時的心解脫退失。

  第三次,住於不放逸的、熱心的、自我努力的尊者瞿低迦觸達暫時的心解脫,第三次,尊者瞿低迦從那個……(中略)退失。

  第四次,當尊者瞿低迦住於不放逸……(中略)觸達暫時的心解脫,第四次,尊者瞿低迦從那個……(中略)退失。

  第五次,當尊者瞿低迦……(中略)觸達暫時的心解脫,第五次,尊者……(中略)心解脫退失。

  第六次,住於不放逸的、熱心的、自我努力的尊者瞿低迦觸達暫時的心解脫,第六次,尊者瞿低迦從那個暫時的心解脫退失。

  第七次,住於不放逸的、熱心的、自我努力的尊者瞿低迦觸達暫時的心解脫。

  那時,尊者瞿低迦想這個:

  「直到第六次為止,我已從暫時的心解脫退失了,讓我取刀[自殺]吧!」

  那時,魔\twnr{波旬}{49.0}以心了知尊者瞿低迦心中的深思後,去見世尊。抵達後,以\twnr{偈頌}{281.0}對世尊說:

  「大英雄、大慧者,以神通以名聲閃耀者,

   已超越一切怨恨恐懼者,我禮拜在\twnr{有眼者}{629.0}的腳上。

   大英雄、死的征服者,你的弟子對死,

   希望、意圖,光輝者請制止他!

   世尊!為何你的,在教說上愛好的弟子,

   心意未達成的\twnr{有學}{193.0},能死?名聞者!」

  當時,刀已被尊者瞿低迦取了。

  那時,世尊知道:「這是魔波旬。」後,以偈頌對魔波旬說:

  「明智者們確實這麼做:不期待活命,

   連根拔除渴愛後,瞿低迦已般涅槃。」

  那時,世尊召喚\twnr{比丘}{31.0}們:

  「比丘們!我們走,我們將去仙吞山坡的黑岩處,在那裡刀已被瞿低迦善男子取了。」

  「是的,\twnr{大德}{45.0}!」那些比丘回答世尊。

  那時,世尊與眾多比丘一起去仙吞山坡的黑岩處。世尊就從遠處看見尊者瞿低迦在床\twnr{肩膀轉回}{x127}地躺臥。

  當時,冒煙狀態的黑闇雲煙走向東方後,走向西方;走向北方;走向南方;走向上方;走向下[方];走向四方的中間方位。

  那時,世尊召喚比丘們:

  「比丘們!你們看見這股冒煙狀態的黑闇雲煙走向東方後,走向西方;走向北方;走向南方;走向上方;走向下[方];走向四方的中間方位嗎?」

  「是的,大德!」

  「比丘們!這是魔波旬探求瞿低迦善男子的識:『瞿低迦善男子的識住立在哪裡?』比丘們!然而,以識已不住立,瞿低迦善男子已般涅槃了。」

  那時,魔波旬拿起淡黃色的橡木琵琶琴後,去見世尊。抵達後,以偈頌對世尊說:

  「上下與水平方位,四方的中間方位我對他,

   當探求時我沒得到,瞿低迦他已去何處。」

  「那位明智者、具足堅固心者,經常愛好禪定的禪定者,

   是日夜實踐者,不希求活命者。

   征服死神軍後,不來再有後,

   連根拔除渴愛後,瞿低迦已般涅槃。」

  「被愁打敗者,琵琶琴從腋下掉落,

   之後那不快樂的\twnr{夜叉}{126.0},就在那裡消失。」



\sutta{24}{24}{七年追蹤經}{https://agama.buddhason.org/SN/sn.php?keyword=4.24}
  \twnr{被我這麼聽聞}{1.0}:

  \twnr{有一次}{2.0},\twnr{世尊}{12.0}住在優樓頻螺,尼連禪河邊牧羊人的榕樹下。

  當時,魔波旬已跟隨世尊七年,\twnr{期待機會}{x128},沒得到機會中。那時,魔波旬去見世尊。抵達後,以\twnr{偈頌}{281.0}對世尊說:

  「你已陷入愁而在林中修禪不是嗎?財產已失去或者欲求著不是嗎?

   你在村落中做了什麼罪行不是嗎?你為什麼不與人作朋友呢?

   \twnr{為何}{x129}你的友情不生起呢?」

  「挖出一切愁之根後,我無罪行無憂愁地修禪,

   切斷一切\twnr{有貪的熱望}{x130}後,我無漏地修禪,放逸者的親族!」

  「凡他們說『這是我的』,以及凡他們說『我的』,

   如果你的意(心)是在那裡,沙門!你沒從我脫離。」

  「凡他們說的那不屬於我,凡他們說的我非那些,

   波旬!你要這麼知道,你甚至沒看見我的道。」[\suttaref{SN.4.19}]

  「如果道被隨覺:安穩的、導向\twnr{不死}{123.0}的,

   請你離開、請你就一個人走,你為何教誡其他人呢?」

  「他們問不死之領域:凡往彼岸的人們,

   當被詢問時我告訴他們:無\twnr{依著}{198.0}的真理。」

  「\twnr{大德}{45.0}!
猶如在村落或城鎮的不遠處有蓮花池,在那裡有螃蟹。大德!那時,眾多男孩或女孩從那村落或城鎮走出去後,去那個蓮花池。抵達後[跳入那個蓮花池後],從水中抓起那隻螃蟹後,使之住立在陸地上。大德!不論那隻螃蟹使哪個蟹爪轉動,就在那時那些男孩或女孩就以木棒或以小石切斷、打裂、破壞它。大德!這樣,那隻螃蟹以全都切斷的、打裂的、破壞的蟹爪,不能夠[再]進入那個蓮花池,猶如之前。同樣的,大德!凡任何歪曲的、相違的、扭曲的,那些全都被世尊切斷、打裂、破壞[\ccchref{MN.35}{https://agama.buddhason.org/MN/dm.php?keyword=35}, 360段]。大德!現在,我不能再以期待機會接近世尊了。」

  那時,魔波旬在世尊的面前以這些厭的偈頌說:

  「脂肪容色的岩石,烏鴉繞行,

   在那裡或許我們能發現柔軟的,或許會有\twnr{樂味}{295.0}。

   在那裡沒得到樂味,烏鴉會從此離開,

   如烏鴉攻擊岩石般,我們厭\twnr{喬達摩}{80.0}後離開。」



\sutta{25}{25}{魔的女兒經}{https://agama.buddhason.org/SN/sn.php?keyword=4.25}
  那時,魔\twnr{波旬}{49.0}在\twnr{世尊}{12.0}面前以這些厭的偈頌說後,從那處離開後,在世尊的不遠處沈默地、羞愧地、垂肩地、低頭地、鬱悶地、不能反應地以盤腿坐在地上,以木棒刮著地。

  那時,魔的女兒:渴愛、不喜樂、貪,去見魔波旬。抵達後,以偈頌對魔波旬說:

  「父親!為何你是不快樂的?你對哪位男子憂愁呢?

   我們以貪網對他,如對\twnr{林野}{142.0}的象,

   繫縛後我們將帶來,他將成為被你控制者。」

  「世間中的\twnr{阿羅漢}{5.0}、\twnr{善逝}{8.0},是不易以貪誘惑的,

   他已超越魔之領域,因此我非常地憂愁。」

  那時,魔的女兒:渴愛、不喜樂、貪,去見世尊。抵達後,對世尊說這個:

  「\twnr{沙門}{29.0}!使我們在你的足下服侍。」

  那時,世尊不理會(不作意),正如他是在無上\twnr{依著}{198.0}盡滅上的解脫者那樣。

  那時,魔的女兒:渴愛、不喜樂、貪,離開到一旁後這麼計劃:

  「男人的欲求是多方面的,讓我們一一化作一百個少女模樣。」

  那時,魔的女兒:渴愛、不喜樂、貪,一一化作一百個少女模樣後,去見世尊。抵達後,對世尊說這個:

  「沙門!使我們在你的足下服侍。」

  那時,世尊不理會,正如他是在無上依著盡滅上的解脫者那樣。

  那時,魔的女兒:渴愛、不喜樂、貪,離開到一旁後這麼計劃:

  「男人的欲求是多方面的,讓我們一一化作一百個未生過小孩的女人模樣。」

  那時,魔的女兒:渴愛、不喜樂、貪,一一化作一百個未生過小孩的女人模樣後,去見世尊。抵達後,對世尊說這個:

  「沙門!使我們在你的足下服侍。」

  那時,世尊不理會,正如他是在無上依著盡滅上的解脫者那樣。

  那時,魔的女兒:渴愛……(中略)讓我們一一化作一百個生過一次小孩的女人模樣。

  那時,魔的女兒:渴愛……(中略)一一化作一百個生過一次小孩的女人模樣後,去見世尊。抵達後,對世尊說這個:

  「沙門!使我們在你的足下服侍。」

  那時,世尊不理會,正如他是在無上依著盡滅上的解脫者那樣。

  那時,魔的女兒:渴愛……(中略)讓我們一一化作一百個生過兩次小孩的女人模樣。

  那時,魔的女兒:渴愛……(中略)一一化作一百個生過兩次小孩的女人模樣後,去見世尊。……(中略)正如他是在無上依著盡滅上的解脫者那樣。

  那時,魔的女兒:渴愛……(中略)讓我們一一化作一百個中年女人模樣。

  那時,魔的女兒:渴愛……(中略)一一化作一百個中年女人模樣後……(中略)正如他是在無上依著盡滅上的解脫者那樣。

  那時,魔的女兒:渴愛……(中略)讓我們一一化作一百個老年女人模樣。

  那時,魔的女兒:渴愛……(中略)一一化作一百個老年女人模樣後,去見世尊。……(中略)正如他是在無上依著盡滅上的解脫者那樣。

  那時,魔的女兒:渴愛、不喜樂、貪,離開到一旁後說這個:「確實是真的,父親對我們說:

  『世間中的阿羅漢、善逝,是不易以貪誘惑的, 

   他已超越魔之領域,因此我非常地憂愁。』

  因為凡如果我們以這些攻擊,攻擊未離貪的沙門或\twnr{婆羅門}{17.0},他的心臟會破裂,或熱血會從口中湧出,或會到達瘋狂,或心散亂。又或猶如被割斷的綠蘆葦乾枯、枯萎、凋謝。同樣的,他會乾枯、枯萎、凋謝。」

  那時,魔的女兒:渴愛、不喜樂、貪,去見世尊。抵達後,在一旁站立。在一旁站立的魔的女兒渴愛,以偈頌對世尊說:

  「你已陷入愁而在林中修禪不是嗎?財產已失去或者欲求著不是嗎? 

   你在村落中做了什麼罪行不是嗎?你為什麼不與人作朋友呢? 

   為何你的友情不生起呢?」[\suttaref{SN.4.24}]

  「目標的達成、心的寂靜,征服可愛、\twnr{合意形色}{962.0}的軍隊後,

   我獨自修禪而\twnr{隨覺樂}{x131},因此我不與人作朋友,

   任何我的友情不生起。」

  那時,魔的女兒不喜樂,以偈頌對世尊說:

  「這裡怎樣多住的\twnr{比丘}{31.0},\twnr{已渡過五暴流}{x132}這裡渡過第六?

   他怎樣多修禪而[使]諸欲想,成為沒得到他的局外者?」

  「\twnr{身已寧靜}{318.0}、心善解脫,不造作、有念、無家,

   了知法後的無尋之禪定者,他不發怒、\twnr{不流動}{x133}、不惛沈。

   這裡這樣多住的比丘,已渡過五暴流這裡渡過第六,

   他這樣多修禪而[使]諸欲想,成為沒得到他的局外者。」

  那時,魔的女兒貪,以偈頌對世尊說:

  「\twnr{切斷渴愛後}{x134}與群眾僧團生活者,且許多有信者確實將實行,

   唉!這位無家者對眾人,搶奪後引導到死神之王的彼岸。」

  「大英雄、如來,他們確實以正法引導,

   在經由法引導者們中,了知者有什麼嫉妒?」

  那時,魔的女兒:渴愛、不喜樂、貪,去見魔波旬。

  魔波旬看見正從遠處到來的魔的女兒:渴愛、不喜樂、貪。看見後,以偈頌說:

  「愚者!以蓮花莖,你們劈山,

   你們以指甲挖山,你們以牙齒咬鐵。

   如以頭舉起岩石後,你們在深淵中尋求立足處,

   如以胸攻擊殘株,厭後你們離開喬達摩。」

  「她們輝耀地來:渴愛、不喜樂、貪,

   在那裡\twnr{大師}{145.0}驅散她們,\twnr{如風對落下的棉花}{x135}。

  第三品,其\twnr{攝頌}{35.0}:

  「眾多、三彌提,瞿低迦、七年,

   女兒-這是,佛陀以最勝教導的魔之五則。」

  魔相應完成。





\page

\xiangying{5}{比丘尼相應}
\sutta{1}{1}{阿羅毘迦經}{https://agama.buddhason.org/SN/sn.php?keyword=5.1}
  \twnr{被我這麼聽聞}{1.0}:

  \twnr{有一次}{2.0},\twnr{世尊}{12.0}住在舍衛城祇樹林給孤獨園。

  那時,阿羅毘迦\twnr{比丘尼}{31.0}午前時穿衣、拿起衣鉢後,\twnr{為了托鉢}{87.0}進入舍衛城。

  在舍衛城為了托鉢行走後,\twnr{餐後已從施食返回}{512.0},獨處欲求者去\twnr{盲者的樹林}{88.0}。

  那時,魔波旬想要阿羅毘迦比丘尼生出害怕、僵硬狀態、\twnr{身毛豎立的}{152.0};想要使她從獨處撤退而去見阿羅毘迦比丘尼。抵達後,以\twnr{偈頌}{281.0}對阿羅毘迦比丘尼說:

  「世間中沒有\twnr{出離}{294.0},你以獨處將做什麼?

   請你享受欲、喜樂,不要以後成為後悔者。」

  那時,阿羅毘迦比丘尼想這個:

  「誰說偈頌?這是人或\twnr{非人}{130.0}?」

  那時,阿羅毘迦比丘尼想這個:

  「這是魔波旬,他想要使我生出害怕、僵硬、身毛豎立;想要使我從獨處撤退而說偈頌。」

  那時,阿羅毘迦比丘尼知道:「這是魔波旬。」後,以偈頌回應魔波旬:

  「世間中有出離,被我以慧善觸達,

   波旬!放逸者的親族,你不知道那個境界(足跡)。

   欲如刀叉,諸蘊是它們的砧板,

   凡你所說欲、喜樂者,那對我是不喜樂。」

  那時,魔波旬:「阿羅毘迦比丘尼知道我。」痛苦地、不快意地就在那裡消失。



\sutta{2}{2}{受摩經}{https://agama.buddhason.org/SN/sn.php?keyword=5.2}
  起源於舍衛城。

  那時,受摩\twnr{比丘尼}{31.0}午前時穿衣、拿起衣鉢後,\twnr{為了托鉢}{87.0}進入舍衛城。

  在舍衛城為了托鉢行走後,\twnr{餐後已從施食返回}{512.0},\twnr{為了白天的住處}{128.0}去\twnr{盲者的樹林}{88.0}。

  進入往盲者的樹林後,坐在某棵樹下為了白天的住處。

  那時,魔波旬想要使受摩比丘尼生出害怕、僵硬狀態、\twnr{身毛豎立的}{152.0};想要使她從定撤退而去見受摩比丘尼。抵達後,以\twnr{偈頌}{281.0}對受摩比丘尼說:

  「凡那個能被仙人到達之處(境界),是難達到的,

   那個以女人\twnr{二指慧}{x136},不能到達。」

  那時,受摩比丘尼想這個:

  「誰說偈頌?這是人或\twnr{非人}{130.0}?」

  那時,受摩比丘尼想這個:

  「這是魔波旬,他想要使我生出害怕、僵硬、身毛豎立;想要使我從定撤退而說偈頌。」

  那時,受摩比丘尼知道:「這是魔波旬。」後,以偈頌回應魔波旬:

  「女人的狀態會有(作)什麼差別?在心已善入定時,

   \twnr{在智轉起中時}{x137},當正確地對法作毘婆舍那時。

   確實有想這個者:『我是女人或男人』,

   又或我是其他任何者,魔適合說那個。」

  那時,魔波旬:「受摩比丘尼知道我。」痛苦地、不快意地就在那裡消失。



\sutta{3}{3}{居沙喬達彌經}{https://agama.buddhason.org/SN/sn.php?keyword=5.3}
  起源於舍衛城。

  那時,居沙喬達彌\twnr{比丘尼}{31.0}午前時穿衣、拿起衣鉢後,\twnr{為了托鉢}{87.0}進入舍衛城。

  在舍衛城為了托鉢行走後,\twnr{餐後已從施食返回}{512.0},\twnr{為了白天的住處}{128.0}去\twnr{盲者的樹林}{88.0}。

  進入往盲者的樹林後,坐在某棵樹下為了白天的住處。

  那時,魔波旬想要使居沙喬達彌比丘尼生出害怕、僵硬狀態、\twnr{身毛豎立的}{152.0};想要使她從定撤退而去見居沙喬達彌比丘尼。抵達後,以\twnr{偈頌}{281.0}對居沙喬達彌比丘尼說:

  「你為何像死了兒子那樣呢?獨自有哭泣的臉,

   獨自進入林園,你尋找男人嗎?」

  那時,居沙喬達彌比丘尼想這個:

  「誰說偈頌?這是人或\twnr{非人}{130.0}?」

  那時,居沙喬達彌比丘尼想這個:

  「這是魔波旬,他想要使我生出害怕、僵硬、身毛豎立;想要使我從定撤退而說偈頌。」

  那時,居沙喬達彌比丘尼知道:「這是魔波旬。」後,以偈頌回應魔波旬:

  「\twnr{我是兒子死亡的終結者}{x138},\twnr{為這個之男人們的結束者}{x139},

   我不憂愁我不哭泣,我不怕你,\twnr{朋友}{201.0}!

   到處歡喜已被破壞,黑闇聚集已被碎破,

   征服死神軍後,我住於無\twnr{漏}{188.0}。」

  那時,魔波旬:「居沙喬達彌比丘尼知道我。」痛苦地、不快意地就在那裡消失。



\sutta{4}{4}{毘闍耶經}{https://agama.buddhason.org/SN/sn.php?keyword=5.4}
  起源於舍衛城。

  那時,毘闍耶\twnr{比丘尼}{31.0}午前時穿好衣服後……(中略)坐在某棵樹下\twnr{為了白天的住處}{128.0}。

  那時,魔波旬想要使毘闍耶比丘尼生出害怕、僵硬狀態、\twnr{身毛豎立的}{152.0};想要使她從定撤退而去見毘闍耶比丘尼。抵達後,以\twnr{偈頌}{281.0}對毘闍耶比丘尼說:

  「你是年輕美女子,而我是年輕少男,

   \twnr{以五種樂器}{x140},來!聖尼!我們尋歡。」

  那時,毘闍耶比丘尼想這個:

  「誰說偈頌?這是人或\twnr{非人}{130.0}?」

  那時,毘闍耶比丘尼想這個:

  「這是魔波旬,他想要使我生出害怕、僵硬、身毛豎立;想要使我從定撤退而說偈頌。」

  那時,毘闍耶比丘尼知道:「這是魔波旬。」後,以偈頌回應魔波旬:

  「色、聲音、氣味、味道,與\twnr{所觸}{220.2}之諸悅意的,

   我就贈與你,魔!因為我是無欲求者。

   以這個腐爛的,破壞的、易壞的身體,

   我厭惡我羞恥,欲、渴愛已被根除。

   凡進入色的眾生,與凡住於無色的,

   以及凡寂靜\twnr{等至}{129.0}的:在一切處黑暗已被破壞。」

  那時,魔波旬:「毘闍耶比丘尼知道我。」痛苦地、不快意地就在那裡消失。



\sutta{5}{5}{蓮華色經}{https://agama.buddhason.org/SN/sn.php?keyword=5.5}
  起源於舍衛城。

  那時,蓮華色\twnr{比丘尼}{31.0}午前時穿好衣服後……(中略)站在某棵盛開花朵的沙羅樹下。

  那時,魔波旬想要使蓮華色比丘尼生出害怕、僵硬狀態、\twnr{身毛豎立的}{152.0};想要使她從定撤退而去見蓮華色比丘尼。抵達後,以\twnr{偈頌}{281.0}對蓮華色比丘尼說:

  「比丘尼!走進頂端盛開花朵的後,你獨自站在沙羅樹下,

   你的美麗外表是無雙的,傻姑娘!你不怕壞人?」

  那時,蓮華色比丘尼想這個:

  「誰說偈頌?這是人或\twnr{非人}{130.0}?」

  那時,蓮華色比丘尼想這個:

  「這是魔波旬,他想要使我生出害怕、僵硬、身毛豎立;想要使我從定撤退而說偈頌。」

  那時,蓮華色比丘尼知道:「這是魔波旬。」後,以偈頌回應魔波旬:

  「即使如果有十萬個壞人,像你那樣的到達這裡者,

   我不搖動身毛我不驚怖,魔!我一個人也不怕你。」

  「這個我隱沒,或進入你的肚子,

   甚至對在睫毛中間,站立的我你沒看見。」

  「我是在心上已成為自在者,神足已善\twnr{修習}{94.0}者,

   我是已解脫一切繫縛者,我不怕你,\twnr{朋友}{201.0}!」

  那時,魔波旬:「蓮華色比丘尼知道我。」痛苦地、不快意地就在那裡消失。



\sutta{6}{6}{遮羅經}{https://agama.buddhason.org/SN/sn.php?keyword=5.6}
  起源於舍衛城。

  那時,遮羅\twnr{比丘尼}{31.0}午前時穿好衣服後……(中略)坐在某棵樹下\twnr{為了白天的住處}{128.0}。

  那時,魔波旬去見遮羅比丘尼。抵達後,對遮羅比丘尼說這個:

  「比丘尼!你不同意什麼呢?」

  「\twnr{朋友}{201.0}!我不同意出生。」

  「你為何不同意出生?被出生者享受諸欲,

   誰使你接受這個:『比丘尼!你不要喜歡出生!』呢?」

  「對被出生者來說有死亡,被出生者接觸諸苦:

   捕縛、殺害、苦難,因此不應該喜歡出生。

   佛陀教導法:對出生的超越,

   為了捨斷一切苦,他使我在真理上安頓。

   凡進入色的眾生,與凡住於無色的,

   不知道滅者,為返回再有者。」

  那時,魔波旬:「遮羅比丘尼知道我。」痛苦地、不快意地就在那裡消失。



\sutta{7}{7}{優波遮羅經}{https://agama.buddhason.org/SN/sn.php?keyword=5.7}
  起源於舍衛城。

  那時,優波遮羅\twnr{比丘尼}{31.0}午前時穿好衣服後……(中略)坐在某棵樹下\twnr{為了白天的住處}{128.0}。

  那時,魔波旬去見優波遮羅比丘尼。抵達後,對優波遮羅比丘尼說這個:

  「比丘尼!你想要往生哪裡?」

  「\twnr{朋友}{201.0}!我不想往生任何地方。」

  「三十三與夜摩,還有兜率天神,

   化樂天神,那些自在天神,

   請你將心志向那裡,你將體驗喜樂。」

  「三十三與夜摩,還有兜率天神,

   化樂天神,那些自在天神,

   他們被欲的繫縛所繫縛,再來到魔的控制。

   一切世界都已燃燒,一切世界都已冒煙,

   一切世界都已熾然,一切世界都已顫抖。

   不動搖的、不熾然的,一般人(凡夫)不親近的,

   於魔的不去之處,該處是我的意(心)喜好的。」

  那時,魔波旬:「優波遮羅比丘尼知道我。」痛苦地、不快意地就在那裡消失。



\sutta{8}{8}{尸蘇波遮羅經}{https://agama.buddhason.org/SN/sn.php?keyword=5.8}
  起源於舍衛城。

  那時,尸蘇波遮羅\twnr{比丘尼}{31.0}午前時穿好衣服後……(中略)坐在某棵樹下\twnr{為了白天的住處}{128.0}。

  那時,魔波旬去見尸蘇波遮羅比丘尼。抵達後,對尸蘇波遮羅比丘尼說這個:

  「比丘尼!你同意誰的\twnr{教條}{x141}呢?」

  「\twnr{朋友}{201.0}!我不同意任何教條。」

  「你是指定誰後剃頭的呢?你看起來像\twnr{沙門尼}{29.0},

   而你不同意教條,你為何像愚鈍者般地行走呢?」

  「在這裡之外的教條者,他們在見上\twnr{淨信}{340.0},

   我不同意他們的法,他們是法的不熟知者。

   有位在釋迦族出生者,佛陀、無與倫比者,

   戰勝一切者、驅逐魔者,在一切處不敗者,

   在一切處解脫的不依止者,\twnr{有眼者}{629.0}看見一切。

   到達一切業滅盡者,在\twnr{依著}{198.0}之盡滅上解脫者,

   那位\twnr{世尊}{12.0}是我的\twnr{大師}{145.0},我同意他的教說。」

  那時,魔波旬:「尸蘇波遮羅比丘尼知道我。」痛苦地、不快意地就在那裡消失。



\sutta{9}{9}{謝勒經}{https://agama.buddhason.org/SN/sn.php?keyword=5.9}
  起源於舍衛城。

  那時,謝勒\twnr{比丘尼}{31.0}午前時穿好衣服後……(中略)坐在某棵樹下\twnr{為了白天的住處}{128.0}。

  那時,魔波旬想要使謝勒比丘尼生出害怕、僵硬狀態、\twnr{身毛豎立的}{152.0}……(中略)以\twnr{偈頌}{281.0}對謝勒比丘尼說:

  「這個形體被誰造作?形體的作者在哪裡?

   形體在哪裡被生起呢?形體在哪裡被滅呢?」

  那時,謝勒比丘尼想這個:

  「誰說偈頌?這是人或\twnr{非人}{130.0}?」

  那時,謝勒比丘尼想這個:

  「這是魔波旬,他想要使我生出害怕、僵硬、身毛豎立;想要使我從定撤退而說偈頌。」

  那時,謝勒比丘尼知道:「這是魔波旬。」後,以偈頌回應魔波旬:

  「這個形體不是自我所作的,這個\twnr{痛苦}{752.0}不是其他者所作的,

   是\twnr{緣於}{252.0}因所生成的,以因的壞滅而被滅。

   如某顆種子,被播種在田中生長:

   由於土地的作用,與溼潤那兩者。

   這樣蘊與界,以及這六處,

   是緣於因所生成的,以因的壞滅而被滅。」

  那時,魔波旬:「謝勒比丘尼知道我。」痛苦地、不快意地就在那裡消失。



\sutta{10}{10}{金剛經}{https://agama.buddhason.org/SN/sn.php?keyword=5.10}
  起源於舍衛城。

  那時,金剛\twnr{比丘尼}{31.0}午前時穿衣、拿起衣鉢後,\twnr{為了托鉢}{87.0}進入舍衛城。在舍衛城為了托鉢行走後,\twnr{餐後已從施食返回}{512.0},\twnr{為了白天的住處}{128.0}去\twnr{盲者的樹林}{88.0}。

  進入盲者的樹林後,坐在某棵樹下為了白天的住處。

  那時,魔波旬想要使金剛比丘尼生出害怕、僵硬狀態、\twnr{身毛豎立的}{152.0};想要使她從定撤退而去見金剛比丘尼。抵達後,以\twnr{偈頌}{281.0}對金剛比丘尼說:

  「這眾生被誰造作?眾生的作者在哪裡?

   眾生在哪裡被生起呢?眾生在哪裡被滅呢?」

  那時,金剛比丘尼想這個:

  「誰說偈頌?這是人或\twnr{非人}{130.0}?」

  那時,金剛比丘尼想這個:

  「這是魔波旬,他想要使我生出害怕、僵硬、身毛豎立;想要使我從定撤退而說偈頌。」

  那時,金剛比丘尼知道:「這是魔波旬。」後,以偈頌回應魔波旬:

  「為何你相信『眾生』呢?魔!那是你的\twnr{惡見}{722.0}嗎?

   這是單純的行之堆積,這裡沒有眾生被發現。

   如以各部分的集起,有『車子』的話語,

   這樣在有諸蘊時,有『眾生』的俗稱。

   僅苦生成,苦存續與消失,

   無除了苦生成之外的,無除了苦被滅之外的。」

  那時,魔波旬:「金剛比丘尼知道我。」痛苦地、不快意地就在那裡消失。

  比丘尼相應完成,其\twnr{攝頌}{35.0}:

  「阿羅毘迦與受摩,喬達彌與毘闍耶,

   蓮華色與遮羅,優波遮羅與尸蘇波遮羅,

   謝勒與金剛它們為十。」





\page

\xiangying{6}{梵天相應}
\pin{第一品}{1}{10}
\sutta{1}{1}{梵天勸請經}{https://agama.buddhason.org/SN/sn.php?keyword=6.1}
  \twnr{被我這麼聽聞}{1.0}:

  \twnr{有一次}{2.0},初\twnr{現正覺}{75.0}的\twnr{世尊}{12.0}住在優樓頻螺,尼連禪河邊牧羊人的榕樹處。

  那時,當世尊獨處、\twnr{獨坐}{92.0}時,這樣心的深思生起:

  「被我到達(證得)的這個法是甚深的、難見的、難隨覺的、寂靜的、勝妙的、\twnr{超越推論的}{718.0}、微妙的、被賢智者體驗的。然而,這\twnr{世代}{38.0}有\twnr{阿賴耶}{391.0}的快樂,樂於阿賴耶,\twnr{喜於阿賴耶}{633.0}。又,對有阿賴耶的快樂,樂於阿賴耶,喜於阿賴耶的世代,此處是難見的,即:\twnr{特定條件性}{636.0}、\twnr{緣起}{225.0};此處也是難見的,即:\twnr{一切行的止}{55.0}、一切\twnr{依著}{198.0}的\twnr{斷念}{211.0}、渴愛的滅盡、\twnr{離貪}{77.0}、\twnr{滅}{68.0}、涅槃。那樣的話,如果我教導法,對方不能了解我,那對我是疲勞,\twnr{那對我是傷害}{898.0}。」

  於是,這些在以前前所未聞的,\twnr{不可思議}{924.0}的\twnr{偈頌}{281.0}在世尊心中出現:

  「被我困難地到達的,現在沒有被說明的必要,

   被貪瞋征服者,此法是不容易正覺的。

   逆流而行的、微妙的,甚深的、難見的、精細的[法],

   被貪染著者沒看見:被大黑暗覆蓋者。」

  在這裡,當世尊像這樣深慮時,心傾向\twnr{不活動}{906.0}、無法的教導。

  那時,\twnr{梵王娑婆主}{215.0}以心了知世尊心中的深思後,想這個:「唉!\twnr{先生}{202.0}!世界滅亡,唉!先生!世界消失,確實是因為世尊、\twnr{阿羅漢}{5.0}、\twnr{遍正覺者}{6.0}的心傾向不活動,無法的教導。」那時,梵王娑婆主就猶如有力氣的男子伸直彎曲的手臂,或彎曲伸直的手臂,就像這樣在梵天世界消失,出現在世尊的面前。那時,梵王娑婆主置(作)上衣到一邊肩膀後,右膝蓋觸地、向世尊\twnr{合掌}{377.0}鞠躬後,對世尊說這個:

  「\twnr{大德}{45.0}!請世尊教導法!請\twnr{善逝}{8.0}教導法!有少塵垢之類的眾生以法的未聽聞情況而退失,他們將會是法的了知者。」

  梵王娑婆主說這個,說這個後,他又更進一步說這個:

  「從前在摩揭陀出現,被有垢者構思的不清淨法,

   請你打開這\twnr{不死}{123.0}之門,令他們聽聞被離垢者\twnr{領悟}{355.0}之法。

   如在岩山山頂上站立者,能到處看見人那樣,

   極聰明者!如像那樣的,\twnr{一切眼者}{629.0}!登上法所成高樓後,

   已離愁者請你看陷入愁,被生老征服的人們。

   英雄!戰場上的勝利者!請你起來,商隊領袖!無負債者!請你在世間走動,

   世尊請你教導法,將會有了知者。」

  那時,世尊知道梵天的勸請後,\twnr{緣於}{252.0}對眾生的悲愍,以\twnr{佛眼}{629.0}觀察世間。當世尊以佛眼觀察世間時,看見少塵垢的、多塵垢的;利根的、鈍根的;\twnr{善行相的}{640.0}、惡行相的;易受教的、難受教的;一些住於看見在其他世界的罪過與恐怖的、另一些不住於看見在其他世界的罪過與恐怖的眾生,就猶如在青蓮池、紅蓮池、白蓮池中,一些青蓮、紅蓮、白蓮被生於水中,被長於水中,不被上升水中,被沈在內部養育的;一些青蓮、紅蓮、白蓮被生於水中,被長於水中,與水面同高而住立;一些青蓮、紅蓮、白蓮被生於水中,被長於水中,從水中上升後住立,不被水污染。同樣的,當世尊以佛眼觀察世間時,看見少塵垢的、多塵垢的;利根的、鈍根的;善行相的、惡行相的;易受教的、難受教的;一些住於看見在其他世界的罪過與恐怖的、另一些不住於看見在其他世界的罪過與恐怖的眾生。看見後,以偈頌回答梵王娑婆主:

  「不死之門已對他們開啟,讓那些有耳者捨棄[邪]信,

   \twnr{惱害想的熟知者}{669.0},梵天!我不在人間說勝妙法。」

  那時,梵王娑婆主[心想]:

  「對法的教導,我已被世尊允許。」向世尊\twnr{問訊}{46.0}、\twnr{作右繞}{47.0}後,就在那裡消失。[\ccchref{MN.26}{https://agama.buddhason.org/MN/dm.php?keyword=26},281段/\ccchref{MN.85}{https://agama.buddhason.org/MN/dm.php?keyword=85},337段/\ccchref{DN.14}{https://agama.buddhason.org/DN/dm.php?keyword=14},64段]



\sutta{2}{2}{尊重經}{https://agama.buddhason.org/SN/sn.php?keyword=6.2}
  \twnr{被我這麼聽聞}{1.0}:

  \twnr{有一次}{2.0},初\twnr{現正覺}{75.0}的\twnr{世尊}{12.0}住在優樓頻螺,尼連禪河邊牧羊人的榕樹處。那時,當世尊獨處、\twnr{獨坐}{92.0}時,這樣心的深思生起:「不尊重、不順從者住於苦,哪位\twnr{沙門}{29.0}或\twnr{婆羅門}{17.0}我應該恭敬、尊敬、依止後而生活(住)呢?」

  那時,世尊想這個:「為了未完成\twnr{戒蘊}{374.0}的完成,恭敬、尊敬、依止其他沙門或婆羅門後,我應該生活,然而,我不見在包括天,在包括魔,在包括梵的世間;在包括沙門婆羅門,在包括天-人的\twnr{世代}{38.0}中,比我更具足戒的其他沙門或婆羅門,凡我恭敬、尊敬、依止後應該生活。

  為了未完成定蘊的完成,恭敬、尊敬、依止其他沙門或婆羅門後,我應該生活,然而,我不見在包括天……(中略)比我更具足定的其他沙門或婆羅門,凡我恭敬、尊敬、依止後應該生活。

  為了未完成慧蘊的完成,恭敬、尊敬、依止其他沙門或婆羅門後,我應該生活,然而,我不見在包括天……(中略)比我更具足慧的其他沙門或婆羅門,凡我恭敬、尊敬、依止後應該生活。

  為了未完成解脫蘊的完成,恭敬、尊敬、依止其他沙門或婆羅門後,我應該生活,然而,我不見在包括天……(中略)比我更具足解脫的其他沙門或婆羅門,凡我恭敬、尊敬、依止後應該生活。

  為了未完成的\twnr{解脫智見}{27.0}蘊的完成,恭敬、尊敬、依止其他沙門或婆羅門後,我應該生活,然而,我不見在包括天,在包括魔,在包括梵的世間;在包括沙門婆羅門,在包括天-人的世代中,比我更具足解脫智見的其他沙門或婆羅門,凡我恭敬、尊敬、依止後應該生活。讓我凡這個被我現正覺的法,就對那個法恭敬、尊敬、依止這後生活。」

  那時,\twnr{梵王娑婆主}{215.0}以心了知世尊心中的深思後,就猶如有力氣的男子伸直彎曲的手臂,或彎曲伸直的手臂,就像這樣在梵天世界消失,出現在世尊的面前。

  那時,梵王娑婆主置(作)上衣到一邊肩膀,向世尊\twnr{合掌}{377.0}鞠躬後,對世尊說這個:「這是這樣,世尊!這是這樣,\twnr{善逝}{8.0}!\twnr{大德}{45.0}!又凡那些過去時是\twnr{阿羅漢}{5.0}、遍正覺者們,那些世尊也只恭敬、尊敬、依止法後生活;大德!又凡那些未來時將是阿羅漢、遍正覺者們,那些世尊也只恭敬、尊敬、依止法後將生活;大德!又世尊現在是阿羅漢、遍正覺者,也請只恭敬、尊敬、依止法後生活。」梵王娑婆主說這個,說這個後,又更進一步說這個:

  「凡過去的諸\twnr{正覺者}{185.1},與凡未來的諸佛,

   以及凡現在正覺者:許多愁的破壞者。

   全部對正法的尊敬,他們曾生活與生活,

   還有也將生活,這是諸佛的法性。

   因此\twnr{以愛惜自己}{943.0},以期待偉大者,

   正法應該被尊敬:憶念諸佛陀教說者。」[\ccchref{AN.4.21}{https://agama.buddhason.org/AN/an.php?keyword=4.21}]



\sutta{3}{3}{梵天經}{https://agama.buddhason.org/SN/sn.php?keyword=6.3}
  \twnr{被我這麼聽聞}{1.0}:

  \twnr{有一次}{2.0},\twnr{世尊}{12.0}住在舍衛城祇樹林給孤獨園。

  當時,有某位女婆羅門的兒子名叫梵天,在世尊的面前\twnr{從在家出家成為無家者}{48.0}。

  那時,住於單獨的、隱離的、不放逸的、熱心的、自我努力的\twnr{尊者}{200.0}梵天不久就以證智自作證後,在當生中\twnr{進入後住於}{66.0}凡\twnr{善男子}{41.0}們為了利益正確地從在家出家成為無家者的那個無上梵行結尾,他證知:「\twnr{出生已盡}{18.0},\twnr{梵行已完成}{19.0},\twnr{應該被作的已作}{20.0},\twnr{不再有此處[輪迴]的狀態}{21.1}。」然後尊者梵天成為眾\twnr{阿羅漢}{5.0}之一。

  那時,尊者梵天午前時穿衣、拿起衣鉢後,\twnr{為了托鉢}{87.0}進入舍衛城。

  當在舍衛城\twnr{為了托鉢次第地行走著}{127.0}時,他來到自己母親住處。當時,尊者梵天的母親女婆羅門,常高舉對梵天(神)的供奉。那時,\twnr{梵王娑婆主}{215.0}想這個:

  「這位尊者梵天的母親女婆羅門,常高舉對梵天的供奉,讓我去見她後激起她的\twnr{急迫感}{373.0}。」

  那時,梵王娑婆主就猶如有力氣的男子伸直彎曲的手臂,或彎曲伸直的手臂,就像這樣在梵天世界消失,出現在尊者梵天的母親住處。

  那時,梵王娑婆主站在空中,對尊者梵天的母親女婆羅門以\twnr{偈頌}{281.0}說:

  「女婆羅門!梵天世界是在離這裡的遠處,你常高舉對祂的供奉,

   女婆羅門!梵天的食物不是這樣的,為何當不知道梵天之路時你喃喃(祈禱)呢?

   女婆羅門!確實你的這位梵天,是無\twnr{依著}{198.0}者、已到達超越天神者,

   無所有的、無養育其他者的\twnr{比丘}{31.0},他為了托鉢已進入你家。

   應該被供養者、明智者、\twnr{已自我修習}{658.0}者,應該被天與人供養者,

   拒斥諸惡後不被染著者,已清涼者對食物的尋求行動。

   \twnr{對他來說沒有前後}{x142},寂靜的、無[瞋恚]煙的、無苦惱的、無願望的,

   在懦弱者與堅強者上放下棍棒者,請他吃你供奉的最上食物。

   已成為無敵的(離軍團的)、心已寂靜的,行走如已調御的、無貪愛的龍象,

   善戒德的、心\twnr{善解脫}{28.0}的比丘,請他吃你供奉的最上食物。

   於他\twnr{淨信的}{340.0}、不動搖的,請你建立供養物在應該被供養者上,

   請你作未來安樂的福德:女婆羅門!看見渡過\twnr{暴流}{369.0}的\twnr{牟尼}{125.0}後。」

  「於他淨信的、不動搖的,她建立供養物在應該被供養者上。

   她作了未來安樂的福德:女婆羅門看見渡過暴流的牟尼後。」



\sutta{4}{4}{巴迦梵天經}{https://agama.buddhason.org/SN/sn.php?keyword=6.4}
  \twnr{被我這麼聽聞}{1.0}:

  \twnr{有一次}{2.0},\twnr{世尊}{12.0}住在舍衛城祇樹林給孤獨園。

  當時,巴迦梵天有像這樣的邪惡\twnr{惡見}{722.0}生起:

  「這是常的,這是堅固的,這是永恆的,這是全部的,這是不死亡法。確實,這個不被生、不衰老、不死亡、不死沒、不再生,而且,沒有其它從這裡超出的出離。」

  那時,世尊以心了知巴迦梵天心中的深思後,就猶如有力氣的男子伸直彎曲的手臂,或彎曲伸直的手臂,就像這樣在祇樹林消失,出現在那個梵天世界中。

  巴迦梵天看見正從遠處到來的世尊。看見後對世尊說這個:

  「來,\twnr{親愛的先生}{204.0}!歡迎你,親愛的先生!親愛的先生終於作這個安排,即:這裡的到來。親愛的先生!確實,這是常的,這是堅固的,這是永恆的,這是全部的,這是不死亡法。確實,這個不被生、不衰老、不死亡、不死沒、不再生,而且,沒有其它從這裡超出的出離。」

  在這麼說時,世尊對巴迦梵天說這個:

  「唉!\twnr{先生}{202.0}!巴迦梵天是\twnr{進入無明者}{645.0},唉!先生!巴迦梵天是進入無明者,確實是因為他將說:『無常的就等同是常的。』他將說:『不堅固的就等同是堅固的。』他將說:『非永恆的就等同是永恆的。』他將說:『非全部的就等同是全部的。』他將說:『衰變法就等同是不衰變法。』而且,在被生、衰老、死亡、死沒、再生之處,他將像這樣說它:『確實,這個不被生、不衰老、不死亡、不死沒、不再生。』而且,有其他超出的出離者,他將說:『沒有其它超出的出離。』」[\ccchref{MN.49}{https://agama.buddhason.org/MN/dm.php?keyword=49}]

  「\twnr{喬達摩}{80.0}!七十二位福業者,是有自在力者、生老的超越者,

   這是通曉吠陀者最終的梵天之往生,多數人祈求我們。」

  「巴迦!凡你認為長的壽命,這個壽命確實是少的、確實非長的,

   梵天!我了知你的壽命,是十萬\twnr{尼羅部陀}{885.2}。」

  「世尊!\twnr{我是無邊之見者}{x143},已越過生老愁者,

   什麼是我以前的\twnr{禁戒與德行之行法}{x144}?請你告訴我那個凡我能了知的。」

  「凡你使許多人喝飲:在炎暑中被口渴逼惱者,

   那是你以前的禁戒與德行之行法,我像從睡眠清醒般地回憶起。

   在耶尼河岸被捕捉的人們,你使可能被捕捉者被引導著脫離,

   那是你以前的禁戒與德行之行法,我像從睡眠清醒般地回憶起。

   在恒河水流中,被想要人的兇惡龍捕捉的船,

   你們以力量征服後使之釋放,那是你以前的禁戒與德行之行法,

   我像從睡眠清醒般地回憶起。

   我曾是你的\twnr{徒弟}{x145}迦葉,你認為\twnr{有正覺禁戒的}{x146},

   那是你以前的禁戒與德行之行法,我像從睡眠清醒般地回憶起。」

  「你的確知道我的這個壽命,你也知道其他人的-像這樣確實是佛陀,

   像這樣這確實是你輝耀的威力,持續使梵天世界閃耀著。」



\sutta{5}{5}{某位梵天經}{https://agama.buddhason.org/SN/sn.php?keyword=6.5}
  起源於舍衛城。

  當時,某位梵天有這樣邪惡的\twnr{惡見}{722.0}生起:「沒有那種\twnr{沙門}{29.0}或\twnr{婆羅門}{17.0}能到這裡的。」

  那時,\twnr{世尊}{12.0}以心了知那位梵天心中的深思後,猶如有力氣的男子……(中略)出現在那梵天世界中。

  那時,世尊以盤腿坐在那位梵天的上方虛空:\twnr{入火界定後}{972.0}。

  那時,\twnr{尊者}{200.0}大目揵連想這個:「世尊現在住於何處呢?」尊者大目揵連以清淨、超越常人的天眼看見以盤腿坐在那位梵天的上方虛空,入火界定的世尊。看見後,就猶如有力氣的男子伸直彎曲的手臂,或彎曲伸直的手臂,就像這樣在祇樹林消失,出現在那個梵天世界中。那時,尊者大目揵連依止東方後以盤腿坐在那位梵天的上方虛空,較世尊低:入火界定後。

  那時,尊者大迦葉想這個:「世尊現在住於何處呢?」尊者大迦葉以清淨、超越常人的天眼看見……(中略)。看見後,猶如有力氣的男子……(中略)就像這樣在祇樹林消失,出現在那個梵天世界中。那時,尊者大迦葉依止南方後以盤腿坐在那位梵天的上方虛空,較世尊低:入火界定後。

  那時,尊者大劫賓那想這個:「世尊現在住於何處呢?」尊者大劫賓那以清淨、超越常人的天眼看見……(中略)入火界定的世尊。看見後,猶如有力氣的男子……(中略)就像這樣在祇樹林消失,出現在那個梵天世界中。那時,尊者大劫賓那依止西方後以盤腿坐在那位梵天的上方虛空,較世尊低:入火界定後。

  那時,尊者阿那律想這個:「世尊現在住於何處呢?」尊者阿那律以清淨、超越常人的天眼看見……(中略)入火界定的世尊。看見後,猶如有力氣的男子……(中略)就像這樣在祇樹林消失,出現在那個梵天世界中。那時,尊者阿那律依止北方後以盤腿坐在那位梵天的上方虛空,較世尊低:入火界定後。

  那時,尊者大目揵連以\twnr{偈頌}{281.0}對那位梵天說:

  「\twnr{朋友}{201.0}!現在你還有那個見:凡你之前有的見,

   你看見輝耀者,在梵天世界中的超越者。」

  「\twnr{親愛的先生}{204.0}!我沒有那個見:凡我之前有的見,

   我看見輝耀者,在梵天世界中的超越者,

   現在那個我如何會說:我是常的、常恆的?」

  那時,世尊激起那位梵天的\twnr{急迫感}{373.0}後,就猶如有力氣的男子伸直彎曲的手臂,或彎曲伸直的手臂,就像這樣在那個梵天世界消失,出現在祇樹林中。

  那時,那位梵天召喚某位梵天眾:

  「來!親愛的先生!你去見尊者大目揵連。抵達後,請對尊者大目揵連說這個:『親愛的目揵連尊師!還有其他那位世尊的弟子這麼大神通力、這麼大威力猶如\twnr{尊師}{203.0}目揵連、迦葉、劫賓那、阿那律嗎?』」

  「是的,親愛的先生!」那位梵天眾回答那位梵天後,去見尊者目揵連。抵達後,對尊者目揵連說這個:「親愛的目揵連尊師!還有其他那位世尊的弟子這麼大神通力、這麼大威力猶如尊師目揵連、迦葉、劫賓那、阿那律嗎?」

  那時,尊者目揵連以偈頌對那位梵天眾說:

  「三明者、得神通者以及,知他心的熟練者,

   諸漏已滅盡的\twnr{阿羅漢}{5.0},佛陀的弟子有許多。」

  那時,那位梵天眾歡喜、\twnr{隨喜}{85.0}尊者目揵連所說後,去見那位梵天。抵達後,對那位梵天說這個:

  「親愛的先生!尊者目揵連說這個:

  『三明者、得神通者以及,知他心的熟練者,

   諸漏已滅盡的阿羅漢,佛陀的弟子有許多。』」

  那位梵天眾說這個,而那位悅意的梵天歡喜那位梵天眾所說。



\sutta{6}{6}{梵天世界經}{https://agama.buddhason.org/SN/sn.php?keyword=6.6}
  起源於舍衛城。

  當時,\twnr{世尊}{12.0}已進入\twnr{白天的住處}{128.0},已\twnr{獨坐}{92.0}。

  那時,善梵\twnr{辟支梵天}{884.0}與淨居辟支梵天去見世尊。抵達後,各依門兩側站立。

  那時,善梵辟支梵天對淨居辟支梵天說這個:

  「\twnr{親愛的先生}{204.0}!拜訪世尊,大致上是不適時的,世尊已進入白天的住處,已獨坐。某某梵天世界是富有的且繁榮的,而且在那裡的梵天住於放逸住,親愛的先生!我們走,我們將去那個梵天世界。抵達後,讓我們對那位梵天\twnr{激起急迫感}{373.0}。」

  「是的,親愛的先生!」淨居辟支梵天回答善梵辟支梵天。

  那時,善梵辟支梵天與淨居辟支梵天就猶如有力氣的男子能……(中略)就像這樣在世尊前消失,出現在那個梵天世界中。

  那位梵天看見正從遠處到來的那些梵天。看見後對那些梵天說這個:

  「喂!親愛的先生!你們從哪裡來?」

  「親愛的先生!我們是從那位世尊、\twnr{阿羅漢}{5.0}、遍正覺者的面前來的,又,親愛的先生!你應該去侍候那位世尊、阿羅漢、遍正覺者。」

  在這麼說時,不同意那個言語的那位梵天化作千個(回)自我後,對善梵辟支梵天說這個:

  「親愛的先生!你看到我像這樣的神通威力了嗎?」

  「親愛的先生!我看到你像這樣的神通威力了。」

  「親愛的先生!我是這麼大神通力、這麼大威力者,我要去侍候哪位其他\twnr{沙門}{29.0}或婆羅門呢?」

  那時,善梵辟支梵天化作兩千個自我後,對那位梵天說這個:

  「親愛的先生!你看到我像這樣的神通威力了嗎?」

  「親愛的先生!我看到你像這樣的神通威力了。」

  「親愛的先生!那位世尊就是比你與我有更大神通力與更大威力者,親愛的先生!你應該去侍候那位世尊、阿羅漢、遍正覺者。」

  那時,那位梵天以\twnr{偈頌}{281.0}對善梵辟支梵天說:

  「三[百]隻金翅鳥與四[百]隻天鵝,以及五百隻鷹-禪定者的,

   梵天!這個天宮發光,使北方閃耀著。」

  「即使你的那天宮發光,使北方閃耀著,

   在色上看見過失、經常顫抖的後,因此極聰明者不樂諸色。」

  那時,善梵辟支梵天與淨居辟支梵天激起那位梵天的急迫感後,就在那裡消失。

  過些時候,那位梵天去侍候那位世尊、阿羅漢、遍正覺者。



\sutta{7}{7}{瞿迦梨迦經}{https://agama.buddhason.org/SN/sn.php?keyword=6.7}
  起源於舍衛城。

  當時,\twnr{世尊}{12.0}已進入\twnr{白天的住處}{128.0},已\twnr{獨坐}{92.0}。

  那時,善梵\twnr{辟支梵天}{884.0}與淨居辟支梵天去見世尊。抵達後,各依門兩側站立。

  那時,善梵辟支梵天在世尊的面前說這關於瞿迦梨迦\twnr{比丘}{31.0}的\twnr{偈頌}{281.0}:

  「測量著\twnr{不能被測量者}{883.0},在這裡哪位智者能分類?

   不能被測量者的衡量者,我想他是被覆蓋的凡夫。」



\sutta{8}{8}{迦達摩大迦低舍經}{https://agama.buddhason.org/SN/sn.php?keyword=6.8}
  起源於舍衛城。

  當時,\twnr{世尊}{12.0}已進入\twnr{白天的住處}{128.0},已\twnr{獨坐}{92.0}。

  那時,善梵\twnr{辟支梵天}{884.0}與淨居辟支梵天去見世尊。抵達後,各依門兩側站立。

  那時,善梵辟支梵天在世尊的面前說這關於迦達摩大迦低舍迦\twnr{比丘}{31.0}的偈頌:

  「測量著\twnr{不能被測量者}{883.0},在這裡哪位智者能分類?

   不能被測量者的衡量者,我想他是被覆蓋的無慧者。」



\sutta{9}{9}{都路梵天經}{https://agama.buddhason.org/SN/sn.php?keyword=6.9}
  起源於舍衛城。

  當時,瞿迦梨迦\twnr{比丘}{31.0}是生病者、受苦者、重病者。

  那時,在夜已深時,容色絕佳的都路\twnr{辟支梵天}{884.0}使整個祇樹林發光後,去見瞿迦梨迦比丘。抵達後,站在空中,對瞿迦梨迦比丘說這個:

  「瞿迦梨迦!請你在舍利弗、目揵連上使心淨信,舍利弗、目揵連是\twnr{美善的}{947.0}。」

  「\twnr{朋友}{201.0}!你是誰?」

  「我是都路辟支梵天。」

  「朋友!你被\twnr{世尊}{12.0}記說為\twnr{不還者}{209.0},不是嗎?那樣的話,為何還是來這裡者呢?看!你的這個罪過有多少。」

  「當男子被生時,斧頭被生在口中,

   以此切斷自己:當愚癡者說惡語時。

   凡稱讚應該被責備者,或凡責備應該被稱讚者,

   他以口選擇不幸運的骰子,以那個不幸運的骰子找不到安樂。

   這個不幸運的骰子是小事:凡在骰子上有財產的損敗,

   甚至一切的也連同以自己,這就是更大不幸運的骰子:

   凡在\twnr{善逝}{8.0}上會使意(心)成為邪惡者。

   十萬\twnr{尼羅部陀}{885.2},[加]三十六、又五阿浮陀,

   凡呵責聖人者來到地獄:置邪惡的言語與意後。」[\ccchref{AN.10.89}{https://agama.buddhason.org/AN/an.php?keyword=10.89}]



\sutta{10}{10}{瞿迦梨迦經}{https://agama.buddhason.org/SN/sn.php?keyword=6.10}
  起源於舍衛城。

  那時,瞿迦梨迦\twnr{比丘}{31.0}去見\twnr{世尊}{12.0}。抵達後,向世尊\twnr{問訊}{46.0}後,在一旁坐下。在一旁坐下的瞿迦梨迦比丘對世尊說這個:

  「\twnr{大德}{45.0}!舍利弗、目揵連是惡欲求者,被惡欲求支配者。」

  在這麼說時,世尊對瞿迦梨迦比丘說這個:

  「瞿迦梨迦!不要這麼說!瞿迦梨迦!不要這麼說!瞿迦梨迦!請你在舍利弗、目揵連上使心淨信,舍利弗、目揵連是\twnr{美善的}{947.0}。」

  第二次,瞿迦梨迦比丘又對世尊說這個:

  「大德!即使世尊對我是值得信者、能信賴者,然大德!而舍利弗、目揵連是惡欲求者,被惡欲求支配者。」

  第二次,世尊又對瞿迦梨迦比丘說這個:

  「瞿迦梨迦!不要這麼說!瞿迦梨迦!不要這麼說!瞿迦梨迦!請你在舍利弗、目揵連上使心淨信,舍利弗、目揵連是美善的。」

  第三次,瞿迦梨迦比丘又對世尊說這個:

  「大德!即使……(中略)被惡欲求支配者。」

  第三次,世尊又對瞿迦梨迦比丘說這個:

  「瞿迦梨迦!不要這麼說!……(中略)舍利弗、目揵連是美善的。」

  那時,瞿迦梨迦比丘從座位起來、向世尊問訊、\twnr{作右繞}{47.0}後,離開。

  當瞿迦梨迦比丘離開不久,全身被遍滿芥子大小的膿腫,成為芥子大小後,成為綠豆大小;成為綠豆大小後,成為豌豆大小;成為豌豆大小後,成為棗核大小;成為棗核大小後,成為棗子大小;成為棗子大小後,成為餘甘子大小;成為餘甘子大小後,成為未成熟的木瓜大小;成為未成熟的木瓜大小後,成為木瓜大小;成為木瓜大小後破裂,膿汁與血液流出。

  那時,瞿迦梨迦比丘就以那個疾病命終。而瞿迦梨迦比丘命終時,在舍利弗、目揵連上心起瞋怒後往生紅蓮地獄。

  那時,在夜已深時,容色絕佳的\twnr{梵王娑婆主}{215.0}使整個祇樹林發光後,去見世尊。抵達後,向世尊問訊後,在一旁站立。在一旁站立的梵王娑婆主對世尊說這個:「大德!瞿迦梨迦比丘已命終,大德!瞿迦梨迦比丘命終時,在舍利弗、目揵連上心起瞋怒後往生紅蓮地獄。」梵王娑婆主說這個,說這個後,梵王娑婆主向世尊問訊、作右繞後,就在那裡消失。

  那時,那夜過後,世尊召喚比丘們:「比丘們!這夜,在夜已深時,容色絕佳的梵王娑婆主使整個祇樹林發光後,來見我。抵達後,向我問訊後,在一旁站立。比丘們!在一旁站立的梵王娑婆主對我說這個:『大德!瞿迦梨迦比丘已命終,大德!瞿迦梨迦比丘命終時,在舍利弗、目揵連上心起瞋怒後往生紅蓮地獄。』比丘們!梵王娑婆主說這個,說這個後,梵王娑婆主向世尊問訊、作右繞後,就在那裡消失。」

  在這麼說時,某位比丘對世尊說這個:

  「大德!在紅蓮地獄中,壽命量有多長呢?」

  「比丘!在紅蓮地獄中,壽命量是長的,那不容易計算『幾年』,或『幾百年』,或『幾千年』,或『幾十萬年』。」

  「大德!那麼能作譬喻嗎?」

  「比丘!能。」世尊說。

  「比丘!猶如二十\twnr{佉梨}{864.0}份量的憍薩羅國芝麻車,如果男子每過一百年從那裡取出一粒芝麻,比丘!二十佉梨份量的憍薩羅國芝麻車以這個行動會比較快地走到遍盡、耗盡,而非一\twnr{阿浮陀}{885.2}地獄。比丘!猶如二十阿浮陀地獄,這樣一尼羅部陀地獄。比丘!猶如二十尼羅部陀地獄,這樣一阿婆婆地獄。比丘!猶如二十阿婆婆地獄,這樣一阿得得地獄。比丘!猶如二十阿得得地獄,這樣一阿訶訶地獄。比丘!猶如二十阿訶訶地獄,這樣一古木達地獄。比丘!猶如二十古木達地獄,這樣一受更狄迦地獄。比丘!猶如二十受更狄迦地獄,這樣一優鉢羅地獄。比丘!猶如二十優鉢羅地獄,這樣一\twnr{分陀利}{264.3}迦地獄。比丘!猶如二十分陀利迦地獄,這樣一紅蓮地獄,比丘!瞿迦梨迦比丘在舍利弗、目揵連上心起瞋怒後往生紅蓮地獄。」

  世尊說這個,說這個後,\twnr{善逝}{8.0}、\twnr{大師}{145.0}又更進一步說這個:

  「當男子被生時,斧頭被生在口中,

   以此切斷自己:當愚癡者說惡語時。

   凡稱讚應該被責備者,或凡責備應該被稱讚者,

   他以口選擇不幸運的骰子,以那個不幸運的骰子找不到安樂。

   這個不幸運的骰子是小事:凡在骰子上有財產的損敗,

   甚至一切的也連同以自己,這就是更大不幸運的骰子:

   凡在\twnr{善逝}{8.0}上會使意(心)成為邪惡者。

   十萬\twnr{尼羅部陀}{885.2},[加]三十六、又五阿浮陀,

   凡呵責聖人者來到地獄:置邪惡的言語與意後。」[\ccchref{AN.10.89}{https://agama.buddhason.org/AN/an.php?keyword=10.89}]

  第一品,其\twnr{攝頌}{35.0}:

  「勸請、尊重、梵天,巴迦梵天與見在後,

   放逸、瞿迦梨迦、迦低舍迦,都路梵天與另一瞿迦梨迦。」





\pin{第二品}{11}{15}
\sutta{11}{11}{常童子經}{https://agama.buddhason.org/SN/sn.php?keyword=6.11}
  \twnr{有一次}{2.0},\twnr{世尊}{12.0}住在王舍城奢逼尼河邊。那時,在夜已深時,容色絕佳的\twnr{梵王常童子}{215.0}使整個奢逼尼河邊發光後,去見世尊。抵達後,向世尊\twnr{問訊}{46.0}後,在一旁站立。在一旁站立的梵王常童子在世尊的面前說這個\twnr{偈頌}{281.0}:

  「\twnr{剎帝利}{116.0}在這人們中是最上的:凡歸屬種姓者們,

   \twnr{明行具足者}{7.0},在天人中他是最上的。[\ccchref{MN.53}{https://agama.buddhason.org/MN/dm.php?keyword=53}, \ccchref{DN.3}{https://agama.buddhason.org/DN/dm.php?keyword=3}, \ccchref{DN.27}{https://agama.buddhason.org/DN/dm.php?keyword=27}, \ccchref{AN.11.10}{https://agama.buddhason.org/AN/an.php?keyword=11.10}]」

  梵王常童子說這個,\twnr{大師}{145.0}是認可者。那時,梵王常童子[想]:「大師是我的認可者。」向世尊問訊、\twnr{作右繞}{47.0}後,就在那裡消失。



\sutta{12}{12}{提婆達多經}{https://agama.buddhason.org/SN/sn.php?keyword=6.12}
  \twnr{被我這麼聽聞}{1.0}:

  \twnr{有一次}{2.0},在提婆達多離開不久,世尊住在王舍城\twnr{耆闍崛山}{258.0}。

  那時,在夜已深時,容色絕佳的\twnr{梵王娑婆主}{215.0}使整個王舍城耆闍崛山發光後,去見世尊。抵達後,向世尊\twnr{問訊}{46.0}後,在一旁站立。在一旁站立的梵王娑婆主在世尊的面前說這關於提婆達多的\twnr{偈頌}{281.0}:

  「果實確實殺害芭蕉,果實對竹子、果實對蘆葦,

   恭敬殺害邪惡人,如胎對騾。」[\suttaref{SN.17.35}, \ccchref{AN.4.68}{https://agama.buddhason.org/AN/an.php?keyword=4.68}]



\sutta{13}{13}{案達葛衛達經}{https://agama.buddhason.org/SN/sn.php?keyword=6.13}
  \twnr{有一次}{2.0},\twnr{世尊}{12.0}住在摩揭陀國的案達葛衛達。

  當時,世尊在漆黑的夜晚坐在露天處,\twnr{而天空下著毛毛雨}{385.0}。

  那時,容色絕佳的\twnr{梵王娑婆主}{215.0}使整個案達葛衛達發光後,去見世尊。抵達後,向世尊\twnr{問訊}{46.0}後,在一旁站立。在一旁站立的梵王娑婆主在世尊的面前說這\twnr{偈頌}{281.0}:

  「應該親近\twnr{邊地臥坐處}{30.0},應該為了結的解脫而行,

   如果在那裡沒得到喜樂,應該自我守護、有念地住在僧團中。

   \twnr{為了托鉢}{87.0}從家至家行走著,根已守護地、明智地、有念地,

   應該親近邊地臥坐處,已解脫恐怖、在無恐怖中解脫。

   在有可怕的蛇之處,閃電往來、天空打雷,

   在黑暗的黑夜裡,已離\twnr{身毛豎立}{152.0}的\twnr{比丘}{31.0}坐在那裡。

   這確實被我看見,這不是傳聞,

   在單一[生]梵行中,有一千位\twnr{死亡的捨棄者}{x147}。

   超過五百位\twnr{有學}{193.0},與十的十倍的十倍,

   全部是進入流者,非去畜生者。

   而這其他人,是我心(意)中的『福分者』,

   我甚至不能夠計算:\twnr{對妄語的愧}{x148}。[\ccchref{DN.18}{https://agama.buddhason.org/DN/dm.php?keyword=18}]」



\sutta{14}{14}{阿盧那哇低經}{https://agama.buddhason.org/SN/sn.php?keyword=6.14}
  \twnr{被我這麼聽聞}{1.0}:

  \twnr{有一次}{2.0},\twnr{世尊}{12.0}住在舍衛城……(中略)。

  在那裡,世尊召喚\twnr{比丘}{31.0}們:「比丘們!」

  「\twnr{尊師}{480.0}!」那些比丘回答世尊。

  世尊說這個:

  「比丘們!從前,有位國王名叫阿盧那哇(明相),比丘們!阿盧那哇國王的王都名叫阿盧那哇低,比丘們!尸棄\twnr{世尊}{12.0}、\twnr{阿羅漢}{5.0}、\twnr{遍正覺者}{383.0}依止阿盧那哇低居住,比丘們!又,尸棄世尊、阿羅漢、遍正覺者的第一雙弟子是名叫阿毘浮、三巴哇的雙賢。比丘們!那時,尸棄世尊、阿羅漢、遍正覺者召喚阿毘浮比丘:『婆羅門!我們走,我們將去某個梵天世界,直到用餐時間為止。』『是的,\twnr{大德}{45.0}!』比丘們!阿毘浮比丘回答尸棄世尊、阿羅漢、遍正覺者。比丘們!那時,尸棄世尊、阿羅漢、遍正覺者與阿毘浮比丘就猶如有力氣的男子伸直彎曲的手臂,或彎曲伸直的手臂,就像這樣在阿盧那哇低王都消失,出現在那個梵天世界。

  比丘們!那時,尸棄世尊、阿羅漢、遍正覺者召喚阿毘浮比丘:『婆羅門!請你對梵天、梵天眾、梵眾天顯現法談。』『是的,大德!』比丘們!阿毘浮比丘回答尸棄世尊、阿羅漢、遍正覺者後,對梵天、梵天眾、梵眾天以法說開示、勸導、鼓勵、\twnr{使歡喜}{86.0}。比丘們!在那裡,梵天、梵天眾、梵眾天譏嫌、不滿、責難:『實在不可思議啊,\twnr{先生}{202.0}!實在\twnr{未曾有}{206.0}啊,先生!弟子為何竟然在大師是面對者時教導法!』

   比丘們!那時,尸棄世尊、阿羅漢、遍正覺者召喚阿毘浮比丘:『婆羅門!梵天、梵天眾、梵眾天譏嫌你:「實在不可思議啊,先生!實在未曾有啊,先生!弟子為何竟然在大師是面對者時教導法!」婆羅門!那樣的話,請你以更大程度對梵天、梵天眾、梵眾天\twnr{激起急迫感}{373.0}。』『是的,大德!』比丘們!阿毘浮比丘回答尸棄世尊、阿羅漢、遍正覺者後,看得見身體地教導法,也沒看見身體地教導法,也看得見下半身沒看見上半身地教導法,也身看得見上半沒看見下半身地教導法。比丘們!在那裡,梵天、梵天眾、梵眾天有不可思議的、未曾有的心生起:『實在不可思議啊,先生!實在未曾有啊,先生!\twnr{沙門}{29.0}的\twnr{大神通力}{405.0}狀態、大威力狀態!』

  那時,阿毘浮比丘對尸棄世尊、阿羅漢、遍正覺者說這個:『大德!我記得(證知)在比丘僧團中說過像這樣的話:「\twnr{學友}{201.0}們!當站在梵天世界中時,我能夠以聲音教導千世間界。」』『婆羅門!是為了這個的適當時機,婆羅門!是為了這個的適當時機:婆羅門!凡在站在梵天世界中時,如果你以聲音教導千世間界。』『是的,大德!』比丘們!阿毘浮比丘回答尸棄世尊、阿羅漢、遍正覺者後,當站在梵天世界中時,說這些\twnr{偈頌}{281.0}:

  『請你們發勤、請你們精勤,請你們致力於佛陀的教說,

   請你們掃蕩死魔軍,如象對蘆葦茅屋。

   凡在這法、律中,住於不放逸者,

   捨斷出生的輪迴後,將\twnr{作苦的終結}{54.0}。』

  比丘們!那時,尸棄世尊、阿羅漢、遍正覺者與阿毘浮比丘對梵天、梵天眾、梵眾天激起急迫感後,猶如……(中略)在那個梵天世界消失,出現在阿盧那哇低王都。那時,尸棄世尊、阿羅漢、遍正覺者召喚比丘們:『比丘們!你們聽到當阿毘浮比丘站在梵天世界中說的偈頌了嗎?』『大德!我們聽到當阿毘浮比丘站在梵天世界中說的偈頌了。』『比丘們!那麼,如怎樣你們聽到當阿毘浮比丘站在梵天世界中說的偈頌?』『大德!我們這樣聽到當阿毘浮比丘站在梵天世界中說的偈頌:

  「請你們發勤、請你們精勤,請你們致力於佛陀的教說,

   請你們掃蕩死魔軍,如象對蘆葦茅屋。

   凡在這法、律中,住於不放逸者,

   捨斷出生的輪迴後,將作苦的終結。」

  大德!我們這樣聽到當阿毘浮比丘站在梵天世界中說的偈頌。』

  「\twnr{好}{44.0}!好!比丘們!好!比丘們!你們聽到當阿毘浮比丘站在梵天世界中說的偈頌。」

  世尊說這個,那些悅意的比丘歡喜世尊的所說。



\sutta{15}{15}{般涅槃經}{https://agama.buddhason.org/SN/sn.php?keyword=6.15}
  \twnr{有一次}{2.0},\twnr{世尊}{12.0}住在拘尸那羅的烏玻瓦達那,末羅的沙羅樹林雙沙羅樹中間,在\twnr{般涅槃}{72.0}時。

  那時,世尊召喚\twnr{比丘}{31.0}們:

  「來吧,比丘們!我現在召喚你們:『諸行是\twnr{消散法}{155.0},你們應該以不放逸使[目標]完成。』這是如來最後的話。」

  那時,世尊進入初禪,從初禪出來後,進入第二禪,從第二禪出來後,進入第三禪,從第三禪出來後,進入第四禪,從第四禪出來後,進入空無邊處,從空無邊處出來後,進入識無邊處,從識無邊處出來後,進入\twnr{無所有處}{533.0},從無所有處出來後,進入\twnr{非想非非想處}{534.0},從非想非非想處出來後,進入\twnr{想受滅}{416.0}。

  從想受滅出來後,進入非想非非想處,從非想非非想處出來後,進入無所有處,從無所有處出來後,進入識無邊處,從識無邊處出來後,進入空無邊處,從空無邊處出來後,進入第四禪,從第四禪出來後,進入第三禪,從第三禪出來後,進入第二禪,從第二禪出來後,進入初禪,從初禪出來後,進入第二禪,從第二禪出來後,進入第三禪,從第三禪出來後,進入第四禪,從第四禪出來後,世尊直接般涅槃。

  在世尊般涅槃時,與般涅槃同時,\twnr{梵王娑婆主}{215.0}說了這\twnr{偈頌}{281.0}:

  「全都將捨棄:世間中生類的身體,

   於該處像這樣的大師,世間中無與倫比者,

   得到力量如來,\twnr{正覺者}{185.1}已\twnr{般涅槃}{72.0}。」

  在世尊般涅槃時,與般涅槃同時,\twnr{天帝釋}{263.0}說了這偈頌:

  「諸行確實是無常的,是\twnr{生起與消散法的}{681.0},

   生起後被滅,它們的寂滅是樂。」[\suttaref{SN.1.11}]

  在世尊般涅槃時,與般涅槃同時,\twnr{尊者}{200.0}阿難說了這偈頌:

  「那時是令人恐懼的,那時是令人\twnr{身毛豎立}{152.0}的:

   在具有一切殊勝行相的,正覺者般涅槃時。」

  在世尊般涅槃時,與般涅槃同時,尊者阿那律說了這偈頌:

  「沒有了入息出息,心已住立的\twnr{像這樣者}{632.0},

   不動者著手寂靜後,\twnr{有眼者}{629.0}已般涅槃。

   以不動搖的心,忍受了痛苦(受),

   就如燈火的熄滅,它是心的解脫。」[\ccchref{DN.16}{https://agama.buddhason.org/DN/dm.php?keyword=16}, 219-222段]

  第二品,其\twnr{攝頌}{35.0}:

  「梵天[常童子]、提婆達多,案達葛衛達、阿盧那哇低,

   以及般涅槃,這是梵天五則所教導的。」

  梵天相應完成。





\page

\xiangying{7}{婆羅門相應}
\pin{阿羅漢品}{1}{10}
\sutta{1}{1}{大那若尼經}{https://agama.buddhason.org/SN/sn.php?keyword=7.1}
  \twnr{被我這麼聽聞}{1.0}:

  \twnr{有一次}{2.0},\twnr{世尊}{12.0}住在王舍城栗鼠飼養處的竹林中。

  當時,某位婆羅墮若姓婆羅門的妻子,名叫大那若尼婆羅門女[\ccchref{MN.100}{https://agama.buddhason.org/MN/dm.php?keyword=100}],對佛、法、僧是\twnr{極淨信者}{340.1}。

  那時,大那若尼婆羅門女為婆羅墮若姓婆羅門帶來食物,絆倒後吟出\twnr{優陀那}{184.0}三次:

  「對那位世尊、\twnr{阿羅漢}{5.0}、\twnr{遍正覺者}{6.0}禮敬,對那位世尊、阿羅漢、遍正覺者禮敬,對那位世尊、阿羅漢、遍正覺者禮敬。」

  在這麼說時,婆羅墮若姓婆羅門對大那若尼婆羅門女說這個:

  「就像這樣,而這位女賤民在每一個場合都稱讚那位禿頭\twnr{沙門}{29.0},女賤民!現在,我將論破你的那位老師(使登上你那位老師的論說)。」

  「婆羅門!我不見那個:在包括天,在包括魔,在包括梵的世間;在包括沙門婆羅門,在包括天-人的\twnr{世代}{38.0}中能論破那位世尊、阿羅漢、遍正覺者的,婆羅門!但你去吧!去了後你必將了知。」

  那時,婆羅墮若姓婆羅門生氣、不悅意地去見世尊。抵達後,與世尊一起互相問候。交換應該被互相問候的友好交談後,在一旁坐下。在一旁坐下的婆羅墮若姓婆羅門以\twnr{偈頌}{281.0}對世尊說:

  「切斷什麼後\twnr{睡得安樂}{532.0}?切斷什麼後不憂愁?

   對哪一法的殺害,\twnr{喬達摩}{80.0}同意?」

  「切斷憤怒後睡得安樂,切斷憤怒後不憂愁,

   婆羅門!對端蜜,\twnr{而根毒之憤怒}{929.0}的殺害,

   聖者稱讚,因為切斷它後不憂愁。」[\suttaref{SN.1.71}]

  在這麼說時,婆羅墮若姓婆羅門對世尊說這個:

  「太偉大了,喬達摩尊師!太偉大了,喬達摩尊師!喬達摩尊師!猶如扶正顛倒的,或揭開隱藏的,或告知迷路者的道路,或在黑暗中持燈火:『有眼者們看見諸色。』同樣的,法被喬達摩尊師以種種\twnr{法門}{562.0}說明。\twnr{大德}{45.0}!這個我\twnr{歸依}{284.0}喬達摩\twnr{尊師}{203.0}、法、\twnr{比丘僧團}{65.0},願我得到在喬達摩\twnr{尊師}{203.0}的面前出家,願我\twnr{得到具足戒}{124.1}。」

  婆羅墮若姓婆羅門得到在世尊的面前出家,得到具足戒。

  還有,已受具足戒不久,住於單獨的、隱離的、不放逸的、熱心的、自我努力的\twnr{尊者}{200.0}婆羅墮若不久就以證智自作證後,在當生中\twnr{進入後住於}{66.0}凡\twnr{善男子}{41.0}們為了利益正確地\twnr{從在家出家成為無家者}{48.0}的那個無上梵行結尾,他證知:「\twnr{出生已盡}{18.0},\twnr{梵行已完成}{19.0},\twnr{應該被作的已作}{20.0},\twnr{不再有此處[輪迴]的狀態}{21.1}。」然後尊者婆羅墮若成為眾阿羅漢之一。



\sutta{2}{2}{辱罵經}{https://agama.buddhason.org/SN/sn.php?keyword=7.2}
  \twnr{有一次}{2.0},\twnr{世尊}{12.0}住在王舍城栗鼠飼養處的竹林中。

  \twnr{辱罵婆羅墮若婆羅門}{x149}聽聞:

  「聽說婆羅墮若姓婆羅門在沙門喬答摩面前\twnr{從在家出家成為無家者}{48.0}。」生氣、不悅意地去見世尊。抵達後,以無禮的、粗惡的言語責罵、誹謗世尊。

  在這麼說時,世尊對辱罵婆羅墮若婆羅門說這個:

  「\twnr{婆羅門}{17.0}!你怎麼想它:你的朋友、同事、親族、血親、客人們是否到來呢?」

  「\twnr{喬達摩}{80.0}尊師!有時我的朋友、同事、親族、血親、客人到來。」

  「婆羅門!你怎麼想它:是否你提供他們\twnr{硬食或軟食}{153.0}或美食呢?」

  「喬達摩尊師!有時我提供他們硬食或軟食或美食。」

  「婆羅門!但,如果他們不領受,那個是誰的呢?」

  「喬達摩尊師!如果他們不領受,那個還是我們的。」

  「同樣的,婆羅門!凡你辱罵、激怒、爭論不辱罵、不激怒、不爭論的我們,我們不要領受你的那個,婆羅門!那個還是你的;婆羅門!那個還是你的。

  婆羅門!凡回罵辱罵者,回激激怒者,回爭爭論者,婆羅門!這被稱為『一起吃;交換。』那些我們既不與你一起吃,也不與你交換,婆羅門!那個還是你的;婆羅門!那個還是你的。」

  「包括國王的群眾這麼知道喬達摩\twnr{尊師}{203.0}:『\twnr{沙門}{29.0}喬達摩是\twnr{阿羅漢}{5.0}。』然而喬達摩尊師生氣。」

  「從哪裡有憤怒:對已調御、正確生活的無憤怒者,

   對以完全智解脫者,對\twnr{像這樣的}{632.0}寂靜者?

   因為那樣對他只是更糟的:凡對生氣者生氣回去,

   不對生氣者生氣回去者,他打勝難勝利的戰鬥。

   行兩者的利益:自己的與對方的,

   知道對方已被激怒後,凡具念地平靜下來者。

   當治療兩者時:自己的與對方的,

   人們思量『他是愚者』:凡法的不熟知者。[\suttaref{SN.7.3}, \suttaref{SN.10.4}]」

  在這麼說時,辱罵婆羅墮若婆羅門對世尊說這個:

  「太偉大了,喬達摩尊師!……(中略)這個我\twnr{歸依}{284.0}喬達摩尊師、法、\twnr{比丘僧團}{65.0},願我得到在喬達摩尊師的面前出家,願我\twnr{得到具足戒}{124.1}。」

  那時,辱罵婆羅墮若婆羅門得到在世尊的面前出家、受具足戒。

  還有,已受具足戒不久,住於單獨的、隱離的、不放逸的、熱心的、自我努力的\twnr{尊者}{200.0}辱罵婆羅墮若不久就以證智自作證後,在當生中\twnr{進入後住於}{66.0}凡\twnr{善男子}{41.0}們為了利益正確從在家出家成為無家者的那個無上梵行結尾,他證知:「\twnr{出生已盡}{18.0},\twnr{梵行已完成}{19.0},\twnr{應該被作的已作}{20.0},\twnr{不再有此處[輪迴]的狀態}{21.1}。」然後尊者辱罵婆羅墮若成為眾阿羅漢之一。



\sutta{3}{3}{阿修羅王經}{https://agama.buddhason.org/SN/sn.php?keyword=7.3}
  \twnr{有一次}{2.0},\twnr{世尊}{12.0}住在王舍城栗鼠飼養處的竹林中。

  \twnr{阿修羅王婆羅墮若婆羅門}{x150}聽聞:

  「聽說婆羅墮若姓婆羅門在沙門喬答摩面前\twnr{從在家出家成為無家者}{48.0}。」生氣、不悅意地去見世尊。抵達後,以無禮的、粗惡的言語責罵、誹謗世尊。

  在這麼說時,世尊保持沈默。

  那時,阿修羅王婆羅墮若婆羅門對世尊說這個:

  「\twnr{沙門}{29.0}!你被征服;沙門!你被征服。」

  「愚者確實認為勝利者:以粗惡的言語說話者,

   但他才會是勝利者:凡了知忍耐者。

   因為那樣對他只是更糟的:凡對生氣者生氣回去,

   不對生氣者生氣回去者,他打勝難勝利的戰鬥。

   行兩者的利益:自己的與對方的,

   知道對方已被激怒後,凡具念地平靜下來者。

   當治療兩者時:自己的與對方的,

   人們思量『他是愚者』:凡法的不熟知者。[\suttaref{SN.7.2}]」

  在這麼說時,阿修羅王婆羅墮若婆羅門對世尊說這個:

  「太偉大了,\twnr{喬達摩}{80.0}尊師!……(中略)他證知:……然後\twnr{尊者}{200.0}婆羅墮若成為眾\twnr{阿羅漢}{5.0}之一。」



\sutta{4}{4}{酸粥經}{https://agama.buddhason.org/SN/sn.php?keyword=7.4}
  \twnr{有一次}{2.0},\twnr{世尊}{12.0}住在王舍城栗鼠飼養處的竹林中。

  \twnr{酸粥婆羅墮若婆羅門}{x151}聽聞:

  「聽說婆羅墮若姓婆羅門在沙門喬答摩面前\twnr{從在家出家成為無家者}{48.0}。」生氣、不悅意地去見世尊。抵達後,沈默地在一旁站立。

  那時,世尊以心了知酸粥婆羅墮若婆羅門心中的深思後,以\twnr{偈頌}{281.0}對酸粥婆羅墮若婆羅門說:

  「凡冒犯無犯錯的人,純淨無穢的人, 

   惡就回到那位愚者,如細塵被逆風地拋出。」[\suttaref{SN.1.22}]

  在這麼說時,酸粥婆羅墮若婆羅門對世尊說這個:

  「太偉大了,\twnr{喬達摩}{80.0}尊師!……(中略)他證知:……然後\twnr{尊者}{200.0}婆羅墮若成為眾\twnr{阿羅漢}{5.0}之一。」



\sutta{5}{5}{無害經}{https://agama.buddhason.org/SN/sn.php?keyword=7.5}
  起源於舍衛城。

  那時,\twnr{無害婆羅墮若婆羅門}{x152}去見\twnr{世尊}{12.0}。抵達後,與世尊一起互相問候。交換應該被互相問候的友好交談後,在一旁坐下。在一旁坐下的無害婆羅墮若婆羅門對世尊說這個:

  「\twnr{喬達摩}{80.0}尊師!我是無害;喬達摩尊師!我是無害。」

  「而如果你如名字那樣,你會是無害,

   凡不以身與語,以及意傷害者,

   他確實是無害:凡不傷害他人者。」

  在這麼說時,無害婆羅墮若婆羅門對世尊說這個:

  「太偉大了,喬達摩尊師!……(中略)他證知:……然後\twnr{尊者}{200.0}婆羅墮若成為眾\twnr{阿羅漢}{5.0}之一。」



\sutta{6}{6}{結縛經}{https://agama.buddhason.org/SN/sn.php?keyword=7.6}
  起源於舍衛城:

  那時,結縛(結髮)婆羅墮若婆羅門去見\twnr{世尊}{12.0}。抵達後,與世尊一起互相問候。交換應該被互相問候的友好交談後,在一旁坐下。在一旁坐下的結縛婆羅墮若婆羅門以\twnr{偈頌}{281.0}對世尊說:

  「內結縛外結縛,\twnr{世代}{38.0}被結縛糾纏,

   \twnr{喬達摩}{80.0}!我問你這個:誰能解開這個結縛?」

  「有慧的人在戒上確立後,心與慧的修習者,

   熱心明智的\twnr{比丘}{31.0},他能解開這個結縛。

   凡他們的貪與瞋,以及\twnr{無明}{207.0}已脫離者,

   諸漏已滅盡的\twnr{阿羅漢}{5.0},他們的結縛已被解開。

   於名與色處,被無餘地破壞,

   有對與色想,那個結縛在這裡能被切斷。」[\suttaref{SN.1.23}]

  在這麼說時,結縛婆羅墮若婆羅門對世尊說這個:

  「太偉大了,喬達摩尊師!……(中略)。」

  然後\twnr{尊者}{200.0}婆羅墮若成為眾阿羅漢之一。



\sutta{7}{7}{概要經}{https://agama.buddhason.org/SN/sn.php?keyword=7.7}
  起源於舍衛城。

  那時,純淨婆羅墮若婆羅門去見\twnr{世尊}{12.0}。抵達後,與世尊一起互相問候。交換應該被互相問候的友好交談後,在一旁坐下。在一旁坐下的純淨婆羅墮若婆羅門在世尊的面前說這\twnr{偈頌}{281.0}:

  「世間任何婆羅門都不變純淨,即使是持戒者與執行(作)苦行者,

   \twnr{明行具足者}{7.0},他變純淨-非別的其他人。」

  「即使許多無用咒的呢喃者,婆羅門也不是以出生:

   內部有污濁的、污染的,依靠詭計者。

   \twnr{剎帝利}{116.0}、\twnr{婆羅門}{17.0}、\twnr{毘舍}{476.0},\twnr{首陀羅}{472.0},\twnr{旃陀羅}{120.0}、清垃圾者,

   活力已發動者、自我努力者,常堅固努力者,

   他得到最高的純淨,婆羅門!請你這麼知道。」

  在這麼說時,純淨婆羅墮若婆羅門對世尊說這個:

  「太偉大了,\twnr{喬達摩}{80.0}尊師!……(中略)。」

  然後\twnr{尊者}{200.0}婆羅墮若成為眾\twnr{阿羅漢}{5.0}之一。



\sutta{8}{8}{拜火經}{https://agama.buddhason.org/SN/sn.php?keyword=7.8}
  \twnr{有一次}{2.0},\twnr{世尊}{12.0}住在王舍城栗鼠飼養處的竹林中。

  當時,拜火婆羅墮若婆羅門的參酥油乳粥被放置:「我將獻供火,我將祀拜火。」

  那時,世尊午前時穿衣、拿起衣鉢後,\twnr{為了托鉢}{87.0}進入王舍城。

  當在王舍城\twnr{為了托鉢次第地行走著}{127.0}時,來到拜火婆羅墮若婆羅門的住處。抵達後,在一旁站立。

  拜火婆羅墮若婆羅門看見為了托鉢站立的世尊。看見後,拜火婆羅墮若婆羅門以\twnr{偈頌}{281.0}對世尊說:

  「\twnr{三明}{133.0}具足者、出生良好者、多聞者,

   \twnr{明行具足者}{7.0},他能享用這乳粥。」

  「即使許多無用咒的呢喃者,婆羅門也不是以出生:

   內部被污濁污染,以詭計成為被圍繞者。

   凡知道前世住處,以及看見天界與\twnr{苦界}{109.0}者,

   又已到達生的滅盡者,\twnr{已完成證智}{x153}的\twnr{牟尼}{125.0}。

   以這三種明,是三明婆羅門,

   明行具足者,他能享用這乳粥。」

  「請\twnr{喬達摩}{80.0}\twnr{尊師}{203.0}享用,尊師是婆羅門。」

  「以吟誦偈頌獲得的不應該被我享用,婆羅門!這不是看見者之法,

   諸佛拒絕以吟誦偈頌獲得的,婆羅門!在法存在時這是行為模式。

   而圓滿的大仙,諸漏已滅盡者、惡作已平息者,

   請你以其它食物飲料服侍,因為那是期待福德者的田地。」

  在這麼說時,拜火婆羅墮若婆羅門對世尊說這個:

  「太偉大了,喬達摩尊師!……(中略)。」

  然後\twnr{尊者}{200.0}拜火婆羅墮若成為眾\twnr{阿羅漢}{5.0}之一。



\sutta{9}{9}{孫陀利葛經}{https://agama.buddhason.org/SN/sn.php?keyword=7.9}
  \twnr{有一次}{2.0},\twnr{世尊}{12.0}住在憍薩羅國孫陀利葛河邊。

  當時,孫陀利葛婆羅墮若婆羅門在孫陀利葛河邊獻供火、祀拜火。那時,孫陀利葛婆羅墮若婆羅門獻供火、祀拜火後,從座位起來後觀察四方一切:「誰能享用這\twnr{殘餘供物}{x154}呢?」

  孫陀利葛婆羅墮若婆羅門看見裹著頭坐在某棵樹下的世尊。看見後,以左手拿殘餘供物、以右手拿長口水瓶後,去見世尊。

  那時,世尊以孫陀利葛婆羅墮若婆羅門的腳步聲而敞開頭。

  那時,孫陀利葛婆羅墮若婆羅門[心想]:

  「這位\twnr{尊者}{200.0}是剃頭的;這位尊者是剃頭者。」就從那裡又想要折返。

  那時,孫陀利葛婆羅墮若婆羅門想這個:

  「這裡,某些婆羅門\twnr{尊師}{203.0}也確實是剃頭的,讓我去見他後,問出生。」

  那時,孫陀利葛婆羅墮若婆羅門去見世尊。抵達後,對世尊說這個:「尊師是什麼出生的?」

  「你不要問出生但要問行為,火確實從柴被出生,

   卑下家系的堅定牟尼,有\twnr{慚}{250.0}抑止者成為高貴的。

   被真理調御、具備調御者,\twnr{通曉吠陀者}{x155}、已完成梵行者,

   帶來獻供者應該招請他,在適當時間他對應該被供養者獻供。」

  「這裡確實有我的善獻供善供奉的:凡我看見像那樣的通曉吠陀者,

   以沒看見像你們那樣的,其他人享用殘餘供物。

   請\twnr{喬達摩}{80.0}尊師享用,尊師是婆羅門。」

  「以吟誦偈頌獲得的不應該被我享用,婆羅門!這不是看見者之法,

   諸佛拒絕以吟誦偈頌獲得的,婆羅門!在法存在時這是行為模式。

   而圓滿的大仙,諸漏已滅盡者、惡作已平息者,

   請你以其它食物飲料服侍,因為那是期待福德者的田地。」

  「喬達摩尊師!那麼,我應將這殘餘供物施與誰?」

  「婆羅門!我不見那個:在包括天,在包括魔,在包括梵的世間;在包括沙門婆羅門,在包括天-人的\twnr{世代}{38.0}中,凡這個殘餘供物的食者能走到\twnr{完全消化}{x156}的,婆羅門!除了\twnr{如來}{4.0}或如來的弟子外。婆羅門!那樣的話,請你捨棄那個殘餘供物在少草處,或使沈入在無蟲的水中。」

  那時,孫陀利葛婆羅墮若婆羅門使那個殘餘供物沈入無蟲的水中。

  那時,那個丟入水中的殘餘供物嘶嘶響、嗤嗤響、冒煙、冒出大煙。猶如白天被曬熱丟入水中的鋤頭嘶嘶響、嗤嗤響、冒煙、冒出大煙。同樣的,那個丟入水中的殘餘供物嘶嘶響、嗤嗤響、冒煙、冒出大煙。

  那時,孫陀利葛婆羅墮若婆羅門驚怖、生起\twnr{身毛豎立}{152.0}地去見世尊。抵達後,在一旁站立。

  世尊以偈頌對在一旁站立的孫陀利葛婆羅墮若婆羅門說:

  「婆羅門!當柴燃燒時,你不要認為這個在外的確實是純淨,

   因為善者們不以那個說純淨:凡如果以外部希求遍純淨者。

   婆羅門!捨斷柴的燃燒後,我只使自身內的火燃燒,

   \twnr{經常火的}{x157}、經常得定的狀態,婆羅門!我是行梵行的\twnr{阿羅漢}{5.0}。

   婆羅門!慢確實是你的荷擔,憤怒是煙、妄語是灰,

   舌頭是獻祭的杓子、心是祭祀的火爐,\twnr{善調御的自我是男子的火}{x158}。

   婆羅門!法是\twnr{具有戒為渡場}{x159}的湖,不濁的、被善人們對善人們稱讚,

   於該處沐浴的通曉吠陀者們,就肢體不濕地渡過彼岸。

   真理、法、[自我]抑制、梵行,婆羅門!在中間依止者有梵的到達,

   請你對那位已成為正直者作禮敬,我說那人是『\twnr{法之行者}{x160}』。」

  在這麼說時,孫陀利葛婆羅墮若婆羅門對世尊說這個:

  「太偉大了,喬達摩尊師!……(中略)。」

  然後尊者婆羅墮若成為眾阿羅漢之一。



\sutta{10}{10}{許多女兒經}{https://agama.buddhason.org/SN/sn.php?keyword=7.10}
  \twnr{有一次}{2.0},\twnr{世尊}{12.0}住在憍薩羅國某處叢林中。

  當時,某位婆羅墮若姓婆羅門的十四頭公牛被遺失了。

  那時,當婆羅墮若姓婆羅門尋找那些公牛時,來到那處叢林。抵達後,看見在那處叢林中坐下,盤腿、定置端直的身體、\twnr{建立面前的念後}{529.0}的世尊。看見後,去見世尊。抵達後,在世尊的面前以這些\twnr{偈頌}{281.0}說:

  「確實沒有這位\twnr{沙門}{29.0}的,十四頭公牛,

   今日第六天沒被看見,因為那樣這位沙門是安樂者。

   確實沒有這位沙門的,壞的[芝麻株]在芝麻田中,

   有一葉以及二葉,因為那樣這位沙門是安樂者。

   確實沒有這位沙門的,老鼠在空穀倉中,

   熱烈地跳舞,因為那樣這位沙門是安樂者。

   確實沒有這位沙門的,敷具七個月,

   被跳蚤覆蓋,因為那樣這位沙門是安樂者。

   確實沒有這位沙門的,七個寡婦女兒,

   有一個兒子以及二個兒子,因為那樣這位沙門是安樂者。

   確實沒有這位沙門的,被黃褐色斑點侵襲女(黃臉婆),

   以腳使睡覺者醒來,因為那樣這位沙門是安樂者。

   確實沒有這位沙門的,債權人在黎明時,

   斥責『你要給[還債]!你要給!』,因為那樣這位沙門是安樂者。」

  「婆羅門!確實沒有我的,十四頭公牛,

   今日第六天沒被看見,婆羅門!因為那樣我是安樂者。

   婆羅門!確實沒有我的,壞的在芝麻田中,

   有一葉以及二葉,婆羅門!因為那樣我是安樂者。

   婆羅門!確實沒有我的,老鼠在空穀倉中,

   熱烈地跳舞,婆羅門!因為那樣我是安樂者。

   婆羅門!確實沒有我的,敷具七個月,

   被跳蚤覆蓋,婆羅門!因為那樣我是安樂者。

   婆羅門!確實沒有我的,七個寡婦女兒,

   有一個兒子以及二個兒子,婆羅門!因為那樣我是安樂者。

   婆羅門!確實沒有我的,被黃褐色斑點侵襲女,

   以腳使睡覺者醒來,婆羅門!因為那樣我是安樂者。

   婆羅門!確實沒有我的,債權人在黎明時,

   斥責『你要給!你要給!』,婆羅門!因為那樣我是安樂者。」

  在這麼說時,婆羅墮若姓婆羅門對世尊說這個:

  「太偉大了,\twnr{喬達摩}{80.0}尊師!太偉大了,喬達摩尊師!喬達摩尊師!猶如扶正顛倒的,或揭開隱藏的,或告知迷路者的道路,或在黑暗中持燈火:『有眼者們看見諸色。』同樣的,法被喬達摩尊師以種種\twnr{法門}{562.0}說明。這個我\twnr{歸依}{284.0}喬達摩\twnr{尊師}{203.0}、法、\twnr{比丘僧團}{65.0},願我得到在喬達摩尊師的面前出家,願我得到具足戒。」

  婆羅墮若姓婆羅門得到在世尊的面前出家,得到具足戒。

  還有,已受具足戒不久,住於單獨的、隱離的、不放逸的、熱心的、自我努力的\twnr{尊者}{200.0}婆羅墮若不久就以證智自作證後,在當生中\twnr{進入後住於}{66.0}凡\twnr{善男子}{41.0}們為了利益正確地\twnr{從在家出家成為無家者}{48.0}的那個無上梵行結尾,他證知:「\twnr{出生已盡}{18.0},\twnr{梵行已完成}{19.0},\twnr{應該被作的已作}{20.0},\twnr{不再有此處[輪迴]的狀態}{21.1}。」然後尊者婆羅墮若成為眾\twnr{阿羅漢}{5.0}之一。

  阿羅漢品第一,其\twnr{攝頌}{35.0}:

  「大那若尼與辱罵,阿修羅王、酸粥,

   無害者與結縛,概要與拜火,

   孫陀利葛與許多女兒它們為十。」





\pin{優婆塞品}{11}{22}
\sutta{11}{11}{耕田婆羅墮若經}{https://agama.buddhason.org/SN/sn.php?keyword=7.11}
  \twnr{被我這麼聽聞}{1.0}:

  \twnr{有一次}{2.0},\twnr{世尊}{12.0}住在摩揭陀國\twnr{南山一蘆葦}{x161}[地方]的婆羅門村落。

  當時是在播種時節,有\twnr{耕田婆羅墮若}{x162}婆羅門的五百具之多(量)上軛的犁。

  那時,世尊午前時穿衣、拿起衣鉢後,去耕田婆羅墮若婆羅門工作處。

  當時正進行耕田婆羅墮若婆羅門的食物分配。那時,世尊去食物分配處。抵達後,在一旁站立。

  耕田婆羅墮若婆羅門看見為食物站立的世尊。看見後,對世尊說這個:

  「\twnr{沙門}{29.0}!我耕田與播種,耕田與播種後我享用。沙門!也請你自己耕田與播種,耕田與播種後你自己享用。」

  「婆羅門!我也耕田與播種,耕田與播種後我享用。」

  「我們確實沒看到\twnr{喬達摩}{80.0}\twnr{尊師}{203.0}的軛,或犁,或鋤,或刺棒,或牛,然而喬達摩尊師卻說這個:『婆羅門!我也耕田與播種,耕田與播種後我享用。』」

  那時,耕田婆羅墮若婆羅門以\twnr{偈頌}{281.0}對世尊說:

  「你自稱耕作者,但我不見你的耕田,

   被質問的耕作者請你說,我們如何能知道你耕田。」

  「信為種子\twnr{苦行}{x163}為雨,慧為我的軛和犁,

   \twnr{慚}{250.0}為轅桿意為繫繩,念為我的鋤和刺棒。

   守護身守護語,\twnr{胃裡}{x164}的食物已被自制,

   我執行(作)真理為割草,\twnr{柔和是我的脫離}{x165}。

   活力是我的負載牛:\twnr{軛安穩}{192.0}的帶來者,

   牠不折返(退轉)地走,走到後於該處牠不悲傷。

   這位耕作者像這樣耕田,他有\twnr{不死}{123.0}之果,

   耕作這個耕田後,從一切苦被釋放。」

  「請喬達摩尊師享用,尊師是耕作者,那是因為喬達摩尊師甚至耕作有不死之果的耕田。」

  「以吟誦偈頌獲得的不應該被我享用,婆羅門!這不是看見者之法,

   諸佛拒絕以吟誦偈頌獲得的,婆羅門!在法存在時這是行為模式。

   而圓滿的大仙,諸漏已滅盡者、惡作已平息者,

   請你以其它食物飲料服侍,因為那是期待福德者的田地。」

  在這麼說時,耕田婆羅墮若婆羅門對世尊說這個:

  「太偉大了,喬達摩尊師!……(中略)從今天起\twnr{已終生歸依}{64.0}。」



\sutta{12}{12}{優達亞經}{https://agama.buddhason.org/SN/sn.php?keyword=7.12}
  起源於舍衛城。

  那時,\twnr{世尊}{12.0}午前時穿衣、拿起衣鉢後,去優達亞婆羅門的住處。

  那時,優達亞婆羅門以米飯使世尊的鉢充滿。

  第二次,世尊午前時穿衣、拿起衣鉢後,又去優達亞婆羅門的住處。……(中略)第三次,優達亞婆羅門又以米飯使世尊的鉢充滿後,對世尊說這個:

  「這位討厭的\twnr{沙門}{29.0}\twnr{喬達摩}{80.0}一再地到來。」

  「但他們就一再地播種,天王一再地下雨,

   耕作者們一再地耕田,穀物一再地來到國家。

   乞求者們一再地乞求,施主們一再地施與,

   施主們一再地施與後,一再地到達天界處。

   他們一再地對產奶牛擠奶,仔牛一再地來到母親,

   他一再地疲倦、顫抖,愚鈍者一再地來到母胎。

   他一再地被生、死亡,他們一再地帶去墓場,

   但得到不再有之道後,廣慧者不一再地被生。」

  在這麼說時,優達亞婆羅門對世尊說這個:

  「太偉大了,喬達摩尊師!……(中略)請喬達摩\twnr{尊師}{203.0}記得我為\twnr{優婆塞}{98.0},從今天起\twnr{已終生歸依}{64.0}。」



\sutta{13}{13}{提婆西多經}{https://agama.buddhason.org/SN/sn.php?keyword=7.13}
  起源於舍衛城。

  當時,\twnr{世尊}{12.0}是因風生病者,而\twnr{尊者}{200.0}優波哇那為世尊的侍者。

  那時,世尊召喚尊者優波哇那:

  「優波哇那!來吧,請你為我知道(找)熱水。」

  「是的,\twnr{大德}{45.0}!」尊者優波哇那回答世尊後,穿衣、拿起衣鉢後,去提婆西多婆羅門的住處。抵達後,沈默地在一旁站立。

  提婆西多婆羅門看見保持沈默在一旁站立的尊者優波哇那。看見後,以\twnr{偈頌}{281.0}對尊者優波哇那說:

  「保持沈默站立的\twnr{尊師}{203.0},剃光頭的穿\twnr{僧伽梨}{270.0}者,

   欲求著什麼尋求著什麼?來者要乞求什麼呢?」

  「世間中的\twnr{阿羅漢}{5.0}、\twnr{善逝}{8.0},牟尼是因風生病者,

   如果有熱水,婆羅門!請施與牟尼。」

  「他被應該被供養者供養,他被應該被恭敬者恭敬,

   他被應該被尊敬者尊敬,我願意為他帶去。」

  那時,提婆西多婆羅門以男子取熱水擔與糖蜜袋給尊者優波哇那。

  那時,尊者優波哇那去見世尊。抵達後,以熱水沐浴世尊後,使糖蜜與熱水混合後,給世尊。

  那時,世尊的病止息了。

  那時,提婆西多婆羅門去見世尊。抵達後,與世尊一起互相問候。交換應該被互相問候的友好交談後,在一旁坐下。在一旁坐下的提婆西多婆羅門以偈頌對世尊說:

  「應該施與施物於何處?所施於何處有大果?

   對供養者來說怎樣,\twnr{供養}{953.0}怎樣成功?」

  「凡知道前世住處者,以及他看見天界\twnr{苦界}{109.0},

   又已到達出生的滅盡者,完成證智的牟尼。

   應該施與施物於這裡,所施於這裡有大果,

   對供養者來說這樣,供養這樣成功。」

  在這麼說時,提婆西多婆羅門對世尊說這個:

  「太偉大了,\twnr{喬達摩}{80.0}尊師!……(中略)請喬達摩尊師記得我為\twnr{優婆塞}{98.0},從今天起\twnr{已終生歸依}{64.0}。」



\sutta{14}{14}{大財富者經}{https://agama.buddhason.org/SN/sn.php?keyword=7.14}
  起源於舍衛城。

  那時,某位粗弊的、穿粗弊外衣的大財富的婆羅門去見世尊。抵達後,與世尊一起互相問候。交換應該被互相問候的友好交談後,在一旁坐下。世尊對在一旁坐下的大財富的婆羅門說這個:

  「婆羅門!為何你是粗弊的、穿粗弊外衣的?」

  「\twnr{喬達摩}{80.0}尊師!這裡,我的四個兒子,他們與妻子\twnr{商量後}{x166},使我離家。」

  「婆羅門!那樣的話,記住這些\twnr{偈頌}{281.0}後,在大群人集會的集會所中,在兒子們共坐時請你說:

  『在他們出生時我歡喜,又我希望他們幸福,

   他們與妻子商量後,阻擋我如狗對豬。

   確實是卑鄙不善者:他們「爹!爹!」地叫我,

   \twnr{以兒子樣子}{x167}的\twnr{羅剎}{122.0},他們捨棄走到年老者。

   如衰老無用的馬,被除去食物,

   孩子們的年長父親,在別人家乞食。

   我的拐杖確實就比,那些如果不順從的兒子好,

   阻擋兇惡的牛,還有甚至兇惡的狗。

   在黑暗中它在前面,在深處得到立足處,

   以拐杖的威力,跌倒後再站起來。』」

  那時,那位大財富的婆羅門在世尊的面前記住這些偈頌後,在大群人集會的集會所中,在兒子們共坐時說:

  「在他們出生時我歡喜,又我希望他們幸福,

   他們與妻子商量後,阻擋我如狗對豬。

   確實是卑鄙不善者:他們「爹!爹!」地叫我,

   以兒子樣子的羅剎,他們捨棄走到年老者。

   如衰老無用的馬,被除去食物,

   孩子們的年長父親,在別人家乞食。

   我的拐杖確實就比,那些如果不順從的兒子好,

   阻擋兇惡的牛,還有甚至兇惡的狗。

   在黑暗中它在前面,在深處得到立足處,

   以拐杖的威力,跌倒後再站起來。」

  那時,兒子們帶那位大財富的婆羅門回家、使之沐浴後,各各以一套衣服使之裹上。

  那時,那位大財富的婆羅門拿一套衣服後去見世尊。抵達後,與世尊一起互相問候。交換應該被互相問候的友好交談後,在一旁坐下。在一旁坐下的大財富的婆羅門對世尊說這個:

  「喬達摩尊師!我們婆羅門遍求名為老師的學費,請喬達摩\twnr{尊師}{203.0}接受我的學費。」

  世尊\twnr{出自憐愍}{121.0}使(令人)接受了。

  那時,那位大財富的婆羅門對世尊說這個:

  「太偉大了,喬達摩尊師!……(中略)請喬達摩尊師記得我為\twnr{優婆塞}{98.0},從今天起\twnr{已終生歸依}{64.0}。」



\sutta{15}{15}{慢剛愎經}{https://agama.buddhason.org/SN/sn.php?keyword=7.15}
  起源於舍衛城。

  當時,名為慢剛愎的婆羅門住在舍衛城,他既不\twnr{問訊}{46.0}母親,也不問訊父親,不問訊老師,不問訊哥哥。

  當時,\twnr{世尊}{12.0}被大眾圍繞,教導法。

  那時,慢剛愎婆羅門想這個:

  「這位\twnr{沙門}{29.0}\twnr{喬達摩}{80.0}被大眾圍繞,教導法,讓我去見沙門喬達摩。如果沙門喬達摩對我說話,我也將對他說話,如果沙門喬達摩不對我說話,我也將不對他說話。」

  那時,慢剛愎婆羅門去見世尊。抵達後,沈默地在一旁站立。

  那時,世尊\twnr{沒對他說話}{x168}。

  那時,慢剛愎婆羅門[心想]:「這位沙門喬達摩什麼也不知道。」就從那裡想要再返回。

  那時,世尊以心了知慢剛愎婆羅門心中的深思後,以\twnr{偈頌}{281.0}對慢剛愎婆羅門說:

  「婆羅門!\twnr{慢是不好的}{x169},婆羅門!這裡對希求者,

   凡你以目的來者,這樣應該隨增大那個(目的)。」

  那時,慢剛愎婆羅門[心想]:「沙門喬達摩知道我的心。」就在那裡,他以頭落在世尊的腳上,並以嘴遍吻世尊的腳,又以手遍擦拭,以及告知名字:

  「喬達摩尊師!我是慢剛愎,喬達摩尊師!我是慢剛愎。」

  那時,那個群眾有\twnr{未曾有}{206.0}之心生起:

  「實在不可思議啊,先生!實在未曾有啊,先生!因為這位慢剛愎婆羅門既不問訊母親,也不問訊父親,不問訊老師,不問訊哥哥,然而卻對沙門喬達摩作像這樣最高的\twnr{身體伏在地上之禮}{40.0}。」

  那時,世尊對慢剛愎婆羅門說這個:

  「夠了,婆羅門!請你起來,請你在自己的座位坐下,因為你的心於我\twnr{已淨信}{340.0}。」

  那時,慢剛愎婆羅門在自己的座位坐下後,以偈頌對世尊說:

  「他應該對誰不作慢?且他應該對誰是有尊敬的?

   誰應該被他尊敬?誰應該被善尊重是\twnr{好的}{44.0}?」

  「對母親與父親,其次對哥哥,

   對老師第四位,他應該對他們不作慢,

   他應該對他們是有尊敬的,他們應該被他尊敬,

   他們應該被善尊重是好的。

   \twnr{阿羅漢}{5.0}們-\twnr{清涼已生者}{996.0}們,\twnr{應該被作的已作}{20.0}者們、無\twnr{漏}{188.0}者們,

   破壞慢後是不剛愎者:他應該禮敬那些無上者。」

  在這麼說時,慢剛愎婆羅門對世尊說這個:

  「太偉大了,喬達摩尊師!……(中略)請喬達摩尊師記得我為\twnr{優婆塞}{98.0},從今天起\twnr{已終生歸依}{64.0}。」



\sutta{16}{16}{反對者經}{https://agama.buddhason.org/SN/sn.php?keyword=7.16}
  起源於舍衛城。

  當時,名為愛反對者的\twnr{婆羅門}{17.0}住在舍衛城。

  那時,愛反對者婆羅門想這個:

  「讓我去見\twnr{沙門}{29.0}\twnr{喬達摩}{80.0},沙門喬達摩說任何事,就對那個,我會是對他反對者。」

  當時,\twnr{世尊}{12.0}在露天處\twnr{經行}{150.0}。

  那時,愛反對者婆羅門去見世尊。抵達後,對經行中的世尊說這個:「沙門!請你說法。」

  「被善說的不容易,被愛反對者,

   被心雜染者,被多激情者了知。

   凡能調伏激情,與心的不淨信者,

   \twnr{斷念}{211.0}瞋害後,他確實能了知被善說的。」

  在這麼說時,愛反對者婆羅門對世尊說這個:

  「太偉大了,喬達摩尊師!太偉大了,喬達摩尊師!……(中略)請喬達摩\twnr{尊師}{203.0}記得我為\twnr{優婆塞}{98.0},從今天起\twnr{已終生歸依}{64.0}。」



\sutta{17}{17}{監工經}{https://agama.buddhason.org/SN/sn.php?keyword=7.17}
  \twnr{有一次}{2.0},\twnr{世尊}{12.0}住在憍薩羅國某處叢林中。

  當時,監工婆羅墮若婆羅門在那處叢林中使(人)作工作。

  監工婆羅墮若婆羅門看見在某棵沙羅樹下坐下,盤腿、定置端直的身體、\twnr{建立面前的念後}{529.0}的世尊。看見後,想這個:

  「我在這處叢林中樂於使作的工作,這位\twnr{沙門}{29.0}\twnr{喬達摩}{80.0}樂於使作的什麼?」

  那時,監工婆羅墮若婆羅門去見世尊。抵達後,以\twnr{偈頌}{281.0}對世尊說:

  「\twnr{比丘}{31.0}!在沙羅林中你的,什麼工作被作呢?

   單獨在\twnr{林野}{142.0}中,喬達摩找到喜樂嗎?」

  「樹林中沒有應該被我作的,根已被切斷我的樹林\twnr{已枯萎}{x170},

   那個我\twnr{在林中是無稠林的}{x171}、無箭的,捨斷不樂後我樂於單獨。」

  在這麼說時,監工婆羅墮若婆羅門對世尊說這個:

  「太偉大了,喬達摩尊師!……(中略)請喬達摩\twnr{尊師}{203.0}記得我為\twnr{優婆塞}{98.0},從今天起\twnr{已終生歸依}{64.0}。」



\sutta{18}{18}{打柴者經}{https://agama.buddhason.org/SN/sn.php?keyword=7.18}
  \twnr{有一次}{2.0},\twnr{世尊}{12.0}住在憍薩羅國某處叢林中。

  當時,某位婆羅墮若姓\twnr{婆羅門}{17.0}的眾多打柴學生婆羅門徒弟去叢林。抵達後,看見在那處叢林中坐下,盤腿、定置端直的身體、\twnr{建立面前的念後}{529.0}的世尊。看見後,去見婆羅墮若姓婆羅門。抵達後,對婆羅墮若姓婆羅門說這個:

  「真的,\twnr{尊師}{203.0}!你應該知道,有\twnr{沙門}{29.0}坐在那樣的叢林中,盤腿、定置端直的身體、建立面前的念後。」

  那時,婆羅墮若姓婆羅門與那些青年一起去那處叢林,看見在那處叢林中坐下,盤腿、定置端直的身體、\twnr{建立面前的念後}{529.0}的世尊。看見後,去見世尊。抵達後,以\twnr{偈頌}{281.0}對世尊說:

  「在極恐怖的叢林深處,進入無人的空\twnr{林野}{142.0}後,

   以不動搖的以住立的以可愛的,\twnr{比丘}{31.0}!你實在極英俊地修禪。

   在既沒有歌曲之處也沒有音樂之處,單獨在林野中依止樹林的牟尼,

   這在我心中出現不可思議:凡單獨能\twnr{意喜}{320.0}地住在樹林者。

   我想你希望著\twnr{世界主共住}{x172},\twnr{無上的天界}{x173},

   為何尊師依止無人的林野,你為了梵天的到達在這裡作苦行?」

  「凡任何期待或歡喜,許多經常被依止在種種界上,

   從無智根的根源有被希求的,一切都被我從包含根處剷除。

   那個我是無期待的、無依止的、無執著的,在一切法上有清淨的見,

   到達吉祥的無上正覺後,婆羅門!我自信地獨處修禪。」

  在這麼說時,婆羅墮若姓婆羅門對世尊說這個:

  「太偉大了,\twnr{喬達摩}{80.0}尊師!太偉大了,喬達摩尊師!……(中略)從今天起\twnr{已終生歸依}{64.0}。」



\sutta{19}{19}{扶養母親者經}{https://agama.buddhason.org/SN/sn.php?keyword=7.19}
  起源於舍衛城。

  那時,扶養母親的婆羅門去見\twnr{世尊}{12.0}。抵達後,與世尊一起互相問候。交換應該被互相問候的友好交談後,在一旁坐下。在一旁坐下的扶養母親的婆羅門對世尊說這個:

  「\twnr{喬達摩}{80.0}尊師!我依法遍求\twnr{施食}{196.0},依法遍求施食後,我扶養父母,喬達摩尊師!我是這麼作者,是否我是盡義務者呢?」

  「婆羅門!你是這麼作者,你當然是盡義務者。婆羅門!凡依法遍求施食,依法遍求施食後,扶養父母者,他產生許多福德。」

  「凡母親或父親,\twnr{不免一死的人}{600.0}依法扶養,

   他以這個侍奉:在母親父親上,

   就在這裡賢智者們稱讚他,死後在天界喜悅。」

  在這麼說時,扶養母親的婆羅門對世尊說這個:

  「太偉大了,喬達摩尊師!太偉大了,喬達摩尊師!……(中略)請喬達摩\twnr{尊師}{203.0}記得我為\twnr{優婆塞}{98.0},從今天起\twnr{已終生歸依}{64.0}。」



\sutta{20}{20}{乞食者經}{https://agama.buddhason.org/SN/sn.php?keyword=7.20}
  起源於舍衛城。

  那時,乞食者婆羅門去見\twnr{世尊}{12.0}。抵達後,與世尊一起互相問候。交換應該被互相問候的友好交談後,在一旁坐下。在一旁坐下的乞食者婆羅門對世尊說這個:

  「\twnr{喬達摩}{80.0}尊師!我是乞食者,\twnr{尊師}{203.0}也是乞食者,這裡,有什麼差別?」

  「不因為那樣是乞食者:只對他人乞食的程度,

   只要受持\twnr{在家法}{x174}後,只那些他就不是\twnr{乞食者(比丘)}{31.0}。

   這裡凡福德與惡,梵行者拒斥了後,

   考量後行於世間,他確實被稱為『乞食者(比丘)』。」 

  在這麼說時,乞食者婆羅門對世尊說這個:

  「太偉大了,喬達摩尊師!……(中略)請喬達摩\twnr{尊師}{203.0}記得我為\twnr{優婆塞}{98.0},從今天起\twnr{已終生歸依}{64.0}。」



\sutta{21}{21}{散額樂窪經}{https://agama.buddhason.org/SN/sn.php?keyword=7.21}
  起源於舍衛城。

  當時,名為散額樂窪(共尊重)的婆羅門住在舍衛城,是由水純淨者、信仰以水遍純淨,住於黃昏、黎明水浴之實踐的實踐者。

  那時,\twnr{尊者}{200.0}阿難午前時穿衣、拿起衣鉢後,\twnr{為了托鉢}{87.0}進入舍衛城。在舍衛城為了托鉢行走後,\twnr{餐後已從施食返回}{512.0},去見世尊。抵達後,向世尊\twnr{問訊}{46.0}後,在一旁坐下。在一旁坐下的尊者阿難對\twnr{世尊}{12.0}說這個:

  「\twnr{大德}{45.0}!這裡,名為散額樂窪的婆羅門住在舍衛城,是由水純淨者、信仰以水遍純淨,住於黃昏、黎明水浴之實踐的實踐者。大德!願世尊\twnr{出自憐愍}{121.0}去散額樂窪婆羅門的住處,\twnr{那就好了}{44.0}!」

  世尊以沈默狀態同意。

  那時,世尊午前時穿衣、拿起衣鉢後,去散額樂窪婆羅門的住處。抵達後,在設置的座位坐下。

  那時,散額樂窪婆羅門去見世尊。抵達後,與世尊一起互相問候。交換應該被互相問候的友好交談後,在一旁坐下。世尊對在一旁坐下的散額樂窪婆羅門說這個:

  「婆羅門!傳說是真的?你是由水純淨者、信仰以水遍純淨,住於黃昏、黎明水浴之實踐的實踐者。」

  「是的,\twnr{喬達摩}{80.0}尊師!」 

  「婆羅門!但,當看見什麼理由時,你是由水純淨者、信仰以水遍純淨,住於黃昏、黎明水浴之實踐的實踐者?」

  「喬達摩尊師!這裡,凡在白天被我做的惡業,在黃昏令我以沐浴沖走(除去);凡在夜間被我做的惡業,在黎明令我以沐浴沖走,當看見這個理由時,我是由水純淨者、信仰以水遍純淨,住於黃昏、黎明水浴之實踐的實踐者。」 

  「婆羅門!法是具有戒為渡場的湖,不濁的、被善人們對善人們稱讚, 

   於該處沐浴的通曉吠陀者們,就肢體不濕地渡過彼岸。」[\suttaref{SN.7.9}]

  在這麼說時,散額樂窪婆羅門對世尊說這個: 

  「太偉大了,喬達摩尊師!太偉大了,喬達摩尊師!……(中略)請喬達摩\twnr{尊師}{203.0}記得我為\twnr{優婆塞}{98.0},從今天起\twnr{已終生歸依}{64.0}。」



\sutta{22}{22}{摳麼度色經}{https://agama.buddhason.org/SN/sn.php?keyword=7.22}
  \twnr{被我這麼聽聞}{1.0}:

  \twnr{有一次}{2.0},\twnr{世尊}{12.0}住在釋迦族中,名為摳麼度色(亞麻布)釋迦族的城鎮。

  那時,世尊午前時穿衣、拿起衣鉢後,\twnr{為了托鉢}{87.0}進入摳麼度色城。

  當時,摳麼度色的婆羅門\twnr{屋主}{103.0}們正以某些應該被作的被集合在集會所,\twnr{而天空下著毛毛雨}{385.0}。

  那時,世尊去那個集會所。

  摳麼度色的婆羅門屋主們看見正從遠處到來的世尊。看見後說這個:

  「誰是\twnr{禿頭假沙門}{439.0}?以及\twnr{誰將知道集會所之法}{x175}?」

  那時,世尊以\twnr{偈頌}{281.0}對摳麼度色的婆羅門屋主們說:

  「沒有善人們之處這非集會所,凡不說法者他們不是善人,

   捨斷貪瞋癡後,說法者們是善人。」

  在這麼說時,摳麼度色的婆羅門屋主們對世尊說這個:

  「太偉大了,\twnr{喬達摩}{80.0}尊師!太偉大了,喬達摩尊師!喬達摩尊師!猶如扶正顛倒的,或揭開隱藏的,或告知迷路者的道路,或在黑暗中持燈火:『有眼者們看見諸色。』同樣的,法被喬達摩尊師以種種\twnr{法門}{562.0}說明。這些我們\twnr{歸依}{284.0}喬達摩\twnr{尊師}{203.0}、法、\twnr{比丘僧團}{65.0},請喬達摩尊師記得我們為\twnr{優婆塞}{98.0},從今天起\twnr{已終生歸依}{64.0}。」

  優婆塞品第二,其\twnr{攝頌}{35.0}:

  「耕田、優達亞、提婆西多,某位大財富者,

   慢剛愎、反對者,監工、打柴者,

   扶養母親者、乞食者,散額樂窪與摳麼度色第十二。」

  婆羅門相應完成。





\page

\xiangying{8}{婆耆舍相應}
\sutta{1}{1}{已出離經}{https://agama.buddhason.org/SN/sn.php?keyword=8.1}
  \twnr{被我這麼聽聞}{1.0}:

  \twnr{有一次}{2.0},\twnr{尊者}{200.0}婆耆舍與尊者尼拘律葛波\twnr{和尚}{108.0},共住在阿羅婆處的阿格羅婆\twnr{塔廟}{366.0}。

  當時,尊者婆耆舍是\twnr{新學}{210.0}、出家不久者、守護住處的留守者。

  那時,眾多女子極好裝飾後,為了觀看住處(精舍)去阿格羅婆的園林。

  那時,看見那些女人後,尊者婆耆舍的不喜樂被生起,貪使心墮落。

  那時,尊者婆耆舍想這個:

  「確實是我的無利得,確實不是我的利得;確實是我的惡得的,確實不是我的善得的:我的不喜樂被生起,貪使心墮落。在這裡,\twnr{那如何可得}{847.0}:凡其他人為我除去不喜樂後,能使喜樂生起。讓我就自己為自己除去不喜樂後,能使喜樂生起。」

  那時,尊者婆耆舍就自己為自己除去不喜樂後,使喜樂生起後,那時候說這些\twnr{偈頌}{281.0}:

  「確實是已出離的我:從在家成為非家者,

   諸尋圍繞:這些從黑暗[來]的大膽者。

   顯貴之子、大弓箭手,已學得者、堅固法者們,

   一千位不逃跑者,如果他們到處包圍。

   即使如果比這更多的,女人們將到來,

   將確實不使我動搖:諸法已被住立在自己上。

   因為這個被我當面聽聞,佛陀、太陽族人的,

   導向涅槃之道,在那裡被我的意(心)喜好的。

   如果當這樣住時,\twnr{波旬}{49.0}!你靠近我,

   死神!我將像這樣做,你沒看見我的道。」



\sutta{2}{2}{不喜樂經}{https://agama.buddhason.org/SN/sn.php?keyword=8.2}
  \twnr{有一次}{2.0}……(中略)。

  \twnr{尊者}{200.0}婆耆舍與尊者尼拘律葛波\twnr{和尚}{108.0},共住在阿羅婆的阿格羅婆\twnr{塔廟}{366.0}。

  當時,尊者尼拘律葛波\twnr{餐後已從施食返回}{512.0},進入住處,或傍晚時出來,或在隔天時。

  當時,尊者婆耆舍的不喜樂被生起,貪使心墮落。

  那時,尊者婆耆舍想這個:

  「確實是我的無利得,確實不是我的利得;確實是我的惡得的,確實不是我的善得的:我的不喜樂被生起,貪使心墮落。在這裡,\twnr{那如何可得}{847.0}:凡其他人為我除去不喜樂後,能使喜樂生起。讓我就自己為自己除去不喜樂後,能使喜樂生起。」

  那時,尊者婆耆舍就自己為自己除去不喜樂後,使喜樂生起後,那時候說這些\twnr{偈頌}{281.0}:

  「完全捨斷不喜樂與喜樂,\twnr{以及掛慮家的尋}{x176}後,

   無論何處都不應該作欲林,無欲林者、不喜樂者他確實是\twnr{比丘}{31.0}。

   這裡凡大地與虛空,色之類與進入世界者,

   無論什麼都衰損而一切是無常的,\twnr{有所覺者們}{x177}這麼知道後而行。

   被繫縛在\twnr{依著}{198.0}上的人們:在所見、所聞、\twnr{有對}{x178}與所覺上,

   在這裡驅離意欲後成為不動者,凡在這裡不沾染者他們說他是\twnr{牟尼}{125.0}。

   \twnr{而依止六十的有尋}{x179}:在人們中多數是非法的安頓者,

   無論在哪裡他都不會是入群者,又非粗惡語者他是比丘。

   有能力的、長久得定的,非詭計的、明智的、無熱望的,

   \twnr{緣於}{252.0}牟尼到達寂靜之境,他等待\twnr{般涅槃}{72.0}時。」



\sutta{3}{3}{美善者經}{https://agama.buddhason.org/SN/sn.php?keyword=8.3}
  \twnr{有一次}{2.0},\twnr{尊者}{200.0}婆耆舍與尊者尼拘律葛波\twnr{和尚}{108.0},共住在阿羅婆的阿格羅婆\twnr{塔廟}{366.0}。

  當時,尊者婆耆舍以自己的辯才輕蔑其他美善\twnr{比丘}{31.0}們。

  那時,尊者婆耆舍想這個:

  「確實是我的無利得,確實不是我的利得;確實是我的惡得的,確實不是我的善得的:我以自己的辯才輕蔑其他美善比丘們。」

  那時,尊者婆耆舍就以自己對自己的懊悔生起後,那時候說這些\twnr{偈頌}{281.0}:

  「\twnr{喬達摩}{80.0}!請你捨斷慢,以及請你無殘餘地捨斷\twnr{慢的道路}{x180},

   在慢的道路上成為癡迷者,你長久是後悔者。

   人們被詆毀塗污,被慢殺害者們跌落地獄,

   人們長久地悲傷(憂愁):往生地獄的被慢殺害者們。

   比丘任何時候都不悲傷:\twnr{道的勝利者}{x181}、正行者,

   他經驗名聲與安樂,他們說他是『自我努力的法之看見者』。

   因此這裡不頑固的、勤奮的,捨斷諸蓋後成為清淨的,

   而無餘地捨斷慢後,成為以明得到結束的寂靜者。」



\sutta{4}{4}{阿難經}{https://agama.buddhason.org/SN/sn.php?keyword=8.4}
  \twnr{有一次}{2.0},\twnr{尊者}{200.0}阿難住在舍衛城祇樹林給孤獨園。

  那時,尊者阿難午前時穿衣、拿起衣鉢後,以尊者婆耆舍為隨從\twnr{沙門}{29.0},\twnr{為了托鉢}{87.0}進入舍衛城。

  當時,尊者婆耆舍的不喜樂被生起,貪使心墮落。那時,尊者婆耆舍以\twnr{偈頌}{281.0}對尊者阿難說:

  「我被欲貪燃燒,我的心被遍燃燒,

   \twnr{喬達摩}{80.0}!請你出於憐愍,說使之熄滅的事\twnr{那就好了}{44.0}。」

  「由於想的顛倒,你的心被遍燃燒,

   請你避開相:淨的、伴隨貪的。

   請你視諸行為\twnr{另一邊的}{431.0},為苦的以及不要[視]為我,

   請你使大貪熄滅,你不要一再地被燃燒。

   請你以不淨\twnr{修習}{94.0}心,\twnr{一境}{255.0}、善入定,

   願你的\twnr{身至念}{521.0}存在,願你成為多\twnr{厭}{15.0}者。

   \twnr{請你修習無相}{x182},請你捨棄\twnr{慢煩惱潛在趨勢}{26.0},

   之後\twnr{以慢的止滅}{592.1},你將寂靜地行。」



\sutta{5}{5}{善說的經}{https://agama.buddhason.org/SN/sn.php?keyword=8.5}
  起源於舍衛城。

  在那裡,\twnr{世尊}{12.0}召喚\twnr{比丘}{31.0}們:「比丘們!」

  「\twnr{尊師}{480.0}!」那些比丘回答世尊。

  「比丘們!具備四支的言語是善說的;非惡說的,是無過失的與在智者們中不能被非難的。哪四個?比丘們!這裡,比丘只說善說的;非惡說的、只說法;非非法、只說可愛的;非不可愛的、只說真實;非虛偽。比丘們!具備這四支的言語是善說的;非惡說的,是無過失的與在智者們中不能被非難的。」

  世尊說這個,說這個後,\twnr{善逝}{8.0}、\twnr{大師}{145.0}又更進一步說這個:

  「善人們說善說的是最先的,應該說法、非非法那是第二的,

   應該說可愛的、非不可愛的那是第三的,應該說真實、非虛偽那是第四的。」

  那時,\twnr{尊者}{200.0}婆耆舍從座位起來後,置(作)上衣到一邊肩膀,向世尊合掌鞠躬後,對世尊說這個:

  「世尊!它在我心中出現;善逝!它在我心中出現。」

  「婆耆舍!請你說明。」世尊說。

  那時,尊者婆耆舍以適合的\twnr{偈頌}{281.0}當面稱讚世尊:

  「只應該說那個言語:凡不會使自己苦惱者,

   以及不會傷害他人,那確實是善說的言語。

   只應該說可愛的言語:凡受歡迎的言語,

   凡不帶著諸惡的,說對他人所愛的(可愛的)。

   真實確實是\twnr{不死}{123.0}的言語,這是古老的法,

   利益與法,善人們說被住立在真實上。

   凡佛陀說言語者,為了安穩涅槃的到達,

   為了苦的結束,那確實是最上的言語。」



\sutta{6}{6}{舍利弗經}{https://agama.buddhason.org/SN/sn.php?keyword=8.6}
  \twnr{有一次}{2.0},\twnr{尊者}{200.0}舍利弗住在舍衛城祇樹林給孤獨園。

  當時,尊者舍利弗以法說,以優雅的、明瞭的、\twnr{清晰的}{927.0}、義理令知的話語對\twnr{比丘}{31.0}們開示、勸導、鼓勵、\twnr{使歡喜}{86.0}。而那些比丘\twnr{作目標後}{316.0}、作意後、\twnr{全心注意後}{479.0}傾耳聽法。

  那時,尊者婆耆舍想這個:

  「這位尊者舍利弗以法說,以優雅的、明瞭的、清晰的、義理令知的話語對比丘們開示、勸導、鼓勵、使歡喜。而那些比丘作目標後、作意後、全心注意後傾耳聽法,讓我當面以適合的\twnr{偈頌}{281.0}稱讚尊者舍利弗。」

  那時,尊者婆耆舍從座位起來後,置(作)上衣到一邊肩膀,向尊者舍利弗合掌鞠躬後,對尊者舍利弗說這個:

  「舍利弗\twnr{學友}{201.0}!它在我心中出現;舍利弗學友!它在我心中出現。」

  「婆耆舍學友!請你說明。」

  那時,尊者婆耆舍當面以適合的\twnr{偈頌}{281.0}稱讚尊者舍利弗:

  「深慧者、有智慧者,道非道的熟知者,

   大慧的舍利弗,為比丘們教導法。

   既簡要地教導,也詳細地說,

   如九官鳥的聲音,他說辯才。

   當他教導那個時,他們聽聞如蜜的話語,

   以誘人的聲音,以悅耳的以可愛的,

   使心踊躍的、喜悅的,比丘們傾耳。」



\sutta{7}{7}{自恣經}{https://agama.buddhason.org/SN/sn.php?keyword=8.7}
  \twnr{有一次}{2.0},\twnr{世尊}{12.0}與約五百位\twnr{比丘}{31.0}全部都是\twnr{阿羅漢}{5.0}的大比丘\twnr{僧團}{375.0},住在舍衛城東園鹿母講堂。

  當時,在十五\twnr{那個布薩}{222.0}\twnr{自恣日}{344.2},世尊被比丘僧團圍繞,坐\twnr{在屋外}{385.0}。

  那時,世尊觀察沈默的比丘僧團後,召喚比丘們:

  「比丘們!來吧,現在我邀請你們:但是否你們呵責我任何與身體有關的或與言語有關的呢?」

  在這麼說時,\twnr{尊者}{200.0}舍利弗從座位起來後,置(作)上衣到一邊肩膀,向世尊合掌鞠躬後,對世尊說這個:

  「\twnr{大德}{45.0}!我們不呵責世尊任何與身體有關的或與言語有關的。因為,世尊是未生起道的使生起者,未出生道的使出生者,未宣說道的宣說者、道的知者、道的熟練者、道的熟知者,大德!而且,現在弟子們住於道的跟隨者,之後為具備者。

  大德!而我邀請世尊:但是否世尊呵責我任何與身體有關的或與言語有關的呢?」

  「舍利弗!我不呵責你任何與身體有關的或與言語有關的。舍利弗!你是賢智者;舍利弗!你是大慧者;舍利弗!你是博慧者;舍利弗!你是捷慧者;舍利弗!你是速慧者;舍利弗!你是\twnr{利慧}{978.0}者;舍利弗!你是\twnr{洞察慧}{566.0}者[\suttaref{SN.2.29}, \ccchref{MN.111}{https://agama.buddhason.org/MN/dm.php?keyword=111}]。舍利弗!猶如\twnr{轉輪王}{278.0}的長子,使被父親轉起的輪子完全地隨轉動(持續轉動)。同樣的,舍利弗!你使被我轉起的無上法輪完全地隨轉動。」

  「大德!如果世尊不呵責我任何與身體有關的或與言語有關的,大德!那麼是否世尊呵責這五百位比丘任何與身體有關的或與言語有關的呢?」

  「舍利弗!我不呵責這五百位比丘任何與身體有關的或與言語有關的。舍利弗!因為這五百位比丘中,六十位比丘是\twnr{三明}{133.0}者,六十位比丘是六神通者,六十位比丘是\twnr{俱分解脫}{351.0}者,而其他是\twnr{慧解脫者}{539.0}。」

  那時,尊者婆耆舍從座位起來後,置(作)上衣到一邊肩膀,向世尊合掌鞠躬後,對世尊說這個:

  「世尊!它在我心中出現;\twnr{善逝}{8.0}!它在我心中出現。」

  「婆耆舍!請你說明。」世尊說。

  那時,尊者婆耆舍以適合的\twnr{偈頌}{281.0}當面稱讚世尊:

  「在十五的今日為了清淨,五百位比丘已來集合:

   結繫縛切斷的無惱亂的,\twnr{再有已盡的仙人}{x183}。

   如轉輪王,被大臣圍繞,

   繞著全部,海洋為邊界的這個大地走。

   像這樣戰場上的勝利者,\twnr{無上的商隊領袖}{x184},

   超越死亡的三明者:弟子們侍奉。

   全部都是世尊之子,在這裡閒聊不被發現,

   願我禮拜\twnr{太陽族人}{662.0}:渴愛刺箭的破壞者。」



\sutta{8}{8}{超過一千經}{https://agama.buddhason.org/SN/sn.php?keyword=8.8}
  \twnr{有一次}{2.0},\twnr{世尊}{12.0}與一千二百五十位\twnr{比丘}{31.0}的大比丘\twnr{僧團}{375.0},住在舍衛城祇樹林給孤獨園。

  當時,世尊以涅槃關聯的法說對比丘們開示、勸導、鼓勵、\twnr{使歡喜}{86.0},而那些比丘\twnr{作目標後}{316.0}、作意後、\twnr{全心注意後}{479.0}傾耳聽法。

  那時,\twnr{尊者}{200.0}婆耆舍想這個:

  「這位世尊以涅槃關聯的法說對比丘們開示、勸導、鼓勵、使歡喜,而那些比丘作目標後、作意後、全心注意後傾耳聽法,讓我以適合的\twnr{偈頌}{281.0}當面稱讚世尊。」

  那時,尊者婆耆舍從座位起來後,置(作)上衣到一邊肩膀,向世尊\twnr{合掌}{377.0}鞠躬後,對世尊說這個:

  「世尊!它在我心中出現;\twnr{善逝}{8.0}!它在我心中出現。」

  「婆耆舍!請你說明。」世尊說。

  那時,尊者婆耆舍以適合的偈頌當面稱讚世尊:

  「超過一千位比丘,侍奉善逝,

   當他教導遠塵之法,任何地方都無恐懼的涅槃時。

   他們聽聞離垢之法:被\twnr{遍正覺者}{6.0}教導的,

   \twnr{正覺者}{185.1}確實輝耀:被比丘僧團尊敬的。

   世尊!你的名字是龍,仙人中最上的仙人,

   如大雨雲生成後,下雨到諸弟子。

   從白天的住處出來後,從看見大師的翼求,

   大英雄!弟子婆耆舍,禮拜在你的腳上。」

  「婆耆舍!是否這些偈頌之前已遍尋思,或者僅立即在你心中出現呢?」

  「\twnr{大德}{45.0}!這些偈頌非之前已遍尋思,反而僅立即在我心中出現。」

  「婆耆舍!那樣的話,令更多非之前已遍尋思的偈頌在你心中出現。」

  「是的,大德!」尊者婆耆舍回答世尊後,以更多非之前已遍尋思的偈頌當面稱讚世尊:

  「征服魔的邪道之路後,破壞諸\twnr{荒蕪}{599.0}後而行,

   請你們看他-繫縛的令脫者,不依止的、分析成部分者。

   為了暴流的渡過,他講述許多種的道,

   而在那個被講述的\twnr{不死}{123.0}上,法的看見者成為不能被動搖的住立者。

   光明的作者洞察後,看見一切住止處的超越,

   知道後與作證後,他對五位教導最上的。

   在法被這麼善教導時,了知法者有什麼放逸?

   因此在那位世尊的教說下,不放逸者應該總是禮敬地\twnr{隨學}{398.0}。」



\sutta{9}{9}{憍陳如經}{https://agama.buddhason.org/SN/sn.php?keyword=8.9}
  \twnr{有一次}{2.0},\twnr{世尊}{12.0}住在王舍城栗鼠飼養處的竹林中。

  那時,\twnr{尊者}{200.0}\twnr{阿若憍陳如}{x185}隔了很久後去見世尊。抵達後,以頭落在世尊的腳上,並以嘴遍吻世尊的腳,又以手遍擦拭,以及告知名字:「世尊!我是憍陳如,\twnr{善逝}{8.0}!我是憍陳如。」

  那時,尊者婆耆舍想這個:

  「這位尊者阿若憍陳如隔了很久後去見世尊。抵達後,以頭落在世尊的腳上,並以嘴遍吻世尊的腳,又以手遍擦拭,以及告知名字:『世尊!我是憍陳如,善逝!我是憍陳如。』讓我在世尊的面前以適合的\twnr{偈頌}{281.0}稱讚尊者阿若憍陳如。」

  那時,尊者婆耆舍從座位起來後,置(作)上衣到一邊肩膀,向世尊合掌鞠躬後,對世尊說這個:

  「世尊!它在我心中出現;善逝!它在我心中出現。」

  「婆耆舍!請你說明。」世尊說。

  那時,尊者婆耆舍在世尊的面前以適合的偈頌稱讚尊者阿若憍陳如:

  「那位佛陀的隨覺者上座,極精勤的憍陳如,

   是安樂的住處,經常獨處的利得者。

   凡應該被弟子達成的:以大師的教說作的,

   他已達到一切:不放逸的學習者。

   三明的大威力者,知他心的熟練者,

   憍陳如-佛陀的繼承者,在大師的腳上禮拜。」



\sutta{10}{10}{目揵連經}{https://agama.buddhason.org/SN/sn.php?keyword=8.10}
  \twnr{有一次}{2.0},\twnr{世尊}{12.0}與約五百位\twnr{比丘}{31.0}全部都是\twnr{阿羅漢}{5.0}的大比丘\twnr{僧團}{375.0},住在王舍城仙吞山坡的黑岩處。

  \twnr{尊者}{200.0}大目揵連以心探查他們的心:[他們]是解脫者、無\twnr{依著}{198.0}者。

  那時,尊者婆耆舍想這個:

  「這位世尊與約五百位比丘全部都是阿羅漢的大比丘僧團,住在王舍城仙吞山坡的黑岩處,尊者大目揵連以心探查他們的心:[他們]是解脫者、無依著者,讓我在世尊的面前以適合的\twnr{偈頌}{281.0}稱讚尊者大目揵連。」

  那時,尊者婆耆舍從座位起來後,置(作)上衣到一邊肩膀,向世尊合掌鞠躬後,對世尊說這個:

  「世尊!它在我心中出現;\twnr{善逝}{8.0}!它在我心中出現。」

  「婆耆舍!請你說明。」世尊說。

  那時,尊者婆耆舍在世尊的面前以適合的偈頌稱讚尊者大目揵連:

  「當坐在山坡時:已到達苦之彼岸的\twnr{牟尼}{125.0},

   弟子們侍奉:\twnr{三明}{133.0}者、死亡的捨棄者們。

   以心繞著他們走:大神通力的目揵連,

   探查著他們的心,[他們]是解脫者、無依著者。

   像這樣具足所有部分者:已到達苦之彼岸的牟尼,

   具足種種行相者:他們侍奉\twnr{喬達摩}{80.0}。」



\sutta{11}{11}{伽伽羅經}{https://agama.buddhason.org/SN/sn.php?keyword=8.11}
  \twnr{有一次}{2.0},\twnr{世尊}{12.0}與約五百位\twnr{比丘}{31.0}的大比丘\twnr{僧團}{375.0}、七百位\twnr{優婆塞}{98.0}、七百位\twnr{優婆夷}{99.0}、好幾千位天神,共住在瞻波城伽伽羅蓮花池邊,世尊以容色與名聲比他們更輝耀。

  那時,\twnr{尊者}{200.0}婆耆舍想這個:

  「這位世尊與約五百位比丘的大比丘僧團、七百位優婆塞、七百位優婆夷、好幾千位天神,共住在瞻波城伽伽羅蓮花池邊,世尊以容色與名聲比他們更輝耀,讓我以適合的\twnr{偈頌}{281.0}當面稱讚世尊。」

  那時,尊者婆耆舍從座位起來後,置(作)上衣到一邊肩膀,向世尊合掌鞠躬後,對世尊說這個:

  「世尊!它在我心中出現;\twnr{善逝}{8.0}!它在我心中出現。」

  「婆耆舍!請你說明。」世尊說。

  那時,尊者婆耆舍以適合的偈頌當面稱讚世尊:

  「如月在雲被驅離的天空,如離垢的太陽照耀,

   \twnr{放光者}{662.2}!像這樣你是大\twnr{牟尼}{125.0},以名聲比一切世間更輝耀。」



\sutta{12}{12}{婆耆舍經}{https://agama.buddhason.org/SN/sn.php?keyword=8.12}
  \twnr{有一次}{2.0},\twnr{尊者}{200.0}婆耆舍住在舍衛城祇樹林給孤獨園。

  當時,尊者婆耆舍成為到達\twnr{阿羅漢}{5.0}境界者不久後,在感受著解脫樂的那時,說這些\twnr{偈頌}{281.0}:

  「以前陶醉在詩中的我們遊蕩,\twnr{從村到村從城到城}{x186},

   然後看見\twnr{正覺者}{185.1},我們的信生起。

   他為我教導法:蘊、處、界,

   我聽聞他的法後,出家成為非家者。

   確實為了許多[眾生]的利益,\twnr{牟尼}{125.0}到達覺(菩提),

   為了\twnr{比丘}{31.0}與比丘尼們,\twnr{到達決定的看見者}{x187}。

   確實是我的善來到的:到我佛陀的面前,

   三明已被達到,佛陀的教說已被作。

   我知道前世住處,天眼已被淨化,

   知他心的熟練者:我是三明者、已得到神通者。」

  婆耆舍相應完成,其\twnr{攝頌}{35.0}:

  「已出離與不喜樂,輕蔑美善者,

   阿難與善說的,舍利弗、自恣, 

   超過一千、憍陳如,目揵連與伽伽羅,

   與婆耆舍為十二。」





\page

\xiangying{9}{林相應}
\sutta{1}{1}{遠離經}{https://agama.buddhason.org/SN/sn.php?keyword=9.1}
  \twnr{被我這麼聽聞}{1.0}:

  \twnr{有一次}{2.0},\twnr{某位}{39.0}\twnr{比丘}{31.0}住在憍薩羅國某處叢林中。

  當時,進入白天的住處的那位比丘,在掛慮家的惡不善尋中尋思。

  那時,那位住在那處叢林中、對那位比丘有憐愍、想要利益、想要激起那位比丘\twnr{急迫感}{373.0}的天神去見那位比丘。抵達後,以\twnr{偈頌}{281.0}對那位比丘說:

  「你想要遠離進入樹林,然後你的意(心)出去外面,

   人!請你調伏對人的意欲,從那裡你將會是安樂、離貪的。

   你捨斷不喜樂你成為具念的,我們使你憶念善的,

   塵垢深淵確實是難越過的,不要讓欲的塵垢抓下你。

   如\twnr{佈滿塵土的}{x188}鳥,當抖動時使附著的塵垢落下,

   像這樣勤奮有念的比丘,當抖動時使附著的塵垢落下。」

  那時,那位比丘被那位天神激起急迫感,來到了急迫感。



\sutta{2}{2}{起來經}{https://agama.buddhason.org/SN/sn.php?keyword=9.2}
  \twnr{有一次}{2.0},\twnr{某位比丘}{39.0}住在憍薩羅國某處叢林中。

  當時,進入白天住處的那位比丘睡覺。

  那時,那位住在那處叢林中、對那位比丘有憐愍、想要利益、想要激起那位比丘\twnr{急迫感}{373.0}的天神去見那位比丘。抵達後,以\twnr{偈頌}{281.0}對那位比丘說:

  「比丘!請你起來-你為何躺下?睡眠對你有什麼利益?

   對生病者睡眠有什麼?對被箭射穿破壞者呢?

   凡\twnr{以信從在家出家成為無家者}{48.0},

   請你增加那個信,不要落入睡眠的控制。」

  「欲是無常的、不堅固的,僅愚鈍者在那些上被迷昏頭,

   對在束縛上\twnr{不依止的}{x189}解脫者,對出家人為何會苦惱?

   以意欲貪的調伏,以無明的超越,

   對那個智最高明淨者,對出家人為何會苦惱?

   \twnr{以明切斷無明後}{x190},以諸\twnr{漏}{188.0}的遍盡,

   對無憂愁無\twnr{絕望}{342.0}者,對出家人為何會苦惱?

   對活力已被發動、自我努力者,對常堅固努力者,

   對寂滅(涅槃)期待者,對出家人為何會苦惱?」



\sutta{3}{3}{迦葉氏經}{https://agama.buddhason.org/SN/sn.php?keyword=9.3}
  \twnr{有一次}{2.0},\twnr{尊者}{200.0}迦葉氏住在憍薩羅國某處叢林中。

  當時,進入白天住處的尊者迦葉氏教誡某位\twnr{獵人}{x191}。

  那時,那位住在那處叢林中、想要激起尊者迦葉氏\twnr{急迫感}{373.0}的天神去見尊者迦葉氏。抵達後,以\twnr{偈頌}{281.0}對尊者迦葉氏說:

  「對難行山路行走的獵人,對少慧的無心者,

   \twnr{比丘}{31.0}在不適當時機的教誡,在我心中出現像愚鈍者。

   他聽而不知,視而不見,

   在被說的法上,愚者不覺義理。

   迦葉!即使如果你將攜帶,十盞燈火來,

   他甚至沒看見諸色,因為他的眼睛不存在。」

  那時,尊者迦葉氏被那位天神激起急迫感,來到了急迫感。



\sutta{4}{4}{眾多經}{https://agama.buddhason.org/SN/sn.php?keyword=9.4}
  \twnr{有一次}{2.0},眾多\twnr{比丘}{31.0}住在憍薩羅國某處叢林中。

  那時,雨季已下過,經過三個月,那些比丘出發遊行。

  那時,那位住在那處叢林中、不見那些比丘、悲泣的天神在那時說這\twnr{偈頌}{281.0}:

  「今天對我來說看起來像不喜樂的樣子:看見許多遠離的(空的)座位後,

   那些巧說的多聞者,這些\twnr{喬達摩}{80.0}的弟子到哪裡了?」

  在這麼說時,另一位天神以偈頌回應那位天神:

  「到摩揭陀到憍薩羅,而某些是在跋耆地區,

   如野鹿般無執著的行者:比丘們無家屋地住。」



\sutta{5}{5}{阿難經}{https://agama.buddhason.org/SN/sn.php?keyword=9.5}
  \twnr{有一次}{2.0},\twnr{尊者}{200.0}阿難住在憍薩羅國某處叢林中。

  當時,尊者阿難住於過多\twnr{對在家人的撫慰}{x192}。

  那時,那位住在那處叢林中、對尊者阿難有憐愍、想要利益、想要激起尊者阿難\twnr{急迫感}{373.0}的天神去見尊者阿難。抵達後,以\twnr{偈頌}{281.0}對尊者阿難說:

  「到樹叢的樹下後,涅槃放進心中後,

   \twnr{喬達摩}{80.0}!請你修禪不要放逸,說個不停對你將能作什麼?」

  那時,尊者阿難被那位天神激起急迫感,來到了急迫感。



\sutta{6}{6}{阿那律經}{https://agama.buddhason.org/SN/sn.php?keyword=9.6}
  \twnr{有一次}{2.0},\twnr{尊者}{200.0}阿那律住在憍薩羅國某處叢林中。

  那時,某位名叫渣里尼(有網)、尊者阿那律的\twnr{前妻}{756.0}、\twnr{三十三天眾}{280.0}的天神去見尊者阿那律。抵達後,以\twnr{偈頌}{281.0}對尊者阿那律說:

  「請你將心志向那裡,往你以前所住之處,

   往三十三天:所有欲的成功處,

   被天女們尊敬、圍繞,你將輝耀。」

  「天女們是不幸的:在\twnr{有身}{93.0}中住立者,

   那些眾生也是不幸的:\twnr{被天女欲求者}{x193}。」

  「他們不了知樂:凡沒見過\twnr{歡喜園}{704.0}者,

   是天人們的住所:三十三天的有名聲者。」

  「愚者!你不了知,關於\twnr{阿羅漢}{5.0}們的言語:

   一切行[確實]是無常的,是\twnr{生起與消散法的}{681.0},

   生起後被滅,它們的平息是樂。[\suttaref{SN.1.11}]

   渣里尼!在天眾中,現在不再有住所,

   出生的輪迴已消盡,現在沒有\twnr{再有}{21.0}。」



\sutta{7}{7}{那格勒得經}{https://agama.buddhason.org/SN/sn.php?keyword=9.7}
  \twnr{有一次}{2.0},\twnr{尊者}{200.0}那格勒得住在憍薩羅國某處叢林中。

  當時,尊者那格勒得太早進入村落,午後回來。

  那時,那位住在那處叢林中、對尊者那格勒得有憐愍、想要利益、想要激起尊者那格勒得\twnr{急迫感}{373.0}的天神去見尊者那格勒得。抵達後,以\twnr{偈頌}{281.0}對尊者那格勒得說:

  「那格勒得!經常進入後(羅馬拼音版),以及在白天返回後為過度的行者,

   為與在家人交際的,同苦樂者。

   我害怕那格勒得成為極大膽的,被繫縛於家者,

   不要就來到有力死王的,死神的控制。」

  那時,尊者那格勒得被那位天神激起急迫感,來到了急迫感。



\sutta{8}{8}{家婦經}{https://agama.buddhason.org/SN/sn.php?keyword=9.8}
  \twnr{有一次}{2.0},某位\twnr{比丘}{31.0}住在憍薩羅國某處叢林中。

  當時,那位比丘在某個家庭上住於過度\twnr{到達深入的}{x194}。

  那時,那位住在那處叢林中、對那位比丘有憐愍、想要利益、想要激起那位比丘\twnr{急迫感}{373.0}的天神化作那個家婦的容色後,去見那位比丘。抵達後,以\twnr{偈頌}{281.0}對那位比丘說:

  「在河岸邊、休息處,集會所與道路中,

   人們會集後,討論我與你之間怎麼了。」

  「確實有許多反對的聲音,應該被苦行者忍受(忍耐),

   不應該被那個成為沮喪的,因為他不因為那樣而染污。

   而凡對聲音有恐懼者,如林中的羚羊,

   他們說他是『\twnr{輕心者}{x195}』,他的禁制(淨行)不會成功。」



\sutta{9}{9}{跋耆子經}{https://agama.buddhason.org/SN/sn.php?keyword=9.9}
  \twnr{有一次}{2.0},某位跋耆子\twnr{比丘}{31.0}住在毘舍離某處叢林中。

  當時,在毘舍離的跋耆子有\twnr{整夜之行}{x196}。

  那時,那位比丘聽聞毘舍離的樂器、敲打的[節奏]、音樂吵鬧聲後,悲嘆著,在那時說這\twnr{偈頌}{281.0}:

  「我們獨自住在\twnr{林野}{142.0}中,像被捨棄在林中的圓木,

   在像這樣的夜晚,有誰比我們更糟?」

  那時,那位住在那處叢林中、對那位比丘有憐愍、想要利益、想要激起那位比丘\twnr{急迫感}{373.0}的天神去見那位比丘。抵達後,以偈頌對那位比丘說:

  「正是你獨自住在林野中,像被捨棄在林中的圓木,

   多數人羨慕你的那個,如墮地獄者對生天者般。」

  那時,那位比丘被那位天神激起急迫感,來到了急迫感。



\sutta{10}{10}{誦讀經}{https://agama.buddhason.org/SN/sn.php?keyword=9.10}
  \twnr{有一次}{2.0},\twnr{某位比丘}{39.0}住在憍薩羅國某處叢林中。

  當時,凡之前住於過度熱心誦讀的那位比丘,他過些時候\twnr{不活動}{906.0}、變沈默地保持靜止。

  那時,那位住在那處叢林中、聽不到那位比丘的法的天神去見那位比丘。抵達後,以\twnr{偈頌}{281.0}對那位比丘說:

  「比丘!你為何不用心學習諸\twnr{法句}{944.0}:當比丘們共住時?

   聽聞法後得到淨信,當生得到稱讚。」

  「之前在諸法句上有意欲,直到我們與離貪會合為止,

   自從我們與離貪會合,凡任何所見的或所聞的或所覺的,

   了知後善人們說放下(捨棄)。」



\sutta{11}{11}{不善尋經}{https://agama.buddhason.org/SN/sn.php?keyword=9.11}
  \twnr{有一次}{2.0},\twnr{某位比丘}{39.0}住在憍薩羅國某處叢林中。

  當時,進入白天住處的那位比丘尋思諸惡不善尋,即:欲尋、惡意尋、\twnr{加害尋}{376.2}。

  那時,那位住在那處叢林中、對那位比丘有憐愍、想要利益、想要激起那位比丘\twnr{急迫感}{373.0}的天神去見那位比丘。抵達後,以\twnr{偈頌}{281.0}對那位比丘說:

  「不\twnr{如理作意}{114.0},\twnr{你被那個諸尋吃掉}{x197},

   捨離(斷念)不如理後,你應該如理思索。

   對\twnr{大師}{145.0}、法,\twnr{僧團}{375.0}、自己的戒精勤後,

   你得到欣悅,喜、安樂、無疑,

   之後富於欣悅的,你將\twnr{作苦的終結}{54.0}。」

  那時,那位比丘被那位天神激起急迫感,來到了急迫感。



\sutta{12}{12}{中午經}{https://agama.buddhason.org/SN/sn.php?keyword=9.12}
  \twnr{有一次}{2.0},\twnr{某位比丘}{39.0}住在憍薩羅國某處叢林中。

  那時,那位住在那處叢林中的天神去見那位比丘。抵達後,在那位比丘面前說這\twnr{偈頌}{281.0}:

  「中午時分已住立時,在鳥兒們已安靜時,

   廣大的\twnr{林野}{142.0}就出聲,那個恐怖在我心中出現。」

  「中午時分已住立時,在鳥兒們已安靜時,

   廣大的林野就出聲,那個喜樂在我心中出現。」[\suttaref{SN.1.15}]



\sutta{13}{13}{諸根不控制經}{https://agama.buddhason.org/SN/sn.php?keyword=9.13}
  \twnr{有一次}{2.0},掉舉的、傲慢的、浮躁的、饒舌的、言語散亂的、\twnr{念已忘失的}{216.0}、不正知的、不得定的、散亂心的、根不控制的眾多\twnr{比丘}{31.0}住在憍薩羅國某處叢林中。

  那時,那位住在那處叢林中、對那些比丘有憐愍、想要利益、想要激起那些比丘\twnr{急迫感}{373.0}的天神去見那些比丘。抵達後,以\twnr{偈頌}{281.0}對那些比丘說:

  「之前比丘們是快樂生活者:\twnr{喬達摩}{80.0}的弟子們,

   \twnr{無欲求的食物尋求者}{x198},無欲求的住所[尋求者],

   知道世間中的無常性後,他們得到了苦的結束。

   置(作)自己為難養者後,如村落中的村長,

   一再吃後他們躺臥,在他人家中被迷昏頭。

   對\twnr{僧團}{375.0}作\twnr{合掌}{377.0}後,這裡我說某些人,

   他們是被拋棄者、無庇護者,就如那些亡者那樣地。

   凡住於放逸者,關於他們被我說,

   凡住於不放逸者,我要對他們作禮敬。[\suttaref{SN.2.25}]」

  那時,那些比丘被那位天神激起急迫感,來到了急迫感。



\sutta{14}{14}{盜香者經}{https://agama.buddhason.org/SN/sn.php?keyword=9.14}
  \twnr{有一次}{2.0},\twnr{某位比丘}{39.0}住在憍薩羅國某處叢林中。

  當時,那位比丘\twnr{餐後已從施食返回}{512.0},進入蓮花池後,聞紅蓮花。

  那時,那位住在那處叢林中、對那位比丘有憐愍、想要利益、想要激起那位比丘\twnr{急迫感}{373.0}的天神去見那位比丘。抵達後,以\twnr{偈頌}{281.0}對那位比丘說:

  「凡你聞,這未被給與的蓮花,

   這是偷盜行為的一部分,\twnr{親愛的先生}{204.0}!你是盜香者。」

  「我沒拿走也沒破壞,我從遠處聞蓮,

   那麼以什麼理由,被稱為『盜香者』呢?

   凡這挖掘蓮藕者,破壞白蓮者,

   這樣殘酷的行為者,為何這位不被稱為?」

  「殘酷兇暴的男子,如弄髒的奶媽衣,

   對他沒有我的言語,但我值得說你。

   對無穢的男子,常尋求清淨者:

   \twnr{對毛端大小的惡來說}{x199},看起來像雲大小般。」

  「\twnr{夜叉}{126.0}!你確實知道我,而且你憐愍我,

   夜叉!願你也再說說:每當你看見像這樣時。」

  「我們既不依靠你生活,我們也不是你的傭人,

   比丘!你就應該知道,能走到\twnr{善趣}{112.0}之處。」

  那時,那位比丘被那位天神激起急迫感,來到了急迫感。

  林相應完成,其\twnr{攝頌}{35.0}:

  「遠離與起來,迦葉氏與眾多,

   阿難與阿那律,那格勒得與家婦

   毘舍離的跋耆子以及,誦讀、不如理,

   中午時、諸根不控制,與紅蓮花成為十四。」





\page

\xiangying{10}{夜叉相應}
\sutta{1}{1}{因陀羅迦經}{https://agama.buddhason.org/SN/sn.php?keyword=10.1}
  \twnr{被我這麼聽聞}{1.0}:

  \twnr{有一次}{2.0},\twnr{世尊}{12.0}住在王舍城因陀羅頂山,因陀羅迦\twnr{夜叉}{126.0}的領域。

  那時,因陀羅迦夜叉去見世尊。抵達後,以\twnr{偈頌}{281.0}對世尊說:

  「諸佛說:『色不是命』,這位如何得到(知道)這個身體?

   他的骨頭肝臟從哪裡來,這位如何\twnr{被黏著於洞穴(子宮)}{x200}?」

  「首先有\twnr{凝滑}{x201},從凝滑有\twnr{胞}{885.1},

   從胞生出\twnr{肉片}{x202},肉片生出\twnr{堅肉}{x203},

   從堅肉\twnr{諸肢節}{x204}被生起,\twnr{以及頭髮體毛及指甲}{x205}。

   而凡他母親吃的,食物飲料與飲食,

   以那個使他在那裡存續:進入母親子宮的人。」



\sutta{2}{2}{沙卡那摩經}{https://agama.buddhason.org/SN/sn.php?keyword=10.2}
  \twnr{有一次}{2.0},\twnr{世尊}{12.0}住在王舍城\twnr{耆闍崛山}{258.0}。

  那時,沙卡那摩迦\twnr{夜叉}{126.0}去見世尊。抵達後,以\twnr{偈頌}{281.0}對世尊說:

  「對一切繫縛已被捨斷的,成為已擺脫的你來說,

   那對\twnr{沙門}{29.0}不好:凡你教誡其他人。」

  「沙卡!凡無論以任何理由,親交被生起,

   有慧者不適合對他,以意(心)憐憫。

   如果以明淨意(心),凡他教誡其他人,

   他不因為那樣被結合(被束縛):凡憐愍、同情。」



\sutta{3}{3}{針毛經}{https://agama.buddhason.org/SN/sn.php?keyword=10.3}
  \twnr{有一次}{2.0},\twnr{世尊}{12.0}住在伽耶的登居得床\twnr{針毛夜叉}{x206}的領域。

  當時,柯樂\twnr{夜叉}{126.0}與針毛夜叉在世尊的不遠處走過。

  那時,柯樂夜叉對針毛夜叉說這個:「這是\twnr{沙門}{29.0}。」

  「這不是沙門,這是假沙門。或者直到我知道他是沙門又或他是假沙門為止。」

  那時,針毛夜叉去見世尊。抵達後,使身體斜向(靠近)世尊。

  那時,世尊使身體離開。

  那時,針毛夜叉對世尊說這個:「沙門!你怕我嗎?」

  「\twnr{朋友}{201.0}!我不怕你,但你的接觸是惡的。」

  「沙門!那麼,我將要問你問題,如果你不回答我,我將會使你的心混亂,或我將會使你的心臟破裂,或我將會抓住腳後拋到恒河彼岸。」 

  「朋友!我不見那個:在包括天,在包括魔,在包括梵的世間;在包括沙門婆羅門,在包括天-人的\twnr{世代}{38.0}中,凡能使我的心混亂,或能使我的心臟破裂,或能抓住腳後拋到恒河彼岸者,但,朋友!當你期待(疑惑)時,你問吧!」[\suttaref{SN.10.12}]

  那時,針毛夜叉以\twnr{偈頌}{281.0}對世尊說:

  「貪與瞋從什麼因緣?不喜樂、喜樂、\twnr{身毛豎立的}{152.0}是從哪裡生的?

   從什麼起來後有\twnr{意尋}{x207},\twnr{如男童們放烏鴉}{x208}?」

  「貪與瞋從這裡的因緣,不喜樂、喜樂、身毛豎立的是從這裡生的,

   從這裡起來後有意尋,如男童們放烏鴉。

   從情愛生的從自己生起的,\twnr{如從榕樹樹幹生的}{x209},

   在諸欲上糾纏的各種,\twnr{如葛藤在林中伸展的}{x210}。

   凡了知它從哪裡的因緣者,他們驅離它-你聽吧!夜叉!

   他們渡這難渡的\twnr{暴流}{369.0},以前未渡過的-為了不再有。」



\sutta{4}{4}{吉祥寶珠經}{https://agama.buddhason.org/SN/sn.php?keyword=10.4}
  \twnr{有一次}{2.0},\twnr{世尊}{12.0}住在摩揭陀的華鬘寶珠\twnr{塔廟}{366.0},吉祥寶珠\twnr{夜叉}{126.0}的領域。

  那時,吉祥寶珠夜叉去見世尊。抵達後,以\twnr{偈頌}{281.0}對世尊說:

  「有念者經常有吉祥:有念者增大安樂,

   有念者的\twnr{明天是更好的}{x211},且從敵意被釋放。」

  「有念者經常有吉祥:有念者增大安樂,

   有念者的明天是更好的,[但]沒從敵意被釋放。

   凡一切日夜,意(心)樂於不害者,

   在一切生類上有\twnr{慈分}{157.0},對他沒有任何敵意[\ccchref{AN.8.1}{https://agama.buddhason.org/AN/an.php?keyword=8.1}, \ccchref{It.27}{https://agama.buddhason.org/It/dm.php?keyword=27}]。」



\sutta{5}{5}{沙奴經}{https://agama.buddhason.org/SN/sn.php?keyword=10.5}
  \twnr{有一次}{2.0},\twnr{世尊}{12.0}住在舍衛城祇樹林給孤獨園。

  當時,某位\twnr{優婆夷}{99.0}名叫沙奴的兒子\twnr{被夜叉捉住}{x212}。

  那時,那位優婆夷在那時哭泣著說這些\twnr{偈頌}{281.0}:

  「在十四、\twnr{十五日裡}{222.0},與凡\twnr{半月的第八日}{737.0},

   \twnr{以及神變月}{738.0},\twnr{善具備八支}{467.0}。

   他們\twnr{入布薩}{246.0}:凡行梵行者,

   \twnr{夜叉}{126.0}們不以他們玩,像這樣被我在\twnr{阿羅漢}{5.0}們中聽聞,

   現在今日我自己看見,夜叉們以沙奴玩。」

  「十四、十五日,與凡半月的第八日,

   以及神變月,善具備八支。

   他們入布薩:凡行梵行者,

   夜叉們不以他們玩,被你在阿羅漢們中聽聞-\twnr{好}{44.0}!

   當沙奴醒來時請你告知,這夜叉的言語:

   不要作惡業,公開地或者私下地。

   而如果惡業,你將做或你做,

   沒有你從苦的解脫,即使飛起來後逃走。」

  「娘!他們對死者哭泣,或凡不被看見的生者,

   娘!他們看見活著的我,娘!你為何對我哭泣?」

  「兒!他們對死者哭泣,或凡不被看見的生者,

   但凡當捨棄諸欲後,再回來這裡者,

   兒!他們也對他哭泣,因為他又是已死的生者。

   親愛的!已從熱灰拉出,你想要跌落熱灰,

   親愛的!已從地獄拉出,你想要跌落地獄。

   請你們趕快跑-祝你幸福,\twnr{我們抱怨誰}{x213}?

   從已燃燒被取出的物品,你又想要被燃燒。」



\sutta{6}{6}{匹亞迦勒經}{https://agama.buddhason.org/SN/sn.php?keyword=10.6}
  \twnr{有一次}{2.0},\twnr{尊者}{200.0}阿那律住在舍衛城祇樹林給孤獨園。

  當時,尊者阿那律在破曉時起來後,[誦]說諸法句。

  那時,匹亞迦勒的母親女\twnr{夜叉}{126.0}這麼使幼子滿足(哄):

  「匹亞迦勒!不要作聲!\twnr{比丘}{31.0}誦說諸法句,

   進一步了知法句後,我們應該行動-那會是為了我們的利益。

   而我們要對生命類[自我]抑制,願我們不說故意的虛妄(妄語),

   我們應該學自己的善戒,或許我們能從惡鬼胎被釋放(脫離)。」



\sutta{7}{7}{富那婆蘇經}{https://agama.buddhason.org/SN/sn.php?keyword=10.7}
  \twnr{有一次}{2.0},\twnr{世尊}{12.0}住在舍衛城祇樹林給孤獨園。

  當時,世尊以涅槃關聯的法說對\twnr{比丘}{31.0}們開示、勸導、鼓勵、\twnr{使歡喜}{86.0},而那些比丘\twnr{作目標後}{316.0}、作意後、\twnr{全心注意後}{479.0}傾耳聽法。

  那時,富那婆蘇的母親女\twnr{夜叉}{126.0}這麼使幼子滿足(哄):

  「巫德里葛!安靜!富那婆蘇!安靜!

   在我將要聽\twnr{大師}{145.0}、最上佛陀之法的期間。

   世尊說涅槃:使一切繫縛的脫離,

   那對我來說是長時間的:在這個法上有喜愛。

   在世間中自己的兒子是可愛的,在世間中自己的丈夫是可愛的,

   對我來說比那個更可愛的,是那個法的探求。

   因為既非可愛的兒子也非丈夫,能使從苦脫離,

   如正法之聽聞,使有生命者從苦脫離。

   在被苦征服的世間中,在被老死結合中,

   為了老死的脫離,凡被\twnr{現正覺}{75.0}的法,

   那個法我想要聽聞,富那婆蘇!安靜!」

  「娘!我將不出聲,這巫德里葛[也]保持安靜,

   請你就傾聽法,正法的聽聞是樂的,

   不了知正法後,娘!令我們過苦生活(行苦)。

   這位是為天-人們,為癡昧者們帶來光明者,

   最後身的佛陀,\twnr{有眼者}{629.0}教導法。」

  「\twnr{好}{44.0}!確實是賢智者:躺在胸部所生的兒子,

   我的兒子喜愛,最上佛陀的純淨法。

   富那婆蘇!你要快樂,\twnr{因為今日我已出脫了}{x214},

   諸聖諦已被看見,巫德里葛!你也要聽我的[話]。」



\sutta{8}{8}{須達多經}{https://agama.buddhason.org/SN/sn.php?keyword=10.8}
  \twnr{有一次}{2.0},\twnr{世尊}{12.0}住在王舍城的\twnr{寒林}{199.0}。

  當時,\twnr{屋主}{103.0}給孤獨正以某些應該被作的已抵達王舍城。

  屋主給孤獨聽聞:「\twnr{佛陀}{3.0}確實已出現在世間。」而就立刻為了見世尊想要能前往。那時,那位屋主給孤獨想這個:「今日不是為了見世尊能前往的適當時機,現在,我將在明日的適當時機,為了見世尊而去。」

  他以念向佛躺臥,在夜間起來三次,想著:『已破曉。』

  那時,屋主給孤獨去往墓場的[城]門,\twnr{非人}{130.0}們打開門。

  那時,當屋主給孤獨從城市出去時,光明消失了,黑暗出現,他生起害怕、僵硬狀態、\twnr{身毛豎立的}{152.0},因此就想要再折返。

  那時,隱形的尸婆迦\twnr{夜叉}{126.0}說出聲音:

  「百象、百馬,百騾車,

   百千少女:寶石耳環裝飾的,

   比往前一步,不值得十六分之一。

   屋主!請你前進,屋主!請你前進,

   前進是對你比較好的,非返回。」

  那時,屋主給孤獨的黑暗消失了,光明出現,凡存在害怕、僵硬、身毛豎立者,他止息了。

  第二次,屋主給孤獨的光明消失了,黑暗出現,他生起害怕、僵硬、身毛豎立,因此就想要再折返。

  第二次,隱形的尸婆迦夜叉說出聲音:

  「百象、百馬……(中略),

   ……,不值得十六分之一。

   屋主!請你前進,屋主!請你前進,

   前進是對你比較好的,非返回。」

  那時,屋主給孤獨的黑暗消失了,光明出現,凡存在害怕、僵硬、身毛豎立者,他止息了。

  第三次,屋主給孤獨的光明消失了,黑暗出現,他生起害怕、僵硬、身毛豎立,因此就想要再折返。

  第三次,隱形的尸婆迦夜叉說出聲音:

  「百象、百馬……(中略),

   ……,不值得十六分之一。

   屋主!請你前進,屋主!請你前進,

   前進是對你比較好的,非返回。」

  那時,屋主給孤獨的黑暗消失了,光明出現,凡存在害怕、僵硬、身毛豎立者,他止息了。

  那時,屋主給孤獨去寒林去見世尊。

  當時,世尊在破曉時起來後,在\twnr{屋外}{385.0}\twnr{經行}{150.0}。

  世尊看見正從遠處到來的屋主給孤獨。看見後,從經行下來後,在設置的座位坐下。坐下後,世尊對屋主給孤獨說這個:「來!須達多!」

  那時,屋主給孤獨[心想]:「世尊以名字稱呼我。」大喜的、踊躍的,就在那裡\twnr{以頭落在世尊的腳上後}{40.0},對世尊說這個:

  「\twnr{大德}{45.0}!世尊是否\twnr{睡得安樂}{532.0}?」

  「確實總是睡得安樂:已般涅槃的婆羅門,

   凡在諸欲上不沾染,清涼、無\twnr{依著}{198.0}者。

   切斷一切執著後,調伏心中的不安後,

   得到心的寂靜後,寂靜者睡得安樂。」



\sutta{9}{9}{白淨經第一}{https://agama.buddhason.org/SN/sn.php?keyword=10.9}
  \twnr{有一次}{2.0},\twnr{世尊}{12.0}住在舍衛城栗鼠飼養處的竹林中。

  當時,白淨\twnr{比丘尼}{31.0}被大眾圍繞,教導法。

  那時,對白淨比丘尼有\twnr{極淨信}{340.1}的\twnr{夜叉}{126.0}到舍衛城後,從街道到街道;從十字路口到十字路口,在那時說這些\twnr{偈頌}{281.0}:

  「這些舍衛城中的人作了什麼?\twnr{像喝了蜜酒那樣睡著了}{x215},

   凡不侍奉白淨:\twnr{不死}{123.0}之境(句)的教導者。

   而且那是不能被抗拒的,美味的、富有營養的,

   有慧者們喝飲,看起來像如旅人對雲。」



\sutta{10}{10}{白淨經第二}{https://agama.buddhason.org/SN/sn.php?keyword=10.10}
  \twnr{有一次}{2.0},\twnr{世尊}{12.0}住在舍衛城栗鼠飼養處的竹林中。

  當時,某位\twnr{優婆塞}{98.0}施與白淨\twnr{比丘尼}{31.0}食物。

  那時,對白淨比丘尼有\twnr{極淨信的}{340.1}\twnr{夜叉}{126.0}到舍衛城後,從街道到街道;從十字路口到十字路口,在那時說這些偈頌:

  「他確實產出許多福德,這位優婆塞確實是有慧者,

   凡施與白淨食物者,他從一切繫縛被釋放。」



\sutta{11}{11}{基臘經}{https://agama.buddhason.org/SN/sn.php?keyword=10.11}
  \twnr{被我這麼聽聞}{1.0}:

  \twnr{有一次}{2.0},\twnr{世尊}{12.0}住在舍衛城栗鼠飼養處的竹林中。

  當時,某位\twnr{優婆塞}{98.0}施與基臘\twnr{比丘尼}{31.0}衣服。

  那時,對基臘比丘尼有\twnr{極淨信的}{340.1}\twnr{夜叉}{126.0}到舍衛城後,從街道到街道;從十字路口到十字路口,在那時說這些\twnr{偈頌}{281.0}:

  「他確實產出許多福德,這位優婆塞確實是有慧者,

   凡施與基臘衣服者,他從一切繫縛被釋放。」



\sutta{12}{12}{阿羅婆迦經}{https://agama.buddhason.org/SN/sn.php?keyword=10.12}
  \twnr{被我這麼聽聞}{1.0}:

  \twnr{有一次}{2.0},\twnr{世尊}{12.0}住在阿羅毘的阿羅婆迦\twnr{夜叉}{126.0}領域。

  那時,阿羅婆迦夜叉去見世尊。抵達後,對世尊說這個:「你出去!\twnr{沙門}{29.0}!」

  「好!\twnr{朋友}{201.0}!」世尊出去。

  「你進來!沙門!」

  「好!朋友!」世尊進來。

  第二次,阿羅婆迦夜叉又對世尊說這個:「你出去!沙門!」

  「好!朋友!」世尊出去。

  「你進來!沙門!」

  「好!朋友!」世尊進來。

  第三次,阿羅婆迦夜叉又對世尊說這個:「你出去!沙門!」

  「好!朋友!」世尊出去。

  「你進來!沙門!」

  「好!朋友!」世尊進來。

  第四次,阿羅婆迦夜叉又對世尊說這個:「出去!沙門!」

  「朋友!那我將不出去,凡應該被你做的(你想做的),你做吧!」

  「沙門!那麼,我將問你,如果你不回答我,我將會使你的心混亂,或我將會使你的心臟破裂,或我將會抓住腳後拋到恒河彼岸。」

  「朋友!我不見那個:在包括天,在包括魔,在包括梵的世間;在包括沙門婆羅門,在包括天-人的\twnr{世代}{38.0}中,凡能使我的心混亂,或能使我的心臟破裂,或能抓住腳後拋到恒河彼岸者,但,朋友!當你期待(疑惑)時,你問吧!」[\suttaref{SN.10.3}]

  「這裡什麼是男子的最上財產?什麼被善實行帶來樂?

   什麼確實是味道中最(更)美味的?如何生活是他們說最上活命者?」

  「這裡信是男子的最上財產,法被善實行帶來樂,

   真理確實是味道中最美味的,以慧生活是他們說最上活命者。」

  「如何渡\twnr{暴流}{369.0}?如何渡海洋?

   如何越過苦?如何淨化?」

  「以信渡暴流,以不放逸對海洋,

   以活力越過苦,以慧淨化。」

  「如何得到慧?如何找到財富?

   如何獲得(達成)名望?如何綁住朋友們?

   從此世到後世,如何死後不悲傷?」

  「信著\twnr{阿羅漢}{5.0},為了涅槃的到達之法,

   欲聽聞者得慧:不放逸地明察地。

   相稱的行為者,盡責奮起者找到財富,

   以真理獲得名望,給與者綁住朋友們。

   從此世到後世,這樣死後不悲傷,

   凡這四法,有信家主的:

   真理、\twnr{應該被調御的}{x216}、堅定、施捨,他死後確實不悲傷。

   來吧!請你也問其他,多數的沙門婆羅門,

   在這裡如果有比真理、應該被調御的、施捨,忍耐更多的被發現。」

  「現在我為何應該問,多數的沙門婆羅門呢?

   凡今日我了知的,凡來世利益者。

   佛陀確實為了我的利益,來到阿羅毘住所,

   凡今日我了知的,所施有大果之處。

   那我將遊走:從村落到村落從城市到城市,

   禮敬著\twnr{正覺者}{185.1},法的善法性。」

  夜叉相應完成,其\twnr{攝頌}{35.0}:

  「因陀羅迦、沙卡、針,吉祥寶珠與沙奴,

   匹亞迦勒、富那婆蘇、須達多,白淨二則、基臘、阿羅婆迦。」十二則。



  附註:菩提比丘長老概述《顯揚真義》的解說:阿羅毘地方的國王被阿羅婆迦夜叉抓住,要脅他每天要提供一個人給他吃,否則就要吃掉國王。起先,國王每天送一位罪犯去,罪犯送完了,送國境內的小孩。國境內有小孩的人家都逃光了,國王打算送他的王子去。就在前一天,佛陀去阿羅婆迦夜叉住的洞穴,但阿羅婆迦夜叉剛好外出到喜馬拉雅山參加聚會,於是,佛陀坐在夜叉的座位上,為夜叉的妻后說法。夜叉聽到了,很生氣地趕回來,於是開始本經的對話。起初,佛陀隨順夜叉的指令,想軟化他,但到第四次後,佛陀知道夜叉打算整夜戲弄,就拒絕他的要求。而夜叉所提問的四個問題,是其父母從迦葉佛那裡聽來而教他的,但現在夜叉只記得問題,已忘了答案,於是,就趁機拿來問佛陀。





\page

\xiangying{11}{帝釋相應}
\pin{第一品}{1}{10}
\sutta{1}{1}{蘇毘羅經}{https://agama.buddhason.org/SN/sn.php?keyword=11.1}
  \twnr{被我這麼聽聞}{1.0}:

  \twnr{有一次}{2.0},\twnr{世尊}{12.0}住在舍衛城祇樹林給孤獨園。

  在那裡,世尊召喚\twnr{比丘}{31.0}們:「比丘們!」

  「\twnr{尊師}{480.0}!」那些比丘回答世尊。

  世尊說這個:

  「比丘們!從前,阿修羅攻打天神。那時,\twnr{天帝釋}{263.0}召喚蘇毘羅\twnr{天子}{282.0}:『親愛的蘇毘羅!這些阿修羅攻打天神,親愛的蘇毘羅!你去迎戰阿修羅!』『是的,\twnr{你的[話]是吉祥的}{757.0}。』比丘們!那時,蘇毘羅天子回答天帝釋後,放逸地度過。

  比丘們!第二次,天帝釋又召喚蘇毘羅天子:『親愛的蘇毘羅!這些阿修羅攻打諸天,親愛的蘇毘羅!你去迎戰阿修羅!』『是的,你的[話]是吉祥的。』比丘們!那時,蘇毘羅天子回答天帝釋後,第二次又放逸地度過。

  比丘們!第三次,天帝釋又召喚蘇毘羅天子:『親愛的蘇毘羅!這些阿修羅攻打諸天,親愛的蘇毘羅!你去迎戰阿修羅!』『是的,你的[話]是吉祥的。』比丘們!那時,蘇毘羅天子回答天帝釋後,第三次又放逸地度過。

  比丘們!那時,天帝釋說以\twnr{偈頌}{281.0}對蘇毘羅天子說:

  『不奮起不精進者,到達安樂之處,

   蘇毘羅!請你去那裡,且帶我(使我)到那裡。』

  『會有懶惰不奮起者,而他不會作應該被作的,

   他會有一切欲的成功,帝釋!\twnr{請你指示我那個殊勝的}{x217}。』

  『不奮起不精進者於該處,他得到究竟的安樂,

   蘇毘羅!請你去那裡,且帶我到那裡。』

  『最上的天!以不工作(不作業),帝釋!我們會找到安樂,

   無憂愁、無\twnr{絕望}{342.0},帝釋!請你指示我那個殊勝的。』

  『如果以不工作有,任何人不管在哪裡都\twnr{不衰退}{x218},

   那確實是涅槃之道,蘇毘羅!請你去那裡,

   且帶我到那裡。』

  比丘們!如果那位確實依自己福果生活、執行三十三天諸天最高主權統治的天帝釋是奮起、活力的稱讚者,那麼,比丘們!這裡,你們應該\twnr{使它輝耀}{x219}:凡在這麼善說的法律中出家的你們,應該為了未獲得的之獲得、未達到的之達到、未作證的之作證奮起、努力、精進。」



\sutta{2}{2}{蘇尸摩經}{https://agama.buddhason.org/SN/sn.php?keyword=11.2}
  起源於舍衛城。

  在那裡,\twnr{世尊}{12.0}召喚\twnr{比丘}{31.0}們:「比丘們!」

  「\twnr{尊師}{480.0}!」那些比丘回答世尊。

  世尊說這個:

  「比丘們!從前,阿修羅攻打天神。那時,\twnr{天帝釋}{263.0}召喚蘇尸摩\twnr{天子}{282.0}:『親愛的蘇尸摩!這些阿修羅攻打諸天,親愛的蘇尸摩!你去迎戰阿修羅!』『是的,\twnr{你的[話]是吉祥的}{757.0}。』比丘們!那時,蘇尸摩天子回答天帝釋後,放逸地度過。

  比丘們!第二次,天帝釋又召喚蘇尸摩天子………(中略),第二次又放逸地度過。

  比丘們!第三次,天帝釋又召喚蘇尸摩天子………(中略),第三次又放逸地度過。

  比丘們!那時,天帝釋以\twnr{偈頌}{281.0}對蘇尸摩天子說:

  『不奮起不精進者,到達安樂之處,

   蘇尸摩!請你去那裡,且帶我(使我)到那裡。』

  『會有懶惰不奮起者,而他不會作應該被作的,

   他會有一切欲的成功,帝釋!請你指示我那個殊勝的。』

  『不奮起不精進者於該處,他得到究竟的安樂,

   蘇尸摩!請你去那裡,且帶我到那裡。』

  『最上的天!以不工作(不作業),帝釋!我們會找到安樂,

   無憂愁、無\twnr{絕望}{342.0},帝釋!請你指示我那個殊勝的。』

  『如果以不工作有,任何人不管在哪裡都不衰退,

   那確實是涅槃之道,蘇尸摩!請你去那裡,

   且帶我到那裡。』

  比丘們!如果那位確實依自己福果生活、執行三十三天諸天最高主權統治的天帝釋是奮起、活力的稱讚者,那麼,比丘們!這裡,你們應該使它輝耀:凡在這麼善說的法律中出家的你們,應該為了未獲得的之獲得、未達到的之達到、未作證的之作證奮起、努力、精進。」[\suttaref{SN.11.1}]



\sutta{3}{3}{旗幟頂端經}{https://agama.buddhason.org/SN/sn.php?keyword=11.3}
  在舍衛城。

  在那裡,\twnr{世尊}{12.0}召喚\twnr{比丘}{31.0}們:「比丘們!」「\twnr{尊師}{480.0}!」那些比丘回答世尊。

  世尊說這個:

  「比丘們!從前,天神、阿修羅的戰鬥已群集,比丘們!那時,天帝釋召喚三十三天中的天神:『\twnr{親愛的先生}{204.0}們!當天神們來到戰鬥時,如果生起害怕,或僵硬狀態,或\twnr{身毛豎立的}{152.0},那時,你們就應該仰視我的旗幟頂端。因為當你們仰視我的旗幟頂端時,凡會有害怕,或僵硬狀態,或身毛豎立的,它將被捨斷。

  如果你們不能仰視我的旗幟頂端,那時,你們應該仰視波闍波提天王的旗幟頂端。因為當你們仰視波闍波提天王的旗幟頂端時,凡會有害怕,或僵硬狀態,或身毛豎立的,它將被捨斷。

  如果你們不能仰視波闍波提天王的旗幟頂端,那時,你們應該仰視伐盧那天王的旗幟頂端。因為當你們仰視伐盧那天王的旗幟頂端時,凡會有害怕,或僵硬狀態,或身毛豎立的,它將被捨斷。

  如果你們不能視伐盧那天王的旗幟頂端,那時,你們應該仰視伊舍那天王的旗幟頂端。因為當你們仰視伊舍那天王的旗幟頂端時,凡會有害怕,或僵硬狀態,或身毛豎立的,它將被捨斷。』

  比丘們!然而,當仰視天帝釋的旗幟頂端時,或當波闍波提天王的旗幟頂端時,或當仰視伐盧那天王的旗幟頂端時,或當仰視伊舍那天王的旗幟頂端時,凡會有害怕,或僵硬狀態,或身毛豎立的,它或會被捨斷,或不會被捨斷。

  那是什麼原因?比丘們!因為天帝釋是未離貪者、未離瞋者、未離癡者、怯懦者、驚恐者、有恐懼者,他逃跑。」

  「比丘們!但我這麼說:『比丘們!當你們到\twnr{林野}{142.0},或到樹下,或到空屋時,如果生起害怕,或僵硬狀態,或身毛豎立的,那時,你們就應該隨念(回憶)我:「像這樣,那位世尊是\twnr{阿羅漢}{5.0}、\twnr{遍正覺者}{6.0}、\twnr{明行具足者}{7.0}、\twnr{善逝}{8.0}、\twnr{世間知者}{9.0}、\twnr{應該被調御人的無上調御者}{10.0}、\twnr{天-人們的大師}{11.0}、佛陀、世尊。」比丘們!因為當你們隨念我時,凡會有害怕,或僵硬狀態,或身毛豎立的,它將被捨斷。

  比丘們!如果你們不能隨念我,那時,你們應該隨念法:「被世尊善說的法是直接可見的、即時的、請你來看的、能引導的、\twnr{應該被智者各自經驗的}{395.0}。」比丘們!因為當你們隨念法時,凡會有害怕,或僵硬狀態,或身毛豎立的,它將被捨斷。

  比丘們!如果你們不能隨念法,那時,你們應該隨念\twnr{僧團}{375.0}:「世尊的弟子僧團是\twnr{善行者}{518.0},世尊的弟子僧團是正直行者,世尊的弟子僧團是真理行者,世尊的弟子僧團是\twnr{方正行者}{764.0},即:四雙之人、\twnr{八輩之士}{347.0},這世尊的弟子僧團應該被奉獻、應該被供奉、應該被供養、應該被\twnr{合掌}{377.0},為世間的無上\twnr{福田}{101.0}。」』比丘們!因為當你們隨念僧團時,凡會有害怕,或僵硬狀態,或身毛豎立的,它將被捨斷。那是什麼原因?比丘們!因為如來、阿羅漢、遍正覺者是已離貪者、已離瞋者、已離癡者、不怯懦者、不驚恐者、無恐懼者,他不逃跑。」

  世尊說這個,說這個後,善逝、\twnr{大師}{145.0}又更進一步說這個:

  「比丘們!在林野、樹下,或就在空屋中,

   你們應該隨念\twnr{正覺者}{185.1},不會有你們的恐怖。

   如果你們不能隨念佛陀:世間最勝者、人中之牛王,

   那時你們應該隨念法:\twnr{出離的}{294.0}、善教導的。

   如果你們不能隨念法:出離的、善教導的,

   那時你們應該隨念僧團:無上福田。

   比丘們!這樣隨念佛陀,法與僧團者,

   害怕或僵硬狀態,身毛豎立的那將不存在。」



\sutta{4}{4}{毘摩質多經}{https://agama.buddhason.org/SN/sn.php?keyword=11.4}
  起源於舍衛城。

  「\twnr{比丘}{31.0}們!從前,天神、阿修羅的戰鬥已群集。

  比丘們!那時,毘摩質多阿修羅王召喚阿修羅們:『\twnr{親愛的先生}{204.0}們!在阿修羅對天神的戰鬥已群集中,如果阿修羅們勝,天神們敗,這樣,\twnr{以脖子為第五繫縛}{x220},繫縛那位\twnr{天帝釋}{263.0}後,請你們帶來阿修羅城我的面前。』

  比丘們!那時,天帝釋也召喚三十三天天神:『親愛的先生們!在阿修羅對天神的戰鬥已群集中,如果天神們勝,阿修羅們敗,這樣,以脖子為第五繫縛,繫縛那位毘摩質多阿修羅王後,請你們帶來善法堂我的面前。』

  比丘們!又,在那場戰鬥中,天神們勝,阿修羅們敗。

  比丘們!那時,三十三天天神以脖子為第五繫縛,繫縛毘摩質多阿修羅王後,帶來天神善法堂天帝釋的面前。

  比丘們!在那裡,當天帝釋進出善法堂時,以脖子為第五繫縛的毘摩質多阿修羅王以粗俗、粗惡的言語辱罵、誹謗。

  比丘們!那時,戰車御車手摩得利以\twnr{偈頌}{281.0}對天帝釋說:

  『\twnr{摩伽婆}{263.0}、帝釋!以害怕,以弱你忍耐嗎?

   在毘摩質多的面前,當聽聞粗惡的言語時?』

  『我非由於害怕非由於弱,忍耐毘摩質多,

   像我這樣的有智者為何,會與愚者糾葛(連接)?』

  『愚者們會更加地爆發,如果沒有禁止者,

   因此明智者應該,以強力的懲罰阻止愚者。』

  『我認為這就是,對愚者的阻止:

   知道對方已被激怒後,凡具念地平靜下來。』

  『\twnr{襪瑟哇}{263.0}!我看見,這就是忍耐的過失:

   當愚者認為他,由於害怕我-這位忍耐,

   劣智慧者更爬上來,如牛對逃跑者。』

  『任他認為或不[認為]:由於害怕我-這位忍耐,

   在自己最高利益的中,比忍耐更高的還沒被發現。

   凡當確實是有力者時,他忍耐弱者,

   他們說那是最高的忍耐,\twnr{弱者總是忍耐}{x221}。

   他們說那個力量並非力量:凡屬於愚者力量的力量,

   對被法守護的力量,反對者沒被發現。

   因為那樣對他只是更糟的:凡對生氣者生氣回去,

   不對生氣者生氣回去者,他打勝難勝利的戰鬥。

   行兩者的利益:自己的與對方的,

   知道對方已被激怒後,凡具念地平靜下來者。

   當治療兩者時:自己的與對方的,

   人們思量『他是愚者』:凡法的不熟知者。[\suttaref{SN.7.2}]」

  比丘們!如果那位確實依自己福果生活、執行三十三天諸天最高主權統治的天帝釋是忍耐、柔和的稱讚者,那麼,比丘們!這裡,你們應該使它輝耀[\suttaref{SN.11.1}]:凡在這麼善說的法律中出家的你們,應該有忍耐與柔和。」



\sutta{5}{5}{善說勝利者經}{https://agama.buddhason.org/SN/sn.php?keyword=11.5}
  起源於舍衛城。

  「\twnr{比丘}{31.0}們!從前,天神、阿修羅的戰鬥已群集。

  比丘們!那時,毘摩質多阿修羅王對天帝釋說這個:『天帝!令以善說為勝利者!』

  『毘摩質多!令以善說為勝利者!』

  比丘們!那時,天神與阿修羅們使評審團設立:『我們的這些人將了知善說的或惡說的。』

  比丘們!那時,毘摩質多阿修羅王對天帝釋說這個:『天帝!請你說\twnr{偈頌}{281.0}。』

  比丘們!在這麼說時,天帝釋對毘摩質多阿修羅王說這個:『毘摩質多!在這裡,你們是\twnr{以前的天神}{x222},請你說偈頌。』

  比丘們!在這麼說時,毘摩質多阿修羅王說這偈頌:

  『愚者們會更加地爆發,如果沒有禁止者,

   因此明智者應該,以強力的懲罰阻止愚者。』

  比丘們!又,在偈頌被毘摩質多阿修羅王說時,阿修羅們\twnr{隨喜}{85.0},天神們保持沈默。

  比丘們!那時,毘摩質多阿修羅王對天帝釋說這個:『天帝!請你說偈頌。』

  比丘們!在這麼說時,天帝釋說這偈頌:

  『我認為這就是,對愚者的阻止:

   知道對方已被激怒後,凡具念地平靜下來。』

  比丘們!又,在偈頌被天帝釋說時,天神們隨喜,阿修羅們保持沈默。

  比丘們!在這麼說時,天帝釋對毘摩質多阿修羅王說這個:『毘摩質多!請你說偈頌。』

  比丘們!在這麼說時,毘摩質多阿修羅王說這偈頌:

  『\twnr{襪瑟哇}{263.0}!我看見,這就是忍耐的過失:

   當愚者認為他,由於害怕我-這位忍耐,

   劣智慧者更爬上來,如牛對逃跑者。』

  比丘們!又,在偈頌被毘摩質多阿修羅王說時,阿修羅們隨喜,天神們保持沈默。

  比丘們!那時,毘摩質多阿修羅王對天帝釋說這個:『天帝!請你說偈頌。』

  比丘們!在這麼說時,天帝釋說這偈頌:

  『任他認為或不[認為]:由於害怕我-這位忍耐,

   在自己最高利益的中,比忍耐更高的還沒被發現。

   凡當確實是有力者時,他忍耐弱者,

   他們說那是最高的忍耐,弱者總是忍耐。

   他們說那個力量並非力量:凡屬於愚者力量的力量,

   對被法守護的力量,反對者沒被發現。

   因為那樣對他只是更糟的:凡對生氣者生氣回去,

   不對生氣者生氣回去者,他打勝難勝利的戰鬥。

   行兩者的利益:自己的與對方的,

   知道對方已被激怒後,凡具念地平靜下來者。

   當治療兩者時:自己的與對方的,

   人們思量「他是愚者」:凡法的不熟知者。』[\suttaref{SN.10.4}]

  比丘們!又,在偈頌被天帝釋說時,天神們隨喜,阿修羅們保持沈默。

  比丘們!那時,天神與阿修羅們的評審團說這個:『偈頌已被毘摩質多阿修羅王說,而那是有棍棒的行境、有刀劍的行境,像這樣有爭吵;像這樣有爭執;像這樣有紛爭。而被天帝釋說的偈頌是無棍棒的行境、無刀劍的行境,像這樣無爭吵;像這樣無爭執;像這樣無紛爭。天帝釋以善說有勝利。』

  比丘們!像這樣,天帝釋以善說有勝利。」



\sutta{6}{6}{鳥巢經}{https://agama.buddhason.org/SN/sn.php?keyword=11.6}
  起源於舍衛城。

  「\twnr{比丘}{31.0}們!從前,天神、阿修羅的戰鬥已群集。比丘們!又,在那場戰鬥中,阿修羅勝,天神敗。比丘們!而敗北的天神就以北邊口退卻,阿修羅正攻打(追擊)他們。

  比丘們!那時,天帝釋以\twnr{偈頌}{281.0}對戰車御車手摩得利說:

  『摩得利!在絹綿樹上有鳥巢,請你\twnr{以轅桿的前端}{x223}避開,

   寧願我們捨生命在阿修羅上,也不要這些鳥成為無巢者。』

  『是的,\twnr{你的[話]是吉祥的}{757.0}。』

  比丘們!戰車御車手摩得利回答天帝釋後,回轉連結千匹駿馬的馬車。

  比丘們!那時,阿修羅們想這個:『現在,天帝釋連結千匹駿馬的馬車已回轉,天神與阿修羅又將戰第二次。』以被驚嚇,他們就驚恐地進入阿修羅城。

  比丘們!像這樣,天帝釋以法成為勝利者。」



\sutta{7}{7}{不傷害經}{https://agama.buddhason.org/SN/sn.php?keyword=11.7}
  在舍衛城。

  「\twnr{比丘}{31.0}們!從前,當天帝釋獨處、\twnr{獨坐}{92.0}時,這樣心的深思生起:『凡即使他會是我的極敵對者,我也不應該傷害他。』

  比丘們!那時,毘摩質多阿修羅王以心了知天帝釋心中的深思後,去見天帝釋。

  比丘們!那時,天帝釋看見正從遠處到來的毘摩質多阿修羅王。看見後,對毘摩質多阿修羅王說這個:

  「站住!毘摩質多!你被捉住。」

  「\twnr{親愛的先生}{204.0}!你不要捨棄,那個你先前的心。」

  「毘摩質多!但請你發誓,以無害意對我。」

  「凡虛妄地說者的惡,凡斥責聖者的惡,

   以及凡背叛朋友的惡,凡不感恩者的惡,

   \twnr{須闍之夫}{263.0}!就令那種惡觸達,凡如果傷害你。」



\sutta{8}{8}{毘盧遮那阿修羅王經}{https://agama.buddhason.org/SN/sn.php?keyword=11.8}
  在舍衛城祇樹林。

  當時,\twnr{世尊}{12.0}已進入\twnr{白天的住處}{128.0},已\twnr{獨坐}{92.0}。

  那時,天帝釋與毘盧遮那阿修羅王去見世尊。抵達後,各依門兩側站立。

  那時,毘盧遮那阿修羅王在世尊的面前說這\twnr{偈頌}{281.0}:

  「男子正應該努力,直到成為目標(利益)完成者,

   \twnr{已完成的目標是閃亮的}{x224},這是毘盧遮那之語。」

  「男子正應該努力,直到成為目標完成者,

   已完成的目標是閃亮的,比忍耐更高的還沒被發現。」

  「一切眾生有生起的目標:處處依合適的,

   然而一切生物的受用,\twnr{結合是最高的}{x225},

   已完成的目標是閃亮的,這是毘盧遮那之語。」

  「一切眾生有生起的目標:處處依合適的,

   然而一切生物的受用,結合是最高的,

   已完成的目標是閃亮的,比忍耐更高的還沒被發現。」



\sutta{9}{9}{林野處仙人經}{https://agama.buddhason.org/SN/sn.php?keyword=11.9}
  在舍衛城。

  「\twnr{比丘}{31.0}們!從前,眾多持戒的善法(良好行為)\twnr{仙人}{x226}住在\twnr{林野}{142.0}處的葉屋。

  比丘們!那時,\twnr{天帝釋}{263.0}與毘摩質多阿修羅王去見那些持戒的善法仙人。

  比丘們!那時,毘摩質多阿修羅王穿多層襯裡的鞋後,配劍後,以持著傘,經由屋門進入草屋後,對那些持戒的善法仙人\twnr{作左繞}{47.0}後走過。

  比丘們!那時,天帝釋脫下多層襯裡的鞋後,劍給其他人後,收傘後,只經由門進入草屋後,站在下風處\twnr{合掌}{377.0}禮敬著那些持戒的善法仙人。

  比丘們!那時,那些持戒的善法仙人以\twnr{偈頌}{281.0}對天帝釋說:

  『長久出家仙人們的味道,被從身體散發隨風行,

   \twnr{千眼}{750.1}!請你從這裡退回,天帝!仙人們的味道是不淨的。』

  『長久出家仙人們的味道,被從身體散發隨風行,

   如在頭上極多樣花的花環,

   \twnr{大德}{45.0}!我們期待這個味道,在這裡天神確實沒有厭逆想。』」



\sutta{10}{10}{海經}{https://agama.buddhason.org/SN/sn.php?keyword=11.10}
  在舍衛城。

  「\twnr{比丘}{31.0}們!從前,眾多持戒的善法(良好行為)\twnr{仙人}{x226}住在海邊的葉屋。

  當時,天神、阿修羅的戰鬥已群集。

  比丘們!那時,那些持戒的善法仙人想這個:『天神是如法的,阿修羅是非法的,從阿修羅也許會有我們害怕的,讓我們去見三婆羅阿修羅王,讓我們乞求\twnr{無恐怖之施(無畏施)}{x227}。』

  比丘們!那時,那些持戒的善法仙人就猶如有力氣的男子伸直彎曲的手臂,或彎曲伸直的手臂,就像這樣在海邊葉屋消失,出現在三婆羅阿修羅王的面前。

  比丘們!那時,那些持戒的善法仙人以\twnr{偈頌}{281.0}對三婆羅阿修羅王說:

  『到達三婆羅的仙人們,乞求無恐怖之施,

   因為你是隨心所欲的施與者:恐怖或無恐怖的。』

  『沒有仙人們的無恐怖,邪惡的帝釋親近者,

   對無恐怖的乞求者,我就施與你們恐怖。』

  『對無恐怖的乞求者,你就施與我們恐怖,

   我們領受你的這個,令有你不消逝的恐怖。

   播什麼樣的種子,收像那樣的果實,

   作善者有善的,而作惡者有惡的,

   親愛的!種子被你播種,你必將承受果實。』

  比丘們!那時,那些持戒的善法仙人詛咒三婆羅阿修羅王後,就猶如有力氣的男子伸直彎曲的手臂,或彎曲伸直的手臂,就像這樣在三婆羅阿修羅王的面前消失,出現在海邊的葉屋。

  比丘們!那時,被那些持戒的善法仙人詛咒的三婆羅阿修羅王,在夜間驚嚇三次。」

  第一品,其\twnr{攝頌}{35.0}:

  「蘇毘羅、蘇尸摩,旗幟頂端、毘摩質多,

   善說勝利者,鳥巢、不傷害,

   毘盧遮那阿修羅王,林野處仙人

   以及海的仙人。」





\pin{第二品}{11}{20}
\sutta{11}{11}{誓言經}{https://agama.buddhason.org/SN/sn.php?keyword=11.11}
  在舍衛城。

  「\twnr{比丘}{31.0}們!從前,當天帝釋為人時,曾有七個誓言的受持、達成,以那些的受持情況,帝釋達到帝釋的地位。哪七個誓言?『願我終生是扶養父母者,願我終生是尊敬家中年長的者,願我終生是柔和語者,願我終生是離\twnr{離間語}{234.0}者,願我終生以離慳垢之心住於在家,是\twnr{自由施捨者}{348.0}、親手施與者、樂於捨者、回應乞求者、\twnr{樂於布施物均分者}{349.0},願我終生是真實語者,願我終生是不憤怒者-即使如果我的憤怒生起,我會急速地排除它。』比丘們!從前,當天帝釋為人時,曾有這七個誓言的受持、達成,以那些的受持情況,帝釋達到帝釋的地位。」

  「人-扶養父母者,尊敬家中年長的者,

   柔和、親切話語者,捨斷離間語者。

   在慳吝的調伏上努力者,真實者、克服憤怒的人,

   三十三天天神們說:『他確實是善人。』」



\sutta{12}{12}{帝釋的名字經}{https://agama.buddhason.org/SN/sn.php?keyword=11.12}
  在舍衛城祇樹林。

  在那裡,\twnr{世尊}{12.0}對\twnr{比丘}{31.0}們說這個:

  「比丘們!從前,當天帝釋為人時,他是名叫摩伽的學生婆羅門,因此被稱為『摩伽婆』。

  比丘們!從前,當天帝釋為人時,他\twnr{在城市施與布施}{x228},因此被稱為『\twnr{城市施與者}{x229}』。

  比丘們!從前,當天帝釋為人時,他恭敬地施與布施,因此被稱為『釋[恭敬的縮短字音譯]』。

  比丘們!從前,當天帝釋為人時,他施與住處,因此被稱為『\twnr{襪瑟哇}{x230}』。

  比丘們!天帝釋以片刻即思惟一千個道理,因此被稱為『\twnr{千眼}{750.1}』。

  比丘們!天帝釋的妻子是名叫須闍的阿修羅女孩,因此被稱為『\twnr{須闍之夫}{x231}』。

  比丘們!天帝釋對三十三天執行最高主權統治,因此被稱為『天帝釋』。

  比丘們!從前,當天帝釋為人時,曾有七個誓言的受持、達成,以那些的受持情況,帝釋達到帝釋的地位。哪七個誓言?『願我終生是扶養父母者,願我終生是尊敬家中年長的者,願我終生是柔和語者,願我終生是離\twnr{離間語}{234.0}者,願我終生以離慳垢之心住於在家,是\twnr{自由施捨者}{348.0}、親手施與者、樂於捨者、回應乞求者、\twnr{樂於布施物均分者}{349.0},願我終生是真實語者,願我終生是不憤怒者-即使如果我的憤怒生起,我會急速地排除它。』比丘們!從前,當天帝釋為人時,曾有這七個誓言的受持、達成,以那些的受持情況,帝釋達到帝釋的地位。」

  「人-扶養父母者,尊敬家中年長的者,

   柔和、親切話語者,捨斷離間語者。

   在慳吝的調伏上努力者,真實者、克服憤怒的人,

   三十三天天神們說:『他確實是善人。』」[\suttaref{SN.11.11}]



\sutta{13}{13}{摩訶里經}{https://agama.buddhason.org/SN/sn.php?keyword=11.13}
  \twnr{被我這麼聽聞}{1.0}:

  \twnr{有一次}{2.0},\twnr{世尊}{12.0}住在毘舍離大林重閣講堂。

  那時,離車族人摩訶里去見世尊。抵達後,向世尊\twnr{問訊}{46.0}後,在一旁坐下。在一旁坐下的離車族人摩訶里對世尊說這個:

  「\twnr{大德}{45.0}!天帝釋被世尊看見?」

  「摩訶里!天帝釋被我看見。」

  「大德!確實,那當然必定是看似帝釋者,大德!因為天帝釋是難見的。」

  「摩訶里!我了知帝釋,並且我了知帝釋所作諸法,以那些的受持情況,帝釋達到帝釋的地位。

  摩訶里!從前,當天帝釋為人時,他是名叫摩伽的學生婆羅門,因此被稱為『摩伽婆』。

  摩訶里!從前,當天帝釋為人時,他\twnr{在城市施與布施}{x228},因此被稱為『\twnr{城市施與者}{x229}』。

  摩訶里!從前,當天帝釋為人時,他恭敬地施與布施,因此被稱為『釋[恭敬的縮短字音譯]』。

  摩訶里!從前,當天帝釋為人時,他施與住處,因此被稱為『\twnr{襪瑟哇}{x230}』。

  摩訶里!天帝釋以片刻即思惟一千個道理,因此被稱為『\twnr{千眼}{750.1}』。

  摩訶里!天帝釋的妻子是名叫須闍的阿修羅女孩,因此被稱為『\twnr{須闍之夫}{x231}』。

  摩訶里!天帝釋對三十三天執行最高主權統治,因此被稱為『天帝釋』。

  摩訶里!從前,當天帝釋為人時,曾有七個誓言的受持、達成,以那些的受持情況,帝釋達到帝釋的地位。哪七個誓言?『願我終生是扶養父母者,願我終生是尊敬家中年長的者,願我終生是柔和語者,願我終生是離\twnr{離間語}{234.0}者,願我終生以離慳垢之心住於在家,是\twnr{自由施捨者}{348.0}、親手施與者、樂於捨者、回應乞求者、\twnr{樂於布施物均分者}{349.0},願我終生是真實語者,願我終生是不憤怒者-即使如果我的憤怒生起,我會急速地排除它。』摩訶里!從前,當天帝釋為人時,曾有這七個誓言的受持、達成,以那些的受持情況,帝釋達到帝釋的地位。」

  「人-扶養父母者,尊敬家中年長的者,

   柔和、親切話語者,捨斷離間語者。

   在慳吝的調伏上努力者,真實者、克服憤怒的人,

   三十三天天神們說:『他確實是善人。』」[\suttaref{SN.11.12}]



\sutta{14}{14}{貧窮經}{https://agama.buddhason.org/SN/sn.php?keyword=11.14}
  \twnr{有一次}{2.0},\twnr{世尊}{12.0}住在王舍城栗鼠飼養處的竹林中。

  在那裡,世尊召喚\twnr{比丘}{31.0}們:「比丘們!」

  「\twnr{尊師}{480.0}!」那些比丘回答世尊。

  世尊說這個:

  「比丘們!從前,就在這王舍城中,某位男子是貧窮人、窮困人、貧困人,他在如來宣說的法律中受持信,受持戒,受持所聞,受持施捨,受持慧。他在如來宣說的法律中受持信後,受持戒後,受持所聞後,受持施捨後,受持慧後,以身體的崩解,死後往生\twnr{善趣}{112.0}、天界,三十三天們的共住狀態,他以容色與名聲比其他天神們更輝耀。

  比丘們!在那裡,三十三天天神們譏嫌、不滿、責難:『實在\twnr{不可思議}{206.0}啊,\twnr{先生}{202.0}!實在\twnr{未曾有}{206.0}啊,先生!當這位\twnr{天子}{282.0}從前為人時,他是貧窮人、窮困人、貧困人,他以身體的崩解,死後往生善趣、天界,三十三天們的共住狀態,他以容色與名聲比其他天神們更輝耀。』

  比丘們!那時,天帝釋召喚三十三天天神們:『\twnr{親愛的先生}{204.0}們!你們不要譏嫌這位天子,親愛的先生們!當這位天子從前為人時,他在如來宣說的法律中受持信,受持戒,受持所聞,受持施捨,受持慧。他在如來宣說的法律中受持信後,受持戒後,受持所聞後,受持施捨後,受持慧後,以身體的崩解,死後往生善趣、天界,三十三天們的共住狀態,他以容色與有名聲使其他諸天失色。』

  比丘們!那時,安撫三十三天天神們的天帝釋在那時說這些\twnr{偈頌}{281.0}:

  『該者在如來上有信:不動的、善住立的,

   以及該者的戒是善的:聖者喜愛的、讚賞的。

   該者在僧團上有\twnr{淨信}{507.0},且見成為正直者,

   他們說他是「不貧窮者」,他的活命是不空的。

   因此對信與戒,對淨信、法的看見,

   有智慧者應該實踐,憶念著諸佛的教說。[\ccchref{AN.4.52}{https://agama.buddhason.org/AN/an.php?keyword=4.52}, \ccchref{AN.5.47}{https://agama.buddhason.org/AN/an.php?keyword=5.47}]』」



\sutta{15}{15}{能令人愉悅經}{https://agama.buddhason.org/SN/sn.php?keyword=11.15}
  在舍衛城祇樹[給孤獨園]。

  那時,\twnr{天帝釋}{263.0}去見世尊。抵達後,向世尊\twnr{問訊}{46.0}後,在一旁站立。在一旁站立的天帝釋對世尊說這個:

  「\twnr{大德}{45.0}!什麼是能令人愉悅的地方呢?」

  「園林\twnr{塔廟}{366.0}、森林塔廟,善創造的蓮花池,

   比能令人愉悅的,不值得十六分之一。

   不論在村落或\twnr{林野}{142.0},不論在低地或高地,

   \twnr{阿羅漢}{5.0}們居住之處,那是能令人愉悅的地方。」



\sutta{16}{16}{供奉者經}{https://agama.buddhason.org/SN/sn.php?keyword=11.16}
  \twnr{有一次}{2.0},\twnr{世尊}{12.0}住在王舍城\twnr{耆闍崛山}{258.0}。

  那時,天帝釋去見世尊。抵達後,向世尊\twnr{問訊}{46.0}後,在一旁站立。在一旁站立的天帝釋以\twnr{偈頌}{281.0}對世尊說:

  「對供奉的人們來說,對期待福德的有生命者,

   \twnr{對作有依著的福德者來說}{881.0},所施於何處有大果?」

  「四種行道者,\twnr{與四種在果位上已住立者}{746.0},

   這是已成為正直的\twnr{僧團}{375.0}:慧戒等持者。

   對供奉的人們來說,對期待福德的有生命者,

   對作有依著的福德者,所施於僧團有大果。」[\ccchref{AN.8.59}{https://agama.buddhason.org/AN/an.php?keyword=8.59}]



\sutta{17}{17}{佛之禮敬經}{https://agama.buddhason.org/SN/sn.php?keyword=11.17}
  在舍衛城祇樹[給孤獨園]。

  當時,\twnr{世尊}{12.0}已進入\twnr{白天的住處}{128.0},已\twnr{獨坐}{92.0}。

  那時,\twnr{天帝釋}{263.0}與\twnr{梵王娑婆主}{215.0}去見世尊。抵達後,抵達後,各依門兩側站立。

  那時,天帝釋在世尊的面前說這\twnr{偈頌}{281.0}:

  「英雄!戰場上的勝利者!請你起來,已卸下重擔者!無負債者!請你在世間走動,

   你的心已\twnr{善解脫}{28.0},如在十五日夜晚的月亮。」

  「天帝釋!如來不應該這麼被禮敬,天帝釋!如來應該這麼被禮敬:

  『英雄!戰場上的勝利者!請你起來,商隊領袖!無負債者!請你在世間走動,

   世尊請你教導法,將會有了知者。』[\ccchref{DN.14}{https://agama.buddhason.org/DN/dm.php?keyword=14}, 70段]」



\sutta{18}{18}{在家者之禮敬經}{https://agama.buddhason.org/SN/sn.php?keyword=11.18}
  在舍衛城。

  在那裡……(中略)說這個:

  「\twnr{比丘}{31.0}們!從前,天帝釋召喚戰車御車手摩得利:『親愛的摩得利!請你準備連結千匹駿馬的馬車,我們去遊樂園看美景。』

  『是的,\twnr{你的[話]是吉祥的}{757.0}。』 

  比丘們!戰車御車手摩得利回答天帝釋後,準備連結千匹駿馬的馬車後,回報天帝釋:『\twnr{親愛的先生}{204.0}!那些連結千匹駿馬的馬車已準備,現在是那個\twnr{你考量的時間}{84.0}。』

  比丘們!那時,當天帝釋從最勝殿高樓下來時,作合掌後,禮敬諸方。

  比丘們!那時,戰車御車手摩得利以\twnr{偈頌}{281.0}對天帝釋說:

  『他們禮敬你:\twnr{三明}{133.0}者與一切地上的\twnr{剎帝利}{116.0},

   以及四大王[天],與有名聲的三十三天,

   帝釋!那麼凡你禮敬者,那位\twnr{夜叉}{126.0}又是誰?』

  『他們禮敬我,三明者與一切地上的剎帝利,

   以及四大王[天],與有名聲的三十三天。

   但我應該禮敬,戒具足者、長久得定者,

   正確出家者、梵行為所趣處者。

   凡作福德的在家者、持戒的\twnr{優婆塞}{98.0},

   他們依法扶養妻子,摩得利!我禮敬他們。』

  『確實是世間中最上者:帝釋!凡你禮敬者,

   我也禮敬他們:\twnr{襪瑟哇}{263.0}!凡你禮敬者。』

  說這個後,摩伽婆、天王、須闍之夫,

  禮敬諸方後,登上首席馬車。」



\sutta{19}{19}{大師之禮敬經}{https://agama.buddhason.org/SN/sn.php?keyword=11.19}
  在舍衛城祇樹林。

  「\twnr{比丘}{31.0}們!從前,天帝釋召喚戰車御車手摩得利:『親愛的摩得利!請你準備連結千匹駿馬的馬車,我們去遊樂園看美景。』

  『是的,\twnr{你的[話]是吉祥的}{757.0}。』 

  比丘們!戰車御車手摩得利回答天帝釋後,準備連結千匹駿馬的馬車後,回報天帝釋:『\twnr{親愛的先生}{204.0}!那些連結千匹駿馬的馬車已準備,現在是那個\twnr{你考量的時間}{84.0}。』

  比丘們!那時,當天帝釋從最勝殿高樓下來時,作合掌後,禮敬\twnr{世尊}{12.0}。

  比丘們!那時,戰車御車手摩得利以\twnr{偈頌}{281.0}對天帝釋說:

  『\twnr{襪瑟哇}{263.0}!凡天與人,他們禮敬你,

   帝釋!那麼凡你禮敬者,那位\twnr{夜叉}{126.0}又是誰?』

  『這裡凡遍正覺者:在這個包括天的世間中,

   最高名字的\twnr{大師}{145.0},摩得利!我禮敬他。

   凡使其貪與瞋,以及\twnr{無明}{207.0}被離脫者,

   諸漏已滅盡的\twnr{阿羅漢}{5.0}們,摩得利!我禮敬他們。

   凡為了貪瞋的調伏,為了無明的超越,

   樂於拆解的\twnr{有學}{193.0},他們不放逸地\twnr{隨學}{398.0},

   摩得利!我禮敬他們。』

  『確實是世間中最上者:帝釋!凡你禮敬者,

   我也禮敬他們:\twnr{襪瑟哇}{263.0}!凡你禮敬者。』

  說這個後,摩伽婆、天王、須闍之夫,

  禮敬世尊後,登上首席馬車。」



\sutta{20}{20}{僧團之禮敬經}{https://agama.buddhason.org/SN/sn.php?keyword=11.20}
  在舍衛城祇樹林。

  在那裡……(中略)說這個:

  「\twnr{比丘}{31.0}們!從前,天帝釋召喚戰車御車手摩得利:『親愛的摩得利!請你準備連結千匹駿馬的馬車,我們去遊樂園看美景。』

  『是的,\twnr{你的[話]是吉祥的}{757.0}。』 

  比丘們!戰車御車手摩得利回答天帝釋後,準備連結千匹駿馬的馬車後,回報天帝釋:『\twnr{親愛的先生}{204.0}!那些連結千匹駿馬的馬車已準備,現在是那個\twnr{你考量的時間}{84.0}。』

  比丘們!那時,當天帝釋從最勝殿高樓下來時,作合掌後,禮敬比丘\twnr{僧團}{375.0}。

  比丘們!那時,戰車御車手摩得利以\twnr{偈頌}{281.0}對天帝釋說:

  『這些應該禮敬你:躺臥在腐爛身體的人類,

   這些陷在屍體中者,具備飢渴者。

   \twnr{襪瑟哇}{263.0}!你為何羨慕,那些無家者呢?

   請你說仙人們的正行,我們要聽你的那個話語。』

  『摩得利!我羨慕,那些無家者的這個:

   從任何村落出發,他們無關注(掛慮)地走去。

   不在儲藏室放置他們的,不在瓶子不在籃子,

   尋求著他人完成的,善行者們以那個維持生活,

   以善忠告忠告的明智者們,是保持沈默者寂靜行者。

   摩得利!天神被阿修羅懷敵意,以及個個\twnr{不免一死的人}{600.0}[互相],

   在懷敵意者中的不懷敵意者,在拿棍棒者中的冷卻(寂滅)者,

   在有取著者中的無取著者:摩得利!我禮敬他們。』

  『確實是世間中最上者:帝釋!凡你禮敬者,

   我也禮敬他們:\twnr{襪瑟哇}{263.0}!凡你禮敬者。』

  說這個後,摩伽婆、天王、須闍之夫,

  禮敬比丘僧團後,登上首席馬車。」

  第二品,其\twnr{攝頌}{35.0}:

  「天神連三說,貧窮,能令人愉悅的,

   供奉者,禮敬,帝釋的禮敬三則。」





\pin{第三品}{21}{25}
\sutta{21}{21}{切斷後經}{https://agama.buddhason.org/SN/sn.php?keyword=11.21}
  在舍衛城祇樹林。

  那時,\twnr{天帝釋}{263.0}去見\twnr{世尊}{12.0}。抵達後,向世尊\twnr{問訊}{46.0}後,在一旁站立。在一旁站立的天帝釋以\twnr{偈頌}{281.0}對世尊說:

  「切斷什麼後\twnr{睡得安樂}{532.0}?切斷什麼後不憂愁?

   對哪一法的殺害,\twnr{喬達摩}{80.0}同意?」

  「切斷憤怒後睡得安樂,切斷憤怒後不憂愁,

   \twnr{襪瑟哇}{263.0}!對端蜜,\twnr{而根毒之憤怒}{929.0}的殺害,

   聖者稱讚,因為切斷它後不憂愁。」[\suttaref{SN.1.71}/\suttaref{SN.2.3}]



\sutta{22}{22}{醜陋經}{https://agama.buddhason.org/SN/sn.php?keyword=11.22}
  在舍衛城祇樹林。

  在那裡……(中略)說這個:

  「\twnr{比丘}{31.0}們!從前,某位醜陋、矮小的\twnr{夜叉}{126.0}坐在天帝釋的座位上。

  比丘們!在那裡,三十三天天神們譏嫌、不滿、責難:『實在\twnr{不可思議}{206.0}啊,\twnr{先生}{202.0}!實在\twnr{未曾有}{206.0}啊,先生!這位醜陋、矮小的夜叉坐在天帝釋的座位上。』

  比丘們!三十三天天神們如是如是譏嫌、不滿、責難,那位夜叉如是如是變成更英俊的同時也更好看的與更端正的。

  比丘們!那時,三十三天天神們去見天帝釋。抵達後,對天帝釋說這個:『\twnr{親愛的先生}{204.0}!這裡,某位醜陋、矮小的夜叉坐在你的天帝釋座位上。親愛的先生!在那裡,三十三天天神們譏嫌、不滿、責難:「實在不可思議啊,先生!實在未曾有啊,先生!這位醜陋、矮小的夜叉坐在天帝釋的座位上。」親愛的先生!三十三天天神們如是如是譏嫌、不滿、責難,那位夜叉如是如是變成更英俊的同時也更好看的與更端正的。』

  『親愛的先生們!那位當然必將是\twnr{食憤怒之夜叉}{x232}。』

  比丘們!那時,天帝釋去見那位食憤怒之夜叉。抵達後,置(作)上衣到一邊肩膀後,右膝蓋觸地、向那位食憤怒之夜叉合掌鞠躬後,告知名字三次:『親愛的先生!我是天帝釋,親愛的先生!我是天帝釋。』

  比丘們!天帝釋如是如是告知名字,那位食憤怒之夜叉如是如是變成更醜陋的同時也更矮小的;變成更醜陋的同時也更矮小的後,就在那裡消失。

  比丘們!那時,天帝釋坐在自己的位子後,安撫著三十三天天神們,在那時說這些\twnr{偈頌}{281.0}:

  『我是心不易被惱害者,不易\twnr{以轉起的}{x233}誘惑者,

   我確實長久不生氣,憤怒不住立於我。

   生氣的我不說粗暴的,並且我不會稱讚[我的]\twnr{諸法}{x234},

   我抑止自己,是自己利益的看見者。』」



\sutta{23}{23}{三婆羅的幻術經}{https://agama.buddhason.org/SN/sn.php?keyword=11.23}
  在舍衛城。

  ……(中略)\twnr{世尊}{12.0}說這個:

  「\twnr{比丘}{31.0}們!從前,毘摩質多阿修羅王是生病者、受苦者、重病者。

  比丘們!那時,\twnr{天帝釋}{263.0}去見毘摩質多阿修羅王,為探病者。

  比丘們!毘摩質多阿修羅王看見正從遠處到來的天帝釋。看見後,對天帝釋說這個:『天帝!請你醫治我。』

  『毘摩質多!請你教我三婆羅[\suttaref{SN.11.10}]的幻術。』

  『\twnr{親愛的先生}{204.0}!我不教,直到我詢問阿修羅們為止。』

  比丘們!那時,摩質多阿修羅王詢問阿修羅們:『親愛的先生們!我[能]教天帝釋三婆羅的幻術?』

  『親愛的先生!你不要教天帝釋三婆羅的幻術。』

  比丘們!那時,摩質多阿修羅王以\twnr{偈頌}{281.0}對天帝釋說:

  『摩伽婆!天帝釋!須闍之夫!幻術者,

   到\twnr{恐怖地獄}{x235}百年,如三婆羅。』



\sutta{24}{24}{罪過經}{https://agama.buddhason.org/SN/sn.php?keyword=11.24}
  在舍衛城……(中略)[給孤獨]園。

  當時,兩位\twnr{比丘}{31.0}爭論。在那裡,一位比丘違越。那時,那位比丘在另一位比丘的當面懺悔罪過為罪過,那位比丘不接受。

  那時,眾多比丘去見世尊。抵達後,向世尊\twnr{問訊}{46.0}後,在一旁坐下。在一旁坐下的那些比丘對世尊說這個:

  「\twnr{大德}{45.0}!這裡,兩位比丘爭論。在那種情況下,一位比丘違越。那時,那位比丘在另一位比丘的當面懺悔罪過為罪過,那位比丘不接受。」

  「比丘們!有這兩類愚者:『凡不見罪過為罪過者,以及凡對如法懺悔罪過者不接受。』比丘們!這是兩類愚者。

  比丘們!有這兩賢智者:『凡見罪過為罪過者,以及凡對如法懺悔罪過者接受。』比丘們!這是兩類賢智者。

  比丘們!從前,\twnr{天帝釋}{263.0}在善法堂安撫著三十三天天神們,在那時說這\twnr{偈頌}{281.0}:

  『令憤怒來到你們的控制,且在你們的友誼上不要成為衰退,

   不要呵責不應該被呵責者,且不要說離間語,

   又-憤怒如山,壓碎惡人。』」



\sutta{25}{25}{無憤怒經}{https://agama.buddhason.org/SN/sn.php?keyword=11.25}
  \twnr{被我這麼聽聞}{1.0}:

  \twnr{有一次}{2.0},\twnr{世尊}{12.0}住在舍衛城祇樹林給孤獨園。

  在那裡,世尊召喚\twnr{比丘}{31.0}們……(中略)世尊說這個:

  「比丘們!從前,\twnr{天帝釋}{263.0}在善法堂安撫著三十三天天神們,在那時說這\twnr{偈頌}{281.0}:

  『不要憤怒征服你們,以及你們不要對生氣者生氣, 

   無憤怒與無加害,在聖者們中是道跡, 

   又-憤怒如山,壓碎惡人。』」

  第三品,其\twnr{攝頌}{35.0}:

  「切斷後、醜陋、幻術,罪過與無憤怒,

   已被最上的佛教導,這帝釋五經。」 

  帝釋相應完成。

  有偈篇第一,其攝頌:

  「諸天與天子,王、魔、比丘尼,

   梵天、婆羅門、婆耆舍,林、夜叉與襪瑟哇。」 

  有偈篇相應經典終了。





\page

\pian{因緣篇}{12}{21}
\xiangying{12}{因緣相應}
\pin{佛陀品}{1}{10}
\sutta{1}{1}{緣起經}{https://agama.buddhason.org/SN/sn.php?keyword=12.1}
  \twnr{被我這麼聽聞}{1.0}:

  \twnr{有一次}{2.0},\twnr{世尊}{12.0}住在舍衛城祇樹林給孤獨園。

  在那裡,世尊召喚\twnr{比丘}{31.0}們:「比丘們!」

  「\twnr{尊師}{480.0}!」那些比丘回答世尊。

  世尊說這個:

  「比丘們!我將為你們教導\twnr{緣起}{225.0},你們要聽它!你們\twnr{要好好作意}{43.1}!我將說。」

  「是的,\twnr{大德}{45.0}!」那些比丘回答世尊。

  世尊說這個:

  「比丘們!而什麼是緣起?比丘們!以\twnr{無明}{207.0}為緣有諸行(而諸行存在);以行\twnr{為緣}{180.0}有識;以識為緣有名色;以名色為緣有六處;以六處為緣有觸;以觸為緣有受;以受為緣有渴愛;以渴愛為緣有取;以取為緣有有;以有為緣有生;以生為緣而老、死、愁、悲、苦、憂、\twnr{絕望}{342.0}生成,這樣是這整個\twnr{苦蘊}{83.0}的\twnr{集}{67.0},比丘們!這被稱為緣起。

  但就以無明的\twnr{無餘褪去與滅}{491.0}有行\twnr{滅}{68.0}(而行滅存在);以行滅有識滅;以識滅有名色滅;以名色滅有六處滅;以六處滅有觸滅;以觸滅有受滅;以受滅有渴愛滅;以渴愛滅有取滅;以取滅有有滅;以有滅有生滅;以生滅而老、死、愁、悲、苦、憂、絕望被滅,這樣是這整個苦蘊的滅。」  

   世尊說這個,那些悅意的比丘歡喜世尊的所說。



\sutta{2}{2}{解析經}{https://agama.buddhason.org/SN/sn.php?keyword=12.2}
  住在舍衛城……(中略)。

  「\twnr{比丘}{31.0}們!我將為你們教導\twnr{緣起}{225.0},我將解析,你們要聽它!你們要\twnr{好好作意}{43.1}!我將說。」

  「是的,\twnr{大德}{45.0}!」那些比丘回答\twnr{世尊}{12.0}。

  世尊說這個:

  「比丘們!而什麼是緣起?比丘們!以\twnr{無明}{207.0}為緣有諸行(而諸行存在);以行\twnr{為緣}{180.0}有識;以識為緣有名色;以名色為緣有六處;以六處為緣有觸;以觸為緣有受;以受為緣有渴愛;以渴愛為緣有取;以取為緣有有;以有為緣有生;以生為緣而老、死、愁、悲、苦、憂、\twnr{絕望}{342.0}生成,這樣是這整個\twnr{苦蘊}{83.0}的\twnr{集}{67.0}。

  比丘們!而什麼是老死?凡一一那些眾生中,以一一那個眾生部類的老、老衰、齒落、髮白、皮皺、壽命的衰退、諸根的退化,這被稱為老。凡一一那些眾生中,以一一那個眾生部類的過世、滅亡、崩解、消失、死亡、壽終、諸蘊的崩解、屍體的捨棄[、\twnr{命根斷絕}{445.1}-\ccchref{MN.9}{https://agama.buddhason.org/MN/dm.php?keyword=9}, 92段],這被稱為死。像這樣,這個老與這個死,比丘們!這被稱為老死。

  比丘們!而什麼是生?凡一一那些眾生中,以一一那個眾生部類的生、出生、進入[胎]、生起、\twnr{生出}{604.0}、諸蘊的顯現、諸處的獲得,比丘們!這被稱為生。

  比丘們!而什麼是有?比丘們!有這三種有:欲有、色有、\twnr{無色有}{261.0},比丘們!這被稱為有。

  比丘們!而什麼是取?比丘們!有這四種取:欲取、見取、\twnr{戒禁取}{194.0}、\twnr{[真]我論取}{444.0},比丘們!這被稱為取。

  比丘們!而什麼是渴愛?比丘們!有這六類渴愛:色的渴愛、聲的渴愛、氣味的渴愛、味道的渴愛、\twnr{所觸}{220.2}的渴愛、法的渴愛,比丘們!這被稱為渴愛。

  比丘們!而什麼是受?比丘們!有這六類受:眼觸所生的受、耳觸所生的受、鼻觸所生的受、舌觸所生的受、身觸所生的受、意觸所生的受,比丘們!這被稱為受。

  比丘們!而什麼是觸?比丘們!有這六類觸:眼觸、耳觸、鼻觸、舌觸、身觸、意觸,比丘們!這被稱為觸。

  比丘們!而什麼是六處?眼處、耳處、鼻處、舌處、身處、意處,比丘們!這被稱為六處。

  比丘們!而什麼是名色?受、想、\twnr{思}{x236}、觸、\twnr{作意}{114.0},這被稱為名。\twnr{四大}{646.0}與四大之所造色,這被稱為色。像這樣,這個名與這個色,比丘們!這被稱為名色。

  比丘們!而什麼是識?比丘們!有這\twnr{六類識}{232.0}:眼識、耳識、鼻識、舌識、身識、意識,比丘們!這被稱為識。

  比丘們!而什麼是諸行?比丘們!有這三種行:身行、語行、心行,比丘們!這些被稱為諸行。

  比丘們!而什麼是無明?比丘們!凡在苦上的無知、在苦集上的無知、在苦滅上的無知、在導向苦\twnr{滅道跡}{69.0}上的無知,比丘們!這被稱為無明。

  比丘們!像這樣,以無明為緣有諸行;以行為緣有識……(中略)這樣是這整個\twnr{苦蘊}{83.0}的\twnr{集}{67.0}。但就以無明的\twnr{無餘褪去與滅}{491.0}有行\twnr{滅}{68.0}(而行滅存在);以行滅有識滅……這樣是這整個苦蘊的滅。」



\sutta{3}{3}{道跡經}{https://agama.buddhason.org/SN/sn.php?keyword=12.3}
  住在舍衛城……(中略)。

  「\twnr{比丘}{31.0}們!我將為你們教導錯誤的道跡與正確的道跡,你們要聽它!你們要\twnr{好好作意}{43.1}!我將說。」

  「是的,\twnr{大德}{45.0}!」那些比丘回答\twnr{世尊}{12.0}。

  世尊說這個:

  「比丘們!而什麼是錯誤的道跡?比丘們!以\twnr{無明}{207.0}為緣有諸行(而諸行存在);以行\twnr{為緣}{180.0}有識……(中略)這樣是這整個\twnr{苦蘊}{83.0}的\twnr{集}{67.0},比丘們!這被稱為錯誤的道跡。

  比丘們!而什麼是正確的道跡?但就以無明的\twnr{無餘褪去與滅}{491.0}有行滅(而行滅存在);以行滅有識滅……(中略)這樣是這整個苦蘊的滅,比丘們!這被稱為正確的道跡。」   



\sutta{4}{4}{毘婆尸經}{https://agama.buddhason.org/SN/sn.php?keyword=12.4}
  住在舍衛城……(中略)。

  「\twnr{比丘}{31.0}們!當就在毘婆尸\twnr{世尊}{12.0}、\twnr{阿羅漢}{5.0}、遍正覺者\twnr{正覺}{185.1}以前,還是未\twnr{現正覺}{75.0}的\twnr{菩薩}{186.0}時想這個:『唉!這個世間確實已陷入苦難:被生、衰老、死去、\twnr{死沒}{x237}、再生,然而,不知道這老死苦的\twnr{出離}{294.0},什麼時候這老死苦的出離才將被知道?』[\ccchref{DN.14}{https://agama.buddhason.org/DN/dm.php?keyword=14}, 57段]

  比丘們!那時,毘婆尸菩薩想這個:『在什麼存在時老死存在呢?以什麼\twnr{為緣}{180.0}有老死(而老死存在)?』比丘們!那時,毘婆尸菩薩從\twnr{如理作意}{114.0},以慧有\twnr{現觀}{53.0}:『在生存在時老死存在;以生為緣而有老死)。』

  比丘們!那時,毘婆尸菩薩想這個:『在什麼存在時生存在呢?以什麼為緣有生?』比丘們!那時,毘婆尸菩薩從如理作意,以慧有現觀:『在有存在時生存在;以有為緣而有生。』

  比丘們!那時,毘婆尸菩薩想這個:『在什麼存在時有存在呢?以什麼為緣而有有?』比丘們!那時,毘婆尸菩薩從如理作意,以慧有現觀:『在取存在時有存在;以取為緣而有有。』

  比丘們!那時,毘婆尸菩薩想這個:『在什麼存在時取存在呢?以什麼為緣而有取?』比丘們!那時,毘婆尸菩薩從如理作意,以慧有現觀:『在渴愛存在時取存在;以渴愛為緣而有取。』

  比丘們!那時,毘婆尸菩薩想這個:『在什麼存在時渴愛存在呢?以什麼為緣而有渴愛?』比丘們!那時,毘婆尸菩薩從如理作意,以慧有現觀:『在受存在時渴愛存在;以受為緣而有渴愛。』

  比丘們!那時,毘婆尸菩薩想這個:『在什麼存在時受存在呢?以什麼為緣而有受?』比丘們!那時,毘婆尸菩薩從如理作意,以慧有現觀:『在觸存在時受存在;以觸為緣而有受。』

  比丘們!那時,毘婆尸菩薩想這個:『在什麼存在時觸存在呢?以什麼為緣而有觸?』比丘們!那時,毘婆尸菩薩從如理作意,以慧有現觀:『在六處存在時觸存在;以六處為緣而有觸。』

  比丘們!那時,毘婆尸菩薩想這個:『在什麼存在時六處存在呢?以什麼為緣而有六處?』比丘們!那時,毘婆尸菩薩從如理作意,以慧有現觀:『在名色存在時六處存在;以名色為緣而有六處。』

  比丘們!那時,毘婆尸菩薩想這個:『在什麼存在時名色存在呢?以什麼為緣而有名色?』比丘們!那時,毘婆尸菩薩從如理作意,以慧有現觀:『在識存在時名色存在;以識為緣而有名色。』

  比丘們!那時,毘婆尸菩薩想這個:『在什麼存在時識存在呢?以什麼為緣而有識?』比丘們!那時,毘婆尸菩薩從如理作意,以慧有現觀:『在諸行存在時識存在;以行為緣而有識。』

  比丘們!那時,毘婆尸菩薩想這個:『在什麼存在時諸行存在呢?以什麼為緣而有諸行?』比丘們!那時,毘婆尸菩薩從如理作意,以慧有現觀:『在\twnr{無明}{207.0}存在時諸行存在;以無明為緣而有諸行。』

  像這樣,以無明為緣而有諸行;以行為緣而有識……(中略)這樣是這整個\twnr{苦蘊}{83.0}的\twnr{集}{67.0}。

  『\twnr{集!集!}{x238}』比丘們!在以前不曾聽聞的諸法上,毘婆尸菩薩的眼生起,智生起,慧生起,明生起,\twnr{光生起}{511.0}。

  比丘們!那時,毘婆尸菩薩想這個:『在什麼不存在時老死不存在呢?以什麼滅而有老死滅(而老死滅存在)?』比丘們!那時,毘婆尸菩薩從如理作意,以慧有現觀:『在生不存在時老死不存在;以生滅而有老死滅。』

  比丘們!那時,毘婆尸菩薩想這個:『在什麼不存在時生不存在呢?以什麼滅而有生滅?』比丘們!那時,毘婆尸菩薩從如理作意,以慧有現觀:『在有不存在時生不存在;以有滅而有生滅。』

  比丘們!那時,毘婆尸菩薩想這個:『在什麼不存在時有不存在呢?以什麼滅而有有滅?』比丘們!那時,毘婆尸菩薩從如理作意,以慧有現觀:『在取不存在時有不存在;以取滅而有有滅。』

  比丘們!那時,毘婆尸菩薩想這個:『在什麼不存在時取不存在呢?以什麼滅而有取滅?』比丘們!那時,毘婆尸菩薩從如理作意,以慧有現觀:『在渴愛不存在時取不存在;以渴愛滅而有取滅。』

  比丘們!那時,毘婆尸菩薩想這個:『在什麼不存在時渴愛不存在呢?以什麼滅而有渴愛滅?』比丘們!那時,毘婆尸菩薩從如理作意,以慧有現觀:『在受不存在時渴愛不存在;以受滅而有渴愛滅。』

  比丘們!那時,毘婆尸菩薩想這個:『在什麼不存在時受不存在呢?以什麼滅而有受滅?』比丘們!那時,毘婆尸菩薩從如理作意,以慧有現觀:『在觸不存在時受不存在;以觸滅而有受滅。』

  比丘們!那時,毘婆尸菩薩想這個:『在什麼不存在時觸不存在呢?以什麼滅而有觸滅?』比丘們!那時,毘婆尸菩薩從如理作意,以慧有現觀:『在六處不存在時觸不存在;以六處滅而有觸滅。』

  比丘們!那時,毘婆尸菩薩想這個:『在什麼不存在時六處不存在呢?以什麼滅而有六處滅?』比丘們!那時,毘婆尸菩薩從如理作意,以慧有現觀:『在名色不存在時六處不存在;以名色\twnr{滅}{68.0}而六處滅。』

  比丘們!那時,毘婆尸菩薩想這個:『在什麼不存在時名色不存在呢?以什麼滅而有名色滅?』比丘們!那時,毘婆尸菩薩從如理作意,以慧有現觀:『在識不存在時名色不存在;以識滅而有名色滅。』

  比丘們!那時,毘婆尸菩薩想這個:『在什麼不存在時識不存在呢?以什麼滅而有識滅?』比丘們!那時,毘婆尸菩薩從如理作意,以慧有現觀:『在諸行不存在時識不存在;以行滅而有識滅。』

  比丘們!那時,毘婆尸菩薩想這個:『在什麼不存在時諸行不存在呢?以什麼滅而有行滅?』比丘們!那時,毘婆尸菩薩從如理作意,以慧有現觀:『在無明不存在時諸行不存在;以無明滅而有行滅。』

  像這樣,以無明滅而有行滅;以行滅而有識滅……(中略)這樣是這整個苦蘊的\twnr{滅}{68.0}。

  『滅!滅!』比丘們!在以前不曾聽聞的諸法上,毘婆尸菩薩的眼生起,智生起,慧生起,明生起,光生起。」

   (對七佛都應該那樣使之被細說)



\sutta{5}{5}{尸棄經}{https://agama.buddhason.org/SN/sn.php?keyword=12.5}
  「\twnr{比丘}{31.0}們!當尸棄\twnr{世尊}{12.0}、\twnr{阿羅漢}{5.0}、遍正覺者……(中略)。」



\sutta{6}{6}{毘舍浮經}{https://agama.buddhason.org/SN/sn.php?keyword=12.6}
  「\twnr{比丘}{31.0}們!當毘舍浮\twnr{世尊}{12.0}、\twnr{阿羅漢}{5.0}、遍正覺者……(中略)。」



\sutta{7}{7}{拘留孫經}{https://agama.buddhason.org/SN/sn.php?keyword=12.7}
  「\twnr{比丘}{31.0}們!當拘留孫\twnr{世尊}{12.0}、\twnr{阿羅漢}{5.0}、遍正覺者……(中略)。」



\sutta{8}{8}{拘那含經}{https://agama.buddhason.org/SN/sn.php?keyword=12.8}
  「\twnr{比丘}{31.0}們!當拘那含\twnr{世尊}{12.0}、\twnr{阿羅漢}{5.0}、遍正覺者……(中略)。」



\sutta{9}{9}{迦葉經}{https://agama.buddhason.org/SN/sn.php?keyword=12.9}
  「\twnr{比丘}{31.0}們!當迦葉\twnr{世尊}{12.0}、\twnr{阿羅漢}{5.0}、遍正覺者……(中略)。」



\sutta{10}{10}{喬達摩經}{https://agama.buddhason.org/SN/sn.php?keyword=12.10}
  「\twnr{比丘}{31.0}們!當就在我\twnr{正覺}{185.1}以前,還是未\twnr{現正覺}{75.0}的\twnr{菩薩}{186.0}時想這個:『唉!這個世間確實已陷入苦難:被生、衰老、死去、\twnr{死沒}{x239}、再生,然而,不知道這老死苦的\twnr{出離}{294.0},什麼時候這老死苦的出離才將被知道?』

  比丘們!那個我想這個:『在什麼存在時老死存在呢?以什麼\twnr{為緣}{180.0}有老死(而老死存在)?』比丘們!那個我從\twnr{如理作意}{114.0},以慧有\twnr{現觀}{53.0}:『在生存在時老死存在;以生為緣有老死)。』

  比丘們!那個我想這個:『在什麼存在時生存在呢?……(中略)有……取……渴愛……受……觸……六處……名色……識……諸行存在?以什麼為緣有諸行(而諸行存在)?』

  比丘們!那個我從如理作意,以慧有現觀:『在\twnr{無明}{207.0}存在時諸行存在;以無明為緣有諸行(而諸行存在)。』

  像這樣,以無明為緣有諸行;以行為緣有識……(中略)這樣是這整個\twnr{苦蘊}{83.0}的\twnr{集}{67.0}。

  『\twnr{集!集!}{x238}』比丘們!在以前不曾聽聞的諸法上,我的眼生起,智生起,慧生起,明生起,\twnr{光生起}{511.0}。

  比丘們!那個我想這個:『在什麼不存在時老死不存在呢?以什麼滅有老死滅(而老死滅存在)』比丘們!那個我從如理作意,以慧有現觀:『在生不存在時老死不存在;以生滅有老死滅。』

  比丘們!那個我想這個:『在什麼不存在時生不存在?……(中略)有……取……渴愛……受……觸……六處……名色……識……諸行不存在?以什麼滅有行滅呢?』

  比丘們!那個我從如理作意,以慧有現觀:『在無明不存在時諸行不存在;以無明滅有行滅。』

  像這樣,以無明滅有行滅;以行滅有識滅……(中略)這樣是這整個苦蘊的\twnr{滅}{68.0}。

  『滅!滅!』比丘們!在以前不曾聽聞的諸法上,我的眼生起,智生起,慧生起,明生起,光生起。」

  佛陀品第一,其\twnr{攝頌}{35.0}:

  「教說與解析、道跡,毘婆尸、尸棄及毘舍浮,

   拘留孫、拘那含、迦葉,以及大釋迦牟尼喬達摩。」





\pin{食品}{11}{20}
\sutta{11}{11}{食經}{https://agama.buddhason.org/SN/sn.php?keyword=12.11}
  \twnr{被我這麼聽聞}{1.0}:

  \twnr{有一次}{2.0},\twnr{世尊}{12.0}住在舍衛城祇樹林給孤獨園。……(中略)說這個:

  「\twnr{比丘}{31.0}們!有這\twnr{四種食}{241.0}:為了已生成眾生的存續,或為了\twnr{求出生者}{711.0}的資助。哪四種?\twnr{或粗或細的物質食物}{387.0},第二、\twnr{觸}{388.0},第三、\twnr{意思}{389.0},第四、\twnr{識}{390.0}。比丘們!這是四種食:為了已生成眾生的存續,或為了求出生者的資助。

  比丘們!這四種食,什麼為因?什麼為集?什麼生的?\twnr{什麼為根源}{668.0}?這四種食,渴愛為因,渴愛為集,渴愛生的,渴愛為根源。

  比丘們!而這個渴愛,什麼為因?什麼為集?什麼生的?什麼為根源?渴愛,受為因,受為集,受生的,受為根源。

  比丘們!而這個受,什麼為因?什麼為集?什麼生的?什麼為根源?受,觸為因,觸為集,觸生的,觸為根源。

  比丘們!而這個觸,什麼為因?什麼為集?什麼生的?什麼為根源?觸,六處為因,六處為集,六處生的,六處為根源。

  比丘們!而這個六處,什麼為因?什麼為集?什麼生的?什麼為根源?六處,名色為因,名色為集,名色生的,名色為根源。

  比丘們!而這個名色,什麼為因?什麼為集?什麼生的?什麼為根源?名色,識為因,識為集,識生的,識為根源。

  比丘們!而這個識,什麼為因?什麼為集?什麼生的?什麼為根源?識,行為因,行為集,行生的,行為根源。

  比丘們!而這些行,什麼為因?什麼為集?什麼生的?什麼為根源?諸行,\twnr{無明}{207.0}為因,無明為集,無明生的,無明為根源。

  比丘們!像這樣,以無明\twnr{為緣}{180.0}有諸行(而諸行存在);以行為緣有識……(中略)這樣是這整個\twnr{苦蘊}{83.0}的\twnr{集}{67.0}。

  但就以無明的\twnr{無餘褪去與滅}{491.0}有行滅(而行滅存在)……(中略)這樣是這整個苦蘊的滅。」[\ccchref{MN.38}{https://agama.buddhason.org/MN/dm.php?keyword=38}, 402段]



\sutta{12}{12}{摩利亞帕辜那經}{https://agama.buddhason.org/SN/sn.php?keyword=12.12}
  住在舍衛城……(中略)。

  「\twnr{比丘}{31.0}們!有這\twnr{四種食}{241.0}:為了已生成眾生的存續,或為了\twnr{求出生者}{711.0}的資助。哪四種?\twnr{或粗或細的物質食物}{387.0},第二、\twnr{觸}{388.0},第三、\twnr{意思}{389.0},第四、\twnr{識}{390.0}。比丘們!這是四種食:為了已生成眾生的存續,或為了求出生者的資助。」

  在這麼說時,\twnr{尊者}{200.0}摩利亞\twnr{帕辜那}{x240}對\twnr{世尊}{12.0}說這個:

  「\twnr{大德}{45.0}!誰吃識食呢?」

  「不適當的問題。」世尊說。

  「我不說『\twnr{他吃}{x241}』,而如果我說『他吃』,在那種情況下,那會是適當的問題:『大德!誰吃呢?』但我不說這個。

  當不說這個時,凡如果這麼問我:『大德!識食是為了什麼呢?』這是適當的問題。在那種情況下,適當的解答是:『識食是\twnr{未來再有的出生}{804.0}之緣:在那位已生成者存在時六處存在;以六處\twnr{為緣}{180.0}有觸(而觸存在)。』」

  「大德!\twnr{誰觸呢}{x242}?」

  「不適當的問題。」世尊說。

  「我不說『他觸』,而如果我說『他觸』,在那種情況下,那會是適當的問題:『大德!誰觸呢?』但我不說這個。

  當不說這個時,凡如果這麼問我:『大德!以什麼為緣有觸呢?』這是適當的問題。在那種情況下,適當的解答是:『以六處為緣有觸;以觸為緣有受。』」

  「大德!誰感受呢?」

  「不適當的問題。」世尊說。

  「我不說『他感受』,而如果我說『他感受』,在那種情況下,那會是適當的問題:『大德!誰感受呢?』但我不說這個。

  當不說這個時,凡如果這麼問我:『大德!以什麼為緣有受呢?』這是適當的問題。在那種情況下,適當的解答是:『以觸為緣有受;以受為緣有渴愛。』」

  「大德!誰渴愛?」

  「不適當的問題。」世尊說。

  「我不說『他渴愛』,而如果我說『他渴愛』,在那種情況下,那會是適當的問題:『大德!誰渴愛呢?』但我不說這個。

  當不說這個時,凡如果這麼問我:『大德!以什麼為緣有渴愛呢?』這是適當的問題。在那種情況下,適當的解答是:『以受為緣有渴愛;以渴愛為緣有取。』」

  「大德!誰執取呢?」

  「不適當的問題。」世尊說。

  「我不說『他執取』,而如果我說『他執取』,在那種情況下,那會是適當的問題:『大德!誰執取呢?』但我不說這個。

  當不說這個時,凡如果這麼問我:『大德!以什麼為緣有取呢?』這是適當的問題。在那種情況下,適當的解答是:『以渴愛為緣有取;以取為緣有有……(中略)這樣是這整個\twnr{苦蘊}{83.0}的\twnr{集}{67.0}。

  帕辜那!但就以\twnr{六觸處}{78.0}的\twnr{無餘褪去與滅}{491.0}有觸滅(而觸滅存在);以觸滅有受滅;以受滅有渴愛滅;以渴愛滅有取滅;以取滅有有滅;以有滅有生滅;以生滅而老、死、愁、悲、苦、憂、\twnr{絕望}{342.0}被滅,這樣是這整個苦蘊的\twnr{滅}{68.0}。』」



\sutta{13}{13}{沙門婆羅門經}{https://agama.buddhason.org/SN/sn.php?keyword=12.13}
  住在舍衛城……(中略)。

  「\twnr{比丘}{31.0}們!凡任何\twnr{沙門}{29.0}或\twnr{婆羅門}{17.0}不知道老死,不知道老死\twnr{集}{67.0},不知道老死\twnr{滅}{68.0},不知道導向老死\twnr{滅道跡}{69.0};生……(中略)有……取……渴愛……受……觸……六處……名色……識……不知道諸行,不知道行集,不知道行滅,不知道導向行滅道跡者,比丘們!那些沙門或婆羅門不被我認同為\twnr{沙門中的沙門}{560.0},或婆羅門中的婆羅門,而且,那些\twnr{尊者}{200.0}也不以證智自作證後,在當生中\twnr{進入後住於}{66.0}\twnr{沙門義}{327.0}或婆羅門義。

  比丘們!而凡任何沙門或婆羅門知道老死,知道老死集,知道老死滅,知道導向老死滅道跡;生……(中略)有……取……渴愛……受……觸……六處……名色……識……知道諸行,知道行集,知道行滅,知道導向行滅道跡者,比丘們!那些沙門或婆羅門被我認同為沙門中的沙門,或婆羅門中的婆羅門,而且,那些尊者也以證智自作證後,在當生中進入後住於沙門義或婆羅門義。」



\sutta{14}{14}{沙門婆羅門經第二}{https://agama.buddhason.org/SN/sn.php?keyword=12.14}
  住在舍衛城……(中略)。

  「\twnr{比丘}{31.0}們!凡任何\twnr{沙門}{29.0}或\twnr{婆羅門}{17.0}不知道這些法,不知道這些法的\twnr{集}{67.0},不知道這些法的\twnr{滅}{68.0},不知道導向這些法的\twnr{滅道跡}{69.0}者,他們不知道哪些法?不知道哪些法的集?不知道哪些法的滅?不知道哪些法的導向滅道跡?

  他們不知道老死,不知道老死集,不知道老死滅,不知道導向老死滅道跡;生……(中略)有……取……渴愛……受……觸……六處……名色……識……不知道諸行,不知道行集,不知道行滅,不知道導向行滅道跡,他們不知道這些法,不知道這些法的集,不知道這些法的滅,不知道導向這些法的滅道跡。比丘們!那些沙門或婆羅門不被我認同為\twnr{沙門中的沙門}{560.0},或婆羅門中的婆羅門,而且,那些\twnr{尊者}{200.0}也不以證智自作證後,在當生中\twnr{進入後住於}{66.0}\twnr{沙門義}{327.0}或婆羅門義。

  比丘們!而凡任何沙門或婆羅門知道這些法,知道這些法的集,知道這些法的滅,知道導向這些法的滅道跡者,他們知道哪些法?知道哪些法的集?知道哪些法的滅?知道哪些法的導向滅道跡?

  他們知道老死,知道老死集,知道老死滅,知道導向老死滅道跡;生……(中略)有……取……渴愛……受……觸……六處……名色……識……知道諸行,知道行集,知道行滅,知道導向行滅道跡,他們知道這些法,知道這些法的集,知道這些法的滅,知道導向這些法的滅道跡。比丘們!那些沙門或婆羅門被我認同為沙門中的沙門,或婆羅門中的婆羅門,而且,那些尊者也以證智自作證後,在當生中進入後住於沙門義或婆羅門義。」



\sutta{15}{15}{迦旃延氏經}{https://agama.buddhason.org/SN/sn.php?keyword=12.15}
  住在舍衛城。

  那時,\twnr{尊者}{200.0}迦旃延氏去見世尊。抵達後,向世尊\twnr{問訊}{46.0}後,在一旁坐下。在一旁坐下的尊者迦旃延氏對世尊說這個:

  「\twnr{大德}{45.0}!被稱為『正見、正見』,大德!什麼情形是正見呢?」

  「迦旃延!這世間多數是依止兩種者:實有性(實有的觀念)與虛無性。迦旃延!對以正確之慧如實看見世間集者,凡關於世間的虛無性它不存在(對世間沒有虛無的觀念);迦旃延!對以正確之慧如實看見世間滅者,凡關於世間的實有性它不存在。

  迦旃延!這世間多數有攀住、執取、執持的束縛,但對那個攀住、執取、心的依處、執持、\twnr{煩惱潛在趨勢}{253.1},這位\twnr{不攀取}{717.0}、不執取,他不執持『我的真我』,『當生起時僅苦生起;當被滅時苦被滅。』他不疑惑、不懷疑,不從緣於他人就在這裡有他的智(他的智存在),迦旃延!這個情形是正見。

  迦旃延!『一切存在』,這是一邊(極端);『一切不存在』,這是第二邊,迦旃延!不走入這些那些兩個邊後,如來以中間教導法:『以無明\twnr{為緣}{180.0}有諸行(而諸行存在);以行為緣有識……(中略)這樣是這整個\twnr{苦蘊}{83.0}的\twnr{集}{67.0}。但就以無明的\twnr{無餘褪去與滅}{491.0}有行滅(而行滅存在);以行滅有識滅……(中略)這樣是這整個苦蘊的滅。』」



\sutta{16}{16}{說法者經}{https://agama.buddhason.org/SN/sn.php?keyword=12.16}
  在舍衛城……(中略)。

  那時,\twnr{某位}{39.0}比丘去見\twnr{世尊}{12.0}。抵達後,向世尊\twnr{問訊}{46.0}後,在一旁坐下。在一旁坐下的那位比丘對世尊說這個:

  「\twnr{大德}{45.0}!被稱為『說法者,說法者』,大德!什麼情形是說法者呢?」

  「比丘!如果對老死為了\twnr{厭}{15.0}、\twnr{離貪}{77.0}、\twnr{滅}{68.0}而教導法,『說法者\twnr{比丘}{31.0}』是適當的言語。

  比丘!如果對老死是為了厭、離貪、\twnr{滅的行者}{519.0},『\twnr{法、隨法行者}{58.0}比丘』是適當的言語。

  比丘!如果對老死從厭、離貪、滅,不執取後成為解脫者,『得當生涅槃者比丘』是適當的言語。

  比丘!如果對生……(中略)比丘!如果對有……比丘!如果對取……比丘!如果對渴愛……比丘!如果對受……比丘!如果對觸……比丘!如果對六處……比丘!如果對名色……比丘!如果對識……比丘!如果對行……比丘!如果對\twnr{無明}{207.0}為了厭、離貪、滅而教導法,『說法者比丘』是適當的言語。

  比丘!如果對無明是為了厭、離貪、滅的行者,『法、隨法行者比丘』是適當的言語。

  比丘!如果對無明從厭、離貪、滅,不執取後成為解脫者,『得當生涅槃者比丘』是適當的言語。」



\sutta{17}{17}{裸行者迦葉經}{https://agama.buddhason.org/SN/sn.php?keyword=12.17}
  \twnr{被我這麼聽聞}{1.0}:

  \twnr{有一次}{2.0},\twnr{世尊}{12.0}住在王舍城栗鼠飼養處的竹林中。

  那時,世尊午前時穿衣、拿起衣鉢後,\twnr{為了托鉢}{87.0}進入王舍城。

  \twnr{裸行者}{x243}迦葉看見正從遠處到來的世尊。看見後,前去見世尊。抵達後,與世尊一起互相問候。交換應該被互相問候的友好交談後,在一旁站立。在一旁站立的裸行者迦葉對世尊說這個:

  「我們想要就某一點問\twnr{喬達摩}{80.0}\twnr{尊師}{203.0},如果喬達摩尊師為我們的問題之解答給機會。」

  「迦葉!為了問題,大致上是不適時的,我們[要]進入俗家內。」

  第二次,裸行者迦葉對世尊說這個:

  「我們想要就某一點問喬達摩尊師,如果喬達摩尊師為我們的問題之解答給機會的話。」

  「迦葉!為了問題,大致上是不適時的,我們[要]進入俗家內。」

  第三次裸行者迦葉……(中略)「……我們[要]進入俗家內。」

  在這麼說時,裸行者迦葉對世尊說這個這個:

  「可是我們並非想要問喬達摩尊師很多。」

  「迦葉!請你問凡你希望的。」

  「喬達摩尊師!是否苦是自己作的嗎?」

  「不是這樣的,迦葉!」世尊說。

  「喬達摩尊師!那麼,苦是其他者作的嗎?」

  「不是這樣的,迦葉!」世尊說。

  「喬達摩尊師!是否苦是\twnr{自己作的與其他者作的}{172.1}嗎?」

  「不是這樣的,迦葉!」世尊說。

  「喬達摩尊師!那麼,苦是非自己也非其他者作的;\twnr{自然生}{173.0}的嗎?」

  「不是這樣的,迦葉!」世尊說。

  「喬達摩尊師!是否沒有苦嗎?」

  「迦葉!非沒有苦,迦葉!有苦。」

  「那樣的話,喬達摩尊師不知、不見苦?」

  「迦葉!我非不知、不見苦,迦葉!我知苦,迦葉!我見苦。」

  「當被像這樣問:『喬達摩尊師!是否苦是自己作的嗎?』你說:『不是這樣的,迦葉!』當被像這樣問:『喬達摩尊師!那麼,苦是其他者作的嗎?』你說:『不是這樣的,迦葉!』當被像這樣問:『喬達摩尊師!是否苦是自己作的與其他者作的?』你說:『不是這樣的,迦葉!』當被像這樣問:『喬達摩尊師!那麼,苦是非自己也非其他者作的;自然生的嗎?』你說:『不是這樣的,迦葉!』當被像這樣問:『喬達摩尊師!是否沒有苦嗎?』你說:『迦葉!非沒有苦,迦葉!有苦。』當被像這樣問:『那樣的話,喬達摩尊師不知、不見苦?』你說:『我非不知、不見苦,迦葉!我知苦,迦葉!我見苦。』\twnr{大德}{45.0}!請世尊為我解說苦,大德!請世尊為我教導苦。」

  「迦葉!『他做,他感受。』從最初的存在者想:『苦是自己作的』,像這樣他來到這個常恆論。

  迦葉!『一個做,另一個感受。』被受征服的存在者想:『苦是其他者作的』,像這樣他來到這個斷滅論。

  迦葉!不走入這些那些兩個邊後,如來以中間教導法:『以無明\twnr{為緣}{180.0}有諸行(而諸行存在);以行為緣有識……(中略)這樣是這整個\twnr{苦蘊}{83.0}的\twnr{集}{67.0}。但就以無明的\twnr{無餘褪去與滅}{491.0}有行滅(而行滅存在);以行滅有識滅……(中略)這樣是這整個苦蘊的滅。』」

  在這麼說時,裸行者迦葉對世尊說這個:

  「大德!太偉大了,大德!太偉大了,大德!猶如扶正顛倒的……(中略):『有眼者們看見諸色。』同樣的,法被世尊以種種\twnr{法門}{562.0}說明。大德!這個我歸依世尊、法、\twnr{比丘僧團}{65.0},大德!願我得到在世尊的面前出家,願我\twnr{得到具足戒}{124.1}。」

  「迦葉!凡先前為其他外道希望在這法律中出家;希望受具足戒,他\twnr{別住四個月}{870.0}。經四個月後,發心的比丘們使出家;使受具足戒為比丘狀態,但個人差異由我發現。」[\ccchref{DN.8}{https://agama.buddhason.org/DN/dm.php?keyword=8}]

  「大德!如果先前為其他外道希望在這法律中出家者;希望受具足戒者別住四個月。經四個月後,發心的比丘們使之出家;使之受具足戒為比丘狀態,我將別住四年。經四年後,請發心的比丘們使出家;使受具足戒為比丘狀態。」

  裸行者迦葉得到在世尊的面前出家、受具足戒。

  還有,已受具足戒不久,住於單獨的、隱離的、不放逸的、熱心的、自我努力的\twnr{尊者}{200.0}迦葉不久就以證智自作證後,在當生中\twnr{進入後住於}{66.0}凡\twnr{善男子}{41.0}們為了利益正確地\twnr{從在家出家成為無家者}{48.0}的那個無上梵行結尾,他證知:「\twnr{出生已盡}{18.0},\twnr{梵行已完成}{19.0},\twnr{應該被作的已作}{20.0},\twnr{不再有此處[輪迴]的狀態}{21.1}。」然後尊者迦葉成為眾\twnr{阿羅漢}{5.0}之一。



\sutta{18}{18}{丁巴盧葛經}{https://agama.buddhason.org/SN/sn.php?keyword=12.18}
  住在舍衛城。

  那時,\twnr{遊行者}{79.0}丁巴盧葛去見\twnr{世尊}{12.0}。抵達後,與世尊一起互相問候。交換應該被互相問候的友好交談後,在一旁坐下。在一旁坐下的遊行者丁巴盧葛對世尊說這個:

  「\twnr{喬達摩}{80.0}尊師!是否苦樂(樂、苦)是自己作的嗎?」

  「不是這樣的,丁巴盧葛!」世尊說。

  「喬達摩尊師!那麼,苦樂是其他者作的嗎?」

  「不是這樣的,丁巴盧葛!」世尊說。

  「喬達摩尊師!是否苦樂是\twnr{自己作的與其他者作的}{172.1}嗎?」

  「不是這樣的,丁巴盧葛!」世尊說。

  「喬達摩尊師!那麼,苦樂是非自己也非其他者作的;\twnr{自然生}{173.0}的嗎?」

  「不是這樣的,丁巴盧葛!」世尊說。

  「喬達摩尊師!是否沒有苦樂嗎?」

  「丁巴盧葛!非沒有苦樂,丁巴盧葛!有苦樂。」

  「那樣的話,喬達摩尊師不知、不見苦樂?」

  「丁巴盧葛!我非不知、不見苦樂,丁巴盧葛!我知苦樂,丁巴盧葛!我見苦樂。」

  「當被像這樣問:『喬達摩尊師!是否苦樂是自己作的嗎?』你說:『不是這樣的,丁巴盧葛!』當被像這樣問:『喬達摩尊師!那麼,苦樂是其他者作的嗎?』你說:『不是這樣的,丁巴盧葛!』當被像這樣問:『喬達摩尊師!是否苦樂是自己作的與其他者作的?』你說:『不是這樣的,丁巴盧葛!』當被像這樣問:『喬達摩尊師!那麼,苦樂被非自己也非其他所作;自然生的嗎?』你說:『不是這樣的,丁巴盧葛!』當被像這樣問:『喬達摩尊師!是否沒有苦樂嗎?』你說:『丁巴盧葛!不是沒有苦樂,丁巴盧葛!有苦樂。』當被像這樣問:『那樣的話,喬達摩尊師不知、不見苦樂?』你說:『我確實非不知、不見苦樂,丁巴盧葛!我確實知苦樂,丁巴盧葛!我確實見苦樂。』\twnr{大德}{45.0}!請世尊為我解說苦樂,大德!請世尊為我教導苦樂。」

  「丁巴盧葛!『有那個感受,\twnr{那位感受}{x244}。』從最初的存在者想:『苦樂是自己作的。』我不說這個。

  丁巴盧葛!『有另一個感受,\twnr{另一位感受}{x245}。』被受征服的存在者想:『苦樂是其他者作的。』我不說這個。

  丁巴盧葛!不走入這些那些兩個邊後,\twnr{如來}{4.0}以中間教導法:『以\twnr{無明}{207.0}\twnr{為緣}{180.0}有諸行(而諸行存在);以行為緣有識……(中略)這樣是這整個\twnr{苦蘊}{83.0}的\twnr{集}{67.0}。但就以無明的\twnr{無餘褪去與滅}{491.0}有行滅(而行滅存在);以行滅有識滅……(中略)這樣是這整個苦蘊的滅。』」

  在這麼說時,遊行者丁巴盧葛對世尊說這個:

  「太偉大了,喬達摩尊師!……這個我\twnr{歸依}{284.0}喬達摩\twnr{尊師}{203.0}、法、\twnr{比丘僧團}{65.0},請喬達摩尊師記得我為\twnr{優婆塞}{98.0},從今天起\twnr{已終生歸依}{64.0}。」



\sutta{19}{19}{愚癡者賢智者經}{https://agama.buddhason.org/SN/sn.php?keyword=12.19}
  住在舍衛城……(中略)。

  「\twnr{比丘}{31.0}們!\twnr{無明蓋}{158.0}、被渴愛連結的愚癡者這樣的這個身被生成。像這樣,\twnr{這個身與外名色}{x246},在這裡這是一對,\twnr{緣於一對而有觸}{x247},就被六處或它們的某個接觸,愚癡者感受苦樂(樂、苦)。

  比丘們!無明蓋、與渴愛相應的賢智者這樣的這個身生成。像這樣,這個身與外名色,為一對,緣於一對而有觸,就被六處或它們的某個接觸,賢智者感受苦樂。

  比丘們!在那裡,對賢智者比愚癡者,什麼是差別,什麼是不同,什麼是區別?」「\twnr{大德}{45.0}!我們的法是\twnr{世尊}{12.0}為根本的、\twnr{世尊為導引的}{56.0}、世尊為依歸的,大德!就請世尊說明這個所說的義理,\twnr{那就好了}{44.0}!聽聞世尊的[教說]後,比丘們將會\twnr{憶持}{57.0}。」

  「比丘們!那樣的話,你們要聽!你們要\twnr{好好作意}{43.1}!我將說。」

  「是的,\twnr{大德}{45.0}!」那些比丘回答世尊。

  世尊說這個:

  「比丘們!凡被無明蓋與凡被渴愛連結的愚癡者的這個身被生成。愚癡者的那個無明未被捨斷連同那個渴愛未被遍滅盡,那是什麼原因?比丘們!愚癡者不為了苦的完全滅盡行梵行,因此,以身體的崩解,愚癡者\twnr{是身體的轉生者}{x248},當是身體的轉生者時,他不從生、老、死、愁、悲、苦、憂、\twnr{絕望}{342.0}被釋放,我說:『不從苦被釋放。』

  比丘們!凡被無明蓋與凡被渴愛連結的賢智者的這個身被生成。賢智者的那個無明被捨斷連同那個渴愛被遍滅盡,那是什麼原因?比丘們!賢智者為了苦的完全滅盡行梵行,因此,以身體的崩解,賢智者不是身體的轉生者,當不是身體的轉生者時,他從生、老、死、愁、悲、苦、憂、絕望被釋放,我說:『從苦被釋放。』

  比丘們!對賢智者比愚癡者,這是差別,這是不同,這是區別,即:梵行的生活。」



\sutta{20}{20}{緣經}{https://agama.buddhason.org/SN/sn.php?keyword=12.20}
  住在舍衛城……(中略)。

  「\twnr{比丘}{31.0}們!我將為你們教導緣起與諸\twnr{緣所生法}{225.1},你們要聽它!你們要\twnr{好好作意}{43.1}!我將說。」

  「是的,\twnr{大德}{45.0}!」那些比丘回答世尊。

  世尊說這個:

  「比丘們!而什麼是緣起?比丘們!以生\twnr{為緣}{180.0}有老死(而老死存在):諸如來的出現,或諸如來的不出現,\twnr{那個界住立}{x249}、\twnr{法安住性}{634.0}、\twnr{法決定性}{635.0}、\twnr{特定條件性}{636.0}。如來\twnr{現正覺}{75.0}、\twnr{現觀}{53.0}它;現正覺、現觀後,告知、教導、\twnr{使知}{143.0}、建立、開顯、解析、闡明:『比丘們!你們看!以生為緣有老死。』

  比丘們!以有為緣有生……(中略)比丘們!以取為緣有有……比丘們!以渴愛為緣有取……比丘們!以受為緣有渴愛……比丘們!以觸為緣有受……比丘們!以六處為緣有觸……比丘們!以名色為緣有六處……比丘們!以識為緣有名色……比丘們!以行為緣有識……比丘們!以無明為緣有諸行:諸如來的出現,或諸如來不的出現,那個界住立、法安住性、法決定性、特定條件性。如來現正覺、現觀它;現正覺、現觀後,告知、教導、使知、建立、開顯、解析、闡明:『比丘們!你們看!以無明為緣有諸行。』

  比丘們!像這樣,在那裡,凡真實性、無誤性、\twnr{無例外性}{855.2}、特定條件性者,比丘們!這被稱為緣起。

  比丘們!又,而什麼是諸緣所生法?比丘們!老死是無常的、\twnr{有為的}{90.0}、\twnr{緣所生的}{557.0},是\twnr{滅盡法}{273.1}、\twnr{消散法}{155.0}、\twnr{褪去}{77.0}法、\twnr{滅法}{68.1}。

  比丘們!生是無常的、有為的、緣所生的,是滅盡法、消散法、褪去法、滅法。

  比丘們!有是無常的、有為的、緣所生的,是滅盡法、消散法、褪去法、滅法。

  比丘們!取……(中略)比丘們!渴愛……比丘們!受……比丘們!觸……比丘們!六處……比丘們!名色……比丘們!識……比丘們!諸行……比丘們!無明是無常的、有為的、緣所生的,是滅盡法、消散法、褪去法、滅法。比丘們!這些被稱為緣所生法。

  比丘們!當聖弟子的這個緣起與這些緣所生法被正確之慧如實善見,他將跑回過去:『我過去世曾存在嗎?我過去世不曾存在嗎?我過去世曾是什麼(誰)呢?我過去世曾是怎樣呢?成為什麼後我過去世曾是什麼?』或他將跑到未來:『我未來世將存在嗎?我未來世將不存在嗎?我未來世將是什麼?我未來世將是怎樣呢?成為什麼後我未來世將是什麼?』或現在他自身內{將-\ccchref{MN.2}{https://agama.buddhason.org/MN/dm.php?keyword=2}}有現在世的疑惑:『我存在嗎?我不存在嗎?我是什麼?我是怎樣呢?這眾生從哪裡來的,他將去哪裡呢?』\twnr{這不存在可能性}{650.0},那是什麼原因?比丘們!因為,像這樣,聖弟子的這個緣起與這些緣所生法被正確之慧如實善見。」

  食品第二,其\twnr{攝頌}{35.0}:

  「食與帕辜那,以及沙門與婆羅門二則,

   迦旃延氏、說法者,裸行者與丁巴盧葛,

   愚癡者與賢智者,以及以緣為第十。」





\pin{十力品}{21}{30}
\sutta{21}{21}{十力經}{https://agama.buddhason.org/SN/sn.php?keyword=12.21}
  住在舍衛城……(中略)。

  「\twnr{比丘}{31.0}們!具備十力與具備\twnr{四無畏}{357.1},如來自稱為\twnr{最上位}{684.0},在群眾中吼獅子吼,使\twnr{梵輪}{309.0}轉起:『像這樣是色,像這樣是色的\twnr{集}{67.0},像這樣是色的滅沒;像這樣是受,像這樣是受的集,像這樣是受的滅沒;像這樣是想,像這樣是想的集,像這樣是想的滅沒;像這樣是諸行,像這樣是諸行的集,像這樣是諸行的滅沒;像這樣是識,像這樣是識的集,像這樣是識的滅沒。像這樣,在這個存在時那個存在,以這個的生起那個生起。在這個不存在時那個不存在,以這個的滅\twnr{那個被滅}{394.0},即:以\twnr{無明}{207.0}為緣有諸行(而諸行存在);以行\twnr{為緣}{180.0}有識……(中略)這樣是這整個\twnr{苦蘊}{83.0}的集。但就以無明的\twnr{無餘褪去與滅}{491.0}有行滅(而行滅存在);以行滅有識滅……(中略)這樣是這整個苦蘊的滅。』」



\sutta{22}{22}{十力經第二}{https://agama.buddhason.org/SN/sn.php?keyword=12.22}
  住在舍衛城……(中略)。

  「\twnr{比丘}{31.0}們!具備十力與具備\twnr{四無畏}{357.1},如來自稱為\twnr{最上位}{684.0},在群眾中吼獅子吼,使\twnr{梵輪}{309.0}轉起:『像這樣是色,像這樣是色的\twnr{集}{67.0},像這樣是色的滅沒;像這樣是受,像這樣是受的集,像這樣是受的滅沒;像這樣是想,像這樣是想的集,像這樣是想的滅沒;像這樣是諸行,像這樣是諸行的集,像這樣是諸行的滅沒;像這樣是識,像這樣是識的集,像這樣是識的滅沒。像這樣,在這個存在時那個存在,以這個的生起那個生起。在這個不存在時那個不存在,以這個的滅\twnr{那個被滅}{394.0},即:以\twnr{無明}{207.0}為緣有諸行(而諸行存在);以行\twnr{為緣}{180.0}有識……(中略)這樣是這整個\twnr{苦蘊}{83.0}的集。但就以無明的\twnr{無餘褪去與滅}{491.0}有行滅(而行滅存在);以行滅有識滅……(中略)這樣是這整個苦蘊的滅。』

  比丘們!被我這麼善說的法成為明瞭的、開顯的、變明亮的、切開舊衣的。比丘們!在被我這麼善說的法成為明瞭的、開顯的、變明亮的、切開舊衣的時,以從信出家的\twnr{善男子}{41.0},他就足以要發動活力:『寧願剩下皮膚、肌腱、骨骸,身體中的血肉變乾,凡那個應該被人的力量、人的活力、人的努力達成的,那個沒達成後,他將沒有活力的止息。』

  比丘們!被諸惡不善法覆蓋的\twnr{怠惰者住於苦}{x250},且使大善利退失。

  比丘們!但與諸惡不善法遠離的活力已發動者\twnr{住於樂}{317.0},且使大善利完成。

  比丘們!不以下劣而有最高的達成;比丘們!但以最高而有最高的達成。

  比丘們!\twnr{大師是面對者,這梵行是}{145.0}\twnr{醍醐味}{361.1}[\ccchref{Ps.10}{https://agama.buddhason.org/Ps/Ps10.htm}],比丘們!因此,在這裡,為了未達成的達成;為了未證得的證得;為了未作證的作證,你們要發動活力:『這樣,我們的這個出家確實將是\twnr{功不唐捐的}{334.1}、有果實的、有結果的,而且,凡我們受用衣服、\twnr{施食}{196.0}、臥坐處、病人需物、醫藥必需品,他們於我們那些的行為將有大果、\twnr{大效益}{113.0}。』比丘們!應該被你們這麼學。

  比丘們!以看見自己的利益者就足以要以不放逸使之(目標)達成;比丘們!以看見別人的利益者就足以要以不放逸使之達成;比丘們!以看見兩者的利益者就足以要以不放逸使之達成。」



\sutta{23}{23}{近因經}{https://agama.buddhason.org/SN/sn.php?keyword=12.23}
  住在舍衛城……(中略)。 

  「\twnr{比丘}{31.0}們!我說,知者、見者有諸\twnr{漏}{188.0}的滅盡,非不知者、非不見者。比丘們!而知什麼者與見什麼者有諸漏的滅盡?[\ccchref{MN.2}{https://agama.buddhason.org/MN/dm.php?keyword=2}]『像這樣是色,像這樣是色的\twnr{集}{67.0},像這樣是色的滅沒;像這樣是受……(中略)像這樣是想……像這樣是諸行……像這樣是識,像這樣是識的集,像這樣是識的滅沒。』比丘們!這樣知者、這樣見者有諸漏的滅盡。  

  比丘們!我說凡他的那個關於滅盡的\twnr{滅盡智}{437.0}也是\twnr{有近因的}{614.1},非無近因的。比丘們!而什麼是滅盡智的近因?『解脫』應該被回答。

  比丘們!我說解脫也是有近因的,非無近因的,比丘們!而什麼是解脫的近因?『\twnr{離貪}{77.0}』應該被回答。

  比丘們!我說離貪也是有近因的,非無近因的,比丘們!而什麼是離貪的近因?『\twnr{厭}{15.0}』應該被回答。

  比丘們!我說厭也是有近因的,非無近因的,比丘們!而什麼是厭的近因?『如實\twnr{智見}{433.0}』應該被回答。

  比丘們!我說如實智見也是有近因的,非無近因的,比丘們!而什麼是如實智見的近因?『\twnr{定}{182.0}』應該被回答。

  比丘們!我說定也是有近因的,非無近因的,比丘們!而什麼是定的近因?『樂』應該被回答。

  比丘們!我說樂也是有近因的,非無近因的,比丘們!而什麼是樂的近因?『\twnr{寧靜}{313.0}』應該被回答。

  比丘們!我說寧靜也是有近因的,非無近因的,比丘們!而什麼是寧靜的近因?『\twnr{喜}{428.0}』應該被回答。

  比丘們!我說喜也是有近因的,非無近因的,比丘們!而什麼是喜的近因?『欣悅』應該被回答。

  比丘們!我說欣悅也是有近因的,非無近因的,比丘們!而什麼是欣悅的近因?『信』應該被回答。

  比丘們!我說信也是有近因的,非無近因的,比丘們!而什麼是信的近因?『苦』應該被回答。

  比丘們!我說苦也是有近因的,非無近因的,比丘們!而什麼是苦的近因?『生』應該被回答。

  比丘們!我說生也是有近因的,非無近因的,比丘們!而什麼是生的近因?『有』應該被回答。

  比丘們!我說有也是有近因的,非無近因的,比丘們!而什麼是有的近因?『取』應該被回答。

  比丘們!我說取也是有近因的,非無近因的,比丘們!而什麼是取的近因?『渴愛』應該被回答。

  比丘們!我說渴愛也是有近因的,非無近因的,比丘們!而什麼是渴愛的近因?『受』應該被回答……(中略)『觸』應該被回答……『六處』應該被回答……『名色』應該被回答……『識』應該被回答……『諸行』應該被回答。

  比丘們!我說諸行也是有近因的,非無近因的,比丘們!而什麼是諸行的近因?『\twnr{無明}{207.0}』應該被回答。

  比丘們!像這樣,以無明為近因而有行;以行為近因而有識;以識為近因而有名色;以名色為近因而有六處;以六處為近因而有觸;以觸為近因而有受;以受為近因而有渴愛;以渴愛為近因而有取;以取為近因而有有;以有為近因而有生;以生為近因而有苦;以苦為近因而有信;以信為近因而有欣悅;以欣悅為近因而有喜;以喜為近因而有寧靜;以寧靜為近因而有樂;以樂為近因而有定;以定為近因而有如實智見;以如實智見為近因而有厭;以厭為近因而有離貪;以離貪為近因而有解脫;以解脫為近因而有滅盡智。

  比丘們!猶如在天下大雨時,在有大雨的山上,那個依向下流動的水使山洞、裂縫、支流充滿,填滿(充滿)的山洞、裂縫、支流使小水池充滿,填滿的小水池使大水池充滿,填滿的大水池使小河充滿,填滿的小河使大河充滿,填滿的大河使大海洋充滿。

  同樣的,比丘們!以無明為近因而有諸行;以諸行為近因而有識;以識為近因而有名色;以名色為近因而有六處;以六處為近因而有觸;以觸為近因而有受;以受為近因而有渴愛;以渴愛為近因而有取;以取為近因而有有;以有為近因而有生;以生為近因而有苦;以苦為近因而有信;以信為近因而有欣悅;以欣悅為近因而有喜;以喜為近因而有寧靜;以寧靜為近因而有樂;以樂為近因而有定;以定為近因而有如實智見;以如實智見為近因而有厭;以厭為近因而有離貪;以離貪為近因而有解脫;以解脫為近因而有滅盡智。」



\sutta{24}{24}{其他外道遊行者經}{https://agama.buddhason.org/SN/sn.php?keyword=12.24}
  住在王舍城竹林中。

  那時,\twnr{尊者}{200.0}舍利弗午前時穿衣、拿起衣鉢後,\twnr{為了托鉢}{87.0}進入王舍城。那時,尊者舍利弗想這個:「在王舍城為了托鉢行走大致上還太早,讓我去\twnr{其他外道遊行者}{79.0}們的園林。」

  那時,尊者舍利弗去其他外道遊行者們的園林。抵達後,與其他外道遊行者一起互相問候。交換應該被互相問候的友好交談後,在一旁坐下。那些其他外道遊行者對在一旁坐下的尊者舍利弗說這個:

  「舍利弗\twnr{道友}{201.0}!有一些論說業的\twnr{沙門}{29.0}、\twnr{婆羅門}{17.0}\twnr{安立}{143.0}苦是自己作的;舍利弗道友!有一些論說業的沙門、婆羅門安立苦是其他作的;舍利弗道友!有一些論說業的沙門、婆羅門安立苦是\twnr{自己作的與其他者作的}{172.1};舍利弗道友!有一些論說業的沙門、婆羅門安立苦是非自己非其他作的;\twnr{自然生}{173.0}的,舍利弗道友!而這裡,沙門\twnr{喬達摩}{80.0}是論說什麼者?宣說什麼者?當怎樣回答時,我們才\twnr{會是沙門喬達摩的所說之說者}{115.0},而且不會以不實的誹謗沙門喬達摩,以及會\twnr{法隨法地解說}{415.0},而任何如法的種種說不會來到應該被呵責處?」

  「\twnr{道友們}{201.0}!苦是\twnr{緣所生的}{557.0}被\twnr{世尊}{12.0}說。\twnr{緣於}{252.0}什麼?\twnr{緣於觸}{407.1}。像這樣說者就會是世尊的所說之說者,而且不會以不實的誹謗世尊,會法隨法地回答,而任何如法的種種說不會來到應該被呵責處。

  道友們!在那裡,凡那些論說業的沙門、婆羅門安立苦是自己作的者,那也是緣於觸;凡那些論說業的沙門、婆羅門安立苦是其他作的者,那也是緣於觸;凡那些論說業的沙門、婆羅門安立苦是自己作的與其他者作的者,那也是緣於觸;凡那些論說業的沙門、婆羅門安立苦是非自己非其他作的;自然生的者,那也是緣於觸。

  道友們!在那裡,凡那些論說業的沙門、婆羅門安立苦是自己作的者,『他們將從觸以外感受。』\twnr{這不存在可能性}{650.0};凡那些論說業的沙門、婆羅門安立苦是其他作的者,『他們將從觸以外感受。』這不存在可能性;凡那些論說業的沙門、婆羅門安立苦是自己作的與其他者作的者,『他們將從觸以外感受。』這不存在可能性;凡那些論說業的沙門、婆羅門安立苦是非自己非其他作的;自然生的者,『他們將從觸以外感受。』這不存在可能性。」

  尊者阿難聽到尊者舍利弗與那些其他外道遊行者一起的這個交談。

  那時,尊者阿難在王舍城為了托鉢行走後,\twnr{餐後已從施食返回}{512.0},去見世尊。抵達後,向世尊\twnr{問訊}{46.0}後,在一旁坐下。在一旁坐下的尊者阿難告訴世尊尊者舍利弗與那些其他外道遊行者一起有交談之所及的那一切。

  「阿難!\twnr{好}{44.0}!好!依那個,當舍利弗回答時,會是正確地回答。阿難!苦是緣所生的被我說,緣於什麼?緣於觸。像這樣說者就會是我的所說之說者,而且不會以不實的誹謗我,會法隨法地回答,而任何如法的種種說不會來到應該被呵責處。

  阿難!在那裡,凡那些論說業的沙門、婆羅門安立苦是自己作的者,那也是緣於觸;凡那些……(中略)凡那些……(中略)凡那些論說業的沙門、婆羅門安立苦是非自己非其他作的;自然生的者,那也是緣於觸。

  阿難!在那裡,凡那些論說業的沙門、婆羅門安立苦是自己作的者,『他們將從觸以外感受。』這不存在可能性;凡那些……(中略)凡那些……(中略)凡那些論說業的沙門、婆羅門安立苦是非自己非其他作的;自然生的者,『他們將從觸以外感受。』這不存在可能性。

  阿難!有這一次,我就住在這王舍城栗鼠飼養處的竹林中。那時,阿難!我午前時穿衣、拿起衣鉢後,為了托鉢進入王舍城。阿難!那時,那個我想這個:『在王舍城為了托鉢行走大致上還太早,讓我前往其他外道遊行者們的園林。』

  阿難!那時,我去其他外道遊行者們的園林。抵達後,與其他外道遊行者一起互相問候。交換應該被互相問候的友好交談後,在一旁坐下。阿難!那些其他外道遊行者對在一旁坐下的我說這個:

  『喬達摩道友!有一些論說業的沙門、婆羅門安立苦是自己作的;喬達摩道友!有一些論說業的沙門、婆羅門安立苦是其他作的;喬達摩道友!有一些論說業的沙門、婆羅門安立苦是自己作的與其他者作的;喬達摩道友!有一些論說業的沙門、婆羅門安立苦是非自己非其他作的;自然生的,而這裡,尊者喬達摩是論說什麼者?宣說什麼者?當怎樣回答時,我們才會是沙門喬達摩的所說之說者,而且不會以不實的誹謗沙門喬達摩,會法隨法地回答,而任何如法的種種說不會來到應該被呵責處?』

  阿難!在這麼說時,我對那些其他外道遊行者說這個:『道友們!苦是緣所生的被我說,緣於什麼?緣於觸。像這樣說者就會是我的所說之說者,而且不會以不實的誹謗謗我,會法隨法地回答,而任何如法的種種說不會來到應該被呵責。

  道友們!在那裡,凡那些論說業的沙門、婆羅門安立苦是自己作的者,那也是緣於觸;凡那些……(中略)凡那些……(中略)凡那些論說業的沙門、婆羅門安立苦是非自己非其他作的;自然生的者,那也是緣於觸。

  道友們!在那裡,凡那些論說業的沙門、婆羅門安立苦是自己作的者,「他們將從觸以外感受。」這不存在可能性;凡那些……(中略)凡那些……(中略)凡那些論說業的沙門、婆羅門安立苦是非自己非其他作的;自然生的者,「他們將從觸以外感受。」這不存在可能性。』」

  「\twnr{不可思議}{206.0}啊,\twnr{大德}{45.0}!\twnr{未曾有}{206.0}啊,大德!確實是因為全部的道理被一句話說了,大德!就這個道理如果被詳細說時,會有甚深的與顯現甚深的,不是嗎?」

  「阿難!那樣的話,就這情況請你說明。」

  「大德!如果他們這麼問我:『阿難\twnr{學友}{201.0}!老死,什麼為因?什麼為集?什麼生的?什麼為根源?』大德!被這麼問,我會這麼回答:『學友們!老死,生為因,生為集,生生的,生為根源。』大德!被這麼問,我會這麼回答。

  大德!如果他們這麼問我:『阿難學友!那麼,生,什麼為因?什麼為集?什麼生的?什麼為根源?』大德!被這麼問,我會這麼回答:『學友們!生,有為因,有為集,有生的,有為根源。』大德!被這麼問,我會這麼回答。

  大德!如果他們這麼問我:『阿難學友!那麼,有,什麼為因?什麼為集?什麼生的?什麼為根源?』大德!被這麼問,我會這麼回答:『學友們!有,取為因,取為集,取生的,取為根源。』大德!被這麼問,我會這麼回答。

  大德!如果他們這麼問我:『阿難學友!那麼,取……(中略)阿難學友!那麼,渴愛……(中略)阿難學友!那麼,受……(中略)。』大德!如果他們這麼問我:『阿難學友!那麼,觸,什麼為因?什麼為集?什麼生的?什麼為根源?』大德!被這麼問,我會這麼回答:『學友們!觸,六處為因,六處為集,六處生的,六處為根源。』

  學友們!但就以\twnr{六觸處}{78.0}的\twnr{無餘褪去與滅}{491.0}有觸滅(而觸滅存在);以觸滅有受滅;以受滅有渴愛滅;以渴愛滅有取滅;以取滅有有滅;以有滅有生滅;以生滅而老、死、愁、悲、苦、憂、\twnr{絕望}{342.0}被滅,這樣是這整個苦蘊的滅。』大德!被這麼問,我會這麼回答。」



\sutta{25}{25}{地生經}{https://agama.buddhason.org/SN/sn.php?keyword=12.25}
  住在舍衛城。

  那時,\twnr{尊者}{200.0}地生傍晚時,從\twnr{獨坐}{92.0}出來,去見尊者舍利弗。抵達後,與尊者舍利弗一起互相問候。交換應該被互相問候的友好交談後,在一旁坐下。在一旁坐下的尊者地生對尊者舍利弗說這個:

  「舍利弗\twnr{學友}{201.0}!
有一些論說業的\twnr{沙門}{29.0}、\twnr{婆羅門}{17.0}\twnr{安立}{143.0}苦樂(樂、苦)是自己作的;舍利弗學友!有一些論說業的沙門、婆羅門安立苦樂是其他作的;舍利弗學友!有一些論說業的沙門、婆羅門安立苦樂是\twnr{自己作的與其他者作的}{172.1};舍利弗學友!有一些論說業的沙門、婆羅門安立苦樂是非自己非其他作的;\twnr{自然生}{173.0}的,舍利弗學友!而這裡,\twnr{世尊}{12.0}是論說什麼者?宣說什麼者?當怎樣回答時,我們才\twnr{會是世尊的所說之說者}{115.0},而且不會以不實的誹謗世尊,會\twnr{法隨法地回答}{415.0},而任何如法的種種說不會來到應該被呵責處?

  「學友!苦樂是\twnr{緣所生的}{557.0}被\twnr{世尊}{12.0}說。\twnr{緣於}{252.0}什麼?\twnr{緣於觸}{407.1}。像這樣說者就\twnr{會是世尊的所說之說者}{115.0},而且不會以不實的誹謗世尊,會法隨法地回答,而任何如法的種種說不會來到應該被呵責處。

  學友!在那裡,凡那些論說業的沙門、婆羅門安立苦樂是自己作的者,那也是緣於觸;凡那些……(中略)凡那些……(中略)凡那些論說業的沙門、婆羅門安立苦樂是非自己非其他作的;自然生的者,那也是緣於觸。

  學友!在那裡,凡那些論說業的沙門、婆羅門安立苦樂是自己作的者,『他們將從觸以外感受。』\twnr{這不存在可能性}{650.0};凡那些……(中略)凡那些……(中略)凡那些論說業的沙門、婆羅門安立苦樂是非自己非其他作的;自然生的者,『他們將從觸以外感受。』這不存在可能性。」

  尊者阿難聽到尊者舍利弗與尊者地生一起的這個交談。

  那時,尊者阿難去見世尊。抵達後,向世尊\twnr{問訊}{46.0}後,在一旁坐下。在一旁坐下的尊者阿難告訴世尊尊者舍利弗與尊者地生一起有交談之所及的那一切。

  「阿難!\twnr{好}{44.0}!好!依那個,當舍利弗回答時,會是正確地回答。阿難!苦樂是緣所生的被我說,緣於什麼?緣於觸。像這樣說者就會是我的所說之說者,而且不會以不實的誹謗我,會法隨法地回答,而任何如法的種種說不會來到應該被呵責處。

  阿難!在那裡,凡那些論說業的沙門、婆羅門安立苦樂是自己作的者,那也是緣於觸;凡那些……(中略)凡那些……(中略)凡那些論說業的沙門、婆羅門安立苦樂是非自己非其他作的;自然生的者,那也是緣於觸。

  阿難!在那裡,凡那些論說業的沙門、婆羅門安立苦樂是自己作的者,『他們將從觸以外感受。』這不存在可能性;凡那些……(中略)凡那些……(中略)凡那些論說業的沙門、婆羅門安立苦樂是非自己非其他作的;自然生的者,『他們將從觸以外感受。』這不存在可能性。

  阿難!在有身體時,\twnr{身故思之因}{x251},自身內的苦樂生起;比丘們!或在有言語時,言語故思之因,自身內的苦樂生起;比丘們!或在有意時,意故思之因,自身內的苦樂生起,就以無明\twnr{為緣}{180.0}。

  阿難!自己造作那個身行,以那個為緣,他的那個自身內的苦樂生起。比丘們!或諸他人造作他的那個身行,以那個為緣,他的那個自身內的苦樂生起。比丘們!或故意者(正知者)造作那個身行,以那個為緣,他的那個自身內的苦樂生起。比丘們!或非故意者造作那個身行,以那個為緣,他的那個自身內的苦樂生起。

  阿難!自己造作那個語行,以那個為緣,他的那個自身內的苦樂生起,比丘們!或諸他人造作他的那個語行,以那個為緣,他的那個自身內的苦樂生起,比丘們!或故意者造作那個語行,以那個為緣,他的那個自身內的苦樂生起。比丘們!或非故意者造作那個語行,以那個為緣,他的那個自身內的苦樂生起。

  阿難!自己造作那個\twnr{意行}{230.0},以那個為緣,他的那個自身內的苦樂生起。比丘們!或諸他人造作他的那個意行,以那個為緣,他的那個自身內的苦樂生起。比丘們!或故意者造作那個意行,以那個為緣,他的那個自身內的苦樂生起。比丘們!或非故意者造作那個意行,以那個為緣,他的那個自身內的苦樂生起。

  阿難!無明已降落在這些法中。但就以無明的無餘褪去與滅,那個身體不存在:以那個為緣,他的那個自身內的苦樂生起;那個言語不存在:以那個為緣,他的那個自身內的苦樂生起;那個意不存在:以那個為緣,他的那個自身內的苦樂生起……(中略)那個\twnr{田}{x252}不存在……(中略)那個\twnr{地}{x253}不存在……(中略)那個\twnr{處}{x254}不存在;那個\twnr{事件}{x255}不存在:以那個為緣,他的那個自身內的苦樂生起。」



\sutta{26}{26}{優波梵那經}{https://agama.buddhason.org/SN/sn.php?keyword=12.26}
  住在舍衛城。 

  那時,\twnr{尊者}{200.0}優波梵那去見世尊。抵達後,向世尊\twnr{問訊}{46.0}後,在一旁坐下。在一旁坐下的尊者優波梵那對世尊說這個:

  「\twnr{大德}{45.0}!有一些論說業的\twnr{沙門}{29.0}、\twnr{婆羅門}{17.0}\twnr{安立}{143.0}苦是自己作的;大德!有一些論說業的沙門、婆羅門安立苦是其他作的;大德!有一些論說業的沙門、婆羅門安立苦是\twnr{自己作的與其他者作的}{172.1};大德!有一些論說業的沙門、婆羅門安立苦是非自己非其他作的;\twnr{自然生}{173.0}的,大德!而這裡,\twnr{世尊}{12.0}是論說什麼者?宣說什麼者?當怎樣回答時,我們才會是世尊的所說之說者,而且不會以不實的誹謗世尊,會\twnr{法隨法地回答}{415.0},而任何如法的種種說不會來到應該被呵責處?」

  「優波梵那!苦是\twnr{緣所生的}{557.0}被我說。緣於什麼?\twnr{緣於觸}{407.1}。像這樣說者就會是我的所說之說者,而且不會以不實的誹謗我,會法隨法地回答說,而任何如法的種種說不會來到應該被呵責處。

  優波梵那!在那裡,凡那些沙門、婆羅門安立苦是自己作的者,那也是緣觸;凡那些……(中略)凡那些……(中略)凡那些沙門、婆羅門安立苦是非自己非其他作的;自然生的者,那也是緣觸。

  優波梵那!在那裡,凡那些沙門、婆羅門安立苦是自己作的者,『他們將從觸以外感受。』\twnr{這不存在可能性}{650.0};凡那些……(中略)凡那些……(中略)凡那些沙門、婆羅門安立苦是非自己非其他作的;自然生的者,『他們將從觸以外感受。』這不存在可能性。」



\sutta{27}{27}{緣經}{https://agama.buddhason.org/SN/sn.php?keyword=12.27}
  住在舍衛城。……(中略)  

  「\twnr{比丘}{31.0}們!以\twnr{無明}{207.0}為緣有諸行(而諸行存在);以行\twnr{為緣}{180.0}有識……(中略)這樣是這整個\twnr{苦蘊}{83.0}的\twnr{集}{67.0}。

  比丘們!而什麼是老死?凡一一那些眾生中,以一一那個眾生部類的老、老衰、齒落、髮白、皮皺、壽命的衰退、諸根的退化,這被稱為老。凡一一那些眾生中,以一一那個眾生部類的過世、滅亡、崩解、消失、死亡、壽終、諸蘊的崩解、屍體的捨棄[、\twnr{命根斷絕}{445.1}-\ccchref{MN.9}{https://agama.buddhason.org/MN/dm.php?keyword=9}, 92段],這被稱為死。像這樣,這個老與這個死,比丘們!這被稱為老死。以生集而有老死集,以生滅有老死滅,這\twnr{八支聖道}{525.0}就是導向老死\twnr{滅道跡}{69.0},即:正見、正志、正語、正業、正命、正精進、正念、正定。

  比丘們!而什麼是生?……(中略)比丘們!而什麼是有?……比丘們!而什麼是取?……比丘們!而什麼是渴愛?……比丘們!而什麼是受?……比丘們!而什麼是觸?……比丘們!而什麼是六處?……比丘們!而什麼是名色?……比丘們!而什麼是識?……比丘們!而什麼是諸行?比丘們!有這些三行:身行、語行、心行,比丘們!這些被稱為諸行。以無明集而有行集,以無明滅有行滅,這八支聖道就是導向行滅道跡,即:正見……(中略)正定。

  比丘們!當\twnr{聖弟子}{24.0}這麼知道緣,這麼知道緣集,這麼知道緣滅,這麼知道導向緣滅道跡,比丘們!這位聖弟子被稱為『\twnr{見具足者}{575.0}』,及『\twnr{看見具足者}{572.0}』,及『來到這正法者』,及『他看見這正法』,及『具備\twnr{有學}{193.0}之智者』,及『具備有學之明者』,及『進入法流者』,及『\twnr{洞察慧}{566.0}之聖者』,及『他敲打\twnr{不死}{123.0}之門後住立』。」



\sutta{28}{28}{比丘經}{https://agama.buddhason.org/SN/sn.php?keyword=12.28}
  住在舍衛城……(中略)在那裡……(中略)。

  「比丘們!這裡,\twnr{比丘}{31.0}知道老死,知道老死集,知道老死滅,知道導向老死\twnr{滅道跡}{69.0};知道生……(中略)知道有……知道取……知道渴愛……知道受……知道觸……知道六處……知道名色……知道識……知道諸行,知道行集,知道行滅,知道導向行滅道跡。

  比丘們!而什麼是老死?凡一一那些眾生中,以一一那個眾生部類的老、老衰、齒落、髮白、皮皺、壽命的衰退、諸根的退化,這被稱為老。凡一一那些眾生中,以一一那個眾生部類的過世、滅亡、崩解、消失、死亡、壽終、諸蘊的崩解、屍體的捨棄[、\twnr{命根斷絕}{445.1}-\ccchref{MN.9}{https://agama.buddhason.org/MN/dm.php?keyword=9}, 92段],這被稱為死。像這樣,這個老與這個死,比丘們!這被稱為老死。以生集而有老死集(而老死集存在),以生滅有老死滅,這\twnr{八支聖道}{525.0}就是導向老死\twnr{滅道跡}{69.0},即:正見……(中略)正定。

  比丘們!而什麼是生?……(中略)比丘們!而什麼是有?……比丘們!而什麼是取?[……渴愛]……受……觸……六處……名色……識……比丘們!而什麼是諸行?比丘們!有這些三行:身行、語行、心行,比丘們!這些被稱為諸行。以無明集而有行集,以無明滅有行滅,這八支聖道就是導向行滅道跡,即:正見……(中略)正定。

  比丘們!當比丘這麼知道老死,這麼知道老死集,這麼知道老死滅,這麼知道導向老死滅道跡;這麼知道生……(中略)有……取……渴愛……受……觸……六處……名色……識……諸行……行集……行滅,這麼知道導向行滅道跡,比丘們!這位比丘被稱為『\twnr{見具足者}{575.0}』,及『\twnr{看見具足者}{572.0}』,及『來到這正法者』,及『他看見這正法』,及『具備\twnr{有學}{193.0}之智者』,及『具備有學之明者』,及『進入法流者』,及『\twnr{洞察慧}{566.0}之聖者』,及『他敲打\twnr{不死}{123.0}之門後住立』。」



\sutta{29}{29}{沙門婆羅門經}{https://agama.buddhason.org/SN/sn.php?keyword=12.29}
  住在舍衛城……(中略)在那裡……(中略)。

  「\twnr{比丘}{31.0}們!凡任何\twnr{沙門}{29.0}或\twnr{婆羅門}{17.0}不知道老死,不知道老死集,不知道老死滅,不知道導向老死\twnr{滅道跡}{69.0};不知道生……(中略)有……取……渴愛……受……觸……六處……名色……識……不知道諸行,不知道行集,不知道行滅,不知道導向行滅道跡者,比丘們!那些沙門或婆羅門不被我認同為\twnr{沙門中的沙門}{560.0},或婆羅門中的婆羅門,而且,那些\twnr{尊者}{200.0}也不以證智自作證後,在當生中\twnr{進入後住於}{66.0}\twnr{沙門義}{327.0}或婆羅門義。

  比丘們!而凡任何沙門或婆羅門知道老死,知道老死集,知道老死滅,知道導向老死滅道跡;知道生……(中略)有……取……渴愛……受……觸……六處……名色……識……知道行,知道行集,知道行滅,知道導向行滅道跡者,比丘們!那些沙門或婆羅門被我認同為沙門中的沙門,或婆羅門中的婆羅門,而且,那些尊者也以證智自作證後,在當生中進入後住於沙門義或婆羅門義。」



\sutta{30}{30}{沙門婆羅門經第二}{https://agama.buddhason.org/SN/sn.php?keyword=12.30}
  住在舍衛城……(中略)在那裡……(中略)。

  「\twnr{比丘}{31.0}們!凡任何\twnr{沙門}{29.0}或\twnr{婆羅門}{17.0}不知道老死,不知道老死集,不知道老死滅,不知道導向老死\twnr{滅道跡}{69.0}者,『他們將\twnr{超越}{x256}老死後住立』,\twnr{這不存在可能性}{650.0};不知道生……(中略)有……取……渴愛……受……觸……六處……名色……識……不知道諸行,不知道行集,不知道行滅,不知道導向行滅道跡者,『他們將超越諸行後住立』,這不存在可能性。

  比丘們!而凡任何沙門或婆羅門知道老死,知道老死集,知道老死滅,知道導向老死滅道跡者,『他們將超越老死後住立』,這存在可能性;知道生……(中略)有……取……渴愛……受……觸……六處……名色……識……知道諸行,知道行集,知道行滅,知道導向行滅道跡者,『他們將超越行後住立』,這存在可能性。」

  十力品第三,其\twnr{攝頌}{35.0}:

  「十力二則以及近因,其他外道遊行者、地生,

   優波梵那、緣、比丘,以及沙門婆羅門二則。」





\pin{剎帝利黑齒品}{31}{40}
\sutta{31}{31}{已生成的經}{https://agama.buddhason.org/SN/sn.php?keyword=12.31}
  \twnr{有一次}{2.0},\twnr{世尊}{12.0}住在舍衛城。

  在那裡,世尊召喚\twnr{尊者}{200.0}舍利弗:

  「舍利弗!這個在〈\twnr{波羅延阿逸多所問}{x257}〉中被說:

  『\twnr{凡法的察悟者}{x258},以及這裡\twnr{凡個個有學}{x259},

   賢明者被我詢問他們的舉止時,\twnr{親愛的先生}{204.0}!請你講述。』[\ccchref{Ni.18}{https://agama.buddhason.org/Ni/dm.php?keyword=18}, 第7段]

  舍利弗!對這個簡要地說的義理,應該怎樣詳細地被看見呢?」

  在這麼說時,尊者舍利弗保持沈默。

  第二次,世尊又召喚尊者舍利弗……(中略)。

  第二次,尊者舍利弗又保持沈默。

  第三次,世尊又召喚尊者舍利弗說:

  「舍利弗!這個在〈波羅延阿逸多所問〉中被說:

  『凡法的察悟者,以及這裡凡個個有學,

   賢明者被我詢問他們的舉止時,親愛的先生!請你講述。』

  舍利弗!對這個簡要地說的義理,應該怎樣詳細地被看見呢?」

  第三次,尊者舍利弗又保持沈默。

  「舍利弗!你看見『\twnr{這是已生成的}{x260}』嗎?」

  「\twnr{大德}{45.0}!他以正確之慧如實看見『這是已生成的』;以正確之慧如實看見『這是已生成的』後,他對已生成的是為了\twnr{厭}{15.0}、\twnr{離貪}{77.0}、\twnr{滅的行者}{519.0}。

  他以正確之慧如實看見『那個是食的生成』;以正確之慧如實看見『那個是食的生成』後,他對食的生成是為了厭、離貪、滅的行者。

  他以正確之慧如實看見『以那個食的滅凡已生成的成為\twnr{滅法}{68.1}』;以正確之慧如實看見『以那個食的滅凡已生成的成為滅法』後,他對滅法是為了厭、離貪、滅的行者。

  大德!這樣是\twnr{有學}{193.0}。

  大德!而怎樣是法的察悟者?

  大德!他以正確之慧如實看見『這是已生成的』;以正確之慧如實看見『這是已生成的』後,他對已生成的從厭、離貪、滅,不執取後成為解脫者。

  他以正確之慧如實看見『那個是食的生成』;以正確之慧如實看見『那個是食的生成』後,他對食的生成從厭、離貪、滅,不執取後成為解脫者。

  他以正確之慧如實看見『以那個食的滅凡已生成的成為滅法』;以正確之慧如實看見『以那個食的滅凡已生成的成為滅法』後,他對滅法從厭、離貪、滅,不執取後成為解脫者。

  大德!這樣是法的察悟者。

  大德!這樣是在〈波羅延阿逸多所問〉中所說:

  『凡法的察悟者,以及這裡,凡個個有學,

   賢明者被我詢問他們的舉止時,親愛的先生!請你講述。』

  大德!我這樣詳細地了知這個以簡要所說的義理。」

  「舍利弗!好!好!舍利弗!他以正確之慧如實看見『這是已生成的』;以正確之慧如實看見『這是已生成的』後,他對已生成的是為了厭、離貪、滅的行者。

  他以正確之慧如實看見『那個是食的生成』;以正確之慧如實看見『那個是食的生成』後,他對食的生成是為了厭、離貪、滅的行者。

  他以正確之慧如實看見『以那個食的滅凡已生成的成為滅法』;以正確之慧如實看見『以那個食的滅凡已生成的成為滅法』後,他對滅法是為了厭、離貪、滅的行者。

  舍利弗!這樣是有學。

  舍利弗!而怎樣是法的察悟者?

  舍利弗!他以正確之慧如實看見『這是已生成的』;以正確之慧如實看見『這是已生成的』後,他對已生成的從厭、離貪、滅,不執取後成為解脫者。

  他以正確之慧如實看見『那個是食的生成』;以正確之慧如實看見『那個是食的生成』後,他對食的生成從厭、離貪、滅,不執取後成為解脫者。

  他以正確之慧如實看見『以那個食的滅凡已生成的成為滅法』;以正確之慧如實看見『以那個食的滅凡已生成的成為滅法』後,他對滅法從厭、離貪、滅,不執取後成為解脫者。

  舍利弗!這樣是法的察悟者。

  舍利弗!這樣是在〈波羅延阿逸多所問〉中所說:

  『凡法的察悟者,以及這裡,凡個個有學,

   賢明者被我詢問他們的舉止時,親愛的先生!請你講述。』

  舍利弗!對這個被簡要地說的,義理應該這樣被詳細地看見。」



\sutta{32}{32}{黑齒經}{https://agama.buddhason.org/SN/sn.php?keyword=12.32}
  住在舍衛城。

  那時,\twnr{剎帝利}{116.0}黑齒\twnr{比丘}{31.0}去見\twnr{尊者}{200.0}舍利弗。抵達後,與尊者舍利弗一起互相問候。交換應該被互相問候的友好交談後,在一旁坐下。在一旁坐下的剎帝利黑齒比丘對尊者舍利弗說這個:

  「舍利弗\twnr{學友}{201.0}!摩利亞帕辜那比丘放棄學後還俗。」「因為那位尊者確實沒在這法與律中得到穌息。」「那樣的話,尊者舍利弗在這法與律中已達到穌息?」

  「學友!我不懷疑。」

  「學友!那麼,未來?」

  「學友!我不疑惑。」

  那時,剎帝利黑齒比丘從座位起來後去見世尊。抵達後,向世尊\twnr{問訊}{46.0}後,在一旁坐下。在一旁坐下的剎帝利黑齒比丘對世尊說這個:

  「\twnr{大德}{45.0}!\twnr{完全智}{489.0}被尊者舍利弗\twnr{記說}{179.0}:『我了知:「\twnr{出生已盡}{18.0},\twnr{梵行已完成}{19.0},\twnr{應該被作的已作}{20.0},\twnr{不再有此處[輪迴]的狀態}{21.1}。」』」

  那時,世尊召喚某位比丘:「來!比丘!請你以我的名義召喚舍利弗:『舍利弗學友!\twnr{大師}{145.0}召喚你。』」

  「是的,大德!」那位比丘回答世尊後,去見尊者舍利弗。抵達後,對尊者舍利弗說這個:「舍利弗學友!大師召喚你。」

  「是的,學友!」尊者舍利弗回答那位比丘後,去見世尊。抵達後,向世尊問訊後,在一旁坐下。世尊對在一旁坐下的尊者舍利弗說這個:

  「舍利弗!傳說是真的?完全智被你記說:『我了知:「出生已盡,梵行已完成,應該被作的已作,不再有此處[輪迴]的狀態。」』」

  「大德!事情不被這些詞句、這些文句說。」

  「舍利弗!凡\twnr{善男子}{41.0}以任何方式記說完全智,那時,所記說的應該從記說被看見。」

  「大德!我也說這個:『大德!事情不被這些詞句、這些文句說。』不是嗎?」

  「舍利弗!如果他們這麼問你:『舍利弗學友!當怎樣知又怎樣見時,完全智被你記說:「我了知:『出生已盡,梵行已完成,應該被作的已作,不再有此處[輪迴]的狀態。』呢?」』舍利弗!被這麼問,你應該怎麼回答?」

  「大德!如果他們這麼問我:『舍利弗學友!當怎樣知又怎樣見時,完全智被你記說:「我了知:『出生已盡,梵行已完成,應該被作的已作,不再有此處[輪迴]的狀態。』呢?」』大德!被這麼問,我應該這麼回答:『學友!凡出生的因緣,以那個因緣的滅盡,在已盡時,「\twnr{我是已盡者}{x261}」被知道,知道「我是已盡者」後,我了知:「出生已盡,梵行已完成,應該被作的已作,不再有此處[輪迴]的狀態。」』大德!被這麼問,我應該這麼回答。」

  「舍利弗!那麼,如果他們這麼問你:『舍利弗學友!那麼,生,什麼為因?什麼為集?什麼生的?\twnr{什麼為根源}{668.0}?』舍利弗!被這麼問,你應該怎麼回答?」

  「大德!如果我被這麼問:『舍利弗學友!那麼,生,什麼為因?什麼為集?什麼生的?什麼為根源?』大德!被這麼問,我應該這麼回答:『學友!生,有為因,有為集,有生的,有為根源。』大德!被這麼問,我應該這麼回答。」

  「舍利弗!那麼,如果他們這麼問你:『舍利弗學友!那麼,有,什麼為因?什麼為集?什麼生的?什麼為根源?』舍利弗!被這麼問,你應該怎麼回答?」

  「大德!如果我被這麼問:『舍利弗學友!那麼,有,什麼為因?什麼為集?什麼生的?什麼為根源?』大德!被這麼問,我應該這麼回答:『學友!有,取為因,取為集,取生的,取為根源。』大德!被這麼問,我應該這麼回答。」

  「舍利弗!那麼,如果他們這麼問你:『舍利弗學友!那麼,取……(中略)』」

  「舍利弗!那麼,如果他們這麼問你:『舍利弗學友!那麼,渴愛,什麼為因?什麼為集?什麼生的?什麼為根源?』舍利弗!被這麼問,你應該怎麼回答?」

  「大德!如果我被這麼問:『舍利弗學友!那麼,渴愛,什麼為因?什麼為集?什麼生的?什麼為根源?』大德!被這麼問,我應該這麼回答:『學友!渴愛,受為因,受為集,受生的,受為根源。』大德!被這麼問,我應該這麼回答。」

  「舍利弗!那麼,如果他們這麼問你:『舍利弗學友!那麼,當你怎樣知又怎樣見時,在諸受上的歡喜它不出現?』舍利弗!被這麼問,你應該怎麼回答?」

  「大德!如果我被這麼問:『舍利弗學友!那麼,當你怎樣知又怎樣見時,在諸受上的歡喜它不出現?』大德!被這麼問,我應該這麼回答:『學友!有這些三受,哪三個?樂受、苦受、不苦不樂受。學友!這三受是無常的,「凡是無常的,那個是苦的」被知道,凡在諸受上的歡喜它不出現。』大德!被這麼問,我應該這麼回答。」

  「舍利弗!好!好!舍利弗!這也是就這件事情以簡要記說的方式:『凡任何受,它是在苦中。』

  舍利弗!那麼,如果他們這麼問你:『舍利弗學友!那麼,從怎樣的解脫而完全智被你記說:「出生已盡,梵行已完成,應該被作的已作,不再有此處[輪迴]的狀態。」呢?』舍利弗!被這麼問,你應該怎麼回答?」

  「大德!如果我被這麼問:『舍利弗學友!那麼,從怎樣的解脫而完全智被你記說:「出生已盡,梵行已完成,應該被作的已作,不再有此處[輪迴]的狀態。」呢?』大德!被這麼問,我應該這麼回答:『學友!\twnr{從自身內的解脫}{x262},從一切執取的滅盡,我像這樣具念地住,當那樣具念地住時,諸\twnr{漏}{188.0}不隨流出,而且我不輕蔑自己。』大德!被這麼問,我應該這麼回答。」

  「舍利弗!好!好!舍利弗!這也是就這件事情以簡要記說的方式:『凡被\twnr{沙門}{29.0}說的諸漏,在那些上我不懷疑,「那些被我捨斷」我不疑惑。』」

  世尊說這個。世尊說這個後,從座位起來後進入住處。

  在那裡,當世尊離開不久,尊者舍利弗召喚比丘們:

  「學友們!世尊對我問之前沒經歷的第一個問題,那個我有遲鈍。學友們!但當世尊對我歡喜第一個問題,學友們!那個我想這個:『即使世尊對我整日以各種詞句、\twnr{各種法門}{82.1}問這件事情,我對世尊也整日以各種詞句、各種法門回答這件事情;即使世尊對我整夜以各種詞句、各種法門問這件事情,我對世尊也整夜以各種詞句、各種法門回答這件事情;即使世尊對我整日夜以各種詞句、各種法門問這件事情,我對世尊也整日夜以各種詞句、各種法門回答這件事情;即使世尊對我二日夜……問這件事情……(中略)即使世尊對我三日夜……問這件事情……(中略)即使世尊對我四日夜……問這件事情……(中略)即使世尊對我五日夜……問這件事情……(中略)即使世尊對我六日夜……問這件事情……(中略)即使世尊對我七日夜以各種詞句、各種法門問這件事情,我對世尊也七日夜以各種詞句、各種法門回答這件事情。』」

  那時,剎帝利黑齒比丘從座位起來後去見世尊。抵達後,向世尊問訊後,在一旁坐下。在一旁坐下的剎帝利黑齒比丘對世尊說這個:

  「大德!尊者舍利弗吼獅子吼:『學友們!世尊對我問之前沒經歷的第一個問題,那個我有遲鈍。學友們!但當世尊對我歡喜第一個問題,學友們!那個我想這個:「即使世尊對我整日以各種詞句、各種法門問我這件事情,我也對世尊整日以各種詞句、各種法門回答這件事情;即使整夜……(中略)即使世尊整日夜對我……(中略)即使世尊兩日夜對我……(中略)三……四……五……六……(中略)即使世尊對我七天七夜以各種詞句、各種法門問這件事情,我也對世尊七天七夜以各種詞句、各種法門回答這件事情。」』」

  「比丘!舍利弗的那個\twnr{法界}{547.0}已被善通達,以該法界已被善通的狀態,即使我對舍利弗整日以各種詞句、各種法門問這件事情,舍利弗對我也整日以各種詞句、各種法門回答這件事情;即使我整夜對舍利弗以各種詞句、各種法門問這件事情,舍利弗對我也整夜……(中略)回答這件事情;即使我對舍利弗整日夜……問這件事情,舍利弗對我也整日夜……回答這件事情;即使我對舍利弗二日夜……問這件事情,舍利弗對我也二日夜……回答這件事情;即使我對舍利弗三日夜……問這件事情,舍利弗對我也三日夜……回答這件事情;即使我對舍利弗四日夜……問這件事情,舍利弗對我也四日夜……回答這件事情;即使我對舍利弗五日夜……問這件事情,舍利弗對我也五日夜……回答這件事情;即使我對舍利弗六日夜……問這件事情,舍利弗對我也六日夜……回答這件事情;即使我七日夜以各種詞句、各種法門問舍利弗這件事情,舍利弗對我也七日夜以各種詞句、各種法門回答這件事情。」



\sutta{33}{33}{智之事經}{https://agama.buddhason.org/SN/sn.php?keyword=12.33}
  在舍衛城……(中略)。

  「\twnr{比丘}{31.0}們!我將為你們教導四十四智之事,你們要聽它!你們要\twnr{好好作意}{43.1}!我將說。」

  「是的,\twnr{大德}{45.0}!」那些比丘回答\twnr{世尊}{12.0}。

  世尊說這個:

  「比丘們!而什麼是四十四智之事?

  在老死上的智、在老死集上的智、在老死滅上的智、在導向老死\twnr{滅道跡}{69.0}上的智,在生上的智、在生集上的智、在生滅上的智、在導向生滅道跡上的智,在有上的智、在有集上的智、在有滅上的智、在導向有滅道跡上的智,在取上的智、在取集上的智、在取滅上的智、在導向取滅道跡上的智,在渴愛上的智、在渴愛集上的智、在渴愛滅上的智、在導向渴愛滅道跡上的智,在受上的智、在受集上的智、在受滅上的智、在導向受滅道跡上的智,在觸上的智……(中略)在六處上的智……在名色上的智……識上的智……在諸行上的智、在行集上的智、在行滅上的智、在導向行滅道跡上的智。比丘們!這些被稱為四十四智之事。

  比丘們!而什麼是老死?凡一一那些眾生中,以一一那個眾生部類的老、老衰、齒落、髮白、皮皺、壽命的衰退、諸根的退化,這被稱為老。凡一一那些眾生中,以一一那個眾生部類的過世、滅亡、崩解、消失、死亡、壽終、諸蘊的崩解、屍體的捨棄[、\twnr{命根斷絕}{445.1}-\ccchref{MN.9}{https://agama.buddhason.org/MN/dm.php?keyword=9}, 92段],這被稱為死。像這樣,這個老與這個死,比丘們!這被稱為老死。

  以生集而有老死集(而老死集存在),以生滅有老死滅,這\twnr{八支聖道}{525.0}就是導向老死\twnr{滅道跡}{69.0},即:正見……(中略)正定。

  比丘們!當\twnr{聖弟子}{24.0}這樣知道老死,這樣知道老死集,這樣知道老死滅,這樣知道導向老死滅道跡,這是他的\twnr{法智}{984.0}。他以已見、已知、已即時獲得、已深入這個法而\twnr{導}{x263}此於過去、未來:

  『凡過去世的\twnr{沙門}{29.0}或\twnr{婆羅門}{17.0}曾\twnr{證知}{242.0}老死,曾證知老死集,曾證知老死滅,曾證知導向老死滅道跡者,他們就曾這麼證知一切,猶如現在的我。

  凡\twnr{未來世}{308.0}的沙門或婆羅門將證知老死,將證知老死集,將證知老死滅,將證知導向老死滅道跡者,他們就將這麼證知一切,猶如現在的我。』這是他的\twnr{類比智}{985.0}。

  比丘們!當聖弟子的這二種智:法智與類比智,成為遍純淨的、淨化的,比丘們!這位聖弟子被稱為『\twnr{見具足者}{575.0}』,及『\twnr{看見具足者}{572.0}』,及『來到這正法者』,及『他看見這正法』,及『具備\twnr{有學}{193.0}之智者』,及『具備有學之明者』,及『進入法流者』,及『\twnr{洞察慧}{566.0}之聖者』,及『他敲打\twnr{不死}{123.0}之門後住立』。

  比丘們!而什麼是生?……(中略)比丘們!而什麼是有?……比丘們!而什麼是取?……比丘們!而什麼是渴愛?……比丘們!而什麼是受?……比丘們!而什麼是觸?……比丘們!而什麼是六處?……比丘們!而什麼是名色?……比丘們!而什麼是識?……比丘們!而什麼是諸行?比丘們!有這些三行:身行、語行、心行,比丘們!這些被稱為諸行。

  以\twnr{無明}{207.0}集而有行集,以無明滅有行滅,這八支聖道就是導向行滅道跡,即:正見……(中略)正定。

  比丘們!當聖弟子這麼知道諸行,這麼知道行集,這麼知道行滅,這麼知道導向行滅道跡,這是他的法智。他以已見、已知、已即時獲得、已深入此法而導此於過去與未來:

  『凡過去世的沙門或婆羅門曾證知諸行,曾證知行集,曾證知行滅,曾證知導向行滅道跡者,他們就曾這麼證知一切,猶如現在的我。

  凡未來世的沙門或婆羅門將證知諸行,將證知行集,將證知行滅,將證知導向行滅道跡者,他們就將這麼證知一切,猶如現在的我。』這是他的類比智。

  比丘們!當聖弟子的這二種智:法智與類比智,成為遍純淨的、淨化的,比丘們!這位聖弟子被稱為『見具足者』,及『看見具足者』,及『來到這正法者』,及『他看見這正法』,及『具備有學之智者』,及『具備有學之明者』,及『進入法流者』,及『洞察慧之聖者』,及『他敲打不死之門後住立』。」



\sutta{34}{34}{智之事經第二}{https://agama.buddhason.org/SN/sn.php?keyword=12.34}
  住在舍衛城……(中略)。

  「\twnr{比丘}{31.0}們!我將為你們教導七十七智之事,你們要聽它!你們要\twnr{好好作意}{43.1}!我將說。」

  「是的,\twnr{大德}{45.0}!」那些比丘回答\twnr{世尊}{12.0}。

  世尊說這個:

  「比丘們!而什麼是七十七智之事?『以生\twnr{為緣}{180.0}有老死(而老死存在)』智、『在生不存在時老死不存在』智,過去世『以生為緣有老死』智、『在生不存在時老死不存在』智,未來世『以生為緣有老死』智、『在生不存在時老死不存在』智,『凡他的那個\twnr{法住智}{634.0}也是\twnr{滅盡法}{273.1}、\twnr{消散法}{155.0}、\twnr{褪去}{77.0}法、\twnr{滅法}{68.1}』\twnr{智}{x264}。

  『以有為緣有生』智……(中略)『以取為緣有有』智……『以渴愛為緣有取』智……『以受為緣有渴愛』智……『以觸為緣有受』智……『以六處為緣有觸』智……『以名色為緣有六處』智……『以識為緣有名色』智……『以行為緣有識』智……『以無明為緣有諸行』智、『在無明不存在時諸行不存在』智,過去世『以無明為緣有諸行』智、『在無明不存在時諸行不存在』智,未來世『以無明為緣有諸行』智、『在無明不存在時諸行不存在』智,『凡他的那個法住智也是滅盡法、消散法、褪去法、滅法』智,比丘們!這些被稱為七十七智之事。」



\sutta{35}{35}{無明為緣經}{https://agama.buddhason.org/SN/sn.php?keyword=12.35}
  住在舍衛城……(中略)。

  「\twnr{比丘}{31.0}們!以無明\twnr{為緣}{180.0}有諸行(而諸行存在);以行為緣有識……(中略)這樣是這整個\twnr{苦蘊}{83.0}的\twnr{集}{67.0}。」

  在這麼說時,某位比丘對\twnr{世尊}{12.0}說這個:

  「\twnr{大德}{45.0}!什麼是老死呢?還有,這個老死屬於誰?」

  「不適當的問題。」世尊說。

  「比丘!凡如果他說:『什麼是老死?還有,這個老死屬於誰?』比丘!或凡如果他說:『老死是一,還有,這個老死屬於另一。(老死與有老死者不同)』這兩者同一意義,僅字句不同。

  比丘!存在於『命即是身體』之見中,梵行生活不存在,比丘!或存在於『命是一,\twnr{身體是另一}{169.0}』之見中,梵行生活不存在。

  比丘!不走入這些那些兩個邊後,\twnr{如來}{4.0}以中間教導法:『以生為緣有老死。』」

  「大德!什麼是生?還有,這個生屬於誰?」

  「不是適當的問題。」世尊說。

  「比丘!凡如果他說:『什麼是生?還有,這個生屬於誰?』比丘!或凡如果他說:『生是一,還有,這個生屬於另一。』這兩者同一意義,僅字句不同。

  比丘!存在於『命即是身體』之見中,梵行生活不存在,比丘!或存在於『命是一身體是另一』之見中,梵行生活不存在。

  比丘!不走入這些那些兩個邊後,如來以中間教導法:『以有為緣有生。』」

  「大德!什麼是有呢?還有,這個有屬於誰?」

  「不是適當的問題。」世尊說。

  「比丘!凡如果他說:『什麼是有?還有,這個有屬於誰?』比丘!或凡如果他說:『有是一,還有,這個有屬於另一。』這兩者同一意義,僅字句不同。

  比丘!存在於『命即是身體』之見中,梵行生活不存在,比丘!或存在於『命是一身體是另一』之見中,梵行生活不存在。

  比丘!不走入這些那些兩個邊後,如來以中間教導法:『以取為緣有有。』」

  「……(中略)『以渴愛為緣有取。』……『以受為緣有渴愛。』……『以觸為緣有受。』……『以六處為緣有觸。』……『以名色為緣有六處。』……『以識為緣有名色。』……『以行為緣有識。』」

  「大德!什麼是諸行呢?還有,這些諸行屬於誰?」

  「不是適當的問題。」世尊說。

  「比丘!凡如果他說:『什麼是諸行?還有,這個行屬於誰?』比丘!或凡如果他說:『諸行是一,還有,這些諸行屬於另一。』這兩者同一意義,僅字句不同。

  比丘!存在於『命即是身體』之見中,梵行生活不存在,比丘!或存在於『命是一身體是另一』之見中,梵行生活不存在。

  比丘!不走入這些那些兩個邊後,如來以中間教導法:『以無明為緣有諸行。』

  比丘!但就以無明的\twnr{無餘褪去與滅}{491.0},凡他的那些歪曲、相違、扭曲:『什麼是老死?還有,這個老死屬於誰?』或『老死是一,還有,這個老死屬於另一。』或『命即是身體。』或『命是一身體是另一。』他的那一切都被捨斷,根已被切斷,\twnr{[如]已斷根的棕櫚樹}{147.1},\twnr{成為非有}{408.0},\twnr{為未來不生之物}{229.0}。

  比丘!但就以無明的無餘褪去與滅,凡他的那些歪曲、相違、扭曲:『什麼是生?還有,這個生屬於誰?』或『生是一,還有,這個生屬於另一。』或『命即是身體。』或『命是一身體是另一。』他的那一切都被捨斷,根已被切斷,[如]已斷根的棕櫚樹,成為非有,為未來不生之物。

  比丘!但就以無明的無餘褪去與滅,凡他的那些歪曲、相違、扭曲:『什麼是有?……(中略)什麼是取?……什麼是渴愛?……什麼是受?……什麼是觸?……什麼是六處?……什麼是名色?……什麼是識?……』……(中略)。

  比丘!但就以無明的無餘褪去與滅,凡他的那些歪曲、相違、扭曲:『什麼是諸行?還有,這些諸行屬於誰?』或『諸行是一,還有,這些諸行屬於另一。』或『命即是身體。』或『命是一身體是另一。』他的那一切都被捨斷,根已被切斷,[如]已斷根的棕櫚樹,成為非有,為未來不生之物。」



\sutta{36}{36}{無明為緣經第二}{https://agama.buddhason.org/SN/sn.php?keyword=12.36}
  住在舍衛城……(中略)。

  「\twnr{比丘}{31.0}們!以無明\twnr{為緣}{180.0}有諸行(而諸行存在);以行為緣有識……(中略)這樣是這整個\twnr{苦蘊}{83.0}的\twnr{集}{67.0}。

  比丘們!凡如果他說:『什麼是老死?還有,這個老死屬於誰?』比丘們!或凡如果他說:『老死是一,還有,這個老死屬於另一。(老死與有老死者不同)』這兩者同一意義,僅字句不同。

  比丘們!存在於『命即是身體』之見中,梵行生活不存在,比丘們!或存在於『命是一,\twnr{身體是另一}{169.0}』之見中,梵行生活不存在。

  比丘們!不走入這些那些兩個邊後,\twnr{如來}{4.0}以中間教導法:『以生為緣有老死。』

  『什麼是生?……(中略)什麼是有?……什麼是取?……什麼是渴愛?……什麼是受?……什麼是觸?……什麼是六處?……什麼是名色?……什麼是識?……什麼是諸行?還有,這些諸行屬於誰?』比丘們!或凡如果他說:『諸行是一,還有,這些諸行屬於另一。』這兩者同一意義,僅字句不同。

  比丘們!存在於『命即是身體』之見中,梵行生活不存在,比丘們!或存在於『命是一身體是另一』之見中,梵行生活不存在。

  比丘們!不走入這些那些兩個邊後,如來以中間教導法:『以無明為緣有諸行。』

  比丘們!但就以無明的\twnr{無餘褪去與滅}{491.0},凡他的那些歪曲、相違、扭曲:『什麼是老死?而這老死又是誰的?』或者『老死是一件事,而這老死又屬誰是另一件事。』或者『命即是身體。』或者『命是一身體是另一。』這一切都已被捨斷,根已被切斷,\twnr{[如]已斷根的棕櫚樹}{147.1},\twnr{成為非有}{408.0},\twnr{為未來不生之物}{229.0}。

  比丘們!但就以無明的無餘褪去與滅,凡他的那些歪曲、相違、扭曲:『什麼是生?……(中略)什麼是有?……什麼是取?……什麼是渴愛?……什麼是受?……什麼是觸?……什麼是六處?……什麼是名色?……什麼是識……?『什麼是諸行?還有,這些諸行屬於誰?』或『諸行是一,還有,這些諸行屬於另一。』或『命即是身體。』或『命是一身體是另一。』他的那一切都被捨斷,根已被切斷,[如]已斷根的棕櫚樹,成為非有,為未來不生之物。」 



\sutta{37}{37}{非你們的經}{https://agama.buddhason.org/SN/sn.php?keyword=12.37}
  住在舍衛城……(中略)。

  「\twnr{比丘}{31.0}們!這個身體不是你們的,也不是其他人的,比丘們!這是舊業,應該被看作被造作的、\twnr{被思惟的}{827.0}、能被感受的。

  比丘們!在那裡,\twnr{有聽聞的聖弟子}{24.0}就徹底地\twnr{如理作意}{114.0}緣起:『像這樣,在這個存在時那個存在,以這個的生起那個生起。在這個不存在時那個不存在,以這個的滅\twnr{那個被滅}{394.0},即:以\twnr{無明}{207.0}\twnr{為緣}{180.0}有諸行(而諸行存在);以行為緣有識……(中略)這樣是這整個\twnr{苦蘊}{83.0}的\twnr{集}{67.0}。但就以無明的\twnr{無餘褪去與滅}{491.0}有行滅(而行滅存在);以行滅有識滅……(中略)這樣是這整個苦蘊的\twnr{滅}{68.0}。』」



\sutta{38}{38}{思經}{https://agama.buddhason.org/SN/sn.php?keyword=12.38}
  起源於舍衛城。

  「\twnr{比丘}{31.0}們!\twnr{凡他意圖}{x265},與凡他計畫(遍計),以及凡他\twnr{潛伏}{253.0},\twnr{這個識存續的所緣存在}{x266}。在所緣存在時,識的立足處存在;在那個識被住立被增長時,\twnr{未來再有的出生}{804.0}存在;在未來再有的出生存在時,未來生、老、死、愁、悲、苦、憂、\twnr{絕望}{342.0}生成,這樣是這整個\twnr{苦蘊}{83.0}的\twnr{集}{67.0}。

  比丘們!如果他不意圖,如果他不計畫,如果他還潛伏,這個識存續的所緣存在。在所緣存在時,識的立足處存在;在那個識被住立被增長時,未來再有的出生存在;在未來再有的出生存在時,未來生、老、死、愁、悲、苦、憂、絕望生成,這樣是這整個苦蘊的集。

  比丘們!但當他既不意圖,他也不計畫、不潛伏,這個識存續的所緣不存在。在所緣不存在時,識的立足處不存在;在那個識不被住立不被增長時,未來再有的出生不存在;在未來再有的出生不存在時,未來生、老、死、愁、悲、苦、憂、絕望被滅,這樣是這整個苦蘊的\twnr{滅}{68.0}。」



\sutta{39}{39}{思經第二}{https://agama.buddhason.org/SN/sn.php?keyword=12.39}
  住在舍衛城……(中略)。

  「\twnr{比丘}{31.0}們!\twnr{凡他意圖}{x267},與凡他計畫(遍計),以及凡他\twnr{潛伏}{253.0},這個識存續的所緣存在。在所緣存在時,識的立足處存在;在那個識被住立被增長時,\twnr{名色的下生}{673.0}存在;以名色\twnr{為緣}{180.0}有六處(而六處存在);以六處為緣有觸;以觸為緣有受……(中略)渴愛……取……有……生……老、死、愁、悲、苦、憂、\twnr{絕望}{342.0}生成,這樣是這整個\twnr{苦蘊}{83.0}的\twnr{集}{67.0}。

  比丘們!如果他不意圖,如果他不計畫,如果他還潛伏,這個識存續的所緣存在。在所緣存在時,識的立足處存在;在那個識被住立被增長時,名色的下生存在;以名色為緣有六處……(中略)這樣是這整個苦蘊的集。

  比丘們!但當他既不意圖,他也不計畫、不潛伏,這個識存續的所緣不存在。在所緣不存在時,識的立足處不存在;在那個識不被住立不被增長時,名色的下生不存在;以名色\twnr{滅}{68.0}而六處滅……(中略)這樣是這整個苦蘊的滅。」



\sutta{40}{40}{思經第三}{https://agama.buddhason.org/SN/sn.php?keyword=12.40}
  住在舍衛城……(中略)。

  「\twnr{比丘}{31.0}們!\twnr{凡他意圖}{x267},與凡他計畫(遍計),以及凡他\twnr{潛伏}{253.0},這個識存續的所緣存在。在所緣存在時,識的立足處存在;在那個識被住立被增長時,傾向存在;在傾向存在時,來去存在;在來去存在時,死亡與往生存在;在死亡與往生存在時,未來生、老、死、愁、悲、苦、憂、\twnr{絕望}{342.0}生成,這樣是這整個\twnr{苦蘊}{83.0}的\twnr{集}{67.0}。

  比丘們!如果他不意圖,如果他不計畫,如果他還潛伏,這個識存續的所緣存在。在所緣存在時,識的立足處存在;在那個識被住立被增長時,傾向存在;在傾向存在時,來去存在;在來去存在時,死亡與往生存在;在死亡與往生存在時,未來生、老、死、愁、悲、苦、憂、絕望生成,這樣是這整個苦蘊的集。

  比丘們!但當他既不意圖,他也不計畫、不潛伏,這個識存續的所緣不存在。在所緣不存在時,識的立足處不存在;在那個識不被住立不被增長時,傾向不存在;在傾向不存在時,來去不存在;在來去不存在時,死亡與往生不存在;在死亡與往生不存在時,未來老、死、愁、悲、苦、憂、絕望被滅,這樣是這整個苦蘊的\twnr{滅}{68.0}。」

  剎帝利黑齒品第四,其\twnr{攝頌}{35.0}:

  「這已生成的與黑齒,智之事二則,

   以無明為緣二則,非你們的、思三則。」





\pin{屋主品}{41}{50}
\sutta{41}{41}{五恐怖怨恨經}{https://agama.buddhason.org/SN/sn.php?keyword=12.41}
  住在舍衛城[……(中略)]。 

  那時,\twnr{屋主}{103.0}給孤獨去見\twnr{世尊}{12.0}。抵達後,向世尊\twnr{問訊}{46.0}後,在一旁坐下。世尊對在一旁坐下的屋主給孤獨說這個:

  「屋主!當\twnr{聖弟子}{24.0}的五恐怖、怨恨被平息,與具備四\twnr{入流支}{370.0},以及他的聖方法(理趣)被慧善見、善洞察,當他希望時,就能以自己\twnr{記說}{179.0}自己:『我是地獄已盡者,畜生界已盡者,\twnr{餓鬼界}{362.0}已盡者,\twnr{苦界}{109.0}、\twnr{惡趣}{110.0}、\twnr{下界}{111.0}已盡者,我是\twnr{入流者}{165.0}、不墮惡趣法者、\twnr{決定者}{159.0}、\twnr{正覺為彼岸者}{160.0}。』

  哪五個恐怖、怨恨被平息?屋主!凡殺生者,以殺生\twnr{為緣}{180.0}產生當生的恐怖、怨恨,也產生來生的恐怖、怨恨,也感受心的憂苦,這樣,離殺生者的那個恐怖、怨恨被平息。

  屋主!凡未給予而取者,以\twnr{未給予而取}{104.0}為緣產生當生的恐怖、怨恨,也產生來生的恐怖、怨恨,也感受心的憂苦,這樣,離未給予而取者的那個恐怖、怨恨被平息。

  屋主!凡邪淫者,以\twnr{邪淫}{105.0}為緣產生當生的恐怖、怨恨,也產生來生的恐怖、怨恨,也感受心的憂苦,這樣,離邪淫者的那個恐怖、怨恨被平息。

  屋主!凡妄語者,以\twnr{妄語}{106.0}為緣產生當生的恐怖、怨恨,也產生來生的恐怖、怨恨,也感受心的憂苦,這樣,離妄語者的那個恐怖、怨恨被平息。

  屋主!凡榖酒、果酒、酒放逸處者,以榖酒、果酒、\twnr{酒放逸處}{107.0}為緣產生當生的恐怖、怨恨,也產生來生的恐怖、怨恨,也感受心的憂苦,這樣,離榖酒、果酒、酒放逸處者的那個恐怖、怨恨被平息。這是五個恐怖、怨恨被平息。

  具備哪四入流支?屋主!這裡,聖弟子在佛上具備\twnr{不壞淨}{233.0}:『像這樣,那位世尊是\twnr{阿羅漢}{5.0}、\twnr{遍正覺者}{6.0}、\twnr{明行具足者}{7.0}、\twnr{善逝}{8.0}、\twnr{世間知者}{9.0}、\twnr{應該被調御人的無上調御者}{10.0}、\twnr{天-人們的大師}{11.0}、\twnr{佛陀}{3.0}、世尊。』

  在法上具備不壞淨:『被世尊善說的法是直接可見的、即時的、請你來看的、能引導的、\twnr{應該被智者各自經驗的}{395.0}。』

  在\twnr{僧團}{375.0}上具備不壞淨:『世尊的弟子僧團是\twnr{善行者}{518.0},世尊的弟子僧團是正直行者,世尊的弟子僧團是真理行者,世尊的弟子僧團是\twnr{方正行者}{764.0},即:四雙之人、\twnr{八輩之士}{347.0},這世尊的弟子僧團應該被奉獻、應該被供奉、應該被供養、應該被合掌,為世間的無上\twnr{福田}{101.0}。』

  具備聖者喜愛的諸戒:無毀壞的、無瑕疵的、無污點的、無雜色的、自由的、智者稱讚的、不取著的、轉起定的。具備這四入流支。

  而什麼是以及他的聖方法被慧善見、善洞察?屋主!這裡,聖弟子就徹底地\twnr{如理作意}{114.0}\twnr{緣起}{225.0}:『像這樣,在這個存在時那個存在,在這個不存在時那個不存在。以這個的生起那個生起,以這個的滅\twnr{那個被滅}{394.0},即:以\twnr{無明}{207.0}\twnr{為緣}{180.0}有諸行(而諸行存在);以行為緣有識……(中略)這樣是這整個\twnr{苦蘊}{83.0}的\twnr{集}{67.0}。但以無明的\twnr{無餘褪去與滅}{491.0}有行滅(而行滅存在)……(中略)以觸滅有受滅;以受滅有渴愛滅……(中略)這樣是這整個苦蘊的滅。』這是他的聖方法(理趣)被慧善見、善洞察。

  屋主!當聖弟子的這些五恐怖、怨恨被平息,與具備這些四入流支,以及他的聖方法(理趣)被慧善見、善洞察,當他希望時,就能以自己記說自己:『我是地獄已盡者,畜生界已盡者,餓鬼界已盡者,苦界、惡趣、下界已盡者,我是入流者、不墮惡趣法者、決定者、以正覺為彼岸。』」[\suttaref{SN.55.28}, \ccchref{AN.10.92}{https://agama.buddhason.org/AN/an.php?keyword=10.92}]



\sutta{42}{42}{五恐怖怨恨經第二}{https://agama.buddhason.org/SN/sn.php?keyword=12.42}
  住在舍衛城……(中略)。 

  「\twnr{比丘}{31.0}們!當\twnr{聖弟子}{24.0}的五恐怖、怨恨被平息,與具備四\twnr{入流支}{370.0},以及他的聖方法(理趣)被慧善見、善洞察,當他希望時,就能以自己\twnr{記說}{179.0}自己:『我是地獄已盡者,畜生界已盡者,\twnr{餓鬼界}{362.0}已盡者,\twnr{苦界}{109.0}、\twnr{惡趣}{110.0}、\twnr{下界}{111.0}已盡者,我是\twnr{入流者}{165.0}、不墮惡趣法者、\twnr{決定者}{159.0}、\twnr{正覺為彼岸者}{160.0}。』[\suttaref{SN.55.29}]

  哪五個恐怖、怨恨被平息?比丘們!凡殺生者……(中略)比丘們!凡為\twnr{邪淫}{105.0}者……(中略)比丘們!凡為\twnr{妄語}{106.0}者……(中略)比丘們!凡為榖酒、果酒、\twnr{酒放逸處}{107.0}者……(中略)。這是五個恐怖、怨恨被平息。

  具備哪四入流支?比丘們!這裡,聖弟子對佛……(中略)對法……對\twnr{僧團}{375.0}……具備聖者喜愛的諸戒……具備這四入流支。

  什麼是以及他的聖方法被慧善見、善洞察?比丘們!這裡,就徹底地\twnr{如理作意}{114.0}\twnr{緣起}{225.0}……(中略)[\suttaref{SN.12.41}]。這是他的聖方法(理趣)被慧善見、善洞察。

  「比丘們!當聖弟子的這些五恐怖、怨恨被平息,與具備這些四入流支,以及他的聖方法(理趣)被慧善見、善洞察,當他希望時,就能以自己記說自己:『我是地獄已盡者,畜生界已盡者,餓鬼界已盡者,苦界、惡趣、下界已盡者,我是入流者、不墮惡趣法者、決定者、正覺為彼岸者。』」[\ccchref{AN.9.28}{https://agama.buddhason.org/AN/an.php?keyword=9.28}]



\sutta{43}{43}{苦經}{https://agama.buddhason.org/SN/sn.php?keyword=12.43}
  住在舍衛城……(中略)。 

  「\twnr{比丘}{31.0}們!我將教導苦的\twnr{集起}{67.0}與滅沒,你們要聽它!你們要\twnr{好好作意}{43.1}!我將說。」

  「是的,\twnr{大德}{45.0}!」那些比丘回答\twnr{世尊}{12.0}。

  世尊說這個:

  「比丘們!而什麼是苦的集起?\twnr{緣於}{252.0}眼與諸色眼識生起,三者的會合有觸,以觸\twnr{為緣}{180.0}有受(而受存在),以受為緣有渴愛,比丘們!這是苦的集起。

  緣於耳與諸聲音生起耳識……(中略)緣於鼻與諸氣味……(中略)緣於舌與諸味道……(中略)緣於身與諸\twnr{所觸}{220.2}……(中略)緣於意與諸法生起意識,三者的會合為觸,以觸為緣有受,以受為緣有渴愛,比丘們!這是苦的集起。

  比丘們!而什麼是苦的滅沒?緣於眼與諸色眼識生起,三者的會合有觸,以觸為緣有受,以受為緣有渴愛,就以那個渴愛的\twnr{無餘褪去與滅}{491.0}有取\twnr{滅}{68.0}(而取滅存在),以取滅有有滅,以有滅有生滅,以生滅而老、死、愁、悲、苦、憂、\twnr{絕望}{342.0}被滅,這樣是這整個苦蘊的滅,比丘們!這是苦的滅沒。

  緣於耳與諸聲音生起耳識……(中略)緣於鼻與諸氣味……(中略)緣於舌與諸味道……(中略)緣於身與諸所觸……(中略)緣於意與諸法生起意識,三者的會合為觸,以觸為緣有受,以受為緣有渴愛,就以那個渴愛的無餘褪去與滅有取滅,以取滅有有滅,以有滅有生滅,以生滅而老、死、愁、悲、苦、憂、絕望被滅,這樣是這整個苦蘊的滅,比丘們!這是苦的滅沒。」[\suttaref{SN.35.106}]



\sutta{44}{44}{世間經}{https://agama.buddhason.org/SN/sn.php?keyword=12.44}
  住在舍衛城……(中略)。 

  「\twnr{比丘}{31.0}們!我將教導世間的\twnr{集起}{67.0}與滅沒,你們要聽它!你們要\twnr{好好作意}{43.1}!我將說。」

  「是的,\twnr{大德}{45.0}!」那些比丘回答\twnr{世尊}{12.0}。

  世尊說這個:

  「比丘們!而什麼是世間的集起?\twnr{緣於}{252.0}眼與諸色眼識生起,三者的會合有觸,以觸\twnr{為緣}{180.0}有受(而受存在),以受為緣有渴愛,以渴愛為緣有取,以取為緣有有,以有為緣有生,以生為緣而老、死、愁、悲、苦、憂、\twnr{絕望}{342.0}生成,比丘們!這是世間的集起。

  緣於耳與諸聲音……(中略)緣於鼻與諸氣味……緣於舌與諸味道……緣於身與諸\twnr{所觸}{220.2}……緣於意與諸法生起意識,三者的會合為觸,以觸為緣有受……(中略)以生為緣而老、死、愁、悲、苦、憂、絕望生成,比丘們!這是世間的集起。

  比丘們!而什麼是世間的滅沒?緣於眼與諸色眼識生起,三者的會合有觸,以觸為緣有受,以受為緣有渴愛,就以那個渴愛的\twnr{無餘褪去與滅}{491.0}有取\twnr{滅}{68.0}(而取滅存在),以取滅有有滅……(中略)這樣是這整個苦蘊的滅,比丘們!這是世間的滅沒。

  緣於耳與諸聲音……(中略)緣於鼻與諸氣味……緣於舌與諸味道……緣於身與諸所觸……緣於意與諸法生起意識,三者的會合為觸,以觸為緣有受,以受為緣有渴愛,就以那個渴愛的無餘褪去與滅有取滅,以取滅有有滅……(中略)這樣是這整個苦蘊的滅,比丘們!這是世間的滅沒。」[\suttaref{SN.35.107}]



\sutta{45}{45}{親戚村經}{https://agama.buddhason.org/SN/sn.php?keyword=12.45}
  \twnr{被我這麼聽聞}{1.0}:

  \twnr{有一次}{2.0},\twnr{世尊}{12.0}住在\twnr{親戚村}{709.0}的磚屋中。

  那時,獨處、\twnr{獨坐}{92.0}的世尊說這個\twnr{法的教說}{562.1}:  

  「\twnr{緣於}{252.0}眼與諸色眼識生起,三者的會合有觸,以觸\twnr{為緣}{180.0}有受(而受存在),以受為緣有渴愛,以渴愛為緣有取……(中略)這樣是這整個\twnr{苦蘊}{83.0}的\twnr{集}{67.0}。

  緣於耳與諸聲音……(中略)緣於鼻與諸氣味……(中略)緣於舌與諸味道……(中略)緣於身與諸所觸……(中略)緣於意與諸法生起意識,三者的會合為觸,以觸為緣有受,以受為緣有渴愛,以渴愛為緣有取……(中略)這樣是這整個\twnr{苦蘊}{83.0}的\twnr{集}{67.0}。

  緣於眼與諸色眼識生起,三者的會合有觸,以觸為緣有受,以受為緣有渴愛,就以那個渴愛的\twnr{無餘褪去與滅}{491.0}而取\twnr{滅}{68.0}(而取滅存在),以取滅有有滅……(中略)這樣是這整個苦蘊的滅。

  緣於耳與諸聲音……(中略)緣於意與諸法生起意識,三者的會合為觸,以觸為緣有受,以受為緣有渴愛,就以那個渴愛的無餘褪去與滅有取滅,以取滅有有滅……(中略)這樣是這整個苦蘊的滅。」

  當時,\twnr{某位比丘}{39.0}站立對世尊屏息側聽。世尊看見站立屏息側聽的那位比丘。看見後,對那位比丘說這個:

  「比丘!你聽到這個法的教說了嗎?」

  「是的,\twnr{大德}{45.0}!」

  「比丘!你要學習這法的教說,比丘!你要學得這個法的教說,比丘!你要\twnr{憶持}{57.0}這個法的教說,比丘!這個法的教說是\twnr{伴隨利益的}{50.0},是\twnr{梵行基礎的}{446.0}。」[\suttaref{SN.35.113}]



\sutta{46}{46}{某位婆羅門經}{https://agama.buddhason.org/SN/sn.php?keyword=12.46}
  住在舍衛城。

  那時,\twnr{某位}{39.0}婆羅門去見世尊。抵達後,與世尊一起互相問候。交換應該被互相問候的友好交談後,在一旁坐下。在一旁坐下的那位\twnr{婆羅門}{17.0}對世尊說這個:

  「\twnr{喬達摩}{80.0}尊師!怎麼樣,『他做,他感受。』嗎?」

  「『他做,他感受』,婆羅門!這是一邊。」

  「喬達摩尊師!那麼,『一個做,另一個感受。』嗎?」

  「『一個做,另一個感受。』婆羅門!這是第二邊。

  婆羅門!不走入這些那些兩個邊後,\twnr{如來}{4.0}以中間教導法:『以無明\twnr{為緣}{180.0}有諸行(而諸行存在);以行為緣有識……(中略)這樣是這整個苦蘊的集。

  但就以無明的\twnr{無餘褪去與滅}{491.0}有行滅(而行滅存在);以行滅有識滅……(中略)這樣是這整個苦蘊的滅。』」

  在這麼說時,那位婆羅門對世尊說這個:

  「太偉大了,喬達摩尊師!太偉大了,喬達摩尊師!……(中略)請喬達摩\twnr{尊師}{203.0}記得我為\twnr{優婆塞}{98.0},從今天起\twnr{已終生歸依}{64.0}。」



\sutta{47}{47}{若奴索尼經}{https://agama.buddhason.org/SN/sn.php?keyword=12.47}
  住在舍衛城。

  那時,\twnr{若奴索尼}{980.0}婆羅門去見\twnr{世尊}{12.0}。抵達後,與世尊互相……(中略)在一旁坐下的若奴索尼\twnr{婆羅門}{17.0}對世尊說這個:

  「\twnr{喬達摩}{80.0}尊師!怎麼樣,一切存在嗎?」

  「婆羅門!『一切存在』,這是一邊。」

  「喬達摩尊師!那麼,一切不存在嗎?」

  「婆羅門!『一切不存在』,這是第二邊。婆羅門!不走入這些那些兩個邊後,如來以中間教導法:『以\twnr{無明}{207.0}為緣有諸行(而諸行存在);以行\twnr{為緣}{180.0}有識……(中略)這樣是這整個\twnr{苦蘊}{83.0}的\twnr{集}{67.0}。但就以無明的\twnr{無餘褪去與滅}{491.0}而行\twnr{滅}{68.0};以行滅有識滅……這樣是這整個苦蘊的滅。」

  在這麼說時,若奴索尼婆羅門對世尊說這個:

  「太偉大了,喬達摩尊師!……(中略)從今天起\twnr{已終生歸依}{64.0}。」



\sutta{48}{48}{順世派經}{https://agama.buddhason.org/SN/sn.php?keyword=12.48}
  住在舍衛城。

  那時,\twnr{順世派}{x268}婆羅門去見\twnr{世尊}{12.0}。抵達後,與世尊互相……(中略)在一旁坐下的順世派婆羅門對世尊說這個:

  「\twnr{喬達摩}{80.0}尊師!怎麼樣,一切存在嗎?」

  「婆羅門!『一切存在』,這是最古老的世間論(順世論)。」

  「喬達摩尊師!那麼,一切不存在嗎?」

  「婆羅門!『一切不存在』,這是第二種世間論。」

  「喬達摩尊師!怎麼樣,一切是單一狀態嗎?」

  「婆羅門!『一切是單一的』,這是第三種世間論。」

  「喬達摩尊師!那麼,一切是別異狀態嗎?」

  「婆羅門!『一切是別異狀態』,這是第四種世間論。

  婆羅門!不走入這些那些兩個邊後,如來以中間教導法:『以\twnr{無明}{207.0}為緣有諸行(而諸行存在);以行\twnr{為緣}{180.0}有識……(中略)這樣是這整個\twnr{苦蘊}{83.0}的\twnr{集}{67.0}。但就以無明的\twnr{無餘褪去與滅}{491.0}而行\twnr{滅}{68.0};以行滅有識滅……(中略)這樣是這整個苦蘊的滅。」

  在這麼說時,順世派婆羅門對世尊說這個:

  「太偉大了,喬達摩尊師!……(中略)從今天起\twnr{已終生歸依}{64.0}。」



\sutta{49}{49}{聖弟子經}{https://agama.buddhason.org/SN/sn.php?keyword=12.49}
  住在舍衛城……(中略)。

  「\twnr{比丘}{31.0}們!\twnr{有聽聞的聖弟子}{24.0}不這麼想:『在什麼存在時什麼存在呢?以什麼的生起什麼生起?(在什麼存在時諸行存在?在什麼存在時識存在?)在什麼存在時名色存在?在什麼存在時六處存在?在什麼存在時觸存在?在什麼存在時受存在?在什麼存在時渴愛存在?在什麼存在時取存在?在什麼存在時有存在?在什麼存在時生存在?在什麼存在時老死存在?』

  比丘們!然後,有聽聞的聖弟子在這裡就有\twnr{不緣於他}{380.0}的智:『在這個存在時那個存在,以這個的生起那個生起,(在無明存在時諸行存在,在諸行存在時識存在,)在識存在時名色存在,在名色存在時六處存在,在六處存在時觸存在,在觸存在時受存在,在受存在時渴愛存在,在渴愛存在時取存在,在取存在時有存在,在有存在時生存在,在生存在時老死存在。』

  他這麼知道:『這樣,這世間\twnr{集起}{67.0}。』

  比丘們!有聽聞的聖弟子不這麼想:『在什麼不存在時什麼不存在呢?以什麼的滅什麼被滅?(在什麼不存在時諸行不存在?在什麼不存在時識不存在?)在什麼不存在時名色不存在?在什麼不存在時六處不存在?在什麼不存在時觸不存在?在什麼不存在時受不存在?在什麼不存在時渴愛不存在?在什麼不存在時取不存在?在什麼不存在時有不存在?在什麼不存在時生不存在?在什麼不存在時老死不存在?』

  比丘們!然後,有聽聞的聖弟子在這裡就有不依賴他人之智:『在這個不存在時那個不存在,以這個的滅那個被滅,(在無明不存在時諸行不存在,在諸行不存在時識不存在,)在識不存在時名色不存在,在名色不存在時六處不存在……(中略)有不存在……生不存在,在生不存在時老死不存在。』

  他這麼知道:『這樣,這世間被滅。』

  比丘們!當聖弟子這麼如實知道世間的集起與滅沒,比丘們!這位聖弟子被稱為『\twnr{見具足者}{575.0}』……(中略)及『他敲打\twnr{不死}{123.0}之門後住立』。」



\sutta{50}{50}{聖弟子經第二}{https://agama.buddhason.org/SN/sn.php?keyword=12.50}
  住在舍衛城……(中略)。

  「\twnr{比丘}{31.0}們!\twnr{有聽聞的聖弟子}{24.0}不這麼想:『在什麼存在時什麼存在呢?以什麼的生起什麼生起?在什麼存在時諸行存在?在什麼存在時識存在?在什麼存在時名色存在?在什麼存在時六處存在?在什麼存在時觸存在?在什麼存在時受存在?在什麼存在時渴愛存在?在什麼存在時取存在?在什麼存在時有存在?在什麼存在時生存在?在什麼存在時老死存在?』

  比丘們!然後,有聽聞的聖弟子在這裡就有\twnr{不緣於他}{380.0}的智:『在這個存在時那個存在,以這個的生起那個生起,在無明存在時諸行存在,在諸行存在時識存在,在識存在時名色存在,在名色存在時六處存在,在六處存在時觸存在,在觸存在時受存在,在受存在時渴愛存在,在渴愛存在時取存在,在取存在時有存在,在有存在時生存在,在生存在時老死存在。』

  他這麼知道:『這樣,這世間\twnr{集起}{67.0}。』

  比丘們!有聽聞的聖弟子不這麼想:『在什麼不存在時什麼不存在呢?以什麼的滅什麼被滅?在什麼不存在時諸行不存在?在什麼不存在時識不存在?在什麼不存在時名色不存在?在什麼不存在時六處不存在?在什麼不存在時觸不存在?在什麼不存在時受不存在?在什麼不存在時渴愛不存在?……(中略)取……有……生……在什麼不存在時老死不存在?』

  比丘們!然後,有聽聞的聖弟子在這裡就有不依賴他人之智:『在這個不存在時那個不存在,以這個的滅那個被滅,在無明不存在時諸行不存在,在諸行不存在時識不存在,在識不存在時名色不存在,在名色不存在時六處不存在……(中略)在生不存在時老死不存在。』

  他這麼知道:『這樣,這世間被滅。』

  比丘們!當聖弟子這麼如實知道世間的集起與滅沒,比丘們!這位聖弟子被稱為『\twnr{見具足者}{575.0}』,及『\twnr{看見具足者}{572.0}』,及『來到這正法者』,及『他看見這正法』,及『具備\twnr{有學}{193.0}之智者』,及『具備有學之明者』,及『進入法流者』,及『\twnr{洞察慧}{566.0}之聖者』,及『他敲打\twnr{不死}{123.0}之門後住立』。」

  \twnr{屋主}{103.0}品第五,其\twnr{攝頌}{35.0}:

  「五恐怖怨恨兩說,苦、世間以及親戚村,

   某位與若奴索尼,以順世派為第八,

   聖弟子兩說,以那個被稱為品。」





\pin{苦品}{51}{60}
\sutta{51}{51}{審慮經}{https://agama.buddhason.org/SN/sn.php?keyword=12.51}
  \twnr{被我這麼聽聞}{1.0}:

  \twnr{有一次}{2.0},\twnr{世尊}{12.0}住在舍衛城祇樹林給孤獨園。

  在那裡,世尊召喚\twnr{比丘}{31.0}們:「比丘們!」

  「\twnr{尊師}{480.0}!」那些比丘回答世尊。

  世尊說這個:

  「比丘們!當比丘審慮時,為了全部苦的完全滅盡,應該從哪方面審慮呢?」

  「大德!我們的法是世尊為根本的、\twnr{世尊為導引的}{56.0}、世尊為依歸的,大德!就請世尊說明這個所說的義理,\twnr{那就好了}{44.0}!聽聞世尊的[教說]後,比丘們將會\twnr{憶持}{57.0}。」

  「比丘們!那樣的話,你們要聽!你們\twnr{要好好作意}{43.1}!我將說。」

  「是的,\twnr{大德}{45.0}!」那些比丘回答世尊。

  世尊說這個:

  「比丘們!這裡,當比丘審慮時,他審慮:『凡這種種不同種類的老死苦在世間生起,這個苦,什麼為因?什麼為集?什麼生的?\twnr{什麼為根源}{668.0}呢?在什麼存在時老死存在?在什麼不存在時老死不存在?』當他審慮時,他這麼知道:『凡這種種不同種類的老死苦在世間生起,這個苦,生為因,生為集,生生的,生為根源,在生存在時老死存在;在生不存在時老死不存在。』

  他知道老死、知道老死集、知道老死滅、知道適合導向老死\twnr{滅道跡}{69.0}。而像這樣的行者是\twnr{隨法行者}{765.0},比丘們!這位比丘被稱為,為了全部苦的完全滅盡、為了老死滅的行者。

  然後,當更進一步審慮時,他審慮:『那麼,這個生,什麼為因?什麼為集?什麼生的?什麼為根源?在什麼存在時生存在?在什麼不存在時生不存在?』當他審慮時,他這麼知道:『生,有為因,有為集,有生的,有為根源,在有存在時生存在;在有不存在時生不存在。』

  他知道生、知道生集、知道生滅、知道適合導向生滅道跡。而像這樣的行者是隨法行者,比丘們!這位比丘被稱為,為了全部苦的完全滅盡、為了生滅的行者。

  然後,當更進一步審慮時,他審慮:『那麼,這個有,什麼是其因?……(中略)那麼,這個取,什麼是其因?……那麼,這個渴愛,什麼是其因?……受……觸……那麼,這個六處,什麼是其因?……那麼,這個名色……那麼,這個識……那麼,這些諸行,什麼為因?什麼為集?什麼生的?什麼為根源?在什麼存在時諸行存在?在什麼不存在時諸行不存在?』當他審慮時,他這麼知道:『諸行,\twnr{無明}{207.0}為因,無明為集,無明生的,無明為根源,在無明存在時諸行存在;在無明不存在時諸行不存在。』

  他知道諸行、知道行集、知道行滅、知道適合導向行滅道跡。而像這樣的行者是隨法行者,比丘們!這位比丘被稱為,為了全部苦的完全滅盡、為了行滅的行者。

  比丘們!凡\twnr{進入無明的}{645.0}男子個人如果造作\twnr{福行}{x269},則識是進入福的(轉生於福的);如果造作非福行,則識是進入非福的;如果造作\twnr{不動行}{x270},則識是進入不動的。

  比丘們!當比丘的無明被捨斷,明已生起,他以無明的\twnr{褪去}{77.0},明的生起,既不造作福行,也不造作非福行、不造作不動行。當不造作、\twnr{不思時}{x271},不執取世間中任何事物。不執取者不\twnr{戰慄}{436.0},不戰慄者\twnr{就自己證涅槃}{71.0},他知道:『\twnr{出生已盡}{18.0},\twnr{梵行已完成}{19.0},\twnr{應該被作的已作}{20.0},\twnr{不再有此處[輪迴]的狀態}{21.1}。』

  他如果感受樂受,知道:『那是無常的。』知道:『是不被固執的。』知道:『是不被歡喜的。』他如果感受苦受,知道:『那是無常的。』知道:『是不被固執的。』知道:『是不被歡喜的。』他如果感受不苦不樂受,知道:『那是無常的。』知道:『是不被固執的。』知道:『是不被歡喜的。』他如果感受樂受,離結縛地感受它;他如果感受苦受,離結縛地感受它;他如果感受不苦不樂受,離結縛地感受它。

  他\twnr{當感受身體終了的感受時}{720.0},知道:『我感受身體終了的感受。』\twnr{當感受生命終了的感受時}{721.0},知道:『我感受生命終了的感受。』他知道:『以身體的崩解,隨後生命耗盡,就在這裡,一切所感受的、不被歡喜的將成為清涼,遺骸被留下(剩下)。』[\suttaref{SN.22.88}, \ccchref{MN.140}{https://agama.buddhason.org/MN/dm.php?keyword=140}, 362-365段]

  比丘們!猶如男子從陶匠的窯取出熱陶壺後,使之住立在平地上,在那種情況下,這個熱它就在那裡平息,\twnr{陶瓷碎片會剩下}{x272}。同樣的,比丘們!比丘當感受身體終了的感受時,知道:『我感受身體終了的感受。』或當感受生命終了的感受時,知道:『我感受生命終了的感受。』他知道:『以身體的崩解,隨後生命耗盡,就在這裡,一切所感受的、不被歡喜的將成為清涼,遺骸被留下。』

  比丘們!你們怎麼想它:漏盡比丘會造作福行、會造作非福行、會造作不動行嗎?」「大德!這確實不是。」

  「又或在全部行不存在時,以行滅,識會被你們知道嗎?」「大德!這確實不是。」

  「又或在全部識不存在時,以識滅,名色會被你們知道嗎?」「大德!這確實不是。」

  「又或在全部名色不存在時,以名色滅,六處會被你們知道嗎?」「大德!這確實不是。」

  「又或在全部六處不存在時,以六處滅,觸會被你們知道嗎?」「大德!這確實不是。」

  「又或在全部觸不存在時,以觸滅,受會被你們知道嗎?」「大德!這確實不是。」

  「又或在全部受不存在時,以受滅,渴愛會被你們知道嗎?」「大德!這確實不是。」

  「又或在全部渴愛不存在時,以渴愛滅,取會被你們知道嗎?」「大德!這確實不是。」

  「又或在全部取不存在時,以取滅,有會被你們知道嗎?」「大德!這確實不是。」

  「又或在全部有不存在時,以有滅,生會被你們知道嗎?」「大德!這確實不是。」

  「又或在全部生不存在時,以生滅,老死會被你們知道嗎?」「大德!這確實不是。」

  「比丘們!好!好!比丘們!這是這樣,沒有其它的了。比丘們!那件事你們對我要相信、要\twnr{勝解}{257.0},在這裡,你們要成為無疑惑的、無猶豫的,這就是苦的結束。」



\sutta{52}{52}{執取經}{https://agama.buddhason.org/SN/sn.php?keyword=12.52}
  住在舍衛城……(中略)。

  「\twnr{比丘}{31.0}們!在\twnr{與執取有關的}{551.0}法上住於\twnr{隨看著}{59.0}\twnr{樂味}{295.0}者的渴愛增長,以渴愛\twnr{為緣}{180.0}有取(而取存在),以取為緣有有,以有為緣有生,以生為緣而老、死、愁、悲、苦、憂、\twnr{絕望}{342.0}生成,這樣是這整個\twnr{苦蘊}{83.0}的\twnr{集}{67.0}。

  比丘們!猶如大火聚使十車的柴,或二十車的柴,或三十車的柴,或四十車的柴燃燒。在那裡,如果男子經常地投入乾草,與投入乾牛糞,以及投入乾柴,比丘們!這樣,那個大火聚有\twnr{那個食物、那個燃料}{667.0},它會長久地、長時間地燃燒。同樣的,比丘們!在與執取有關的法上住於隨看著樂味者的渴愛增長,以渴愛為緣有取……(中略)這樣是這整個苦蘊的集。

  比丘們!在與執取有關的法上住於隨看著\twnr{過患}{293.0}者的渴愛被滅,以渴愛滅有取滅(而取滅存在),以取滅有有滅,以有滅有生滅,以生滅而老、死、愁、悲、苦、憂、絕望被滅,這樣是這整個苦蘊的\twnr{滅}{68.0}。

  比丘們!猶如大火聚使十車的柴,或二十[車的柴],或三十[車的柴],或四十車的柴燃燒。在那裡,如果男子不經常投入乾草,也不投入乾牛糞,也不投入乾柴,比丘們!這樣,那大火聚以先前燃料的耗盡,與以其它的沒帶來,無食物者被熄滅。同樣的,比丘們!在與執取有關的法上住於隨看著過患者的渴愛被滅,以渴愛滅有取滅……(中略)這樣是這整個苦蘊的滅。」



\sutta{53}{53}{結經}{https://agama.buddhason.org/SN/sn.php?keyword=12.53}
  住在舍衛城……(中略)。

  「\twnr{比丘}{31.0}們!\twnr{在會被結縛的諸法上}{666.0}住於\twnr{隨看著}{59.0}\twnr{樂味}{295.0}者的渴愛增長,以渴愛\twnr{為緣}{180.0}有取(而取存在),以取為緣有有,以有為緣有生,以生為緣而老、死、愁、悲、苦、憂、\twnr{絕望}{342.0}生成,這樣是這整個\twnr{苦蘊}{83.0}的\twnr{集}{67.0}。

  比丘們!猶如\twnr{緣於}{252.0}油與緣於燈芯,油燈燃燒。在那裡,如果男子經常地灌入油、放置燈芯,比丘們!這樣,那個油燈有\twnr{那個食物、那個燃料}{667.0},它會長久地、長時間地燃燒。同樣的,比丘們!在會被結縛的諸法上住於隨看著樂味者的渴愛增長,以渴愛為緣有取,以取為緣有有,以有為緣有生,以生為緣而老、死、愁、悲、苦、憂、絕望生成,這樣是這整個苦蘊的集。

  比丘們!在會被結縛的諸法上住於隨看著\twnr{過患}{293.0}者的渴愛被滅,以渴愛滅有取滅(而取滅存在),以取滅有有滅,以有滅有生滅,以生滅而老、死、愁、悲、苦、憂、絕望被滅,這樣是這整個苦蘊的\twnr{滅}{68.0}。

  比丘們!猶如緣於油與緣於燈芯,油燈燃燒。在那裡,如果男子不經常地灌入油、放置燈芯,比丘們!這樣,那個油燈以先前燃料的耗盡,與以其它的沒帶來,無食物者被熄滅。同樣的,比丘們!在會被結縛的諸法上住於隨看著過患者的渴愛被滅,以渴愛滅有取滅……(中略)這樣是這整個苦蘊的滅。」



\sutta{54}{54}{結經第二}{https://agama.buddhason.org/SN/sn.php?keyword=12.54}
  住在舍衛城……(中略)。

  「\twnr{比丘}{31.0}們!猶如\twnr{緣於}{252.0}油與緣於燈芯,油燈燃燒。在那裡,如果男子經常地灌入油、放置燈芯,比丘們!這樣,那個油燈有\twnr{那個食物、那個燃料}{667.0},它會長久地、長時間地燃燒。同樣的,比丘們!\twnr{在會被結縛的諸法上}{666.0}住於\twnr{隨看著}{59.0}\twnr{樂味}{295.0}者的渴愛增長,以渴愛\twnr{為緣}{180.0}有取(而取存在)……(中略)這樣是這整個\twnr{苦蘊}{83.0}的\twnr{集}{67.0}。

  比丘們!猶如緣於油與緣於燈芯,油燈燃燒。在那裡,如果男子不經常地灌入油、放置燈芯,比丘們!這樣,那個油燈以先前燃料的耗盡,與以其它的沒帶來,無食物者被熄滅。同樣的,比丘們!在會被結縛的諸法上住於隨看著\twnr{過患}{293.0}者的渴愛被滅,以渴愛滅有取滅(而取滅存在)……(中略)這樣是這整個苦蘊的\twnr{滅}{68.0}。」



\sutta{55}{55}{大樹經}{https://agama.buddhason.org/SN/sn.php?keyword=12.55}
  住在舍衛城……(中略)。

  「\twnr{比丘}{31.0}們!在\twnr{與執取有關的}{551.0}法上住於\twnr{隨看著}{59.0}\twnr{樂味}{295.0}者的渴愛增長,以渴愛\twnr{為緣}{180.0}有取(而取存在),以取為緣有有……(中略)這樣是這整個\twnr{苦蘊}{83.0}的\twnr{集}{67.0}。

  比丘們!猶如大樹,凡它向下走、向橫向走的諸根,那些全都向上帶來滋養素,比丘們!這樣,那棵大樹有\twnr{那個食物、那個燃料}{667.0},它會長久地、長時間地住立。同樣的,比丘們!在與執取有關的法上住於隨看著樂味者的渴愛增長,以渴愛為緣有取……(中略)這樣是這整個苦蘊的集。

  比丘們!在與執取有關的法上住於\twnr{隨看著}{59.0}\twnr{過患}{293.0}者的渴愛被滅,以渴愛滅有取滅(而取滅存在),以取滅有有滅……(中略)這樣是這整個苦蘊的\twnr{滅}{68.0}。

  比丘們!猶如大樹,那時,男子拿鋤頭、籃子後到來,他在根處切斷那棵樹,在根處切斷後挖出,挖出後拉出諸根及乃至鬚根,他切那棵樹成細片,切成細片後使之破裂,使之破裂後做成碎碎的,做成碎碎的後在風熱中使之乾枯,在風熱中使之乾枯後以火燃燒,以火燃燒後轉成(作)灰,轉成灰後在大風處暴露,或在河中急流處沖走。比丘們!這樣,那棵大樹根被切斷,\twnr{[如]已斷根的棕櫚樹}{147.1},\twnr{成為非有}{408.0},\twnr{為未來不生之物}{229.0}。同樣的,比丘們!在與執取有關的法上住於隨看著過患者的渴愛被滅,以渴愛滅有取滅,以取滅有有滅……(中略)這樣是這整個苦蘊的滅。」



\sutta{56}{56}{大樹經第二}{https://agama.buddhason.org/SN/sn.php?keyword=12.56}
  住在舍衛城……(中略)。

  「\twnr{比丘}{31.0}們!猶如大樹,凡它向下走、向橫向走的諸根,那些全都向上帶來滋養素,比丘們!這樣,那棵大樹有\twnr{那個食物、那個燃料}{667.0},它會長久地、長時間地住立。同樣的,比丘們!在\twnr{與執取有關的}{551.0}法上住於隨看著樂味者的渴愛增長,以渴愛為緣有取(而取存在)……(中略)這樣是這整個\twnr{苦蘊}{83.0}的\twnr{集}{67.0}。

  比丘們!猶如大樹,那時,男子拿鋤頭、籃子後到來,他在根處切斷那棵樹,切斷後挖出根,挖出後拉出諸根……(中略[\suttaref{SN.12.55}])或在急流的河中沖走。比丘們!這樣,那棵大樹根被切斷,\twnr{[如]已斷根的棕櫚樹}{147.1},\twnr{成為非有}{408.0},\twnr{為未來不生之物}{229.0}。同樣的,比丘們!在與執取有關的法上住於隨看著過患者的渴愛被滅,以渴愛滅有取滅(而取滅存在)……(中略)這樣是這整個苦蘊的\twnr{滅}{68.0}。」



\sutta{57}{57}{幼樹經}{https://agama.buddhason.org/SN/sn.php?keyword=12.57}
  住在舍衛城……(中略)。

  「\twnr{比丘}{31.0}們!\twnr{在會被結縛的諸法上}{666.0}住於\twnr{隨看著}{59.0}\twnr{樂味}{295.0}者的渴愛增長,以渴愛\twnr{為緣}{180.0}有取(而取存在)……(中略)這樣是這整個\twnr{苦蘊}{83.0}的\twnr{集}{67.0}。

  比丘們!猶如幼樹,如果男子對它經常梳理諸根,經常給與土壤(培土),經常給與水(灌溉),比丘們!這樣,那棵幼樹有\twnr{那個食物、那個燃料}{667.0},它會來到成長、增長、成滿。同樣的,比丘們!在會被結縛的諸法上住於隨看樂味者的渴愛增長,以渴愛為緣有取……(中略)這樣是這整個苦蘊的集。

  比丘們!在會被結縛的諸法上住於隨看著\twnr{過患}{293.0}者的渴愛被滅,以渴愛滅有取滅(而取滅存在)……(中略)這樣是這整個苦蘊的\twnr{滅}{68.0}。

  比丘們!猶如幼樹,那時,男子拿鋤頭、籃子後到來……(中略[\suttaref{SN.12.55}])或在急流的河中沖走。比丘們!這樣,那棵幼樹根被切斷,\twnr{[如]已斷根的棕櫚樹}{147.1},\twnr{成為非有}{408.0},\twnr{為未來不生之物}{229.0}。同樣的,比丘們!在會被結縛的諸法上住於隨看著過患者的渴愛被滅,以渴愛滅有取滅……(中略)這樣是這整個苦蘊的滅。」



\sutta{58}{58}{名色經}{https://agama.buddhason.org/SN/sn.php?keyword=12.58}
  住在舍衛城……(中略)。

  「\twnr{比丘}{31.0}們!\twnr{在會被結縛的諸法上}{666.0}住於\twnr{隨看著}{59.0}\twnr{樂味}{295.0}者\twnr{有名色的下生}{673.0},以名色\twnr{為緣}{180.0}有六處(而六處存在)……(中略)這樣是這整個\twnr{苦蘊}{83.0}的\twnr{集}{67.0}。

  比丘們!猶如大樹,凡它向下走、向橫向走的諸根,那些全都向上帶來滋養素,比丘們!這樣,那棵大樹有\twnr{那個食物、那個燃料}{667.0},它會長久地、長時間地住立。同樣的,比丘們!在會被結縛的諸法上住於隨看著樂味者有名色的下生……(中略)。

  比丘們!在會被結縛的諸法上住於隨看著\twnr{過患}{293.0}者沒有名色的下生,以名色滅有六處滅(而六處滅存在)……(中略)這樣是這整個苦蘊的\twnr{滅}{68.0}。

  比丘們!猶如大樹,那時,男子拿鋤頭、籃子後到來……(中略[\suttaref{SN.12.55}])\twnr{為未來不生之物}{229.0}。同樣的,比丘們!在會被結縛的諸法上住於隨看著過患者沒有名色的下生,以名色滅有六處滅……(中略)這樣是這整個苦蘊的滅。」



\sutta{59}{59}{識經}{https://agama.buddhason.org/SN/sn.php?keyword=12.59}
  住在舍衛城……(中略)。

  「\twnr{比丘}{31.0}們!\twnr{在會被結縛的諸法上}{666.0}住於\twnr{隨看著}{59.0}\twnr{樂味}{295.0}者\twnr{有識的下生}{x273},以識\twnr{為緣}{180.0}有名色(而名色存在)……(中略)這樣是這整個\twnr{苦蘊}{83.0}的\twnr{集}{67.0}。

  比丘們!猶如大樹,凡它向下走、向橫向走的諸根……(中略[\suttaref{SN.12.55}])。同樣的,比丘們!在會被結縛的諸法上住於隨看著樂味者有識的下生……(中略)。

  比丘們!在會被結縛的諸法上住於隨看著\twnr{過患}{293.0}者沒有識的下生,以識滅有名色滅(而名色滅存在)……(中略)這樣是這整個苦蘊的\twnr{滅}{68.0}。

  比丘們!猶如大樹,那時,男子拿鋤頭、籃子後到來……(中略)\twnr{為未來不生之物}{229.0}。同樣的,比丘們!在會被結縛的諸法上住於隨看著過患者沒有識的下生,以識滅有名色滅……(中略)這樣是這整個苦蘊的滅。」



\sutta{60}{60}{因緣經}{https://agama.buddhason.org/SN/sn.php?keyword=12.60}
  \twnr{有一次}{2.0},\twnr{世尊}{12.0}住在俱盧國,名叫葛馬沙達馬的俱盧國城鎮。  

  那時,\twnr{尊者}{200.0}阿難去見世尊。抵達後,向世尊\twnr{問訊}{46.0}後,在一旁坐下。在一旁坐下的尊者阿難對世尊說這個:

  「不可思議啊,\twnr{大德}{45.0}!\twnr{未曾有}{206.0}啊,大德!大德!這緣起是多麼\twnr{甚深與顯現甚深}{x274},然而,對我,看起來像\twnr{明顯明顯的}{x275}。」

  「阿難!不要這樣[說-\ccchref{DN.15}{https://agama.buddhason.org/DN/dm.php?keyword=15}],阿難!不要這樣[說],阿難!這緣起是甚深與顯現甚深的。阿難!由於這個法的不隨覺、\twnr{不通達}{355.0},這樣,這\twnr{世代}{38.0}\twnr{變成糾纏線軸的}{802.0}、變成打結線球的,\twnr{成為蘆草與燈心草團的}{803.0},不超越\twnr{苦界}{109.0}、\twnr{惡趣}{110.0}、\twnr{下界}{111.0}、輪迴。

  阿難!對在\twnr{與執取有關的}{551.0}法上住於\twnr{隨看著}{59.0}\twnr{樂味}{295.0}者的渴愛增長,以渴愛\twnr{為緣}{180.0}有取(而取存在),以取為緣有有,以有為緣有生,以生為緣而老、死、愁、悲、苦、憂、\twnr{絕望}{342.0}生成,這樣是這整個\twnr{苦蘊}{83.0}的\twnr{集}{67.0}。

  阿難!猶如大樹,凡它向下走、向橫向走的諸根,那些全都向上帶來滋養素,阿難!這樣,那棵大樹有\twnr{那個食物、那個燃料}{667.0},它會長久地、長時間地住立。同樣的,阿難!對在與執取有關的法上住於隨看著樂味者的渴愛增長,以渴愛為緣有取,以取為緣有有……(中略)這樣是這整個苦蘊的集。

  阿難!對在與執取有關的法上住於隨看著\twnr{過患}{293.0}者的渴愛被滅,以渴愛滅有取滅(而取滅存在),以取滅有有滅……(中略)這樣是這整個苦蘊的\twnr{滅}{68.0}。

  阿難!猶如大樹,那時,男子拿鋤頭、籃子後到來,他在根處切斷那棵樹,在根處切斷後挖出,挖出後拉出諸根及乃至鬚根,他切那棵樹成細片,切成細片後使之破裂,使之破裂後做成碎碎的,做成碎碎的後在風熱中使之乾枯,在風熱中使之乾枯後以火燃燒,以火燃燒後轉成(作)灰,轉成灰後在大風處暴露,或在河中急流處沖走。阿難!這樣,那棵大樹根被切斷,\twnr{[如]已斷根的棕櫚樹}{147.1},\twnr{成為非有}{408.0},\twnr{為未來不生之物}{229.0}。同樣的,阿難!對在與執取有關的法上住於隨看著過患者的渴愛被滅,以渴愛滅有取滅,以取滅有有滅,以有滅有生滅,以生滅而老、死、愁、悲、苦、憂、絕望被滅,這樣是這整個苦蘊的滅。」

  苦品第六,其\twnr{攝頌}{35.0}:

  「審慮、執取,以及結二則,

   大樹二則被說,以幼樹為第七,

   名色、識,以及以因緣它們為十。」





\pin{大品}{61}{70}
\sutta{61}{61}{未聽聞經}{https://agama.buddhason.org/SN/sn.php?keyword=12.61}
  \twnr{被我這麼聽聞}{1.0}:

  \twnr{有一次}{2.0},\twnr{世尊}{12.0}住在舍衛城祇樹林給孤獨園……(中略)。

  「\twnr{比丘}{31.0}們!\twnr{未聽聞的一般人}{74.0}在這\twnr{四大}{646.0}之身上會\twnr{厭}{15.0}、\twnr{離染}{558.0}、被解脫,那是什麼原因?比丘們!這四大之身的成長、衰退、拿起、捨棄被看見,因此,在那裡,未聽聞的一般人會厭、離染、被解脫。

  比丘們!而凡這被這樣稱為『心』、『意』、『識』者,在那裡,未聽聞的一般人不足以厭,不足以離染,不足以被解脫,那是什麼原因?比丘們!因為對未聽聞的一般人,這個長久地被固執,被當做自己的,被執取:『\twnr{這是我的}{32.0},\twnr{我是這個}{33.0},\twnr{這是我的真我}{34.1}。』因此,在那裡,未聽聞的一般人不足以厭,不足以離染,不足以被解脫。

  比丘們!未聽聞的一般人如果握持這四大身為我比較好,而非心,那是什麼原因?比丘們!四大身一年住立著、二年住立著、三年住立著、四年住立著、五年住立著、十年住立著、二十年住立著、三十年住立著、四十年住立著、五十年住立著、一百年住立著、更久住立著被看見,比丘們!而凡這被稱為『心』、『意』、『識』者,它日與夜地都生起一個,\twnr{另一個被滅}{x276}。

  比丘們!猶如在林野、森林中行走的猴子,牠抓住樹枝,放開那枝後抓住另一枝,[再]放開那枝後抓住另一枝。同樣的,比丘們!凡這被稱為『心』、『意』、『識』者,它日與夜地都生起一個,另一個被滅。

  比丘們!在那裡,\twnr{有聽聞的聖弟子}{24.0}都徹底地\twnr{如理作意}{114.0}緣起:『像這樣,在這個存在時那個存在,以這個的生起那個生起。在這個不存在時那個不存在,以這個的滅\twnr{那個被滅}{394.0},即:以\twnr{無明}{207.0}\twnr{為緣}{180.0}有諸行(而諸行存在);以行為緣有識……(中略)這樣是這整個\twnr{苦蘊}{83.0}的\twnr{集}{67.0}。但以無明的\twnr{無餘褪去與滅}{491.0}有行滅(而行滅存在);以行滅有識滅……(中略)這樣是這整個苦蘊的滅。』

  比丘們!這麼看的有聽聞的聖弟子在色上厭,也在受上厭,也在想上厭,也在諸行上厭,也在識上厭。厭者離染,從\twnr{離貪}{77.0}被解脫,在已解脫時,\twnr{有『[這是]解脫』之智}{27.0},他知道:『\twnr{出生已盡}{18.0},\twnr{梵行已完成}{19.0},\twnr{應該被作的已作}{20.0},\twnr{不再有此處[輪迴]的狀態}{21.1}。』」



\sutta{62}{62}{未聽聞經第二}{https://agama.buddhason.org/SN/sn.php?keyword=12.62}
  住在舍衛城……(中略)。

  「\twnr{比丘}{31.0}們!\twnr{未聽聞的一般人}{74.0}在這\twnr{四大}{646.0}之身上會\twnr{厭}{15.0}、\twnr{離染}{558.0}、被解脫,那是什麼原因?比丘們!這四大之身的成長、衰退、拿起、捨棄被看見,因此,在那裡,未聽聞的一般人會厭、離染、被解脫。

  比丘們!而凡這被這樣稱為『心』、『意』、『識』者,在那裡,未聽聞的一般人不足以厭,不足以離染,不足以被解脫,那是什麼原因?比丘們!因為對未聽聞的一般人,這個長久地被固執,被當做自己的,被執取:『\twnr{這是我的}{32.0},\twnr{我是這個}{33.0},\twnr{這是我的真我}{34.1}。』因此,在那裡,未聽聞的一般人不足以厭,不足以離染,不足以被解脫。

  比丘們!未聽聞的一般人如果握持這四大身為我比較好,而非心,那是什麼原因?比丘們!四大身一年住立著、二年住立著、三年住立著、四年住立著、五年住立著、十年住立著、二十年住立著、三十年住立著、四十年住立著、五十年住立著、一百年住立著、更久住立著被看見,比丘們!而凡這被稱為『心』、『意』、『識』者,它日與夜地都生起一個,\twnr{另一個被滅}{x277}。

  比丘們!在那裡,\twnr{有聽聞的聖弟子}{24.0}都徹底地\twnr{如理作意}{114.0}緣起:『像這樣,在這個存在時那個存在,以這個的生起那個生起。在這個不存在時那個不存在,以這個的滅\twnr{那個被滅}{394.0}。』

  比丘們!\twnr{緣於}{252.0}能被感受為樂之\twnr{觸}{407.0},樂受生起,就以那個能被感受為樂之觸的\twnr{滅}{68.0},凡對應那個所感受的:緣於能被感受為樂之觸所生起的樂受,它被滅,它被平息。

  比丘們!緣於能被感受為苦之觸生起苦受,就以那個能被感受為苦之觸的滅,凡對應那個所感受的:緣於能被感受為苦之觸所生起的苦受,它被滅,它被平息。

  比丘們!緣於能被感受為不苦不樂之觸,不苦不樂受生起,就以那個能被感受為不苦不樂之觸的滅,凡對應那個所感受的:緣於能被感受為不苦不樂之觸生起的不苦不樂受,它被滅,它被平息。

  比丘們!猶如從兩塊柴的磨擦、結合,熱被產生,火生起。就從那兩塊柴的分離分置,凡對應那個的熱,它被滅,它被平息。同樣的,比丘們!緣於能被感受為樂之觸,樂受生起,就以那個能被感受為樂之觸的滅,凡對應那個所感受的:緣於能被感受為樂之觸所生起的樂受,它被滅,它被平息。……(中略)緣於能被感受為不苦不樂之觸,不苦不樂受生起,就以那個能被感受為不苦不樂之觸的滅,凡對應那個所感受的:緣於能被感受為不苦不樂之觸生起的不苦不樂受,它被滅,它被平息。[\suttaref{SN.36.10}, \suttaref{SN.48.39}, \ccchref{MN.140}{https://agama.buddhason.org/MN/dm.php?keyword=140}, 354-359段]

  比丘們!這麼看的有聽聞的聖弟子在觸上厭,也在受上厭,也在想上厭,也在諸行上厭,也在識上厭。厭者離染,從\twnr{離貪}{77.0}被解脫,在已解脫時,\twnr{有『[這是]解脫』之智}{27.0},他知道:『\twnr{出生已盡}{18.0},\twnr{梵行已完成}{19.0},\twnr{應該被作的已作}{20.0},\twnr{不再有此處[輪迴]的狀態}{21.1}。』」



\sutta{63}{63}{如兒子的肉經}{https://agama.buddhason.org/SN/sn.php?keyword=12.63}
  在舍衛城……(中略)。

  「\twnr{比丘}{31.0}們!有這\twnr{四種食}{241.0}:為了已生成眾生的存續,或為了\twnr{求出生者}{711.0}的資助。哪四種?\twnr{或粗或細的物質食物}{387.0},第二、\twnr{觸}{388.0},第三、\twnr{意思}{389.0},第四、\twnr{識}{390.0}。比丘們!這是四種食:為了已生成眾生的存續,或為了求出生者的資助。

  比丘們!物質食物應該怎樣被看待?

  比丘們!猶如夫婦二人取少的糧食後,走向荒野道路,他們有可愛的、合意的獨子[同行]。比丘們!那時,來到荒野的那夫婦二人的凡少量糧食,那個走到遍盡、耗盡,而還有他們剩餘未越過的荒野。比丘們!那時,那夫婦二人這麼想:『凡我們的少量糧食,那個已遍盡、已耗盡,而還有這剩餘未越過的荒野。讓我們殺這個可愛的、合意的獨子後,做肉乾與胡椒醃肉後,吃著兒子的肉,這樣我們越過那剩餘的荒野,不要我們三人全部都滅亡。』

  比丘們!那時,那夫婦二人殺那個可愛的、合意的獨子後,做肉乾與胡椒醃肉後,吃著兒子的肉,這樣越過那剩餘的荒野。他們吃兒子的肉,同時也搥胸:『在哪裡啊?獨子!在哪裡啊?獨子!』

  比丘們!你們怎麼想它:是否他們會為了享樂吃食物,或會為了陶醉吃食物,或會為了裝飾吃食物,或會\twnr{為了莊嚴}{520.0}吃食物呢?」

  「\twnr{大德}{45.0}!這確實不是。」

  「比丘們!他們最多為了荒野的越過目的吃食物,不是嗎?」

  「是的,大德!」

  「同樣的,比丘們!我說:『物質食物應該被這麼看待。』

  比丘們!在物質食物\twnr{被遍知}{x278}時,\twnr{五種欲的貪被遍知}{x279};在\twnr{五種欲}{187.0}的貪被遍知時,沒有那個結,聖弟子被該結結合會再來到這個世間。

  比丘們!觸食應該怎樣被看待?

  比丘們!猶如有無皮膚的母牛,如果牠依止牆站立,凡依止牆的那些生物類會叮咬牠;如果牠依止樹站立,凡依止樹的生物類會叮咬牠;如果牠依止水站立,凡依止水的生物類會叮咬牠;如果牠依止虛空站立,凡依止虛空的生物類會叮咬牠,比丘們!不論那頭無皮膚的母牛依止哪裡站立,凡依止該處的生物類都會叮咬牠。同樣的,比丘們!我說:『觸食應該這麼被看待。』

  比丘們!在觸食被遍知時,\twnr{三受被遍知}{x280};在三受被遍知時,我說:『聖弟子沒有任何更應該做的。』

  比丘們!意思食應該怎樣被看待?

  比丘們!猶如有超過一人深的炭火坑被無焰的、無煙的炭火充滿。那時,想要活命、不想要死,想要樂、厭逆苦的男子到來,兩位有力氣的男子隨即在不同手臂處捉住後,拉他向那個炭火坑。比丘們!那時,那位男子的思是遠離、希求是遠離、願求是遠離,那是什麼原因?比丘們!因為那位男子這麼想:『我將會跌落這個炭火坑,從那個因由我遭受死亡,或死亡程度的苦。』同樣的,比丘們!我說:『意思食應該這麼被看待。』

  比丘們!在意思食被遍知時,\twnr{三類渴愛}{244.0}被遍知;在三類渴愛被遍知時,我說:『聖弟子沒有任何更應該做的。』

  比丘們!識食應該怎樣被看待?

  比丘們!猶如捕捉盜賊、罪犯後他們對國王展示:『陛下!這位是你的盜賊、罪犯,請你對這位判決凡你想要的那個處罰。』

  國王隨即這麼說:『\twnr{先生}{202.0}!請你們去,請你們午前時以百槍擊這位男子。』午前時他們以百槍擊他。

  然後,中午時國王這麼說:『喂!那位男子怎樣了?』『大王!同樣活著。』國王隨即這麼說:『先生!請你們去,請你們中午時以百槍打擊那位男子。』中午時他們以百槍擊他。

  然後,傍晚時國王這麼說:『喂!那位男子怎樣了?』『大王!同樣活著。』國王隨即這麼說:『先生!請你們去,請你們傍晚時以百槍打擊那位男子。』傍晚時他們以百槍擊他。

  比丘們!你們怎麼想它:是否日間被三百槍打擊的那位男子,從那個因由感受苦憂呢?」

  「大德!即使被一槍打擊的那個男子,從那個因由感受苦憂,更不用說被三百槍。」

  「同樣的,比丘們!我說:『識食應該這麼被看待。』

  比丘們!在識食被遍知時,名色被遍知;在名色被遍知時,我說:『聖弟子沒有任何更應該做的。』」



\sutta{64}{64}{有貪經}{https://agama.buddhason.org/SN/sn.php?keyword=12.64}
  住在舍衛城……(中略)。

  「\twnr{比丘}{31.0}們!有這\twnr{四種食}{241.0}:為了已生成眾生的存續,或為了\twnr{求出生者}{711.0}的資助,哪四種?\twnr{或粗或細的物質食物}{387.0},第二、\twnr{觸}{388.0},第三、\twnr{意思}{389.0},第四、\twnr{識}{390.0}。比丘們!這是四種食:為了已生成眾生的存續,或為了求出生者的資助。

  比丘們!如果在物質食物上有貪,有歡喜,有渴愛,識在那裡被住立、被增長。識被住立、被增長之處,在那裡\twnr{有名色的下生}{673.0}。有名色的下生之處,在那裡有諸行的生長。有諸行的生長之處,在那裡有\twnr{未來再有的出生}{804.0}。有未來再有的出生之處,在那裡有未來的生、老、死。有未來的生、老、死之處,比丘們!我說:『那是有愁的,有悲的,有\twnr{絕望}{342.0}的。』

  比丘們!如果在觸食上……(中略)比丘們!如果在意思食上……比丘們!如果在識食上有貪,有歡喜,有渴愛,在那裡識被住立、被增長。識被住立、被增長之處,在那裡有名色的下生。有名色的下生之處,在那裡有諸行的增長。有諸行的增長之處,在那裡有未來再有的出生。有未來再有的出生之處,在那裡有未來的生、老、死。有未來的生、老、死之處,比丘們!我說:『那是有愁的,有悲的,有絕望的。』

  比丘們!猶如染工或畫家在有染料,或胭脂紅,或鬱金黃,或藍的,或深紅的時,在已善磨光的板上,或在壁上,或在白布上,創造全部大小肢體的男人形色或女人形色。同樣的,比丘們!如果在物質食物上有貪,有歡喜,有渴愛,在那裡識被住立、被增長。識被住立、被增長之處,在那裡有名色的下生。有名色的下生之處,在那裡有諸行的增長。有諸行的增長之處,在那裡有未來再有的出生。有未來再有的出生之處,在那裡有未來的生、老、死。有未來的生、老、死之處,比丘們!我說:『那是有愁的,有悲的,有絕望的。』

  比丘們!如果在觸食上……(中略)比丘們!如果在意思食上……(中略)比丘們!如果在識食上有貪,有歡喜,有渴愛,在那裡識被住立、被增長。識被住立、被增長之處,在那裡有名色的下生。有名色的下生之處,在那裡有諸行的增長。有諸行的增長之處,在那裡有未來再有的出生。有未來再有的出生之處,在那裡有未來的生、老、死。有未來的生、老、死之處,比丘們!我說:『那是有愁的,有悲的,有絕望的。』

  比丘們!如果在物質食物上沒有貪,沒有歡喜,沒有渴愛,在那裡識不被確立、不被增長。識不被確立、不被增長之處,在那裡沒有名色的下生。沒有名色的下生之處,在那裡沒有諸行的增長。沒有諸行的增長之處,在那裡沒有未來再有的出生。沒有未來再有的出生之處,在那裡沒有未來的生、老、死。沒有未來的生、老、死之處,比丘們!我說:『那是無愁的,無悲的,無絕望的。』

  比丘們!如果在觸食上……(中略)比丘們!如果在意思食上……比丘們!如果在識食上沒有貪,沒有歡喜,沒有渴愛,在那裡識不被確立、不被增長。識不被確立、不被增長之處,在那裡沒有名色的下生。沒有名色的下生之處,在那裡沒有諸行的增長。沒有諸行的生長之處,在那裡沒有未來再有的出生。沒有未來再有的出生之處,在那裡沒有未來的生、老、死。沒有未來的生、老、死之處,比丘們!我說:『那是無愁的,無悲的,無絕望的。』

  比丘們!猶如\twnr{重閣}{213.0}或重閣會堂,在北邊、南邊、東邊有窗戶,在太陽昇起時,光線經窗戶進入後,會被住立在何處?」

  「\twnr{大德}{45.0}!在西邊的牆壁。」

  「比丘們!如果西邊沒有牆壁,會被住立在何處?」

  「大德!在地上。」

  「比丘們!如果沒有地,會被住立在何處?」

  「大德!在水上。」

  「比丘們!如果沒有水,會被住立在何處?」

  「大德!不被住立。」

  「同樣的,比丘們!如果在物質食物上沒有貪,沒有歡喜,沒有渴愛……(中略)比丘們!如果在觸食上……比丘們!如果在意思食上……比丘們!如果在識食上沒有貪,沒有歡喜,沒有渴愛,在那裡識不被確立、不被增長。識不被確立、不被增長之處,在那裡沒有名色的下生。沒有名色的下生之處,在那裡沒有諸行的增長。沒有諸行的增長之處,在那裡沒有未來再有的出生。沒有未來再有的出生之處,在那裡沒有未來的生、老、死。沒有未來的生、老、死之處,比丘們!我說:『那是無愁的,無悲的,無絕望的。』」



\sutta{65}{65}{城市經}{https://agama.buddhason.org/SN/sn.php?keyword=12.65}
  住在舍衛城……(中略)。

  「\twnr{比丘}{31.0}們!當就在我\twnr{正覺}{185.1}以前,還是未\twnr{現正覺}{75.0}的\twnr{菩薩}{186.0}時想這個:『唉!這個世間確實已陷入苦難:被生、衰老、死去、\twnr{死沒}{x281}、再生,然而,不知道這老死苦的\twnr{出離}{294.0},什麼時候這老死苦的出離才將被知道?』

  比丘們!那個我想這個:『在什麼存在時老死存在呢?以什麼\twnr{為緣}{180.0}有老死(而老死存在)呢?』

  比丘們!那個我從\twnr{如理作意}{114.0},以慧有\twnr{現觀}{53.0}:『在生存在時老死存在;以生為緣有老死。』

  比丘們!那個我想這個:『在什麼存在時生存在呢?……(中略)有存在……取存在……渴愛存在……受存在……觸存在……六處存在……名色存在?以什麼為緣有名色?』

  比丘們!那個我從如理作意,以慧有現觀:『在識存在時名色存在;以識為緣有名色。』

  比丘們!那個我想這個:『在什麼存在時識存在?以什麼為緣有識呢?』

  比丘們!那個我從如理作意,以慧有現觀:『在名色存在時識存在;以名色為緣有識。』

  比丘們!那個我想這個:『這個識從名色回轉,\twnr{不更進一步走}{x282},\twnr{就這個範圍}{x283},會被生,或會被衰老,或會死去,或會死沒,或會再生,即:以名色為緣有識;以識為緣有名色;以名色為緣有六處;以六處為緣有觸……(中略)這樣是這整個\twnr{苦蘊}{83.0}的\twnr{集}{67.0}。』

  『\twnr{集!集!}{x238}』

  比丘們!在以前不曾聽過的諸法上,我的眼生起,智生起,慧生起,明生起,\twnr{光生起}{511.0}。

  比丘們!那個我想這個:『在什麼不存在時老死不存在呢?以什麼滅有老死滅(而老死滅存在)?』

  比丘們!那個我從如理作意,以慧有現觀:『在生不存在時老死不存在;以生滅有老死滅。』

  比丘們!那個我想這個:『在什麼不存在時生不存在呢?……(中略)有不存在……取不存在……渴愛不存在……受不存在……觸不存在……六處不存在……名色不存在?以什麼滅有名色滅?』

  比丘們!那個我從如理作意,以慧有現觀:『在識不存在時名色不存在;以識滅有名色滅。』

  比丘們!那個我想這個:『在什麼不存在時識不存在?以什麼滅有識滅呢?』

  比丘們!那個我從如理作意,以慧有現觀:『在名色不存在時識不存在;以名色滅有識滅。』

  比丘們!那個我想這個:『這為了覺的道路被我證得,即:以名色滅有識滅;以識滅有名色滅;以名色滅有六處滅;以六處滅有觸滅……(中略)這樣是這整個苦蘊的滅。』

  『滅!滅!』

  比丘們!在以前不曾聽過的諸法上,我的眼生起,智生起,慧生起,明生起,光生起。

  比丘們!猶如當在\twnr{林野}{142.0}、森林中漫遊時,男子看見古道;被以前人們隨行的古徑,他跟隨那個。當跟隨那個時,他看見被以前人們居住,林園、森林、蓮花池具足的,有城壁的,美麗的古王都古城。

  比丘們!那時,那位男子告知國王或國王的大臣:『真的,\twnr{大德}{45.0}!你應該知道,當在林野、森林中漫遊時,我看見古道;被以前人們隨行的古徑,我跟隨那個。當跟隨那個時,我看見被以前人們居住,林園、森林、蓮花池具足的,有城壁的,美麗的古王都古城。大德!請你建造那個城市吧。』

  比丘們!那時,那位國王或國王的大臣建造那個城市。過些時候,那個城市就是繁榮的,同時也富裕的,以及人多的、人雜亂的,到達增長的、廣大的。同樣的,比丘們!我看見古道;被過去遍正覺者們隨行的古徑。

  比丘們!而什麼是那個古道;被過去遍正覺者們隨行的古徑呢?就是這\twnr{八支聖道}{525.0},即:正見……(中略)正定。比丘們!這個古道;被過去遍正覺者們隨行的古徑,我隨行它,當隨行它時,我\twnr{證知}{242.0}老死,證知老死集,證知老死滅,證知導向老死滅道跡。我隨行它,當隨行它時,我證知生……(中略)證知有……證知取……證知渴愛……證知受……證知觸……證知六處……證知名色……證知識……我隨行它,當隨行它時,我證知諸行,證知行集,證知行滅,證知導向行滅道跡。證知它後,我告知比丘們、比丘尼們、優婆塞們、優婆夷們。比丘們!這梵行成為成功的、繁榮的、被廣知的、人口眾多的、廣為流傳的,直到使被天-人們善知道。」



\sutta{66}{66}{觸知經}{https://agama.buddhason.org/SN/sn.php?keyword=12.66}
  \twnr{被我這麼聽聞}{1.0}:

  \twnr{有一次}{2.0},\twnr{世尊}{12.0}住在俱盧國,名叫葛馬沙達馬的俱盧國城鎮。

  在那裡,世尊召喚\twnr{比丘}{31.0}們:「比丘們!」

  「\twnr{尊師}{480.0}!」那些比丘回答世尊。

  世尊說這個:「比丘們!\twnr{你們觸知}{x284}內在的觸知嗎?」

  在這麼說時,某位比丘對世尊說這個:「大德!我觸知\twnr{內在的觸知}{x285}。」

  「比丘!那麼,你如怎樣觸知內在的觸知?」

  那時,那位比丘回答。如那位比丘回答,那位比丘不使世尊的心合意。

  在這麼說時,\twnr{尊者}{200.0}阿難對世尊說這個:

  「世尊!是為了這個的適當時機,\twnr{善逝}{8.0}!是為了這個的適當時機,凡世尊如果說內在的觸知者,聽聞世尊的[教說]後,比丘們將會\twnr{憶持}{57.0}。」

  「阿難!那樣的話,你們要聽!你們要\twnr{好好作意}{43.1}!我將說。」

  「是的,\twnr{大德}{45.0}!」那些比丘回答世尊。

  世尊說這個:

  「比丘們!這裡,當比丘觸知時,他觸知內在的觸知:『凡這種種不同種類的老死苦在世間生起,這個苦,什麼為因?什麼為集?什麼生的?\twnr{什麼為根源}{668.0}?在什麼存在時老死存在?在什麼不存在時老死不存在?』

  當他觸知時,他這麼知道:『凡這種種不同種類的老死苦在世間生起,這個苦,\twnr{依著}{198.0}為因,依著為集,依著生的,依著為根源,在依著存在時老死存在;在依著不存在時老死不存在。』

  他知道老死、知道老死集、知道老死滅、知道適合導向老死\twnr{滅道跡}{69.0}。而像這樣的行者是\twnr{隨法行者}{765.0},比丘們!這位比丘被稱為,為了全部苦的完全滅盡、為了老死滅的行者。

  然後,當更進一步觸知時,他觸知內在的觸知:『那麼,這個依著,什麼為因?什麼為集?什麼生的?什麼為根源?在什麼存在時依著存在?在什麼不存在時依著不存在?』

  當他觸知時,他這麼知道:『依著,渴愛為因,渴愛為集,渴愛生的,渴愛為根源,在渴愛存在時依著存在;在渴愛不存在時依著不存在。』

  他知道依著、知道依著的集、知道依著的滅、知道適合導向依著滅道跡。而像這樣的行者是隨法行者,比丘們!這位比丘被稱為,為了全部苦的完全滅盡、為了依著滅的行者。

  然後,當更進一步觸知時,他觸知內在的觸知:『那麼,這個渴愛當生起時,在哪裡生起?當安頓時,在哪裡安頓?』

  當他觸知時,他這麼知道:『凡世間中的可愛形色、\twnr{合意形色}{962.0},這個渴愛當生起時,在這裡生起,當安頓時,在這裡安頓。』

  而什麼是世間中的可愛形色、合意形色?

  眼是世間中的可愛形色、合意形色,這個渴愛當生起時,在這裡生起,當安頓時,在這裡安頓;耳是世間中的可愛形色、合意形色……(中略)鼻是世間中的可愛形色、合意形色……舌是世間中的可愛形色、合意形色……身是世間中的可愛形色、合意形色……意是世間中的可愛形色、合意形色,這個渴愛當生起時,在這裡生起,當安頓時,在這裡安頓。[\ccchref{MN.10}{https://agama.buddhason.org/MN/dm.php?keyword=10}, 133段 \ccchref{DN.22}{https://agama.buddhason.org/DN/dm.php?keyword=22}, 400段]

  比丘們!凡過去世任何\twnr{沙門}{29.0}或\twnr{婆羅門}{17.0}對凡世間中的可愛形色、合意形色看作是常的,看作是樂的,看作是我,看作是無病,看作是安穩者,他們增長渴愛。凡增長渴愛者,他們增長依著。凡增長依著者,他們增長苦。凡增長苦者,他們不被生、老、死、愁、悲、苦、憂、\twnr{絕望}{342.0}被釋放,我說:『不從苦被釋放。』

  比丘們!凡\twnr{未來世}{308.0}任何沙門或婆羅門對凡世間中的可愛形色、合意形色看作是常的,看作是樂的,看作是我,看作是無病,看作是安穩者,他們增長渴愛。凡增長渴愛者,他們增長依著。凡增長依著者,他們增長苦。凡增長苦者,他們不被生、老、死、愁、悲、苦、憂、絕望被釋放,我說:『不從苦被釋放。』

  比丘們!凡現在任何沙門或婆羅門對凡世間中的可愛形色、合意形色看作是常的,看作是樂的,看作是我,看作是無病,看作是安穩者,他們增長渴愛。凡增長渴愛者,他們增長依著。凡增長依著者,他們增長苦。凡增長苦者,他們不被生、老、死、愁、悲、苦、憂、絕望被釋放,我說:『不從苦被釋放。』

  比丘們!猶如有杯\twnr{容色具足}{713.0}、香氣具足、食味具足的[酒],但它被參雜毒。那時,被熱壓迫的、被熱折磨的、疲累的、乾涸的、口渴的男子到來,他們對他說這個:『喂!男子!這杯是你的容色具足、香氣具足、食味具足的[酒],但它被參雜毒。如果你願意,請你喝,當喝時,它以顏色,又以香氣,又以食味取悅,但喝後,你從那個因由將遭受死亡,或死亡程度的苦。』未省察後他急速地喝那杯,不斷念,他從那個因由遭受死亡,或死亡程度的苦。同樣的,比丘們!凡過去世任何沙門或婆羅門對凡世間中的可愛形色、合意形色……(中略)未來世的……(中略)現在任何沙門或婆羅門對凡世間中的可愛形色、合意形色看作是常的,看作是樂的,看作是我,看作是無病,看作是安穩者,他們增長渴愛。凡增長渴愛者,他們增長依著。凡增長依著者,他們增長苦。凡增長苦者,他們不被生、老、死、愁、悲、苦、憂、絕望被釋放,我說:『不從苦被釋放。』

  比丘們!但,凡過去世任何沙門或婆羅門對凡世間中的可愛形色、合意形色看作是無常的,看作是苦的,看作是無我,看作是病的,看作是恐怖的者,他們捨斷渴愛。凡捨斷渴愛者,他們捨斷依著。凡捨斷依著者,他們捨斷苦。凡捨斷苦者,他們被生、老、死、愁、悲、苦、憂、絕望被釋放,我說:『從苦被釋放。』

  比丘們!凡未來世任何沙門或婆羅門對凡世間中的可愛形色、合意形色看作是無常的,看作是苦的,看作是無我,看作是病的,看作是恐怖的者,他們捨斷渴愛……(中略)我說:『從苦被釋放。』

  比丘們!凡現在任何沙門或婆羅門對凡世間中的可愛形色、合意形色看作是無常的,看作是苦的,看作是無我,看作是病的,看作是恐怖的者,他們捨斷渴愛。凡捨斷渴愛者,他們捨斷依著。凡捨斷依著者,他們捨斷苦。凡捨斷苦者,他們被生、老、死、愁、悲、苦、憂、絕望被釋放,我說:『從苦被釋放。』

  比丘們!猶如有杯容色具足、香氣具足、食味具足的[酒],但它被參雜毒。那時,被熱壓迫的、被熱折磨的、疲累的、乾涸的、口渴的男子到來,他們對他說這個:『喂!男子!這杯是你的容色具足、香氣具足、食味具足的[酒],但它被參雜毒。如果你願意,請你喝,當喝時,它以顏色,又以香氣,又以食味取悅,但喝後,你從那個因由將遭受死亡,或死亡程度的苦。』

  比丘們!那時,那位男子這麼想:『我的這個渴求酒的,能以水除去(調伏),或以生酥除去,或以\twnr{鹹麵漿}{x286}除去,或以\twnr{鹹酸粥}{x287}除去,就像這樣我不喝它:凡對我有長久的利益、安樂。』省察後他不喝那杯,他斷念,他不從那個因由遭受死亡或會死亡程度的苦。同樣的,比丘們!凡過去世任何沙門或婆羅門對凡世間中的可愛形色、合意形色看作是無常的,看作是苦的,看作是無我,看作是病的,看作是恐怖的者,他們捨斷渴愛。凡捨斷渴愛者,他們捨斷依著。凡捨斷依著者,他們捨斷苦。凡捨斷苦者,他們被生、老、死、愁、悲、苦、憂、絕望被釋放,我說:『從苦被釋放。』

  比丘們!凡未來世的……(中略)凡現在任何沙門或婆羅門對凡世間中的可愛形色、合意形色看作是無常的,看作是苦的,看作是無我,看作是病的,看作是恐怖的者,他們捨斷渴愛。凡捨斷渴愛者,他們捨斷依著。凡捨斷依著者,他們捨斷苦。凡捨斷苦者,他們被生、老、死、愁、悲、苦、憂、絕望被釋放,我說:『從苦被釋放。』」



\sutta{67}{67}{蘆葦束經}{https://agama.buddhason.org/SN/sn.php?keyword=12.67}
  \twnr{有一次}{2.0},\twnr{尊者}{200.0}舍利弗與尊者摩訶拘絺羅,住在波羅奈仙人墜落處的鹿林。

  那時,尊者摩訶拘絺羅傍晚時,從\twnr{獨坐}{92.0}出來,去見尊者舍利弗。抵達後,與尊者舍利弗一起互相問候。交換應該被互相問候的友好交談後,在一旁坐下。在一旁坐下的尊者摩訶拘絺羅對尊者舍利弗說這個:

  「舍利弗\twnr{學友}{201.0}!怎麼樣,老死是自己作的嗎?其他作的?\twnr{自己作的與其他者作的}{172.1}?或非自己非其他作的;\twnr{自然生的}{173.0}?」

  「拘絺羅學友!老死不是自己作的、不是其他作的、不是自己作的與其他者作的、也不是非自己非他人作的;自然生的,而是以生\twnr{為緣}{180.0}有老死(而老死存在)。」

  「舍利弗學友!怎麼樣,生是自己作的嗎?其他作的?自己作的與其他者作的?或非自己非其他作的;自然生的?」

  「拘絺羅學友!生不是自己作的、不是其他作的、不是自己作的與其他者作的、也不是非自己非他人作的;自然生的,而是以有為緣有生。」

  「舍利弗學友!怎麼樣,有是自己作的嗎?……(中略)取是自己作的嗎?……渴愛是自己作的嗎?……受是自己作的嗎?……觸是自己作的嗎?……六處是自己作的嗎?……名色是自己作的嗎?其他作的?自己作的與其他者作的?或非自己非其他作的;自然生的?」 

  「拘絺羅學友!名色不是自己作的、不是其他作的、不是自己作的與其他者作的、也不是非自己非其他作的;自然生的,而是以識為緣有名色。」

  「舍利弗學友!怎麼樣,識是自己作的嗎?其他作的?自己作的與其他者作的?或非自己非其他作的;自然生的?」

  「拘絺羅學友!識不是自己作的、不是其他作的、不是自己作的與其他者作的、也不是非自己非其他作的;自然生的,而是以名色為緣有識。」

  「就現在,我們這麼了知尊者舍利弗所說:『拘絺羅學友!名色不是自己作的、不是其他作的、不是自己作的與其他者作的、也不是非自己非其他作的;自然生的,而是以識為緣有名色。』但,就現在,我們又這麼了知尊者舍利弗所說:『拘絺羅學友!識不是自己作的、不是其他作的、不是自己作的與其他者作的、也不是非自己非其他作的;自然生的,而是以名色為緣有識。』舍利弗學友!那麼,如怎樣這所說的義理應該被看見?」

  「學友!那樣的話,我將為你作譬喻,這裡,一些有智的男子也以譬喻了知所說的義理。學友!猶如兩把蘆葦束互相依靠後站立。同樣的,學友!以名色為緣有識;以識為緣有名色;以名色為緣有六處;以六處為緣有觸……(中略)這樣是這整個\twnr{苦蘊}{83.0}的\twnr{集}{67.0}。

  學友!如果拉那些蘆葦束中之一個,另一個倒下;如果拉另一個(另外的),這一個(另外的)倒下。同樣的,學友!以名色\twnr{滅}{68.0}有識滅(而識滅存在);以識滅有名色滅,以名色滅有六處滅;以六處滅有觸滅……(中略)這樣是這整個苦蘊的滅。」

  「不可思議啊,舍利弗學友!\twnr{未曾有}{206.0}啊,舍利弗學友!而這被尊者舍利弗多麼善說。而且,我們以這三十六事隨喜這位尊者舍利弗的所說:『學友!如果\twnr{比丘}{31.0}對老死是為了\twnr{厭}{15.0}、\twnr{離貪}{77.0}、\twnr{滅}{68.0}而教導法,「說法者比丘」是適當的言語。

  學友!如果比丘是對老死是為了厭、離貪、滅的行者,「\twnr{法、隨法行者}{58.0}比丘」是適當的言語。

  學友!如果比丘對老死從\twnr{厭}{15.0}、\twnr{離貪}{77.0}、滅,不執取後成為解脫者,「得當生涅槃者比丘」是適當的言語。

  學友!如果比丘教導對生……對有……對取……對渴愛……對受……對觸……對六處……對名色……對識……對行……如果比丘對\twnr{無明}{207.0}是為了厭、離貪、滅而教導法,「說法者比丘」是適當的言語。

  學友!如果比丘對無明是為了厭、離貪、滅的行者,「法、隨法行者比丘」是適當的言語。

  學友!如果比丘對無明從厭、離貪、滅,不執取後成為解脫者,「得當生涅槃者比丘。」是適當的言語』」



\sutta{68}{68}{憍賞彌經}{https://agama.buddhason.org/SN/sn.php?keyword=12.68}
  \twnr{有一次}{2.0},\twnr{尊者}{200.0}茂師羅、尊者殊勝、尊者那羅、尊者阿難住在\twnr{憍賞彌}{140.0}瞿師羅園。

  那時,尊者殊勝對尊者茂師羅說這個:

  「茂師羅\twnr{學友}{201.0}!\twnr{除了}{908.0}就從信[某人],除了從愛好,除了從口傳,除了從理由的遍尋思,除了\twnr{從見解的審慮接受}{609.0}外,尊者茂師羅就有自己的智:『以生\twnr{為緣}{180.0}有老死(而老死存在)』嗎?」

  「殊勝學友!除了就從信[某人],除了從[個人的]愛好,除了從口傳,除了從理由的遍尋思,除了從見解的審慮接受外,我知道這個,我看見這個:『以生為緣有老死』。」

  「茂師羅學友!除了就從信,除了從愛好,除了從口傳,除了從理由的遍尋思,除了從見解的審慮接受外,尊者茂師羅就有自己的智:『以有為緣有生』……(中略)『以取為緣有有』……『以渴愛為緣有取』……『以受為緣有渴愛』……『以觸為緣有受』……『以六處為緣有觸』……『以名色為緣有六處』……『以識為緣有名色』……『以行為緣有識』……『以\twnr{無明}{207.0}為緣有諸行』嗎?」

  「殊勝學友!除了就從信,除了從愛好,除了從口傳,除了從理由的遍尋思,除了從見解的審慮接受外,我知道這個,我看見這個:『以無明為緣有諸行』。」

  「茂師羅學友!除了就從信,除了從愛好,除了從口傳,除了從理由的遍尋思,除了從見解的審慮接受外,尊者茂師羅就有自己的智:『以生滅有老死滅(老死滅存在)』嗎?」

  「殊勝學友!除了就從信,除了從愛好,除了從口傳,除了從理由的遍尋思,除了從見解的審慮接受外,我知道這個,我看見這個:『以生滅有老死滅』。」

  「茂師羅學友!除了就從信,除了從愛好,除了從口傳,除了從理由的遍尋思,除了從見解的審慮接受外,尊者茂師羅就有自己的智:『以有滅有生滅』……(中略)『以取滅有有滅』……(中略)『以渴愛滅有取滅』……『以受滅有渴愛滅』……『以觸滅有受滅』……『以六處滅有觸滅』……『以名色\twnr{滅}{68.0}而六處滅』……『以識滅有名色滅』……『以行滅有識滅』……『以無明滅有行滅』嗎?」

  「殊勝學友!除了就從信,除了從愛好,除了從口傳,除了從理由的遍尋思,除了從見解的審慮接受外,我知道這個,我看見這個:『以無明滅有行滅』。」

  「茂師羅學友!除了就從信,除了從愛好,除了從口傳,除了從理由的遍尋思,除了從見解的審慮接受外,尊者茂師羅就有自己的智:『\twnr{有之滅為涅槃}{x288}』嗎?」

  「殊勝學友!除了就從信,除了從愛好,除了從口傳,除了從理由的遍尋思,除了從見解的審慮接受外,我知道這個,我看見這個:『有之滅為涅槃』。」

  「那樣的話,尊者茂師羅是諸漏已滅盡的\twnr{阿羅漢}{5.0}了。」

  在這麼說時,尊者茂師羅保持沈默。

  那時,尊者那羅對尊者殊勝說這個:

  「殊勝學友!如果我得到這個問題,\twnr{那就好了}{44.0}!請你問我這個問題,我將回答你這個問題。」

  「請尊者那羅得到這個問題,我問尊者那羅這個問題,且請尊者那羅回答我這個問題。那羅學友!除了就從信,除了從愛好,除了從口傳,除了從理由的遍尋思,除了從見解的審慮接受外,尊者那羅就有自己的智:『以生為緣有老死』嗎?」

  「殊勝學友!除了就從信,除了從愛好,除了從口傳,除了從理由的遍尋思,除了從見解的審慮接受外,我知道這個,我看見這個:『以生為緣有老死』。」

  「那羅學友!除了就從信,除了從愛好,除了從口傳,除了從理由的遍尋思,除了從見解的審慮接受外,尊者那羅就有自己的智:『以有為緣有生』……(中略)『以無明為緣有諸行』嗎?」

  「殊勝學友!除了就從信,除了從愛好,除了從口傳,除了從理由的遍尋思,除了從見解的審慮接受外,我知道這個,我看見這個:『以無明為緣有諸行』。」

  「那羅學友!除了就從信,除了從愛好,除了從口傳,除了從理由的遍尋思,除了從見解的審慮接受外,尊者那羅就有自己的智:『以生滅有老死滅』嗎?」

  「殊勝學友!除了就從信,除了從愛好,除了從口傳,除了從理由的遍尋思,除了從見解的審慮接受外我知道這個,我看見這個:『以生滅有老死滅』。」

  「那羅學友!除了就從信,除了從愛好,除了從口傳,除了從理由的遍尋思,除了從見解的審慮接受外,尊者那羅就有自己的智:『以有滅有生滅』……(中略)『以無明滅有行滅』嗎?」

  「殊勝學友!除了就從信,除了從愛好,除了從口傳,除了從理由的遍尋思,除了從見解的審慮接受外,我知道這個,我看見這個:『以無明滅有行滅』。」

  「那羅學友!除了就從信,除了從愛好,除了從口傳,除了從理由的遍尋思,除了從見解的審慮接受外,尊者那羅就有自己的智:『有之滅為涅槃』嗎?」

  「殊勝學友!除了就從信,除了從愛好,除了從口傳,除了從理由的遍尋思,除了從見解的審慮接受外,我知道這個,我看見這個:『有之滅為涅槃』。」

  「那樣的話,尊者那羅是諸漏已滅盡的阿羅漢了。」

  「學友!『有之滅為涅槃』被我以正確之慧如實善見,但我不是諸漏已滅盡的阿羅漢。學友!猶如在荒野道路處有水井,在那裡,既沒有繩,也沒有水桶。那時,被熱壓迫的、被熱折磨的、疲累的、乾涸的、口渴的男子到來,他注視那個水井,他有『是水』之智,\twnr{但非以身觸達後而住}{x289}。同樣的,學友!『有之滅為涅槃』被[我]以正確之慧如實善見,但我不是諸漏已滅盡的阿羅漢。」

  在這麼說時,尊者阿難對尊者殊勝說這個:「殊勝學友!對這麼說的尊者那羅,你怎麼說?」

  「阿難學友!對這麼說的尊者那羅,除了好,除了善巧外,我不說什麼了。」



\sutta{69}{69}{高漲經}{https://agama.buddhason.org/SN/sn.php?keyword=12.69}
  \twnr{被我這麼聽聞}{1.0}:

  \twnr{有一次}{2.0},\twnr{世尊}{12.0}住在舍衛城祇樹林給孤獨園。

  在那裡……(中略)

  「\twnr{比丘}{31.0}們!\twnr{高漲}{x290}的大海使諸大河高漲;高漲的諸大河使諸小河高漲;高漲的諸小河使諸大池高漲;高漲的諸大池使諸小池高漲。同樣的,比丘們!高漲的無明使諸行高漲;高漲的諸行使識高漲;高漲的識使名色高漲;高漲的名色使六處高漲;高漲的六處使觸高漲;高漲的觸使受高漲;高漲的受使渴愛高漲;高漲的渴愛使取高漲;高漲的取使有高漲;高漲的有使生高漲;高漲的生使老死高漲。

  比丘們!消退的大海使諸大河消退;消退的諸大河使諸小河消退;消退的諸小河使諸大池消退;消退的諸大池使諸小池消退。同樣的,比丘們!消退的無明使諸行消退;消退的諸行使識消退;消退的識使名色消退;消退的名色使六處消退;消退的六處使觸消退;消退的觸使受消退;消退的受使渴愛消退;消退的渴愛使取消退;消退的取使有消退;消退的有使生消退;消退的生使老死消退。



\sutta{70}{70}{蘇尸摩經}{https://agama.buddhason.org/SN/sn.php?keyword=12.70}
  \twnr{被我這麼聽聞}{1.0}:

  \twnr{有一次}{2.0},\twnr{世尊}{12.0}住在王舍城栗鼠飼養處的竹林中。

  當時,世尊被恭敬、被敬重、被尊重、被禮敬、被崇敬,是衣服、\twnr{施食}{196.0}、臥坐處、病人需物、醫藥必需品的利得者,\twnr{比丘}{31.0}\twnr{僧團}{375.0}也被恭敬、被敬重、被尊重、被禮敬、被崇敬,是衣服、施食、臥坐處、病人需物、醫藥必需品的利得者。而其他外道\twnr{遊行者}{79.0}們不被恭敬、不被敬重、不被尊重、不被禮敬、不被崇敬,不是衣服、施食、臥坐處、病人需物、醫藥必需品的利得者。

  當時,遊行者蘇尸摩與大遊行者群眾一起住在王舍城。

  那時,蘇尸摩的遊行者群眾對遊行者蘇尸摩說這個:

  「來!蘇尸摩\twnr{學友}{201.0}!請你在\twnr{沙門}{29.0}\twnr{喬達摩}{80.0}處行梵行,學得法後,願你教我們,我們學得那個法後,將對在家人說,這樣,我們也將被恭敬、被敬重、被尊重、被禮敬、被崇敬,將是衣服、施食、臥坐處、病人需物、醫藥必需品的利得者。」

  「是的,學友!」遊行者蘇尸摩回答自己的群眾後,去見\twnr{尊者}{200.0}阿難。抵達後,與尊者阿難一起互相問候。交換應該被互相問候的友好交談後,在一旁坐下。在一旁坐下的遊行者蘇尸摩對尊者阿難說這個:

  「阿難道友!我想要在這法、律中行梵行。」

  那時,尊者阿難帶著遊行者蘇尸摩去見世尊。抵達後,向世尊\twnr{問訊}{46.0}後,在一旁坐下。在一旁坐下的尊者阿難對世尊說這個:

  「\twnr{大德}{45.0}!這位遊行者蘇尸摩說這個:『阿難道友!我想要在這法、律中行梵行。』」

  「阿難!那樣的話,請你們使蘇尸摩出家。」

  遊行者蘇尸摩在世尊的面前得到出家,\twnr{得到具足戒}{124.1}。

  又,當時,\twnr{完全智}{489.0}被眾多比丘在世尊的面前記說:「我們知道:『\twnr{出生已盡}{18.0},\twnr{梵行已完成}{19.0},\twnr{應該被作的已作}{20.0},\twnr{不再有此處[輪迴]的狀態}{21.1}。』」

  尊者蘇尸摩聽聞:「聽說完全智被眾多比丘在世尊的面前記說:『我們知道:「出生已盡,梵行已完成,應該被作的已作,不再有此處[輪迴]的狀態。」』」

  那時,尊者蘇尸摩去見那些比丘。抵達後,與那些比丘一起互相問候。交換應該被互相問候的友好交談後,在一旁坐下。在一旁坐下的尊者蘇尸摩對那些比丘說這個:「傳說是真的?完全智被尊者們在世尊的面前記說:『我們知道:「出生已盡,梵行已完成,應該被作的已作,不再有此處[輪迴]的狀態。」』」「是的,學友!」

  「尊者們!但這麼知的、這麼見的你們體驗各種神通種類:是一個後變成多個,又,是多個後變成一個;現身、隱身、穿牆、穿壘、穿山無阻礙地行走猶如在虛空中;在地中作浮沈猶如在水中,又,在不被破裂的水上行走猶如在地上;在空中以盤腿來去猶如有翅膀的鳥,又,以手碰觸、撫摸這些這麼大神通力、這麼大威力的日月;以身體行使自在直到梵天世界嗎?」「學友!這確實不是。」

  「尊者們!但這麼知的、這麼見的你們以清淨、超越常人的天耳界聽到二者的聲音:「天與人,以及在遠處、近處。」嗎?」「學友!這確實不是。」

  「尊者們!但這麼知的、這麼見的你們對其他眾生、其他個人\twnr{以心熟知心後}{393.0}知道:有貪的心為『有貪的心』,或知道離貪的心為『離貪的心』,或知道有瞋的心為『有瞋的心』,或知道離瞋的心為『離瞋的心』,或知道有癡的心為『有癡的心』,或知道離癡的心為『離癡的心』,或知道\twnr{收斂的心}{674.0}為『收斂的心』,或知道散亂的心為『散亂的心』,或知道廣大的心為『廣大的心』,或知道非廣大的心為『非廣大的心』,或知道有更上的心為『有更上的心』,或知道無更上的心為『無更上的心』,或知道得定的心為『得定的心』,或知道未得定的心為『未得定的心』,或知道已解脫的心為『已解脫的心』,或知道未解脫的心為『未解脫的心』嗎?」「學友!這確實不是。」

  「尊者們!但這麼知的、這麼見的你們回憶(隨念)許多前世住處,即:一生、二生、三生、四生、五生、十生、二十生、三十生、四十生、五十生、百生、千生、十萬生、許多壞劫、許多成劫、許多\twnr{壞成劫}{403.0}:『在那裡我是這樣的名、這樣的姓氏、這樣的容貌、這樣的食物、這樣的苦樂感受、這樣的壽長,那位從那裡死後我出生在那裡,在那裡我又是這樣的名、這樣的姓氏、這樣的容貌、這樣的食物、這樣的苦樂感受、這樣的壽長,那位從那裡死後被再生於這裡。』像這樣,你們回憶許多\twnr{有行相的、有境遇的}{500.0}前世住處嗎?」「學友!這確實不是。」

  「尊者們!但這麼知的、這麼見的你們以清淨、超越常人的天眼看見死沒往生的眾生:下劣的、勝妙的,美的、醜的,善去的、惡去的,知道依業到達的眾生:『確實,這些尊師眾生具備身惡行、具備語惡行、具備意惡行,是對聖者斥責者、邪見者、邪見行為的受持者,他們以身體的崩解,死後已往生\twnr{苦界}{109.0}、\twnr{惡趣}{110.0}、\twnr{下界}{111.0}、地獄,又或這些尊師眾生具備身善行、具備語善行、具備意善行,是對聖者不斥責者、正見者、正見行為的受持者,他們以身體的崩解,死後已往生\twnr{善趣}{112.0}、天界。』像這樣,你們以清淨、超越常人的天眼看見死沒往生的眾生:下劣的、勝妙的,美的、醜的,善去的、惡去的,知道依業到達的眾生嗎?」「學友!這確實不是。」

  「尊者們!但這麼知的、這麼見的你們,凡那些\twnr{超越色}{x291}的無色\twnr{寂靜解脫}{501.0},你們以身觸達後住於那些嗎?」「學友!這確實不是。」

  「尊者們!現在,這裡,這個回答與這些法的未\twnr{等至}{129.0},學友們!這是怎樣呢?」

  「蘇尸摩學友!我們是\twnr{慧解脫者}{539.0}。」

  「我對這個被尊者們簡要地說的,不詳細地了知義理,請尊者們為我說,如是我能對這個被尊者們簡要地說的,詳細地了知義理,\twnr{那就好了}{44.0}!」

  「蘇尸摩學友![不論]你能了知或你不能了知,但我們是慧解脫者。」

  那時,尊者蘇尸摩從座位起來後去見世尊。抵達後,向世尊問訊後,在一旁坐下。在一旁坐下的尊者蘇尸摩告訴世尊與那些比丘一起有交談之所及的那一切。

  「蘇尸摩!\twnr{法住智}{634.0}在前,\twnr{涅槃智在後}{x292}。」

  「大德!我對這個被世尊簡要地說的義理,不詳細地了知,請世尊為我說,如是我能對這個被世尊簡要地說的義理,詳細地了知,那就好了!」

  「蘇尸摩![不論]你能了知或你不能了知,法住智在前,涅槃智在後。

  蘇尸摩!你怎麼想它:色是常的,或是無常的?」

  「無常的,大德!」

  「那麼,凡為無常的,那是苦的或樂的?」

  「苦的,大德!」

  「那麼,凡為無常的、苦的、\twnr{變易法}{70.0},適合認為它:『\twnr{這是我的}{32.0},\twnr{我是這個}{33.0},這是\twnr{我的真我}{34.0}。』嗎?」

  「大德!這確實不是。」

  「受是常的,或是無常的?」

  「無常的,大德!」

  「那麼,凡為無常的,那是苦的或樂的?」

  「苦的,大德!」

  「那麼,凡為無常的、苦的、變易法,適合認為它:『這是我的,我是這個,這是我的真我。』嗎?」

  「大德!這確實不是。」

  「想是常的,或是無常的?」

  「無常的,大德!」……(中略)

  「諸行是常的,或是無常的?」

  「無常的,大德!」

  「那麼,凡為無常的,那是苦的或樂的?」

  「苦的,大德!」

  「那麼,凡為無常的、苦的、變易法,適合認為它:『這是我的,我是這個,這是我的真我。』嗎?」

  「大德!這確實不是。」

  「識是常的,或是無常的?」

  「無常的,大德!」

  「那麼,凡為無常的,那是苦的或樂的?」

  「苦的,大德!」

  「那麼,凡為無常的、苦的、變易法,適合認為它:『這是我的,我是這個,這是我的真我。』嗎?」

  「大德!這確實不是。」

  「蘇尸摩!因此,在這裡,凡任何色:過去、未來、現在,或內、或外,或粗、或細,或下劣、或勝妙,或凡在遠處、在近處,所有色:『\twnr{這不是我的}{32.1},\twnr{我不是這個}{33.1},\twnr{這不是我的真我}{34.2}。』這樣,這個應該以正確之慧如實被看見。

  凡任何受:過去、未來、現在,或內、或外,或粗、或細,或下劣、或勝妙,或凡在遠處、在近處,所有受:『這不是我的,我不是這個,這不是我的真我。』這樣,這個應該以正確之慧如實被看見。

  凡任何想……(中略)凡任何諸行:過去、未來、現在,或內、或外,或粗、或細,或下劣、或勝妙,或凡在遠處、在近處,所有諸行:『這不是我的,我不是這個,這不是我的真我。』這樣,這個應該以正確之慧如實被看見。凡任何識:過去、未來、現在,或內、或外,或粗、或細,或下劣、或勝妙,或凡在遠處、在近處,所有識:『這不是我的,我不是這個,這不是我的真我。』這樣,這個應該以正確之慧如實被看見。

  蘇尸摩!這麼看的\twnr{有聽聞的聖弟子}{24.0},在色上\twnr{厭}{15.0},也在受上厭,也在想上厭,也在諸行上厭,也在識上厭。厭者\twnr{離染}{558.0},從\twnr{離貪}{77.0}被解脫,在已解脫時,\twnr{有『[這是]解脫』之智}{27.0},他知道:『出生已盡,梵行已完成,應該被作的已作,不再有此處[輪迴]的狀態。』」

  蘇尸摩!你看到『以生\twnr{為緣}{180.0}有老死(而老死存在)』嗎?」

  「是的,大德!」

  「蘇尸摩!你看到『以有為緣有生』嗎?」

  「是的,大德!」

  「蘇尸摩!你看到『以取為緣有有』嗎?」

  「是的,大德!」

  「蘇尸摩!你看到『以渴愛為緣有取』嗎?」

  「是的,大德!」

  「『以受為緣有渴愛』……『以觸為緣有受』……『以六處為緣有觸』……『以名色為緣有六處』……『以識為緣有名色』……『以行為緣有識』……蘇尸摩!你看到『以無明為緣有諸行』嗎?」

  「是的,大德!」

  「蘇尸摩!你看到『以生滅有老死滅(而老死滅存在)』嗎?」

  「是的,大德!」

  「蘇尸摩!你看到『以有滅有生滅』嗎?」

  「是的,大德!」

  「『以取滅有有滅』……『以渴愛滅有取滅』……『以受滅有渴愛滅』……『以觸滅有受滅』……『以六處滅有觸滅』……『以名色\twnr{滅}{68.0}而六處滅』……『以識滅有名色滅』……『以行滅有識滅』……蘇尸摩!你看到『以無明滅有行滅』嗎?」

  「是的,大德!」

  「蘇尸摩!但當這麼知的、這麼見的你體驗各種神通種類:是一個後變成多個,又,是多個後變成一個;現身、隱身、穿牆、穿壘、穿山無阻礙地行走猶如在虛空中;在地中作浮沈猶如在水中,又,在不被破裂的水上行走猶如在地上;在空中以盤腿來去猶如有翅膀的鳥,又,以手碰觸、撫摸這些這麼大神通力、這麼大威力的日月;以身體行使自在直到梵天世界嗎?」

  「大德!這確實不是。」

  「蘇尸摩!但這麼知的、這麼見的你以清淨、超越常人的天耳界聽到二者的聲音:「天與人,以及在遠處、近處。」嗎?」

  「大德!這確實不是。」

  「蘇尸摩!但這麼知的、這麼見的你對其他眾生、其他個人以心熟知心後知道:有貪的心為『有貪的心』……(中略)知道已解脫的心為『已解脫的心』嗎?」

  「大德!這確實不是。」

  「蘇尸摩!但這麼知的、這麼見的你回憶(隨念)許多前世住處,即一生……(中略)像這樣,你回憶許多有行相的、有境遇的前世住處嗎?」

  「大德!這確實不是。」

  「蘇尸摩!但這麼知的、這麼見的你以清淨、超越常人的天眼死沒往生的眾生……(中略)知道依業到達的眾生嗎?」

  「大德!這確實不是。」

  「蘇尸摩!但這麼知的、這麼見的你,凡那些超越色的無色寂靜解脫,你以身觸達後住於那些嗎?」

  「大德!這確實不是。」

  「蘇尸摩!現在,這裡,這個回答與這些法的未等至,蘇尸摩!這是怎樣呢?」

  那時,尊者蘇尸摩以頭落在世尊的腳上後對世尊說這個:

  「大德!罪過征服如是愚的、如是愚昧的、如是不善的我:凡我在這麼被善說的法律中為盜法的出家者。大德!為了未來的\twnr{自制}{217.0},請世尊接受那個我的罪過為罪過。」

  「蘇尸摩!確實,罪過征服如是愚的、如是愚昧的、如是不善的你:凡你在這麼被善說的法律中為盜法的出家者。

  蘇尸摩!猶如捕捉盜賊、罪犯後他們對國王展示:『陛下!這位是你的盜賊、罪犯,請你對這位判決凡你想要的那個處罰。』

  國王隨即這麼說:『\twnr{先生}{202.0}!請你們去,對這位男子以堅固的繩索手在背後緊緊地捆綁後,剃光頭後,以猛烈聲的銅鼓,從街道到街道;從十字路口到十字路口遍帶領後,經南門出去後,請你們在城南斬首。』

  國王的人們對他以堅固的繩索手在背後緊緊地捆綁後,剃光頭後,以猛烈聲的銅鼓,從街道到街道;從十字路口到十字路口遍帶領後,經南門出去後,在城南斬首。

  蘇尸摩!你怎麼想它:是否那位男子會從那個因由感受憂苦呢?」

  「是的,大德!」

  「蘇尸摩!那位男子從那個因由感受憂苦。凡在這麼被善說的法律中為盜法的出家者,這位因此有更苦的果報與更辛辣的果報,進一步導向下界中。

  蘇尸摩!但由於你看見罪過為罪過後如法懺悔,我們接受你的那個[懺悔]。蘇尸摩!在聖者之律中這是增長:凡看見罪過為罪過後如法懺悔,未來來到自制。」

  大品第七,其\twnr{攝頌}{35.0}:

  「未聽聞的二則,與隨後以兒子的肉,

   有貪與城市,觸知、蘆葦束,

   憍賞彌與漲高,與以蘇尸摩為第十。」





\pin{沙門婆羅門品}{71}{93}
\sutta{71}{71}{老死經}{https://agama.buddhason.org/SN/sn.php?keyword=12.71}
  \twnr{被我這麼聽聞}{1.0}:

  \twnr{有一次}{2.0},\twnr{世尊}{12.0}住在舍衛城祇樹林給孤獨園。

  在那裡,世尊……(中略)。

  「\twnr{比丘}{31.0}們!凡任何\twnr{沙門}{29.0}或\twnr{婆羅門}{17.0}不知道老死,不知道老死集,不知道老死滅,不知道導向老死\twnr{滅道跡}{69.0}者,比丘們!那些沙門或婆羅門不被我認同為\twnr{沙門中的沙門}{560.0},或婆羅門中的婆羅門,而且,那些\twnr{尊者}{200.0}也不以證智自作證後,在當生中\twnr{進入後住於}{66.0}\twnr{沙門義}{327.0}或婆羅門義。

  「比丘們!而凡任何沙門或婆羅門知道老死……(中略)知道導向老死滅道跡者,比丘們!那些沙門或婆羅門被我認同為沙門中的沙門,或婆羅門中的婆羅門,而且,那些尊者也以證智自作證後,在當生中進入後住於沙門義或婆羅門義。」[\suttaref{SN.12.13}]



\sutta{72}{81}{生經等十則}{https://agama.buddhason.org/SN/sn.php?keyword=12.72}
  住在舍衛城……(中略)。

  不知道生……(中略)。

  (3)不知道有……(中略)。

  (4)不知道取……(中略)。

  (5)不知道渴愛……(中略)。

  (6)不知道受……(中略)。

  (7)不知道觸……(中略)。

  (8)不知道六處……(中略)。

  (9)不知道名色……(中略)。

  (10)不知道識……(中略)。

  (11)不知道行,不知道行集,不知道行滅,不知道導向行\twnr{滅道跡}{69.0}者……(中略)。

  知道行……(中略)以證智自作證後,在當生中\twnr{進入後住於}{66.0}\twnr{沙門義}{327.0}或婆羅門義。」[\suttaref{SN.12.31}]

  沙門婆羅門品第八,其\twnr{攝頌}{35.0}:

  「緣十一說,以四諦解析,

   沙門婆羅門品,在因緣[相應]中是第八的。」

  品之攝頌:

  「佛陀、食、十力,黑齒、屋主第五,

   苦品、大品,沙門婆羅門是第八的。」





\sutta{82}{82}{師經}{https://agama.buddhason.org/SN/sn.php?keyword=12.82}
  住在舍衛城……(中略)。

  「\twnr{比丘}{31.0}們!不如實知見老死者(如實不知者不見者),為了在老死上的如實智,\twnr{老師}{x293}應該被遍求;不如實知見老死\twnr{集}{67.0}者,為了在老死集上的如實智,老師應該被遍求;不如實知見老死\twnr{滅}{68.0}者,為了在老死滅上的如實智,老師應該被遍求;不如實知見導向老死\twnr{滅道跡}{69.0}者,為了在導向老死滅道跡上的如實智,老師應該被遍求。」(一經)

  (對[以下]一切中略應該這樣使之被細說)



\sutta{83}{93}{師經第二等十則}{https://agama.buddhason.org/SN/sn.php?keyword=12.83}
  (2)「\twnr{比丘}{31.0}們!\twnr{不如實知見生者}{x294}……(中略)。」

  (3)「比丘們!不如實知見有者……(中略)。」

  (4)「比丘們!不如實知見取者……(中略)。」

  (5)「比丘們!不如實知見渴愛者……(中略)。」

  (6)「比丘們!不如實知見受者……(中略)。」

  (7)「比丘們!不如實知見觸者……(中略)。」

  (8)「比丘們!不如實知見六處者……(中略)。」

  (9)「比丘們!不如實知見名色者……(中略)。」

  (10)「比丘們!不如實知見識者……(中略)。」

  (11)「比丘們!不如實知見諸行者,為了在諸行上的如實智,\twnr{老師}{x293}應該被遍求;不如實知見行\twnr{集}{67.0}者,為了在行集上的如實智,老師應該被遍求;不如實知見行\twnr{滅}{68.0}者,為了在行滅上的如實智,老師應該被遍求;不如實知見導向行\twnr{滅道跡}{69.0}者,為了在導向行滅道跡上的如實智,老師應該被遍求。」

  (對一切應該作四諦的[解說])



相應部12相應83-93經/學經等(中略)十一則(因緣相應/因緣篇/修多羅)(莊春江譯)

  (2)「比丘們!不如實知見老死者,為了在老死上的如實智,學應該被做(執行)……(中略,應該作四諦)。」

  (3)「比丘們!不如實知見老死者,為了在老死上的如實智,努力應該被做……(中略)。」

  (4)「比丘們!不如實知見老死者,為了在老死上的如實智,意欲應該被做……(中略)。」

  (5)「比丘們!不如實知見老死者,為了在老死上的如實智,諸熱忱應該被做……(中略)。」

  (6)「比丘們!不如實知見老死者,為了在老死上的如實智,不畏縮應該被做……(中略)。」

  (7)「比丘們!不如實知見老死者,為了在老死上的如實智,熱心應該被做……(中略)。」

  (8)「比丘們!不如實知見老死者,為了在老死上的如實智,活力應該被做……(中略)。」

  (9)「比丘們!不如實知見老死者,為了在老死上的如實智,堅忍(毅力)應該被做……(中略)。」

  (10)「比丘們!不如實知見老死者,為了在老死上的如實智,念應該被做……(中略)。」

  (11)「比丘們!不如實知見老死者,為了在老死上的如實智,正知應該被做……(中略)。」

  (12)「比丘們!不如實知見老死者,為了在老死上的如實智,不放逸應該被做……(中略)。」

  中間中略[品]第九,其\twnr{攝頌}{35.0}:

  「師、學與努力,意欲、熱忱為第五,

   不畏縮、熱心,活力、堅忍被釋放,

   念與正知,以不放逸為十二則。」

  中間中略經終了。

  「那十二則其他的有,一百三十二經,

   那些以四諦被說的:凡在中略內。」

  關於中略之攝頌完成。

  因緣相應完成。





\page

\xiangying{13}{現觀相應}
\sutta{1}{1}{指甲尖經}{https://agama.buddhason.org/SN/sn.php?keyword=13.1}
  \twnr{被我這麼聽聞}{1.0}:

  \twnr{有一次}{2.0},\twnr{世尊}{12.0}住在舍衛城祇樹林給孤獨園。

  那時,世尊使微少塵土沾在指甲尖後,召喚\twnr{比丘}{31.0}們:

  「比丘們!你們怎麼想它,哪個是比較多的呢:凡這被我沾在指甲尖的微少塵土,或凡這大地?」

  「\twnr{大德}{45.0}!這正是比較多的,即:大地,被世尊沾在指甲尖的微少塵土是少量的。被世尊沾在指甲尖的微少塵土比較大地後,既不來到百分之一,也不來到千分之一、不來到十萬分之一。」

  「同樣的,比丘們!對\twnr{見具足之人}{575.0}、已\twnr{現觀}{53.0}的\twnr{聖弟子}{24.0},這正是比較多的,即:已遍滅盡、已耗盡(遍取)的苦,殘留的是少量的,即:\twnr{最多七次}{161.0}的狀態,比較先前已遍滅盡、已耗盡的苦蘊後,既不來到百分之一,也不來到千分之一、不來到十萬分之一。比丘們!法的現觀有這麼大利益,法眼的獲得有這麼大利益。」



\sutta{2}{2}{蓮花池經}{https://agama.buddhason.org/SN/sn.php?keyword=13.2}
  住在舍衛城。……(中略)

  「\twnr{比丘}{31.0}們!猶如被水充滿的、滿到邊緣的、能被烏鴉喝飲的蓮花池長五十\twnr{由旬}{148.0},寬五十由旬,深五十由旬。男子以茅草尖從那裡取出水,比丘們!你們怎麼想它,哪個是比較多的呢:凡被茅草尖取出的水,或凡蓮花池的水?」

  「\twnr{大德}{45.0}!這正是比較多的,即:蓮花池的水,被茅草尖取出的水是少量的。被茅草尖取出的水比較蓮花池的水後,既不來到百分之一,也不來到千分之一、不來到十萬分之一。」

  「同樣的,比丘們!對\twnr{見具足之人}{575.0}、已\twnr{現觀}{53.0}的\twnr{聖弟子}{24.0},這正是比較多的,即:已遍滅盡、已耗盡(遍取)的苦,殘留的是少量的,即:\twnr{最多七次}{161.0}的狀態,比較先前已遍滅盡、已耗盡的苦蘊後,既不來到百分之一,也不來到千分之一、不來到十萬分之一。比丘們!法的現觀有這麼大利益,法眼的獲得有這麼大利益。」



\sutta{3}{3}{合流水經}{https://agama.buddhason.org/SN/sn.php?keyword=13.3}
  住在舍衛城。……(中略)

  「\twnr{比丘}{31.0}們!猶如在這些大河會合、集合之處,即:恒河、耶牟那河、阿致羅筏底河、薩羅浮河、摩醯河,男子從那裡取出二或三滴水,比丘們!你們怎麼想它,哪個是比較多的呢:凡被取出的二或三滴水,或凡合流的水?」

  「\twnr{大德}{45.0}!這正是比較多的,即:合流的水,被取出的二或三滴水是少量的。被取出的二或三滴水比較合流的水後,既不來到百分之一,也不來到千分之一、不來到十萬分之一。」

  「同樣的,比丘們!……(中略)法眼的獲得有這麼大的利益。」



\sutta{4}{4}{合流水經第二}{https://agama.buddhason.org/SN/sn.php?keyword=13.4}
  住在舍衛城。……(中略)

  「\twnr{比丘}{31.0}們!猶如在這些大河會合、集合之處,即:恒河、耶牟那河、阿致羅筏底河、薩羅浮河、摩醯河,那個水除了二或三滴水外,走到遍盡、耗盡(遍取),比丘們!你們怎麼想它,哪個是比較多的呢:凡已遍盡、已耗盡的合流水,或凡二或三滴殘留的水?」

  「\twnr{大德}{45.0}!這正是比較多的,即:已遍盡、已耗盡的合流水,二或三滴殘留的水是少量的。二或三滴殘留的水比較已遍盡、已耗盡的合流水後,既不來到百分之一,也不來到千分之一、不來到十萬分之一。」

  「同樣的,比丘們!……(中略)法眼的獲得有這麼大的利益。」



\sutta{5}{5}{地經}{https://agama.buddhason.org/SN/sn.php?keyword=13.5}
  住在舍衛城。……(中略)

  「\twnr{比丘}{31.0}們!猶如男子在大地上放置七顆棗核大小的土團,比丘們!你們怎麼想它,哪個是比較多的呢:凡被放置的七顆棗核大小土團,或凡這大地?」

  「\twnr{大德}{45.0}!這正是比較多的,即:大地,被放置的七顆棗核大小土團是少量的。被放置的七顆棗核大小土團比較大地後,既不來到百分之一,也不來到千分之一、不來到十萬分之一。」

  「同樣的,比丘們!……(中略)法眼的獲得有這麼大的利益。」



\sutta{6}{6}{地經第二}{https://agama.buddhason.org/SN/sn.php?keyword=13.6}
  住在舍衛城。……(中略)

  「\twnr{比丘}{31.0}們!猶如大地如果除了七顆棗核大小的土團外,走到遍盡、耗盡(遍取),比丘們!你們怎麼想它,哪個是比較多的呢:凡已遍盡、已耗盡的大地,或凡殘留的七顆棗核大小的土團?」

  「\twnr{大德}{45.0}!這正是比較多的,即:已遍盡、已耗盡的大地,殘留的七顆棗核大小的土團是少量的。殘留的七顆棗核大小的土團比較已遍盡、已耗盡的大地後,既不來到百分之一,也不來到千分之一、不來到十萬分之一。」

  「同樣的,比丘們!……(中略)法眼的獲得有這麼大的利益。」



\sutta{7}{7}{海洋經}{https://agama.buddhason.org/SN/sn.php?keyword=13.7}
  住在舍衛城。……(中略)

  「比丘們!猶如男子從大海取出二或三滴水,比丘們!你們怎麼想它,哪個是比較多的呢:凡被取出的二或三滴水,或凡大海的水?」

  「\twnr{大德}{45.0}!這正是比較多的,即:大海的水,被取出的二或三滴水是少量的。被取出的二或三滴水比較大海的水後,既不來到百分之一,也不來到千分之一、不來到十萬分之一。」

  「同樣的,比丘們!……(中略)法眼的獲得有這麼大的利益。」



\sutta{8}{8}{海洋經第二}{https://agama.buddhason.org/SN/sn.php?keyword=13.8}
  住在舍衛城。……(中略)

  「比丘們!猶如大海除了二或三滴水外,走到遍盡、耗盡(遍取),比丘們!你們怎麼想它,哪個是比較多的呢:凡已遍盡、已耗盡的大海水,或凡二或三滴殘留的水?」

  「\twnr{大德}{45.0}!這正是比較多的,即:已遍盡、已耗盡的大海水,二或三滴殘留的水是少量的。二或三滴殘留的水比較已遍盡、已耗盡的大海水後,既不來到百分之一,也不來到千分之一、不來到十萬分之一。

  「同樣的,比丘們!……(中略)法眼的獲得有這麼大的利益。」



\sutta{9}{9}{山經}{https://agama.buddhason.org/SN/sn.php?keyword=13.9}
  住在舍衛城。……(中略)

  「\twnr{比丘}{31.0}們!猶如男子對喜瑪拉雅山山王放置七顆芥子大小的碎石,比丘們!你們怎麼想它,哪個是比較多的呢:凡七顆芥子大小的碎石,或凡喜瑪拉雅山山王?」

  「\twnr{大德}{45.0}!這正是比較多的,即:喜瑪拉雅山山王,七顆芥子大小的碎石是少量的。七顆芥子大小的碎石比較喜瑪拉雅山山王後,既不來到百分之一,也不來到千分之一、不來到十萬分之一。」

  「同樣的,比丘們!……(中略)法眼的獲得有這麼大的利益。」



\sutta{10}{10}{山經第二}{https://agama.buddhason.org/SN/sn.php?keyword=13.10}
  住在舍衛城。……(中略)

  「\twnr{比丘}{31.0}們!猶如喜瑪拉雅山山王除了七顆芥子大小的碎石外,走到遍盡、耗盡(遍取),比丘們!你們怎麼想它,哪個是比較多的呢:凡七顆芥子大小的碎石,或凡已遍盡、已耗盡的喜瑪拉雅山山王?」

  「\twnr{大德}{45.0}!這正是比較多的,即:已遍盡、已耗盡的喜瑪拉雅山山王,殘留的七顆芥子大小碎石是少量的。殘留的七顆芥子大小碎石比較已遍盡、已耗盡的喜瑪拉雅山山王後,既不來到百分之一,也不來到千分之一、不來到十萬分之一。」

  「同樣的,比丘們!對\twnr{見具足之人}{575.0}、已\twnr{現觀}{53.0}的\twnr{聖弟子}{24.0},這正是比較多的,即:已遍滅盡、已耗盡(遍取)的苦,殘留的是少量的,即:\twnr{最多七次}{161.0}的狀態,比較先前已遍滅盡、已耗盡的苦蘊後,既不來到百分之一,也不來到千分之一、不來到十萬分之一。比丘們!法的現觀有這麼大利益,法眼的獲得有這麼大利益。」



\sutta{11}{11}{山經第三}{https://agama.buddhason.org/SN/sn.php?keyword=13.11}
  住在舍衛城。……(中略)

  「\twnr{比丘}{31.0}們!猶如男子對\twnr{須彌山山王}{272.0}放置七顆綠豆大小的碎石,比丘們!你們怎麼想它,哪個是比較多的呢:凡被放置的七顆綠豆大小的碎石,或凡須彌山山王?」

  「\twnr{大德}{45.0}!這正是比較多的,即:須彌山山王,被放置的七顆綠豆大小碎石是少量的。被放置的七顆綠豆大小碎石比須彌山山王後,既不來到百分之一,也不來到千分之一、不來到十萬分之一。」

  「同樣的,比丘們!其他外道\twnr{沙門}{29.0}、\twnr{婆羅門}{17.0}、\twnr{遊行者}{79.0}之證得比較\twnr{見具足之人}{575.0}的\twnr{聖弟子}{24.0}之證得後,既不來到百分之一,也不來到千分之一、不來到十萬分之一。比丘們!見具足之人有這麼大證得,有這麼大證智。」

  現觀相應完成,其\twnr{攝頌}{35.0}:

  「指甲尖、蓮花池,合流水二則,

   二則地、二則海洋,三則山的比喻。」





\page

\xiangying{14}{界相應}
\pin{種種品}{1}{10}
\sutta{1}{1}{種種界經}{https://agama.buddhason.org/SN/sn.php?keyword=14.1}
  住在舍衛城……(中略)。

  「\twnr{比丘}{31.0}們!我將為你們教導種種界,你們要聽它!你們要\twnr{好好作意}{43.1}!我將說。」

  「是的,\twnr{大德}{45.0}!」比丘們回答\twnr{世尊}{12.0}。

  世尊說這個:

  「比丘們!而什麼是種種界?眼界、色界、眼識界,耳界、聲界、耳識界,鼻界、氣味界、鼻識界,舌界、味道界、舌識界,身界、\twnr{所觸}{220.2}界、身識界,\twnr{意界}{337.0}、\twnr{法界}{547.0}、\twnr{意識界}{337.1}。比丘們!這被稱為種種界。」



\sutta{2}{2}{種種觸經}{https://agama.buddhason.org/SN/sn.php?keyword=14.2}
  住在舍衛城……(中略)。

  「\twnr{比丘}{31.0}們!\twnr{緣於}{252.0}種種界(界種種性)種種觸生起。

  比丘們!而什麼是種種界?眼界、耳界、鼻界、舌界、身界、\twnr{意界}{337.0}。比丘們!這被稱為種種界。

  比丘們!而什麼是緣於種種界種種觸生起?比丘們!緣於眼界眼觸生起;緣於耳界……緣於鼻界……緣於舌界……緣於身界……緣於意界意觸生起。比丘們!緣於種種界這樣種種觸生起。」



\sutta{3}{3}{非種種觸經}{https://agama.buddhason.org/SN/sn.php?keyword=14.3}
  住在舍衛城……(中略)。

  「\twnr{比丘}{31.0}們!\twnr{緣於}{252.0}種種界(界種種性)種種觸生起,非緣於種種觸種種界生起。

  比丘們!而什麼是種種界?眼界……(中略)\twnr{意界}{337.0}。比丘們!這被稱為種種界。

  比丘們!而什麼是緣於種種界種種觸生起,非緣於種種觸種種界?比丘們!緣於眼界眼觸生起,非緣於眼觸眼界生起……(中略)緣於意界意觸生起,非緣於意觸意界生起。

  比丘們!緣於種種界這樣種種觸生起,非緣於種種觸種種界生起。」



\sutta{4}{4}{種種受經}{https://agama.buddhason.org/SN/sn.php?keyword=14.4}
  住在舍衛城……(中略)。

  「\twnr{比丘}{31.0}們!\twnr{緣於}{252.0}種種界(界種種性)種種觸生起;緣於種種觸種種受生起。

  比丘們!而什麼是種種界?眼界……(中略)\twnr{意界}{337.0}。比丘們!這被稱為種種界。

  比丘們!而什麼是緣於種種界種種觸生起;緣於種種觸種種受生起?

  比丘們!緣於眼界眼觸生起;緣於眼觸眼觸所生受生起……(中略)緣於意界意觸生起;緣於意觸意觸所生受生起。

  比丘們!緣於種種界這樣種種觸生起;緣於種種觸種種受生起。」



\sutta{5}{5}{種種受經第二}{https://agama.buddhason.org/SN/sn.php?keyword=14.5}
  住在舍衛城……(中略)。

  「\twnr{比丘}{31.0}們!\twnr{緣於}{252.0}種種界(界種種性)種種觸生起;緣於種種觸種種受生起,非緣於種種受種種觸生起,非緣於種種觸種種界。

  比丘們!而什麼是種種界?眼界……(中略)\twnr{意界}{337.0}。比丘們!這被稱為種種界。

  比丘們!而什麼是緣於種種界種種觸生起;緣於種種觸種種受生起,非緣於種種受種種觸生起,非緣於種種觸種種界呢?比丘們!緣於眼界眼觸生起;緣於眼觸眼觸所生受生起,非緣於眼觸所生受眼觸生起,非緣於眼觸眼界生起……(中略)緣於意界意觸生起;緣於意觸意觸所生受生起,非緣於意觸所生受意觸生起,非緣於意觸意界生起。

  比丘們!緣於種種界這樣種種觸生起;緣於種種觸種種受生起,非緣於種種受種種觸生起,非緣於種種觸種種界生起。」



\sutta{6}{6}{外部的種種界經}{https://agama.buddhason.org/SN/sn.php?keyword=14.6}
  住在舍衛城……(中略)。

  「\twnr{比丘}{31.0}們!我將為你們教導種種界(界種種性),你們要聽它!……(中略)。

  比丘們!而什麼是種種界?色界、聲界、氣味界、味道界、\twnr{所觸}{220.2}界、\twnr{法界}{547.0}。比丘們!這被稱為種種界。」



\sutta{7}{7}{種種想經}{https://agama.buddhason.org/SN/sn.php?keyword=14.7}
  住在舍衛城……(中略)。

  「\twnr{尊師}{480.0}!」那些\twnr{比丘}{31.0}回答\twnr{世尊}{12.0}。

  世尊說這個:

  「比丘們!\twnr{緣於}{252.0}種種界(界種種性)種種想生起;緣於種種想種種意向生起;緣於種種意向種種意欲生起;緣於種種意欲種種熱惱生起;緣於種種熱惱種種遍求生起。

  比丘們!而什麼是種種界?色界……(中略)\twnr{法界}{547.0}。比丘們!這被稱為種種界。

  比丘們!而什麼是緣於種種界種種想生起;緣於種種想種種意向生起;緣於種種意向種種意欲生起;緣於種種意欲種種熱惱生起;緣於種種熱惱種種遍求生起?

  比丘們!緣於色界色想生起;緣於色想色意向生起;緣於色意向色意欲生起;緣於色意欲色熱惱生起;緣於色熱惱色遍求生起。……(中略)緣於法界法想生起;緣於法想法意向生起;緣於法意向法意欲生起;緣於法意欲法熱惱生起;緣於法熱惱法遍求生起。

  比丘們!緣於種種界這樣種種想生起;緣於種種想種種意向生起;緣於種種意向種種意欲生起;緣於種種意欲種種熱惱生起;緣於種種熱惱種種遍求生起。」



\sutta{8}{8}{非種種遍求經}{https://agama.buddhason.org/SN/sn.php?keyword=14.8}
  住在舍衛城……(中略)。

  「\twnr{比丘}{31.0}們!\twnr{緣於}{252.0}種種界(界種種性)種種想生起;緣於種種想種種意向生起;緣於種種意向種種意欲生起;緣於種種意欲種種熱惱生起;緣於種種熱惱種種遍求,非緣於種種遍求種種熱惱生起;非緣於種種熱惱種種意欲生起;非緣於種種意欲種種意向生起;非緣於種種意向種種想生起;非緣於種種想種種界生起。

  比丘們!而什麼是種種界?

  色界……(中略)\twnr{法界}{547.0}。

  比丘們!這被稱為種種界。

  比丘們!而什麼是緣於種種界種種想生起;緣於種種想……生起……(中略)種種遍求,非緣於種種遍求種種熱惱生起;非緣於種種熱惱種種意欲生起;非緣於種種意欲種種意向生起;非緣於種種意向種種想生起;非緣於種種想種種界生起?

  比丘們!緣於色界色想生起……(中略)緣於法界法想生起;緣於法想……生起(中略)法遍求生起;非緣於法遍求法熱惱生起;非緣於法熱惱法意欲生起;非緣於法意欲法意向生起;非緣於法意向法想生起;非緣於法想法界生起。

  比丘們!緣於種種界這樣種種想生起;緣於種種想……生起……(中略)種種遍求生起;非緣於種種遍求種種熱惱生起;非緣於種種熱惱種種意欲生起;非緣於種種意欲種種意向生起;非緣於種種意向種種想生起;非緣於種種想種種界生起。」



\sutta{9}{9}{外部的種種觸經}{https://agama.buddhason.org/SN/sn.php?keyword=14.9}
  住在舍衛城……(中略)。

  「\twnr{比丘}{31.0}們!\twnr{緣於}{252.0}種種界(界種種性)種種想生起;緣於種種想種種意向生起;緣於種種意向種種觸生起;緣於種種觸種種受生起;緣於種種受種種意欲生起;緣於種種意欲種種熱惱生起;緣於種種熱惱種種遍求生起;緣於種種遍求種種\twnr{獲得}{x295}生起。

  比丘們!而什麼是種種界?

  色界……(中略)\twnr{法界}{547.0}。

  比丘們!這被稱為種種界。

  比丘們!而什麼是緣於種種界種種想生起;緣於種種想……生起……(中略)種種獲得生起?

  比丘們!緣於色界色想生起;緣於色想色意向生起;緣於色意向色觸生起;緣於色觸色觸所生受生起;緣於色觸所生受色意欲生起;緣於色意欲色熱惱生起;緣於色熱惱色遍求生起;緣於色遍求色獲得生起。……(中略)緣於法界法想生起;緣於法想法意向生起;緣於法意向法觸生起;緣於法觸法觸所生受生起;緣於法觸所生受法意欲生起;緣於法意欲法熱惱生起;緣於法熱惱法遍求生起;緣於法遍求法獲得生起。

  比丘們!緣於種種界這樣種種想生起;緣於種種想……生起……(中略)種種遍求生起;緣於種種遍求種種獲得生起。」



\sutta{10}{10}{外部的種種觸經第二}{https://agama.buddhason.org/SN/sn.php?keyword=14.10}
  住在舍衛城……(中略)。

  「\twnr{比丘}{31.0}們!\twnr{緣於}{252.0}種種界(界種種性)種種想生起;緣於種種想種種意向生起……觸……受……意欲……熱惱……緣於種種遍求種種獲得,非緣於種種獲得種種遍求生起;非緣於種種遍求種種熱惱生起;非緣於種種熱惱……生起……(中略)意欲……受……觸……意向……想生起;非緣於種種想種種界生起。

  比丘們!而什麼是種種界?

  色界……(中略)\twnr{法界}{547.0}。

  比丘們!這被稱為種種界。

  比丘們!而什麼是緣於種種界種種想生起;緣於種種想種種意向生起……觸……受……意欲……熱惱……遍求……獲得,非緣於種種獲得種種遍求生起;非緣於種種遍求種種熱惱……生起意欲……受……觸……非緣於種種意向種種想生起;非緣於種種想種種界?

  比丘們!緣於色界色想……(中略)緣於法界法想生起;緣於法想……生起……(中略)法遍求生起;緣於法遍求法獲得,非緣於法獲得法遍求生起;非緣於法遍求法熱惱生起;非緣於法熱惱法意欲生起;非緣於法意欲法觸所生受生起;非緣於法觸所生受法觸生起;非緣於法觸法意向生起;非緣於法意向法想生起;非緣於法想法界。

  比丘們!緣於種種界這樣種種想生起;緣於種種想……生起……(中略)意向……觸……受……意欲……熱惱……遍求……獲得生起;非緣於種種獲得種種遍求生起;非緣於種種遍求種種熱惱生起;非緣於種種熱惱種種意欲生起;非緣於種種意欲種種受生起;非緣於種種受種種觸生起;非緣於種種觸種種意向生起;非緣於種種意向種種想生起;非緣於種種想種種界生起。」

  種種品第一,其\twnr{攝頌}{35.0}:

  「界、觸與非這個,受二則在後,

   此自身內者五則,界、想與非這個,

   觸二則隨後,此外部者五則。」



  附註:

  北傳「觸→受→想→欲→覺→熱→求」次第,南傳作「想→意向→觸→受→意欲→熱惱→遍求→獲得」,菩提比丘長老並指出此處南傳經文「想→意向→觸→受……」與「觸」為「受、想、行」的條件(緣)之經文(\suttaref{SN.22.82}、\suttaref{SN.35.93})相違背,而認為\ccchref{MN.18}{https://agama.buddhason.org/MN/dm.php?keyword=18}所說的「觸→受→想→尋→虛妄→習慣」(contact→feeling→perception→ thought→conceptual proliferation→obsession by perceptions and notions arisen from proliferation)是比較中肯的次第(a more cogent version of the series),並解說經文中常將「尋」(vitakka)與「意向」(saṅkappa)視為等同,「虛妄」(papañca)則包含「渴愛」(taṇhā),「渴愛」又是「欲求」(chanda)的同義詞,「習慣」(samudācāra)又可以包含「熱惱」與「遍求」(The texts often treat as identical with); proliferation (papañca) includes craving (taṇhā), which is synonymous with desire (chanda); and obsession (samudācāra) may comprise passions and quests),這與北傳經文所說相近。





\pin{第二品}{11}{22}
\sutta{11}{11}{七界經}{https://agama.buddhason.org/SN/sn.php?keyword=14.11}
  住在舍衛城……(中略)。

  「\twnr{比丘}{31.0}們!有這七界,哪七個?光界、淨界、虛空無邊處界、識無邊處界、\twnr{無所有處}{533.0}界、\twnr{非想非非想處}{534.0}界、\twnr{想受滅}{416.0}界。比丘們!這些是七界。」

  在這麼說時,某位比丘對\twnr{世尊}{12.0}說這個:

  「\twnr{大德}{45.0}!凡這個光界、凡淨界、凡虛空無邊處界、凡識無邊處界、凡無所有處界、凡非想非非想處界、凡想受滅界,大德!這些界\twnr{緣於}{252.0}什麼被知道(被辨別)呢?」

  「比丘!凡這個光界,這個界緣於黑暗被知道。比丘!凡這個淨界,這個界緣於不淨被知道。比丘!凡這個虛空無邊處界,這個界緣於色被知道。比丘!凡這個識無邊處界,這個界緣於虛空無邊處被知道。比丘!凡這個無所有處界,這個界緣於識無邊處被知道。比丘!凡這個非想非非想處界,這個界緣於無所有處被知道。比丘!凡這個想受滅界,這個界緣於滅被知道。」

  「大德!凡這個光界、凡淨界、凡虛空無邊處界、凡識無邊處界、凡無所有處界、凡非想非非想處界、凡想受滅界,大德!這些界的\twnr{等至}{129.0}應該如何被到達(被得到)呢?」

  「比丘!凡這個光界、淨界、虛空無邊處界、識無邊處界、無所有處界,這些界應該被想等至到達,比丘!凡這個非想非非想處界,這個界應該被殘行等至到達,比丘!凡這個想受滅界,這個界應該被滅等至到達。」



\sutta{12}{12}{有因緣經}{https://agama.buddhason.org/SN/sn.php?keyword=14.12}
  住在舍衛城……(中略)。

  「\twnr{比丘}{31.0}們!欲尋有因緣生起,非無因緣;惡意尋有因緣生起,非無因緣;\twnr{加害尋}{376.2}有因緣生起,非無因緣。

  比丘們!而如何欲尋有因緣生起,非無因緣;惡意尋有因緣生起,非無因緣;加害尋有因緣生起,非無因緣?比丘們!\twnr{緣於}{252.0}欲界欲想生起;緣於欲想欲的意向生起;緣於欲的意向\twnr{欲的意欲}{118.0}生起;緣於欲的意欲欲的熱惱生起;緣於欲的熱惱欲的遍求生起,比丘們!遍求欲的遍求之\twnr{未聽聞的一般人}{74.0}以三處錯誤地行動:以身、語、意。

  比丘們!緣於惡意界惡意想生起;緣於惡意想惡意的意向生起……(中略)惡意的意欲……惡意的熱惱……惡意的遍求,比丘們!遍求惡意的遍求之未聽聞的一般人以三處錯誤地行動:以身、語、意。

  比丘們!緣於加害界\twnr{加害想}{938.2}生起;緣於加害想加害的意向生起……(中略)加害的意欲……加害的熱惱……加害的遍求,比丘們!遍求加害的遍求之未聽聞的一般人以三處錯誤地行動:以身、語、意。

  比丘們!猶如男子掉落燃燒的草火把到乾草原中,如果不以手與以腳急速地使之熄滅,比丘們!這樣,凡依止草木的生物類,牠們會來到不幸與災厄。同樣的,比丘們!凡生起進入不正想的任何\twnr{沙門}{29.0}或\twnr{婆羅門}{17.0}不急速地捨斷、驅離、作終結、使之走到不存在者,他\twnr{當生}{42.0}住於苦,有惱害,有\twnr{絕望}{342.0},有熱惱,以身體的崩解,死後\twnr{惡趣}{110.0}能被預期。

  比丘們!離欲尋有因緣生起,非無因緣;無惡意尋有因緣生起,非無因緣;不加害尋有因緣生起,非無因緣。比丘們!而如何離欲尋有因緣生起,非無因緣;無惡意尋有因緣生起,非無因緣;不加害尋有因緣生起,非無因緣?

  比丘們!\twnr{緣於離欲界}{x296}離欲想生起;緣於離欲想\twnr{離欲的意向}{514.0}生起;緣於離欲的意向離欲的意欲生起;緣於離欲的意欲離欲的熱惱生起;緣於離欲的熱惱離欲的遍求生起,比丘們!遍求離欲的遍求的\twnr{有聽聞的聖弟子}{24.0}以三處正確地行動:以身、語、意。

  比丘們!緣於無惡意界無惡意想生起;緣於無惡意想\twnr{無惡意的意向}{515.0}生起……(中略)無惡意的意欲……無惡意的熱惱……無惡意的遍求,比丘們!遍求無惡意的遍求的有聽聞的聖弟子以三處正確地行動:以身、語、意。

  比丘們!緣於無加害界無加害想生起;緣於無加害想\twnr{無加害的意向}{516.0}生起;緣於無加害的意向無加害的意欲生起;緣於加害意的意欲無加害的熱惱生起;緣於無加害的熱惱無加害的遍求生起,比丘們!遍求不加害的遍求的有聽聞的聖弟子以三處正確地行動:以身、語、意。

  比丘們!猶如男子掉落燃燒的草火把到乾草原中,他立即以手與以腳急速地使之熄滅,比丘們!這樣,凡依止草木的生物類,牠們不會來到不幸與災厄。同樣的,比丘們!凡生起進入不正想的任何沙門或婆羅門急速地捨斷、驅離、作終結、使之走到不存在者,他當生\twnr{住於樂}{317.0},不惱害,不絕望,不熱惱,以身體的崩解,死後\twnr{善趣}{112.0}能被預期。」



\sutta{13}{13}{磚屋經}{https://agama.buddhason.org/SN/sn.php?keyword=14.13}
  \twnr{有一次}{2.0},\twnr{世尊}{12.0}住在那低葛的磚屋中。

  在那裡,世尊召喚\twnr{比丘}{31.0}們:「比丘們!」

  「\twnr{尊師}{480.0}!」那些比丘回答世尊。

  世尊說這個:

  「比丘們!\twnr{緣於}{252.0}界,想生起,見生起,尋生起。」

  在這麼說時,\twnr{尊者}{200.0}迦旃延對世尊說這個:

  「\twnr{大德}{45.0}!凡這個見:『關於諸非遍正覺者為遍正覺者』,大德!緣於什麼這個見被知道(被辨別)呢?」

  「迦旃延!這是大的界(這個界是大的),即:無明界。

  迦旃延!緣於下劣的界,下劣的想、下劣的見、下劣的尋、下劣的思、下劣的欲求、下劣的願求、下劣的個人、下劣的言語生起,他告知、教導、\twnr{使知}{143.0}、建立、開顯、解析、闡明下劣的,我說:『他的往生是下劣的。』

  迦旃延!緣於中等的界,中等的想、中等的見、中等的尋、中等的思、中等的欲求、中等的願求、中等的個人、中等的言語生起,他告知、教導、使知、建立、開顯、解析、闡明中等的,我說:『他的往生是中等的。』

  迦旃延!緣於勝妙的界,勝妙的想、勝妙的見、勝妙的尋、勝妙的思、勝妙的欲求、勝妙的願求、勝妙的個人、勝妙的言語生起,他告知、教導、使知、建立、開顯、解析、闡明勝妙的,我說:『他的往生是勝妙的。』」



\sutta{14}{14}{下劣志向者經}{https://agama.buddhason.org/SN/sn.php?keyword=14.14}
  住在舍衛城……(中略)。

  「\twnr{比丘}{31.0}們!眾生就從界(由於界)會合、集合:下劣\twnr{志向}{257.0}者與下劣志向者一起會合、集合;善的志向者與善的志向者一起會合、集合。

  比丘們!過去世的眾生也曾就從界會合、集合:下劣志向者曾與下劣志向者一起會合、集合;善的志向者曾與善的志向者一起會合、集合。

  比丘們!\twnr{未來世}{308.0}的眾生也將就從界會合、集合:下劣志向者將與下劣志向者一起會合、集合;善的志向者將與善的志向者一起會合、集合。

  比丘們!現在,現在世的眾生也就從界會合、集合:下劣志向者與下劣志向者一起會合、集合;善的志向者與善的志向者一起會合、集合。」



\sutta{15}{15}{經行經}{https://agama.buddhason.org/SN/sn.php?keyword=14.15}
  \twnr{有一次}{2.0},\twnr{世尊}{12.0}住在王舍城\twnr{耆闍崛山}{258.0}。

  當時,\twnr{尊者}{200.0}舍利弗與眾多\twnr{比丘}{31.0}一起在世尊附近\twnr{經行}{150.0};尊者大目揵連也與眾多比丘一起在世尊附近經行;尊者大迦葉也與眾多比丘一起在世尊附近經行;尊者阿那律也與眾多比丘一起在世尊附近經行;尊者富樓那滿慈子也與眾多比丘一起在世尊附近經行;尊者優波離也與眾多比丘一起在世尊附近經行;尊者阿難也與眾多比丘一起在世尊附近經行;提婆達多也與眾多比丘一起在世尊附近經行。

  那時,世尊召喚比丘們:

  「比丘們!你們看見舍利弗與眾多比丘一起正在經行嗎?」

  「是的,\twnr{大德}{45.0}!」

  「比丘們!這些比丘全部是大慧者。

  比丘們!你們看見大目揵連與眾多比丘一起正在經行嗎?」

  「是的,大德!」

  「比丘們!這些比丘全部是大神通力者。

  比丘們!你們看見大迦葉與眾多比丘一起正在經行嗎?」

  「是的,大德!」

  「比丘們!這些比丘全部是\twnr{頭陀論者}{249.0}。

  比丘們!你們看見阿那律與眾多比丘一起正在經行嗎?」

  「是的,大德!」

  「比丘們!這些比丘全部是天眼者。

  比丘們!你們看見富樓那滿慈子與眾多比丘一起正在經行嗎?」

  「是的,大德!」

  「比丘們!這些比丘全部是論法者。

  比丘們!你們看見優波離與眾多比丘一起正在經行嗎?」

  「是的,大德!」

  「比丘們!這些比丘全部是持律者。

  比丘們!你們看見阿難與眾多比丘一起正在經行嗎?」

  「是的,大德!」

  「比丘們!這些比丘全部是多聞者。

  比丘們!你們看見提婆達多與眾多比丘一起正在經行嗎?」

  「是的,大德!」

  「比丘們!這些比丘全部是惡欲求者。

  比丘們!眾生就從界(由於界)會合、集合:下劣\twnr{志向}{257.0}者與下劣志向者一起會合、集合;善的志向者與善的志向者一起會合、集合。

  比丘們!過去世的眾生也曾就從界會合、集合:下劣志向者曾與下劣志向者一起會合、集合;善的志向者曾與善的志向者一起會合、集合。

  比丘們!\twnr{未來世}{308.0}的眾生也將就從界會合、集合:下劣志向者將與下劣志向者一起會合、集合;善的志向者將與善的志向者一起會合、集合。

  比丘們!現在,現在世的眾生也就從界會合、集合:下劣志向者與下劣志向者一起會合、集合;善的志向者與善的志向者一起會合、集合。」



\sutta{16}{16}{有偈經}{https://agama.buddhason.org/SN/sn.php?keyword=14.16}
  住在舍衛城……(中略)。

  「\twnr{比丘}{31.0}們!眾生就從界(由於界)會合、集合:下劣\twnr{志向}{257.0}者與下劣志向者一起會合、集合。

  比丘們!過去世的眾生也曾就從界會合、集合:下劣志向者曾與下劣志向者一起會合、集合。

  比丘們!\twnr{未來世}{308.0}的眾生也將就從界會合、集合:下劣志向者將與下劣志向者一起會合、集合。

  比丘們!現在,現在世的眾生也就從界會合、集合:下劣志向者與下劣志向者一起會合、集合。

  比丘們!猶如糞與糞會合、集合;尿與尿會合、集合;唾液與唾液會合、集合;膿與膿會合、集合;血與血會合、集合。同樣的,比丘們!眾生就從界會合、集合:下劣志向者與下劣志向者一起會合、集合。過去世……(中略)未來世……(中略)[比丘們!]現在,現在世的眾生也就從界會合、集合:下劣志向者與下劣志向者一起會合、集合。

  比丘們!眾生就從界會合、集合:善的志向者與善的志向者一起會合、集合。

  比丘們!過去世的眾生也曾就從界會合、集合:善的志向者曾與善的志向者一起會合、集合。

  比丘們!未來世……(中略)比丘們!現在,現在世的眾生也就從界會合、集合:善良志向者與善良志向者一起會合、集合。

  比丘們!猶如乳與乳會合、集合;油與油會合、集合;酥與酥會合、集合;蜜與蜜會合、集合;糖蜜與糖蜜會合、集合。同樣的,比丘們!眾生就從界會合、集合:善的志向者與善的志向者一起會合、集合。過去世……(中略)未來世……(中略)[比丘們!]現在,現在世的眾生也就從界會合、集合:善的志向者與善的志向者一起會合、集合。」

  \twnr{世尊}{12.0}說這個,說這個後,\twnr{善逝}{8.0}、\twnr{大師}{145.0}又更進一步說這個:

  「從接觸欲林被生,以不接觸被切斷,

   如登上小木頭後,會沈沒在大海中。

   這樣由於懈怠者,善生活者也沈沒,

   因此應該避開他:懈怠缺乏活力者。

   與獨居的聖者們,與自我努力的禪修者,

   與常發動活力者,應該與賢智者共住。」[\ccchref{It.78}{https://agama.buddhason.org/It/dm.php?keyword=78}]



\sutta{17}{17}{無信者合流經}{https://agama.buddhason.org/SN/sn.php?keyword=14.17}
  住在舍衛城……(中略)。

  「\twnr{比丘}{31.0}們!眾生就從界會合、集合:無信者與無信者一起會合、集合,無慚者與無慚者一起會合、集合,無愧者與無愧者一起會合、集合,少聞者與少聞者一起會合、集合,懈怠者與懈怠者一起會合、集合,\twnr{念已忘失}{216.0}者與念已忘失者一起會合、集合,劣慧者與劣慧者一起會合、集合。

  比丘們!過去世的眾生也曾就從界會合、集合:無信者曾與無信者一起會合、集合,無慚者曾與無慚者一起會合、集合,無愧者曾與無愧者一起會合、集合,少聞者曾與少聞者一起會合、集合,懈怠者曾與懈怠者一起會合、集合,念已忘失者曾與念已忘失者一起會合、集合,劣慧者曾與劣慧者一起會合、集合。

  比丘們!\twnr{未來世}{308.0}的眾生也將就從界會合、集合:無信者將與無信者一起會合、集合,無慚者將與無慚者一起會合、集合,無愧者將與無愧者一起……(中略)少聞者將與少聞者一起……(中略)懈怠者將與懈怠者一起……(中略)念已忘失者將與念已忘失者一起……(中略)劣慧者將與劣慧者一起會合、集合。

  比丘們!現在,現在世的眾生也就從界會合、集合:無信者與無信者一起會合、集合,無慚者與無慚者一起會合、集合,無愧者與無愧者一起……(中略)少聞者與少聞者一起……(中略)懈怠者與懈怠者一起……(中略)念已忘失者與念已忘失者一起……(中略)劣慧者與劣慧者一起會合、集合。

  比丘們!眾生就從界會合、集合:有信者與有信者一起會合、集合,有慚者與有慚者一起會合、集合,有愧者與有愧者一起會合、集合,多聞者與多聞者一起會合、集合,活力已發動者與活力已發動者一起會合、集合,\twnr{念已現起}{341.0}者與念已現起者一起會合、集合,有慧者與有慧者一起會合、集合。

  比丘們!過去世的……(中略)比丘們!未來世的……(中略)比丘們!現在,現在世的眾生也就從界會合、集合:有信者與有信者一起會合、集合……有慧者與有慧者一起會合、集合。」



\sutta{18}{18}{無信者之根經}{https://agama.buddhason.org/SN/sn.php?keyword=14.18}
  住在舍衛城……(中略)。

  「\twnr{比丘}{31.0}們!眾生就從界會合、集合:無信者與無信者一起會合、集合,無慚者與無慚者一起會合、集合,劣慧者與劣慧者一起會合、集合,有信者與有信者一起會合、集合,有慚者與有慚者一起會合、集合,有慧者與有慧者一起會合、集合。

  比丘們!過去世的眾生也曾就從界會合、集合:……(中略)比丘們!未來世的眾生也將就從界會合、集合:……(中略)比丘們!現在,現在世的眾生也就從界會合、集合:無信者與無信者一起會合、集合,無慚者與無慚者一起會合、集合,劣慧者與劣慧者一起會合、集合,有信者與有信者一起會合、集合,有慚者與有慚者一起會合、集合,有慧者與有慧者一起會合、集合。(1)

  比丘們!眾生就從界會合、集合:無信者與無信者一起會合、集合,無愧者與無愧者一起會合、集合,劣慧者與劣慧者一起會合、集合,有信者與有信者一起會合、集合,有愧者與有愧者一起會合、集合,有慧者與有慧者一起會合、集合。……(中略)應詳如第一部分那樣使之被細說。(2)

  比丘們!眾生就從界……(中略)無信者與無信者一起會合、集合,少聞者與少聞者一起會合、集合,劣慧者與劣慧者一起會合、集合,有信者與有信者一起會合、集合,多聞者與多聞者一起會合、集合,有慧者與有慧者一起會合、集合。……(中略)。(3)

  比丘們!眾生就從界……(中略)無信者與無信者一起會合、集合,懈怠者與懈怠者一起會合、集合,劣慧者與劣慧者一起會合、集合,有信者與有信者一起會合、集合,活力已發動者與活力已發動者一起會合、集合,有慧者與有慧者一起會合、集合。……(中略)。(4)

  比丘們!眾生就從界……(中略)無信者與無信者一起會合、集合,\twnr{念已忘失}{216.0}者與念已忘失者一起會合、集合,劣慧者與劣慧者一起會合、集合,有信者與有信者一起會合、集合,念已現起者與念已現起者一起會合、集合,有慧者與有慧者一起會合、集合。……(中略)。」(5)



\sutta{19}{19}{無慚者之根經}{https://agama.buddhason.org/SN/sn.php?keyword=14.19}
  住在舍衛城……(中略)。

  \twnr{比丘}{31.0}們!眾生就從界……(中略)無慚者與無慚者一起會合、集合,無愧者與無愧者一起會合、集合,劣慧者與劣慧者一起會合、集合,有慚者與有慚者一起會合、集合,有愧者與有愧者一起會合、集合,有慧者與有慧者一起會合、集合。……(中略)。(1)

  無慚者與無慚者一起會合、集合,少聞者與少聞者一起會合、集合,劣慧者與劣慧者一起會合、集合,有慚者與有慚者一起會合、集合,多聞者與多聞者一起會合、集合,有慧者與有慧者一起會合、集合。……(中略)。(2)

  無慚者與無慚者一起會合、集合,懈怠者與懈怠者一起會合、集合,劣慧者與劣慧者一起會合、集合,有慚者與有慚者一起會合、集合,活力已發動者與活力已發動者一起會合、集合,有慧者與有慧者一起會合、集合。……(中略)。(3)

  無慚者與無慚者一起會合、集合,\twnr{念已忘失}{216.0}者與念已忘失者一起會合、集合,劣慧者與劣慧者一起會合、集合,有慚者與有慚者一起會合、集合,念已現起者與念已現起者一起會合、集合,有慧者與有慧者一起會合、集合。……(中略)。(4)



\sutta{20}{20}{無愧者之根經}{https://agama.buddhason.org/SN/sn.php?keyword=14.20}
  住在舍衛城……(中略)。

  \twnr{比丘}{31.0}們!眾生就從界會合、集合:無愧者與無愧者一起會合、集合,少聞者與少聞者一起會合、集合,劣慧者與劣慧者一起會合、集合,有愧者與有愧者一起會合、集合,多聞者與多聞者一起會合、集合,有慧者與有慧者一起會合、集合。……(中略)。(1)

  [比丘們!眾生就從界會合、集合:]無愧者與無愧者一起會合、集合,懈怠者與懈怠者一起會合、集合,劣慧者與劣慧者一起會合、集合,有愧者與有愧者一起會合、集合,活力已發動者與活力已發動者一起會合、集合,有慧者與有慧者一起會合、集合。……(中略)。(2)

  [比丘們!眾生就從界會合、集合:]無愧者與無愧者一起會合、集合,\twnr{念已忘失}{216.0}者與念已忘失者一起會合、集合,劣慧者與劣慧者一起會合、集合,有愧者與有愧者一起會合、集合,念已現起者與念已現起者一起會合、集合,有慧者與有慧者一起會合、集合。……(中略)。(3)



\sutta{21}{21}{少聞者之根經}{https://agama.buddhason.org/SN/sn.php?keyword=14.21}
  住在舍衛城……(中略)。

  \twnr{比丘}{31.0}們!眾生就從界會合、集合:少聞者與少聞者一起會合、集合,懈怠者與懈怠者一起會合、集合,劣慧者與劣慧者一起會合、集合,多聞者與多聞者一起會合、集合,活力已發動者與活力已發動者一起會合、集合,有慧者與有慧者一起會合、集合。……(中略)。(1)

  比丘們!眾生就從界會合、集合:少聞者與少聞者一起會合、集合,\twnr{念已忘失}{216.0}者與念已忘失者一起會合、集合,劣慧者與劣慧者一起會合、集合,多聞者與多聞者一起會合、集合,念已現起者與念已現起者一起會合、集合,有慧者與有慧者一起會合、集合。……(中略)。(2)



\sutta{22}{22}{懈怠者之根經}{https://agama.buddhason.org/SN/sn.php?keyword=14.22}
  住在舍衛城……(中略)。

  \twnr{比丘}{31.0}們!眾生就從界會合、集合:懈怠者與懈怠者一起會合、集合,\twnr{念已忘失}{216.0}者與念已忘失者一起會合、集合,劣慧者與劣慧者一起會合、集合,活力已發動者與活力已發動者一起會合、集合,念已現起者與念已現起者一起會合、集合,有慧者與有慧者一起會合、集合。……(中略)

  第二品,其\twnr{攝頌}{35.0}:

  「這七[界]與有因緣,與以磚屋,

   下劣志向者、經行,有偈、無信者為第七,

   無信者之根五則,四則無慚者之根,

   無愧之根三則,二則少聞者之根,

   懈怠者之根一說,諸經有三[乘]五則,

   二十二經已說,被稱為第二品。」





\pin{業路品}{23}{29}
\sutta{23}{23}{不得定者經}{https://agama.buddhason.org/SN/sn.php?keyword=14.23}
  住在舍衛城……(中略)。

  「\twnr{比丘}{31.0}們!眾生就從界會合、集合:無信者與無信者一起會合、集合,無慚者與無慚者一起會合、集合,無愧者與無愧者一起會合、集合,不得定者與不得定者一起會合、集合,劣慧者與劣慧者一起會合、集合。

  有信者與有信者一起會合、集合,有慚者與有慚者一起會合、集合,有愧者與有愧者一起會合、集合,得定者與得定者一起會合、集合,有慧者與有慧者一起會合、集合。」



\sutta{24}{24}{破戒者經}{https://agama.buddhason.org/SN/sn.php?keyword=14.24}
  住在舍衛城……(中略)。

  「\twnr{比丘}{31.0}們!眾生就從界會合、集合:無信者與無信者一起會合、集合,無慚者與無慚者一起會合、集合,無愧者與無愧者一起會合、集合,破戒者與破戒者一起會合、集合,劣慧者與劣慧者一起會合、集合。

  有信者與有信者一起會合、集合,有慚者與有慚者一起會合、集合,有愧者與有愧者一起會合、集合,持戒者與持戒者一起會合、集合,有慧者與有慧者一起會合、集合。」







\sutta{25}{25}{五學處經}{https://agama.buddhason.org/SN/sn.php?keyword=14.25}
  住在舍衛城……(中略)。

  「\twnr{比丘}{31.0}們!眾生就從界會合、集合:殺生者與殺生者一起會合、集合;\twnr{未給予而取}{104.0}者與未給予而取者一起會合、集合;\twnr{邪淫}{105.0}者與邪淫者一起會合、集合;\twnr{妄語}{106.0}者與妄語者一起會合、集合;榖酒、果酒、\twnr{酒放逸處}{107.0}者與榖酒、果酒、酒放逸處者一起會合、集合。

  離殺生者與離殺生者一起會合、集合;離未給予而取者與離未給予而取者一起會合、集合;離邪淫者與離邪淫者一起會合、集合;離妄語者與離妄語者一起會合、集合;離榖酒、果酒、酒放逸處者與離榖酒、果酒、酒放逸處者一起會合、集合。」



\sutta{26}{26}{七種業路經}{https://agama.buddhason.org/SN/sn.php?keyword=14.26}
  住在舍衛城……(中略)。

  「\twnr{比丘}{31.0}們!眾生就從界會合、集合:殺生者與殺生者一起會合、集合;\twnr{未給予而取}{104.0}者與未給予而取者一起會合、集合;\twnr{邪淫}{105.0}者與邪淫者一起會合、集合;\twnr{妄語}{106.0}者與妄語者一起會合、集合;\twnr{離間語}{234.0}者與離間語者一起會合、集合;\twnr{粗惡語}{235.0}者與粗惡語者一起會合、集合;\twnr{雜穢語}{236.0}者與雜穢語者一起會合、集合。

  離殺生者……(中略)離未給予而取者……離邪淫者……離妄語者……離離間語者與離離間語者一起會合、集合;離粗惡語者與離粗惡語者一起會合、集合;離雜穢語者與離雜穢語者一起會合、集合。」



\sutta{27}{27}{十種業路經}{https://agama.buddhason.org/SN/sn.php?keyword=14.27}
  住在舍衛城……(中略)。

  「\twnr{比丘}{31.0}們!眾生就從界會合、集合:殺生者與殺生者一起會合、集合;\twnr{未給予而取}{104.0}者……(中略)\twnr{邪淫}{105.0}者……\twnr{妄語}{106.0}者……\twnr{離間語}{234.0}者……\twnr{粗惡語}{235.0}者……\twnr{雜穢語}{236.0}者與雜穢語者一起會合、集合;\twnr{貪婪者}{435.0}與貪婪者一起會合、集合;有瞋害心者與有瞋害心者一起會合、集合;邪見者與邪見者一起會合、集合。

  離殺生者與殺生者一起會合、集合;離未給予而取者……離邪淫者……離妄語者……[離]離間語者……[離]粗惡語者……離雜穢語者與離雜穢語者一起會合、集合;不貪婪者與不貪婪者一起會合、集合;無瞋害心者與無瞋害心者一起會合、集合;正見者與正見者一起會合、集合。」



\sutta{28}{28}{八支經}{https://agama.buddhason.org/SN/sn.php?keyword=14.28}
  住在舍衛城……(中略)。

  「\twnr{比丘}{31.0}們!眾生就從界會合、集合:邪見者與邪見者一起會合、集合;邪志者……(中略)邪語者……邪業者……邪命者……邪精進者……邪念者……邪定者與邪定者一起會合、集合。

  正見者與正見者一起會合、集合;正志者……(中略)正語者……正業者……正命者……正精進者……正念者……正定者與正定者一起會合、集合。」



\sutta{29}{29}{十支經}{https://agama.buddhason.org/SN/sn.php?keyword=14.29}
  住在舍衛城……(中略)。

  「\twnr{比丘}{31.0}們!眾生就從界會合、集合:邪見者與邪見者一起會合、集合;邪志者……(中略)邪語者……邪業者……邪命者……邪精進者……邪念者……邪定者與邪定者一起會合、集合;\twnr{邪智}{325.0}者與邪智者一起會合、集合;\twnr{邪解脫}{326.0}者與邪解脫者一起會合、集合。

  正見者與正見者一起會合、集合;正志者……(中略)正語者……正業者……正命者……正精進者……正念者……正定者……\twnr{正智}{976.0}者與正智者一起會合、集合;正解脫者與正解脫者一起會合、集合。」

  七經的\twnr{攝頌}{35.0}:

  不得定者、破戒者,與五學處,

  七種業路被說,與以十種業路,

  八支為第六說,與以十支為第七。

  業路品第三。





\pin{第四品}{30}{39}
\sutta{30}{30}{四界經}{https://agama.buddhason.org/SN/sn.php?keyword=14.30}
  \twnr{有一次}{2.0},\twnr{世尊}{12.0}住在舍衛城祇樹林給孤獨園。

  ……(中略)

  「\twnr{比丘}{31.0}們!有這四界,哪四個?地界、水界、火界、風界,比丘們!這些是四界。」



\sutta{31}{31}{正覺以前經}{https://agama.buddhason.org/SN/sn.php?keyword=14.31}
  住在舍衛城……(中略)。

  「\twnr{比丘}{31.0}們!當就在我\twnr{正覺}{185.1}以前,還是未\twnr{現正覺}{75.0}的\twnr{菩薩}{186.0}時想這個:『什麼是地界的\twnr{樂味}{295.0}呢?什麼是\twnr{過患}{293.0}?什麼是\twnr{出離}{294.0}?什麼是水界的樂味?什麼是過患?什麼是出離?什麼火界的是樂味?什麼是過患?什麼是出離?什麼是風界的樂味?什麼是過患?什麼是出離?』

  比丘們!那個我想這個:『凡\twnr{緣於}{252.0}地界樂、喜悅生起,這是地界的樂味;凡地界是無常的、苦的、\twnr{變易法}{70.0},這是地界的過患;凡在地界上意欲貪的調伏、意欲貪的捨斷,這是地界的出離。

  凡緣於水界……(中略)凡緣於火界……(中略)凡緣於風界樂、喜悅生起,這是風界的樂味;凡風界是無常的、苦的、變易法,這是風界的過患;凡在風界上意欲貪的調伏、意欲貪的捨斷,這是風界的出離。』

  比丘們!只要我不這樣如實證知這些四界的樂味為樂味、過患為過患、出離為出離,比丘們!我在包括天,在包括魔,在包括梵的世間;在包括沙門婆羅門,在包括天-人的\twnr{世代}{38.0}中,就不自稱『\twnr{已現正覺}{75.0}\twnr{無上遍正覺}{37.0}』。

  比丘們!但當我這樣如實證知這些四界的樂味為樂味、過患為過患、出離為出離,比丘們!那時,我在包括天,在包括魔,在包括梵的世間;在包括沙門婆羅門,在包括天-人的世代中,才自稱『已現正覺無上遍正覺』。而且,我的\twnr{智與見}{433.0}生起:『我的解脫是不動搖的,這是最後的出生,現在,沒有\twnr{再有}{21.0}。』」



\sutta{32}{32}{我曾行經}{https://agama.buddhason.org/SN/sn.php?keyword=14.32}
  住在舍衛城……(中略)。

  「\twnr{比丘}{31.0}們!我\twnr{曾行}{954.0}地界的\twnr{樂味}{295.0}之遍求,凡地界的樂味,我曾到達那個。地界的樂味之所及,那個被我以慧善見。

  比丘們!我曾行地界的\twnr{過患}{293.0}之遍求,凡地界的過患,我曾到達那個。地界的過患之所及,那個被我以慧善見。

  比丘們!我曾行地界的\twnr{出離}{294.0}之遍求,凡地界的出離,我曾到達那個。地界的出離之所及,那個被我以慧善見。

  比丘們!我[曾行]水界的……比丘們!我[曾行]火界的……比丘們!我曾行風界的樂味之遍求,凡風界的樂味,我曾到達那個。風界的樂味之所及,那個被我以慧善見。比丘們!我曾行風界的過患之遍求,凡風界的過患,我曾到達那個。風界的過患之所及,那個被我以慧善見。比丘們!我曾行風界的出離之遍求,凡風界的出離,我曾到達那個。風界的出離之所及,那個被我以慧善見。

  比丘們!只要我不如實證知這些四界的樂味為樂味、過患為過患、出離為出離,比丘們!我在包括天,在包括魔,在包括梵的世間;在包括沙門婆羅門,在包括天-人的\twnr{世代}{38.0}中,就不自稱『\twnr{已現正覺}{75.0}\twnr{無上遍正覺}{37.0}』。

  比丘們!但當我這樣如實證知這些四界的樂味為樂味、過患為過患、出離為出離,比丘們!那時,我在包括天,在包括魔,在包括梵的世間;在包括沙門婆羅門,在包括天-人的世代中,才自稱『已現正覺無上遍正覺』。而且,我的\twnr{智與見}{433.0}生起:『我的解脫是不動搖的,這是最後的出生,現在,沒有\twnr{再有}{21.0}。』」



\sutta{33}{33}{如果沒有這個經}{https://agama.buddhason.org/SN/sn.php?keyword=14.33}
  住在舍衛城……(中略)。

  「\twnr{比丘}{31.0}們!如果沒有地界的\twnr{樂味}{295.0},眾生不在地界上\twnr{貪著}{x297}。比丘們!但因為有地界的樂味,因此眾生在地界上貪著。

  比丘們!如果沒有地界的\twnr{過患}{293.0},眾生不在地界上\twnr{厭}{15.0}。比丘們!但因為有地界的過患,因此眾生在地界上厭。

  比丘們!如果沒有地界的\twnr{出離}{294.0},眾生不從地界出離。比丘們!但因為有地界的出離,因此眾生從地界出離。

  比丘們!如果沒有水界的樂味……(中略)比丘們!如果沒有火界的樂味……比丘們!如果沒有風界的樂味,眾生不為了風界貪著。比丘們!但因為有風界的樂味,因此眾生為了風界貪著。比丘們!如果沒有風界的過患,眾生不在風界上厭。比丘們!但因為有風界的過患,因此眾生在風界上厭。比丘們!如果沒有風界的出離,眾生不從這風界出離。比丘們!但因為有風界的出離,因此眾生從風界出離。

  比丘們!只要眾生不如實證知這四界的樂味為樂味、過患為過患、出離為出離,比丘們!這些眾生就還未從這包括天、魔、梵的世間;包括沙門婆羅門,包括天-人的\twnr{世代}{38.0}出離、離縛、脫離,\twnr{以離被限制之心}{555.0}而住。

  比丘們!但當眾生如實證知這四界的樂味為樂味、過患為過患、出離為出離,比丘們!眾生已從這包括天、魔、梵的世間;包括沙門婆羅門,包括天-人的世代中出離、離縛、脫離,以離被限制之心而住。」[≃\suttaref{SN.22.28}, \suttaref{SN.35.17}, \suttaref{SN.35.18}]



\sutta{34}{34}{一向的苦經}{https://agama.buddhason.org/SN/sn.php?keyword=14.34}
  住在舍衛城……(中略)。

  「\twnr{比丘}{31.0}們!如果這地界存在\twnr{一向的}{168.0}苦,已掉入苦,已進入苦,不被樂進入,眾生不在這地界上貪著。比丘們!但因為地界是樂的,已掉入樂,已進入樂,不被苦進入,因此眾生在地界上貪著。

  比丘們!如果這水界……(中略)比丘們!如果這火界……比丘們!如果這風界存在一向的苦,已掉入苦,已進入苦,不被樂進入,眾生不在這風界上貪著。比丘們!但因為風界是樂的,已掉入樂,已進入樂,不被苦進入,因此眾生在風界上貪著。

  比丘們!如果這地界存在一向的樂,已掉入樂,已進入樂,不被苦進入,眾生不在這地界上\twnr{厭}{15.0}。比丘們!但因為地界是苦的,已掉入苦,已進入苦,不被樂進入,因此眾生在地界上厭。

  比丘們!如果這水界……(中略)比丘們!如果這火界……比丘們!如果這風界存在一向的樂,已掉入樂,已進入樂,不被苦進入,眾生不在這風界上厭。比丘們!但因為風界是苦的,已掉入苦,已進入苦,不被樂進入,因此眾生在風界上厭。



\sutta{35}{35}{歡喜經}{https://agama.buddhason.org/SN/sn.php?keyword=14.35}
  住在舍衛城……(中略)。

  「\twnr{比丘}{31.0}們!凡歡喜地界者,他歡喜苦;凡歡喜苦者,我說:『他不從苦被解脫。』凡歡喜水界者……(中略)凡[歡喜]火界者……凡歡喜風界者,他歡喜苦;凡歡喜苦者,我說:『他不從苦被解脫。』

  比丘們!凡不歡喜地界者,他不歡喜苦;凡不歡喜苦者,我說:『他從苦被解脫。』凡[不歡喜]水界者……(中略)凡[不歡喜]火界者……凡不歡喜風界者,他不歡喜苦;凡不歡喜苦者,我說:『他從苦被解脫。』[≃\suttaref{SN.22.29}, \suttaref{SN.35.19}, \suttaref{SN.35.20}]



\sutta{36}{36}{生起經}{https://agama.buddhason.org/SN/sn.php?keyword=14.36}
  住在舍衛城……(中略)。

  「\twnr{比丘}{31.0}們!凡地界的\twnr{生起}{x298}、存續、生出、顯現,這是苦的生起、諸病的存續、老死的顯現。

  凡水界的……(中略)凡火界的……凡風界的生起、存續、生出、顯現,這是苦的生起、諸病的存續,老死的顯現。

  比丘們!而凡地界的\twnr{滅}{68.0}、平息、滅沒,這是苦的滅、諸病的平息、老死的滅沒。

  凡水界的……(中略)凡火界的……凡風界的滅、平息、滅沒,這是苦的滅、諸病的平息、老死的滅沒。」



\sutta{37}{37}{沙門婆羅門經}{https://agama.buddhason.org/SN/sn.php?keyword=14.37}
  住在舍衛城……(中略)。

  「\twnr{比丘}{31.0}們!有這些四界,哪四個?即:地界、水界、火界、風界,比丘們!凡任何\twnr{沙門}{29.0}或\twnr{婆羅門}{17.0}不如實知道這四界的\twnr{樂味}{295.0}、\twnr{過患}{293.0}、\twnr{出離}{294.0}者,比丘們!那些沙門或婆羅門不被我認同為\twnr{沙門中的沙門}{560.0},或婆羅門中的婆羅門,而且,那些\twnr{尊者}{200.0}也不以證智自作證後,在當生中\twnr{進入後住於}{66.0}\twnr{沙門義}{327.0}或婆羅門義。

  比丘們!而凡任何沙門或婆羅門如實知道這四界的樂味、過患、出離者,比丘們!那些沙門或婆羅門被我認同為沙門中的沙門,或婆羅門中的婆羅門,而且,那些尊者也以證智自作證後,在當生中進入後住於沙門義或婆羅門義。」



\sutta{38}{38}{沙門婆羅門經第二}{https://agama.buddhason.org/SN/sn.php?keyword=14.38}
  住在舍衛城……(中略)。

  「\twnr{比丘}{31.0}們!有這四界,哪四個?即:地界、水界、火界、風界,比丘們!凡任何\twnr{沙門}{29.0}或\twnr{婆羅門}{17.0}不如實知道這四界的\twnr{集起}{67.0}、滅沒、\twnr{樂味}{295.0}、\twnr{過患}{293.0}、\twnr{出離}{294.0}者……(中略)[如實]了知……(中略)以證智自作證後,[\twnr{在當生中}{42.0}]\twnr{進入後住於}{66.0}[\twnr{沙門義}{327.0}或婆羅門義]。」



\sutta{39}{39}{沙門婆羅門經第三}{https://agama.buddhason.org/SN/sn.php?keyword=14.39}
  住在舍衛城……(中略)。

  「比丘們!凡任何\twnr{沙門}{29.0}或\twnr{婆羅門}{17.0}不知道地界,不知道地界\twnr{集}{67.0},不知道地界\twnr{滅}{68.0},不知道導向地界\twnr{滅道跡}{69.0},不知道水界……(中略)不知道火界……不知道風界,不知道風界集,不知道風界滅,不知道導向風界滅道跡者,比丘們!那些沙門或婆羅門不被我認同為\twnr{沙門中的沙門}{560.0},或婆羅門中的婆羅門,而且,那些\twnr{尊者}{200.0}也不以證智自作證後,在當生中\twnr{進入後住於}{66.0}\twnr{沙門義}{327.0}或婆羅門義。

  比丘們!而凡任何沙門或婆羅門知道地界,知道地界集,知道地界滅,知道導向地界滅道跡,而凡任何沙門或婆羅門知道水界……(中略)知道火界……知道風界,知道風界集,知道風界滅,知道導向風界滅道跡者,比丘們!那些沙門或婆羅門被我認同為沙門中的沙門,或婆羅門中的婆羅門,而且,那些尊者也以證智自作證後,在當生中進入後住於沙門義或婆羅門義。」

  第四品,其\twnr{攝頌}{35.0}:

  「四[界]、以前、行,如果沒有這個與苦,

   歡喜與生起,三則沙門婆羅門。」

  界相應完成。





\page

\xiangying{15}{無始相應}
\pin{第一品}{1}{10}
\sutta{1}{1}{草木經}{https://agama.buddhason.org/SN/sn.php?keyword=15.1}
  \twnr{被我這麼聽聞}{1.0}:

  \twnr{有一次}{2.0},\twnr{世尊}{12.0}住在舍衛城祇樹林給孤獨園。

  在那裡,世尊召喚\twnr{比丘}{31.0}們:「比丘們!」

  「\twnr{尊師}{480.0}!」那些比丘回答世尊。

  世尊說這個:

  「比丘們!這輪迴是無始的,\twnr{無明蓋}{158.0}、渴愛結眾生的流轉的、輪迴的\twnr{起始點}{639.0}不被知道。

  比丘們!猶如男子切斷凡在這贍部洲中的草、木、枝、葉、收集在一起後,製作一一四指量(寬)的小樹枝後放置:『這是我的母親,這是我母親她的母親。』比丘們!那位男子的母親的母親還未被遍盡,而在這贍部洲中的草、木、枝、葉走到遍盡、耗盡(遍取),那是什麼原因?比丘們!這輪迴是無始的,無明蓋、渴愛結眾生的流轉的、輪迴的起始點不被知道。

  比丘們!你們已這麼長時間地經驗苦的,經驗尖銳的,經驗不幸,使墓地增加,比丘們!到那個程度,這就足以在一切諸行上\twnr{厭}{15.0},足以\twnr{離染}{558.0},足以解脫。」



\sutta{2}{2}{地經}{https://agama.buddhason.org/SN/sn.php?keyword=15.2}
  住在舍衛城……(中略)。

  「\twnr{比丘}{31.0}們!這輪迴是無始的,\twnr{無明蓋}{158.0}、渴愛結眾生的流轉的、輪迴的\twnr{起始點}{639.0}不被知道。

  比丘們!猶如男子製作這大地成一一棗核大小的泥丸後放置:『這是我的父親,這是我父親他的父親。』比丘們!那位男子的父親的父親還未被遍盡,而這大地走到遍盡、耗盡(遍取),那是什麼原因?比丘們!這輪迴是無始的,無明蓋、渴愛結眾生的流轉的、輪迴的起始點不被知道。

  比丘們!你們已這麼長時間地經驗苦的,經驗尖銳的,經驗不幸,使墓地增加,比丘們!到那個程度,這就足以在一切諸行上\twnr{厭}{15.0},足以\twnr{離染}{558.0},足以解脫。」



\sutta{3}{3}{淚經}{https://agama.buddhason.org/SN/sn.php?keyword=15.3}
  住在舍衛城……(中略)。

  「\twnr{比丘}{31.0}們!這輪迴是無始的,\twnr{無明蓋}{158.0}、渴愛結眾生的流轉的、輪迴的\twnr{起始點}{639.0}不被知道。

  比丘們!你們怎麼想它,哪個是比較多的呢:凡你們經這長時間流轉、輪迴,與不合意的結合、與合意的別離而悲泣、哭泣所流出的淚流,或凡四大海中的水?」

  「\twnr{大德}{45.0}!如我們了知\twnr{世尊}{12.0}教導的法,大德!這正是比較多的:凡我們經這長時間流轉、輪迴,與不合意的結合、與合意的別離而悲泣、哭泣所流出的淚流,而非四大海中的水。」

  「比丘們!\twnr{好}{44.0}!好!比丘們!你們這麼了知我教導的法,好!比丘們!這正是比較多的:凡你們經這長時間流轉、輪迴,與不合意的結合、與合意的別離而悲泣、哭泣所流出的淚流,而非四大海中的水。比丘們!你們已長時間地經驗母親的死亡,是你們經驗那些母親死亡的,與不合意的結合、與合意的別離而悲泣、哭泣所流出的淚流,而非四大海中的水;比丘們!你們已長時間地經驗父親死亡……(中略)經驗兄弟死亡……經驗姊妹死亡……經驗兒子死亡……經驗女兒死亡……經驗親族的損失……經驗財產的損失……比丘們!你們長時間地經驗疾病的損失,是你們經驗那些疾病損失的,與不合意的結合、與合意的別離而悲泣、哭泣所流出的淚流,而非四大海中的水,那是什麼原因?比丘們!這輪迴是無始的……(中略)比丘們!到那個程度,這就足以在一切諸行上\twnr{厭}{15.0},足以\twnr{離染}{558.0},足以解脫。」



\sutta{4}{4}{乳經}{https://agama.buddhason.org/SN/sn.php?keyword=15.4}
  住在舍衛城……(中略)。

  「\twnr{比丘}{31.0}們!這輪迴是無始的,\twnr{無明蓋}{158.0}、渴愛結眾生的流轉的、輪迴的\twnr{起始點}{639.0}不被知道。

  比丘們!你們怎麼想它,哪個是比較多的呢:凡你們經這長時間流轉、輪迴所喝的母乳,或凡四大海中的水?」

  「\twnr{大德}{45.0}!如我們了知\twnr{世尊}{12.0}教導的法,大德!這正是比較多的:凡我們經這長時間流轉、輪迴所喝的母乳,而非四大海中的水。」

  「比丘們!\twnr{好}{44.0}!好!比丘們!你們這麼了知我教導的法,好!比丘們!這正是比較多的:凡你們經這長時間流轉、輪迴所喝的母乳,而非四大海中的水,那是什麼原因?比丘們!這輪迴是無始的……(中略)足以解脫。」



\sutta{5}{5}{山經}{https://agama.buddhason.org/SN/sn.php?keyword=15.5}
  住在舍衛城……(中略)園。

  那時,\twnr{某位比丘}{39.0}去見世尊。抵達後,向世尊\twnr{問訊}{46.0}後,在一旁坐下。在一旁坐下的那位比丘對世尊說這個:

  「\twnr{大德}{45.0}!一劫有多長呢?」

  「比丘!一劫是長的,它不容易計算『幾年』,或『幾百年』,或『幾千年』,或『幾十萬年』。」

  「大德!但能作譬喻嗎?」

  「比丘!能。」世尊說。

  「比丘!猶如大岩山,長一\twnr{由旬}{148.0}、寬一由旬、高一由旬,無裂、無洞、堅固,男子每經過一百年以迦尸出產的布擦它一次,比丘!大岩山以這個行動比較快走到遍盡、耗盡(遍取),而非一劫。

  比丘!一劫這麼長,比丘!對這麼長的劫,不只一劫被輪迴了,不只一百劫被輪迴了,不只一千劫被輪迴了,不只十萬劫被輪迴了,那是什麼原因?比丘!這輪迴是無始的……\twnr{起始點}{639.0}……(中略)比丘!到那個程度,這就足以在一切諸行上\twnr{厭}{15.0},足以\twnr{離染}{558.0},足以解脫。」



\sutta{6}{6}{芥子經}{https://agama.buddhason.org/SN/sn.php?keyword=15.6}
  住在舍衛城。

  那時,\twnr{某位比丘}{39.0}去見\twnr{世尊}{12.0}。……(中略)

  在一旁坐下的那位比丘對世尊說這個:

  「\twnr{大德}{45.0}!一劫有多長呢?」 

  「比丘!一劫是長的,它不容易計算『幾年』,或……(中略)或『幾十萬年』。」

  「大德!但能作譬喻嗎?」

  「比丘!能。」世尊說。

  「比丘!猶如鐵製的城市,長一\twnr{由旬}{148.0}、寬一由旬、高一由旬,被充滿緊壓成團狀的\twnr{芥子}{x299},男子每經過一百年從那裡取出一粒芥子,比丘!大芥子堆以這個行動比較快走到遍盡、耗盡(遍取),而非一劫。

  比丘!一劫這麼長,比丘!對這麼長的劫,不只一劫被輪迴了,不只百劫被輪迴了,不只千劫被輪迴了,不只十萬劫被輪迴了,那是什麼原因?比丘!這輪迴是無始的……(中略)足以解脫。」



\sutta{7}{7}{弟子經}{https://agama.buddhason.org/SN/sn.php?keyword=15.7}
  住在舍衛城。

  那時,眾多\twnr{比丘}{31.0}去見\twnr{世尊}{12.0}。……(中略)

  在一旁坐下的那些比丘對世尊說這個:

  「\twnr{大德}{45.0}!有多少劫已過去、已經過呢?」

  「比丘們!很多劫已過去、已經過,它們不容易計算『幾劫』,或『幾百劫』,或『幾千劫』,或『幾十萬劫』。」

  「大德!但能作譬喻嗎?」

  「比丘們!能。」世尊說。

  「比丘們!這裡,如果有一百年壽命、一百年生命的四位弟子,他們每天各回憶十萬劫,比丘們!仍有劫未被他們回憶,而一百年壽命、一百年生命的那四位弟子經過一百年死了。

  比丘們!這麼多劫已過去、已經過,它們不容易計算『幾劫』,或『幾百劫』,或『幾千劫』,或『幾十萬劫』,那是什麼原因?比丘們!這輪迴是無始的……(中略)足以解脫。」



\sutta{8}{8}{恒河經}{https://agama.buddhason.org/SN/sn.php?keyword=15.8}
  住在王舍城[栗鼠飼養處的]竹林中。

  那時,\twnr{某位}{39.0}婆羅門去見\twnr{世尊}{12.0}。抵達後,與世尊一起互相問候。交換應該被互相問候的友好交談後,在一旁坐下。在一旁坐下的那位\twnr{婆羅門}{17.0}對世尊說這個:

  「\twnr{喬達摩}{80.0}尊師!有多少劫已過去、已經過呢?」

  「婆羅門!很多劫已過去、已經過,它們不容易計算『幾劫』,或『幾百劫』,或『幾千劫』,或『幾十萬劫』。」

  「但,喬達摩尊師!能作譬喻嗎?」

  「婆羅門!能。」世尊說。

  「婆羅門!猶如從這恒河起源到流入大海處,凡在這中間的沙,它們不容易計算『幾粒沙』,或『幾百粒沙』,或『幾千粒沙』,或『幾十萬粒沙』,婆羅門!比那個更多的劫已過去、已經過,它們不容易計算『幾劫』,或『幾百劫』,或『幾千劫』,或『幾十萬劫』,那是什麼原因?婆羅門!這輪迴是無始的,\twnr{無明蓋}{158.0}、渴愛結眾生的流轉的、輪迴的\twnr{起始點}{639.0}不被知道。

  婆羅門!已這麼長時間地經驗苦的,經驗尖銳的,經驗不幸,使墓地增加,婆羅門!到那個程度,這就足以在一切諸行上\twnr{厭}{15.0},足以\twnr{離染}{558.0},足以解脫。」

  在這麼說時,那位婆羅門對世尊說這個:

  「太偉大了,喬達摩尊師!太偉大了,喬達摩尊師!……(中略)請喬達摩\twnr{尊師}{203.0}記得我為\twnr{優婆塞}{98.0},從今天起\twnr{已終生歸依}{64.0}。」



\sutta{9}{9}{棍子經}{https://agama.buddhason.org/SN/sn.php?keyword=15.9}
  住在舍衛城。……(中略)

  「\twnr{比丘}{31.0}們!這輪迴是無始的,\twnr{無明蓋}{158.0}、渴愛結眾生的流轉的、輪迴的\twnr{起始點}{639.0}不被知道。

  比丘們!猶如棍子被向上投擲到空中,有時以底部落下,有時以中間落下,有時以頂端落下。同樣的,比丘們!無明蓋、渴愛結、流轉的、輪迴的眾生有時從這個世界到其它世界,有時從其它世界到這個世界[\suttaref{SN.56.33}],那是什麼原因?比丘們!這輪迴是無始的……(中略)足以解脫。」



\sutta{10}{10}{個人經}{https://agama.buddhason.org/SN/sn.php?keyword=15.10}
  \twnr{有一次}{2.0},\twnr{世尊}{12.0}住在王舍城\twnr{耆闍崛山}{258.0}。

  在那裡,世尊召喚\twnr{比丘}{31.0}們:「比丘們!」

  「\twnr{尊師}{480.0}!」那些比丘回答世尊。

  世尊說這個:

  「比丘們!這輪迴是無始的……(中略)。

  比丘們!當一位個人流轉、輪迴一劫時,如果被收集者收集,且不消失,會有如這毘富羅山這麼大的骨骸累積、骨骸堆積、骨骸聚集,那是什麼原因?比丘們!這輪迴是無始的……(中略)足以解脫。」

  世尊說這個,說這個後,\twnr{善逝}{8.0}、\twnr{大師}{145.0}又更進一步說這個:

  「經一劫一個人的,骨骸累積,

   會成為等同山的聚集,像這樣被大仙說。

   又這被宣說:它有毘富羅山大,

   在摩揭陀的王舍城,耆闍崛山北方的。

   當諸聖諦,他以正確之慧看見:

   苦、苦的生起,與苦的超越,

   以及\twnr{八支聖道}{525.0}:導向苦的寂靜。

   那個人最多七次,流轉後,

   從一切結的滅盡,有苦的結束。」

  第一品,其\twnr{攝頌}{35.0}:

  「草木與地,淚、乳與山,

   芥子、弟子、恒河,棍子與以個人。」





\pin{第二品}{11}{20}
\sutta{11}{11}{不幸者經}{https://agama.buddhason.org/SN/sn.php?keyword=15.11}
  \twnr{有一次}{2.0},\twnr{世尊}{12.0}住在舍衛城。

  在那裡,世尊召喚\twnr{比丘}{31.0}們:「比丘們!」

  「\twnr{尊師}{480.0}!」那些比丘回答世尊。

  世尊說這個:

  「比丘們!這輪迴是無始的,\twnr{無明蓋}{158.0}、渴愛結眾生的流轉的、輪迴的\twnr{起始點}{639.0}不被知道。

  比丘們!凡如果你們看到不幸者、來到不好者,結論應該被走到:『我們經這長時間也已像這樣經驗。』那是什麼原因?……(中略)比丘們!到那個程度,這就足以在一切諸行上\twnr{厭}{15.0},足以\twnr{離染}{558.0},足以解脫。」



\sutta{12}{12}{安樂者經}{https://agama.buddhason.org/SN/sn.php?keyword=15.12}
  住在舍衛城……(中略)。

  「\twnr{比丘}{31.0}們!這輪迴是無始的……(中略)。

  比丘們!凡如果你們看到安樂者、極幸福者,結論應該被走到:『我們經這長時間也已像這樣經驗過。』那是什麼原因?

  比丘們!這輪迴是無始的……\twnr{起始點}{639.0}不被知道……(中略)足以解脫。」



\sutta{13}{13}{約三十位經}{https://agama.buddhason.org/SN/sn.php?keyword=15.13}
  住在王舍城的竹林中。

  那時,約三十位波婆城的\twnr{比丘}{31.0}全是住\twnr{林野}{142.0}者、全是常乞食者、全是穿\twnr{糞掃衣}{352.0}者、全是但\twnr{三衣}{339.0}者、全是有結縛者,他們去見世尊。抵達後,向世尊\twnr{問訊}{46.0}後,在一旁坐下。

  那時,世尊想這個:

  「這約三十位波婆城的比丘全是住林野者、全是常乞食者、全是穿糞掃衣者、全是但三衣者、全是有結縛者,讓我教導他們像這樣的法,如是,就在這座位上,不執取後他們的心從諸\twnr{漏}{188.0}被解脫。」

  那時,世尊召喚比丘們:「比丘們!」

  「\twnr{尊師}{480.0}!」那些比丘回答世尊。

  世尊說這個:

  「比丘們!這輪迴是無始的,\twnr{無明蓋}{158.0}、渴愛結眾生的流轉的、輪迴的\twnr{起始點}{639.0}不被知道。比丘們!你們怎麼想它,哪個是比較多的呢:凡當你們經這長時間流轉、輪迴,頭被切斷時流出的血流,或凡四大海中的水?」

  「\twnr{大德}{45.0}!如我們了知世尊教導的法,大德!這正是比較多的:凡我們經這長時間流轉、輪迴,頭被切斷時流出的血流,而非四大海中的水。」

  「比丘們!好!好!比丘們!你們這麼了知我教導的法,好!比丘們!這正是比較多的:凡你們經這長時間流轉、輪迴,頭被切斷時流出的血流,而非四大海中的水;比丘們!……當你們長久是牛,為牛頭被切斷時流出的血流,而非四大海中的水;比丘們!……當你們長久是水牛,為水牛頭被切斷時流出的血流……(中略)比丘們!……當你們長久是羊,為羊……是山羊,為山羊……是鹿,為鹿……是雞,為雞……是豬,為豬……比丘們!……你們長久為『村落掠奪者』強盜,被捕捉後頭被切斷時流出的血流……比丘們!……你們長久為『攔路搶劫者』強盜,被捕捉後頭被切斷時流出的血流……比丘們!……你們長久為『通夫』強盜,被捕捉後頭被切斷時流出的血流,而非四大海中的水,那是什麼原因?比丘們!這輪迴是無始的……(中略)足以解脫。」

  世尊說這個,那些悅意的比丘歡喜世尊的所說。

  還有,\twnr{在當這個解說被說時}{136.0},這約三十位波婆城不執取後比丘的心從諸漏被解脫。



\sutta{14}{14}{母親經}{https://agama.buddhason.org/SN/sn.php?keyword=15.14}
  住在舍衛城……(中略)。

  「\twnr{比丘}{31.0}們!這輪迴是無始的……(中略)。

  比丘們!經這長時間,那是不容易得到的:非往昔的母親之眾生,那是什麼原因?比丘們!這輪迴是無始的……(中略)足以解脫。」



\sutta{15}{15}{父親經}{https://agama.buddhason.org/SN/sn.php?keyword=15.15}
  住在舍衛城……(中略)。

  「\twnr{比丘}{31.0}們!這輪迴是無始的……(中略)。

  比丘們!……那是不容易得到的:非往昔的父親之眾生……(中略)足以解脫。」



\sutta{16}{16}{兄弟經}{https://agama.buddhason.org/SN/sn.php?keyword=15.16}
  住在舍衛城……(中略)。

  「\twnr{比丘}{31.0}們!……那是不容易得到的:非往昔的兄弟之眾生……(中略)足以解脫。」



\sutta{17}{17}{姊妹經}{https://agama.buddhason.org/SN/sn.php?keyword=15.17}
  住在舍衛城……(中略)。

  「\twnr{比丘}{31.0}們!……那是不容易得到的:非往昔的姊妹之眾生……(中略)足以解脫。」



\sutta{18}{18}{兒子經}{https://agama.buddhason.org/SN/sn.php?keyword=15.18}
  住在舍衛城……(中略)。

  「\twnr{比丘}{31.0}們!……那是不容易得到的:非往昔的兒子之眾生……(中略)足以解脫。」



\sutta{19}{19}{女兒經}{https://agama.buddhason.org/SN/sn.php?keyword=15.19}
  住在舍衛城……(中略)。

  「\twnr{比丘}{31.0}們!這輪迴是無始的,\twnr{無明蓋}{158.0}、渴愛結眾生的流轉的、輪迴的\twnr{起始點}{639.0}不被知道。

  比丘們!經這長時間,那是不容易得到的:非往昔的女兒之眾生,那是什麼原因?比丘們!這輪迴是無始的,無明蓋、渴愛結眾生的流轉的、輪迴的起始點不被知道。

  比丘們!你們已這麼長時間地經驗苦的,經驗尖銳的,經驗不幸,使墓地增加,比丘們!到那個程度,這就足以在一切諸行上\twnr{厭}{15.0},足以\twnr{離染}{558.0},足以解脫。」



\sutta{20}{20}{毘富羅山經}{https://agama.buddhason.org/SN/sn.php?keyword=15.20}
  \twnr{有一次}{2.0},\twnr{世尊}{12.0}住在王舍城\twnr{耆闍崛山}{258.0}。

  在那裡,世尊召喚\twnr{比丘}{31.0}們:「比丘們!」

  「\twnr{尊師}{480.0}!」那些比丘回答世尊。

  世尊說這個:

  「比丘們!這輪迴是無始的,\twnr{無明蓋}{158.0}、渴愛結眾生的流轉的、輪迴的\twnr{起始點}{639.0}不被知道。

  比丘們!從前,這毘富羅山的稱呼就是『巴支那伐沙』,比丘們!又,當時,人們的稱呼就是『低伐羅』,比丘們!低伐羅人的壽量是四萬年,比丘們!低伐羅人以四天登上巴支那伐沙山,以四天下來。比丘們!又,當時,拘留孫世尊、\twnr{阿羅漢}{5.0}、\twnr{遍正覺者}{6.0}在世間出現(生起),比丘們!拘留孫世尊、阿羅漢、遍正覺者的第一雙弟子是名叫威度樂(無可比擬者)、慎伎哇(復活者)的雙賢。比丘們!你們看!這座山的稱呼她已消失、那些人已死、那位世尊已般涅槃。比丘們!諸行是這樣無常的,比丘們!諸行是這樣不堅固的,比丘們!諸行是這樣\twnr{不安的}{x300},比丘們!到那個程度,這就足以在一切諸行上\twnr{厭}{15.0},足以\twnr{離染}{558.0},足以解脫。

  比丘們!從前,這毘富羅山的稱呼就是『梵迦迦』,比丘們!又,當時,人們的稱呼就是『羅西塔沙』,比丘們!羅西塔沙人的壽量是三萬年,比丘們!羅西塔沙人以三天登上梵迦迦山,以三天下來。比丘們!又,當時,拘那含世尊、阿羅漢、遍正覺者在世間出現,比丘們!拘那含世尊、阿羅漢、遍正覺者的第一雙弟子是名叫畢佑沙、鬱多羅的雙賢。比丘們!你們看!這座山的稱呼她已消失、那些人已死、那位世尊已般涅槃。比丘們!諸行這樣無常……(中略)足以解脫。

  比丘們!從前,這毘富羅山的稱呼就是『殊巴沙』,比丘們!又,當時,人們的稱呼就是『殊必雅』,比丘們!殊必雅人的壽量為二萬年,比丘們!殊必雅人以二天登上殊巴沙山,以二天下來。比丘們!又,當時,迦葉世尊、阿羅漢、遍正覺者在世間出現,比丘們!迦葉世尊、阿羅漢、遍正覺者的第一雙弟子是名叫低舍、婆羅墮若的雙賢。比丘們!你們看!這座山的稱呼她已消失、那些人已死、那位世尊已般涅槃。比丘們!諸行這樣無常……(中略)足以解脫。

  比丘們!而現在,這毘富羅山的稱呼就是『毘富羅』,比丘們!又,現在,這些人的稱呼就是『摩揭陀』,比丘們!摩揭陀人的壽量是少的、微的、短的,凡活長久者,他有百年,或少量更多的,比丘們!摩揭陀人以片刻登上毘富羅山,以片刻下來。比丘們!又,現在,我世尊、阿羅漢、遍正覺者在世間出現,比丘們!又,我的第一雙弟子是名叫舍利弗、目揵連的雙賢。

  比丘們!將有那個時期:這座山的稱呼就將消失、這些人將死、我將般涅槃。比丘們!諸行是這樣無常的,比丘們!諸行是這樣不堅固的,比丘們!諸行是這樣不安的,比丘們!到那個程度,這就足以在一切諸行上厭,足以離染,足以解脫。」

  世尊說這個,說這個後,\twnr{善逝}{8.0}、\twnr{大師}{145.0}又更進一步說這個:

  「對低伐羅來說是巴支那伐沙,對羅西塔沙來說是梵迦迦,

   對殊必雅來說是殊巴沙,而對摩揭陀來說是毘富羅。

   諸行確實是無常的,是\twnr{生起與消散法的}{681.0},

   生起後被滅,它們的寂滅是樂。」[\suttaref{SN.6.15}]

  第二品,其\twnr{攝頌}{35.0}:

  「不幸者與快樂者,三十位與以父母,

   兄弟、姊妹與兒子,女兒、毘富羅山。」

  無始相應完成。





\page

\xiangying{16}{迦葉相應}
\sutta{1}{1}{滿足經}{https://agama.buddhason.org/SN/sn.php?keyword=16.1}
  住在舍衛城……(中略)。

  「\twnr{比丘}{31.0}們!這位迦葉被無論怎樣的衣服滿足,以及是無論怎樣的衣服之滿足的稱讚者,以及不因衣服而來到不適當的邪求,以及沒得到衣服後不\twnr{戰慄}{436.0},以及得到衣服後不繫結地、不迷昏頭地、無罪過地、看見\twnr{過患}{293.0}地、出離慧地受用。

  比丘們!這位迦葉被無論怎樣的\twnr{施食}{196.0}滿足,以及是無論怎樣的施食之滿足的稱讚者,以及不因施食而來到不適當的邪求,以及沒得到施食後不戰慄,以及得到施食後不繫結地、不迷昏頭地、無罪過地、看見過患地、出離慧地受用。

  比丘們!這位迦葉被無論怎樣的住處滿足,以及是無論怎樣的住處之滿足的稱讚者,以及不因住處而來到不適當的邪求,以及沒得到住處後不戰慄,以及得到住處後不繫結地、不迷昏頭地、無罪過地、看見過患地、出離慧地受用。

   比丘們!這位迦葉被無論怎樣的病人需要物與醫藥必需品滿足,以及是無論怎樣的病人需要物與醫藥必需品之滿足的稱讚者,以及不因病人需要物與醫藥必需品而來到不適當的邪求,以及沒得到病人需要物與醫藥必需品後不戰慄,以及得到病人需要物與醫藥必需品後不繫結地、不迷昏頭地、無罪過地、看見過患地、出離慧地受用。

  比丘們!因此,在這裡,應該被這麼學:『我們將被無論怎樣的衣服滿足,以及將是無論怎樣的衣服之滿足的稱讚者,以及不因衣服而將來到不適當的邪求,以及沒得到衣服後將不戰慄,以及得到衣服後將不繫結地、不迷昏頭地、無罪過地、看見過患地、出離慧地受用。(全都應被這樣作)我們將被無論怎樣的施食……(中略)將被無論怎樣的住處……(中略)將被無論怎樣的病人需要物與醫藥必需品滿足,以及將是無論怎樣的病人需要物與醫藥必需品之滿足的稱讚者,以及不因病人需要物與醫藥必需品而將來到不適當的邪求,以及沒得到病人需要物與醫藥必需品後將不戰慄,以及得到病人需要物與醫藥必需品後將不繫結地、不迷昏頭地、無罪過地、看見過患地、出離慧地受用。』比丘們!應該被你們這麼學。比丘們!我將以迦葉又或凡會是等同迦葉者教誡你們,而且以諸被教誡的應該被你們照著實行。」



\sutta{2}{2}{無愧者經}{https://agama.buddhason.org/SN/sn.php?keyword=16.2}
  \twnr{被我這麼聽聞}{1.0}:

  \twnr{有一次}{2.0},\twnr{尊者}{200.0}大迦葉與尊者舍利弗住在波羅奈仙人墜落處的鹿林。 

  那時,尊者舍利弗傍晚時,從\twnr{獨坐}{92.0}出來,去見尊者大迦葉。抵達後,與尊者大迦葉一起互相問候。交換應該被互相問候的友好交談後,在一旁坐下。在一旁坐下的尊者舍利弗對尊者大迦葉說這個:

  「迦葉\twnr{學友}{201.0}!這被說:『無熱心者、無\twnr{愧}{251.0}者是正覺的不可能者、涅槃的不可能者、到達無上\twnr{軛安穩}{192.0}的不可能者,而熱心者、有愧者是正覺的可能者、涅槃的可能者、到達無上軛安穩的可能者。』學友!什麼情形無熱心者、無愧者是正覺的不可能者、涅槃的不可能者、到達無上軛安穩的不可能者,學友!還有,什麼情形熱心者、有愧者是正覺的可能者、涅槃的可能者、到達無上軛安穩的可能者呢?」

  「[學友!而怎樣是無熱心者?]學友!這裡,\twnr{比丘}{31.0}對『當我未生起的諸惡不善法生起時,會轉起不利。』不作熱心;對『當我已生起的諸惡不善法不被捨斷時,會轉起不利。』不作熱心;對『當我未生起的諸善法不生起時,會轉起不利。』不作熱心;對『當我已生起的諸善法被滅時,會轉起不利。』不作熱心,學友!這樣是無熱心者。

  學友!而怎樣是無愧者?學友!這裡,比丘對『當我未生起的諸惡不善法生起時,會轉起不利。』不愧;對『當我已生起的諸惡不善法不被捨斷時,會轉起不利。』不愧;對『當我未生起的諸善法不生起時,會轉起不利。』不愧;對『當我已生起的諸善法被滅時,會轉起不利。』不愧,學友!這樣是無愧者。

  學友!這樣,無熱心者、無愧者是正覺的不可能者、涅槃的不可能者、到達無上軛安穩的不可能者。

  學友!而怎樣是熱心者?學友!這裡,比丘對『當我未生起的諸惡不善法生起時,會轉起不利。』作熱心;對『當我已生起的諸惡不善法不被捨斷時,會轉起不利。』作熱心;對『當我未生起的諸善法不生起時,會轉起不利。』作熱心;對『當我已生起的諸善法被滅時,會轉起不利。』作熱心,學友!這樣是熱心者。

  學友!而怎樣是愧者?學友!這裡,比丘對『當我未生起的諸惡不善法生起時,會轉起不利。』愧;對『當我已生起的諸惡不善法不被捨斷時,會轉起不利。』愧;對『當我未生起的諸善法不生起時,會轉起不利。』愧;對『當我已生起的諸善法被滅時,會轉起不利。』愧,學友!這樣是有愧者。

  學友!這樣,熱心者、有愧者是正覺的可能者、涅槃的可能者、到達無上軛安穩的可能者。」



\sutta{3}{3}{如月亮經}{https://agama.buddhason.org/SN/sn.php?keyword=16.3}
  住在舍衛城。……(中略)。

  「\twnr{比丘}{31.0}們!你們要\twnr{如月亮那樣}{x301}去諸家:就收斂身後,收斂心後,常如新手,在諸家上不傲慢的。

  比丘們!猶如男子注視老井,或山崖,或河的難渡處:就收斂身後,收斂心後。同樣的,比丘們!你們要如月亮那樣去諸家:就收斂身後,收斂心後,常如新手,在諸家上不傲慢的。

  比丘們!迦葉如月亮那樣去諸家:就收斂身後,收斂心後,常如新手,在諸家上不傲慢的。

  比丘們!你們怎麼想它:怎樣形色的比丘值得去諸家?」

  「\twnr{大德}{45.0}!我們的法是\twnr{世尊}{12.0}為根本的、\twnr{世尊為導引的}{56.0}、世尊為依歸的,大德!就請世尊說明這個所說的義理,\twnr{那就好了}{44.0}!聽聞世尊的[教說]後,比丘們將會\twnr{憶持}{57.0}。」

  那時,世尊使手在空中移動:

  「比丘們!猶如這隻手在空中不被黏著、不被捉住、不被繫縛。同樣的,比丘們!凡任何去諸家之比丘的心在諸家上不被黏著、不被捉住、不被繫縛:『令想要利得者他們得到!令想要福德者他們作福德!』如以自己的利得為悅意的、快樂的,這樣,以別人的利得為悅意的、快樂的,比丘們!像這樣形色的比丘值得前往諸家。

  比丘們!去諸家之迦葉的心不被黏著、不被捉住、不被繫縛:『令想要利得者他們得到!令想要福德者他們作福德!』如以自己的利得為悅意的、快樂的,這樣,以別人的利得為悅意的、快樂的。

  比丘們!你們怎麼想它:對怎樣形色的比丘來說是不清淨說法者?對怎樣形色的比丘來說是清淨說法者?」

  「大德!我們的法是世尊為根本的、世尊為導引的、世尊為依歸的,大德!就請世尊說明這個所說的義理,那就好了!聽聞世尊的[教說]後,比丘們將會憶持。」

  「比丘們!那樣的話,你們要聽!你們要\twnr{好好作意}{43.1}!我將說。」

  「是的,\twnr{大德}{45.0}!」那些比丘回答世尊。

  世尊說這個:

  「比丘們!凡任何比丘有這樣的心對別人教導法:『啊!願他們從我這裡聽聞法,而且聽聞後,願他們對法變得明淨,又,願淨信者們對我作\twnr{淨信的行為}{340.2}。』比丘們!對這樣形色的比丘來說是不清淨說法者。

  比丘們!凡比丘有這樣的心對別人教導法:『「被世尊善說的法是直接可見的、即時的、請你來看的、能引導的、應該被智者各自經驗的。」啊!願他們從我這裡聽聞法,而且聽聞後,願他們了知法,了知後,願他們照著實行。』像這樣,他\twnr{緣於}{252.0}法的善法性對別人教導法;他緣於悲愍、緣於憐愍、\twnr{出自憐愍}{121.0}對別人教導法,比丘們!對這樣形色的比丘來說是清淨說法者。

  比丘們!迦葉有這樣的心對別人教導法:『「被世尊善說的法是直接可見的、即時的、請你來看的、能引導的、應該被智者各自經驗的。」啊!願他們從我這裡聽聞法,而且聽聞後,願他們了知法,了知後,願他們照著實行。』像這樣,他緣於法的善法性對別人教導法;他緣於悲愍、緣於憐愍、出自憐愍對別人教導法。

  比丘們!我將以迦葉又或凡會是等同迦葉者教誡你們,而且以諸被教誡的應該被你們照著實行。」



\sutta{4}{4}{常出入某家者經}{https://agama.buddhason.org/SN/sn.php?keyword=16.4}
  住在舍衛城。……(中略)。

  「\twnr{比丘}{31.0}們!你們怎麼想它:怎樣形色的比丘值得為常出入某家者?怎樣形色的比丘不值得為常出入某家者?」

  「\twnr{大德}{45.0}!我們的法以\twnr{世尊}{12.0}為根本……(中略)。」

  世尊說這個:

  「比丘們!凡任何比丘有這樣的心去諸家:『令他們就施與我,不要不施與;令他們就施與我許多的,不要一點點的;令他們就施與我勝妙的,不要粗弊的;令他們快速地施與我,不要徐緩地;令他們恭敬地施與我,不要不恭敬地。』比丘們!當那位比丘有這樣的心去諸家時,如果他們不施與,比丘因為那樣被惱怒,他從那個因由感受苦憂。他們施與一點點的而非許多的……(中略)他們施與粗弊的而非勝妙的……他們徐緩地施與而非快速地,比丘因為那樣被惱怒,他從那個因由感受苦憂。他們不恭敬地施與而非恭敬地,比丘因為那樣被惱怒,他從那個因由感受苦憂。比丘們!這樣形色的比丘不值得為常出入某家者。

  比丘們!凡比丘有這樣的心去諸家:『在他人家,在這裡,\twnr{那如何可得}{847.0}:令他們就施與我,不要不施與;令他們就施與我許多的,不要一點點的;令他們就施與我勝妙的,不要粗弊的;令他們快速地施與我,不要徐緩地;令他們恭敬地施與我,不要不恭敬地。』比丘們!當那位比丘有這樣的心去諸家時,如果他們不施與,比丘不因為那樣被惱怒,他不從那個因由感受苦憂。他們施與一點點的而非許多的,比丘不因為那樣被惱怒,他不從那個因由感受苦憂。他們施與粗弊的而非勝妙的,比丘不因為那樣被惱怒,他不從那個因由感受苦憂。他們徐緩地施與而非快速地,比丘不因為那樣被惱怒,他不從那個因由感受苦憂。他們不恭敬地施與而非恭敬地,比丘不因為那樣被惱怒,他不從那個因由感受苦憂。比丘們!這樣形色的比丘值得為常出入某家者。

  比丘們!迦葉有這樣的心去諸家:『在他人家,在這裡,那如何可得:令他們就施與我,不要不施與;令他們就施與我許多的,不要一點點的;令他們就施與我勝妙的,不要粗弊的;令他們快速地施與我,不要徐緩地;令他們恭敬地施與我,不要不恭敬地。』比丘們!當迦葉有這樣的心去諸家時,如果他們不施與,迦葉不因為那樣被惱怒,他不從那個因由感受苦憂。他們施與一點點的而非許多的,迦葉不因為那樣被惱怒,他不從那個因由感受苦憂。他們施與粗弊的而非勝妙的,迦葉不因為那樣被惱怒,他不從那個因由感受苦憂。他們徐緩地施與而非快速地,迦葉不因為那樣被惱怒,他不從那個因由感受苦憂。他們不恭敬地施與而非恭敬地,迦葉不因為那樣被惱怒,他不從那個因由感受苦憂。

  比丘們!我將以迦葉又或凡會是等同迦葉者教誡你們,而且以諸被教誡的應該被你們照著實行。」



\sutta{5}{5}{已年老經}{https://agama.buddhason.org/SN/sn.php?keyword=16.5}
  \twnr{被我這麼聽聞}{1.0}:……(中略)在王舍城竹林。

  那時,\twnr{尊者}{200.0}大迦葉去見世尊。抵達後,向世尊\twnr{問訊}{46.0}後,在一旁坐下。世尊對在一旁坐下的尊者大迦葉說這個:

  「迦葉!現在,你已年老,這些丟棄布的\twnr{粗麻布}{x302}\twnr{糞掃衣}{352.0}對你是重的,迦葉!因此,在這裡,請你穿\twnr{屋主}{103.0}[奉獻]的衣服,同時也吃招待[的食物],以及住到我附近。」

  「\twnr{大德}{45.0}!我長時間是住\twnr{林野}{142.0}者,同時也是住林野狀態的稱讚者;是常乞食者,同時也是常乞食狀態的稱讚者;是穿糞掃衣者,同時也是穿糞掃衣狀態的稱讚者;是但三衣者,同時也是但三衣狀態的稱讚者;是少欲者,同時也是少欲的稱讚者;是知足者,同時也是知足的稱讚者;是獨居者,並同時也是獨居的稱讚者;是離群眾者,同時也是離群眾的稱讚者;是活力已發動者,同時也是活力發動的稱讚者。」

  「迦葉!但當看見什麼理由時,你長時間是住林野者,同時也是住林野狀態的稱讚者;是常乞食者……(中略)是穿糞掃衣者……是但三衣者……是少欲者……是知足者……是獨居者……是離群眾者……是活力已發動者,同時也是活力發動的稱讚者?」

  「大德!當看見二個理由時,我長時間是住林野者,同時也是住林野狀態的稱讚者;是常乞食者……(中略)是穿糞掃衣者……是但三衣者……是少欲者……是知足者……是獨居者……是離群眾者……是活力已發動者,同時也是活力發動的稱讚者:當看見自己的當生樂住處時,以及當憐愍後面的人時:『也許後面的人們會來到跟隨所見。』凡聽說:『那些佛陀的隨覺弟子們,他們長時間是住林野者,同時也是住林野狀態的稱讚者;是常乞食者……(中略)是穿糞掃衣者……是但三衣者……是少欲者……是知足者……是獨居者……是離群眾者……是活力已發動者,同時也是活力發動的稱讚者。』他們將照著實行,那對他們將有長久的利益、安樂。

  大德!當看見這二個理由時,我長時間是住林野者,同時也是住林野狀態的稱讚者;是常乞食者……(中略)是穿糞掃衣者……是但三衣者……是少欲者……是知足者……是獨居者……是離群眾者……是活力已發動者,同時也是活力發動的稱讚者。」

  「迦葉!\twnr{好}{44.0}!好!迦葉!你確實是為了眾人的利益,為了眾人的安樂,為了世間的憐愍,為了天-人們的需要、利益、安樂之行者。迦葉!因此,在這裡,請你穿丟棄布的粗麻布糞掃衣,以及請你\twnr{為了托鉢}{87.0}行走,以及請你住在林野。」



\sutta{6}{6}{教誡經}{https://agama.buddhason.org/SN/sn.php?keyword=16.6}
  在王舍城竹林中。

  那時,\twnr{尊者}{200.0}大迦葉去見世尊。抵達後,向\twnr{世尊}{12.0}\twnr{問訊}{46.0}後,在一旁坐下。世尊對在一旁坐下的尊者大迦葉說這個:「迦葉!請你教誡\twnr{比丘}{31.0}們,請你為比丘們作法談,迦葉!我或你應該教誡比丘們,迦葉!我或你應該為比丘們作法說。」

  「\twnr{大德}{45.0}!現在,比丘們是\twnr{難順從糾正者}{582.0},具備難作勸導法者、不忍耐者、不善於理解教誡者。大德!這裡,我看見阿難的共住者(弟子),名叫玻大的比丘,與阿那律的共住者,名叫阿毘居迦的比丘以所聽聞的相互挑釁:『來!比丘!誰將會說比較多,誰將會說比較好,誰將會說比較久。』」

  那時,世尊召喚某位比丘:「來!比丘!請你以我的名義召喚阿難的共住者玻大比丘,與阿那律的共住者阿毘居迦比丘:『大師召喚尊者們。』」「是的,大德!」那位比丘回答世尊後,去見那些比丘。抵達後,對那些比丘說這個:「大師召喚尊者們。」

  「是的,\twnr{學友}{201.0}!」那些比丘回答那位比丘後,去見世尊。抵達後,向世尊問訊後,在一旁坐下。世尊對在一旁坐下的那些比丘說這個:「比丘們!傳說是真的?你們以所聽聞的相互挑釁:『來!比丘!誰將會說比較多,誰將會說比較好,誰將會說比較久。』」「是的,大德!」「比丘們!你們了知法被我這麼教導:『來!比丘們!你們以所聽聞的相互挑釁:「來!比丘!誰將會說比較多,誰將會說比較好,誰將會說比較久。」』嗎?」「大德!這確實不是。」「比丘們!如果你們不了知法被我這麼教導,無用的男子們!那麼,那樣的話,當知道什麼時,當看見什麼時,在這麼善說的法律中出家的你們以所聽聞的相互挑釁:『來!比丘!誰將會說比較多,誰將會說比較好,誰將會說比較久。』呢?」

  那時,那些比丘\twnr{以頭落在世尊的腳上}{40.0}後,對世尊說這個:「大德!罪過征服如是愚的、如是愚昧的、如是不善的我們:凡在這麼善解說的法律中出家的我們以所學相互挑釁:『來!比丘!誰將會說比較多,誰將會說比較好,誰將會說比較久。』大德!為了未來的\twnr{自制}{217.0},請世尊接受那些我們的罪過為罪過。」

  「比丘們!確實,罪過征服如是愚的、如是愚昧的、如是不善的你們:凡在這麼善解說的法律中出家的你們以所學相互挑釁:『來!比丘!誰將會說比較多,誰將會說比較好,誰將會說比較久。』比丘們!但由於你們看見罪過是罪過後如法懺悔,我們接受你們的那個[懺悔]。比丘們!在聖者之律中這是增長:凡看見罪過為罪過後如法懺悔,未來來到自制。」



\sutta{7}{7}{教誡經第二}{https://agama.buddhason.org/SN/sn.php?keyword=16.7}
  住在王舍城竹林中。那時,\twnr{尊者}{200.0}大迦葉去見\twnr{世尊}{12.0}。……(中略)世尊對在一旁坐下的尊者大迦葉說這個:「迦葉!請你教誡\twnr{比丘}{31.0}們,請你為比丘們作法說,迦葉!我或你應該教誡比丘們,迦葉!我或你應該為比丘們作法說。」

  「\twnr{大德}{45.0}!現在,比丘們是\twnr{難順從糾正者}{582.0},具備難作勸導法者、不忍耐者、不善於理解教誡者。大德!凡任何在諸善法上沒有信者、在諸善法上沒有慚者、在諸善法上沒有愧者、在諸善法上沒有活力者、在諸善法上沒有慧者,對他來說,不論日或夜到來,在諸善法上僅減損能被預期,而非增長。

  大德!猶如\twnr{在黑暗側}{950.0},不論日或夜到來,月亮的容色被減損、圓相被減損、光亮被減損、直徑與圓周被減損。同樣的,大德!凡任何在諸善法上沒有信者、沒有慚者(中略)……沒有愧者……沒有活力者……在諸善法上沒有慧者,不論日或夜到來,對他來說,在諸善法上僅減損能被預期,非增長。

  大德!『無信的人(男子個人)』,這是衰退;大德!『無慚的人』,這是衰退;大德!『無愧的人』,這是衰退;大德!『怠惰的人』,這是衰退;大德!『\twnr{惡慧}{384.0}的人』,這是衰退;大德!『易怒的人』,這是衰退;大德!『懷怨恨的人』,這是衰退;大德!『沒有教誡的比丘們』,這是衰退。

  大德!凡任何在諸善法上有信者、在諸善法上有慚者、在諸善法上有愧者、在諸善法上有活力者、在諸善法上有慧者,對他來說,不論日或夜到來,在諸善法上僅增長能被預期,非減損。

  大德!猶如在明亮側,不論日或夜到來,月亮的容色增長、圓相增長、光亮增長、直徑與圓周增長。同樣的,大德!凡任何在諸善法上有信者……(中略)有慚者……有愧者……有活力者、在諸善法上有慧者,對他來說,不論日或夜到來,在諸善法上僅增長能被預期,非減損。

  大德!『有信的人』,這是不衰退;大德!『有慚的人』,這是不衰退;大德!『有愧的人』,這是不衰退;大德!『不怠惰的人』,這是不衰退;大德!『有慧的人』,這是不衰退;大德!『不易怒的人』,這是不衰退;大德!『不懷怨恨的人』,這是不衰退;大德!『有教誡比丘們』,這是不衰退。」

  「迦葉!好!好!迦葉!凡任何在諸善法上沒有信者……(中略)沒有慚者……沒有愧者……沒有活力者、在諸善法上沒有慧者,對他來說,不論日或夜到來,在諸善法上僅減損能被預期,而非增長。

  迦葉!猶如在黑暗側,不論日或夜到來,月亮的容色被減損……(中略)直徑與圓周被減損。同樣的,迦葉!凡任何在諸善法上沒有信者……(中略)沒有慚者……沒有愧者……沒有活力者、在諸善法上沒有慧者,對他來說,不論日或夜到來,在諸善法上僅減損能被預期,非增長。

  迦葉!『無信的人』,這是衰退;無慚的……(中略)無愧的……怠惰的……惡慧的……易怒的……迦葉!『懷怨恨的人』,這是衰退;迦葉!『沒有教誡的比丘們』,這是衰退。

  迦葉!凡任何在諸善法上有信者……(中略)有慚者……有愧者……有活力者……在諸善法上有慧者,則對他來說,不論日或夜到來,在諸善法上僅增長能被預期,非減損。

  迦葉!猶如在明亮側,不論日或夜到來,月亮的容色增長、圓相增長、光亮增長、直徑與圓周增長。同樣的,迦葉!凡任何在諸善法上有信者、有慚者……有愧者……有活力者……在諸善法上有慧者,對他來說,不論日或夜到來,在諸善法上僅增長能被預期,非減損。

  迦葉!『有信的人』,這是不衰退;有慚的……(中略)有愧的……不怠惰的……有慧的……不易怒的……迦葉!『不懷怨恨的人』,這是不衰退;迦葉!『有教誡的比丘們』,這是不衰退。」



\sutta{8}{8}{教誡經第三}{https://agama.buddhason.org/SN/sn.php?keyword=16.8}
  在王舍城栗鼠飼養處。那時,\twnr{尊者}{200.0}大迦葉去見\twnr{世尊}{12.0}。抵達後,向世尊\twnr{問訊}{46.0}後,在一旁坐下。世尊對在一旁坐下的尊者大迦葉說這個:「迦葉!請你教誡\twnr{比丘}{31.0}們,請你為比丘們作法說,迦葉!我或你應該教誡比丘們,迦葉!我或你應該為比丘們作法說。」

  「\twnr{大德}{45.0}!現在的比丘是\twnr{難順從糾正者}{582.0},具備難作勸導法者、不忍耐者、不善於理解教誡者。」

  「迦葉!確實如此,在以前,\twnr{上座}{135.0}比丘們是住\twnr{林野}{142.0}者,同時也是住林野狀態的稱讚者;是常乞食者,同時也是常乞食狀態的稱讚者;是穿糞掃衣者,同時也是穿糞掃衣狀態的稱讚者;是但\twnr{三衣}{339.0}者,同時也是但三衣狀態的稱讚者;是少欲者,同時也是少欲的稱讚者;是知足者,同時也是知足的稱讚者;是\twnr{獨居者}{x303},同時也是獨居的稱讚者;是離群眾者,同時也是離群眾的稱讚者;是活力已發動者,同時也是活力發動的稱讚者。

  在那裡,凡比丘是住林野者,同時也是住林野狀態的稱讚者;是常乞食者,同時也是常乞食狀態的稱讚者;是穿糞掃衣者,同時也是穿糞掃衣狀態的稱讚者;是但三衣者,同時也是但三衣狀態的稱讚者;是少欲者,同時也是少欲的稱讚者;是知足者,同時也是知足的稱讚者;是獨居者,並同時也是獨居的稱讚者;是離群眾者,同時也是離群眾的稱讚者;是活力已發動者,同時也是活力發動的稱讚者,上座比丘們以座位邀請他:『來!比丘!這位比丘是什麼名字?這位比丘確實是善的,這位比丘確實是想要學者,來!比丘!請你坐這個座位。』

  迦葉!在那裡,新學比丘們這麼想:『聽說凡比丘是住林野者,同時也是住林野狀態的稱讚者;是常乞食者,並且……(中略)是穿糞掃衣者,並且……是但三衣者,並且……是少欲者,並且……是知足者,並且……是獨居者,並且……是離群眾者,並且……是活力已發動者,同時也是活力發動的稱讚者,上座比丘們以座位邀請他:「來!比丘!這位比丘是什麼名字?這位比丘確實是善的,這位比丘確實是學的愛欲者,來!比丘!請你坐這個座位。」』他們照著實行,那對他們有長久的利益、安樂。

  迦葉!但,現在,上座比丘們不是住林野者,且不是住林野狀態的稱讚者;不是常乞食者,且不是常乞食狀態的稱讚者;不是穿糞掃衣者,且不是穿糞掃衣狀態的稱讚者;不是但三衣者,且不是但三衣狀態的稱讚者;不是少欲者,且不是少欲的稱讚者;不是知足者,且不是知足的稱讚者;不是獨居者,且不是獨居的稱讚者;不是離群眾者,且不是離群眾的稱讚者;不是活力已發動者,且不是活力發動的稱讚者。

  在那裡,凡有名聲的知名比丘是衣服、\twnr{施食}{196.0}、臥坐處、病人需物、醫藥必需品的利得者,上座比丘們以座位邀請他:『來!比丘!這位比丘是什麼名字?這位比丘確實是善的,這位比丘確實是\twnr{同梵行者}{138.0}的愛欲者,來!比丘!請你坐這個座位。』

  迦葉!在那裡,新學比丘們這麼想:『聽說凡有名聲的知名比丘是衣服、施食、臥坐處、病人需物、醫藥必需品的利得者,上座比丘們以座位邀請他:「來!比丘!這位比丘是什麼名字?這位比丘確實是善的,這位比丘確實是同梵行者的愛欲者,來!比丘!請你坐這個座位。」』他們照著實行,那對他們有長久的不利、苦。

  迦葉!凡當正確說它時,應該說『梵行者被梵行的禍害所禍害;梵行者被梵行的欲求所欲求』,迦葉!現在,當正確說它時,應該說『梵行者被梵行的禍害所禍害;梵行者被梵行的欲求所欲求』。」



\sutta{9}{9}{禪與證智經}{https://agama.buddhason.org/SN/sn.php?keyword=16.9}
  住在舍衛城……(中略)。

  「\twnr{比丘}{31.0}們!只要我希望,就從離諸欲後,從離諸不善法後,我\twnr{進入後住於}{66.0}有尋、\twnr{有伺}{175.0},\twnr{離而生喜、樂}{174.0}的初禪,比丘們!只要迦葉也希望,就從離諸欲後,從離諸不善法後,他進入後住於有尋、有伺,離而生喜、樂的初禪。

  比丘們!只要我希望,從尋與伺的平息,\twnr{自身內的明淨}{256.0},\twnr{心的專一性}{255.0},我進入後住於無尋、無伺,定而生喜、樂的第二禪,比丘們!只要迦葉也希望,從尋與伺的平息……(中略)他進入後住於……的第二禪。

  比丘們!只要我希望,從喜的\twnr{褪去}{77.0}、住於\twnr{平靜}{228.0}、有念正知、以身體感受樂,我進入後住於凡聖者們告知『他是平靜者、具念者、\twnr{安樂住者}{317.0}』的第三禪,比丘們!只要迦葉也希望,從喜的褪去、住於平靜、有念正知、以身體感受樂,他進入後住於凡聖者們宣說它『是平靜的、具念的、安樂住的』的第三禪。

  比丘們!只要我希望,從樂的捨斷與從苦的捨斷,就在之前諸喜悅、憂的滅沒,我進入後住於不苦不樂,\twnr{平靜、念遍純淨}{494.0}的第四禪,比丘們!只要迦葉也希望,從樂的捨斷……(中略)他進入後住於……的第四禪。

  比丘們!只要我希望,\twnr{從一切色想的超越}{490.0},從\twnr{有對想}{331.0}的滅沒,從不作意種種想[而知]:『虛空是無邊的』,我進入後住於虛空無邊處,比丘們!只要迦葉也希望,以一切色想的超越……(中略)他進入後住於虛空無邊處。

  比丘們!只要我希望,超越一切虛空無邊處後[而知]:『識是無邊的』,我進入後住於識無邊處,比丘們!只要迦葉也希望,超越一切虛空無邊處後[而知]:『識是無邊的』,他進入後住於識無邊處。

  比丘們!只要我希望,超越一切識無邊處後[而知]:『什麼都沒有』,我進入後住於無所有處,比丘們!只要迦葉也希望……(中略)他進入後住於無所有處。

  比丘們!只要我希望,超越一切無所有處後,我進入後住於非想非非想處,比丘們!只要迦葉也希望……(中略)他進入後住於非想非非想處。

  比丘們!只要我希望,超越一切非想非非想處後,我進入後住於想受滅,比丘們!只要迦葉也……(中略)他進入後住於想受滅。

  比丘們!只要我希望,我體驗各種神通種類:是一個後變成多個,又,是多個後變成一個;現身、隱身、穿牆、穿壘、穿山無阻礙地行走猶如在虛空中;在地中作浮沈猶如在水中,又,在不被破裂的水上行走猶如在地上;在空中以盤腿來去猶如有翅膀的鳥,又,以手碰觸、撫摸這些這麼大神通力、這麼大威力的日月;以身體行使自在直到梵天世界,比丘們!只要迦葉也希望,他體驗各種神通種類:……(中略)以身體行使自在直到梵天世界。

  比丘們!只要我希望,我以清淨、超越常人的天耳界聽到二者的聲音:「天與人,以及在遠處、近處。」比丘們!只要迦葉也希望……天耳界……(中略)是遠、是近的聲音。

  比丘們!只要我希望,我對其他眾生、其他個人\twnr{以心熟知心後}{393.0}知道:有貪的心為『有貪的心』,離貪的心知道為『離貪的心』;有瞋的心……(中略)離瞋的心……(中略)有癡的心……(中略)離癡的心……(中略)\twnr{收斂的心}{674.0}……(中略)散亂的心……(中略)廣大的心……(中略)無廣大的心……(中略)有更上的心……(中略)無更上的心……(中略)得定的心……(中略)未得定的心……(中略)已解脫的心……(中略)知道未解脫的心為『未解脫的心』,比丘們!只要迦葉也希望,他對其他眾生、其他個人以心熟知心後知道:有貪的心為『有貪的心』……(中略)知道未解脫的心為『未解脫的心』。

  比丘們!只要我希望,我回憶(隨念)許多前世住處,即:一生、二生、三生、四生、五生、十生、二十生、三十生、四十生、五十生、百生、千生、十萬生、許多壞劫、許多成劫、許多壞成劫:『在那裡我是這樣的名、這樣的姓氏、這樣的容貌、這樣的食物、這樣的苦樂感受、這樣的壽長,那位從那裡死後我出生在那裡,在那裡我又是這樣的名、這樣的姓氏、這樣的容貌、這樣的食物、這樣的苦樂感受、這樣的壽長,那位從那裡死後被再生在這裡。』像這樣,我回憶許多\twnr{有行相的、有境遇的}{500.0}前世住處,比丘們!只要迦葉也希望,他回憶(隨念)許多前世住處,即:一生……(中略)像這樣,回憶許多有行相的、有境遇的前世住處。

  比丘們!只要我希望,我以清淨、超越常人的天眼看見死沒往生的眾生:下劣的、勝妙的,美的、醜的,善去的、惡去的,知道依業到達的眾生:『確實,這些尊師眾生具備身惡行、具備語惡行、具備意惡行,是對聖者斥責者、邪見者、邪見行為的受持者,他們以身體的崩解,死後已往生\twnr{苦界}{109.0}、\twnr{惡趣}{110.0}、\twnr{下界}{111.0}、地獄,又或這些尊師眾生具備身善行、具備語善行、具備意善行,是對聖者不斥責者、正見者、正見行為的受持者,他們以身體的崩解,死後已往生\twnr{善趣}{112.0}、天界。』像這樣,我以清淨、超越常人的天眼看見死沒往生的眾生:下劣的、勝妙的,美的、醜的,善去的、惡去的,知道依業到達的眾生,比丘們!只要迦葉也希望,以清淨、超越常人的天眼,看見當眾生死時、往生時……(中略)知道眾生依業到達的。

  比丘們!我以諸\twnr{漏}{188.0}的滅盡,以證智自作證後,在當生中進入後住於無漏\twnr{心解脫}{16.0}、\twnr{慧解脫}{539.0},比丘們!迦葉也以諸漏的滅盡,以證智自作證後,在當生中進入後住於無漏心解脫、慧解脫。」



\sutta{10}{10}{住所經}{https://agama.buddhason.org/SN/sn.php?keyword=16.10}
  \twnr{被我這麼聽聞}{1.0}:

  \twnr{有一次}{2.0},\twnr{尊者}{200.0}大迦葉住在舍衛城祇樹林給孤獨園。

  那時,尊者阿難午前時穿衣、拿起衣鉢後,去見尊者大迦葉。抵達後,對尊者大迦葉說這個:

  「迦葉\twnr{大德}{45.0}!我們走,我們將去某個\twnr{比丘尼}{31.0}們住所。」

  「阿難\twnr{學友}{201.0}!你去吧!你有許多工作、許多應該被做的 。」

  第二次,尊者阿難又對尊者大迦葉說這個:

  「迦葉大德!我們走,我們將去某個比丘尼們住所。」

  「阿難學友!你去吧!你有許多工作、許多應該被做的 。」

  第三次,尊者阿難又對尊者大迦葉說這個:

  「迦葉大德!我們走,我們將去某個比丘尼們住所。」

  那時,尊者大迦葉午前時穿衣、拿起衣鉢後,以尊者阿難為隨從\twnr{沙門}{29.0},去某個比丘尼們住所。抵達後,在設置的座位坐下。

  那時,眾多比丘尼去見尊者大迦葉。抵達後,向尊者大迦葉\twnr{問訊}{46.0}後,在一旁坐下。尊者大迦葉對在一旁坐下的比丘尼以法說開示、勸導、鼓勵、\twnr{使歡喜}{86.0}。

  那時,尊者大迦葉對那些比丘尼以法說開示、勸導、鼓勵、使歡喜後,從座位起來後離開。

  那時,不悅意的胖低舍比丘尼使不悅意的話發出:

  「但,\twnr{聖}{612.1}大迦葉如何想在聖毘提訶牟尼阿難的面前,法應該被說呢?猶如賣針人想在做針人的面前,針應該被賣。同樣的,聖大迦葉想在聖毘提訶牟尼阿難的面前,法應該被說。」

  尊者大迦葉聽到胖低舍比丘尼說的這些話。

  那時,尊者大迦葉對尊者阿難說這個:

  「阿難學友!如何,我是賣針人,你是做針人嗎?還是,你是賣針人,我是做針人?」

  「迦葉大德!請你容忍,\twnr{女人是愚癡的}{x304}。」

  「阿難學友!請你等一下,不要\twnr{僧團}{375.0}對你進一步調查。阿難學友!你怎麼想它:是否你從\twnr{世尊}{12.0}面前,在比丘僧團中被提出:『比丘們!只要我希望,就從離諸欲後,從離諸不善法後,我\twnr{進入後住於}{66.0}有尋、\twnr{有伺}{175.0},\twnr{離而生喜、樂}{174.0}的初禪,比丘們!只要阿難也希望,就從離諸欲後,從離諸不善法後,他進入後住於有尋、有伺,離而生喜、樂的初禪。』嗎?」

  「大德!這確實不是。」

  「學友!是我從世尊的面前,在比丘僧團中被提出:『比丘們!只要我希望,就從離諸欲後,從離諸不善法後,我進入後住於有尋、有伺,\twnr{離而生喜、樂}{174.0}的初禪,比丘們!只要迦葉也希望,就從離諸欲後,從離諸不善法後,進入後住於……(中略)的初禪。』……[\suttaref{SN.16.9}](中略)(九次第住處\twnr{等至}{129.0}與五證智應該那樣細說使之被知道)。』

  阿難學友!你怎麼想它:是否你從世尊的面前,在比丘僧團中被提出:『比丘們!我以諸\twnr{漏}{188.0}的滅盡,以證智自作證後,在當生中進入後住於無漏\twnr{心解脫}{16.0}、\twnr{慧解脫}{539.0},比丘們!阿難也以諸漏的滅盡,以證智自作證後,在當生中進入後住於無漏心解脫、慧解脫。』嗎?」

  「大德!這確實不是。」

  「學友!是我從世尊的面前,在比丘僧團中被提出:『比丘們!我以諸漏的滅盡,以證智自作證後,在當生中進入後住於無漏心解脫、慧解脫,比丘們!迦葉也以諸漏的滅盡,以證智自作證後,在當生中進入後住於無漏心解脫、慧解脫。』

  學友!凡會想我的六證智能被覆蓋者,他會想七肘或七肘半的象能被棕櫚葉覆蓋。」

  還有,胖低舍比丘尼從梵行脫落。



\sutta{11}{11}{衣經}{https://agama.buddhason.org/SN/sn.php?keyword=16.11}
  \twnr{有一次}{2.0},\twnr{尊者}{200.0}大迦葉住在王舍城栗鼠飼養處的竹林中。

  當時,尊者阿難與大\twnr{比丘}{31.0}\twnr{僧團}{375.0}一起在南山進行遊行。

  當時,約三十位尊者阿難的共住者(弟子)比丘放棄學後還俗,他們多數是少年之類的。

  那時,尊者阿難在南山如其意地進行遊行後,前往王舍城栗鼠飼養處的竹林,去見尊者大迦葉。抵達後,向尊者大迦葉\twnr{問訊}{46.0}後,在一旁坐下。尊者大迦葉對在一旁坐下的尊者阿難說這個:

  「阿難\twnr{學友}{201.0}!\twnr{緣於}{252.0}多少理由,在諸家中三人共食被\twnr{世尊}{12.0}\twnr{安立}{143.0}呢?」

  「迦葉\twnr{大德}{45.0}!緣於三個理由,在諸家中三人共食被世尊安立:為了難羞愧之個人們的折伏、為了\twnr{美善}{947.0}比丘們的\twnr{安樂住}{156.0}:願依止黨翼的惡欲求者們不要分裂僧團,以及為了家的憐愍,迦葉大德!緣於這三個理由,在諸家中三人共食被世尊安立。」

  「阿難學友!那麼,那樣的話,為何你與這些不\twnr{在諸根上守護門的}{468.0}、不在飲食上知適量的、不專修清醒的年輕比丘一起進行遊行呢?看起來像你進行穀物的破壞;看起來像你進行家的破壞,阿難學友!你的群眾破碎,學友!你的新出發者們被瓦解,這位少年仍不知道衡量。」

  「迦葉大德!但在我的頭上已生出白髮,然而,現在我們仍不被從尊者大迦葉少年之語脫離。」

  「阿難學友!因為你像那樣還與這些不在諸根上守護門的、不在飲食上知適量的、不專修清醒的年輕比丘一起進行遊行,看起來像你進行穀物的破壞;看起來像你進行家的破壞,阿難學友!你的群眾破碎,學友!你的新出發者們被瓦解,這位少年仍不知道衡量。」

  胖難陀比丘尼聽聞:

  「聽說\twnr{聖}{612.1}毘提訶牟尼阿難被聖大迦葉以少年之語貶抑。」

  那時,不悅意的胖難陀比丘尼使不悅意的話發出:

  「但為何先前為其他外道\twnr{沙門}{29.0}的聖大迦葉想,聖毘提訶牟尼阿難應該被少年之語貶抑呢?」

  尊者大迦葉聽到胖難陀比丘尼說的這些話。

  那時,尊者大迦葉對尊者阿難說這個:

  「阿難學友!言語確實被胖難陀比丘尼不省察而說,學友!自從我剃除髮鬚、裹上\twnr{袈裟衣}{365.0}後,\twnr{從在家出家成為無家者}{48.0},我不證知(不記得)指定其他老師,除了世尊、\twnr{阿羅漢}{5.0}、遍正覺者外。

  學友!從前,為在家人的我想這個:『居家生活是障礙,是塵垢之路;\twnr{出家是露地}{410.0},當住在家中時,這是不容易行一向圓滿、一向遍純淨的\twnr{磨亮海螺}{411.0}的梵行,讓我剃除髮鬚、裹上袈裟衣後,從在家出家成為無家者。』

  學友!過些時候,那個我自己作布片布料的大衣後,凡世間中的阿羅漢們,指定他們[為師]後,我剃除髮鬚、裹上袈裟衣後,從在家出家成為無家者。

  當那個我成為這樣旅途中行走的出家者時,在王舍城與那爛陀中間,我看見坐在多子\twnr{塔廟}{366.0}的世尊。看見後,我想這個:『如果我確實看見老師,我正應該看見世尊;如果我確實看見\twnr{善逝}{8.0},我正應該看見世尊;如果我確實看見遍正覺者,我正應該看見世尊。』

  學友!那個我就在那裡以頭落在世尊的腳上後,對世尊說這個:『大德!世尊是我的老師,我是弟子;大德!世尊是我的老師,我是弟子。』

  學友!在這麼說時,世尊對我說這個:『迦葉!對凡這麼具備一切心的弟子,不知者如果就說:「我知道。」不見者如果就說:「我看見。」他的頭會破裂。但,迦葉!我是知者,我就說:「我知道。」我是見者,我就說:「我看見。」

  迦葉!因此,在這裡,應該被你這麼學:「我的慚愧將被強烈地現起在\twnr{上座}{135.0}、新學、中臘者上。」迦葉!應該被你這麼學。

  迦葉!因此,在這裡,應該被你這麼學:「凡我將聽聞任何一切與善有關的法,對那個\twnr{作目標後}{316.0}、作意後、\twnr{全心注意後}{479.0},我將傾耳聽聞法。」迦葉!應該被你這麼學。

  迦葉!因此,在這裡,應該被你這麼學:「我的\twnr{悅意俱行的身至念}{x305}將不捨棄。」迦葉!應該被你這麼學。』

  學友!那時,世尊以這個教誡教誡我後,從座位起來後,離開。學友!我只\twnr{{有諍}[負債]地吃國家施食}{854.0}七天,在第八天\twnr{完全智}{489.0}生起。

  學友!那時,世尊離開道路後,去某棵樹下。學友!那時,我將布片布料的大衣摺成四折後,對世尊說這個:『大德!請世尊坐這裡:凡對我會有長久的利益、安樂。』學友!世尊在設置的座位坐下。學友!坐下後,世尊對我說這個:『迦葉!你的這布片布料的大衣是柔軟的。』『大德!請世尊出自憐憫,接受我的布片布料的大衣。』『迦葉!那麼,你將穿我的丟棄布的粗麻布糞掃衣[\suttaref{SN.16.5}]?』『大德!我將穿世尊的丟棄布的粗麻布糞掃衣。』學友!那個我給與世尊布片布料的大衣,並且我走上世尊的丟棄布的粗麻布糞掃衣之路。

  學友!凡當正確說它時,應該說『世尊的親生子、從口出生者、法生者、法化作者、法的繼承人、領受丟棄布的粗麻布糞掃衣者』,那是我,當正確說時,應該說『世尊的親生子、從口出生者、法生者、法化作者、法的繼承人、領受丟棄布的粗麻布糞掃衣者』。

  學友!只要我希望,就從離諸欲後,從離諸不善法後,我\twnr{進入後住於}{66.0}有尋、\twnr{有伺}{175.0},\twnr{離而生喜、樂}{174.0}的初禪。

  學友!只要我希望……[\suttaref{SN.16.9}](中略)(九次第住處\twnr{等至}{129.0}與五證智應該那樣細說使之被知道)。

  學友!我以諸\twnr{漏}{188.0}的滅盡,以證智自作證後,在當生中進入後住於無漏\twnr{心解脫}{16.0}、\twnr{慧解脫}{539.0}。

  學友!凡會想我的六證智能被覆蓋者,他會想七肘或七肘半的象能被棕櫚葉覆蓋。」

  還有,胖難陀比丘尼從梵行脫落。



\sutta{12}{12}{死後經}{https://agama.buddhason.org/SN/sn.php?keyword=16.12}
  \twnr{有一次}{2.0},\twnr{尊者}{200.0}大迦葉與尊者舍利弗住在波羅奈仙人墜落處的鹿林。

  那時,尊者舍利弗傍晚時,從\twnr{獨坐}{92.0}出來,去見尊者大迦葉。抵達後,與尊者大迦葉一起互相問候。交換應該被互相問候的友好交談後,在一旁坐下。在一旁坐下的尊者舍利弗對尊者大迦葉說這個:

  「迦葉\twnr{學友}{201.0}!怎麼樣,死後如來存在嗎?」

  「學友!這不被如來記說:『死後如來存在。』」

  「學友!那麼,死後如來不存在嗎?」

  「學友!這樣也不被如來記說:『死後如來不存在。』」

  「學友!怎麼樣,\twnr{死後如來存在且不存在}{354.0}嗎?」

  「學友!這不被如來記說:『死後如來存在且不存在。』」

  「學友!那麼,死後如來既非存在也非不存在嗎?」

  「學友!這樣也不被如來記說:『死後如來既非存在也非不存在。』」

  「學友!但為何這不被如來記說?」

  「學友!因為這是不\twnr{伴隨利益的}{50.0},非\twnr{梵行基礎的}{446.0},不對\twnr{厭}{15.0}、不對\twnr{離貪}{77.0}、不對\twnr{滅}{68.0}、不對寂靜、不對證智、不對\twnr{正覺}{185.1}、不對涅槃轉起,因此它不被如來記說。」

  「學友!那麼,那樣的話,什麼被如來記說?」

  「學友!『這是苦』被如來記說,『這是苦集』被如來記說,『這是苦滅』被如來記說,『這是導向苦\twnr{滅道跡}{69.0}』被如來記說。」

  「學友!但為何這被如來記說?」

  「學友!因為這是伴隨利益的,這是梵行基礎的,這對厭、對離貪、對滅、對寂靜、對證智、對正覺、對涅槃轉起,因此它被如來記說。」



\sutta{13}{13}{相似正法經}{https://agama.buddhason.org/SN/sn.php?keyword=16.13}
  \twnr{被我這麼聽聞}{1.0}:

  \twnr{有一次}{2.0},\twnr{世尊}{12.0}住在舍衛城祇樹林給孤獨園。

  那時,\twnr{尊者}{200.0}大迦葉去見世尊。抵達後,向世尊\twnr{問訊}{46.0}後,在一旁坐下。在一旁坐下的尊者大迦葉對世尊說這個:

  「\twnr{大德}{45.0}!什麼因、什麼\twnr{緣}{180.0},以那個在以前有較少的\twnr{學處}{392.0},同時也較多的\twnr{比丘}{31.0}在\twnr{完全智}{489.0}上確立呢?大德!又,什麼因、什麼緣,以那個現在有較多的學處,同時也較少的比丘在完全智上確立?」

  「迦葉!而這是這樣:在眾生減少中、在正法消失中時,有較多的學處,同時也較少的比丘在完全智上確立,迦葉!只要\twnr{相似正法}{x306}不在世間出現,正法的消失就不存在,迦葉!但當相似正法在世間出現,那時正法的消失存在。

  迦葉!猶如只要相似黃金不在世間出現,黃金的消失就不存在,迦葉!但當相似黃金在世間出現,那時黃金的消失存在。同樣的,迦葉!只要相似正法不在世間出現,正法的消失就不存在,迦葉!但當相似正法在世間出現,那時正法的消失存在。

  迦葉!地界不使正法消失,水界不使正法消失,火界不使正法消失,風界不使正法消失,在這裡,反而正是那些無用的男子出現,他們使這正法消失。迦葉!猶如船立即就沈沒,迦葉!正法的消失不是這樣。

  迦葉!有這五個能退墮的法,它們轉起正法的混亂、消失,哪五個?迦葉!這裡,比丘、比丘尼、\twnr{優婆塞}{98.0}、\twnr{優婆夷}{99.0}在大師上住於不尊敬的、不順從的;在法上住於不尊敬的、不順從的;在\twnr{僧團}{375.0}上住於不尊敬的、不順從的;在學上上住於不尊敬的、不順從的;在定上住於不尊敬的、不順從的,迦葉!這是五個能退墮的法,它們轉起正法的混亂、消失。

  迦葉!有這五法,它們轉起正法的存續、不混亂、不消失,哪五個?迦葉!這裡,比丘、比丘尼、優婆塞、優婆夷在大師上住於尊敬的、順從的;在法上住於尊敬的、順從的;在僧團上住於尊敬的、順從的;在學上住於尊敬的、順從的;在定上住於尊敬的、順從的,迦葉!這是五法,它們轉起正法的存續、不混亂、不消失。」

  迦葉相應完成,其\twnr{攝頌}{35.0}:

  「滿足與無愧者,如月亮、常出入某家者,

   已年老與三則教誡,禪與證智、住所,

   衣、死後,相似正法。」





\page

\xiangying{17}{利得恭敬相應}
\pin{第一品}{1}{10}
\sutta{1}{1}{恐畏經}{https://agama.buddhason.org/SN/sn.php?keyword=17.1}
  \twnr{被我這麼聽聞}{1.0}:

  \twnr{有一次}{2.0},\twnr{世尊}{12.0}住在舍衛城祇樹林給孤獨園。

  在那裡,世尊召喚\twnr{比丘}{31.0}們:「比丘們!」

  「\twnr{尊師}{480.0}!」那些比丘回答世尊。

  世尊說這個:

  「比丘們!利得、恭敬、名聲是恐怖的、辛辣的、粗惡的,是到達無上\twnr{軛安穩}{192.0}的障礙,比丘們!因此,在這裡,應該被這麼學:『我們要捨斷已生起的利得、恭敬、名聲,而且,已生起的利得、恭敬、名聲不要\twnr{持續遍取}{530.0}我們的心。』比丘們!應該被你們這麼學。」



\sutta{2}{2}{釣鉤經}{https://agama.buddhason.org/SN/sn.php?keyword=17.2}
  住在舍衛城……(中略)。

  「比丘們!利得、恭敬、名聲是恐怖的、辛辣的、粗惡的,是到達無上\twnr{軛安穩}{192.0}的障礙,比丘們!猶如漁夫在深湖中投入有餌的釣鉤,某隻覓食的魚吞下它,比丘們!這樣,那是吞下漁夫的釣鉤、來到不幸、來到災厄、被漁夫為所欲為的魚。

   比丘們!『漁夫』,這是魔\twnr{波旬}{49.0}的同義語;比丘們!『釣鉤』,這是利得、恭敬、名聲的同義語,比丘們!凡任何比丘樂味、欲求已生起的利得、恭敬、名聲者,比丘們!這被稱為吞下魔的釣鉤、來到不幸、來到災厄、被波旬為所欲為的比丘。比丘們!利得、恭敬、名聲是這麼恐怖的、辛辣的、粗惡的,是到達無上\twnr{軛安穩}{192.0}的障礙。比丘們!因此,在這裡,應該被這麼學:『我們要捨斷已生起的利得、恭敬、名聲,而且,已生起的利得、恭敬、名聲不要\twnr{持續遍取}{530.0}我們的心。』比丘們!應該被你們這麼學。」



\sutta{3}{3}{龜經}{https://agama.buddhason.org/SN/sn.php?keyword=17.3}
  住在舍衛城……(中略)。

  「\twnr{比丘}{31.0}們!利得、恭敬、名聲是恐怖的……(中略)。比丘們!從前,在某個水池中有大家族的龜久住,比丘們!那時,某隻龜對另一隻龜說這個:『親愛的龜!你不要去這個地方。』但該隻龜去那個地方,獵人\twnr{以陷阱叉}{x307}刺穿牠,比丘們!那時,該隻龜去見那隻龜。比丘們!那隻龜看見正從遠處到來的該隻龜。見了後,對該隻龜說這個:『親愛的龜!希望你沒去那個地方。』『親愛的龜!我確實去那個地方。』『親愛的龜!那麼,是否你是無受傷的、無損壞的?』『親愛的龜!我是無受傷的、無損壞的,但有這條線一直在我的背後跟隨。』『親愛的龜!你確實是受傷的、確實是損壞的,親愛的龜!因為,你的父親與祖父被這條線來到不幸、來到災厄,親愛的龜!現在,請你走,現在,你不屬於我們了。』

   比丘們!『獵人』,這是魔\twnr{波旬}{49.0}的同義語;比丘們!『陷阱叉』,這是利得、恭敬、名聲的同義語;比丘們!『線』,這是歡喜、貪的同義語。比丘們!凡任何比丘樂味、欲求已生起的利得、恭敬、名聲者,比丘們!這被稱為貪求的比丘被陷阱叉來到不幸、來到災厄,被波旬為所欲為。比丘們!利得、恭敬、名聲是這麼恐怖的……(中略)比丘們!應該被你們這麼學。」



\sutta{4}{4}{長毛經}{https://agama.buddhason.org/SN/sn.php?keyword=17.4}
  住在舍衛城……(中略)。

  「\twnr{比丘}{31.0}們!利得、恭敬、名聲是恐怖的……(中略)。比丘們!猶如長毛母山羊進入荊棘密叢,牠到處被黏著,到處被捕捉,到處被繫縛,到處來到不幸、災厄。同樣的,比丘們!這裡,某位比丘被利得、恭敬、名聲征服而心被遍取,他午前時穿衣、拿起衣鉢後,\twnr{為了托鉢}{87.0}進入村落或城鎮,他到處被黏著,到處被捕捉,到處被繫縛,到處來到不幸、災厄。比丘們!利得、恭敬、名聲是這麼恐怖的……(中略)比丘們!應該被你們這麼學。」



\sutta{5}{5}{糞蟲經}{https://agama.buddhason.org/SN/sn.php?keyword=17.5}
  住在舍衛城……(中略)。

  「\twnr{比丘}{31.0}們!利得、恭敬、名聲是恐怖的……(中略)。比丘們!猶如食糞、滿滿糞、充滿糞的糞蟲,而在牠的前面有一大團糞,牠因為那樣輕蔑其他糞蟲:『我是食糞者、滿滿糞者、充滿糞者,在我的前面有這一大團糞。』同樣的,比丘們!這裡,某位比丘被利得、恭敬、名聲征服而心被遍取,他午前時穿衣、拿起衣鉢後,\twnr{為了托鉢}{87.0}進入村落或城鎮,在那裡,他是盡情的食者、明天被招請者、他的鉢食是滿的。他走到僧園後,在比丘眾中誇示:『我是盡情的食者、我是明天被招請者、我的鉢食是滿的,而且我是衣服、施食、臥坐處、病人需物、醫藥必需品的利得者,但這些其他比丘是少福德、少能力者,衣服、施食、臥坐處、病人需物、醫藥必需品的無利得者。』他被這利得、恭敬、名聲征服而心被遍取,他輕蔑其他美善的比丘,比丘們!那對那位無用的男子有長久的不利、苦。比丘們!利得、恭敬、名聲是這麼恐怖的……(中略)比丘們!應該被你們這麼學。」



\sutta{6}{6}{雷電經}{https://agama.buddhason.org/SN/sn.php?keyword=17.6}
  住在舍衛城……(中略)。

  「\twnr{比丘}{31.0}們!利得、恭敬、名聲是恐怖的……(中略)。比丘們!令落雷來到誰?令到達利得、恭敬、名聲的心意未達成\twnr{有學}{193.0}。

   比丘們!『落雷』,這是利得、恭敬、名聲的同義語。比丘們!利得、恭敬、名聲是這麼恐怖的……(中略)比丘們!應該被你們這麼學。」



\sutta{7}{7}{塗有毒的經}{https://agama.buddhason.org/SN/sn.php?keyword=17.7}
  住在舍衛城……(中略)。

  「\twnr{比丘}{31.0}們!利得、恭敬、名聲是恐怖的……(中略)。比丘們!令\twnr{以到達塗有毒的、塗上毒的箭}{x308}刺穿誰?令到達利得、恭敬、名聲的心意未達成\twnr{有學}{193.0}。

   比丘們!『箭』,這是利得、恭敬、名聲的同義語。比丘們!利得、恭敬、名聲是這麼恐怖的……(中略)比丘們!應該被你們這麼學。」



\sutta{8}{8}{狐狼經}{https://agama.buddhason.org/SN/sn.php?keyword=17.8}
  住在舍衛城……(中略)。

  「\twnr{比丘}{31.0}們!利得、恭敬、名聲是恐怖的……(中略)。比丘們!你們在破曉時聽到老狐狼鳴叫著嗎?」

  「是的,\twnr{大德}{45.0}!」

  「比丘們!那隻老狐狼被名為疥瘡生起的病接觸,牠來到洞穴既不喜樂,來到樹下也不喜樂,來到露地也不喜樂,所到之處、所站之處、所坐之處、所臥之處,處處都來到不幸、災厄。同樣的,比丘們!這裡,某位比丘被利得、恭敬、名聲征服而心被遍取,他來到空屋既不喜樂,來到樹下也不喜樂,來到露地也不喜樂,所到之處、所站之處、所坐之處、所臥之處,處處都來到不幸、災厄。比丘們!利得、恭敬、名聲是這麼恐怖的……(中略)比丘們!應該被你們這麼學。」



\sutta{9}{9}{迅猛風經}{https://agama.buddhason.org/SN/sn.php?keyword=17.9}
  住在舍衛城……(中略)。

  「\twnr{比丘}{31.0}們!利得、恭敬、名聲是恐怖的……(中略)。比丘們!在上空,名叫\twnr{迅猛風}{x309}吹,凡有翅膀的到那裡,迅猛風拋擲牠。當牠被迅猛風拋擲時,腳就到一邊,翅膀到另一邊,頭到另一邊,身體到另一邊。同樣的,比丘們!這裡,某位比丘被利得、恭敬、名聲征服而心被遍取,他午前時穿衣、拿起衣鉢後,以身未守護、以語未守護、以心未守護,以念未現起,以諸根未防護,\twnr{為了托鉢}{87.0}進入村落或城鎮。在那裡,他看見穿著暴露的或衣著暴露的婦女。看見穿著暴露的或衣著暴露的婦女後,貪使他的心墮落,他以貪使墮落的心放棄學後還俗。其他人取走他的衣服,其他人取走鉢,其他人取走坐墊布,其他人取走\twnr{針盒}{x310},如被迅猛風拋擲的鳥。比丘們!利得、恭敬、名聲是這麼恐畏……(中略)。比丘們!應該被你們這麼學!」



\sutta{10}{10}{有偈頌的經}{https://agama.buddhason.org/SN/sn.php?keyword=17.10}
  住在舍衛城……(中略)。

  「\twnr{比丘}{31.0}們!利得、恭敬、名聲是恐怖的……(中略)。比丘們!這裡,我看見某一類的個人被恭敬征服而心被遍取,他以身體的崩解,死後已往生\twnr{苦界}{109.0}、\twnr{惡趣}{110.0}、\twnr{下界}{111.0}、\twnr{地獄}{345.0},比丘們!又,這裡,我看見某一類的個人被不恭敬征服而心被遍取,他以身體的崩解,死後已往生苦界、惡趣、下界、地獄,比丘們!又,這裡,我看見某一類的個人被恭敬與不恭敬這兩者征服而心被遍取,他以身體的崩解,死後已往生苦界、惡趣、下界、地獄。比丘們!利得、恭敬、名聲是這麼恐畏……(中略)。比丘們!應該被你們這麼學!」

  \twnr{世尊}{12.0}說這個,說這個後,\twnr{善逝}{8.0}、\twnr{大師}{145.0}又更進一步說這個:

  「凡被恭敬,以不恭敬與兩者,

   住無量者的定,不動搖。

   那位有毅力的禪修者,微細見的毘婆舍那者,

   樂於執取的盡滅者,他們說像這樣是\twnr{善人}{76.0}。」[\ccchref{It.81}{https://agama.buddhason.org/It/dm.php?keyword=81}]

  第一品,其\twnr{攝頌}{35.0}:

  「恐畏、釣鉤、龜,長毛與糞蟲,

   雷電、塗有毒的、狐狼,迅猛風與有偈頌的。」





\pin{第二品}{11}{20}
\sutta{11}{11}{金鉢經}{https://agama.buddhason.org/SN/sn.php?keyword=17.11}
  住在舍衛城……(中略)。

  「\twnr{比丘}{31.0}們!利得、恭敬、名聲是恐怖的……(中略)。比丘們!這裡,我對某一類的個人這麼\twnr{以心熟知心後}{393.0}知道:『這位\twnr{尊者}{200.0}不會為了裝滿銀粉的金鉢之因而故意說謊。』過些時候,我看見他被利得、恭敬、名聲征服而心被遍取,他故意說謊。比丘們!利得、恭敬、名聲是這麼恐畏……(中略)。比丘們!應該被你們這麼學!」



\sutta{12}{12}{銀鉢經}{https://agama.buddhason.org/SN/sn.php?keyword=17.12}
  住在舍衛城……(中略)。

  「\twnr{比丘}{31.0}們!利得、恭敬、名聲是恐怖的……(中略)。比丘們!這裡,我對某一類的個人這麼\twnr{以心熟知心後}{393.0}知道:『這位\twnr{尊者}{200.0}不會為了裝滿金粉的銀鉢之因而故意說謊。』過些時候,我看見他被利得、恭敬、名聲征服而心被遍取,他故意說謊。比丘們!利得、恭敬、名聲是這麼恐畏……(中略)。比丘們!應該被你們這麼學!」



\sutta{13}{20}{金環經等八則}{https://agama.buddhason.org/SN/sn.php?keyword=17.13}
  住在舍衛城……(中略)。

  「\twnr{比丘}{31.0}們!這裡,我對某一類的個人這麼\twnr{以心熟知心後}{393.0}知道:『這位\twnr{尊者}{200.0}不會為了一個金環之因……(中略)百個金環之因……(中略)一個自然金環之因……(中略)百個自然金環之因……(中略)充滿黃金的土地之因……(中略)些微物質之因……(中略)活命之因……(中略)地方上的美女之因而故意說謊。』過些時候,我看見他被利得、恭敬、名聲征服而心被遍取,他故意說謊。比丘們!利得、恭敬、名聲是這麼恐畏……(中略)。比丘們!應該被你們這麼學!」

  第二品,其\twnr{攝頌}{35.0}:

  「二則鉢、二則金,自然金在後二則,

   土地、一些物質、活命,地方上的美女為十。」





\pin{第三品}{21}{30}
\sutta{21}{21}{婦女經}{https://agama.buddhason.org/SN/sn.php?keyword=17.21}
  住在舍衛城……(中略)。

  「\twnr{比丘}{31.0}們!利得、恭敬、名聲是恐怖的……(中略)。比丘們!一對一(獨處)的婦女不\twnr{持續遍取}{530.0}他的心而利得、恭敬、名聲持續遍取他的心。比丘們!利得、恭敬、名聲是這麼恐畏……(中略)。比丘們!應該被你們這麼學!」



\sutta{22}{22}{美女經}{https://agama.buddhason.org/SN/sn.php?keyword=17.22}
  住在舍衛城……(中略)。

  「\twnr{比丘}{31.0}們!利得、恭敬、名聲是恐怖的……(中略)。比丘們!一對一(獨處)的地方上的美女不\twnr{持續遍取}{530.0}他的心而利得、恭敬、名聲持續遍取他的心。比丘們!利得、恭敬、名聲是這麼恐畏……(中略)。比丘們!應該被你們這麼學!」



\sutta{23}{23}{獨子經}{https://agama.buddhason.org/SN/sn.php?keyword=17.23}
  住在舍衛城……(中略)。

  「\twnr{比丘}{31.0}們!利得、恭敬、名聲是恐怖的……(中略)。比丘們!有信的\twnr{優婆夷}{99.0}對所愛的、合意的獨子當正確地祈願時,會這麼祈願:『兒子!你要成為像無論\twnr{屋主}{103.0}質多與阿拉哇葛的如手那樣。』比丘們!對我的弟子\twnr{優婆塞}{98.0}們,這是秤;\twnr{這是衡量}{x311},即:屋主質多與阿拉哇葛的如手[\ccchref{AN.2.133}{https://agama.buddhason.org/AN/an.php?keyword=2.133}]。『兒子!如果你\twnr{從在家出家成為無家者}{48.0},兒子!你要成為像無論舍利弗、目揵連那樣。』比丘們!我的弟子比丘們中,這是秤;這是衡量,即:舍利弗、目揵連[\ccchref{AN.2.131}{https://agama.buddhason.org/AN/an.php?keyword=2.131}]。『兒子!但對你-心意未達成的\twnr{有學}{193.0},令利得、恭敬、名聲不要得到。』比丘們!如果心意未達成的那位有學比丘得到利得、恭敬、名聲,那是他的障礙。比丘們!利得、恭敬、名聲是這麼恐怖的……(中略)。比丘們!應該被你們這麼學!」



\sutta{24}{24}{獨女經}{https://agama.buddhason.org/SN/sn.php?keyword=17.24}
  住在舍衛城……(中略)。 

  「\twnr{比丘}{31.0}們!利得、恭敬、名聲是恐怖的……(中略)。比丘們!有信的\twnr{優婆夷}{99.0}對所愛的、合意的獨[生]女當正確地祈願時,會這麼祈願:『\twnr{賢女}{x312}!妳要成為像無論辜住桃樂優婆夷與在威魯梗達居[地方]難陀的母親[\ccchref{AN.7.53}{https://agama.buddhason.org/AN/an.php?keyword=7.53}]那樣。』比丘們!我的弟子優婆夷們中,這是秤;\twnr{這是衡量}{x313},即:辜住桃樂優婆夷與在威魯梗達居[地方]難陀的母親[\ccchref{AN.2.132}{https://agama.buddhason.org/AN/an.php?keyword=2.132}]。『賢女!如果妳\twnr{從在家出家成為無家者}{48.0},賢女!妳要成為像無論讖摩比丘尼與蓮華色那樣。』比丘們!對我的弟子比丘尼們,這是秤;這是衡量,即:讖摩比丘尼與蓮華色[\ccchref{AN.2.134}{https://agama.buddhason.org/AN/an.php?keyword=2.134}]。『賢女!但對妳-心意未達成的\twnr{有學}{193.0},令利得、恭敬、名聲不要得到。』比丘們!如果心意未達成的那位有學比丘尼得到利得、恭敬、名聲,那是她的障礙。比丘們!利得、恭敬、名聲是這麼恐怖的……(中略)。比丘們!應該被你們這麼學!」



\sutta{25}{25}{沙門婆羅門經}{https://agama.buddhason.org/SN/sn.php?keyword=17.25}
  住在舍衛城……(中略)。 

  「\twnr{比丘}{31.0}們!凡任何\twnr{沙門}{29.0}或\twnr{婆羅門}{17.0}不如實知道利得、恭敬、名聲的\twnr{樂味}{295.0}、\twnr{過患}{293.0}、\twnr{出離}{294.0}者,比丘們!那些沙門或婆羅門不被我認同為\twnr{沙門中的沙門}{560.0},或婆羅門中的婆羅門,而且,那些\twnr{尊者}{200.0}也不以證智自作證後,在當生中\twnr{進入後住於}{66.0}\twnr{沙門義}{327.0}或婆羅門義。

  比丘們!而凡任何沙門或婆羅門如實知道利得、恭敬、名聲的樂味、過患、出離者,比丘們!那些沙門或婆羅門被我認同為沙門中的沙門,或婆羅門中的婆羅門,而且,那些尊者也以證智自作證後,在當生中進入後住於沙門義或婆羅門義。」



\sutta{26}{26}{沙門婆羅門經第二}{https://agama.buddhason.org/SN/sn.php?keyword=17.26}
  住在舍衛城……(中略)。 

  「\twnr{比丘}{31.0}們!凡任何\twnr{沙門}{29.0}或\twnr{婆羅門}{17.0}不如實知道利得、恭敬、名聲的\twnr{集起}{67.0}、滅沒、\twnr{樂味}{295.0}、\twnr{過患}{293.0}、\twnr{出離}{294.0}者……(中略)了知……(中略)以證智自作證後,[\twnr{在當生中}{42.0}]\twnr{進入後住於}{66.0}\twnr{沙門義}{327.0}或婆羅門義。」



\sutta{27}{27}{沙門婆羅門經第三}{https://agama.buddhason.org/SN/sn.php?keyword=17.27}
  住在舍衛城……(中略)。 

  「\twnr{比丘}{31.0}們!凡任何\twnr{沙門}{29.0}或\twnr{婆羅門}{17.0}不如實知道利得、恭敬、名聲,不如實知道利得、恭敬、名聲的\twnr{集}{67.0},不如實知道利得、恭敬、名聲的\twnr{滅}{68.0},不如實知道導向利得、恭敬、名聲的\twnr{滅道跡}{69.0}者……(中略)了知……(中略)以證智自作證後,[\twnr{在當生中}{42.0}]\twnr{進入後住於}{66.0}\twnr{沙門義}{327.0}或婆羅門義。」



\sutta{28}{28}{表皮經}{https://agama.buddhason.org/SN/sn.php?keyword=17.28}
  住在舍衛城……(中略)。 

  「\twnr{比丘}{31.0}們!利得、恭敬、名聲是恐怖的。比丘們!利得、恭敬、名聲切斷表皮,切斷表皮後切斷皮膚,切斷皮膚後切斷肉,切斷肉後切斷腱,切斷腱後切斷骨,切斷骨後觸及骨髓而後停止。比丘們!利得、恭敬、名聲是這麼恐怖的……(中略)[辛辣的、粗惡的,是到達無上\twnr{軛安穩}{192.0}的障礙,阿難!因此,在這裡,應該被這麼學:『我們要捨斷已生起的利得、恭敬、名聲,而且,已生起的利得、恭敬、名聲不要持續遍取我們的心。』]。比丘們!應該被你們這麼學。」



\sutta{29}{29}{繩經}{https://agama.buddhason.org/SN/sn.php?keyword=17.29}
  住在舍衛城……(中略)。 

  「\twnr{比丘}{31.0}們!利得、恭敬、名聲是恐怖的。比丘們!利得、恭敬、名聲切斷表皮,切斷表皮後切斷皮膚,切斷皮膚後切斷肉,切斷肉後切斷腱,切斷腱後切斷骨,切斷骨後觸及骨髓而後停止。

  比丘們!猶如有力氣的男子以堅固的馬尾毛繩纏繞小腿後摩擦,它切斷表皮,切斷表皮後切斷皮膚,切斷皮膚後切斷肉,切斷肉後切斷腱,切斷腱後切斷骨,切斷骨後觸及骨髓而後停止。同樣的,比丘們!利得、恭敬、名聲切斷表皮,切斷表皮後切斷皮膚,切斷皮膚後切斷肉,切斷肉後切斷腱,切斷腱後切斷骨,切斷骨後觸及骨髓而後停止。比丘們!利得、恭敬、名聲是這麼恐怖的……(中略)。比丘們!應該被你們這麼學!」 



\sutta{30}{30}{比丘經}{https://agama.buddhason.org/SN/sn.php?keyword=17.30}
  住在舍衛城……(中略)。 

  「比丘們!凡即使他是漏已滅盡的\twnr{阿羅漢}{5.0}比丘,我說利得、恭敬、名聲也是他的障礙。」

  在這麼說時,\twnr{尊者}{200.0}阿難對\twnr{世尊}{12.0}說這個:

  「\twnr{大德}{45.0}!但為什麼漏已滅盡比丘的利得、恭敬、名聲是障礙?」

  「阿難!凡他的那個\twnr{不動心解脫}{808.0},我不說利得、恭敬、名聲是他的障礙,阿難!但凡他的住於不放逸的、熱心的、自我努力的到達的\twnr{在當生中的諸樂住處}{x314},對那些,我說利得、恭敬、名聲是他的障礙。阿難!利得、恭敬、名聲是這麼恐怖的、辛辣的、粗惡的,是到達無上\twnr{軛安穩}{192.0}的障礙,阿難!因此,在這裡,應該被這麼學:『我們要捨斷已生起的利得、恭敬、名聲,而且,已生起的利得、恭敬、名聲不要\twnr{持續遍取}{530.0}我們的心。』阿難!應該被你們這麼學。」

  第三品,其\twnr{攝頌}{35.0}:

  「婦女與美女,兒子與獨女,

   沙門婆羅門三則,表皮、繩與比丘。」





\pin{第四品}{31}{43}
\sutta{31}{31}{分裂經}{https://agama.buddhason.org/SN/sn.php?keyword=17.31}
  住在舍衛城……(中略)。 

  「\twnr{比丘}{31.0}們!利得、恭敬、名聲是恐怖的。比丘們!被利得、恭敬、名聲征服而心被遍取,提婆達多分裂\twnr{僧團}{375.0}。比丘們!利得、恭敬、名聲是這麼恐怖的……(中略)。比丘們!應該被你們這麼學!」 



\sutta{32}{32}{善根經}{https://agama.buddhason.org/SN/sn.php?keyword=17.32}
  住在舍衛城……(中略)。 

  「\twnr{比丘}{31.0}們!利得、恭敬、名聲是恐怖的。比丘們!被利得、恭敬、名聲征服而心被遍取,提婆達多的善根走到斷絕。比丘們!利得、恭敬、名聲是這麼恐怖的……(中略)。比丘們!應該被你們這麼學!」 



\sutta{33}{33}{善法經}{https://agama.buddhason.org/SN/sn.php?keyword=17.33}
  住在舍衛城……(中略)。 

  「\twnr{比丘}{31.0}們!利得、恭敬、名聲是恐怖的。比丘們!被利得、恭敬、名聲征服而心被遍取,提婆達多的善法走到斷絕。比丘們!利得、恭敬、名聲是這麼恐怖的……(中略)。比丘們!應該被你們這麼學!」 



\sutta{34}{34}{白法經}{https://agama.buddhason.org/SN/sn.php?keyword=17.34}
  住在舍衛城……(中略)。 

  「\twnr{比丘}{31.0}們!利得、恭敬、名聲是恐怖的。比丘們!被利得、恭敬、名聲征服而心被遍取,提婆達多的\twnr{白法}{x315}走到斷絕。比丘們!利得、恭敬、名聲是這麼恐怖的……(中略)。比丘們!應該被你們這麼學!」 



\sutta{35}{35}{離開不久經}{https://agama.buddhason.org/SN/sn.php?keyword=17.35}
  \twnr{有一次}{2.0},在提婆達多離開不久,\twnr{世尊}{12.0}住在王舍城\twnr{耆闍崛山}{258.0}。

  在那裡,關於提婆達多之事,世尊召喚\twnr{比丘}{31.0}們:

  「比丘們!提婆達多的利得、恭敬、名聲生起對自己的殺害;提婆達多的利得、恭敬、名聲生起敗亡。

  比丘們!猶如芭蕉結果實對自己的殺害;結果實敗亡。同樣的,比丘們!提婆達多的利得、恭敬、名聲生起對自己的殺害;提婆達多的利得、恭敬、名聲生起敗亡。

  比丘們!猶如竹子結果實對自己的殺害;結果實敗亡。同樣的,比丘們!提婆達多的利得、恭敬、名聲生起對自己的殺害;提婆達多的利得、恭敬、名聲生起敗亡。

  比丘們!猶如蘆葦結果實對自己的殺害;結果實敗亡。同樣的,比丘們!提婆達多的利得、恭敬、名聲生起對自己的殺害;提婆達多的利得、恭敬、名聲生起敗亡。

  比丘們!猶如騾子懷胎對自己的殺害;懷胎而敗亡。同樣的,比丘們!提婆達多的利得、恭敬、名聲生起對自己的殺害;提婆達多的利得、恭敬、名聲生起敗亡。

  比丘們!利得、恭敬、名聲是這麼恐怖的[……(中略)、辛辣的、粗惡的,是到達無上\twnr{軛安穩}{192.0}的障礙,比丘們!因此,在這裡,應該被這麼學:『我們要捨斷已生起的利得、恭敬、名聲,而且,已生起的利得、恭敬、名聲不要\twnr{持續遍取}{530.0}我們的心。』]比丘們!應該被你們這麼學。」

  世尊說這個,說這個後,\twnr{善逝}{8.0}、\twnr{大師}{145.0}又更進一步說這個:

  「果實確實殺害芭蕉,果實對竹子、果實對蘆葦,

   恭敬殺害邪惡人,\twnr{如胎對騾}{x316}。」[\suttaref{SN.6.12}, \ccchref{AN.4.68}{https://agama.buddhason.org/AN/an.php?keyword=4.68}]



\sutta{36}{36}{五百車經}{https://agama.buddhason.org/SN/sn.php?keyword=17.36}
  住在王舍城栗鼠飼養處的竹林中。

  當時,阿闍世王子傍晚、早上以五百輛車去伺候提婆達多,以及五百鍋煮熟食物的提供被帶來。

  那時,眾多\twnr{比丘}{31.0}去見世尊。抵達後,向世尊\twnr{問訊}{46.0}後,在一旁坐下。在一旁坐下的那些比丘對世尊說這個:

  「\twnr{大德}{45.0}!阿闍世王子傍晚、早上以五百輛車去伺候提婆達多,以及五百鍋煮熟食物的提供被帶來。」

  「比丘們!你們不要羨慕提婆達多的利得、恭敬、名聲,比丘們!只要阿闍世王子傍晚、早上以五百輛車將去伺候提婆達多,以及五百鍋煮熟食物的提供將被帶來,比丘們!對提婆達多來說,在善法上的減損就能被預期,非增長。

  比丘們!猶如膽汁\twnr{灑}{x317}在兇惡狗的鼻子上,比丘們!那隻狗就成為更兇惡的。同樣的,比丘們!只要阿闍世王子傍晚、早上以五百輛車去伺候提婆達多,以及五百鍋煮熟食物的提供被帶來,比丘們!提對提婆達多來說,在善法上的減損就能被預期,非增長。

  比丘們!利得、恭敬、名聲是這麼恐怖的……(中略)[辛辣的、粗惡的,是到達無上\twnr{軛安穩}{192.0}的障礙,比丘們!因此,在這裡,應該被這麼學:『我們要捨斷已生起的利得、恭敬、名聲,而且,已生起的利得、恭敬、名聲不要\twnr{持續遍取}{530.0}我們的心。』]比丘們!應該被你們這麼學。」



\sutta{37}{37}{母親經}{https://agama.buddhason.org/SN/sn.php?keyword=17.37}
  住在舍衛城……(中略)。

  「\twnr{比丘}{31.0}們!利得、恭敬、名聲是恐怖的、辛辣的、粗惡的,是到達無上\twnr{軛安穩}{192.0}的障礙。比丘們!這裡,我對某一類的個人這麼\twnr{以心熟知心後}{393.0}知道:『這位\twnr{尊者}{200.0}不會為了母親之因而故意說謊。』過些時候,我看見他被利得、恭敬、名聲征服而心被遍取,他故意說謊。比丘們!利得、恭敬、名聲是這麼恐怖的、辛辣的、粗惡的,是到達無上軛安穩的障礙,比丘們!因此,在這裡,應該被這麼學:『我們要捨斷已生起的利得、恭敬、名聲,而且,已生起的利得、恭敬、名聲不要\twnr{持續遍取}{530.0}我們的心。』比丘們!應該被你們這麼學。」 



\sutta{38}{43}{父親經等六則}{https://agama.buddhason.org/SN/sn.php?keyword=17.38}
  住在舍衛城……(中略)。

  「\twnr{比丘}{31.0}們!利得、恭敬、名聲是恐怖的、辛辣的、粗惡的,是到達無上\twnr{軛安穩}{192.0}的障礙。比丘們!這裡,我對某一類的個人這麼\twnr{以心熟知心後}{393.0}知道:『這位\twnr{尊者}{200.0}不會為了父親之因……兄弟之因……姊妹之因……兒子之因……女兒之因……妻子之因而故意說謊。』過些時候,我看見他被利得、恭敬、名聲征服而心被遍取,他故意說謊。比丘們!利得、恭敬、名聲是這麼恐怖的、辛辣的、粗惡的,是到達無上軛安穩的障礙,比丘們!因此,在這裡,應該被這麼學:『我們要捨斷已生起的利得、恭敬、名聲,而且,已生起的利得、恭敬、名聲不要\twnr{持續遍取}{530.0}我們的心。』比丘們!應該被你們這麼學。」 

  第四品,其\twnr{攝頌}{35.0}:

  「分裂、根、二則法,離去、車、母親,

   父親、兄弟與姊妹,兒子、女兒、妻子。」

  利得恭敬相應完成





\page

\xiangying{18}{羅侯羅相應}
\pin{第一品}{1}{10}
\sutta{1}{1}{眼經}{https://agama.buddhason.org/SN/sn.php?keyword=18.1}
  \twnr{被我這麼聽聞}{1.0}:

  \twnr{有一次}{2.0},\twnr{世尊}{12.0}住在舍衛城祇樹林給孤獨園。

  那時,\twnr{尊者}{200.0}羅侯羅去見世尊。抵達後,向世尊\twnr{問訊}{46.0}後,在一旁坐下。在一旁坐下的尊者羅侯羅對世尊說這個:

  「\twnr{大德}{45.0}!請世尊為我簡要地教導法,凡我聽聞世尊的法後,會住於單獨的、隱離的、不放逸的、熱心的、自我努力的,\twnr{那就好了}{44.0}!」

  「羅侯羅!你怎麼想它:眼是常的,或是無常的?」

  「無常的,大德!」

  「那麼,凡為無常的,那是苦的或樂的?」

  「苦的,大德!」

  「那麼,凡為無常的、苦的、\twnr{變易法}{70.0},適合認為它:『\twnr{這是我的}{32.0},\twnr{我是這個}{33.0},\twnr{這是我的真我}{34.1}。』嗎?」

  「大德!這確實不是。」

  「耳是常的,或是無常的?」

  「無常的,大德!」……(中略)。

  「鼻是常的,或是無常的?」

  「無常的,大德!」……(中略)。

  「舌是常的,或是無常的?」

  「無常的,大德!」……(中略)。

  「身是常的,或是無常的?」

  「無常的,大德!」……(中略)。

  「意是常的,或是無常的?」

  「無常的,大德!」

  「那麼,凡為無常的,那是苦的或樂的?」

  「苦的,大德!」

  「那麼,凡為無常的、苦的、變易法,適合認為它:『\twnr{這是我的}{32.0},\twnr{我是這個}{33.0},這是\twnr{我的真我}{34.0}。』嗎?」

  「大德!這確實不是。」

  「羅侯羅!這麼看的\twnr{有聽聞的聖弟子}{24.0}在眼上\twnr{厭}{15.0}……(中略)在耳上厭……在鼻上厭……在舌上厭……在身上厭……在意上厭。厭者\twnr{離染}{558.0},從\twnr{離貪}{77.0}被解脫,在已解脫時,\twnr{有『[這是]解脫』之智}{27.0},他知道:『\twnr{出生已盡}{18.0},\twnr{梵行已完成}{19.0},\twnr{應該被作的已作}{20.0},\twnr{不再有此處[輪迴]的狀態}{21.1}。』」



\sutta{2}{2}{色經}{https://agama.buddhason.org/SN/sn.php?keyword=18.2}
  住在舍衛城……(中略)。

  「羅侯羅!你怎麼想它:色是常的,或是無常的?」

  「無常的,\twnr{大德}{45.0}!」……(中略)。

  「諸聲音……諸氣味……諸味道……諸\twnr{所觸}{220.2}……法是常的,或是無常的?」

  「無常的,大德!」……(中略)。

  「羅侯羅!這麼看的\twnr{有聽聞的聖弟子}{24.0}在色上\twnr{厭}{15.0}……(中略)在聲音上厭……在氣味上厭……在味道上厭……在所觸上厭……在法上厭。厭者\twnr{離染}{558.0}……(中略)他知道:……。」



\sutta{3}{3}{識經}{https://agama.buddhason.org/SN/sn.php?keyword=18.3}
  住在舍衛城……(中略)。

  「羅侯羅!你怎麼想它:眼識是常的,或是無常的?」

  「無常的,\twnr{大德}{45.0}!」……(中略)。

  「耳識……(中略)鼻識……舌識……身識……意識是常的,或是無常的?」

  「無常的,大德!」……(中略)。

  「羅侯羅!這麼看的\twnr{有聽聞的聖弟子}{24.0}在眼識上\twnr{厭}{15.0}……(中略)在耳識上厭……在鼻識上厭……在舌識上厭……在身識上厭……在意識上厭。厭者\twnr{離染}{558.0}……(中略)他知道:……。」



\sutta{4}{4}{觸經}{https://agama.buddhason.org/SN/sn.php?keyword=18.4}
  住在舍衛城……(中略)。

  「羅侯羅!你怎麼想它:眼觸是常的,或是無常的?」

  「無常的,\twnr{大德}{45.0}!」……(中略)。

  「耳觸……(中略)鼻觸……舌觸……身觸……意觸是常的,或是無常的?」

  「無常的,大德!」……(中略)。

  「羅侯羅!這麼看的\twnr{有聽聞的聖弟子}{24.0}在眼觸上\twnr{厭}{15.0}……(中略)在耳觸上厭……在鼻觸上厭……在舌觸上厭……在身觸上厭……在意觸上厭。厭者\twnr{離染}{558.0}……(中略)他知道:……。」



\sutta{5}{5}{受經}{https://agama.buddhason.org/SN/sn.php?keyword=18.5}
  住在舍衛城……(中略)。

  「羅侯羅!你怎麼想它:眼觸所生受是常的,或是無常的?」

  「無常的,\twnr{大德}{45.0}!」……(中略)。

  「耳觸所生受……(中略)鼻觸所生受……舌觸所生受……身觸所生受……意觸所生受是常的,或是無常的?」

  「無常的,大德!」……(中略)。

  「羅侯羅!這麼看的\twnr{有聽聞的聖弟子}{24.0}在眼觸所生受上\twnr{厭}{15.0}……(中略)耳……鼻……舌……身……在意觸所生受上厭……(中略)他知道:……(中略)。」



\sutta{6}{6}{想經}{https://agama.buddhason.org/SN/sn.php?keyword=18.6}
  住在舍衛城……(中略)。

  「羅侯羅!你怎麼想它:色想是常的,或是無常的?」

  「無常的,\twnr{大德}{45.0}!」……(中略)。

  「聲音想……(中略)氣味想……味道想……\twnr{所觸}{220.2}想……法想是常的,或是無常的?」

  「無常的,大德!」……(中略)。

  「羅侯羅!這麼看的\twnr{有聽聞的聖弟子}{24.0}在色想上\twnr{厭}{15.0}……(中略)在聲音想上厭……在氣味想上厭……在味道想上厭……在所觸上想厭……在法想上厭……(中略)他知道:……(中略)。」



\sutta{7}{7}{思經}{https://agama.buddhason.org/SN/sn.php?keyword=18.7}
  住在舍衛城……(中略)。

  「羅侯羅!你怎麼想它:色思是常的,或是無常的?」

  「無常的,\twnr{大德}{45.0}!」……(中略)。

  「聲音思……(中略)氣味思……味道思……\twnr{所觸}{220.2}思……法思是常的,或是無常的?」

  「無常的,大德!」……(中略)。

  「羅侯羅!這麼看的\twnr{有聽聞的聖弟子}{24.0}在色思上\twnr{厭}{15.0}……(中略)在聲音思上厭……在氣味思上厭……在味道思上厭……在所觸思上厭……在法思上厭……(中略)他知道:……(中略)。」



\sutta{8}{8}{渴愛經}{https://agama.buddhason.org/SN/sn.php?keyword=18.8}
  住在舍衛城……(中略)。

  「羅侯羅!你怎麼想它:色的渴愛是常的,或是無常的?」

  「無常的,\twnr{大德}{45.0}!」……(中略)。

  「聲音的渴愛……(中略)氣味的渴愛……味道的渴愛……\twnr{所觸}{220.2}的渴愛……法的渴愛是常的,或是無常的?」

  「無常的,大德!」……(中略)。

  「羅侯羅!這麼看的\twnr{有聽聞的聖弟子}{24.0}在色的渴愛上\twnr{厭}{15.0}……(中略)在聲音的渴愛上厭……在氣味的渴愛上厭……在味道的渴愛上厭……在所觸的渴愛上厭……在法的渴愛上厭……(中略)他知道:……(中略)。」



\sutta{9}{9}{界經}{https://agama.buddhason.org/SN/sn.php?keyword=18.9}
  住在舍衛城……(中略)。

  「羅侯羅!你怎麼想它:地界是常的,或是無常的?」

  「無常的,\twnr{大德}{45.0}!」……(中略)。

  「水界……(中略)火界……風界……空界……識界是常的,或是無常的?」

  「無常的,大德!」……(中略)。

  「羅侯羅!這麼看的\twnr{有聽聞的聖弟子}{24.0}在地界上\twnr{厭}{15.0}……(中略)在水界上厭……在火界上厭……在風界上厭……在空界上厭……在識界上厭……(中略)他知道:……(中略)。」



\sutta{10}{10}{蘊經}{https://agama.buddhason.org/SN/sn.php?keyword=18.10}
  住在舍衛城……(中略)。

  「羅侯羅!你怎麼想它:色是常的,或是無常的?」

  「無常的,\twnr{大德}{45.0}!」……(中略)。

  「受……(中略)想……諸行……識是常的,或是無常的?」

  「無常的,大德!」……(中略)。

  「羅侯羅!這麼看的\twnr{有聽聞的聖弟子}{24.0}在色上\twnr{厭}{15.0}……(中略)在受上厭……在想上厭……在諸行上厭……在識上厭。厭者\twnr{離染}{558.0},從\twnr{離貪}{77.0}被解脫,在已解脫時,\twnr{有『[這是]解脫』之智}{27.0},他知道:『\twnr{出生已盡}{18.0},\twnr{梵行已完成}{19.0},\twnr{應該被作的已作}{20.0},\twnr{不再有此處[輪迴]的狀態}{21.1}。』」

  第一品,其\twnr{攝頌}{35.0}:

  「眼、色、識,以及觸、受,

   想、思、渴愛,界與蘊它們為十。」





\pin{第二品}{11}{22}
\sutta{11}{11}{眼經}{https://agama.buddhason.org/SN/sn.php?keyword=18.11}
  \twnr{被我這麼聽聞}{1.0}:

  \twnr{有一次}{2.0},\twnr{世尊}{12.0}住在舍衛城。

  那時,\twnr{尊者}{200.0}羅侯羅去見世尊。抵達後,向世尊\twnr{問訊}{46.0}後,在一旁坐下。世尊對在一旁坐下的尊者羅侯羅說這個:

  「羅侯羅!你怎麼想它:眼是常的,或是無常的?」

  「無常的,\twnr{大德}{45.0}!」

  「那麼,凡為無常的,那是苦的或樂的?」

  「苦的,大德!」

  「那麼,凡為無常的、苦的、\twnr{變易法}{70.0},適合認為它:『\twnr{這是我的}{32.0},\twnr{我是這個}{33.0},\twnr{這是我的真我}{34.1}。』嗎?」

  「大德!這確實不是。」

  「耳……(中略)鼻……舌……身……意是常的,或是無常的?」

  「無常的,大德!」

  「那麼,凡為無常的,那是苦的或樂的?」

  「苦的,大德!」

  「那麼,凡為無常的、苦的、變易法,適合認為它:『\twnr{這是我的}{32.0},\twnr{我是這個}{33.0},這是\twnr{我的真我}{34.0}。』嗎?」

  「大德!這確實不是。」

  「羅侯羅!這麼看的\twnr{有聽聞的聖弟子}{24.0}在眼上\twnr{厭}{15.0}……(中略)在耳上厭……在鼻上厭……在舌上厭……在身上厭……在意上厭。厭者\twnr{離染}{558.0},從\twnr{離貪}{77.0}被解脫,在已解脫時,\twnr{有『[這是]解脫』之智}{27.0},他知道:『\twnr{出生已盡}{18.0},\twnr{梵行已完成}{19.0},\twnr{應該被作的已作}{20.0},\twnr{不再有此處[輪迴]的狀態}{21.1}。』」

  以這個模式[以下]十經應該被做。



\sutta{12}{20}{色經等九則}{https://agama.buddhason.org/SN/sn.php?keyword=18.12}
  住在舍衛城……(中略)。

  「羅侯羅!你怎麼想它:色是常的,或是無常的?」

  「無常的,\twnr{大德}{45.0}!」……(中略)。

  「諸聲音……諸氣味……諸味道……諸\twnr{所觸}{220.2}……法……

  眼識……(中略)耳識……鼻識……舌識……身識……意識……

  眼觸……(中略)耳觸……鼻觸……舌觸……身觸……意觸……

  眼觸所生受……(中略)耳觸所生受……鼻觸所生受……舌觸所生受……身觸所生受……意觸所生受……

  色想……(中略)聲音想……氣味想……味道想……所觸想……法想……

  色思……(中略)聲音思……氣味思……味道思……所觸思……法思……

  色的渴愛……(中略)聲音的渴愛……氣味的渴愛……味道的渴愛……所觸的渴愛……法的渴愛……

  地界……(中略)水界……火界……風界……空界……識界……

  色……(中略)受……想……諸行……識是常的,或是無常的?」

  「無常的,大德!」……(中略)。

  「羅侯羅!這麼看的……(中略)他知道:『……\twnr{不再有此處[輪迴]的狀態}{21.1}。』」



\sutta{21}{21}{煩惱潛在趨勢經}{https://agama.buddhason.org/SN/sn.php?keyword=18.21}
  住在舍衛城。

  那時,\twnr{尊者}{200.0}羅侯羅去見世尊。抵達後,向世尊\twnr{問訊}{46.0}後,在一旁坐下。在一旁坐下的尊者羅侯羅對世尊說這個:

  「\twnr{大德}{45.0}!當怎樣知、當怎樣見時,在這個有識之身上\twnr{與在一切外部諸相上}{559.0}沒有\twnr{我作}{22.0}、\twnr{我所作}{25.0}、\twnr{慢煩惱潛在趨勢}{26.0}?」

  「羅侯羅!凡任何色:過去、未來、現在,或內、或外,或粗、或細,或下劣、或勝妙,或凡在遠處、在近處,所有色:『\twnr{這不是我的}{32.1},\twnr{我不是這個}{33.1},\twnr{這不是我的真我}{34.2}。』以正確之慧這樣如實看見這個。

  凡任何受……(中略)凡任何想……凡任何諸行……凡任何識:過去、未來、現在,或內、或外,或粗、或細,或下劣、或勝妙,或凡在遠處、在近處,所有識:『這不是我的,我不是這個,這不是我的真我。』以正確之慧這樣如實看見這個。

  羅侯羅!當這樣知、這樣見時,在這個有識之身上與在一切外部諸相上沒有我作、我所作、慢煩惱潛在趨勢。」[\suttaref{SN.22.71}, \suttaref{SN.22.91}, ≃\suttaref{SN.22.72}, \suttaref{SN.22.124}]



\sutta{22}{22}{離經}{https://agama.buddhason.org/SN/sn.php?keyword=18.22}
  起源於舍衛城。

  那時,\twnr{尊者}{200.0}羅侯羅去見世尊。抵達後,向世尊\twnr{問訊}{46.0}後,在一旁坐下。在一旁坐下的尊者羅侯羅對世尊說這個:

  「\twnr{大德}{45.0}!當怎樣知、當怎樣見時,在這個有識之身上\twnr{與在一切外部諸相上}{559.0}離\twnr{我作}{22.0}、\twnr{我所作}{25.0}、慢,成為心意超越\twnr{慢類}{647.0}者、寂靜者、\twnr{善解脫}{28.0}者呢?」

  「羅侯羅!凡任何色:過去、未來、現在,或內、或外,或粗、或細,或下劣、或勝妙,或凡在遠處、在近處,所有色:『\twnr{這不是我的}{32.1},\twnr{我不是這個}{33.1},\twnr{這不是我的真我}{34.2}。』以正確之慧這樣如實看見這個後,不執取後成為解脫者。

  凡任何受……(中略)凡任何想……凡任何諸行……凡任何識:過去、未來、現在,或內、或外,或粗、或細,或下劣、或勝妙,或凡在遠處、在近處,所有識:『這不是我的,我不是這個,這不是我的真我。』以正確之慧這樣如實看見這個後,不執取後成為解脫者。

  羅侯羅!當這樣知、這樣見時,在這個有識之身上與在一切外部諸相上離我作、我所作、慢,成為心意超越慢類者、寂靜者、善解脫者。」[\suttaref{SN.22.92}, ≃\suttaref{SN.22.125}]

  第二品,其\twnr{攝頌}{35.0}:

  「眼、色、識,以及觸、受,

   想、思、渴愛,界與蘊它們為十則,

   離煩惱潛在趨勢與離,以那個被稱為品。」

  羅侯羅相應完成。





\page

\xiangying{19}{勒叉那相應}
\pin{第一品}{1}{10}
\sutta{1}{1}{骨骸經}{https://agama.buddhason.org/SN/sn.php?keyword=19.1}
  \twnr{被我這麼聽聞}{1.0}:

  \twnr{有一次}{2.0},\twnr{世尊}{12.0}住在王舍城栗鼠飼養處的竹林中。

  當時,\twnr{尊者}{200.0}勒叉那與尊者大目揵連住在\twnr{耆闍崛山}{258.0}。

  那時,尊者大目揵連午前時穿衣、拿起衣鉢後,去見尊者勒叉那。抵達後,對尊者勒叉那說這個:

  「勒叉那\twnr{學友}{201.0}!我們走,讓我們\twnr{為了托鉢}{87.0}進入王舍城吧。」

  「是的,學友!」尊者勒叉那回答尊者大目揵連。

  那時,當尊者大目揵連從耆闍崛山下來時,在某個地方顯露微笑。

  那時,尊者勒叉那對尊者大目揵連說這個:

  「目揵連學友!什麼因、什麼\twnr{緣}{180.0},有微笑的顯示呢?」

  「勒叉那學友!對這個問題來說是不適時的,請你在世尊的面前問這個問題。」

  那時,尊者勒叉那與尊者大目揵連在王舍城為了托鉢行走後,\twnr{餐後已從施食返回}{512.0},去見世尊。抵達後,向世尊\twnr{問訊}{46.0}後,在一旁坐下。在一旁坐下的尊者勒叉那對尊者大目揵連說這個:

  「這裡,當尊者大目揵連從耆闍崛山下來時,在某個地方顯露微笑,目揵連學友!什麼因、什麼緣,有微笑的顯示呢?」

  「學友!這裡,當我從耆闍崛山下來時,看見空中行走的骨架,鷲、烏鴉、老鷹一一降臨他後,從肋骨的空隙刺(啄)、拉裂、使之脫離,他發出(作)痛苦的聲音。

  學友!那個我想這個:『實在\twnr{不可思議}{206.0}啊,\twnr{先生}{202.0}!實在\twnr{未曾有}{206.0}啊,先生!竟然也會有這樣形色的眾生,竟然也會有這樣形色的\twnr{夜叉}{126.0},竟然也會有這樣形色的\twnr{自體的獲得}{661.0}。』」

  那時,世尊召喚\twnr{比丘}{31.0}們:

  「比丘們!弟子們確實住於為\twnr{眼已生者}{455.0},比丘們!弟子們確實住於為\twnr{智已生者}{456.0},確實是因為弟子會知道,或會看到,或會見證這樣形色。

  比丘們!那位眾生以前就被我看見,但我沒解說。如果我解說那個,他人對我不相信,凡對我不相信者,那對他們有長久的不利、苦。

  比丘們!這位眾生就在這王舍城曾是屠牛夫,他以那個業的果報在地獄被折磨幾年、好幾百年、好幾千年、好幾十萬年後,\twnr{就以那個業的殘餘果報}{x318}感受這樣形色自體的獲得。」([本相應之]一切經之中略如這個[經])



\sutta{2}{2}{肉片經}{https://agama.buddhason.org/SN/sn.php?keyword=19.2}
  「\twnr{學友}{201.0}!這裡,當我從\twnr{耆闍崛山}{258.0}下來時,看見空中行走的肉片,鷲、烏鴉、老鷹一一降臨他後,拉裂、使之脫離,他發出(作)痛苦的聲音。……(中略)\twnr{比丘}{31.0}們!這位眾生就在這王舍城曾是屠牛夫……(中略)。」



\sutta{3}{3}{團經}{https://agama.buddhason.org/SN/sn.php?keyword=19.3}
  「\twnr{學友}{201.0}!這裡,當我從\twnr{耆闍崛山}{258.0}下來時,看見空中行走的肉團,鷲、烏鴉、老鷹一一降臨他後,拉裂、使之脫離,他發出(作)痛苦的聲音。……(中略)\twnr{比丘}{31.0}們!這位眾生就在這王舍城曾是捕鳥者……(中略)。」



\sutta{4}{4}{無皮膚經}{https://agama.buddhason.org/SN/sn.php?keyword=19.4}
  「\twnr{學友}{201.0}!這裡,當我從\twnr{耆闍崛山}{258.0}下來時,看見空中行走的無皮膚男子,鷲、烏鴉、老鷹一一降臨他後,拉裂、使之脫離,他發出(作)痛苦的聲音。……(中略)\twnr{比丘}{31.0}們!這位眾生就在這王舍城曾是屠羊夫……(中略)。」



\sutta{5}{5}{劍毛經}{https://agama.buddhason.org/SN/sn.php?keyword=19.5}
  「\twnr{學友}{201.0}!這裡,當我從\twnr{耆闍崛山}{258.0}下來時,看見空中行走的\twnr{劍毛}{x319}男子,他的那些劍一一飛起後就落在他的身上,他發出(作)痛苦的聲音。……(中略)\twnr{比丘}{31.0}們!這位眾生就在這王舍城曾是殺豬人……(中略)。」



\sutta{6}{6}{矛經}{https://agama.buddhason.org/SN/sn.php?keyword=19.6}
  「\twnr{學友}{201.0}!這裡,當我從\twnr{耆闍崛山}{258.0}下來時,看見空中行走的\twnr{矛毛}{x320}男子,他的那些矛一一飛起後就落在他的身上,他發出(作)痛苦的聲音。……(中略)\twnr{比丘}{31.0}們!這位眾生就在這王舍城曾是獵鹿人……(中略)。」



\sutta{7}{7}{箭毛經}{https://agama.buddhason.org/SN/sn.php?keyword=19.7}
  「\twnr{學友}{201.0}!這裡,當我從\twnr{耆闍崛山}{258.0}下來時,看見空中行走的\twnr{箭毛}{x321}男子,他的那些箭一一飛起後就落在他的身上,他發出(作)痛苦的聲音。……(中略)\twnr{比丘}{31.0}們!這位眾生就在這王舍城曾是拷問者……(中略)。」



\sutta{8}{8}{針毛經}{https://agama.buddhason.org/SN/sn.php?keyword=19.8}
  「\twnr{學友}{201.0}!這裡,當我從\twnr{耆闍崛山}{258.0}下來時,看見空中行走的針毛男子[\suttaref{SN.19.7}],他的那些針一一飛起後就落在他的身上,他發出(作)痛苦的聲音。……(中略)\twnr{比丘}{31.0}們!這位眾生就在這王舍城曾是\twnr{駕御者}{x322}……(中略)。」



\sutta{9}{9}{針毛經第二}{https://agama.buddhason.org/SN/sn.php?keyword=19.9}
  「\twnr{學友}{201.0}!這裡,當我從\twnr{耆闍崛山}{258.0}下來時,看見空中行走的針毛男子,他的那些針進入頭後從嘴巴出去;進入嘴巴後從胸部出去;進入胸部後從腹部出去;進入腹部後從大腿出去;進入大腿後從小腿出去;進入小腿後從足部出去,他發出(作)痛苦的聲音。……(中略)\twnr{比丘}{31.0}們!這位眾生就在這王舍城曾是中傷者……(中略)。」



\sutta{10}{10}{甕形睾丸經}{https://agama.buddhason.org/SN/sn.php?keyword=19.10}
  「\twnr{學友}{201.0}!這裡,當我從\twnr{耆闍崛山}{258.0}下來時,看見空中行走的甕形睾丸男子,當行走時,他就將那些睾丸放在肩上後行走,當坐下時,就坐在那些睾丸上,鷲、烏鴉、老鷹一一降臨他後,拉裂、使之脫離,他發出(作)痛苦的聲音。……(中略)\twnr{比丘}{31.0}們!這位眾生就在這王舍城曾是\twnr{收賄裁判官}{x323}……(中略)。」

  第一品,其\twnr{攝頌}{35.0}:

  「骨骸肉片兩者都是屠牛夫,肉塊是捕鳥者、無皮膚的是屠羊夫, 

  劍[毛]是殺豬人 、矛[毛]是獵鹿人,箭[毛]是拷問者、針[毛]是御車手,

  而凡被縫者那是中傷者,負擔睾丸者是收賄裁判官。」





\pin{第二品}{11}{21}
\sutta{11}{11}{頭全部經}{https://agama.buddhason.org/SN/sn.php?keyword=19.11}
  \twnr{被我這麼聽聞}{1.0}:

  \twnr{有一次}{2.0},在王舍城的竹林中。

  「\twnr{學友}{201.0}!這裡,當我從\twnr{耆闍崛山}{258.0}下來時,看見頭全部潛入糞坑的男子。……(中略)\twnr{比丘}{31.0}們!這位眾生就在這王舍城曾是姦夫……(中略)。」



\sutta{12}{12}{食糞者經}{https://agama.buddhason.org/SN/sn.php?keyword=19.12}
  「\twnr{學友}{201.0}!這裡,當我從\twnr{耆闍崛山}{258.0}下來時,看見潛入糞坑、以兩手食糞的男子……(中略)\twnr{比丘}{31.0}們!這位眾生就在這王舍城曾是惡心婆羅門,他在迦葉遍正覺者教說時代以食事邀請比丘\twnr{僧團}{375.0}後,木桶盛滿糞後,說這個:『啊!請\twnr{尊師}{203.0}們盡情地吃,連同帶走。』……(中略)。」



\sutta{13}{13}{無皮膚的女子經}{https://agama.buddhason.org/SN/sn.php?keyword=19.13}
  「\twnr{學友}{201.0}!這裡,當我從\twnr{耆闍崛山}{258.0}下來時,看見空中行走的無皮膚的女子,鷲、烏鴉、老鷹一一降臨她後,拉裂、使之脫離,她發出(作)痛苦的聲音。……(中略)\twnr{比丘}{31.0}們!這位女子就在這王舍城曾是姦婦……(中略)。」



\sutta{14}{14}{臉色青白的女子經}{https://agama.buddhason.org/SN/sn.php?keyword=19.14}
  「\twnr{學友}{201.0}!這裡,當我從\twnr{耆闍崛山}{258.0}下來時,看見臉色青白、惡臭的、空中行走的女子,鷲、烏鴉、老鷹一一降臨她後,拉裂、使之脫離,她發出(作)痛苦的聲音。……(中略)\twnr{比丘}{31.0}們!這位女子就在這王舍城曾是女占相者……(中略)。」



\sutta{15}{15}{中暑的經}{https://agama.buddhason.org/SN/sn.php?keyword=19.15}
  「\twnr{學友}{201.0}!這裡,當我從\twnr{耆闍崛山}{258.0}下來時,看見被燒焦的、炎熱的、\twnr{被燻黑}{x324}的、空中行走的女子,她發出(作)痛苦的聲音。……(中略)\twnr{比丘}{31.0}們!這位女子曾是迦利歌王的皇后,嫉妒個性的她以一鍋炭火澆灑另一位妃子……(中略)。」



\sutta{16}{16}{無頭的經}{https://agama.buddhason.org/SN/sn.php?keyword=19.16}
  「\twnr{學友}{201.0}!這裡,當我從\twnr{耆闍崛山}{258.0}下來時,看見空中行走的無頭的身體,他的胸部有眼睛與嘴巴,鷲、烏鴉、老鷹一一降臨他後,拉裂、使之脫離,他發出(作)痛苦的聲音。……(中略)\twnr{比丘}{31.0}們!這位眾生就在這王舍城曾是名叫哈利迦的劊子手……(中略)。」



\sutta{17}{17}{惡比丘經}{https://agama.buddhason.org/SN/sn.php?keyword=19.17}
  「\twnr{學友}{201.0}!這裡,當我從\twnr{耆闍崛山}{258.0}下來時,看見空中行走的\twnr{比丘}{31.0},他的大衣是燃燒的、灼熱的、熾熱的;鉢也是燃燒的、灼熱的、熾熱的;腰帶也是燃燒的、灼熱的、熾熱的;身體也是燃燒的、灼熱的、熾熱的,他發出(作)痛苦的聲音。……(中略)比丘們!那位比丘在迦葉遍正覺者教說時代是惡比丘。……(中略)。」



\sutta{18}{18}{惡比丘尼經}{https://agama.buddhason.org/SN/sn.php?keyword=19.18}
  「看見空中行走的\twnr{比丘尼}{31.0},她的大衣是燃燒的……(中略)是惡比丘尼。……(中略)。」



\sutta{19}{19}{惡式叉摩那經}{https://agama.buddhason.org/SN/sn.php?keyword=19.19}
  「看見空中行走的\twnr{式叉摩那}{630.0},她的大衣是燃燒的……(中略)是惡式叉摩那。……(中略)。」



\sutta{20}{20}{惡沙彌經}{https://agama.buddhason.org/SN/sn.php?keyword=19.20}
  「看見空中行走的\twnr{沙彌}{290.0},他的大衣是燃燒的……(中略)是惡沙彌。……(中略)。」



\sutta{21}{21}{惡沙彌尼經}{https://agama.buddhason.org/SN/sn.php?keyword=19.21}
  「\twnr{學友}{201.0}!這裡,當我從\twnr{耆闍崛山}{258.0}下來時,看見空中行走的\twnr{沙彌尼}{290.1},她的大衣是燃燒的、灼熱的、熾熱的;鉢也是燃燒的、灼熱的、熾熱的;腰帶也是燃燒的、灼熱的、熾熱的;身體也是燃燒的、灼熱的、熾熱的,她發出(作)痛苦的聲音。

  學友!那個我想這個:『實在\twnr{不可思議}{206.0}啊,\twnr{先生}{202.0}!實在\twnr{未曾有}{206.0}啊,先生!竟然也會有如此形色的眾生,竟然也會有如此形色的\twnr{夜叉}{126.0},竟然也會有如此形色的\twnr{自體的獲得}{661.0}。』」

  那時,\twnr{世尊}{12.0}召喚\twnr{比丘}{31.0}們:

  「比丘們!弟子們確實住於為\twnr{眼已生者}{455.0}的,比丘們!弟子們確實住於為\twnr{智已生者}{456.0},確實是因為弟子能知道、看到、見證如此之形色。

  比丘們!那位沙彌尼以前就被我看見,但我沒解說。如果我解說那個,他人對我不相信,凡對我不相信者,那對他們有長久的不利、苦。

  比丘們!這位沙彌尼在迦葉遍正覺者教說時代是惡沙彌尼,她那個業的果報在地獄被折磨幾年、好幾百年、好幾千年、好幾十萬年後,就以那個業的殘餘果報[\suttaref{SN.19.1}]感受這樣形色自體的獲得。」

  第二品,其\twnr{攝頌}{35.0}:

  「潛入坑因為他是姦夫,食糞者是惡心婆羅門,

   無皮膚的女子是姦婦,臉色青白的女子是女占相者,

   中暑者炭火澆灑另一位妃子,斷頭者是劊子手,

   比丘、比丘尼、式叉摩那,沙彌又沙彌尼,

   在迦葉律中出家,在那時作惡業。」

  勒叉那相應完成。





\page

\xiangying{20}{譬喻相應}
\sutta{1}{1}{屋頂經}{https://agama.buddhason.org/SN/sn.php?keyword=20.1}
  \twnr{被我這麼聽聞}{1.0}:

  \twnr{有一次}{2.0},\twnr{世尊}{12.0}住在舍衛城祇樹林給孤獨園。

  在那裡,世尊召喚\twnr{比丘}{31.0}們:「比丘們!」

  「\twnr{尊師}{480.0}!」那些比丘回答世尊。

  世尊說這個:

  「比丘們!猶如凡任何\twnr{重閣}{213.0}的\twnr{椽}{663.0},那些全部是走到屋頂的、屋頂為會合,從屋頂的根除,那一切都走到根除。同樣的,比丘們!凡任何不善法,那一切都是無明為根本的、無明為會合,從無明的根除,那一切都走到根除。

  比丘們!因此,在這裡,應該被這麼學:『我們將要不放逸地生活(住)。』比丘們!應該被你們這麼學。」 



\sutta{2}{2}{指甲尖經}{https://agama.buddhason.org/SN/sn.php?keyword=20.2}
  住在舍衛城。

  那時,\twnr{世尊}{12.0}使微少塵土沾在指甲尖後,召喚\twnr{比丘}{31.0}們:

  「比丘們!你們怎麼想它,哪個是比較多的呢:凡這被我沾在指甲尖的微少塵土,或凡這大地?」

  「\twnr{大德}{45.0}!這正是比較多的,即:大地,被世尊沾在指甲尖的微少塵土是少量的。被世尊沾在指甲尖的微少塵土比較大地後,不來到計算,也不來到比較,也不來到十六分之一的部分。」

  「同樣的,比丘們!那些眾生是少的:凡再生於人,而這些眾生正是更多的:凡從人間再生於他處。比丘們!因此,在這裡,應該被這麼學:『我們將要不放逸地生活(住)。』比丘們!應該被你們這麼學。」[\suttaref{SN.56.102}]



\sutta{3}{3}{家經}{https://agama.buddhason.org/SN/sn.php?keyword=20.3}
  住在舍衛城……(中略)。

  「\twnr{比丘}{31.0}們!猶如凡任何多女人、少男人的家,他們是容易被盜賊、\twnr{小偷}{x325}侵犯的。同樣的,比丘們!凡任何比丘的\twnr{慈心解脫}{589.0}不被\twnr{修習}{94.0}、不被\twnr{多作}{95.0}者,他是容易被\twnr{非人}{130.0}侵犯的。

  比丘們!猶如凡任何多男人、少女人的家,他們是難被盜賊、小偷侵犯的。同樣的,比丘們!凡任何比丘的慈心解脫被修習、被多作者,他是難被非人侵犯的。

  比丘們!因此,在這裡,應該被這麼學:『我們的慈心解脫要被修習、被多作、被作為車輛、被作為基礎、被實行、被累積、\twnr{被善努力}{682.0}。』比丘們!應該被你們這麼學。」



\sutta{4}{4}{大口鍋經}{https://agama.buddhason.org/SN/sn.php?keyword=20.4}
  住在舍衛城……(中略)。

  「\twnr{比丘}{31.0}們!如果他午前時施與\twnr{百大口鍋[食物]}{x326}之布施,如果他中午時施與百大口鍋之布施,如果他傍晚時施與百大口鍋之布施;如果他午前時\twnr{修習}{94.0}慈心即使甚至如擠牛奶時拉一次奶頭那樣短的時間,或如果中午時修習慈心即使甚至如擠牛奶時拉一次奶頭那樣短的時間,或如果傍晚時修習慈心即使甚至如擠牛奶時拉一次奶頭那樣短的時間,這是比那個有更大果的。

  比丘們!因此,在這裡,應該被這麼學:『我們的\twnr{慈心解脫}{589.0}要被修習、被\twnr{多作}{95.0}、被作為車輛、被作為基礎、被實行、被累積、\twnr{被善努力}{682.0}。』比丘們!應該被你們這麼學。」



\sutta{5}{5}{矛經}{https://agama.buddhason.org/SN/sn.php?keyword=20.5}
  住在舍衛城……(中略)。

  「\twnr{比丘}{31.0}們!猶如有銳利尖端的矛(匕首),那時,如果男子走來:『我將以手或拳使這銳利尖端的矛倒退、避開、返回。』

  比丘們!你怎麼想它:那男子是否能夠以手或拳使這銳利尖端的矛倒退、避開、返回呢?」

  「\twnr{大德}{45.0}!這確實不是,那是什麼原因?大德!因為,不容易以手或以拳使那銳利尖端的矛倒退、避開、返回。還有,那位男子最終只會是疲勞的、苦惱的\twnr{有分者}{876.0}。」

  「同樣的,比丘們!凡任何比丘的\twnr{慈心解脫}{589.0}被\twnr{修習}{94.0}、被\twnr{多作}{95.0},被作為車輛、被作為基礎、被實行、被累積、\twnr{被善努力}{682.0},如果\twnr{非人}{130.0}想:他的心能被混亂(拋出)。那時,那位非人只會是疲勞的、苦惱的有分者。

  比丘們!因此,在這裡,應該被這麼學:『我們的慈心解脫要被修習、被多作、被作為車輛、被作為基礎、被實行、被累積、被善努力。』比丘們!應該被你們這麼學。」



\sutta{6}{6}{弓箭手經}{https://agama.buddhason.org/SN/sn.php?keyword=20.6}
  住在舍衛城。……(中略)

  「\twnr{比丘}{31.0}們!猶如有四位精通的、善學的、熟練的、擅長弓術的弓箭手站在四方,那時,如果男子走來:『我將捉住被這四位精通的、善學的、熟練的、擅長弓術的弓箭手從四方射出、未在地上住立(落地)的箭後帶來。』

  比丘們!你們怎麼想它:『男子是疾速者,具備無上速度者。』這是適當的言語嗎?」

  「\twnr{大德}{45.0}!即使能捉住被一位精通的、善學的、熟練的、擅長弓術的弓箭手射出、未在地上住立的箭後帶來,『男子是疾速者,具備無上速度者。』這是適當的言語,更不用說四位精通的、善學的、熟練的、擅長弓術的弓箭手。」

  「比丘們!如那位男子的速度,日月的速度比那個更快。

  比丘們!如那位男子的速度、如日月的速度、如凡天神們跑在日月前面者,\twnr{壽行}{766.0}被耗盡比那些天神的速度更快。

  比丘們!因此,在這裡,應該被這麼學:『我們將要不放逸地生活(住)。』比丘們!應該被你們這麼學。」



\sutta{7}{7}{楔子經}{https://agama.buddhason.org/SN/sn.php?keyword=20.7}
  住在舍衛城。……(中略)

  「\twnr{比丘}{31.0}們!從前,有達沙羅哈人名叫\twnr{阿那葛}{x327}的小鼓,對它,達沙羅哈人在阿那葛[修復]組合時放入其他的楔子。比丘們!有那個時候:凡阿那葛小鼓的舊皮與板消失了,只殘留楔子聚合。同樣的,比丘們!\twnr{未來世}{308.0}比丘們將成為:凡那些如來說的甚深、義之甚深、出世間、\twnr{空關聯的}{637.0}\twnr{經典}{x328},在那些被說時,他們將不想要聽、將不傾耳、將不使諸了知對心現起,也將不認為那些法應該被把握、應該被學得。

  然而,凡那些外部弟子說的:詩人作的詩、美詞、美句的經典,在那些被說時,他們將想要聽、將傾耳、將使諸了知對心現起,也將認為那些法應該被把握、應該被學得。

  比丘們!這樣,這些如來說的甚深、義之甚深、出世間、空關聯的經典將會消失。

  比丘們!因此,在這裡,應該被這麼學:『凡那些如來說的甚深、義之甚深、出世間、空關聯的經典,在那些被說時,我們將想要聽、將傾耳、將使諸了知對心現起,也將認為那些法應該被把握、應該被學得。』比丘們!應該被你們這麼學。」



\sutta{8}{8}{圓木頭經}{https://agama.buddhason.org/SN/sn.php?keyword=20.8}
  \twnr{被我這麼聽聞}{1.0}:

  \twnr{有一次}{2.0},\twnr{世尊}{12.0}住在毘舍離大林重閣講堂。

  在那裡,世尊召喚\twnr{比丘}{31.0}們:「比丘們!」

  「\twnr{尊師}{480.0}!」那些比丘回答世尊。

  世尊說這個:

  「比丘們!現在,離車人住於圓木頭為枕頭、不放逸的、在訓練上熱心的,摩揭陀國阿闍世王韋提希子對他們沒得到機會、沒得到對象(所緣)。

  比丘們!未來時,離車人將成為纖細的,手腳柔軟的、嬌嫩的,他們在柔軟的床與綿枕上將睡到太陽上昇,摩揭陀國阿闍世王韋提希子對他們將得到機會、將得到對象。

  比丘們!現在,比丘們住於圓木頭為枕頭、不放逸的、在勤奮上熱心的,魔\twnr{波旬}{49.0}對他們沒得到機會、沒得到對象。

  比丘們!未來時,比丘們將成為精細的,手腳柔軟的、嬌嫩的,他們在柔軟的床與綿枕上將睡到太陽上昇,魔波旬對他們將得到機會、將得到對象。

  比丘們!因此,在這裡,應該被這麼學:『我們將要住於圓木頭為枕頭、不放逸的、在勤奮上熱心的。』比丘們!應該被你們這麼學。」



\sutta{9}{9}{象經}{https://agama.buddhason.org/SN/sn.php?keyword=20.9}
  \twnr{被我這麼聽聞}{1.0}:

  \twnr{有一次}{2.0},\twnr{世尊}{12.0}住在舍衛城祇樹林給孤獨園。

  當時,某位新進\twnr{比丘}{31.0}過度地去諸家,比丘們對他這麼說:「\twnr{尊者}{200.0}!你不要過度地去諸家。」

  當被比丘們這麼說時,那位比丘這麼說:「這些所謂的\twnr{上座}{135.0}比丘們將會想諸家應該被去,那麼,更何況是我?」

  那時,眾多比丘去見世尊。抵達後,向世尊\twnr{問訊}{46.0}後,在一旁坐下。在一旁坐下的那些比丘對世尊說這個:

  「\twnr{大德}{45.0}!這裡,某位新進比丘過度地去諸家,比丘們對他這麼說:『尊者!你不要過度地去諸家。』當被比丘們這麼說時,那位比丘這麼說:『這些所謂的上座比丘們將會想諸家應該被去,那麼,更何況是我?』」

  「比丘們!從前,在\twnr{林野}{142.0}處有大湖,諸象依止它住,牠們潛入那個湖中後,以鼻拔出蓮的幼芽後,徹底清洗地清洗後,咀嚼後,吃無泥的,這是為了牠們的容色同時也為了力量,不從那個因由遭受死亡,或死亡程度的苦。

  比丘們!但只模仿那些大象的年輕幼象子獸們潛入那個湖中後,以鼻拔出蓮的幼芽後,不徹底清洗地清洗後,不咀嚼後,吃有泥的,這既不是為了牠們的容色也非為了力量,從那個因由遭受死亡,或死亡程度的苦。

  同樣的,比丘們!這裡,上座比丘們午前時穿衣、拿起衣鉢後,\twnr{為了托鉢}{87.0}進入村落或城鎮,在那裡,他們說法,在家人們對他們作\twnr{淨信的行為}{340.2},他們不繫結地、不迷昏頭地、無罪過地、看見\twnr{過患}{293.0}地、\twnr{出離}{294.0}慧地受用那個利得,這是為了他們的容色同時也為了力量,不從那個因由遭受死亡,或死亡程度的苦。

  比丘們!但只模仿那些上座比丘的新進比丘們午前時穿衣、拿起衣鉢後,為了托鉢進入村落或城鎮,在那裡,他們說法,在家人們對他們作淨信的行為,他們繫結地、迷昏頭地、有罪過地、不看見過患地、無出離慧地受用那個利得,這既不是為了他們的容色也非為了力量,從那個因由遭受死亡,或死亡程度的苦。

  比丘們!因此,在這裡,應該被這麼學:『我們將要不繫結地、不迷昏頭地、無罪過地、看見過患地、出離慧地受用那個利得。』比丘們!應該被你們這麼學。」



\sutta{10}{10}{貓經}{https://agama.buddhason.org/SN/sn.php?keyword=20.10}
  住在舍衛城。

  當時,\twnr{某位比丘}{39.0}在諸家中過度地交際,比丘們對他這麼說:「\twnr{尊者}{200.0}!你不要在諸家中過度地交際。」

  當被比丘們這麼說時,那位比丘不停止。

  那時,眾多比丘去見世尊。抵達後,向世尊\twnr{問訊}{46.0}後,在一旁坐下。在一旁坐下的那些比丘對世尊說這個:

  「\twnr{大德}{45.0}!這裡,某位比丘在諸家中過度地交際,比丘們對他這麼說:『尊者!你不要在諸家中過度地交際。』當被比丘們這麼說時,那位比丘不停止。」

  「比丘們!從前,貓站在間隙、下水道、垃圾堆中探求著柔軟的鼠:『每當這隻柔軟的鼠將出發到食物處(覓食),就在那裡,抓住牠後,我將吃。』比丘們!那時,那隻柔軟的鼠出發到食物處,貓抓住牠後,{咀嚼}[未咀嚼]後,匆忙地吞下,那隻柔軟的鼠咬牠的腸,又咬腸膜,牠從那個因由遭受死亡,或死亡程度的苦。

  同樣的,比丘們!這裡,一類比丘午前時穿衣、拿起衣鉢後,以身未守護、以語未守護、以心未守護,以念未現起,以諸根未防護,\twnr{為了托鉢}{87.0}進入村落或城鎮,在那裡,他看見穿著暴露的或衣著暴露的婦女。看見穿著暴露的或衣著暴露的婦女後,貪使他的心墮落,他以貪使墮落的心遭受死亡,或死亡程度的苦。比丘們!因為,在聖者之律中這是死亡:凡放棄學後還俗;比丘們!因為,這是死亡程度的苦,即:來到某個被污染的罪,在像這樣的罪上出罪被\twnr{安立}{143.0}。

  比丘們!因此,在這裡,應該被這麼學:『我們將要以身已守護、以語已守護、以心已守護,以念已現起,以諸根已防護,為了托鉢進入村落或城鎮。』比丘們!應該被你們這麼學。」



\sutta{11}{11}{狐狼經}{https://agama.buddhason.org/SN/sn.php?keyword=20.11}
  住在舍衛城。……(中略)

  「\twnr{比丘}{31.0}們!你們在破曉時聽到老狐狼鳴叫著嗎?」

  「是的,\twnr{大德}{45.0}!」

  「比丘們!那隻老狐狼被名為疥瘡生起的病接觸,牠去牠想要之處,牠站牠想要之處,牠坐牠想要之處,牠躺牠想要之處,涼風強力吹牠。

  比丘們!這裡,凡某類\twnr{釋迦之徒的}{262.1}自稱者,如果他經驗像這樣形色的\twnr{自體的獲得}{661.0},\twnr{那就好了}{44.0}!

  比丘們!因此,在這裡,應該被這麼學:『我們將要不放逸地生活(住)。』比丘們!應該被你們這麼學。」



\sutta{12}{12}{狐狼經第二}{https://agama.buddhason.org/SN/sn.php?keyword=20.12}
  住在舍衛城。……(中略)

  「\twnr{比丘}{31.0}們!你們在破曉時聽到老狐狼鳴叫著嗎?」

  「是的,\twnr{大德}{45.0}!」

  「比丘們!在那隻老狐狼上,會有某些(任何)知恩、感恩,然而,這裡,在某類\twnr{釋迦之徒的}{262.1}自稱者上,不會有任何知恩、感恩。

  比丘們!因此,在這裡,應該被這麼學:『我們將要是知恩者、感恩者,而且,即使於我們少量所作的[恩惠]也將不毀滅。』比丘們!應該被你們這麼學。」

  譬喻相應完成,其\twnr{攝頌}{35.0}:

  「屋頂、指甲尖、家,大口鍋、矛、弓箭手,

   楔子、圓木頭、象,貓、二則狐狼。」





\page

\xiangying{21}{比丘相應}
\sutta{1}{1}{芶里德經}{https://agama.buddhason.org/SN/sn.php?keyword=21.1}
  \twnr{被我這麼聽聞}{1.0}:

  \twnr{有一次}{2.0},\twnr{世尊}{12.0}住在舍衛城祇樹林給孤獨園。

  在那裡,\twnr{尊者}{200.0}大目揵連召喚\twnr{比丘}{31.0}們:「比丘\twnr{學友}{201.0}們!」

  「學友!」那些比丘回答尊者大目揵連。

  尊者大目揵連說這個:

  「學友們!這裡,當我獨處、\twnr{獨坐}{92.0}時,這樣心的深思生起:『被稱為「\twnr{聖沈默狀態}{548.0}、聖沈默狀態」,什麼是聖沈默狀態呢?』

  學友們!那個我想這個:『這裡,比丘從尋與伺的平息,\twnr{自身內的明淨}{256.0},\twnr{心的專一性}{255.0},\twnr{進入後住於}{66.0}無尋、無伺,定而生喜、樂的第二禪,這被稱為聖沈默狀態。』

  學友們!那個我從尋與伺的平息,自身內的明淨,心的專一性,進入後住於無尋、無伺,定而生喜、樂的第二禪。學友們!當我以這個住處住時,與尋俱行的諸想、諸\twnr{作意}{43.1}被執行。

  學友們!那時,世尊以神通來見我後,說這個:『目揵連!目揵連!你不要對聖沈默狀態放逸,\twnr{婆羅門}{17.0}!請你使心\twnr{安頓}{888.0}在聖沈默狀態,請你在聖沈默狀態上作一心,請你在聖沈默狀態上集中心。』

  學友們!過些時候,那個我從尋與伺的平息,自身內的明淨,心的專一性,進入後住於無尋、無伺,定而生喜、樂的第二禪。

  學友們!凡當正確說它時,應該說『被\twnr{大師}{145.0}資助達到\twnr{大通智}{564.0}的弟子』,那是我,當正確說時,應該說『被大師資助達到大通智的弟子』。」



\sutta{2}{2}{優玻低色經}{https://agama.buddhason.org/SN/sn.php?keyword=21.2}
  住在舍衛城。

  在那裡,\twnr{尊者}{200.0}舍利弗召喚\twnr{比丘}{31.0}們:「比丘\twnr{學友}{201.0}們!」

  「學友!」那些比丘回答尊者舍利弗。

  尊者舍利弗說這個:

  「學友們!這裡,當我獨處、\twnr{獨坐}{92.0}時,這樣心的深思生起:『世間中有任何那個凡從變易變異,我的愁、悲、苦、憂、\twnr{絕望}{342.0}會生起嗎?』

  學友們!那個我想這個:『世間中沒有任何那個凡從變易變異,我的愁、悲、苦、憂、絕望會生起。』」

  在這麼說時,尊者阿難對尊者舍利弗說這個:

  「舍利弗學友!從\twnr{大師}{145.0}的變易變異,你的愁、悲、苦、憂、絕望也不會生起嗎?」

  「學友!從大師的變易變異,我的愁、悲、苦、憂、絕望也不會生起。但我會這麼想:『唉!\twnr{先生}{202.0}!大影響力的、\twnr{大神通力的}{405.0}、大威力的大師\twnr{已滅沒了}{x329},如果\twnr{世尊}{12.0}長時間久住,那會是對眾人有利益的,對眾人安樂,對世間憐愍,對天與人的需要、利益、安樂。」

  「而像這樣,因為尊者舍利弗的\twnr{我作}{22.0}、\twnr{我所作}{25.0}、\twnr{慢煩惱潛在趨勢}{26.0}被\twnr{長久}{51.0}地善根除,因此,從大師的變易變異,尊者舍利弗的愁、悲、苦、憂、絕望也不會生起。」



\sutta{3}{3}{甕經}{https://agama.buddhason.org/SN/sn.php?keyword=21.3}
  \twnr{被我這麼聽聞}{1.0}:

  \twnr{有一次}{2.0},\twnr{世尊}{12.0}住在舍衛城祇樹林給孤獨園。

  當時,\twnr{尊者}{200.0}舍利弗與尊者大目揵連住在王舍城栗鼠飼養處的竹林中同一住處。

  那時,傍晚時,從\twnr{獨坐}{92.0}出來的尊者舍利弗去見尊者大目揵連。抵達後,與尊者大目揵連一起互相問候。交換應該被互相問候的友好交談後,在一旁坐下。在一旁坐下的尊者舍利弗對尊者大目揵連說這個:

  「大目揵連\twnr{學友}{201.0}!你的諸根是明淨的,臉色是遍純淨的、皎潔的,尊者大目揵連今日以寂靜的住處住,是嗎?」

  「學友!我今日以粗的住處住,此外,有我的法談。」

  「學友!那麼,與誰一起有尊者大目揵連的法談呢?」

  「學友!與世尊一起有我的法談。」

  「學友!世尊現在住在遠處的舍衛城祇樹林給孤獨園,尊者大目揵連以神通去見世尊,或者世尊以神通來見尊者大目揵連呢?」

  「學友!非我以神通去見世尊,也非世尊以神通來見我,而是世尊只對我淨化天眼與天耳界的情形,我也只對世尊淨化天眼與天耳界的情形。」

  「那麼,與世尊一起有尊者大目揵連關於怎樣的法談呢?」

  「學友!這裡,我對世尊說這個:『\twnr{大德}{45.0}!被稱為「活力已發動者,活力已發動者」,大德!什麼情形是活力已發動者?』

  學友!在這麼說時,世尊對我說這個:『目揵連!這裡,活力已被發動的\twnr{比丘}{31.0}住於:「寧願剩下皮膚、肌腱、骨骸,身體中的血肉變乾,凡那個應該被人的力量、人的活力、人的努力達成的,那個沒達成後,他將沒有活力的止息。」目揵連!這樣是活力已發動者。』學友!與世尊一起有我這樣的法談。」

  「學友!猶如頂多是微小的碎石對喜馬拉雅山山王的比較量,同樣的,頂多是我們對尊者大目揵連的比較量。因為尊者大目揵連是大神通力、大威力者,當他希望時,能住世一劫。」

  「學友!猶如頂多是微小的鹽粒對大甕鹽的比較量,同樣的,頂多是我們對尊者舍利弗的比較量。因為尊者舍利弗被世尊以種種法門稱頌、稱讚、讚美:

  『如舍利弗以慧,以戒以寂靜,

   凡即使到\twnr{彼岸}{226.0}的\twnr{比丘}{31.0},最高者會是這樣程度的。』[\suttaref{SN.1.48}]」

  像這樣,那兩位\twnr{大龍}{x330}互相對善說的、善談話的喜悅。



\sutta{4}{4}{新進者經}{https://agama.buddhason.org/SN/sn.php?keyword=21.4}
  住在舍衛城。

  當時,某位新進\twnr{比丘}{31.0}\twnr{餐後已從施食返回}{512.0},進入住處後\twnr{不活動地}{906.0}、沈默地保持靜止,在縫製衣服時不作比丘們的工作。

  那時,眾多比丘去見世尊。抵達後,向世尊\twnr{問訊}{46.0}後,在一旁坐下。在一旁坐下的那些比丘對世尊說這個:

  「\twnr{大德}{45.0}!這裡,某位新進比丘餐後已從施食返回,進入住處後不活動地、沈默地保持靜止,在縫製衣服時不作比丘們的工作?」

  那時,世尊召喚某位比丘:

  「來!比丘!請你以我的名義召喚那位比丘:『\twnr{學友}{201.0}!\twnr{大師}{145.0}召喚你。』」

  「是的,大德!」那位比丘回答世尊後,去見那位比丘。抵達後,對那位比丘說這個:「學友!大師召喚你。」

  「是的,學友!」那位比丘回答那位比丘後,去見世尊。抵達後,向世尊問訊後,在一旁坐下。世尊對在一旁坐下的那位比丘說這個:

  「比丘們!傳說是真的?你餐後已從施食返回,進入住處後不活動地、沈默地保持靜止,在縫製衣服時不作比丘們的工作?」

  「大德!我也作應該被自己作的。」

  那時,世尊以心了知那位比丘心中的深思後,召喚比丘們:

  「比丘們!你們不要嫌責這位比丘,比丘們!這位比丘是\twnr{增上心、當生樂住處之四禪的}{443.0}隨欲得到者、不困難得到者、無困難得到者,以及是以證智自作證後,在當生中\twnr{進入後住於}{66.0}凡\twnr{善男子}{41.0}們為了利益正確地\twnr{從在家出家成為無家者}{48.0}的那個無上梵行結尾者。」

  世尊說這個,說這個後,\twnr{善逝}{8.0}、大師又更進一步說這個:

  「\twnr{這非鬆弛地精勤後}{x331},這非以少的力量,

   有涅槃能被證得,一切苦的釋放。

   而這位年輕的比丘,這位是無上之人,

   征服有軍隊的魔後,持最後身。」



\sutta{5}{5}{善生經}{https://agama.buddhason.org/SN/sn.php?keyword=21.5}
  住在舍衛城。

  那時,\twnr{尊者}{200.0}善生去見\twnr{世尊}{12.0}。

  世尊看見正從遠處到來的那位尊者善生。看見後,召喚\twnr{比丘}{31.0}們:

  「比丘們!這位\twnr{善男子}{41.0}就在兩方面輝耀:是英俊的、好看的、端正的、具備最美的容色;以及是以證智自作證後,在當生中\twnr{進入後住於}{66.0}凡\twnr{善男子}{41.0}們為了利益正確地\twnr{從在家出家成為無家者}{48.0}的那個無上梵行結尾者。」

  世尊說這個……(中略)[,說這個後\twnr{善逝}{8.0}、\twnr{大師}{145.0}又更進一步這麼]說:

  「這位比丘確實輝耀:以成為正直的心,

   是離繫縛者離結縛者,不執取後到達涅槃者,

   征服有軍隊的魔後,持最後身。」



\sutta{6}{6}{侏儒拔提亞經}{https://agama.buddhason.org/SN/sn.php?keyword=21.6}
  住在舍衛城。

  那時,\twnr{尊者}{200.0}\twnr{侏儒拔提亞}{x332}去見\twnr{世尊}{12.0}。

  世尊看見正從遠處到來的尊者侏儒拔提亞。看見後,召喚\twnr{比丘}{31.0}們:

  「比丘們!你們看見這位走來的醜陋、\twnr{難看}{x333}、矮小、在比丘們中被輕蔑樣子的比丘嗎?」

  「是的,\twnr{大德}{45.0}!」

  「比丘們!這位比丘是大神通力、大威力者,而那種\twnr{等至}{129.0}是不易得到的形色:凡以前未被那位比丘進入的,以及是以證智自作證後,在當生中\twnr{進入後住於}{66.0}凡\twnr{善男子}{41.0}們為了利益正確地\twnr{從在家出家成為無家者}{48.0}的那個無上梵行結尾者。」

  世尊說這個……(中略)[,說這個後,\twnr{善逝}{8.0}、\twnr{大師}{145.0}又更進一步這麼]說:

  「天鵝、白鷺和孔雀,大象、梅花鹿,

   全都害怕獅子,不是在身體的衡量。

   同樣的在人之中,即使是年幼的有慧者,

   在那裡他確實是大者,不是僅有身體的愚者。」

  [「各部分無缺點的、有白色覆蓋物的,單輪輻的二輪車轉動,

    請看無苦惱者正到來:已切斷流者、無繫縛者。」(\ccchref{Ud.65}{https://agama.buddhason.org/Ud/dm.php?keyword=65})]



\sutta{7}{7}{毘舍佉經}{https://agama.buddhason.org/SN/sn.php?keyword=21.7}
  \twnr{被我這麼聽聞}{1.0}:

  \twnr{有一次}{2.0},\twnr{世尊}{12.0}住在毘舍離大林重閣講堂。

  當時,\twnr{尊者}{200.0}毘舍佉般遮羅子在講堂中以法說,以優雅的、明瞭的、\twnr{清晰的}{927.0}、義理令知的、完整的、\twnr{不依止的}{x334}話語對\twnr{比丘}{31.0}們開示、勸導、鼓勵、\twnr{使歡喜}{86.0}。

  那時,世尊傍晚時,從\twnr{獨坐}{92.0}出來,去講堂。抵達後,在設置的座位坐下。坐下後,世尊召喚比丘們:

  「比丘們!誰在講堂中以法說,以優雅的、明瞭的、清晰的、義理令知的、完整的、不依止的話語開示、勸導、鼓勵比丘們呢?」

  「\twnr{大德}{45.0}!尊者毘舍佉般遮羅子在講堂中以法說,以優雅的、明瞭的、清晰的、義理令知的、完整的、不依止的話語開示、勸導、鼓勵比丘們。」

  那時,世尊召喚尊者毘舍佉般遮羅子:

  「毘舍佉!\twnr{好}{44.0}!好!毘舍佉!你以法說,以優雅的、明瞭的、清晰的、義理令知的、完整的、不依止的話語對比丘們開示、勸導、鼓勵、好!」

  世尊說這個,說這個後,\twnr{善逝}{8.0}、\twnr{大師}{145.0}又更進一步說這個:

  「賢智者與愚者混雜,當他沒說話時他們不知道,

   但當他說話時他們知道:\twnr{不死}{123.0}境界的教導者。

   他應該說應該使法輝耀,他應該高舉仙人們的旗幟,

   善說的是仙人們的旗幟,因為法是仙人們的旗幟。」[\ccchref{AN.4.48}{https://agama.buddhason.org/AN/an.php?keyword=4.48}]



\sutta{8}{8}{難陀經}{https://agama.buddhason.org/SN/sn.php?keyword=21.8}
  住在舍衛城。

  那時,世尊\twnr{姨母之子}{x335}尊者難陀,穿捶平壓平的衣服、\twnr{畫眼妝}{x336}、拿起發亮的鉢後去見\twnr{世尊}{12.0}。抵達後,向世尊\twnr{問訊}{46.0}後,在一旁坐下。世尊對在一旁坐下的\twnr{尊者}{200.0}難陀說這個:

  「難陀!那對\twnr{善男子}{41.0}\twnr{以信從在家出家成為無家者}{48.0}的你是不適當的:凡如果你穿捶平壓平的衣服、畫眼妝、持發亮的鉢。難陀!這對善男子以信從在家出家成為無家者的你是適當的:凡如果你是住\twnr{林野}{142.0}者、常乞食者、穿\twnr{糞掃衣}{352.0}者、住於在欲上無期待者。」

  世尊說這個……(中略)[說這個後,\twnr{善逝}{8.0}、大師又更進一步這麼]說:

  「何時我能看見難陀,是住林野者、穿糞掃衣者,

   \twnr{以非有名者的殘食}{x337}維生者,在欲上無期待者?」

  過些時候,尊者難陀成為住林野者、常乞食者、穿糞掃衣者、住於對欲無期待者。



\sutta{9}{9}{低舍經}{https://agama.buddhason.org/SN/sn.php?keyword=21.9}
  住在舍衛城。

  那時,世尊姑媽之子\twnr{尊者}{200.0}低舍去見世尊。抵達後,向世尊\twnr{問訊}{46.0}後,痛苦地、不快意地、流淚地坐在一旁。

  那時,世尊對尊者低舍說這個:

  「低舍!你為何痛苦地、不快意地、流淚地坐在一旁呢?」

  「\twnr{大德}{45.0}!因為\twnr{比丘}{31.0}們像那樣到處以尖刺言語對我作諷刺。」

  「低舍!但因為你是像那樣說者而非言語的容忍者,低舍!對\twnr{善男子}{41.0}\twnr{以信從在家出家成為無家者}{48.0}的你來說,那是不適當的:凡你是說者而非言語的容忍者。低舍!對善男子以信從在家出家成為無家者的你,這是適當的:凡如果你是說者與言語的容忍者。」

  世尊說這個,說這個後,\twnr{善逝}{8.0}、大師又更進一步說這個:

  「你為何生氣?不要生氣!低舍!不憤怒對你比較好,

   因為為了憤怒、慢、詆毀的調伏,低舍!梵行被住。」



\sutta{10}{10}{名叫上座經}{https://agama.buddhason.org/SN/sn.php?keyword=21.10}
  \twnr{有一次}{2.0},\twnr{世尊}{12.0}住在王舍城栗鼠飼養處的竹林中。

  當時,某位名叫上座的\twnr{比丘}{31.0}是獨住者,同時也是獨住者的稱讚者,他獨自\twnr{為了托鉢}{87.0}進入村落、獨自返回、獨自坐靜處、獨自進行\twnr{經行}{150.0}。

  那時,眾多比丘去見世尊。抵達後,向世尊\twnr{問訊}{46.0}後,在一旁坐下。在一旁坐下的那些比丘對世尊說這個:

  「\twnr{大德}{45.0}!這裡,某位名叫上座的比丘是獨住者與獨住者的稱讚者。」

  那時,世尊召喚某位比丘:

  「來!比丘!請你以我的名義召喚上座比丘:『上座\twnr{學友}{201.0}!大師召喚你。』」

  「是的,大德!」那位比丘回答世尊後,去見\twnr{尊者}{200.0}上座。抵達後,對尊者上座說這個:「學友!大師召喚你。」

  「是的,學友!」尊者上座回答那位比丘後,去見世尊。抵達後,向世尊問訊後,在一旁坐下。世尊對在一旁坐下的尊者上座說這個:

  「上座!傳說是真的?你是獨住者與獨住者的稱讚者。」

  「是的,大德!」

  「上座!那麼,如怎樣你是獨住者與獨住者的稱讚者?」

  「大德!這裡,我獨自為了托鉢進入村落、獨自返回、獨自坐靜處、獨自進行經行,大德!我是這樣的獨住者與獨住者的稱讚者。」

  「上座!這是獨住,我不說:『這不是。』但,上座!如是獨住詳細地被圓滿,你要聽!你要\twnr{好好作意}{43.1}!我將說。」

  「是的,大德!」……(中略)

  「上座!而怎樣的獨住詳細地被圓滿呢?上座!這裡,凡過去的被捨斷,凡未來的被\twnr{斷念}{211.0},以及在現在\twnr{自體的獲得}{661.0}上意欲貪被徹底排除,上座!這樣的獨住詳細地被圓滿。」

  世尊說這個,說這個後,\twnr{善逝}{8.0}、大師又更進一步說這個:

  「\twnr{一切的征服者}{x338}、知一切的極聰明者,在一切法上不被染著者,

   一切的捨斷者、在渴愛的滅盡上解脫者,我說他是獨住者。」



\sutta{11}{11}{大劫賓那經}{https://agama.buddhason.org/SN/sn.php?keyword=21.11}
  住在舍衛城。 

  那時,\twnr{尊者}{200.0}大劫賓那去見\twnr{世尊}{12.0}。

  世尊看見正從遠處到來的尊者大劫賓那。看見後,召喚\twnr{比丘}{31.0}們:

  「比丘們!你們看見這位走來的白皙的、瘦的、高鼻子的比丘嗎?」

  「是的,\twnr{大德}{45.0}!」

  「比丘們!這位比丘是大神通力、大威力者,而那種\twnr{等至}{129.0}是不易得到的形色:凡以前未被那位比丘進入的,以及是以證智自作證後,在當生中\twnr{進入後住於}{66.0}凡\twnr{善男子}{41.0}們為了利益正確地\twnr{從在家出家成為無家者}{48.0}的那個無上梵行結尾者。」

  世尊說這個,說這個後,\twnr{善逝}{8.0}、\twnr{大師}{145.0}又更進一步說這個:

  「剎帝利在這人們中是最上:凡歸屬種姓者們,

   \twnr{明行具足者}{7.0},在天人中他是最上的。[\ccchref{DN.3}{https://agama.buddhason.org/DN/dm.php?keyword=3}-277段, \ccchref{DN.27}{https://agama.buddhason.org/DN/dm.php?keyword=27}-140段]

   白天太陽輝耀,夜裡月亮發光,

   武裝的剎帝利輝耀,禪修的婆羅門輝耀,

   日夜一切時候,佛陀\twnr{以威光}{x339}輝耀。」



\sutta{12}{12}{同伴經}{https://agama.buddhason.org/SN/sn.php?keyword=21.12}
  住在舍衛城。 

  那時,二位\twnr{尊者}{200.0}大劫賓那的共住者(弟子)比丘同伴去見\twnr{世尊}{12.0}。

  世尊看見正從遠處到來的那些\twnr{比丘}{31.0}。看見後,召喚比丘們:

  「比丘們!你們看見那些走來的劫賓那的共住者比丘同伴嗎?」

  「是的,\twnr{大德}{45.0}!」

  「這些比丘他們是大神通力、大威力者,而那種\twnr{等至}{129.0}是不易得到的形色:凡以前未被那位比丘進入的,以及是以證智自作證後,在當生中\twnr{進入後住於}{66.0}凡\twnr{善男子}{41.0}們為了利益正確地\twnr{從在家出家成為無家者}{48.0}的那個無上梵行結尾者。」

  世尊說這個,說這個後,\twnr{善逝}{8.0}、\twnr{大師}{145.0}又更進一步說這個:

  「這些比丘同伴確實,是長時間\twnr{結交者}{x340},

   正法會合(結交)他們:在被佛陀宣說的法上。

   被劫賓那善教導:在被聖者告知的法上,

   征服有軍隊的魔後,他們持最後身。」

  比丘相應完成,其\twnr{攝頌}{35.0}:

  「芶里德與優玻低色,還有被說為甕,

   新進者、善生、拔提,以及毘舍佉、難陀、低舍,

   名叫上座與劫賓那,以及以同伴為十二。」

  因緣篇第二,其攝頌:

  「因緣、現觀、界,無始與迦葉,

   恭敬、羅侯羅、勒叉那,譬喻、比丘,

   以那個被稱為第二篇。」

  因緣篇相應經典終了。





\page

\pian{蘊篇}{22}{34}
\xiangying{22}{蘊相應}
\pin{那拘羅的父親品}{1}{11}
\sutta{1}{1}{那拘羅的父親經}{https://agama.buddhason.org/SN/sn.php?keyword=22.1}
  \twnr{被我這麼聽聞}{1.0}:

  \twnr{有一次}{2.0},\twnr{世尊}{12.0}住在婆祇國蘇蘇馬拉山之配沙卡拉林的鹿林。

  那時,\twnr{屋主}{103.0}那拘羅的父親去見世尊。抵達後,向世尊\twnr{問訊}{46.0}後,在一旁坐下。在一旁坐下的屋主那拘羅的父親對世尊說這個:

  「\twnr{大德}{45.0}!我是衰老的、年老的、高齡的、老年的、到達老年的、身體病苦的、經常生病,大德!又,我是世尊與值得尊敬的\twnr{比丘}{31.0}們的不常看見者,大德!請世尊教誡我,大德!請世尊訓誡我:凡我會有長久的利益、安樂。」

  「這是這樣,屋主!這是這樣,屋主!

  屋主!這病苦的身體成為蛋被覆蓋的,屋主!凡帶著這個身體者如果還片刻地自稱無病,除了愚者外還有什麼?屋主!因此,在這裡,應該被你這麼學:『對病苦的身體,我具念的心將是無病苦的。』屋主!應該被你這麼學。」

  那時,屋主那拘羅的父親歡喜、\twnr{隨喜}{85.0}世尊所說後,從座位起來、向世尊問訊、\twnr{作右繞}{47.0}後,去見\twnr{尊者}{200.0}舍利弗。抵達後,向尊者舍利弗問訊後,在一旁坐下。尊者舍利弗對在一旁坐下的屋主那拘羅的父親說這個:

  「屋主!你的諸根是明淨的,臉色是遍純淨的、皎潔的,今日你從世尊的面前得到聽聞法談嗎?」

  「大德!怎樣不會是呢,大德!這裡,我被世尊以\twnr{不死}{123.0}的法談灌頂。」

  「屋主!那麼,如怎樣你被世尊以不死的法談灌頂?」

  「大德!這裡,我去見世尊。抵達後,向世尊問訊後,在一旁坐下。大德!在一旁坐下的我對世尊說這個:『我是衰老的、年老的、高齡的、老年的、到達老年的、身體病苦的、經常生病,大德!又,我是世尊與值得尊敬的比丘們的不常看見者,大德!請世尊教誡我,大德!請世尊訓誡我:凡我會有長久的利益、安樂。』大德!在這麼說時,世尊對我說這個:『這是這樣,屋主!這是這樣,屋主!

  屋主!這病苦的身體成為蛋被覆蓋的,屋主!凡帶著這個身體者如果還片刻地自稱無病,除了愚者外還有什麼?屋主!因此,在這裡,應該被你這麼學:『對病苦的身體,我具念的心將是無病苦的。」屋主!應該被你這麼學。』大德!這樣,我被世尊以不死的法談灌頂。」

  「屋主!那麼,你沒更進一步反問世尊說明:『大德!什麼情形是病苦的身體同時也病苦的心呢?什麼情形是病苦的身體但無病苦的心?』」

  「大德!我們會從遠處來到尊者舍利弗的面前,也為了能了知這個所說的義理,就請尊者舍利弗說明這個所說的義理,\twnr{那就好了}{44.0}!」

  「屋主!那樣的話,請你聽,請你\twnr{好好作意}{43.1}!我將說。」

  「是的,大德!」屋主那拘羅的父親回答尊者舍利弗。

  尊者舍利弗說這個:

  「屋主!而怎樣是病苦的身體同時也病苦的心?

  屋主!這裡,\twnr{未聽聞的一般人}{74.0}是聖者的未看見者,聖者法的不熟知者,在聖者法上未被教導者;是善人的未看見者,\twnr{善人法}{76.0}的不熟知者,在善人法上未被教導者,他\twnr{認為}{964.0}色是我,\twnr{或我擁有色}{13.0},或色在我中,\twnr{或我在色中}{14.0},他是處在『我是色、色是我的』\twnr{的纏縛者}{x341};當他是處在『我是色、色是我的』的纏縛者時,那個色\twnr{變易成為不同的}{146.0},從色的變易變異,他的愁、悲、苦、憂、\twnr{絕望}{342.0}生起。

  他認為受是我,或我擁有受,或受在我中,或我在受中,他是處在『我是受、受是我的』的纏縛者;當他是處在『我是受、受是我的』的纏縛者時,那個受變易成為不同的,從受的變易變異,他的愁、悲、苦、憂、絕望生起。

  他認為想是我,或我擁有想,或想在我中,或我在想中,他是處在『我是想、想是我的』的纏縛者;當他是處在『我是想、想是我的』的纏縛者時,那個想變易成為不同的,從想的變易變異,他的愁、悲、苦、憂、絕望生起。

  他認為諸行是我,或我擁有諸行,或諸行在我中,或我在諸行中,他是處在『我是諸行、諸行是我的』的纏縛者;當他是處在『我是諸行、諸行是我的』的纏縛者時,那些行變易成為不同的,從諸行的變易變異,他的愁、悲、苦、憂、絕望生起。

  他認為識是我,或我擁有識,或識在我中,或我在識中,他是處在『我是識、識是我的』的纏縛者;當他是處在『我是識、識是我的』的纏縛者時,那個識變易成為不同的,從識的變易變異,他的愁、悲、苦、憂、絕望生起。

  屋主!這樣是病苦的身體同時也病苦的心。

  屋主!而怎樣是病苦的身體但無病苦的心?

  屋主!這裡,\twnr{有聽聞的聖弟子}{24.0}是聖者的看見者,聖者法的熟知者,在聖者法上被善教導者;是善人的看見者,善人法的熟知者,在善人法上被善教導者,他認為色不是我,或我不擁有色,或色不在我中,或我不在色中,他不是處在『我是色、色是我的』的纏縛者;當他不是處在『我是色、色是我的』的纏縛者時,那個色變易成為不同的,從色的變易變異,他的愁、悲、苦、憂、絕望不生起。

  他認為受不是我,或我不擁有受,或受不在我中,或我不在受中,他不是處在『我是受、受是我的』的纏縛者;當他不是處在『我是受、受是我的』的纏縛者時,那個受變易成為不同的,從受的變易變異,他的愁、悲、苦、憂、絕望不生起。

  他認為想不是我,或我不擁有想,或想不在我中,或我不在想中,他不是處在『我是想、想是我的』的纏縛者;當他不是處在『我是想、想是我的』的纏縛者時,那個想變易成為不同的,從想的變易變異,他的愁、悲、苦、憂、絕望不生起。

  他認為諸行不是我,或我不擁有諸行,或諸行不在我中,或我不在諸行中,他不是處在『我是諸行、諸行是我的』的纏縛者;當他不是處在『我是諸行、諸行是我的』的纏縛者時,那些行變易成為不同的,從諸行的變易變異,他的愁、悲、苦、憂、絕望不生起。

  他認為識不是我,或我不擁有識,或識不在我中,或我不在識中,他不是處在『我是識、識是我的』的纏縛者;當他不是處在『我是識、識是我的』的纏縛者時,那個識變易成為不同的,從識的變易變異,他的愁、悲、苦、憂、絕望不生起。

  屋主!這樣是病苦的身體但無病苦的心。」

  尊者舍利弗說這個,悅意的屋主那拘羅的父親歡喜尊者舍利弗所說。



\sutta{2}{2}{天臂經}{https://agama.buddhason.org/SN/sn.php?keyword=22.2}
  \twnr{被我這麼聽聞}{1.0}:

  \twnr{有一次}{2.0},\twnr{世尊}{12.0}住在釋迦族中名叫天臂的釋迦族城鎮。

  那時,眾多去西部地方的\twnr{比丘}{31.0}去見世尊。抵達後,向世尊\twnr{問訊}{46.0}後,在一旁坐下。在一旁坐下的那些比丘對世尊說這個:

  「\twnr{大德}{45.0}!我們想要去西部地方國土,準備在西部地方國土居住。」

  「比丘們!那麼,舍利弗被你們求聽許了嗎?」

  「大德!\twnr{尊者}{200.0}舍利弗沒被我們求聽許。」

  「比丘們!請你們向舍利弗求聽許,比丘們!舍利弗是賢智者、同梵行比丘們的助益者。」

  「是的,\twnr{大德}{45.0}!」那些比丘回答世尊。

  當時,尊者舍利弗坐在離世尊不遠某個肉桂樹叢下。

  那時,那些比丘歡喜、隨喜世尊所說後,從座位起來、向世尊問訊、\twnr{作右繞}{47.0}後,去見尊者舍利弗。抵達後,與尊者舍利弗一起互相問候。交換應該被互相問候的友好交談後,在一旁坐下。在一旁坐下的那些比丘對尊者舍利弗說這個:

  「舍利弗\twnr{學友}{201.0}!我們想要去西部地方國土,準備在西部地方國土居住,\twnr{大師}{145.0}被我們求聽許了。」

  「學友們!賢智的\twnr{剎帝利}{116.0}們、賢智的\twnr{婆羅門}{17.0}們、賢智的\twnr{屋主}{103.0}們、賢智的\twnr{沙門}{29.0}們是對到各國比丘的問題詢問者,學友們!因為賢智者們是人間的考察者:『那麼,尊者們的大師是論說什麼者?宣說什麼者?』是否你們諸位尊者的法被善聽聞、被善把握、被善作意、被善考慮(理解)、被以慧善貫通,如是當尊者們回答時,你們才\twnr{會是世尊的所說之說者}{115.0},而且不會以不實的誹謗世尊,以及會\twnr{法隨法地回答}{415.0},而任何如法的種種說不會來到應該被呵責處?」

  「學友!我們會從遠處來到尊者舍利弗的面前,也為了能了知這個所說的義理,就請尊者舍利弗說明這個所說的義理,\twnr{那就好了}{44.0}!」

  「學友們!那樣的話,請你們聽,請你們\twnr{好好作意}{43.1},我將說。」

  「是的,學友!」那些比丘回答尊者舍利弗。

  尊者舍利弗說這個:

  「學友們!賢智的剎帝利們……(中略)賢智的沙門們是對到各國比丘的問題詢問者,學友們!因為賢智者們是人間的考察者:『那麼,尊者們的大師是論說什麼者?宣說什麼者?』學友們!被這麼詢問,你們應該這麼回答:『\twnr{道友}{201.0}們!我們的大師是意欲貪之調伏的宣說者。』

  學友們!在這麼回答時,賢智的剎帝利們、賢智的婆羅門們、賢智的屋主們、賢智的沙門們還會是更進一步問題的詢問者,學友們!因為賢智者們是人間的考察者:『那麼,尊者們的大師是在什麼上意欲貪之調伏的宣說者?』

  學友們!被這麼詢問,你們應該這麼回答:『道友們!大師是在色上意欲貪之調伏的宣說者,在受上……在想上……在諸行上……大師是在識上意欲貪之調伏的宣說者。』

  學友們!在這麼回答時,賢智的剎帝利們、賢智的婆羅門們、賢智的屋主們、賢智的沙門們還會是更進一步問題的詢問者,學友們!因為賢智者們是人間的考察者:『那麼,看見什麼\twnr{過患}{293.0}後,尊者們的大師是在色上意欲貪之調伏的宣說者,在受上……在想上……在諸行上……大師是在識上意欲貪之調伏的宣說者?』

  學友們!被這麼詢問,你們應該這麼回答:『道友們!在色上未離貪、未離意欲、未離情愛、未離渴望、未離熱惱、未離渴愛者,從那個色的變易變異,愁、悲、苦、憂、\twnr{絕望}{342.0}生起。

  在受上……在想上……在諸行上未離貪……(中略)未離渴愛者,從那個諸行的變易變異,愁、悲、苦、憂、絕望生起。在識上未離貪、未離意欲、未離情愛、未離渴望、未離熱惱、未離渴愛者,從那個識的變易變異,愁、悲、苦、憂、絕望生起。

  學友們!看見這個過患後,我們的大師是在色上意欲貪之調伏的宣說者,在受上……在想上……在諸行上……大師是在識上意欲貪之調伏的宣說者。』

  學友們!在這麼回答時,賢智的剎帝利們、賢智的婆羅門們、賢智的屋主們、賢智的沙門們還會是更進一步問題的詢問者,學友們!因為賢智者們是人間的考察者:『那麼,看見什麼效益後,尊者們的大師是在色上意欲貪之調伏的宣說者,在受上……在想上……在諸行上……大師是在識上意欲貪之調伏的宣說者?』

  學友們!被這麼詢問,你們應該這麼回答:『道友們!在色上離貪、離意欲、離情愛、離渴望、離熱惱、離渴愛者,從那個色的變易變異,愁、悲、苦、憂、絕望不生起。

  在受上……在想上……在諸行上離貪、離意欲、離情愛、離渴望、離熱惱、離渴愛者,從那個諸行的變易變異,愁、悲、苦、憂、絕望不生起。在識上離貪、離意欲、離情愛、離渴望、離熱惱、離渴愛者,從那個識的變易變異,愁、悲、苦、憂、絕望不生起。

  學友們!看見這個效益後,我們的大師是在色上意欲貪之調伏的宣說者,在受上……在想上……在諸行上……大師是在識上意欲貪之調伏的宣說者。』

  學友們!如果\twnr{進入後住於}{66.0}不善法者在當生有樂住處,不惱害、不絕望、不熱惱,以身體的崩解,死後\twnr{善趣}{112.0}能被預期,世尊不會稱讚捨斷這不善法,學友們!但因為進入後住於不善法者在當生有苦的住處,有惱害、有絕望、有熱惱,以身體的崩解,死後\twnr{惡趣}{110.0}能被預期,因此世尊稱讚捨斷這不善法。

  學友們!如果進入後住於善法者當生有苦住處,有惱害、有絕望、有熱惱,以身體的崩解,死後惡趣能被預期,世尊不會稱讚這善法的具足,學友們!但因為進入後住於善法者當生有樂住處,不惱害、不絕望、不熱惱,以身體的崩解,死後善趣能被預期,因此世尊稱讚這善法的具足。」

  尊者舍利弗說這個,那些悅意的比丘歡喜尊者舍利弗所說。



\sutta{3}{3}{訶梨迪迦尼經}{https://agama.buddhason.org/SN/sn.php?keyword=22.3}
  \twnr{被我這麼聽聞}{1.0}:

  \twnr{有一次}{2.0},\twnr{尊者}{200.0}大迦旃延住在阿槃提拘拉拉迦拉之斷崖山。

  那時,\twnr{屋主}{103.0}訶梨迪迦尼去見尊者大迦旃延。抵達後,向尊者大迦旃延\twnr{問訊}{46.0}後,在一旁坐下。在一旁坐下的屋主訶梨迪迦尼對尊者大迦旃延說這個:

  「\twnr{大德}{45.0}!在這被\twnr{世尊}{12.0}說的馬更地亞所問\twnr{八群[經]}{x342}中:

  『捨斷家後成為\twnr{無住處的行者}{x343},\twnr{牟尼}{125.0}在村落中是不產生(作)親密交往者,

   從諸欲捨除者、\twnr{不期盼者}{x344},不會與人作爭論的談論。』[\ccchref{Ni.9}{https://agama.buddhason.org/Ni/dm.php?keyword=9}, 79段]

  大德!對這個被世尊簡要地說的義理,應該怎樣被詳細地看見呢?」

  「屋主!色界是識的家,還有,識有色界貪的繫縛,被稱為『[有]家的行者』;屋主!受界是識的家,還有,識有受界貪的繫縛,被稱為『家的行者』;屋主!想界是識的家,還有,識有想界貪的繫縛,被稱為『家的行者』;屋主!行界是識的家,還有,識有行界貪的繫縛,被稱為『家的行者』,屋主!這樣是家的行者。

  屋主!而怎樣是無家的行者?屋主!在色界上凡意欲,凡貪,凡歡喜,凡渴愛,凡攀住,凡執取,凡心的依處,凡執持,凡\twnr{煩惱潛在趨勢}{253.1}:這些被如來捨斷,根已被切斷,\twnr{[如]已斷根的棕櫚樹}{147.1},\twnr{成為非有}{408.0},\twnr{為未來不生之物}{229.0},因此,如來被稱為『無家的行者』。屋主!在受界上……屋主!在想界上……屋主!在行界上……屋主!在識界上凡意欲,凡貪,凡歡喜,凡渴愛,凡攀住,凡執取,凡心的依處,凡執持,凡煩惱潛在趨勢:這些被如來捨斷,根已被切斷,[如]已斷根的棕櫚樹,成為非有,為未來不生之物,因此,如來被稱為『無家的行者』。屋主!這樣是無家的行者。

  屋主!而怎樣是[有]住處的行者?屋主!\twnr{從色相住處的擴散與繫縛}{x345},被稱為『住處的行者』;聲音相……(中略)氣味相……味道相……\twnr{所觸}{220.2}相……從法相住處的擴散與繫縛,被稱為『住處的行者』,屋主!這樣是住處的行者。

  屋主!而怎樣是無住處的行者?屋主!從色相住處的擴散與繫縛被如來捨斷,根已被切斷,[如]已斷根的棕櫚樹,成為非有,為未來不生之物,因此,如來被稱為『無住處的行者』;聲音相……氣味相……味道相……所觸相……屋主!從法相住處的擴散與繫縛被如來捨斷,根已被切斷,[如]已斷根的棕櫚樹,成為非有,為未來不生之物,因此,如來被稱為『無住處的行者』。屋主!這樣是無住處的行者。

  屋主!而怎樣是在村落中生起親密交往者?屋主!這裡,某人住於被諸在家人交際,同歡、同愁:在諸樂者中成為樂者(在他們樂時成為樂者),在諸苦者中成為苦者,以自己在他們生起的應該被作的諸義務中\twnr{來到結合}{x346},屋主!這樣是在村落中生起親密交往者。

  屋主!而怎樣是在村落中不生起親密交往者?屋主!這裡,\twnr{比丘}{31.0}住於不被諸在家人交際,不同歡、不同愁:不在諸樂者中成為樂者,不在諸苦者中成為苦者,不以自己在他們生起的應該被作的諸義務中來到結合,屋主!這樣是在村落中不生起親密交往者。

  屋主!而怎樣是未從諸欲捨除者?屋主!這裡,某人在諸欲上是未\twnr{離貪}{77.0}者、未離意欲者、未離情愛者、未離渴望者、未離熱惱者、未離渴愛者,屋主!這樣是未從諸欲捨除者。

  屋主!而怎樣是從諸欲捨除者?屋主!這裡,某人在諸欲上是離貪者、離意欲者、離情愛者、離渴望者、離熱惱者、離渴愛者,屋主!這樣是從諸欲捨除者。

  屋主!而怎樣是期盼者?屋主!這裡,某人這麼想:『我\twnr{未來時}{308.0}會有這樣的色,我未來時會有這樣的受,我未來時會有這樣的想,我未來時會有這樣的行,我未來時會有這樣的識。』屋主!這樣是期盼者。

  屋主!而怎樣是不期盼者?屋主!這裡,某人不這麼想:『未來時會有這樣的色,我未來時會有這樣的受,我未來時會有這樣的想,我未來時會有這樣的行,我未來時會有這樣的識。』屋主!這樣是不期盼者。

  屋主!而怎樣是與人作爭論的談論者?屋主!這裡,某人是作這樣的談論者:『你不了知這法、律,我了知這法、律;你了知這法、律什麼!你是邪行者,我是正行者;應該先說的你後說,應該後說的你先說;我的是一致的,你的是不一致的;你長時間熟練的是顛倒的;你已被論破(你的理論已被反駁),請你去救(使脫離)理論;你已被折伏,或請你解開,如果你能夠。』屋主!這樣是與人作爭論的談論者。

  屋主!而怎樣是不與人作爭論的談論者?屋主!這裡,比丘不是作這樣的談論者:『你不了知這法、律……(中略)或請你解開,如果你能夠。』屋主!這樣是不與人作爭論的談論者。

  屋主!像這樣,凡這被世尊在馬更地亞所問八群[經]中說:

  『捨斷家後成為無住處的行者,牟尼在村落中是不產生(作)親密交往者,

   從諸欲捨除者、不期盼者,不會與人作爭論的談論。』

  屋主!對這個被世尊簡要地說的義理,應該這樣被詳細地看見。」



\sutta{4}{4}{訶梨迪迦尼經第二}{https://agama.buddhason.org/SN/sn.php?keyword=22.4}
  \twnr{被我這麼聽聞}{1.0}:

  \twnr{有一次}{2.0},\twnr{尊者}{200.0}大迦旃延住在阿槃提拘拉拉迦拉之斷崖山。

  那時,\twnr{屋主}{103.0}訶梨迪迦尼去見尊者大迦旃延。……(中略)在一旁坐下的屋主訶梨迪迦尼對尊者大迦旃延說這個:

  「\twnr{大德}{45.0}!在帝釋的問題中這被\twnr{世尊}{12.0}說:『凡那些\twnr{渴愛之滅盡解脫的}{834.0}\twnr{沙門}{29.0}、\twnr{婆羅門}{17.0},他們是\twnr{究竟終結者}{275.0}、究竟\twnr{軛安穩}{192.0}者、究竟梵行者、\twnr{究竟完結者}{819.0}、天-人們中最上者。』[\ccchref{DN.21}{https://agama.buddhason.org/DN/dm.php?keyword=21}, 366段]

  大德!對這個被世尊簡要地說的義理,應該怎樣被詳細地看見呢?」

  「屋主!在色界上凡意欲,凡貪,凡歡喜,凡渴愛,凡攀住,凡執取,凡心的依處,凡執持,凡\twnr{煩惱潛在趨勢}{253.1},從那些的滅盡、\twnr{褪去}{77.0}、滅、捨棄、\twnr{斷念}{211.0},心被稱為『\twnr{善解脫}{28.0}』。

  屋主!在受界上……屋主!在想界上……屋主!在行界上……屋主!在識界上凡意欲,凡貪,凡歡喜,凡渴愛,凡攀住,凡執取,凡心的依處,凡執持,凡煩惱潛在趨勢,從那些的滅盡、褪去、滅、捨棄、斷念,心被稱為『善解脫』。

  屋主!像這樣,凡在帝釋的問題中那被世尊說:『凡那些渴愛之滅盡解脫的沙門、婆羅門,他們是究竟終結者、究竟軛安穩者、究竟梵行者、究竟完結者、天-人們中最上者。』屋主!對這個被世尊簡要地說的義理,應該這樣被詳細地看見。」



\sutta{5}{5}{定經}{https://agama.buddhason.org/SN/sn.php?keyword=22.5}
  \twnr{被我這麼聽聞}{1.0}:

  \twnr{有一次}{2.0},\twnr{世尊}{12.0}住在舍衛城祇樹林給孤獨園。

  在那裡,世尊召喚\twnr{比丘}{31.0}們:「比丘們!」

  「\twnr{尊師}{480.0}!」那些比丘回答世尊。

  世尊說這個:

  「比丘們!你們要\twnr{修習}{94.0}\twnr{定}{182.0}。比丘們!\twnr{得定的}{x347}比丘如實知道,而如實知道什麼?色的\twnr{集起}{67.0}與滅沒,受的集起與滅沒,想的集起與滅沒,諸行的集起與滅沒,識的集起與滅沒。

  比丘們!而什麼是色的集起?什麼是受的集起?什麼是想的集起?什麼是諸行的集起?什麼是識的集起?

  比丘們!這裡,比丘歡喜、歡迎、持續固持。而歡喜、歡迎、持續固持什麼?歡喜、歡迎、持續固持色。當歡喜、歡迎、持續固持那個色時,歡喜生起。凡在色上有歡喜,那是取。\twnr{以那個的取為緣有有}{x348}(而有存在);以有為緣有生;以生為緣而老、死、愁、悲、苦、憂、\twnr{絕望}{342.0}生成,這樣是這整個\twnr{苦蘊}{83.0}的\twnr{集}{67.0}。

  歡喜受……(中略)歡喜想……歡喜諸行……歡喜、歡迎、持續固持識。當歡喜、歡迎、持續固持那個識時,歡喜生起。凡在識上有歡喜,那是取。以那個的取為緣有有;以有為緣有生;以生為緣……(中略)這樣是這整個苦蘊的集。

  比丘們!這是色的集起,這是受的集起,這是想的集起,這是諸行的集起,這是識的集起。

  比丘們!而什麼是色的滅沒?什麼是受的……什麼是想的……什麼是諸行的……什麼是識的滅沒?

  比丘們!這裡,不歡喜、不歡迎、不持續固持。而不歡喜、不歡迎、不持續固持什麼?不歡喜、不歡迎、不持續固持色。當不歡喜、不歡迎、不持續固持那個色時,凡在色上的那個歡喜被滅。以那個的歡喜滅有取滅;以取滅有有滅……(中略)這樣是這整個苦蘊的滅。

  不歡喜、不歡迎、不持續固持受。當不歡喜、不歡迎、不持續固持那個受時,凡在受上的那個歡喜被滅。以那個的歡喜滅有取滅;以取滅有有滅……(中略)這樣是這整個苦蘊的滅。不歡喜、不歡迎、不持續固持想……(中略)不歡喜、不歡迎、不持續固持諸行。當不歡喜、不歡迎、不持續固持那個諸行時,凡在諸行上的那個歡喜被滅。以那個的歡喜滅有取滅;以取滅有有滅……(中略)這樣是這整個苦蘊的滅。不歡喜、不歡迎、不持續固持識。當不歡喜、不歡迎、不持續固持那個識時。,凡在識上的那個歡喜被滅。以那個的歡喜滅有取滅;以取滅有有滅……(中略)這樣是這整個苦蘊的滅。

  比丘們!這是色的滅沒,這是受的滅沒,這是想的滅沒,這是諸行的滅沒,這是識的滅沒。」



\sutta{6}{6}{獨坐經}{https://agama.buddhason.org/SN/sn.php?keyword=22.6}
  起源於舍衛城。

  「\twnr{比丘}{31.0}們!你們要著手\twnr{努力}{739.0}於\twnr{獨坐}{92.0},比丘們!獨坐的比丘如實知道,如實知道什麼?色的\twnr{集起}{67.0}與滅沒、受的集起與滅沒、想的集起與滅沒、諸行的集起與滅沒、識的集起與滅沒。

  ……(中略)(應該依前經那樣使之被細說)。」



\sutta{7}{7}{由於執取的戰慄經}{https://agama.buddhason.org/SN/sn.php?keyword=22.7}
  起源於舍衛城。

  「\twnr{比丘}{31.0}們!我將為你們教導由於執取的\twnr{戰慄}{436.0},以及由於無執取的無戰慄,你們要聽它!你們\twnr{要好好作意}{43.1}!我將說。」

  「是的,\twnr{大德}{45.0}!」那些比丘回答\twnr{世尊}{12.0}。

  世尊說這個:

  「比丘們!而怎樣是由於執取的戰慄?

  比丘們!這裡,\twnr{未聽聞的一般人}{74.0}是聖者的未看見者,聖者法的不熟知者,在聖者法上未被教導者;是善人的未看見者,\twnr{善人法}{76.0}的不熟知者,在善人法上未被教導者,他\twnr{認為}{964.0}色是我,\twnr{或我擁有色}{13.0},或色在我中,\twnr{或我在色中}{14.0};他的那個色\twnr{變易成為不同的}{146.0},從色的變易變異狀態,\twnr{他的識成為隨色變易轉}{x349},隨色變易轉所生戰慄之法的生起\twnr{持續遍取}{530.0}他的心,由於(從)心的遍取成為有恐懼的、有惱害的、有期待的,執取後他戰慄。

  他認為受是我,或我擁有受,或受在我中,或我在受中;他的那個受變易成為不同的,從受的變易變異狀態,他的識成為隨受變易轉,隨受變易轉所生戰慄之法的生起持續遍取他的心,由於心的遍取成為有恐懼的、有惱害的、有期待的,執取後他戰慄。他認為想是我……(中略)他認為諸行是我,或我擁有諸行,或諸行在我中,或我在諸行中;他的那些行變易成為不同的,從行的變易變異狀態,他的識成為隨行變易轉,隨行變易轉所生戰慄之法的生起持續遍取他的心,由於心的遍取成為有恐懼的、有惱害的、有期待的,執取後他戰慄。他認為識是我,或我擁有識,或識在我中,或我在識中;他的那個識變易成為不同的,從識的變易變異狀態,他的識成為隨識變易轉,隨識變易轉所生戰慄之法的生起持續遍取他的心,由於心的遍取成為有恐懼的、有惱害的、有期待的,執取後他戰慄。

  比丘們!這樣是由於執取的戰慄。

  比丘們!而怎樣是由於無執取的無戰慄?

  比丘們!這裡,\twnr{有聽聞的聖弟子}{24.0}是聖者的看見者,聖者法的熟知者,在聖者法上被善教導者;是善人的看見者,善人法的熟知者,在善人法上被善教導者,他認為色不是我,或我不擁有色,或色不在我中,或我不在色中;他的那個色變易成為不同的,從色的變易變異狀態,他的識不成為隨色變易轉,無隨色變易轉所生戰慄之法的生起持續遍取他的心,由於心的無遍取成為無有恐懼的、無有惱害的、無有期待的,不執取後他不戰慄。

  他認為受不是我,或我不擁有受,或受不在我中,或我不在受中;他的那個受變易成為不同的,從受的變易變異狀態,他的識不成為隨受變易轉,無隨受變易轉所生戰慄之法的生起持續遍取他的心,由於心的無遍取成為無有恐懼的、無有惱害的、無有期待的,不執取後他不戰慄。他認為想不是我……(中略)他認為諸行不是我,或我不擁有諸行,或諸行不在我中,或我不在諸行中;他的那個行變易成為不同的狀態,從行的變易變異狀態,他的識不成為隨行變易轉,無隨行變易轉所生戰慄之法的生起持續遍取他的心,由於心的無遍取成為無有恐懼的、無有惱害的、無有期待的,不執取後他不戰慄。他認為識不是我,或我不擁有識……(中略);他的那個識變易成為不同的,從識的變易變異狀態,他的識不成為隨識變易轉,無隨識變易轉所生戰慄之法的生起持續遍取他的心,由於心的無遍取成為無有恐懼的、無有惱害的、無有期待的,不執取後他不戰慄。

  比丘們!這樣是由於無執取的無戰慄。」



\sutta{8}{8}{由於執取的戰慄經第二}{https://agama.buddhason.org/SN/sn.php?keyword=22.8}
  起源於舍衛城。

  「\twnr{比丘}{31.0}們!我將為你們教導由於執取的\twnr{戰慄}{436.0},以及由於無執取的無戰慄,\twnr{你們要聽}{43.0}它!……(中略)。」……

  「比丘們!而怎樣是由於執取的戰慄?

  比丘們!這裡,\twnr{未聽聞的一般人}{74.0}\twnr{認為}{964.0}色:『\twnr{這是我的}{32.0},\twnr{我是這個}{33.0},這是\twnr{我的真我}{34.0}。』他的那個色\twnr{變易成為不同的}{146.0},從色的變易變異,他的愁、悲、苦、憂、\twnr{絕望}{342.0}生起。

  他認為受:『這是我的……(中略)。』他認為想:『這是我的……。』他認為諸行:『這是我的……。』他認為識:『這是我的,我是這個,這是我的真我。』他的那個識變易成為不同的,從識的變易變異,他的愁、悲、苦、憂、絕望生起。

  比丘們!這樣是由於執取的戰慄。

  比丘們!而怎樣是由於無執取的無戰慄?

  比丘們!這裡,\twnr{有聽聞的聖弟子}{24.0}認為色:『\twnr{這不是我的}{32.1},\twnr{我不是這個}{33.1},\twnr{這不是我的真我}{34.2}。』他的那個色變易成為不同的,從色的變易變異,他的愁、悲、苦、憂、絕望不生起。

  他認為受:『這不是我的……。』他認為想:『這不是我的……。』他認為諸行:『這不是我的……。』他認為識:『這不是我的,我不是這個,這不是我的真我。』他的那個識變易成為不同的,從識的變易變異,他的愁、悲、苦、憂、絕望不生起。

  比丘們!這樣是由於無執取的無戰慄。」



\sutta{9}{9}{三時無常經}{https://agama.buddhason.org/SN/sn.php?keyword=22.9}
  起源於舍衛城。

  「\twnr{比丘}{31.0}們!過去、未來色是無常的,更不用說現在!比丘們!這麼看的\twnr{有聽聞的聖弟子}{24.0}在過去色上是\twnr{無期待者}{x350},\twnr{他不歡喜}{x351}未來色,對現在色是為了\twnr{厭}{15.0}、\twnr{離貪}{77.0}、\twnr{滅的行者}{519.0}。

  過去、未來受是無常的……(中略)過去、未來想是無常的……過去、未來諸行是無常的,更不用說現在!比丘們!這麼看的有聽聞的聖弟子在過去諸行上是無期待者,他不歡喜未來諸行,對現在諸行是為了厭、離貪、滅的行者。過去、未來識是無常的,更不用說現在!比丘們!這麼看的有聽聞的聖弟子在過去識上是無期待者,他不歡喜未來識,對現在識是為了厭、離貪、滅的行者。」



\sutta{10}{10}{三時苦經}{https://agama.buddhason.org/SN/sn.php?keyword=22.10}
  起源於舍衛城。

  「\twnr{比丘}{31.0}們!過去、未來色是苦的,更不用說現在!比丘們!這麼看的\twnr{有聽聞的聖弟子}{24.0}在過去色上是\twnr{無期待者}{x350},\twnr{他不歡喜}{x351}未來色,對現在色是為了\twnr{厭}{15.0}、\twnr{離貪}{77.0}、\twnr{滅的行者}{519.0}。

  過去、未來受是苦的……過去、未來想是苦的……過去、未來諸行是苦的……過去、未來識是苦的,更不用說現在!比丘們!這麼看的有聽聞的聖弟子在過去識上是無期待者,他不歡喜未來識,對現在識是為了厭、離貪、滅的行者。」





\sutta{11}{11}{三時無我經}{https://agama.buddhason.org/SN/sn.php?keyword=22.11}
  起源於舍衛城。

  「\twnr{比丘}{31.0}們!過去、未來色是無我,更不用說現在!比丘們!這麼看的\twnr{有聽聞的聖弟子}{24.0}在過去色上是\twnr{無期待者}{x350},\twnr{他不歡喜}{x351}未來色,對現在色是為了\twnr{厭}{15.0}、\twnr{離貪}{77.0}、\twnr{滅的行者}{519.0}。

  過去、未來受是無我……過去、未來想是無我……過去、未來諸行是無我……過去、未來識是無我,更不用說現在!比丘們!這麼看的有聽聞的聖弟子在過去識上是無期待者,他不歡喜未來識,對現在識是為了厭、離貪、滅的行者。」

  那拘羅的父親品第一,其\twnr{攝頌}{35.0}:

  「那拘羅的父親、天臂,以及還二則訶梨迪迦尼,

   定、獨坐,由執取的戰慄二則,

   過去未來現在,這被稱為品。」





\pin{無常品}{12}{21}
\sutta{12}{12}{無常經}{https://agama.buddhason.org/SN/sn.php?keyword=22.12}
  \twnr{被我這麼聽聞}{1.0}:

  在舍衛城,在那裡……(中略)。

  「\twnr{比丘}{31.0}們!色是無常的,受是無常的,想是無常的,諸行是無常的,識是無常的。

  比丘們!這麼看的\twnr{有聽聞的聖弟子}{24.0}在色上\twnr{厭}{15.0},也在受上厭,也在想上厭,也在諸行上厭,也在識上厭。厭者\twnr{離染}{558.0},從\twnr{離貪}{77.0}被解脫,在已解脫時,\twnr{有『[這是]解脫』之智}{27.0},他知道:『\twnr{出生已盡}{18.0},\twnr{梵行已完成}{19.0},\twnr{應該被作的已作}{20.0},\twnr{不再有此處[輪迴]的狀態}{21.1}。』」



\sutta{13}{13}{苦經}{https://agama.buddhason.org/SN/sn.php?keyword=22.13}
  起源於舍衛城。

  「\twnr{比丘}{31.0}們!色是苦的,受是苦的,想是苦的,諸行是苦的,識是苦的。這麼看的……(中略)他知道:『……\twnr{不再有此處[輪迴]的狀態}{21.1}。」



\sutta{14}{14}{無我經}{https://agama.buddhason.org/SN/sn.php?keyword=22.14}
  起源於舍衛城。

  「\twnr{比丘}{31.0}們!色是無我,受是無我,想是無我,諸行是無我,識是無我。

  比丘們!這麼看的\twnr{有聽聞的聖弟子}{24.0}在色上\twnr{厭}{15.0},也在受上厭,也在想上厭,也在諸行上厭,也在識上厭。厭者\twnr{離染}{558.0},從\twnr{離貪}{77.0}被解脫,在已解脫時,\twnr{有『[這是]解脫』之智}{27.0},他知道:『\twnr{出生已盡}{18.0},\twnr{梵行已完成}{19.0},\twnr{應該被作的已作}{20.0},\twnr{不再有此處[輪迴]的狀態}{21.1}。』」





\sutta{15}{15}{凡無常經}{https://agama.buddhason.org/SN/sn.php?keyword=22.15}
  起源於舍衛城。

  「\twnr{比丘}{31.0}們!色是無常的,凡是無常的,那個是苦的,凡是苦的,那個是無我,凡是無我,那個:『\twnr{這不是我的}{32.1},\twnr{我不是這個}{33.1},\twnr{這不是我的真我}{34.2}。』這樣,這個應該以正確之慧如實被看見。

  受是無常的,凡是無常的,那個是苦的,凡是苦的,那個是無我,凡是無我,那個:『這不是我的,我不是這個,這不是我的真我。』這樣,這個應該以正確之慧如實被看見。想是無常的……(中略)諸行是無常的……識是無常的,凡是無常的,那個是苦的,凡是苦的,那個是無我,凡是無我,那個:『這不是我的,我不是這個,這不是我的真我。』這樣,這個應該以正確之慧如實被看見。這麼看的……(中略)他知道:『……\twnr{不再有此處[輪迴]的狀態}{21.1}。』」



\sutta{16}{16}{凡苦者經}{https://agama.buddhason.org/SN/sn.php?keyword=22.16}
  起源於舍衛城。

  「\twnr{比丘}{31.0}們!色是苦的,凡是苦的,那是無我,凡是無我,那個:『\twnr{這不是我的}{32.1},\twnr{我不是這個}{33.1},\twnr{這不是我的真我}{34.2}。』這樣,這個應該以正確之慧如實被看見。

  受是苦的……想是苦的……諸行是苦的……識是苦的,凡是苦的,那是無我,凡是無我,那個:『這不是我的,我不是這個,這不是我的真我。』這樣,這個應該以正確之慧如實被看見。這麼看的……(中略)他知道:『……\twnr{不再有此處[輪迴]的狀態}{21.1}。』」



\sutta{17}{17}{凡無我者經}{https://agama.buddhason.org/SN/sn.php?keyword=22.17}
  起源於舍衛城。

  「\twnr{比丘}{31.0}們!色是無我,凡是無我,那個:『\twnr{這不是我的}{32.1},\twnr{我不是這個}{33.1},\twnr{這不是我的真我}{34.2}。』這樣,這個應該以正確之慧如實被看見。

  受是無我……想是無我……諸行是無我……識是無我,凡是無我,那個:『這不是我的,我不是這個,這不是我的真我。』這樣,這個應該以正確之慧如實被看見。這麼看的……(中略)他知道:『……\twnr{不再有此處[輪迴]的狀態}{21.1}。』」



\sutta{18}{18}{有因的無常經}{https://agama.buddhason.org/SN/sn.php?keyword=22.18}
  起源於舍衛城。

  「\twnr{比丘}{31.0}們!色是無常的,為了色生起的該因及該緣,那個也是無常的,比丘們!無常所生成的色,將從哪裡有常的?

  受是無常的,為了受生起的該因及該緣,那個也是無常的,比丘們!無常所生成的受,將從哪裡有常的?想是無常的……諸行是無常的,為了諸行生起的該因及該緣,那個也是無常的,比丘們!無常所生成的諸行,將從哪裡有常的?識是無常的,為了識生起的該因及該緣,那個也是無常的,比丘們!無常所生成的識,將從哪裡有常的?

  這麼看的……(中略)他知道:『……\twnr{不再有此處[輪迴]的狀態}{21.1}。』」



\sutta{19}{19}{有因的苦經}{https://agama.buddhason.org/SN/sn.php?keyword=22.19}
  起源於舍衛城。

  「\twnr{比丘}{31.0}們!色是苦的,凡為了色生起的該因及該緣,那個也是苦的,比丘們!苦所生成的色,將從哪裡有樂的? 

  受是苦的……想是苦的……諸行是苦的……識是苦的,凡為了識生起的該因及該緣,那個也是苦的,比丘們!苦所生成的識,從將從哪裡有樂的?

  這麼看的……(中略)他知道:『……\twnr{不再有此處[輪迴]的狀態}{21.1}。』」



\sutta{20}{20}{有因的無我經}{https://agama.buddhason.org/SN/sn.php?keyword=22.20}
  起源於舍衛城。

  「\twnr{比丘}{31.0}們!色是無我,為了色生起的該因及該緣,那個也是無我。比丘們!無我所生成的色,將從哪裡有我? 

  受是無我……想是無我……諸行是無我……識是無我,為了識生起的該因及該緣,那個也是無我。比丘們!無我所生成的識,將從哪裡有我?

  這麼看的……(中略)他知道:『……\twnr{不再有此處[輪迴]的狀態}{21.1}。』」



\sutta{21}{21}{阿難經}{https://agama.buddhason.org/SN/sn.php?keyword=22.21}
  在舍衛城的園林。

  那時,\twnr{尊者}{200.0}阿難去見\twnr{世尊}{12.0}。抵達後,向世尊\twnr{問訊}{46.0}後,在一旁坐下。在一旁坐下的尊者阿難對世尊說這個:

  「\twnr{大德}{45.0}!被稱為『滅、滅』,大德!哪些法的滅被稱為『滅』?」

  「阿難!色是無常的、\twnr{有為的}{90.0}、\twnr{緣所生的}{557.0}、\twnr{滅盡法}{273.1}、\twnr{消散法}{155.0}、\twnr{褪去}{77.0}法、\twnr{滅法}{68.1},它的滅被稱為『滅』。

  受是無常的、有為的、緣所生的、滅盡法、消散法、褪去法、滅法,它的滅被稱為『滅』。想……諸行是無常的、有為的、緣所生的、滅盡法、消散法、褪去法、滅法,它們的滅被稱為『滅』。識是無常的、有為的、緣所生的、滅盡法、消散法、褪去法、滅法,它的滅被稱為『滅』。阿難!這些法的滅被稱為『滅』。」

  無常品第二,其\twnr{攝頌}{35.0}:

  「無常、苦、無我,凡無常隨後三則,

   及有因的三說,與以阿難它們為十。」





\pin{負擔品}{22}{32}
\sutta{22}{22}{負擔經}{https://agama.buddhason.org/SN/sn.php?keyword=22.22}
  在舍衛城。

  在那裡……「\twnr{比丘}{31.0}們!我將為你們教導負擔、荷負擔者、\twnr{負擔的拿起}{x352}、負擔的放下,\twnr{你們要聽}{43.0}它!

  比丘們!而什麼是負擔?『五取蘊』應該被回答。哪五個?色取蘊、受取蘊、想取蘊、行取蘊、識取蘊,比丘們!這被稱為負擔。

  比丘們!而什麼是荷負擔者?『個人』應該被回答。凡這樣名這樣姓的這位\twnr{尊者}{200.0},比丘們!這被稱為荷負擔者。

  比丘們!而什麼是負擔的拿起?凡這個導致再有的、與歡喜及貪俱行的、\twnr{到處歡喜的}{96.2}渴愛,即:欲的渴愛、有的渴愛、\twnr{虛無的渴愛}{244.0},比丘們!這被稱為負擔的拿起。

  比丘們!而什麼是負擔的放下?凡正是那個渴愛的\twnr{無餘褪去與滅}{491.0}、捨棄、\twnr{斷念}{211.0}、解脫、無\twnr{阿賴耶}{391.0},比丘們!這被稱為放下負擔。」[≃\suttaref{SN.22.103}]

  \twnr{世尊}{12.0}說這個,說這個後,\twnr{善逝}{8.0}、\twnr{大師}{145.0}又更進一步說這個:

  「五蘊確實是負擔,而個人是荷負擔者,

   負擔的拿起是世間中的苦,負擔的放下是樂。

   放下重的負擔後,不拿起另一個負擔後,

   連根拔出渴愛後,成為無飢渴者、般涅槃者。」



\sutta{23}{23}{遍知經}{https://agama.buddhason.org/SN/sn.php?keyword=22.23}
  起源於舍衛城。

  「\twnr{比丘}{31.0}們!我將教導應該被遍知的諸法,以及\twnr{遍知}{154.0},\twnr{你們要聽}{43.0}它!

  比丘們!而什麼是應該被遍知的法?比丘們!色是應該被遍知的法,受是應該被遍知的法,想是應該被遍知的法,諸行是應該被遍知的法,識是應該被遍知的法,比丘們!這些被稱為應該被遍知的諸法。

  比丘們!而什麼是遍知?比丘們!凡貪的滅盡、瞋的滅盡、癡的滅盡,比丘們!這被稱為遍知。」



\sutta{24}{24}{證知經}{https://agama.buddhason.org/SN/sn.php?keyword=22.24}
  起源於舍衛城。

  「\twnr{比丘}{31.0}們!不\twnr{證知}{242.0}、不\twnr{遍知}{154.0}、\twnr{不離貪}{77.1}、不捨斷色者是對苦滅盡的不可能者。

  不證知、不遍知、不離貪、不捨斷受者是對苦滅盡的不可能者。不證知想者……不證知、不遍知、不離貪、不捨斷諸行者是對苦滅盡的不可能者。不證知、不遍知、不離貪、不捨斷識者是對苦滅盡的不可能者。

  比丘們!證知、遍知、離貪、捨斷色者是對苦滅盡的可能者。

  證知受者……想……諸行……證知、遍知、離貪、捨斷識者是對苦滅盡的可能者。」[≃\suttaref{SN.35.26}, \suttaref{SN.35.27}, \suttaref{SN.35.111}, \suttaref{SN.35.112}]



\sutta{25}{25}{意欲貪經}{https://agama.buddhason.org/SN/sn.php?keyword=22.25}
  起源於舍衛城。

  「\twnr{比丘}{31.0}們!凡在色上的意欲貪,你們要捨斷它,這樣,那個色必將被捨斷,根被切斷,\twnr{[如]已斷根的棕櫚樹}{147.1},\twnr{成為非有}{408.0},\twnr{為未來不生之物}{229.0}。

  凡在受上的意欲貪,你們要捨斷它,這樣,那個受必將被捨斷,根被切斷,[如]已斷根的棕櫚樹,成為非有,為未來不生之物。

  凡在想上的意欲貪,你們要捨斷它,這樣,那個想必將被捨斷,根被切斷,[如]已斷根的棕櫚樹,成為非有,為未來不生之物。

  凡在諸行上的意欲貪,你們要捨斷它,這樣,那些行必將被捨斷,根被切斷,[如]已斷根的棕櫚樹,成為非有,為未來不生之物。

  凡在識上的意欲貪,你們要捨斷它,這樣,那個識必將被捨斷,根被切斷,[如]已斷根的棕櫚樹,成為非有,為未來不生之物。」



\sutta{26}{26}{樂味經}{https://agama.buddhason.org/SN/sn.php?keyword=22.26}
  起源於舍衛城。

  「\twnr{比丘}{31.0}們!當就在我\twnr{正覺}{185.1}以前,還是未\twnr{現正覺}{75.0}的\twnr{菩薩}{186.0}時想這個:『什麼是色的\twnr{樂味}{295.0}呢?什麼是\twnr{過患}{293.0}?什麼是\twnr{出離}{294.0}?什麼是受的樂味,什麼是過患,什麼是出離?什麼是想的樂味,什麼是過患,什麼是出離?什麼是諸行的樂味,什麼是過患,什麼是出離?什麼是識的樂味,什麼是過患,什麼是出離?』

  比丘們!那個我想這個:『凡\twnr{緣於}{252.0}色,樂、喜悅生起,這是色的樂味;凡色是無常的、苦的、\twnr{變易法}{70.0},這是色的過患;凡在色上意欲貪的調伏、意欲貪的捨斷,這是色的出離。

  凡緣於受,樂、喜悅生起,這是受的樂味;凡受是無常的、苦的、變易法,這是受的過患;凡在受上意欲貪的調伏、意欲貪的捨斷,這是受的出離。凡緣於想……(中略)凡緣於諸行,樂、喜悅生起,這是行的樂味;凡諸行是無常的、苦的、變易法,這是諸行的過患;凡在諸行上意欲貪的調伏、意欲貪的捨斷,這是諸行的出離。凡緣於識,樂、喜悅生起,這是識的樂味;凡識是無常的、苦的、變易法,這是識的過患;凡在識上意欲貪的調伏、意欲貪的捨斷,這是識的出離。』

  比丘們!只要我不這樣如實證知這五取蘊的樂味為樂味、過患為過患、出離為出離,比丘們!我在包括天,在包括魔,在包括梵的世間;在包括沙門婆羅門,在包括天-人的\twnr{世代}{38.0}中,就不自稱『\twnr{已現正覺}{75.0}\twnr{無上遍正覺}{37.0}』。

  比丘們!但當我這樣如實證知這五取蘊的樂味為樂味、過患為過患、出離為出離,比丘們!那時,我在包括天,在包括魔,在包括梵的世間;在包括沙門婆羅門,在包括天-人的世代中,自稱『已現正覺無上遍正覺』。而且,我的\twnr{智與見}{433.0}生起:『我的解脫是不動搖的,這是最後的出生,現在,沒有\twnr{再有}{21.0}。』」



\sutta{27}{27}{樂味經第二}{https://agama.buddhason.org/SN/sn.php?keyword=22.27}
  起源於舍衛城。

  「\twnr{比丘}{31.0}們!\twnr{我曾行}{954.0}色的\twnr{樂味}{295.0}之遍求,凡色的樂味,我曾到達那個。色的樂味之所及,那個被我以慧善見。

  比丘們!我曾行色的\twnr{過患}{293.0}之遍求,凡色的過患,我曾到達那個。色的過患之所及,那個被我以慧善見。

  比丘們!我曾行色的\twnr{出離}{294.0}之遍求,凡色的出離,我曾到達那個。色的出離之所及,那個被我以慧善見。

  比丘們!我……受的……比丘們!我……想的……比丘們!我……諸行的……比丘們!我曾行識的樂味之遍求,凡識的樂味,我曾到達那個。識的樂味之所及,那個被我以慧善見。比丘們!我曾行識的過患之遍求,凡識的過患,我曾到達那個。識的過患之所及,那個被我以慧善見。比丘們!我曾行識的出離之遍求,凡識的出離,我曾到達那個。識的出離之所及,那個被我以慧善見。

  比丘們!只要我不如實證知這五取蘊的樂味為樂味、過患為過患、出離為出離……(中略)證知……。而且,我的\twnr{智與見}{433.0}生起:『我的解脫是不動搖的,這是最後的出生,現在,沒有\twnr{再有}{21.0}。』」



\sutta{28}{28}{樂味經第三}{https://agama.buddhason.org/SN/sn.php?keyword=22.28}
  起源於舍衛城。

  「\twnr{比丘}{31.0}們!如果沒有色的\twnr{樂味}{295.0},眾生不在色上\twnr{貪著}{x297}。比丘們!但因為有色的樂味,因此眾生在色上貪著。

  比丘們!如果沒有色的\twnr{過患}{293.0},眾生不在色上\twnr{厭}{15.0}。比丘們!但因為有色的過患,因此眾生在色上厭。

  比丘們!如果沒有色的\twnr{出離}{294.0},眾生不從色出離。比丘們!但因為有色的出離,因此眾生從色出離。

  比丘們!如果沒有受的……(中略)比丘們!如果沒有想的……比丘們!如果沒有諸行的出離,眾生不從諸行出離。比丘們!但因為有諸行的出離,因此眾生從諸行出離。比丘們!如果沒有識的樂味,眾生不在識上貪著。比丘們!但因為有識的樂味,因此眾生在識上貪著。比丘們!如果沒有識的過患,眾生不在識上厭。比丘們!但因為有識的過患,因此眾生在識上厭。比丘們!如果沒有識的出離,眾生不從識出離。比丘們!但因為有識的出離,因此眾生從識出離。

  比丘們!只要眾生不如實證知這五取蘊的樂味為樂味、過患為過患、出離為出離,比丘們!眾生就不從這包括天、魔、梵的世間;包括沙門婆羅門,包括天-人的\twnr{世代}{38.0}出離、離縛、脫離,\twnr{以離被限制之心}{555.0}而住。

  比丘們!但當眾生如實證知這五取蘊的樂味為樂味、過患為過患、出離為出離,比丘們!那時,眾生從這包括天、魔、梵的世間;包括沙門婆羅門,包括天-人的世代中出離、離縛、脫離,以離被限制之心而住。」[≃\suttaref{SN.14.33}, \suttaref{SN.35.17}, \suttaref{SN.35.18}]



\sutta{29}{29}{歡喜經}{https://agama.buddhason.org/SN/sn.php?keyword=22.29}
  起源於舍衛城。

  「\twnr{比丘}{31.0}們!凡歡喜色者,\twnr{他歡喜苦}{x353};凡歡喜苦者,我說:『他不從苦被解脫。』

  凡歡喜受者……凡歡喜想者……凡歡喜諸行者……凡歡喜識者,他歡喜苦;凡歡喜苦者,我說:『他不從苦被解脫。』

  比丘們!凡不歡喜色者,他不歡喜苦;凡不歡喜苦者,我說:『他從苦被解脫。』

  凡不歡喜受者……凡不歡喜想者……凡不歡喜諸行者……凡不歡喜識者,他不歡喜苦;凡不歡喜苦者,我說:『他從苦被解脫。』」[≃\suttaref{SN.14.35}, \suttaref{SN.35.19}, \suttaref{SN.35.20}]



\sutta{30}{30}{生起經}{https://agama.buddhason.org/SN/sn.php?keyword=22.30}
  起源於舍衛城。

  「\twnr{比丘}{31.0}們!凡色的\twnr{生起}{x354}、\twnr{存續}{x355}、生出、\twnr{顯現}{x356},這是苦的生起、諸病的存續、老死的顯現。

  凡受的……(中略)凡想的……(中略)凡諸行的……(中略)凡識的生起、存續、生出、顯現,這是苦的生起,諸病的存續,老死的顯現。

  比丘們!而凡色的\twnr{滅}{68.0}、\twnr{平息}{x357}、\twnr{滅沒}{x358},這是苦的滅、諸病的平息、老死的滅沒。

  凡受的……(中略)凡想的……(中略)凡諸行的……(中略)凡識的滅、平息、滅沒,這是苦的滅、諸病的平息、老死的滅沒。」



\sutta{31}{31}{痛苦之根經}{https://agama.buddhason.org/SN/sn.php?keyword=22.31}
  起源於舍衛城。

  「\twnr{比丘}{31.0}們!我將教導\twnr{痛苦}{752.0}與痛苦之根,\twnr{你們要聽}{43.0}它!

  比丘們!而什麼是痛苦?比丘們!色是痛苦,受是痛苦,想是痛苦,諸行是痛苦,識是痛苦,比丘們!這被稱為痛苦。

  比丘們!而什麼是痛苦之根?凡這個導致再有的、與歡喜及貪俱行的、\twnr{到處歡喜的}{96.2}渴愛,即:欲的渴愛、有的渴愛、\twnr{虛無的渴愛}{244.0},比丘們!這被稱為痛苦之根。」



\sutta{32}{32}{易壞的經}{https://agama.buddhason.org/SN/sn.php?keyword=22.32}
  起源於舍衛城。

  「\twnr{比丘}{31.0}們!我將教導\twnr{易壞的}{x359}與非易壞的,\twnr{你們要聽}{43.0}它!

  比丘們!而什麼是易壞的?什麼是非易壞的?

  比丘們!色是易壞的,凡它的\twnr{滅}{68.0}、平息、滅沒,這是非易壞的。

  受是易壞的,凡它的滅、平息、滅沒,這是非易壞的。想是易壞的……諸行是易壞的,凡它的滅、平息、滅沒,這是非易壞的。識是易壞的,凡它的滅、平息、滅沒,這是非易壞的。」

  負擔品第三,其\twnr{攝頌}{35.0}:

  「負擔、遍知、證知,意欲貪為第四的,

   樂味三說,歡喜為第八的,

   生起與痛苦之根,易壞的為第十一的。」





\pin{非你們的品}{33}{42}
\sutta{33}{33}{非你們的經}{https://agama.buddhason.org/SN/sn.php?keyword=22.33}
  起源於舍衛城。

  「\twnr{比丘}{31.0}們!凡非你們的,你們要捨斷它!它被捨斷,對你們將有利益、安樂。比丘們!而什麼是非你們的?

  比丘們!色是非你們的,你們要捨斷它!它被捨斷,對你們將有利益、安樂;受是非你們的,你們要捨斷它!它被捨斷,對你們將有利益、安樂;想非你們的……諸行是非你們的,你們要捨斷它們!捨斷它們,那些對你們將有長久的利益、安樂;識是非你們的,你們要捨斷它!它被捨斷,對你們將有利益、安樂。

  比丘們!猶如凡在這祇樹林中的草、薪木、枝條、樹葉,[某]人帶走它,或燃燒,或如需要做,是否你們這麼想:『[某]人帶走我們,或燃燒,或如需要做。』呢?」

  「\twnr{大德}{45.0}!這確實不是,那是什麼原因?大德!因為對我們這不是自己,或自己的。」

  「同樣的,比丘們!色是非你們的,你們要捨斷它!它被捨斷,對你們將有利益、安樂;受是非你們的,你們要捨斷它!它被捨斷,對你們將有利益、安樂;想非你們的……諸行非你們的……識是非你們的,你們要捨斷它!它被捨斷,對你們將有利益、安樂。」[≃\suttaref{SN.35.101}]



\sutta{34}{34}{非你們的經第二}{https://agama.buddhason.org/SN/sn.php?keyword=22.34}
  起源於舍衛城。

  「\twnr{比丘}{31.0}們!凡非你們的,你們要捨斷它!它被捨斷,對你們將有利益、安樂。比丘們!而什麼是非你們的?

  比丘們!色是非你們的,你們要捨斷它!它被捨斷,對你們將有利益、安樂;受非你們的……想非你們的……諸行非你們的……識是非你們的,你們要捨斷它!它被捨斷,對你們將有利益、安樂。

  比丘們!凡非你們的,你們要捨斷它!它被捨斷,對你們將有利益、安樂。」



\sutta{35}{35}{某位比丘經}{https://agama.buddhason.org/SN/sn.php?keyword=22.35}
  起源於舍衛城。

  那時,\twnr{某位比丘}{39.0}去見世尊。抵達後,向世尊\twnr{問訊}{46.0}後,在一旁坐下。在一旁坐下的那位比丘對世尊說這個:

  「\twnr{大德}{45.0}!請世尊為我簡要地教導法,凡我聽聞世尊的法後,會住於單獨的、隱離的、不放逸的、熱心的、自我努力的,\twnr{那就好了}{44.0}!」

  「比丘!凡\twnr{潛伏}{253.0}者,他以那個走到稱呼(\suttaref{SN.22.36});凡不潛伏者,他不以那個走到稱呼。」

  「已了知,世尊!已了知,\twnr{善逝}{8.0}!」

  「比丘!那麼,如怎樣你對被我簡要地說的,詳細地了知義理?」

  「大德!如果對色潛伏,他以那個走到稱呼;如果對受潛伏,他以那個走到稱呼;如果對想潛伏,他以那個走到稱呼;如果對諸行潛伏,他以那個走到稱呼;如果對識潛伏,他以那個走到稱呼。

  大德!如果對色不潛伏,他不以那個走到稱呼;如果對受……如果對想……如果對諸行……如果對識不潛伏,他不以那個走到稱呼。

  大德!我對這個被世尊簡要地說的,這樣詳細地了知義理。」

  「比丘!好!好!比丘!你對被我簡要地說的,詳細地了知義理,好!

  比丘!如果對色潛伏,他以那個走到稱呼;如果對受……如果對想……如果對諸行……如果對識潛伏,他以那個走到稱呼。

  比丘!如果對色不潛伏,他不以那個走到稱呼;如果對受……如果對想……如果對諸行……如果對識不潛伏,他不以那個走到稱呼。

  比丘!對這個被我簡要地說的,義理應該這樣被詳細地看見。」

  那時,那位比丘歡喜、\twnr{隨喜}{85.0}世尊所說後,從座位起來、向世尊問訊、\twnr{作右繞}{47.0}後,離開。

  那時,住於單獨的、隱離的、不放逸的、熱心的、自我努力的那位比丘不久就以證智自作證後,在當生中\twnr{進入後住於}{66.0}凡\twnr{善男子}{41.0}們為了利益正確地\twnr{從在家出家成為無家者}{48.0}的那個無上梵行結尾,他證知:「\twnr{出生已盡}{18.0},\twnr{梵行已完成}{19.0},\twnr{應該被作的已作}{20.0},\twnr{不再有此處[輪迴]的狀態}{21.1}。」然後那位比丘成為眾\twnr{阿羅漢}{5.0}之一。



\sutta{36}{36}{某位比丘經第二}{https://agama.buddhason.org/SN/sn.php?keyword=22.36}
  起源於舍衛城。

  那時,\twnr{某位比丘}{39.0}去見\twnr{世尊}{12.0}……(中略)在一旁坐下的那位比丘對世尊說這個:

  「\twnr{大德}{45.0}!請世尊為我簡要地教導法,凡我聽聞世尊的法後,會住於單獨的、隱離的、不放逸的、熱心的、自我努力的,\twnr{那就好了}{44.0}!」

  「比丘!凡\twnr{潛伏}{253.0}者,\twnr{他被推量}{x360};凡被推量者,\twnr{他以那個走到稱呼}{x361}。凡不潛伏者,不被推量;凡不被推量者,他不以那個走到稱呼。」

  「已了知,世尊!已了知,\twnr{善逝}{8.0}!」

  「比丘!那麼,如怎樣你對被我簡要地說的,詳細地了知義理?」

  「大德!如果對色潛伏,他被推量;凡被推量者,他以那個走到稱呼。

  如果對受潛伏者……如果對想潛伏者……如果對諸行潛伏者……如果對識潛伏者,他被推量;凡被推量者,他以那個走到稱呼。

  大德!如果對色不潛伏,他不被推量;凡不被推量者,他不以那個走到稱呼。

  如果對受不潛伏……如果對想不潛伏……如果對諸行不潛伏……如果對識不潛伏,他不被推量;凡不被推量者,他不以那個走到稱呼。

  大德!我對這個被世尊簡要地說的,這樣詳細地了知義理。」

  「比丘!好!好!比丘!你對被我簡要地說的,詳細地了知義理,好!

  比丘!如果對色潛伏,他被推量;凡被推量者,他以那個走到稱呼。

  比丘!如果對受潛伏者……比丘!如果對想潛伏者……比丘!如果對諸行潛伏者……比丘!如果對識潛伏者,他被推量;凡被推量者,他以那個走到稱呼。

  比丘!如果對色不潛伏,他不被推量;凡不被推量者,他不以那個走到稱呼。

  如果對受不潛伏……如果對想不潛伏……如果對諸行不潛伏……如果對識不潛伏,他不被推量;凡不被推量者,他不以那個走到稱呼。

  比丘!對這個被我簡要地說的,義理應該這樣被詳細地看見。」

  ……(中略)然後那位比丘成為眾\twnr{阿羅漢}{5.0}之一。



\sutta{37}{37}{阿難經}{https://agama.buddhason.org/SN/sn.php?keyword=22.37}
  起源於舍衛城。

  那時,\twnr{尊者}{200.0}阿難去見\twnr{世尊}{12.0}。抵達後,與世尊一起互相問候。交換應該被互相問候的友好交談後,在一旁坐下。世尊對在一旁坐下的尊者阿難說這個:

  「阿難!如果他們這麼問你:『阿難\twnr{學友}{201.0}!什麼法的生起被了知,消散被了知,已住立的變異被了知?』阿難!被這麼問,你應該怎麼回答?」

  「\twnr{大德}{45.0}!如果他們這麼問我:『阿難學友!什麼法的生起被了知,消散被了知,已住立的變異被了知?』大德!被這麼問,我應該這麼回答:『學友們!色的生起被了知,消散被了知,已住立的變異被了知;受的……想的……諸行的……識的生起被了知,消散被了知,已住立的變異被了知,學友們!這些法的生起被了知,消散被了知,已住立的變異被了知。』大德!被這麼問,我會這麼回答。」 

  「阿難!\twnr{好}{44.0}!好!阿難!色的生起被了知,消散被了知,已住立的變異被了知;受的……想的……諸行的……識的生起被了知,消散被了知,已住立的變異被了知。阿難!這些法的生起被了知,消散被了知,已住立的變異被了知。阿難!被這麼問,你應該這麼回答。」



\sutta{38}{38}{阿難經第二}{https://agama.buddhason.org/SN/sn.php?keyword=22.38}
  起源於舍衛城。

  \twnr{世尊}{12.0}對在一旁坐下的\twnr{尊者}{200.0}阿難說這個:

  「阿難!如果他們這麼問你:『阿難\twnr{學友}{201.0}!什麼法的生起曾被了知,消散曾被了知,已住立的變異曾被了知?什麼法的生起將被了知,消散將被了知,已住立的變異將被了知?什麼法的生起被了知,消散被了知,已住立的變異被了知?』阿難!被這麼問,你應該怎麼回答?」

  「\twnr{大德}{45.0}!如果他們這麼問我:『阿難學友!什麼法的生起曾被了知,消散曾被了知,已住立的變異曾被了知?什麼法的生起將被了知,消散將被了知,已住立的變異將被了知?什麼法的生起被了知,消散被了知,已住立的變異被了知?』

  大德!被這麼問,我會這麼回答:『學友們!凡已過去的、已被滅的、已變易的色,它的生起曾被了知,消散曾被了知,已住立的變異曾被了知;凡已過去的、已被滅的、已變易的受,它的生起曾被了知,消散曾被了知,已住立的變異曾被了知;凡……想……凡已過去的、已被滅的、已變易的諸行,它們的生起曾被了知,消散曾被了知,已住立的變異曾被了知;凡已過去的、已被滅的、已變易的識,它的生起曾被了知,消散曾被了知,已住立的變異曾被了知,學友們!這些法的生起曾被了知,消散曾被了知,已住立的變異曾被了知。

  學友們!凡未生起的、未顯現的色,它的生起將被了知,消散將被了知,已住立的變異將被了知;凡未生起的、未顯現的受,它的生起將被了知,消散將被了知,已住立的變異將被了知;凡……想……凡未生起的、未顯現的諸行,它們的生起將被了知,消散將被了知,已住立的變異將被了知;凡未生起的、未顯現的識,它的生起將被了知,消散將被了知,已住立的變異將被了知,學友們!這些法的生起將被了知,消散將被了知,已住立的變異將被了知。

  學友們!凡生起的、顯現的色,它的生起被了知,消散被了知,已住立的變異被了知;凡生起的、顯現的受……(中略)凡想……凡生起的、顯現的諸行,它們的生起被了知,消散被了知,已住立的變異被了知;凡生起的、顯現的識,它的生起被了知,消散被了知,已住立的變異被了知,學友們!這些法的生起被了知,消散被了知,已住立的變異被了知。』大德!被這麼問,我會這麼回答。」 

  「\twnr{好}{44.0}!好!阿難!

  阿難!凡已過去的、已被滅的、已變易的色,它的生起曾被了知,消散曾被了知,已住立的變異曾被了知;凡受……凡想……凡諸行……凡已過去的、已被滅的、已變易的識,它的生起曾被了知,消散曾被了知,已住立的變異曾被了知,阿難!這些法的生起曾被了知,消散曾被了知,已住立的變異曾被了知。

  阿難!凡未生起的、未顯現的色,它的生起將被了知,消散將被了知,已住立的變異將被了知;凡受……凡想……凡諸行……凡未生起的、未顯現的識,它的生起將被了知,消散將被了知,已住立的變異將被了知,阿難!這些法的生起將被了知,消散將被了知,已住立的變異將被了知。

  阿難!凡生起的、顯現的色,它的生起被了知,消散被了知,已住立的變異被了知;凡生起的、顯現的受……凡想……凡諸行……凡生起的、顯現的識,它的生起被了知,消散被了知,已住立的變異被了知,阿難!這些法的生起被了知,消散被了知,已住立的變異被了知。阿難!被這麼問,你應該這麼回答。」 



\sutta{39}{39}{隨法經}{https://agama.buddhason.org/SN/sn.php?keyword=22.39}
  起源於舍衛城。

  「\twnr{比丘}{31.0}們!這是\twnr{法、隨法行}{58.0}比丘的隨法:凡在色上應該住於多\twnr{厭}{15.0},在受上應該住於多厭,在想上應該住於多厭,在諸行上應該住於多厭,在識上應該住於多厭。

  凡在色上住於多厭者;在受上……在想上……在諸行上……在識上住於多厭者\twnr{遍知}{154.0}色……受……想……諸行……遍知識,那位遍知色者……受……想……諸行……遍知識者從色被釋放、從受被釋放、從想被釋放、從諸行被釋放,從識被釋放,從生、老、死、愁、悲、苦、憂、\twnr{絕望}{342.0}被釋放,我說:『從苦被釋放。」[\suttaref{SN.22.146}]



\sutta{40}{40}{隨法經第二}{https://agama.buddhason.org/SN/sn.php?keyword=22.40}
  起源於舍衛城。

  「\twnr{比丘}{31.0}們!這是\twnr{法、隨法行}{58.0}比丘的隨法:在色上應該住於隨看著無常……(中略)我說:『從苦被釋放。』」



\sutta{41}{41}{隨法經第三}{https://agama.buddhason.org/SN/sn.php?keyword=22.41}
  起源於舍衛城。

  「\twnr{比丘}{31.0}們!這是\twnr{法、隨法行}{58.0}比丘的隨法:在色上應該住於\twnr{隨看著}{59.0}苦……(中略)我說:『從苦被釋放。』」



\sutta{42}{42}{隨法經第四}{https://agama.buddhason.org/SN/sn.php?keyword=22.42}
  起源於舍衛城。

  「\twnr{比丘}{31.0}們!這是\twnr{法、隨法行}{58.0}比丘的隨法:凡在色上應該住於\twnr{隨看著}{59.0}無我,在受上……在想上……在諸行上……在識上應該住於隨看無我。

  凡在色上住於隨看無我者……(中略)\twnr{遍知}{154.0}色;受……想……諸行……遍知識。

  那位遍知色者……受……想……諸行……遍知識者從色被釋放、從受被釋放、從想被釋放、從諸行被釋放,從識被釋放,從生、老、死、愁、悲、苦、憂、\twnr{絕望}{342.0}被釋放,我說:『從苦被釋放。』」

  非你們的品第四,其\twnr{攝頌}{35.0}:

  「以非你們的二說,以比丘二則在後,

   與以阿難二說,以隨法兩對。」







\pin{以自己為島品}{43}{52}
\sutta{43}{43}{以自己為島經}{https://agama.buddhason.org/SN/sn.php?keyword=22.43}
  起源於舍衛城。

  「\twnr{比丘}{31.0}們!你們要住於\twnr{以自己為島}{537.0}、\twnr{以自己為歸依}{482.0},不以其他為歸依;以法為島、以法為歸依,不以其他為歸依。

  比丘們!對住於以自己為島、以自己為歸依,不以其他為歸依;以法為島、以法為歸依,不以其他為歸依者,\twnr{它應該被如理考察}{x362}:愁、悲、苦、憂、\twnr{絕望}{342.0}是什麼生的?\twnr{什麼發生的}{x363}?

  比丘們!而愁、悲、苦、憂、絕望是什麼生的?什麼發生的?

  比丘們!這裡,\twnr{未聽聞的一般人}{74.0}是聖者的未看見者,聖者法的不熟知者,在聖者法上未被教導者;是善人的未看見者,\twnr{善人法}{76.0}的不熟知者,在善人法上未被教導者,他\twnr{認為}{964.0}色是我,\twnr{或我擁有色}{13.0},或色在我中,\twnr{或我在色中}{14.0};他的那個色\twnr{變易而成為不同的}{146.0},從色的變易變異,他的愁、悲、苦、憂、絕望生起。

  他認為受是我,或我擁有受,或受在我中,或我在受中;他的那個受變易而成為不同的,從受的變易變異,他的愁、悲、苦、憂、絕望生起。他認為想是我……認為諸行是我……他認為識是我,或我擁有識,或識在我中,或我在識中;他的那個識變易而成為不同的,從識的變易變異,他的愁、悲、苦、憂、絕望生起。

  比丘們!但就知道色的無常性、變易、\twnr{褪去}{77.0}、滅後,『在之前連同現在色,一切色是無常的、苦、\twnr{變易法}{70.0}。』以正確之慧這樣如實看見這個者的凡愁、悲、苦、憂、絕望,它們被捨斷。從它們的捨斷,他不\twnr{戰慄}{436.0};無戰慄者\twnr{住於樂}{317.0};住於樂的比丘\twnr{被稱為『那部分到達涅槃者』}{x364}。

  比丘們!但就知道受的無常性、變易、褪去、滅後,『在之前連同現在受,一切受是無常的、苦、變易法。』以正確之慧這樣如實看見這個者的凡愁、悲、苦、憂、絕望,它們被捨斷。從它們的捨斷,他不戰慄;無戰慄者住於樂;住於樂的比丘被稱為『那部分到達涅槃者』。想……比丘們!但就知道諸行的無常性、變易、褪去、滅後,『在之前連同現在諸行,\twnr{一切行是無常的}{628.0}、苦、變易法。』以正確之慧這樣如實看見這個者的凡愁、悲、苦、憂、絕望,它們被捨斷。從它們的捨斷,他不戰慄;無戰慄者住於樂;住於樂的比丘被稱為『那部分到達涅槃者』。比丘們!但就知道識的無常性、變易、褪去、滅後,『在之前連同現在識,一切識是無常的、苦、變易法。』以正確之慧這樣如實看見這個者的凡愁、悲、苦、憂、絕望,它們被捨斷。從它們的捨斷,他不戰慄;無戰慄者住於樂;住於樂的比丘被稱為『那部分到達涅槃者』。」



\sutta{44}{44}{道跡經}{https://agama.buddhason.org/SN/sn.php?keyword=22.44}
  起源於舍衛城。

  「\twnr{比丘}{31.0}們!我將為你們教導導向\twnr{有身}{93.0}集道跡,以及導向有身\twnr{滅道跡}{69.0},\twnr{你們要聽}{43.0}它!

  比丘們!而什麼是導向有身集道跡?

  比丘們!這裡,\twnr{未聽聞的一般人}{74.0}是聖者的未看見者,聖者法的不熟知者,在聖者法上未被教導者;是善人的未看見者,\twnr{善人法}{76.0}的不熟知者,在善人法上未被教導者,他\twnr{認為}{964.0}色是我,\twnr{或我擁有色}{13.0},或色在我中,\twnr{或我在色中}{14.0}。受是我……想……諸行……他認為識是我,或我擁有識,或識在我中,或我在識中。

  比丘們!這被稱為『導向有身集道跡、導向有身集道跡』。比丘們!像這樣,這被稱為『導向苦集的觀察』,在這裡,這就是義理。

  比丘們!而什麼是導向有身滅道跡?

  比丘們!這裡,\twnr{有聽聞的聖弟子}{24.0}是聖者的看見者,聖者法的熟知者,在聖者法上被善教導者;是善人的看見者,善人法的熟知者,在善人法上被善教導者,他認為色不是我,或我不擁有色,或色不在我中,或我不在色中。受不是我……想不……諸行不……他認為識不是我,或我不擁有識,或識不在我中,或我不在識中。

  比丘們!這被稱為『導向有身滅道跡、導向有身滅道跡』。比丘們!像這樣,這被稱為『導向苦滅的觀察』,在這裡,這就是義理。」



\sutta{45}{45}{無常經}{https://agama.buddhason.org/SN/sn.php?keyword=22.45}
  起源於舍衛城。

  「\twnr{比丘}{31.0}們!色是無常的,凡是無常的,那個是苦的,凡是苦的,那個是無我,凡是無我,那個:『\twnr{這不是我的}{32.1},\twnr{我不是這個}{33.1},\twnr{這不是我的真我}{34.2}。』這樣,這個應該以正確之慧如實被看見。以正確之慧這樣如實看見這個者的心\twnr{離染}{558.0},不執取後從諸\twnr{漏}{188.0}被解脫。

  受是無常的……想……諸行……識是無常的,凡是無常的,那個是苦的,凡是苦的,那個是無我,凡是無我,那個:『這不是我的,我不是這個,這不是我的真我。』這樣,這個應該以正確之慧如實被看見。以正確之慧這樣如實看見這個者的心離染,不執取後從諸\twnr{漏}{188.0}被解脫。

  比丘們!如果比丘的心於色界已離染,不執取後從諸\twnr{漏}{188.0}被解脫;於受界……(中略)……於想界……於行界……比丘們!如果比丘的心於識界已離染,不執取後從諸\twnr{漏}{188.0}被解脫,以解脫狀態成為住止的;以住止狀態成為滿足的;以滿足狀態他不\twnr{戰慄}{436.0},無戰慄者\twnr{就自己證涅槃}{71.0},他知道:『\twnr{出生已盡}{18.0},\twnr{梵行已完成}{19.0},\twnr{應該被作的已作}{20.0},\twnr{不再有此處[輪迴]的狀態}{21.1}。』」



\sutta{46}{46}{無常經第二}{https://agama.buddhason.org/SN/sn.php?keyword=22.46}
  起源於舍衛城。

  「\twnr{比丘}{31.0}們!色是無常的,凡是無常的,那個是苦的,凡是苦的,那個是無我,凡是無我,那個:『\twnr{這不是我的}{32.1},\twnr{我不是這個}{33.1},\twnr{這不是我的真我}{34.2}。』這樣,這個應該以正確之慧如實被看見。 

  受是無常的……想是無常的……諸行是無常的……識是無常的,凡是無常的,那個是苦的,凡是苦的,那個是無我,凡是無我,那個:『這不是我的,我不是這個,這不是我的真我。』這樣,這個應該以正確之慧如實被看見。

  以正確之慧這樣如實看見這個者的\twnr{諸過去隨見}{x365}不存在。在諸過去隨見不存在時,諸未來隨見不存在。在諸未來隨見不存在時,強力取著不存在。在強力取著不存在時,在色上……在受上……在想上……在諸行上……在識上心\twnr{離染}{558.0},不執取後從諸\twnr{漏}{188.0}被解脫,以解脫狀態成為住止的;以住止狀態成為滿足的;以滿足狀態他不\twnr{戰慄}{436.0},無戰慄者\twnr{就自己證涅槃}{71.0},他知道:『\twnr{出生已盡}{18.0},\twnr{梵行已完成}{19.0},\twnr{應該被作的已作}{20.0},\twnr{不再有此處[輪迴]的狀態}{21.1}。』」



\sutta{47}{47}{認為經}{https://agama.buddhason.org/SN/sn.php?keyword=22.47}
  起源於舍衛城。

  「\twnr{比丘}{31.0}們!凡任何\twnr{沙門}{29.0}或\twnr{婆羅門}{17.0}認為\twnr{種種我的認為}{964.1},他們全認為五取蘊,或它們中之一。哪五個?

  比丘們!這裡,\twnr{未聽聞的一般人}{74.0}是聖者的未看見者,聖者法的不熟知者,在聖者法上未被教導者;是善人的未看見者,\twnr{善人法}{76.0}的不熟知者,在善人法上未被教導者,他認為色是我,\twnr{或我擁有色}{13.0},或色在我中,\twnr{或我在色中}{14.0};受……想……諸行……他認為識是我,或我擁有識,或識在我中,或我在識中。

  像這樣,這種認為連同『\twnr{我是}{894.0}』[觀念]對他來說是不離的。比丘們!又,在『我是』[觀念]不離時,\twnr{有五根的下生}{x366}:眼根、耳根、鼻根、舌根、身根。

  比丘們!有意,有諸法,有無明界。

  比丘們!對被\twnr{無明觸}{89.0}所生受接觸之未聽聞的一般人來說,他的『我是』存在,他的『\twnr{我是這個}{894.1}』也存在,他的『我將是』也存在,他的『我將不是』也存在,他的『我將是有色者』也存在,他的『我將是無色者』也存在,他的『我將是有想者』也存在,他的『我將是無想者』也存在,他的『我將是非想非非想者』也存在,比丘們!五根就在那裡仍存續。

  而在這裡,\twnr{有聽聞的聖弟子}{24.0}\twnr{無明}{207.0}被捨斷,明生起,對他來說,以無明的\twnr{褪去}{77.0},以明的生起,他的『我是』不存在,他的『我是這個』也不存在,『我將是』…『我將不是』……有色的……無色的……有想的……無想的……『我將是非想非非想者』也不存在。」



\sutta{48}{48}{蘊經}{https://agama.buddhason.org/SN/sn.php?keyword=22.48}
  起源於舍衛城。

  「\twnr{比丘}{31.0}們!我將教導五蘊與\twnr{五取蘊}{36.0},\twnr{你們要聽}{43.0}它!

  比丘們!而什麼是五蘊?

  比丘們!凡任何色:過去、未來、現在,或內、或外,或粗、或細,或下劣、或勝妙,或凡在遠處、在近處,這被稱為色蘊。

  凡任何受……(中略)凡任何想……凡任何諸行:過去、未來、現在,或內、或外,或粗、或細……(中略)這被稱為行蘊。凡任何識:過去、未來、現在,或內、或外,或粗、或細,或下劣、或勝妙,或凡在遠處、在近處,這被稱為識蘊。比丘們!這些被稱為五蘊。

  比丘們!而什麼是五取蘊?

  比丘們!凡任何色:過去、未來、現在……(中略)或凡在遠處、在近處,有\twnr{漏}{188.0}的、\twnr{與執取有關的}{551.0},這被稱為色取蘊。

  凡任何受……(中略)或凡在遠處、在近處,有漏的、與執取有關的,這被稱為受取蘊。凡任何想……(中略)或凡在遠處、在近處,有漏的、與執取有關的,這被稱為想取蘊。凡任何諸行……(中略)有漏的、與執取有關的,這被稱為行取蘊。凡任何識:過去、未來、現在……(中略)或凡在遠處、在近處,有漏的、與執取有關的,這被稱為識取蘊。比丘們!這些被稱為五取蘊。」



\sutta{49}{49}{輸屢那經}{https://agama.buddhason.org/SN/sn.php?keyword=22.49}
  \twnr{被我這麼聽聞}{1.0}:

  \twnr{有一次}{2.0},\twnr{世尊}{12.0}住在王舍城栗鼠飼養處的竹林中。

  那時,\twnr{屋主}{103.0}之子輸屢那去見世尊……(中略)世尊對在一旁坐下的屋主之子輸屢那說這個:

  「輸屢那!凡任何\twnr{沙門}{29.0}或\twnr{婆羅門}{17.0},對無常的、苦的、\twnr{變易法}{70.0}的色認為『我是優勝者』,或認為『我是同等者』,或認為『我是下劣者』,除了從不如實的見外還有什麼?

  對無常的、苦的、變易法的受認為『我是優勝者』,或認為『我是同等者』,或認為『我是下劣者』,除了從不如實的見外還有什麼?對無常的……想……對無常的、苦的、變易法的諸行認為『我是優勝者』,或認為『我是同等者』,或認為『我是下劣者』,除了從不如實的見外還有什麼?對無常的、苦的、變易法的識認為『我是優勝者』,或認為『我是同等者』,或認為『我是下劣者』,除了從不如實的見外還有什麼?

  輸屢那!凡任何沙門或婆羅門,對無常的、苦的、變易法的色不認為『我是優勝者』,也不認為『我是同等者』,也不認為『我是下劣者』,除了從如實的見外還有什麼?

  對無常的……受……對無常的……想……對無常的……諸行……對無常的、苦的、變易法的識不認為『我是優勝者』,也不認為『我是同等者』,也不認為『我是下劣者』,除了從如實的見外還有什麼?

  輸屢那!你怎麼想它:色是常的,或是無常的?」

  「無常的,\twnr{大德}{45.0}!」

  「那麼,凡為無常的,那是苦的或樂的?」

  「苦的,大德!」

  「那麼,凡為無常的、苦的、變易法,適合認為它:『\twnr{這是我的}{32.0},\twnr{我是這個}{33.0},這是\twnr{我的真我}{34.0}。』嗎?」

  「大德!這確實不是。」

  「受是常的,或是無常的?」

  「無常的,大德!」……

  「想……諸行……識是常的,或是無常的?」

  「無常的,大德!」

  「那麼,凡為無常的,那是苦的或樂的?」

  「苦的,大德!」

  「那麼,凡為無常的、苦的、變易法,適合認為它:『這是我的,我是這個,這是我的真我。』嗎?」

  「大德!這確實不是。」

  「輸屢那!因此,在這裡,凡任何色:過去、未來、現在,或內、或外,或粗、或細,或下劣、或勝妙,或凡在遠處、在近處,所有色:『\twnr{這不是我的}{32.1},\twnr{我不是這個}{33.1},\twnr{這不是我的真我}{34.2}。』這樣,這個應該以正確之慧如實被看見。

  凡任何受……凡任何想……凡任何諸行……凡任何識:過去、未來、現在,或內、或外,或粗、或細,或下劣、或勝妙,或凡在遠處、在近處,所有識:『這不是我的,我不是這個,這不是我的真我。』這樣,這個應該以正確之慧如實被看見。

  輸屢那!這麼看的\twnr{有聽聞的聖弟子}{24.0}在色上\twnr{厭}{15.0},也在受上厭,也在想上厭,也在諸行上厭,也在識上厭。厭者\twnr{離染}{558.0},從\twnr{離貪}{77.0}被解脫,在已解脫時,\twnr{有『[這是]解脫』之智}{27.0},他知道:『\twnr{出生已盡}{18.0},\twnr{梵行已完成}{19.0},\twnr{應該被作的已作}{20.0},\twnr{不再有此處[輪迴]的狀態}{21.1}。』」



\sutta{50}{50}{輸屢那經第二}{https://agama.buddhason.org/SN/sn.php?keyword=22.50}
  \twnr{被我這麼聽聞}{1.0}:

  \twnr{有一次}{2.0},\twnr{世尊}{12.0}住在王舍城栗鼠飼養處的竹林中。

  那時,\twnr{屋主}{103.0}之子輸屢那去見世尊。抵達後,向世尊\twnr{問訊}{46.0}後,在一旁坐下。世尊對在一旁坐下的屋主之子輸屢那說這個:

  「輸屢那!凡任何\twnr{沙門}{29.0}或\twnr{婆羅門}{17.0}不知道色,不知道色\twnr{集}{67.0},不知道色\twnr{滅}{68.0},不知道導向色\twnr{滅道跡}{69.0};不知道受,不知道受集,不知道受滅,不知道導向受滅道跡;不知道想……(中略);不知道諸行,不知道行集,不知道行滅,不知道導向行滅道跡;不知道識,不知道識集,不知道識滅,不知道導向識滅道跡者,輸屢那!那些沙門或婆羅門不被我認同為\twnr{沙門中的沙門}{560.0},或婆羅門中的婆羅門,而且,那些\twnr{尊者}{200.0}也不以證智自作證後,在當生中\twnr{進入後住於}{66.0}\twnr{沙門義}{327.0}或婆羅門義。

  輸屢那!但凡任何沙門或婆羅門知道色,知道色集,知道色滅,知道導向色滅道跡;知道受……(中略)知道想……知道諸行……知道識,知道識集,知道識滅,知道導向識滅道跡者,輸屢那!那些沙門或婆羅門被我認同為沙門中的沙門,或婆羅門中的婆羅門,而且,那些尊者也以證智自作證後,在當生中進入後住於沙門義或婆羅門義。」



\sutta{51}{51}{歡喜的滅盡經}{https://agama.buddhason.org/SN/sn.php?keyword=22.51}
  起源於舍衛城。

  「\twnr{比丘}{31.0}們!比丘\twnr{看}{x367}『無常的色』\twnr{只是無常的}{x368},那是他的\twnr{正見}{x369}。正確看見者\twnr{厭}{15.0},從歡喜的\twnr{滅盡}{273.0}有貪的滅盡;從貪的滅盡有歡喜的滅盡。從歡喜、貪的滅盡心被解脫,被稱為『\twnr{善解脫}{28.0}』。

  比丘們!比丘看『無常的受』只是無常的,那是他的正見,正確看見者厭,從歡喜的滅盡有貪的滅盡;從貪的滅盡有歡喜的滅盡。從歡喜、貪的滅盡心被解脫,被稱為『善解脫』。比丘們!比丘看『無常的想』只是無常的……(中略)比丘們!比丘看『無常的諸行』只是無常的,那是他的正見,正確看見者厭,從歡喜的滅盡有貪的滅盡;從貪的滅盡有歡喜的滅盡。從歡喜、貪的滅盡心被解脫,被稱為『善解脫』。比丘們!比丘看『無常的識』只是無常的,那是他的正見,正確看見者厭,從歡喜的滅盡有貪的滅盡;從貪的滅盡有歡喜的滅盡。從歡喜、貪的滅盡心被解脫,被稱為『善解脫』。」



\sutta{52}{52}{歡喜的滅盡經第二}{https://agama.buddhason.org/SN/sn.php?keyword=22.52}
  起源於舍衛城。

  「\twnr{比丘}{31.0}們!你們要\twnr{如理作意}{114.0}色,你們要如實察覺色的無常性。比丘們!當比丘如理作意色、如實察覺色的無常性時,在色上\twnr{厭}{15.0},從歡喜的\twnr{滅盡}{273.0}有貪的滅盡;從貪的滅盡有歡喜的滅盡。從歡喜、貪的滅盡心被解脫,被稱為『\twnr{善解脫}{28.0}』。

  比丘們!你們要如理作意受,你們要如實察覺受的無常性。比丘們!當比丘如理作意受、如實察覺受的無常性時,在受上厭,從歡喜的滅盡有貪的滅盡;從貪的滅盡有歡喜的滅盡。從歡喜、貪的滅盡心被解,這被稱為『善解脫』。比丘們!想……比丘們!你們要如理作意諸行,你們要如實察覺諸行的無常性。比丘們!當比丘如理作意諸行、如實察覺諸行的無常性時,在諸行上厭,從歡喜的滅盡有貪的滅盡;從貪的滅盡有歡喜的滅盡。從歡喜、貪的滅盡心被解,這被稱為『善解脫』。比丘們!你們要如理作意識,你們要如實察覺識的無常性。比丘們!當比丘如理作意識、如實察覺識的無常性時,在識上厭,從歡喜的滅盡有貪的滅盡;從貪的滅盡有歡喜的滅盡。從歡喜、貪的滅盡心被解,這被稱為『善解脫』。」

  以自己為島品第五,其\twnr{攝頌}{35.0}:

  「以自己為島、道跡,以及有二則無常性,

   認為、蘊,二則輸屢那與二則歡喜的滅盡。」

  根本五十則已完成。

  那個根本五十則的品之攝頌:

  「那拘羅的父親與無常,負擔與以非你們的,

   以自己為島有五十則,以那個被稱為第一的。」





\pin{攀住品}{53}{62}
\sutta{53}{53}{攀住經}{https://agama.buddhason.org/SN/sn.php?keyword=22.53}
  起源於舍衛城。

  「\twnr{比丘}{31.0}們!\twnr{攀住者}{x370}不被解脫,不攀住者被解脫。

  比丘們!當識住立時,或會住立在攀住的色、所緣的色、依止的色,有歡喜的澆灑,會來到增長、生長、成滿;或攀住的受……(中略)或攀住的想……(中略)比丘們!當識住立時,或會住立在攀住的行、所緣的行、依止的行,有歡喜的澆灑,會來到增長、生長、成滿。

  比丘們!凡如果這麼說:『除了色,除了受,除了想,除了諸行外,我將\twnr{安立}{143.0}識的或來或去,或沒或往生,或增長或生長或成滿。』\twnr{這不存在可能性}{650.0}。

  比丘們!如果在色界上比丘的貪已被捨斷,從貪的捨斷,所緣被切斷,識的依止(立足處)不存在。

  比丘們!如果在受界……比丘們!如果在想界……比丘們!如果在行界……比丘們!如果在識界上比丘的貪已被捨斷,從貪的捨斷,所緣被切斷,識的依止不存在。那個無安住處的識不被增長,\twnr{不造作}{751.0}後被解脫,以解脫狀態成為住止的;以住止狀態成為滿足的;以滿足狀態他不\twnr{戰慄}{436.0},無戰慄者\twnr{就自己證涅槃}{71.0},他知道:『\twnr{出生已盡}{18.0},\twnr{梵行已完成}{19.0},\twnr{應該被作的已作}{20.0},\twnr{不再有此處[輪迴]的狀態}{21.1}。』」



\sutta{54}{54}{種子經}{https://agama.buddhason.org/SN/sn.php?keyword=22.54}
  起源於舍衛城。

  「\twnr{比丘}{31.0}們!有這五類種子的種類,哪五類呢?\twnr{根種子}{x371}、\twnr{莖種子}{x372}、\twnr{枝種子}{x373}、\twnr{節種子}{x374},第五就是\twnr{種子種子}{x375}。

  比丘們!如果有不破碎的、不腐爛的、\twnr{無風吹日曬破壞的}{x376}、新成熟的、安全播下的這五類種子的種類、沒有地、沒有水,比丘們!是否這五類種子的種類會來到增長、生長、成滿呢?」

  「\twnr{大德}{45.0}!這確實不是。」

  「比丘們!如果有不被毀壞的……(中略)安全播下的這五類種子的種類、有地、有水,比丘們!是否這五類種子的種類會來到增長、生長、成滿呢?」

  「是的,大德!」

  「比丘們!\twnr{四識住}{x377}應該被看作如同地界;比丘們!\twnr{歡喜、貪}{x378}應該被看作如同水界;比丘們!\twnr{有食物的識}{x379}應該被看作如同五類種子的種類。

  比丘們!當識住立時,或會住立在\twnr{攀住}{x380}的色、所緣的色、依止的色,有歡喜的澆灑,會來到增長、生長、成滿;比丘們!當識住立時,或會住立在攀住的受……(中略)比丘們!當識住立時,或會住立在攀住的想……(中略)比丘們!當識住立時,或會住立在攀住的行、所緣的行、依止的行,有歡喜的澆灑,會來到增長、生長、成滿。

  比丘們!凡如果這麼說:『除了色,除了受,除了想,除了諸行外,我將\twnr{安立}{143.0}識的或來或去,或沒或往生,或增長或生長或成滿。』\twnr{這不存在可能性}{650.0}。

  比丘們!如果在色界上比丘的\twnr{貪已被捨斷}{x381},從貪的捨斷,所緣被切斷,識的依止(立足處)不存在。

  如果在受界……如果在想界……如果在行界……比丘們!如果在識界上比丘的貪已被捨斷,從貪的捨斷,所緣被切斷,識的依止不存在。\twnr{那個無安住處的識}{x382}不被增長,\twnr{不造作}{751.0}後被解脫,以解脫狀態成為\twnr{住止的}{x383};以住止狀態成為\twnr{滿足的}{x384};以滿足狀態他不\twnr{戰慄}{436.0},無戰慄者\twnr{就自己證涅槃}{71.0},他知道:『\twnr{出生已盡}{18.0},\twnr{梵行已完成}{19.0},\twnr{應該被作的已作}{20.0},\twnr{不再有此處[輪迴]的狀態}{21.1}。』」



\sutta{55}{55}{優陀那經}{https://agama.buddhason.org/SN/sn.php?keyword=22.55}
  起源於舍衛城。

  在那裡,\twnr{世尊}{12.0}吟出\twnr{優陀那}{184.0}:

  「『{我不會存在以及我的不會存在,如果我不存在,\twnr{我的將不存在}{616.0}}[彼不會存在以及我的不會存在,彼將不存在以及\twnr{我的將不存在}{595.0}]。』當這樣\twnr{勝解}{257.0}時,\twnr{比丘}{31.0}能切斷\twnr{下分結}{134.0}。」

  在這麼說時,某位比丘對世尊說這個:

  「\twnr{大德}{45.0}!那麼,如怎樣『{我不會存在以及我的不會存在,如果我不存在,我的將不存在}[彼不會存在以及我的不會存在,彼將不存在以及我的將不存在]。』當這樣勝解時,比丘能斷下分結呢?」 

  「比丘!這裡,\twnr{未聽聞的一般人}{74.0}是聖者的未看見者……(中略)在\twnr{善人法}{76.0}上未被教導者,他\twnr{認為}{964.0}色是我,\twnr{或我擁有色}{13.0},或色在我中,\twnr{或我在色中}{14.0};受……想……諸行……認為識是我,或我擁有識,或識在我中,或我在識中。

  他不如實知道無常的色為『無常的色』,不如實知道無常的受為『無常的受』,不如實知道無常的想為『無常的想』,不如實知道無常的諸行為『無常的諸行』,不如實知道無常的識為『無常的識』。不如實知道苦的色為『苦的色』……苦的受……苦的想……苦的諸行……不如實知道苦的識為『苦的識』。不如實知道無我色為『無我色』,不如實知道無我受為『無我受』,不如實知道無我想為『無我想』,不如實知道無我諸行為『無我諸行』,不如實知道無我識為『無我識』。不如實知道\twnr{有為的}{90.0}色為『有為的色』……有為的受……有為的想……有為的諸行……不如實知道有為的識為『有為的識』。不如實知道『色\twnr{將消失}{x385}』……受將消失……想將消失……諸行將消失,不如實知道『識將消失』。

  比丘!\twnr{有聽聞的聖弟子}{24.0}是聖者的看見者,聖者法的熟知者,在聖者法上被善教導者;是善人的看見者,善人法的熟知者,在善人法上被善教導者,他認為色不是我……(中略)受……想……諸行……認為識不是我。

  他如實知道無常的色為『無常的色』,無常的受……無常的想……無常的諸行……如實知道無常的識為『無常的識』。苦的色……(中略)苦的識……無我色……(中略)無我識……如實知道有為的識為『有為的識』。如實知道『色將消失』……受……想……諸行……如實知道『識將消失』。

  他從色的消失,從受的消失,從想的消失,從諸行的消失,從識的消失,比丘!這樣,『{我不會存在以及我的不會存在,如果我不存在,我的將不存在}[彼不會存在以及我的不會存在,彼將不存在以及我的將不存在]。』當這樣勝解時,比丘能切斷下分結。」「大德!當這樣勝解時,比丘能斷下分結。」

  「大德!那麼,怎樣知、怎樣見者有諸\twnr{漏}{188.0}的\twnr{直接的}{615.0}滅盡?」

  「比丘!這裡,未聽聞的一般人在不應該恐懼處來到恐懼,比丘!因為未聽聞的一般人有這個恐懼:『{我不會存在以及我的不會存在,如果我不存在,我的將不存在}[彼不會存在以及我的不會存在,彼將不存在以及我的將不存在]。』

  比丘,但有聽聞的聖弟子在不應該恐懼處不來到恐懼,比丘!因為有聽聞的聖弟子沒有這個恐懼:『{我不會存在以及我的不會存在,如果我不存在,我的將不存在}[彼不會存在以及我的不會存在,彼將不存在以及我的將不存在]。』

  比丘!當識住立時,或會住立在攀住的色、所緣的色、依止的色,有歡喜的澆灑,會來到增長、生長、成滿;比丘!或攀住的受……(中略)比丘!或攀住的想……(中略)比丘!當識住立時,或會住立在攀住的行、所緣的行、依止的行,有歡喜的澆灑,會來到增長、生長、成滿。

  比丘!凡如果這麼說:『除了色,除了受,除了想,除了諸行外,我將\twnr{安立}{143.0}識的或來或去,或沒或往生,或增長或生長或成滿。』\twnr{這不存在可能性}{650.0}。

  比丘!如果在色界上比丘的貪已被捨斷,從貪的捨斷,所緣被切斷,識的依止(立足處)不存在。比丘!如果在受界……比丘!如果在想界……比丘!如果在行界……比丘!如果在識界上比丘的貪已被捨斷,從貪的捨斷,所緣被切斷,識的依止不存在。那個無安住處的識不被增長,\twnr{不造作}{751.0}後被解脫,以解脫狀態成為住止的;以住止狀態成為滿足的;以滿足狀態他不\twnr{戰慄}{436.0},無戰慄者\twnr{就自己證涅槃}{71.0},他知道:『\twnr{出生已盡}{18.0}……(中略)\twnr{不再有此處[輪迴]的狀態}{21.1}。』」

  比丘!這樣知者、這樣見者有諸漏的直接滅盡。」



\sutta{56}{56}{取輪經}{https://agama.buddhason.org/SN/sn.php?keyword=22.56}
  起源於舍衛城。

  「\twnr{比丘}{31.0}們!有這些五取蘊,哪五個?色取蘊,受取蘊,想取蘊,行取蘊,識取蘊。

  比丘們!只要我不如實證知這五取蘊的\twnr{四輪}{x386},比丘們!我在包括天,在包括魔,在包括梵的世間;在包括沙門婆羅門,在包括天-人的\twnr{世代}{38.0}中,就不自稱『已\twnr{現正覺}{75.0}\twnr{無上遍正覺}{37.0}』。

  比丘們!但當我這樣如實證知這五取蘊的四輪,比丘們!那時,我在包括天……(中略)在包括天-人的……才自稱『已現正覺無上遍正覺』。

  哪四輪呢?

  我證知色,我證知色集,我證知色\twnr{滅}{68.0},我證知導向色\twnr{滅道跡}{69.0};受……想……諸行……我證知識,我證知識集,我證知識滅,我證知導向識滅道跡。

  比丘們!而什麼是色?\twnr{四大}{646.0}與四大之所造色,比丘們!這被稱為色。

  以食集而有色集,以食滅有色滅。

  這\twnr{八支聖道}{525.0}就是導向色滅道跡,即:正見……(中略)正定。

  比丘們!凡任何沙門或婆羅門這樣證知色、這樣證知色集、這樣證知色滅、這樣證知導向色滅道跡後,是對色為了\twnr{厭}{15.0}、\twnr{離貪}{77.0}、\twnr{滅的行者}{519.0},他們是\twnr{善行者}{518.0}。凡善行者,他們在這法、律中堅固站立。

  比丘們!而凡任何沙門或婆羅門這樣證知色……(中略)這樣證知導向色滅道跡後,以對色的厭、離貪、\twnr{滅}{68.0},不執取後成為解脫者,他們是\twnr{善解脫}{28.0}者。凡善解脫者,他們是完成者。凡完成者,對他們來說沒有輪迴的\twnr{安立}{143.0}。

  比丘們!而什麼是受?比丘們!有六類受:眼觸所生受、耳觸所生受、鼻觸所生受、舌觸所生受、身觸所生受、意觸所生受,比丘們!這被稱為受。

  以觸集而有受集,以觸滅有受滅。

  這八支聖道就是導向受滅道跡,即:正見……(中略)正定。

  比丘們!凡任何沙門或婆羅門這樣證知受、這樣證知受集、這樣證知受滅、這樣證知導向受滅道跡後,是對受為了厭、離貪、滅的行者,他們是善行者。凡善行者,他們在這法、律中堅固站立。

  比丘們!而凡任何沙門或婆羅門這樣證知受……(中略)這樣證知導向受滅道跡後……(中略)對他們來說沒有輪迴的安立。

  比丘們!而什麼是想?比丘們!有六類想:色想、聲想、氣味想、味道想、\twnr{所觸}{220.2}想、法想,比丘們!這被稱為想。

  以觸集而有想集,以觸滅有想滅。

  這八支聖道就是導向想滅道跡,即:正見……(中略)正定。……(中略)對他們來說沒有輪迴的安立。

  比丘們!而什麼是諸行?比丘們!有\twnr{六類思}{343.0}:色思、聲思、氣味思、味道思、所觸思、法思,比丘們!這些被稱為諸行。

  以觸集而有行集,以觸滅有行滅。

  這八支聖道就是導向行滅道跡,即:正見……(中略)正定。

  比丘們!凡任何沙門或婆羅門這樣證知諸行、這樣證知行集、這樣證知行滅、這樣證知導向行滅道跡後,是對諸行為了厭、離貪、滅的行者,他們是善行者。凡善行者,他們在這法、律中堅固站立。

  比丘們!而凡任何沙門或婆羅門這樣證知諸行、這樣證知行集、這樣證知行滅、這樣證知導向行滅道跡後,以對諸行的厭、離貪、滅,不執取後成為解脫者,他們是善解脫者。凡善解脫者,他們是完成者。凡完成者,對他們來說沒有輪迴的安立。

  比丘們!而什麼是識?比丘們!有六類識:眼識、耳識、鼻識、舌識、身識、意識,比丘們!這被稱為識。

  以名色集而有識集,以名色滅有識滅。

  這八支聖道就是導向識滅道跡,即:正見……(中略)正定。

  比丘們!凡任何沙門或婆羅門這樣證知識、這樣證知識集、這樣證知識滅、這樣證知導向識滅道跡後,是對識為了厭、離貪、滅的行者,他們是善行者。凡善行者,他們在這法、律中堅固站立。

  比丘們!而凡任何沙門或婆羅門這樣證知識、這樣證知識集、這樣證知識滅、這樣證知導向識滅道跡後,以對識的厭、離貪、滅,不執取後成為解脫者,他們是善解脫者。凡善解脫者,他們是完成者。凡完成者,對他們來說沒有輪迴的安立。」



\sutta{57}{57}{七處經}{https://agama.buddhason.org/SN/sn.php?keyword=22.57}
  起源於舍衛城。

  「\twnr{比丘}{31.0}們!\twnr{七處熟練的}{x387}、\twnr{三種考察的}{x388}比丘被稱為在這法、律中完全完成的『最上人』。

  比丘們!而怎樣比丘是七處熟練者呢?

  比丘們!這裡,比丘知道色,知道色集,知道色滅,知道導向色\twnr{滅道跡}{69.0},知道色的\twnr{樂味}{295.0},知道色的\twnr{過患}{293.0},知道色的\twnr{出離}{294.0}。

  知道受……想……諸行……知道識,知道識集,知道識滅,知道導向識滅道跡,知道識的樂味,知道識的過患,知道識的出離。

  比丘們!而什麼是色?\twnr{四大}{646.0}與四大之所造色,比丘們!這被稱為色。

  \twnr{以食集而有色集}{x389};以食滅有色滅。

  這\twnr{八支聖道}{525.0}就是導向色滅道跡,即:正見……(中略)正定。

  凡\twnr{緣於}{252.0}色生起樂、喜悅,這是色的樂味。

  凡色是無常的、苦的、\twnr{變易法}{70.0}者,這是色的過患。

  凡在色上意欲貪的調伏、意欲貪的捨斷,這是色的出離。

  比丘們!凡任何\twnr{沙門}{29.0}、\twnr{婆羅門}{17.0}這樣證知色、這樣證知色集、這樣證知色滅、這樣證知導向色滅道跡、這樣證知色的樂味、這樣證知色的過患、這樣證知色的出離後,是對色為了\twnr{厭}{15.0}、\twnr{離貪}{77.0}、\twnr{滅的行者}{519.0},他們是\twnr{善行者}{518.0}。凡善行者,他們在這法、律中堅固站立。

  比丘們!而凡任何沙門或婆羅門這樣證知色、這樣證知色集、這樣證知色滅、這樣證知導向色滅道跡、這樣證知色的樂味、這樣證知色的過患、這樣證知色的出離後,以對色的厭、離貪、滅,不執取後成為解脫者,他們是\twnr{善解脫}{28.0}者。凡善解脫者,他們是完成者。凡完成者,對他們來說沒有輪迴的\twnr{安立}{143.0}。

  比丘們!而什麼是受?比丘們!有六類受:眼觸所生受……(中略)意觸所生受,比丘們!這被稱為受。

  以觸集而有受集,以觸滅有受滅。

  這八支聖道就是導向受滅道跡,即:正見……(中略)正定。

  凡緣於受,樂、喜悅生起,這是受的樂味。

  凡受是無常的、苦的、變易法者,這是受的過患。

  凡在受上意欲貪的調伏、意欲貪的捨斷,這是受的出離。

  比丘們!凡任何沙門或婆羅門這樣證知受、這樣證知受集、這樣證知受滅、這樣證知導向受滅道跡、這樣證知受的樂味、這樣證知受的過患、這樣證知受的出離後,是對受為了厭、離貪、滅的行者,他們是善行者。凡善行者,他們在這法、律中堅固站立。

  比丘們!而凡任何沙門或婆羅門這樣證知受……(中略)對他們來說沒有輪迴的安立。

  比丘們!而什麼是想?比丘們!有六類想:色想、聲想、氣味想、味道想、\twnr{所觸}{220.2}想、法想,比丘們!這被稱為想。

  以觸集而有想集,以觸滅有想滅。

  這八支聖道就是導向想滅道跡,即:正見……(中略)正定。……(中略)對他們來說沒有輪迴的安立。

  比丘們!而什麼是諸行?比丘們!有\twnr{六類思}{343.0}:色思、聲思、氣味思、味道思、所觸思、法思,比丘們!這些被稱為諸行。

  以觸集而有行集,以觸滅有行滅。

  這八支聖道就是導向行滅道跡,即:正見……(中略)正定。

  凡緣於諸行生起樂、喜悅,這是諸行的樂味。

  凡諸行是無常的、苦的、變易法者,這是諸行的過患。

  凡在諸行上意欲貪的調伏、意欲貪的捨斷,這是諸行的出離。

  比丘們!凡任何沙門或婆羅門這樣證知諸行、這樣證知行集、這樣證知行滅、這樣證知導向行滅道跡……(中略)後,是對諸行為了厭、離貪、滅的行者,他們是善行者。凡善行者,他們在這法、律中堅固站立。……(中略)對他們來說沒有輪迴的安立。

  比丘們!而什麼是識?比丘們!有六類識:眼識、耳識、鼻識、舌識、身識、意識,比丘們!這被稱為識。

  以名色集而有識集,以名色滅有識滅。

  這八支聖道就是導向識滅道跡,即:正見……(中略)正定。

  凡緣於識生起樂、喜悅,這是識的樂味。

  凡識是無常的、苦的、變易法者,這是識的過患。

  凡在識上意欲貪的調伏、意欲貪的捨斷,這是識的出離。

  比丘們!凡任何沙門或婆羅門這樣證知識、這樣證知識集、這樣證知識滅、這樣證知導向識滅道跡、這樣證知識的樂味、這樣證知識的過患、這樣證知識的出離後,是對識為了厭、離貪、滅的行者,他們是善行者。凡善行者,他們在這法、律中堅固站立。

  比丘們!而凡任何沙門或婆羅門這樣證知識、這樣證知識集、這樣證知識滅、這樣證知導向識滅道跡、這樣證知識的樂味、這樣證知識的過患、這樣證知識的出離後,以對識的厭、離貪、滅,他們是善解脫者。凡善解脫者,他們是完成者。凡完成者,對他們來說沒有輪迴的安立。

  比丘們!這樣,比丘是七處熟練者。

  比丘們!而怎樣比丘是三種考察者呢?

  比丘們!這裡,比丘從界考察、從處考察、從\twnr{緣起}{225.0}考察。

  比丘們!這樣,比丘是三種考察者。

  比丘們!七處善巧的、三種考察的比丘被稱為在這法、律中完全完成的『最上人』。」



\sutta{58}{58}{遍正覺者經}{https://agama.buddhason.org/SN/sn.php?keyword=22.58}
  起源於舍衛城。

  「\twnr{比丘}{31.0}們!\twnr{如來}{4.0}、\twnr{阿羅漢}{5.0}、\twnr{遍正覺者}{6.0}以對色的\twnr{厭}{15.0}、\twnr{離貪}{77.0}、\twnr{滅}{68.0},不執取後成為解脫者,被稱為『遍正覺者』。

  比丘們!慧解脫比丘也以對色的厭、離貪、滅,不執取後成為解脫者,被稱為『慧解脫者』。

  比丘們!如來、阿羅漢、遍正覺者以對受的厭、離貪、滅,不執取後成為解脫者,被稱為『遍正覺者』。

  比丘們!慧解脫比丘也以對受的厭……(中略)被稱為『慧解脫者』。

  比丘們!如來、阿羅漢、遍正覺者以對想的……對諸行的……以對識的厭、離貪、滅,不執取後成為解脫者,被稱為『遍正覺者』。

  比丘們!慧解脫比丘也以對識的厭、離貪、滅,不執取後成為解脫者,被稱為『慧解脫者』。

  比丘們!在那裡,對如來、阿羅漢、遍正覺者與慧解脫比丘,什麼是差別,什麼是不同,什麼是區別?」「\twnr{大德}{45.0}!我們的法是\twnr{世尊}{12.0}為根本的、\twnr{世尊為導引的}{56.0}、世尊為依歸的,大德!就請世尊說明這個所說的義理,\twnr{那就好了}{44.0}!聽聞世尊的[教說]後,比丘們將會\twnr{憶持}{57.0}。」

  「比丘們!那樣的話,你們要聽!你們要\twnr{好好作意}{43.1}!我將說。」

  「是的,\twnr{大德}{45.0}!」那些比丘回答世尊。

  世尊說這個:

  「比丘們!如來、阿羅漢、遍正覺者是未生起道的使生起者,未出生道的使出生者,未宣說道的宣說者;是道的知者,道的熟練者,道的熟知者,比丘們!而現在弟子們住於道的隨行者,之後為具備者。

  比丘們!對如來、阿羅漢、遍正覺者與慧解脫比丘,這是差別,這是不同,這是區別。」



\sutta{59}{59}{無我相經}{https://agama.buddhason.org/SN/sn.php?keyword=22.59}
  \twnr{有一次}{2.0},\twnr{世尊}{12.0}住在波羅奈仙人墜落處的鹿林。

  在那裡,世尊召喚\twnr{五位一群的比丘們}{828.0}:「比丘們!」

  「\twnr{尊師}{480.0}!」那些比丘回答世尊。

  世尊說這個:

  「比丘們!色是無我。比丘們!因為,如果這個色是我,這個色不轉起疾病,以及在色上被得到:『令我的色是這樣;\twnr{令我的色不是這樣}{x390}。』比丘們!但因為色是無我,因此,色轉起疾病,也在色上不被得到:『令我的色是這樣;令我的色不是這樣。』

  受是無我。比丘們!因為,如果這個受是我,這個受不轉起疾病,以及在受上被得到:『令我的受是這樣;令我的受不是這樣。』比丘們!但因為受是無我,因此,受轉起疾病,也在受上不被得到:『令我的受是這樣;令我的受不是這樣。』想是無我……(中略)諸行是無我。比丘們!因為,如果這個諸行是我,這個諸行不轉起疾病,以及在諸行上被得到:『令我的諸行是這樣;令我的諸行不是這樣。』比丘們!但因為諸行是無我,因此,諸行轉起疾病,也在諸行上不被得到:『令我的諸行是這樣;令我的諸行不是這樣。』識是無我。比丘們!因為,如果這個識是我,這個識不轉起疾病,以及在識上被得到:『令我的識是這樣;令我的識不是這樣。』比丘們!但因為識是無我,因此,識轉起疾病,也在識上不被得到:『令我的識是這樣;令我的識不是這樣。』

  比丘們!你們怎麼想它:色是常的,或是無常的?」

  「無常的,\twnr{大德}{45.0}!」

  「那麼,凡為無常的,那是苦的或樂的?」

  「苦的,大德!」

  「那麼,凡為無常的、苦的、\twnr{變易法}{70.0},適合認為它:『\twnr{這是我的}{32.0},\twnr{我是這個}{33.0},這是\twnr{我的真我}{34.0}。』嗎?」

  「大德!這確實不是。」

  「受……想……諸行……識是常的,或是無常的?」

  「無常的,大德!」

  「那麼,凡為無常的,那是苦的或樂的?」

  「苦的,大德!」

  「那麼,凡為無常的、苦的、變易法,適合認為它:『這是我的,我是這個,這是我的真我。』嗎?」

  「大德!這確實不是。」

  「比丘們!因此,在這裡,凡任何色:過去、未來、現在,或內、或外,或粗、或細,或下劣、或勝妙,或凡在遠處、在近處,所有色:『\twnr{這不是我的}{32.1},\twnr{我不是這個}{33.1},\twnr{這不是我的真我}{34.2}。』這樣,這個應該以正確之慧如實被看見。

  凡任何受:過去、未來、現在,或內、或外……(中略)或凡在遠處、在近處,所有受:『這不是我的,我不是這個,這不是我的真我。』這樣,這個應該以正確之慧如實被看見。凡任何想……(中略)凡任何諸行:過去、未來、現在,或內、或外……(中略)或凡在遠處、在近處,所有諸行:『這不是我的,我不是這個,這不是我的真我。』這樣,這個應該以正確之慧如實被看見。凡任何識:過去、未來、現在,或內、或外,或粗、或細,或下劣、或勝妙,或凡在遠處、在近處,所有識:『這不是我的,我不是這個,這不是我的真我。』這樣,這個應該以正確之慧如實被看見。

  比丘們!這麼看的\twnr{有聽聞的聖弟子}{24.0}在色上\twnr{厭}{15.0},也在受上厭,也在想上厭,也在諸行上厭,也在識上厭。厭者\twnr{離染}{558.0},從\twnr{離貪}{77.0}被解脫,在已解脫時,\twnr{有『[這是]解脫』之智}{27.0},他知道:『\twnr{出生已盡}{18.0},\twnr{梵行已完成}{19.0},\twnr{應該被作的已作}{20.0},\twnr{不再有此處[輪迴]的狀態}{21.1}。』」

  世尊說這個,[那]群\twnr{悅意的}{x391}五比丘們歡喜世尊的所說。

  還有,\twnr{在當這個解說被說時}{136.0},不執取後五位一群的比丘們的心從諸\twnr{漏}{188.0}被解脫。



\sutta{60}{60}{摩訶里經}{https://agama.buddhason.org/SN/sn.php?keyword=22.60}
  \twnr{被我這麼聽聞}{1.0}:

  \twnr{有一次}{2.0},\twnr{世尊}{12.0}住在毘舍離大林重閣講堂。

  那時,離車族人摩訶里去見世尊……(中略)在一旁坐下的離車族人摩訶里對世尊說這個:

  「\twnr{大德}{45.0}!富蘭那迦葉這麼說:『對眾生的污染,沒有因沒有\twnr{緣}{180.0},眾生們無因無緣地被污染;對眾生的清淨,沒有因沒有緣,眾生們無因無緣地變成清淨。』這裡,世尊怎麼說?」

  「摩訶里!對眾生的污染,有因有緣,眾生們有因有緣地被污染;摩訶里!對眾生的清淨,有因有緣,眾生們有因有緣地變成清淨。」

  「大德!但對眾生的污染,什麼是因?什麼是緣?眾生們怎樣有因有緣地被污染呢?」

  「摩訶里!而如果這個色是一向苦的,已掉入苦的,已進入苦的,不被樂進入的,眾生們不在這個色上染著,摩訶里!但因為色是樂的,已掉入樂的,已進入樂的,不被苦進入的,因此眾生在色上染著;從貪染被結縛;從結縛被污染,摩訶里!對眾生的污染,這是因,這是緣,眾生們這樣有因有緣地被污染。

  摩訶里!而如果這個受是一向苦的,已掉入苦的,已進入苦的,不被樂進入的,眾生們不在這個受上染著,摩訶里!但因為受是樂的,已掉入樂的,已進入樂的,不被苦進入的,因此眾生在受上染著;從貪染被結縛;從結縛被污染,摩訶里!對眾生的污染,這是因,這是緣,眾生們也這樣有因有緣地被污染。摩訶里!而如果這個想……(中略)摩訶里!如果這個諸行是一向苦的,已掉入苦的,已進入苦的,不被樂進入的,眾生們不在這個諸行上染著,摩訶里!但因為諸行是樂的,已掉入樂的,已進入樂的,不被苦進入的,因此眾生在諸行上染著;從貪染被結縛;從結縛被污染,摩訶里!對眾生的污染,這是因,這是緣,眾生們也這樣有因有緣地被污染。摩訶里!如果這個識是一向苦的,已掉入苦的,已進入苦的,不被樂進入的,眾生們不在這個識上染著,摩訶里!但因為識是樂的,已掉入樂的,已進入樂的,不被苦進入的,因此眾生在識上染著;從貪染被結縛;從結縛被污染,摩訶里!對眾生的污染,這是因,這是緣,眾生們也這樣有因有緣地被污染。」

  「大德!但對眾生的清淨,什麼是因?什麼是緣?眾生們怎樣有因有緣地變成清淨呢?」

  「摩訶里!而如果這色是一向樂的,已掉入樂的,已進入樂的,不被苦進入的,眾生不在這個色上厭,摩訶里!但因為色是苦的,已掉入苦的,已進入苦的,不被樂進入的,因此眾生在色上厭。厭者離染,從\twnr{離貪}{77.0}變成清淨,摩訶里!對眾生的清淨,這是因,這是緣,眾生們這樣有因有緣地變成清淨。

  摩訶里!如果這個受是一向樂的……(中略)摩訶里!如果這個想是一向樂的……(中略)摩訶里!如果這個諸行是一向樂的……(中略)摩訶里!如果這個識是一向樂的,已掉入樂的,已進入樂的,不被苦進入的,眾生不在這個識上厭,摩訶里!但因為識是苦的,已掉入苦的,已進入苦的,不被樂進入的,因此眾生在識上厭。厭者離染,從離貪變成清淨,摩訶里!對眾生的清淨,這是因,這是緣,眾生們也這樣有因有緣地變成清淨。」



\sutta{61}{61}{燃燒經}{https://agama.buddhason.org/SN/sn.php?keyword=22.61}
  起源於舍衛城。  

  「\twnr{比丘}{31.0}們!色是燃燒的,受是燃燒的,想是燃燒的,諸行是燃燒的,識是燃燒的,比丘們!這麼看的\twnr{有聽聞的聖弟子}{24.0}在色上\twnr{厭}{15.0},也在受上……也在想上……也在諸行上……也在識上厭。厭者\twnr{離染}{558.0},從\twnr{離貪}{77.0}被解脫,在已解脫時,\twnr{有『[這是]解脫』之智}{27.0},他知道:『\twnr{出生已盡}{18.0},\twnr{梵行已完成}{19.0},\twnr{應該被作的已作}{20.0},\twnr{不再有此處[輪迴]的狀態}{21.1}。』」



\sutta{62}{62}{言語道經}{https://agama.buddhason.org/SN/sn.php?keyword=22.62}
  起源於舍衛城。 

  「\twnr{比丘}{31.0}們!有這三種\twnr{未被摻雜}{877.0}的\twnr{言語道}{788.0}、\twnr{名稱道}{789.0}、\twnr{安立道}{790.0}:在過去未被摻雜,不被摻雜,將不被摻雜,不被有智的\twnr{沙門}{29.0}\twnr{婆羅門}{17.0}非難,哪三種?比丘們!凡已過去、已滅、已變易的色,『曾是(單數過去式)』為它的名稱,『曾是』為它的稱呼,『曾是』為它的安立。『是(單數現在式)』不為它的名稱,『將是(單數未來式)』不為它的名稱。

  凡已過去、已滅、已變易的受,『曾是』為它的名稱,『曾是』為它的稱呼,『曾是』為它的安立。『是』不為它的名稱,『將是』不為它的名稱。

  凡想……凡已過去、已滅、已變易的諸行,『曾是(複數)』為它們的名稱,『曾是』為它們的稱呼,『曾是』為它們的安立。『是(複數)』不為它們的名稱,『將是(複數)』不為它們的名稱。

   凡已過去、已滅、已變易的識,『曾是』為它的名稱,『曾是』為它的稱呼,『曾是』為它的安立。『是』不為它的名稱,『將是』不為它的名稱。

  比丘們!凡未生的、未出現的色,『將是』為它的名稱,『將是』為它的稱呼,『將是』為它的安立。『是』不為它的名稱,『曾是』不為它的名稱。

  凡未生的、未出現的受,『將是』為它的名稱,『將是』為它的稱呼,『將是』為它的安立。『是』不為它的名稱,『曾是』不為它的名稱。

  凡想……凡未生的、未出現的諸行,『將是』為它們的名稱,『將是』為它們的稱呼,『將是』為它們的安立。『是』不為它們的名稱,『曾是』不為它們的名稱。

  凡未生的、未出現的識,『將是』為它的名稱,『將是』為它的稱呼,『將是』為它的安立。『是』不為它的名稱,『曾是』不為它的名稱。

  比丘們!凡已生的、已出現的色,『是』為它的名稱,『是』為它的稱呼,『是』為它的安立。『曾是』不為它的名稱,『將是』不為它的名稱。

  凡已生的、已出現的受,『是』為它的名稱,『是』為它的稱呼,『是』為它的安立。『曾是』不為它的名稱,『將是』不為它的名稱。

  凡想……凡已生的、已出現的諸行,『是(複數現在式)』為它們的名稱,『是』為它們的稱呼,『是』為它們的安立。『曾是(複數過去式)』不為它們的名稱,『將是(複數未來式)』不為它們的名稱。

  凡已生的、已出現的識,『是』為它的名稱,『是』為它的稱呼,『是』為它的安立。『曾是』不為它的名稱,『將是』不為它的名稱。

  比丘們!這些是三種未被摻雜的言語道、名稱道、安立道:在過去未被摻雜,不被摻雜,將不被摻雜,不被有智的沙門婆羅門非難。凡即使他們是\twnr{歐卡拉的瓦砂與巴聶}{926.0}的無因論者、\twnr{無作業論者}{695.0}、\twnr{虛無論者}{694.0},他們也不曾想這三種言語道、名稱道、安立道應該被呵責、應該被反駁,那是什麼原因?以斥責、侮辱、憤怒、指責的害怕。」

  攀住品第六,其\twnr{攝頌}{35.0}:

  「攀住、種子、優陀那,取輪,

   七處與正覺者,五、摩訶里、燃燒,

   以及以言語道為品。」





\pin{阿羅漢品}{63}{72}
\sutta{63}{63}{執取者經}{https://agama.buddhason.org/SN/sn.php?keyword=22.63}
  \twnr{被我這麼聽聞}{1.0}:

  \twnr{有一次}{2.0},\twnr{世尊}{12.0}住在舍衛城祇樹林給孤獨園。

  那時,\twnr{某位比丘}{39.0}去見世尊。抵達後,向世尊\twnr{問訊}{46.0}後,在一旁坐下。在一旁坐下的那位比丘對世尊說這個:

  「\twnr{大德}{45.0}!請世尊為我簡要地教導法,凡我聽聞世尊的法後,會住於單獨的、隱離的、不放逸的、熱心的、自我努力的,\twnr{那就好了}{44.0}!」

  「比丘!\twnr{執取者}{x392}被魔繫縛,不執取者被\twnr{波旬}{49.0}釋放。」

  「已了知,世尊!已了知,\twnr{善逝}{8.0}!」

  「比丘!那麼,如怎樣你對被我簡要地說的,詳細地了知義理?」

  「大德!執取色者被魔繫縛,不執取者被波旬釋放。執取受者被魔繫縛,不執取者被波旬釋放。想……諸行……執取識者被魔繫縛,不執取者被波旬釋放。

  大德!我對這個被世尊簡要地說的,這樣詳細地了知義理。」

  「比丘!好!好!比丘!你對被我簡要地說的,詳細地了知義理,好!

  比丘!執取色者被魔繫縛,不執取者被波旬釋放。受……想……諸行……執取識者被魔繫縛,不執取者被波旬釋放。

  比丘!對這個被我簡要地說的,義理應該這樣被詳細地看見。」

  那時,那位比丘歡喜、\twnr{隨喜}{85.0}世尊所說後,從座位起來、向世尊問訊、\twnr{作右繞}{47.0}後,離開。

  那時,住於單獨的、隱離的、不放逸的、熱心的、自我努力的那位比丘不久就以證智自作證後,在當生中\twnr{進入後住於}{66.0}凡\twnr{善男子}{41.0}們為了利益正確地\twnr{從在家出家成為無家者}{48.0}的那個無上梵行結尾,他證知:「\twnr{出生已盡}{18.0},\twnr{梵行已完成}{19.0},\twnr{應該被作的已作}{20.0},\twnr{不再有此處[輪迴]的狀態}{21.1}。」然後那位比丘成為眾\twnr{阿羅漢}{5.0}之一。



\sutta{64}{64}{思量者經}{https://agama.buddhason.org/SN/sn.php?keyword=22.64}
  起源於舍衛城。

  那時,\twnr{某位比丘}{39.0}……(中略)在一旁坐下的那位比丘對\twnr{世尊}{12.0}說這個:

  「\twnr{大德}{45.0}!請世尊為我簡要地教導法……(中略-[凡我聽聞世尊的法後,])能住於[單獨的、隱離的、不放逸的、]熱心的、自我努力的,\twnr{那就好了}{44.0}!」

  「比丘!\twnr{思量者}{963.0}被魔繫縛,不思量者被\twnr{波旬}{49.0}釋放。」

  「已了知,世尊!已了知,\twnr{善逝}{8.0}!」

  「比丘!那麼,如怎樣你對被我簡要地說的,詳細地了知義理?」

  「大德!思量色者被魔繫縛,不思量者被波旬釋放。受……想……諸行……思量識者被魔繫縛,不思量者被波旬釋放。

  大德!我對這個被世尊簡要地說的,這樣詳細地了知義理。」

  「比丘!好!好!比丘!你對被我簡要地說的,詳細地了知義理,好!

  比丘!思量色者被魔繫縛,不思量者被波旬釋放。受……想……諸行……思量識者被魔繫縛,不思量者被波旬釋放。

  比丘!對這個被我簡要地說的,義理應該這樣被詳細地看見。」

  ……(中略)然後那位比丘成為眾\twnr{阿羅漢}{5.0}之一。



\sutta{65}{65}{歡喜者經}{https://agama.buddhason.org/SN/sn.php?keyword=22.65}
  起源於舍衛城。

  那時,\twnr{某位比丘}{39.0}……(中略)在一旁坐下的那位比丘對\twnr{世尊}{12.0}說這個:

  「\twnr{大德}{45.0}!請世尊簡要地為我……(中略)能住於……自我努力的,\twnr{那就好了}{44.0}!」

  「比丘!\twnr{歡喜者}{x393}被魔繫縛,不歡喜者被\twnr{波旬}{49.0}釋放。」

  「已了知,世尊!已了知,\twnr{善逝}{8.0}!」

  「比丘!那麼,如怎樣你對被我簡要地說的,詳細地了知義理?」

  「大德!歡喜色者被魔繫縛,不歡喜者被波旬釋放。受……想……諸行……歡喜識者被魔繫縛,不歡喜者被波旬釋放。

  大德!我對這個被世尊簡要地說的,這樣詳細地了知義理。」

  「比丘!好!好!比丘!你對被我簡要地說的,詳細地了知義理,好!

  比丘!歡喜色者被魔繫縛,不歡喜者被波旬釋放。受……想……諸行……歡喜識者被魔繫縛,不歡喜者被波旬釋放。

  比丘!對這個被我簡要地說的,義理應該這樣被詳細地看見。」

  ……(中略)然後那位比丘成為眾\twnr{阿羅漢}{5.0}之一。





\sutta{66}{66}{無常經}{https://agama.buddhason.org/SN/sn.php?keyword=22.66}
  起源於舍衛城。

  那時,\twnr{某位比丘}{39.0}……(中略)在一旁坐下的那位比丘對\twnr{世尊}{12.0}說這個:

  「\twnr{大德}{45.0}!請世尊為我簡要地教導法……(中略)能住於……熱心的、自我努力的,\twnr{那就好了}{44.0}!」

  「比丘!凡是無常的,在那裡意欲應該被你捨斷。」

  「已了知,世尊!已了知,\twnr{善逝}{8.0}!」

  「比丘!那麼,如怎樣你對被我簡要地說的,詳細地了知義理?」

  「大德!色是無常的,在那裡意欲應該被我捨斷;受……想……諸行……識是無常的,在那裡意欲應該被我捨斷。大德!我對這個被世尊簡要地說的,這樣詳細地了知義理。」

  「比丘!好!好!比丘!你對被我簡要地說的,詳細地了知義理,好!

  比丘!色是無常的,在那裡意欲應該被你捨斷;受是無常的……想……諸行……識是無常的,在那裡意欲應該被你捨斷。比丘!對這個被我簡要地說的,義理應該這樣被詳細地看見。」

  ……(中略)然後那位比丘成為眾\twnr{阿羅漢}{5.0}之一。



\sutta{67}{67}{苦經}{https://agama.buddhason.org/SN/sn.php?keyword=22.67}
  起源於舍衛城。

  那時,\twnr{某位比丘}{39.0}……(中略)在一旁坐下的那位比丘對\twnr{世尊}{12.0}說這個:

  「\twnr{大德}{45.0}!請世尊為我簡要地教導法……(中略)能住於……熱心的、自我努力的,\twnr{那就好了}{44.0}!」

  「比丘!凡是苦的,在那裡意欲應該被你捨斷。」

  「已了知,世尊!已了知,\twnr{善逝}{8.0}!」

  「比丘!那麼,如怎樣你對被我簡要地說的,詳細地了知義理?」

  「大德!色是苦的,在那裡意欲應該被我捨斷;受……想……諸行……識是苦的,在那裡意欲應該被我捨斷,大德!我對這個被世尊簡要地說的,這樣詳細地了知義理。」

  「比丘!好!好!比丘!你對被我簡要地說的,詳細地了知義理,好!

  比丘!色是苦的,在那裡意欲應該被你捨斷;受……想……諸行……識是苦的,在那裡意欲應該被你捨斷,比丘!對這個被我簡要地說的,義理應該這樣被詳細地看見。」

  ……(中略)然後那位比丘成為眾\twnr{阿羅漢}{5.0}之一。





\sutta{68}{68}{無我經}{https://agama.buddhason.org/SN/sn.php?keyword=22.68}
  起源於舍衛城。

  那時,\twnr{某位比丘}{39.0}……(中略)在一旁坐下的那位比丘對\twnr{世尊}{12.0}說這個:

  「\twnr{大德}{45.0}!請世尊為我簡要地教導法……(中略-[凡我聽聞世尊的法後,])能住於[單獨的、隱離的、不放逸的、]熱心的、自我努力的,\twnr{那就好了}{44.0}!」

  「比丘!凡是\twnr{無我}{23.0}者,在那裡意欲應該被你捨斷。」

  「已了知,世尊!已了知,\twnr{善逝}{8.0}!」

  「比丘!那麼,如怎樣你對被我簡要地說的,詳細地了知義理?」

  「大德!色是無我,在那裡意欲應該被我捨斷;受……想……諸行……識是無我,在那裡意欲應該被我捨斷,大德!我對這個被世尊簡要地說的,這樣詳細地了知義理。」

  「比丘!好!好!比丘!你對被我簡要地說的,詳細地了知義理,好!

  比丘!色是無我,在那裡意欲應該被你捨斷;受……想……諸行……識是無我,在那裡意欲應該被你捨斷,比丘!對這個被我簡要地說的,義理應該這樣被詳細地看見。」

  ……(中略)然後那位比丘成為眾\twnr{阿羅漢}{5.0}之一。



\sutta{69}{69}{非自己的經}{https://agama.buddhason.org/SN/sn.php?keyword=22.69}
  起源於舍衛城。

  那時,\twnr{某位比丘}{39.0}……(中略)在一旁坐下的那位比丘對\twnr{世尊}{12.0}說這個:

  「\twnr{大德}{45.0}!請世尊為我簡要地教導法……(中略)能住於……\twnr{那就好了}{44.0}!」

  「比丘!凡是非自己的,在那裡意欲應該被你捨斷。」

  「已了知,世尊!已了知,\twnr{善逝}{8.0}!」

  「比丘!那麼,如怎樣你對被我簡要地說的,詳細地了知義理?」

  「大德!色是非自己的,在那裡意欲應該被我捨斷;受……想……諸行……識是非自己的,在那裡意欲應該被我捨斷,大德!我對這個被世尊簡要地說的,這樣詳細地了知義理。」

  「比丘!好!好!比丘!你對被我簡要地說的,詳細地了知義理,好!

  比丘!色是非自己的,在那裡意欲應該被你捨斷;受……想……諸行……識是非自己的,在那裡意欲應該被你捨斷,比丘!對這個被我簡要地說的,義理應該這樣被詳細地看見。」

  ……(中略)然後那位比丘成為眾\twnr{阿羅漢}{5.0}之一。



\sutta{70}{70}{會被染住立的經}{https://agama.buddhason.org/SN/sn.php?keyword=22.70}
  起源於舍衛城。

  那時,\twnr{某位比丘}{39.0}……(中略)在一旁坐下的那位比丘對\twnr{世尊}{12.0}說這個:
  「大德!請世尊為我簡要地教導法凡我聽聞世尊的法後,能住於……(中略-[單獨的、隱離的、不放逸的、熱心的、自我努力的\twnr{那就好了}{44.0}!]」

  「比丘!凡是\twnr{會被染住立的}{x394},在那裡意欲應該被你捨斷。」

  「已了知,世尊!已了知,\twnr{善逝}{8.0}!」

  「比丘!那麼,如怎樣你對被我簡要地說的,詳細地了知義理?」

  「大德!色是會被染住立的,在那裡意欲應該被我捨斷;受……想……諸行……識是會被染住立的,在那裡意欲應該被我捨斷,大德!我對這個被世尊簡要地說的,這樣詳細地了知義理。」

  「比丘!好!好!比丘!你對被我簡要地說的,詳細地了知義理,好!

  比丘!色是會被染住立的,在那裡意欲應該被你捨斷;受……想……諸行……識是會被染住立的,在那裡意欲應該被你捨斷,比丘!對這個被我簡要地說的,義理應該這樣被詳細地看見。」

  ……(中略)然後那位比丘成為眾\twnr{阿羅漢}{5.0}之一。



\sutta{71}{71}{羅陀經}{https://agama.buddhason.org/SN/sn.php?keyword=22.71}
  起源於舍衛城。

  那時,\twnr{尊者}{200.0}羅陀去見\twnr{世尊}{12.0}。抵達後,對世尊說這個:

  「\twnr{大德}{45.0}!當怎樣知、當怎樣見時,在這個有識之身上\twnr{與在一切外部諸相上}{559.0}沒有\twnr{我作}{22.0}、\twnr{我所作}{25.0}、\twnr{慢煩惱潛在趨勢}{26.0}?」

  「羅陀!凡任何色:過去、未來、現在,或內、或外,或粗、或細,或下劣、或勝妙,或凡在遠處、在近處,所有色:『\twnr{這不是我的}{32.1},\twnr{我不是這個}{33.1},\twnr{這不是我的真我}{34.2}。』以正確之慧這樣如實看見這個。

  凡任何受……凡任何想……凡任何諸行……凡任何識:過去、未來、現在……或凡在遠處、在近處,所有識:『這不是我的,我不是這個,這不是我的真我。』以正確之慧這樣如實看見這個。

  羅陀!當這樣知、這樣見時,在這個有識之身上與在一切外部諸相上沒有我作、我所作、慢煩惱潛在趨勢。」[\suttaref{SN.18.21}]

  ……(中略)然後尊者羅陀成為眾\twnr{阿羅漢}{5.0}之一。



\sutta{72}{72}{蘇臘達經}{https://agama.buddhason.org/SN/sn.php?keyword=22.72}
  起源於舍衛城。

  那時,\twnr{尊者}{200.0}蘇臘達對\twnr{世尊}{12.0}說這個:

  「\twnr{大德}{45.0}!當怎樣知、當怎樣見時,在這個有識之身上\twnr{與在一切外部諸相上}{559.0}離\twnr{我作}{22.0}、\twnr{我所作}{25.0}、慢,成為心意超越\twnr{慢類}{647.0}者、寂靜者、\twnr{善解脫}{28.0}者呢?」

  「蘇臘達!凡任何色:過去、未來、現在……(中略)或凡在遠處、在近處,所有色:『\twnr{這不是我的}{32.1},\twnr{我不是這個}{33.1},\twnr{這不是我的真我}{34.2}。』以正確之慧這樣如實看見這個後,不執取後成為解脫者。

  凡任何受……(中略)凡任何想……凡任何諸行……凡任何識:過去、未來、現在,或內、或外,或粗、或細,或下劣、或勝妙,或凡在遠處、在近處,所有識:『這不是我的,我不是這個,這不是我的真我。』以正確之慧這樣如實看見這個後,不執取後成為解脫者。

  蘇臘達!當這樣知、這樣見時,在這個有識之身上與在一切外部諸相上離我作、我所作、慢,成為心意超越慢類者、寂靜者、善解脫者。」[≃\suttaref{SN.18.21}]

  ……(中略)然後蘇臘達成為眾\twnr{阿羅漢}{5.0}之一。

  阿羅漢品第七,其\twnr{攝頌}{35.0}:

  「執取者、思量者,以及又歡喜者,

   無常、苦與無我,非自己的、會被染住立的,

   以羅陀、蘇臘達它們為十。」





\pin{被食品}{73}{82}
\sutta{73}{73}{樂味經}{https://agama.buddhason.org/SN/sn.php?keyword=22.73}
  起源於舍衛城。  

  「\twnr{比丘}{31.0}們!\twnr{未聽聞的一般人}{74.0}不如實知道色的\twnr{樂味}{295.0}、\twnr{過患}{293.0}、\twnr{出離}{294.0};受的……想的……諸行的……不如實知道識的樂味、過患、出離。

  比丘們!\twnr{有聽聞的聖弟子}{24.0}如實知道色的樂味、過患、出離;受的……想的……諸行的……如實知道識的樂味、過患、出離。」



\sutta{74}{74}{集經}{https://agama.buddhason.org/SN/sn.php?keyword=22.74}
  起源於舍衛城。  

  「\twnr{比丘}{31.0}們!\twnr{未聽聞的一般人}{74.0}不如實知道色的\twnr{集起}{67.0}、滅沒、\twnr{樂味}{295.0}、\twnr{過患}{293.0}、\twnr{出離}{294.0};受的……想的……諸行的……不如實知道識的集起、滅沒、樂味、過患、出離。

  比丘們!\twnr{有聽聞的聖弟子}{24.0}如實知道色的集起、滅沒、樂味、過患、出離;受的……想的……諸行的……如實知道識的集起、滅沒、樂味、過患、出離。」



\sutta{75}{75}{集經第二}{https://agama.buddhason.org/SN/sn.php?keyword=22.75}
  起源於舍衛城。  

  「\twnr{比丘}{31.0}們!\twnr{有聽聞的聖弟子}{24.0}如實知道色的\twnr{集起}{67.0}、滅沒、\twnr{樂味}{295.0}、\twnr{過患}{293.0}、\twnr{出離}{294.0};受的……想的……諸行的……如實知道識的集起、滅沒、樂味、過患、出離。」



\sutta{76}{76}{阿羅漢經}{https://agama.buddhason.org/SN/sn.php?keyword=22.76}
  起源於舍衛城。

  「\twnr{比丘}{31.0}們!色是無常的,凡是無常的,那個是苦的,凡是苦的,那個是無我,凡是無我,那個:『\twnr{這不是我的}{32.1},\twnr{我不是這個}{33.1},\twnr{這不是我的真我}{34.2}。』這樣,這個應該以正確之慧如實被看見。受……(中略)想……(中略)諸行……識是無常的,凡是無常的,那個是苦的,凡是苦的,那個是無我,凡是無我,那個:『這不是我的,我不是這個,這不是我的真我。』這樣,這個應該以正確之慧如實被看見。

  比丘們!這麼看的\twnr{有聽聞的聖弟子}{24.0}在色上\twnr{厭}{15.0},也在受上……也在想上……也在諸行上……也在識上厭。厭者\twnr{離染}{558.0},從\twnr{離貪}{77.0}被解脫,在已解脫時,\twnr{有『[這是]解脫』之智}{27.0},他知道:『\twnr{出生已盡}{18.0},\twnr{梵行已完成}{19.0},\twnr{應該被作的已作}{20.0},\twnr{不再有此處[輪迴]的狀態}{21.1}。』

  比丘們!\twnr{眾生住處}{641.0}之所及,直到\twnr{有之頂點}{886.0},在世間中,他們是第一的,他們是最上的,即:\twnr{阿羅漢}{5.0}們。」

  世尊說這個,說這個後,\twnr{善逝}{8.0}、\twnr{大師}{145.0}又更進一步說這個:

  「阿羅漢們確實是安樂者,他們的渴愛不被發現,

   \twnr{我是之慢}{400.0}已被斷絕,癡網已被碎破。

   他們是到達不動者,他們的心是不混濁的,

   他們是世間中的不染著者,\twnr{梵已生者}{458.0}、無\twnr{漏}{188.0}者。

   \twnr{遍知}{154.0}五蘊後,\twnr{在七善法的行境中}{x395},

   是應該被讚賞的\twnr{善人}{76.0},佛陀的親生子。

   \twnr{七寶}{x396}具足者,在三學上已學者,

   大英雄們漫遊:捨斷害怕恐懼者。

   具足\twnr{十支}{x397}:得定的大\twnr{龍象}{131.0},

   他們是世間中最上者,他們的渴愛不被發現。

   無學智生起者,這是最後的身體(集聚),

   凡\twnr{梵行的核心}{x398},在那個上不緣於他人者。

   \twnr{他們在慢類上不動搖}{x399},從再有被解脫,

   到達已調御階位者,他們是世間中的勝利者。

   上下四方,他們的歡喜不被發現,

   他們吼獅子吼,覺者們是世間中無上者。」



\sutta{77}{77}{阿羅漢經第二}{https://agama.buddhason.org/SN/sn.php?keyword=22.77}
  起源於舍衛城。

  「\twnr{比丘}{31.0}們!色是無常的,凡是無常的,那個是苦的,凡是苦的,那個是\twnr{無我}{23.0},凡是無我,那個:『\twnr{這不是我的}{32.1},\twnr{我不是這個}{33.1},\twnr{這不是我的真我}{34.2}。』……(中略)這樣,這個應該以正確之慧如實被看見。

  比丘們!這麼看的\twnr{有聽聞的聖弟子}{24.0}在色上\twnr{厭}{15.0},也在受上……也在想上……也在諸行上……也在識上厭。厭者\twnr{離染}{558.0},從\twnr{離貪}{77.0}被解脫,在已解脫時,\twnr{有『[這是]解脫』之智}{27.0},他知道:『\twnr{出生已盡}{18.0},\twnr{梵行已完成}{19.0},\twnr{應該被作的已作}{20.0},\twnr{不再有此處[輪迴]的狀態}{21.1}。』

  比丘們!\twnr{眾生住處}{641.0}之所及,直到\twnr{有之頂點}{886.0},在世間中,他們是第一的,他們是最上的,即:\twnr{阿羅漢}{5.0}們。」



\sutta{78}{78}{獅子經}{https://agama.buddhason.org/SN/sn.php?keyword=22.78}
  起源於舍衛城。 

  「\twnr{比丘}{31.0}們!獸王獅子傍晚時從棲息處出去;從棲息處出去後打哈欠;打哈欠後環視四方各處;環視四方各處後吼三回獅子吼;吼三回獅子吼後出發到食物處。比丘們!凡任何畜生生物聽聞獸王獅子的吼叫聲者,大多數來到害怕、急迫感、畏懼:穴居動物進入洞穴;水居動物進入水中;林居動物進入林中;有翅膀的\twnr{飛入}{x400}空中。比丘們!即使凡那些在村落、城鎮、王都中被堅固韁繩繫縛的國王的象,牠們也破壞、掙斷(切斷)那些繫縛後,驚嚇地排著大小便到處逃(逃往這裡或那裡)。比丘們!獸王獅子對畜生生物是這麼\twnr{大神通力的}{405.0}、這麼大影響力的、這麼大威力的。

  同樣的,比丘們!當\twnr{如來}{4.0}、\twnr{阿羅漢}{5.0}、\twnr{遍正覺者}{6.0}、\twnr{明行具足者}{7.0}、\twnr{善逝}{8.0}、\twnr{世間知者}{9.0}、\twnr{應該被調御人的無上調御者}{10.0}、\twnr{天-人們的大師}{11.0}、\twnr{佛陀}{3.0}、\twnr{世尊}{12.0}在世間出現,他教導法:『像這樣是色,像這樣是色的\twnr{集起}{67.0},像這樣是色的滅沒;像這樣是受……像這樣是想……像這樣是諸行……像這樣是識,像這樣是識的集起,像這樣是識的滅沒。』比丘們!即使凡那些在高天宮處長壽的、美貌的、多樂的、久住的諸天,聽聞如來的說法後,祂們也大多數來到害怕、急迫感、畏懼:『\twnr{先生}{202.0}!唉!當我們顯然是無常的時,我們認為是常的;先生!唉!當我們顯然是非堅固的時,我們認為是堅固的;先生!唉!當我們顯然是非常恆的時,我們認為是常恆的;先生!我們顯然是無常的、非堅固的、非常恆的、被\twnr{有身}{93.0}包含的。』比丘們!如來對包括天的世間是這麼大神通力的、這麼大影響力的、這麼大威力的。」

  世尊說這個……(中略)\twnr{大師}{145.0}說這個:

  「當佛陀\twnr{證知}{242.0}後,轉起法輪,

   包括天的世間中,大師是無與倫比者。

   有身與滅,與有身的生成,

   以及\twnr{八支聖道}{525.0}:導向苦的止息。

   即使凡長壽的諸天:有美貌的、有名聲的,

   來到驚嚇畏懼,如其他諸野獸對獅子。

   未越過有身者,先生!我們顯然是無常的:

   聽聞阿羅漢,解脫者、\twnr{像這樣者}{632.0}的言說後。」[\ccchref{AN.4.33}{https://agama.buddhason.org/AN/an.php?keyword=4.33}]



\sutta{79}{79}{被食經}{https://agama.buddhason.org/SN/sn.php?keyword=22.79}
  起源於舍衛城。

  「\twnr{比丘}{31.0}們!凡任何\twnr{回憶種種前世住處}{x401}的\twnr{沙門}{29.0}或\twnr{婆羅門}{17.0}回憶,他們全回憶五取蘊,或它們中之一,哪五個?

  『我過去世有這樣的色』:比丘們!像這樣回憶者僅回憶色。『我過去世有這樣的受』:比丘們!像這樣回憶者僅回憶受。『我過去世有這樣的想』……『我過去世有這樣的行』……『我過去世有這樣的識』:比丘們!像這樣回憶者僅回憶識。

  比丘們!而為何你們宣稱色?比丘們!『\twnr{它變壞}{x402}』,因此被稱為『色』,以什麼變壞?以寒,也以暑,也以飢,也以渴變壞,也以與虻蚊風烈日蛇的接觸變壞,比丘們!『它變壞』,因此被稱為『色』。

  比丘們!而為何你們宣稱受?比丘們!『\twnr{它感受}{x403}』,因此被稱為『受』。它感受什麼?感受苦,也感受樂,也感受不苦不樂,比丘們!『它感受』,因此被稱為『受』。

  比丘們!而為何你們宣稱想?比丘們!『它\twnr{認知}{583.0}』,因此被稱為『想』。它認知什麼?它認知藍的,它也認知黃的,它也認知紅的,它也認知白的,比丘們!『它認知』,因此被稱為『想』。

  比丘們!而為何你們宣稱諸行?比丘們!『\twnr{它們造作有為的}{x404}』,因此被稱為『諸行』,它們造作什麼\twnr{有為的}{90.0}?對色性它們造作有為的色;對受性它們造作有為的受;對想性它們造作有為的想;對行性它們造作有為的諸行;對識性它們造作有為的識,比丘們!『它們造作有為的』,因此被稱為『諸行』。

  比丘們!而為何你們宣稱識?比丘們!『它\twnr{識知}{833.0}』,因此被稱為『識』。它識知什麼?它識知酸的,它也識知苦的,它也識知辛辣的,它也識知甜的,它也識知鹼性的,它也識知非鹼性的,它也識知鹹的,它也識知不鹹的,比丘們!『它識知』,因此被稱為『識』。

  比丘們!在那裡,\twnr{有聽聞的聖弟子}{24.0}像這樣深慮:『我現在\twnr{被色食}{x405},過去世我也同樣地被色食,猶如現在被當前的色食。又,如果我同樣地歡喜未來色,\twnr{未來世}{308.0}我也同樣地被色食,猶如現在被當前的色食。』他像這樣深慮後食,在過去色上是無期待者,他不歡喜未來色,對現在色是為了\twnr{厭}{15.0}、\twnr{離貪}{77.0}、\twnr{滅的行者}{519.0}。

  『我現在被受食,過去世我也同樣地被受食,猶如現在被當前的受食。又,如果我同樣地歡喜未來受,未來世我也同樣地被受食,猶如現在被當前的受食。』他像這樣深慮後,在過去受上是無期待者,他不歡喜未來受,對現在受是為了厭、離貪、滅的行者。

  『我現在被想食……(中略)』……『我現在被諸行食,過去世我也同樣地被諸行食,猶如現在被當前的諸行食。又,如果我同樣地歡喜未來諸行,未來世我也同樣地被諸行食,猶如現在被當前的諸行食。』他像這樣深慮後,在過去諸行上是無期待者,他不歡喜未來諸行,對現在諸行是為了厭、離貪、滅的行者。

  『我現在被識食,過去世也同樣地被識食,猶如現在被當前的識食。又,如果我同樣地歡喜未來識,未來世我也同樣地被識食,猶如現在被當前的識食。』他像這樣深慮後,在過去識上是無期待者,他不歡喜未來識,對現在識是為了厭、離貪、滅的行者。

  比丘們!你們怎麼想它:色是常的,或是無常的?」

  「無常的,\twnr{大德}{45.0}!」

  「那麼,凡為無常的,那是苦的或樂的?」

  「苦的,大德!」

  「那麼,凡為無常的、苦的、\twnr{變易法}{70.0},適合認為它:『\twnr{這是我的}{32.0},\twnr{我是這個}{33.0},這是\twnr{我的真我}{34.0}。』嗎?」

  「大德!這確實不是。」

  「受……想……諸行……識是常的,或是無常的?」

  「無常的,大德!」

  「那麼,凡為無常的,那是苦的或樂的?」

  「苦,大德!」

  「那麼,凡為無常的、苦的、變易法,適合認為它:『這是我的,我是這個,這是我的真我。』嗎?」

  「大德!這確實不是。」

  「比丘們!因此,在這裡,凡任何色:過去、未來、現在,或內、或外,或粗、或細,或下劣、或勝妙,或凡在遠處、在近處,所有色:『\twnr{這不是我的}{32.1},\twnr{我不是這個}{33.1},\twnr{這不是我的真我}{34.2}。』這樣,這個應該以正確之慧如實被看見。

  凡任何受……凡任何想……凡任何諸行……凡任何識:過去、未來、現在……(中略)或凡在遠處、在近處,所有識:『這不是我的,我不是這個,這不是我的真我。』這樣,這個應該以正確之慧如實被看見。

  比丘們!這被稱為:聖弟子\twnr{拆解不堆積}{x406};\twnr{捨斷不執取}{x407};\twnr{驅散不積聚}{x408};\twnr{熄滅不點燃}{x409}。

  而拆解不堆積什麼?拆解不堆積色;受……想……諸行……拆解不堆積識。

  而捨斷不執取什麼?捨斷不執取色;受……想……諸行……捨斷不執取識。

  而驅散不積聚什麼?驅散不積聚色;受……想……諸行……驅散不積聚識。

  而熄滅不點燃什麼?熄滅不點燃色;受……想……諸行……熄滅不點燃識。

  比丘們!這麼看的有聽聞的聖弟子在色上\twnr{厭}{15.0},也在受上……也在想上……也在諸行上……也在識上厭。厭者\twnr{離染}{558.0},從\twnr{離貪}{77.0}被解脫,在已解脫時,\twnr{有『[這是]解脫』之智}{27.0},他知道:『\twnr{出生已盡}{18.0},\twnr{梵行已完成}{19.0},\twnr{應該被作的已作}{20.0},\twnr{不再有此處[輪迴]的狀態}{21.1}。』

  比丘們!這被稱為:比丘既不堆積也不拆解,\twnr{拆解後為住立者}{x410};既不捨斷也不執取,捨斷後為住立者;既不驅散也不積聚,驅散後為住立者;既不熄滅也不點燃,熄滅後為住立者。

  而既不堆積也不拆解什麼,拆解後為住立者?既不堆積也不拆解色,拆解後為住立者;受……想……諸行……既不堆積也不拆解識,拆解後為住立者。

  而既不捨斷也不執取什麼,捨斷後為住立者?既不捨斷也不執取色,捨斷後為住立者;受……想……諸行……既不捨斷也不執取識,捨斷後為住立者。

  而既不驅散也不積聚什麼,驅散後為住立者?既不驅散也不積聚色,驅散後為住立者;受……想……諸行……既不驅散也不積聚識,驅散後為住立者。

  而既不熄滅也不點燃什麼,熄滅後為住立者?既不熄滅也不點燃色,熄滅後為住立者;受……想……諸行……既不熄滅也不點燃識,熄滅後為住立者。

  比丘們!這樣心解脫的比丘,包括帝釋天的、包括梵天的、包括生主神的,就遠遠地禮敬:

  『對你禮敬,高貴的人!對你禮敬,最上的人!

   我們不\twnr{證知}{242.0},凡你修禪依止的。』[\ccchref{AN.11.9}{https://agama.buddhason.org/AN/an.php?keyword=11.9}]」



\sutta{80}{80}{托鉢經}{https://agama.buddhason.org/SN/sn.php?keyword=22.80}
  \twnr{有一次}{2.0},\twnr{世尊}{12.0}住在釋迦族人的迦毘羅衛城尼拘律園。

  那時,世尊就在某個場合遣離\twnr{比丘}{31.0}\twnr{僧團}{375.0}後,午前時穿衣、拿起衣鉢後,\twnr{為了托鉢}{87.0}進入迦毘羅衛城。

  在迦毘羅衛城為了托鉢行走後,\twnr{餐後已從施食返回}{512.0},\twnr{為了白天的住處}{128.0}去大林中。進入大林後,坐在小橡樹下為了白天的住處。

  那時,當世尊獨處、\twnr{獨坐}{92.0}時,這樣心的深思生起:

  「比丘僧團被我逐出,在這裡有出家不久的新比丘,最近來到這法、律中,當他們見不到我時,會變異(異心),會變易,猶如當幼小牛隻見不到母親時,會變異,會變易。同樣的,這裡有出家不久的新比丘,最近來到這法、律中,當他們見不到我時,會變異,會變易。猶如當新種子沒得到水時,會變異,會變易。同樣的,這裡有……當他們見不到我時,會變異,會變易。[\ccchref{MN.67}{https://agama.buddhason.org/MN/dm.php?keyword=67}, 158段]

  現在讓我資助比丘僧團,就如以前比丘僧團被我資助那樣。」

  那時,\twnr{梵王娑婆主}{215.0}以心了知世尊心中的深思後,就猶如有力氣的男子伸直彎曲的手臂,或彎曲伸直的手臂,就像這樣在梵天世界消失,出現在世尊的面前。

  那時,梵王娑婆主置(作)上衣到一邊肩膀,向世尊\twnr{合掌}{377.0}鞠躬後,對世尊說這個:

  「這是這樣,世尊!這是這樣,\twnr{善逝}{8.0}!\twnr{大德}{45.0}!世尊的比丘僧團被逐出,這裡有出家不久的新比丘,最近來到這法、律中,當他們見不到世尊時,會變異(異心),會變易,猶如當幼小牛隻見不到母親時,會變異,會變易。同樣的,這裡有出家不久的新比丘,最近來到這法、律中,當他們見不到世尊時,會變異,會變易。猶如當新種子沒得到水時,會變異,會變易。同樣的,這裡有出家不久的新比丘,最近來到這法、律中,當他們見不到世尊時,會變異,會變易。

  大德!請世尊接受比丘僧團!大德!請世尊歡迎比丘僧團!現在請世尊資助比丘僧團,就如以前比丘僧團被世尊資助那樣。」

  世尊以沈默狀態同意。

  那時,梵王娑婆主知道世尊同意後,向世尊\twnr{問訊}{46.0}、\twnr{作右繞}{47.0}後,就在那裡消失。

  那時,世尊傍晚時,從獨坐出來,去尼拘律園。抵達後,在設置的座位坐下。坐下後,世尊\twnr{造作像那樣的神通作為}{425.0}:如是,那些比丘會以每次一位二位帶著羞恥的容色來見我。那些比丘也以每次一位二位帶著羞恥的容色去見世尊。抵達後,向世尊問訊後,在一旁坐下。世尊對在一旁坐下的那些比丘說這個:

  「比丘們!這是活命方式中的末端,即:托鉢。比丘們!這在世間是詛咒:『托鉢者(乞食者)你手持鉢遊蕩。』

  比丘們!通曉道理的善男子們\twnr{緣於}{252.0}道理而進入那個與這個[托鉢],既非被國王壓迫的,也非被盜賊壓迫的,也非有負債苦惱的,也非有害怕苦惱的,也\twnr{非生活所迫的}{x411},而是:『我們是陷入生、老、死、愁、悲、苦、憂、絕望者,陷入苦者,被苦征服者,也許這整個\twnr{苦蘊}{83.0}的作終結能被知道。』

  比丘們!而這位這樣出家的善男子,但他是\twnr{貪婪者}{435.0}、在欲上重貪欲者、有瞋害心者、\twnr{有憎惡之意向者}{875.0}、\twnr{念已忘失者}{216.0}、不正知者、不得定者、心散亂者、根不控制者。比丘們!猶如\twnr{火葬場的燃燒木柴}{x412}:兩端被燃燒的,在中間沾糞的,既不能在村落中當木材的目的,也不能在林野中當木材的目的。比丘們!我說這個人像這樣的譬喻:從在家享樂被錯失,且未使\twnr{沙門}{29.0}的利益完成。

  比丘們!有三\twnr{不善尋}{x413}:欲尋、惡意尋、\twnr{加害尋}{376.2}。比丘們!而這三不善尋在何處無殘餘地被滅?對住於在\twnr{四念住}{286.0}中善建立心者,或對修習\twnr{無相定}{265.0}者。比丘們!到那個程度,這就足以要修習無相定。比丘們!無相定已修習、已多作,有大果、大效益。

  比丘們!有這二種見:\twnr{有見與無有見}{782.0}。

  比丘們!在那裡,\twnr{有聽聞的聖弟子}{24.0}像這樣深慮:『世間中有任何當執取時我不會有過失的嗎?』

  他這麼知道:『世間中確實沒有任何當執取時我不會有過失的,因為當執取時,我只會執取色,只受……只想……只諸行……當執取時,我只會執取識。對那個我來說,那會是:以取\twnr{為緣}{180.0}有有(而有存在);以有為緣有生;以生為緣而會有老、死、愁、悲、苦、憂、\twnr{絕望}{342.0}生成,這樣是這整個\twnr{苦蘊}{83.0}的\twnr{集}{67.0}。』

  比丘們!你們怎麼想它:色是常的,或是無常的?」

  「無常的,大德!」

  「那麼,凡為無常的,那是苦的或樂的?」

  「苦的,大德!」

  「那麼,凡為無常的、苦的、\twnr{變易法}{70.0},適合認為它:『\twnr{這是我的}{32.0},\twnr{我是這個}{33.0},\twnr{這是我的真我}{34.1}。』嗎?」

  「大德!這確實不是。」

  「受……想……諸行……識……」……(中略)

  「比丘們!因此,在這裡,這麼看的……他知道:『……\twnr{不再有此處[輪迴]的狀態}{21.1}。』」



\sutta{81}{81}{巴利雷雅經}{https://agama.buddhason.org/SN/sn.php?keyword=22.81}
  \twnr{有一次}{2.0},\twnr{世尊}{12.0}住在\twnr{憍賞彌}{140.0}瞿師羅園。

  那時,世尊午前時穿衣、拿起衣鉢後,\twnr{為了托鉢}{87.0}進入憍賞彌。

  在憍賞彌為了托鉢行走後,\twnr{餐後已從施食返回}{512.0},收起自己的臥坐具、拿起衣鉢後,沒召喚侍者們、沒通知\twnr{比丘}{31.0}眾後,獨自無伴地出發遊行。[\ccchref{Ud.35}{https://agama.buddhason.org/Ud/dm.php?keyword=35}]

  那時,當世尊離開不久,某位比丘去見\twnr{尊者}{200.0}阿難。抵達後,對尊者阿難說這個:

  「這位阿難\twnr{學友}{201.0}!世尊收起自己的臥坐具、拿起衣鉢後,沒召喚侍者們、沒通知比丘眾後,已獨自無伴地出發遊行。」

  「學友!凡在世尊收起自己的臥坐具、拿起衣鉢後,沒召喚侍者們、沒通知比丘僧眾後,獨自無伴地出發遊行時,在那時,世尊就\twnr{想單獨地住}{x414},在那時,世尊不應該被任何人跟隨。」

  那時,世尊次第地進行著遊行,抵達巴利雷雅。在那裡,世尊住在巴利雷雅的\twnr{吉祥沙羅樹下}{x415}。

  那時,眾多比丘去見尊者阿難。抵達後,與尊者阿難一起互相問候。交換應該被互相問候的友好交談後,在一旁坐下。在一旁坐下的那些比丘對尊者阿難說這個:

  「阿難學友!法說被我們從世尊的面前長久[前]聽聞,阿難學友!我們想要從世尊的面前聽聞法說。」

  那時,尊者阿難與那些比丘一同去巴利雷雅吉祥的沙羅樹見世尊。抵達後,向世尊\twnr{問訊}{46.0}後,在一旁坐下。世尊對在一旁坐下的那些比丘以法說開示、勸導、鼓勵、\twnr{使歡喜}{86.0}。

  當時,某位比丘這樣心的\twnr{深思}{x416}生起:

  「怎樣知、怎樣見者有諸\twnr{漏}{188.0}的\twnr{直接}{615.0}滅盡呢?」

  那時,世尊以心了知那位比丘心中的深思後,召喚比丘們:

  「比丘們!法被我檢擇地教導:\twnr{四念住}{286.0}被檢擇地教導,四正勤被檢擇地教導,四神足被檢擇地教導,五根被檢擇地教導,五力被檢擇地教導,七覺支被檢擇地教導,\twnr{八支聖道}{525.0}被檢擇地教導。比丘們!法被我這樣檢擇地教導。比丘們!在被我這樣檢擇地教導的法上,而這裡,一位比丘這樣心的深思生起:『怎樣知、怎樣見者有諸漏的直接滅盡呢?』

  比丘們!而怎樣知、怎樣見者有諸漏的直接滅盡?

  比丘們!這裡,\twnr{未聽聞的一般人}{74.0}是聖者的未看見者,聖者法的不熟知者,在聖者法上未被教導者;是善人的未看見者,\twnr{善人法}{76.0}的不熟知者,在善人法上未被教導者,他\twnr{認為}{964.0}色是我。

  比丘們!而凡\twnr{那種認為,那是行}{x417}。

  而那個行,什麼為因?什麼為集?什麼生的?\twnr{什麼為根源}{668.0}?

  比丘們!當被\twnr{無明觸}{89.0}所生的感受接觸時,未聽聞的一般人的渴愛生起,那個行是從那裡生的。

  比丘們!像這樣,那個行是無常的、\twnr{有為的}{90.0}、\twnr{緣所生的}{557.0}。

  那個渴愛也是無常的、有為的、緣所生的。

  那個受也是無常的、有為的、緣所生的。

  那個觸也是無常的、有為的、緣所生的。

  那個無明也是無常的、有為的、緣所生的。

  比丘們!這樣知者、這樣見者有諸漏的直接滅盡。

  即便認為色不是我,但認為\twnr{我擁有色}{13.0}。比丘們!而凡那種認為,那是行。而那個行,什麼為因?什麼為集?什麼生的?什麼為根源?比丘們!當被無明觸所生的感受接觸時,未聽聞的一般人的渴愛生起,那個行是從那裡生的。比丘們!像這樣,那個行是無常的、有為的、緣所生的。那個渴愛也是……那個受也是……那個觸也是……那個無明也是無常的、有為的、緣所生的。比丘們!這樣知者、這樣見者有諸漏的直接滅盡。

  即便認為色不是我、認為我不擁有色,但認為\twnr{色在我中}{14.0}。比丘們!而凡那種認為,那是行。而那個行,什麼為因?什麼為集?什麼生的?什麼為根源?比丘們!當被無明觸所生的感受接觸時,未聽聞的一般人的渴愛生起,那個行是從那裡生的。比丘們!像這樣,那個行是無常的、有為的、緣所生的。那個渴愛也是……那個受也是……那觸也是……那個無明也是無常的、有為的、緣所生的。比丘們!這樣知者、這樣見者有諸漏的直接滅盡。

  即便認為色不是我、認為我不擁有色、認為色不在我中,但認為我在色中。比丘們!而凡那種認為,那是行。而那個行,什麼為因?什麼為集?什麼生的?什麼為根源?比丘們!當被無明觸所生的感受接觸時,未聽聞的一般人的渴愛生起,那個行是從那裡生的。比丘們!像這樣,那個行是無常的、有為的、緣所生的。那個渴愛也是……那個受也是……那個觸也是……那個無明也是無常的、有為的、緣所生的。比丘們!這樣知者、這樣見者有諸漏的直接滅盡。

  即便認為色不是我、我不擁有色、色不在我中,認為我不在色中,但認為受是我……但認為我擁有受……但認為受在我中……但認為我在受中……但認為想……但認為諸行是我……但認為我擁有諸行……但認為諸行在我中……但認為我在諸行中……但認為識是我……但認為我擁有識……但認為識在我中……但認為我在識中。比丘們!而凡那種認為,那是行。而那個行,什麼是其因……(中略)什麼為根源?比丘們!當被無明觸所生的感受接觸時,未聽聞的一般人的渴愛生起,那個行是從那裡生的。比丘們!像這樣,那個行是無常的、有為的、緣所生的。那個渴愛也是……那個受也是……那個觸也是……那個無明也是無常的、有為的、緣所生的。比丘們!這樣知者、這樣見者有諸漏的直接滅盡。

  即便認為色不是我、認為受不是我、想不是……諸行不是……認為識不是我,但有這樣的見:『\twnr{彼是我者彼即是世間}{887.0},那個我死後將成為常的、堅固的、永恆的、不\twnr{變易法}{70.0}。』比丘們!而凡那種\twnr{常見}{x418},那是行。而那個行,什麼是其因……(中略)比丘們!這樣知者、這樣見者有諸漏的直接滅盡。

  即便認為色不是我、受不……想不……諸行不……認為識不是我,也沒有這樣的見:『彼是我者彼即是世間,那個我死後將成為常的、堅固的、永恆的、不變易法。』但有這樣的見:『我不會存在以及我的不會存在,如果我不存在,\twnr{我的將不存在}{616.0}。』比丘們!而凡那種\twnr{斷滅見}{x419},那是行。而那個行,什麼為因?什麼為集?什麼生的?什麼為根源?比丘們!當被無明觸所生的感受接觸時,未聽聞的一般人的渴愛生起,那個行是從那裡生的。比丘們!像這樣,那個行是無常的……(中略)比丘們!這樣知者、這樣見者有諸漏的直接滅盡。

  即便認為色不是我、受不……想不……諸行不……認為識不是我……(中略)認為我不在識中,也沒有這樣的見:『彼是我者彼即是世間,那個我死後將成為常的、堅固的、永恆的、不變易法。』也沒有這樣的見:『我不會存在以及我的不會存在,如果我不存在,我的將不存在。』但在正法上是疑惑者、懷疑者、\twnr{未達想要者}{x420}。比丘們!而凡那種在正法上疑惑、懷疑、未達想要,那是行。而那個行,什麼為因?什麼為集?什麼生的?什麼為根源?比丘們!當被無明觸所生的感受接觸時,未聽聞的一般人的渴愛生起,那個行是從那裡生的。比丘們!像這樣,那個行是無常的、有為的、緣所生的。那個渴愛也是無常的、有為的、緣所生的。那個受也是無常的、有為的、緣所生的。那個觸也是無常的、有為的、緣所生的。那個無明也是無常的、有為的、緣所生的。比丘們!這樣知者、這樣見者有諸漏的直接滅盡。」



\sutta{82}{82}{滿月經}{https://agama.buddhason.org/SN/sn.php?keyword=22.82}
  \twnr{有一次}{2.0},\twnr{世尊}{12.0}與大\twnr{比丘}{31.0}\twnr{僧團}{375.0}一起住在舍衛城東園鹿母講堂。

  當時,世尊在十五\twnr{那個布薩日}{222.0}的滿月夜晚,被比丘僧團圍繞,坐\twnr{在屋外}{385.0}。

  那時,\twnr{某位比丘}{39.0}從座位起來後,置(作)上衣到一邊肩膀,向世尊\twnr{合掌}{377.0}鞠躬後,對世尊說這個:

  「\twnr{大德}{45.0}!願我就某點詢問世尊,如果世尊為我的問題之解答給機會。」

  「比丘!那樣的話,你坐在自己的座位後,請你問凡你希望的。」

  「是的,大德!」那位比丘回答世尊後,坐在自己的座位後,對世尊說這個:

  「大德!這些不是\twnr{五取蘊}{36.0}嗎?即:色取蘊、受取蘊、想取蘊、行取蘊、識取蘊。」

  「比丘!這些是五取蘊,即:色取蘊……(中略)識取蘊。」

  「\twnr{好}{44.0}!大德!」那位比丘歡喜、隨喜世尊所說後,更進一步問世尊問題:

  「大德!那麼,這些五取蘊,什麼為根源呢?」

  「比丘!這些五取蘊,意欲為根源。」{……(中略)}

  「大德!那個執取就是那些五取蘊,或者,從五取蘊外有執取呢?」

  「比丘!那個執取非就是那些五取蘊,也非從五取蘊外有執取,而是凡哪裡有意欲貪者,那裡有執取。」

  「好!大德!」那位比丘……(中略)更進一步問[世尊]問題:

  「大德!那麼,在五取蘊上會有意欲貪的差別性嗎?」

  「比丘!會有。」世尊說。

  「比丘!這裡,一類人這麼想:『我\twnr{未來時}{308.0}會有這樣的色,我未來時會有這樣的受,我未來時會有這樣的想,我未來時會有這樣的行,我未來時會有這樣的識。』這樣,比丘!在五取蘊上會有意欲貪的差別性。」

  「好!大德!」那位比丘……(中略)更進一步問[世尊]問題:

  「大德!什麼情形是諸蘊的蘊屬性(名稱)呢?」

  「比丘!凡任何色:過去、未來、現在,或內、或外,或粗、或細,或下劣、或勝妙,或凡在遠處、在近處,這被稱為色蘊。凡任何受……凡任何想……凡任何諸行……凡任何識:過去、未來、現在,或內、或外,或粗、或細,或下劣、或勝妙,或凡在遠處、在近處,這被稱為識蘊。比丘!這個情形是諸蘊的蘊屬性。」

  「好!大德!」那位比丘……(中略)問:

  「大德!什麼因、什麼\twnr{緣}{180.0},有色蘊的\twnr{安立}{143.0}?什麼因、什麼緣,有受蘊的安立?什麼因、什麼緣,有想蘊的安立?什麼因、什麼緣,有行蘊的安立?什麼因、什麼緣,有識蘊的安立?」

  「比丘!\twnr{四大}{646.0}為因、四大\twnr{為緣}{180.0}有色蘊的安立;觸為因、觸為緣有受蘊的安立;觸為因、觸為緣有想蘊的安立;觸為因、觸為緣有行蘊的安立;名色為因、名色為緣有識蘊的安立。」

  「好!大德!」那位比丘……(中略)問:

  「大德!怎樣有\twnr{有身見}{93.1}呢?」

  「比丘!這裡,\twnr{未聽聞的一般人}{74.0}是聖者的未看見者,聖者法的不熟知者,在聖者法上未被教導者;是善人的未看見者,\twnr{善人法}{76.0}的不熟知者,在善人法上未被教導者,他\twnr{認為}{964.0}色是我,\twnr{或我擁有色}{13.0},或色在我中,\twnr{或我在色中}{14.0};受……想……諸行……認為識是我,或我擁有識,或識在我中,或我在識中。比丘!這樣有有身見。」

  「好!大德!」那位比丘……(中略)問:

  「大德!那麼,怎樣沒有有身見呢?」

  「比丘!這裡,\twnr{有聽聞的聖弟子}{24.0}是聖者的看見者,聖者法的熟知者,在聖者法上被善教導者;是善人的看見者,善人法的熟知者,在善人法上被善教導者,他認為色不是我,或我不擁有色,或色不在我中,或我不在色中;受……想……諸行……認為識不是我,或我不擁有識,或識不在我中,或我不在識中。比丘!這樣沒有有身見。」

  「好!大德!」那位比丘……(中略)問:

  「大德!什麼是色的\twnr{樂味}{295.0}、什麼是\twnr{過患}{293.0}、什麼是\twnr{出離}{294.0}?什麼是受的……什麼是想的……什麼是諸行的……什麼是識的樂味、什麼是過患、什麼是出離?」

  「比丘!凡\twnr{緣於}{252.0}色生起樂、喜悅者,這是色的樂味。

  凡色是無常的、苦的、\twnr{變易法}{70.0}者,這是色的過患。

  凡在色上意欲貪的調伏、意欲貪的捨斷者,這是色的出離。

  凡緣於受……緣於想……緣於諸行……緣於識生起樂、喜悅者,這是識的樂味。凡識是無常的、苦的、變易法者,這是識的過患。凡在識上意欲貪的調伏、意欲貪的捨斷者,這是識的出離。」

  「好!大德!」那位比丘歡喜、隨喜世尊所說後,更進一步問世尊問題:

  「大德!當怎樣知、當怎樣見時,在這個有識之身上\twnr{與在一切外部諸相上}{559.0}沒有\twnr{我作}{22.0}、\twnr{我所作}{25.0}、\twnr{慢煩惱潛在趨勢}{26.0}?」

  「比丘!凡任何色:過去、未來、現在,或內、或外,或粗、或細,或下劣、或勝妙,或凡在遠處、在近處,所有色:『\twnr{這不是我的}{32.1},\twnr{我不是這個}{33.1},\twnr{這不是我的真我}{34.2}。』以正確之慧這樣如實看見這個。

  凡任何受……凡任何想……凡任何諸行……凡任何識:過去、未來、現在,或內、或外,或粗、或細,或下劣、或勝妙,或凡在遠處、在近處,所有識:『這不是我的,我不是這個,這不是我的真我。』以正確之慧這樣如實看見這個。

  比丘!當這樣知、這樣見時,在這個有識之身上與在一切外部諸相上沒有我作、我所作、慢煩惱潛在趨勢。」

  當時,某位比丘的這樣心的深思生起:

  「\twnr{先生}{202.0}!像這樣,顯然,色是無我的;受……想……諸行……識是無我的,無我所作諸業將如何觸達我呢?」

  那時,世尊以心了知那位比丘心中的深思後,召喚比丘們:

  「比丘們!然而,這存在可能性:凡在這裡,一類無用的、無智的、\twnr{進入無明的}{645.0}男子,以渴愛支配心,想老師的教導應該被超越:『先生!像這樣,顯然,色是無我的;受……想……諸行……識是無我的,無我所作諸業將如何觸達我呢?』

  比丘們!你們在諸法上到處被我\twnr{反問教導(調伏)}{x421},比丘們!你們怎麼想它:色是常的,或是無常的?」

  「無常的,大德!」

  「受……想……諸行……識是常的,或是無常的?」

  「無常的,大德!」

  「而凡是無常的,是苦的,或是樂的?」

  「苦的,大德!」

  「而凡是無常的、苦的、\twnr{變易法}{70.0},適合認為它:『\twnr{這是我的}{32.0},\twnr{我是這個}{33.0},這是\twnr{我的真我}{34.0}』呢?」

  「大德!這確實不是。」

  「[比丘們!]因此,在這裡,這麼看的……(中略)他知道:『……(中略)\twnr{不再有此處[輪迴]的狀態}{21.1}。』」[\ccchref{MN.109}{https://agama.buddhason.org/MN/dm.php?keyword=109}]

  「蘊二則、那就會有,名詞及因,

   有身兩說,樂味及識,此為比丘的十種所問。」

  被食品第八,其\twnr{攝頌}{35.0}:

  「樂味、二則集,以阿羅漢二則在後,

   獅子、被食、托鉢,巴利雷雅與滿月。」





\pin{上座品}{83}{92}
\sutta{83}{83}{阿難經}{https://agama.buddhason.org/SN/sn.php?keyword=22.83}
  起源於舍衛城。

  在那裡,\twnr{尊者}{200.0}阿難召喚\twnr{比丘}{31.0}們:「比丘學友們!」

  「\twnr{學友}{201.0}!」那些比丘回答尊者阿難。

  尊者阿難說這個:

  「學友們!名為富留那彌多羅尼子的尊者對當是\twnr{新學}{210.0}時的我們是多助益者,他以這個教誡教誡我們:『阿難學友!\twnr{執取後有「我是」[的觀念]}{x422},非不執取後。執取什麼後有「我是」,非不執取後呢?執取色後有「我是」,非不執取後;受……想……諸行……執取識後有「我是」,非不執取後。

  阿難學友!猶如年輕、年少、喜好裝飾之類的女子或男子,當在鏡中,或在遍淨、潔淨、清澈的水鉢中省察自己的面相時,會(執)取後看,非不(執)取後。同樣的,阿難學友!執取色後有「我是」,非不執取後;受……想……諸行……執取識後有「我是」,非不執取後。

  阿難學友!你怎麼想它:「色是常的,或是無常的?」』

  『無常的,學友!』

  『受……想……諸行……識是常的,或是無常的?』

  『無常的,學友!』

  因此,在這裡……(中略)這麼看的……(中略)他知道:『……\twnr{不再有此處[輪迴]的狀態}{21.1}。』

  學友們!名為富留那彌多羅尼子的尊者對當是新學時的我們是多助益者,他以這個教誡教誡我們。

  而且,聽聞尊者富留那彌多羅尼子的這個說法後,法已被我\twnr{現觀}{53.0}了。」



\sutta{84}{84}{低舍經}{https://agama.buddhason.org/SN/sn.php?keyword=22.84}
  起源於舍衛城。

  當時,世尊姑媽的兒子\twnr{尊者}{200.0}低舍這麼告訴眾多\twnr{比丘}{31.0}:

  「\twnr{學友}{201.0}們!我的身體恐怕像變成酒醉的,我的諸方向不顯現,諸法也不在我心中出現,惛沈睡眠\twnr{持續遍取}{530.0}我的心,以及我無大喜樂地行梵行,以及在諸法上有我的懷疑。」

  那時,眾多比丘去見世尊。抵達後,向世尊\twnr{問訊}{46.0}後,在一旁坐下。在一旁坐下的那些比丘對世尊說這個:

  「\twnr{大德}{45.0}!世尊姑媽的兒子尊者低舍這麼告訴眾多比丘:『學友們!我的身體恐怕像變成酒醉的,我的諸方向不顯現,諸法也不在我心中出現,惛沈睡眠持續遍取我的心,以及我無大喜樂地行梵行,以及在諸法上有我的懷疑。』」

  那時,世尊召喚某位比丘:

  「來!比丘!請你以我的名義召喚低舍比丘。」

  「是的,大德!」那位比丘回答世尊後,去見尊者低舍。抵達後,對尊者低舍說這個:

  「低舍\twnr{學友}{201.0}!大師召喚你。」

  「是的,學友!」尊者低舍回答那位比丘後,去見世尊。抵達後,向世尊問訊後,在一旁坐下。世尊對在一旁坐下的尊者低舍說這個:

  「低舍!傳說是真的?你這麼告訴眾多比丘:『學友們!我的身體恐怕像變成酒醉的……(中略)以及在諸法上有我的懷疑。』」

  「是的,大德!」

  「低舍!你怎麼想它:在色上未\twnr{離貪}{77.0}、未離意欲、未離情愛、未離渴望、未離熱惱、未離渴愛者,從那個色的變易變異,愁、悲、苦、憂、\twnr{絕望}{342.0}生起?」

  「是的,大德!」

  「低舍!好!好!低舍!這確實是這樣:如那個在色上未離貪……在受上……在想上……在諸行上未離貪……(中略)從那些諸行的變易變異,愁、悲、苦、憂、絕望生起?」

  「是的,大德!」

  「低舍!好!好!低舍!這確實是這樣:如那個在識上未離貪、未離意欲、未離情愛、未離渴望、未離熱惱、未離渴愛者,從那個識的變易變異,愁、悲、苦、憂、絕望生起?」

  「是的,大德!」

  「低舍!好!好!低舍!那確實是這樣:如那個在識上未離貪[……]。低舍!你怎麼想它:在色上離貪、離欲、離情愛、離渴、離熱惱、離渴愛者,從那個色的變易變異,愁、悲、苦、憂、絕望生起?」

  「大德!這確實不是。」

  「低舍!好!好!低舍!這確實是這樣:如那個在色上離貪……在受上……在想上……在諸行上離貪……在識上離貪、離欲、離情愛、離渴、離熱惱、離渴愛者,從那個識的變易變異,愁、悲、苦、憂、絕望生起?」

  「大德!這確實不是。」

  「低舍!好!好!低舍!這確實是這樣:如那個在識上離貪……。低舍!你怎麼想它:色是常的,或是無常的?」

  「無常的,大德!」

  「受……想……諸行……識是常的,或是無常的?」

  「無常的,大德!」

  「因此,在這裡……(中略)這麼看的……(中略)他知道:『……\twnr{不再有此處[輪迴]的狀態}{21.1}。』

  低舍!猶如有二位男子:一位男子是不熟悉道路者,一位男子是熟悉道路者,那位不熟悉道路的男子向那位熟悉道路的男子問道路,他這麼說:『來!\twnr{先生}{202.0}!這是道路:請你以那個走片刻;以那個走片刻後你將看到叉路,在那裡放棄左邊的後請你取右邊的,請你以那個走片刻;以那個走片刻後你將看到極密的叢林,請你以那個走片刻;以那個走片刻後你將看到大的低窪沼澤,請你以那個走片刻;以那個走片刻後你將看到深的懸崖,請你以那個走片刻;以那個走片刻後你將看到能被喜樂的平坦土地。』

  低舍!為了義理的使知這個譬喻被我作。在這裡,這就是義理:低舍!『不熟悉道路的男子』,這是凡夫的同義語;低舍!『熟悉道路的男子』,這是\twnr{如來}{4.0}、\twnr{阿羅漢}{5.0}、\twnr{遍正覺者}{6.0}的同義語;低舍!『岔路』,這是懷疑的同義語;低舍!『左邊的路』,這是八支邪道的同義語,即:邪見……(中略)邪定;低舍!『右邊的路』,這是\twnr{八支聖道}{525.0}的同義語,即:正見……(中略)正定;低舍!『極密的叢林』,這是\twnr{無明}{207.0}的同義語;低舍!『大的低窪沼澤』,這是諸欲的同義語;低舍!『深的懸崖』,這是憤怒絕望的同義語;低舍!『能被喜樂的平坦土地』,這是涅槃的同義語。

  低舍!請你大喜樂!低舍!請你大喜樂!經我告誡,經我資助,經我教誡。」

  世尊說這個,悅意的尊者低舍歡喜世尊的所說。



\sutta{85}{85}{焰摩迦經}{https://agama.buddhason.org/SN/sn.php?keyword=22.85}
  \twnr{有一次}{2.0},\twnr{尊者}{200.0}舍利弗住在舍衛城祇樹林給孤獨園。

  當時,名為焰摩迦的\twnr{比丘}{31.0},有像這樣邪惡的\twnr{惡見}{722.0}生起:

  「我像這樣了知被\twnr{世尊}{12.0}教導的法,如:漏盡比丘,以身體的崩解被斷滅、消失,\twnr{死後不存在}{x423}。」

  眾多比丘聽聞傳說名為焰摩迦的比丘有像這樣邪惡的惡見生起:

  「我像這樣了知被世尊教導的法,如:漏盡比丘,以身體的崩解被斷滅、消失,死後不存在。」

  那時,那些比丘去見尊者焰摩迦。抵達後,與尊者焰摩迦一起互相問候。交換應該被互相問候的友好交談後,在一旁坐下。在一旁坐下的那些比丘對尊者焰摩迦說這個:

  「是真的嗎?焰摩迦\twnr{學友}{201.0}!你已生起像這樣邪惡的惡見:『我像這樣了知被世尊教導的法,如:漏盡比丘,以身體的崩解被斷滅、消失,死後不存在。』」

  「學友們!我確實這樣了知被世尊教導的法:『漏盡比丘,以身體的崩解被斷滅、消失,死後不存在。」

  「焰摩迦學友!你不要這麼說,你不要誹謗世尊,對世尊的誹謗是不好的,世尊不會這麼說:『漏盡比丘,以身體的崩解被斷滅、消失,死後不存在。』」

  當即使被那些比丘這麼說時,尊者焰摩迦仍同樣剛毅地、取著地執著那個邪惡的惡見後說:「我像這樣了知被世尊教導的法,如:漏盡比丘,以身體的崩解被斷滅、消失,死後不存在。」

  由於那些比丘不能夠使尊者焰摩迦從那個邪惡的惡見遠離,然後那些比丘從座位起來後去見尊者舍利弗。抵達後,對尊者舍利弗說這個:

  「舍利弗學友!名為焰摩迦的比丘已生起像這樣邪惡的惡見:『我像這樣了知被世尊教導的法,如:漏盡比丘,以身體的崩解被斷滅、消失,死後不存在。』請尊者舍利弗\twnr{出自憐愍}{121.0}去見焰摩迦比丘,\twnr{那就好了}{44.0}。」

  尊者舍利弗以沈默狀態同意。

  那時,尊者舍利弗傍晚時,從\twnr{獨坐}{92.0}出來,去見尊者焰摩迦。抵達後,與尊者焰摩迦一起互相問候……(中略)在一旁坐下的尊者舍利弗對尊者焰摩迦說這個:

  「是真的嗎?焰摩迦學友!你已生起像這樣邪惡的惡見:我像這樣了知被世尊教導的法,如:漏盡比丘,以身體的崩解被斷滅、消失,死後不存在。」

  「學友!我確實這樣了知被世尊教導的法,如:漏盡比丘,以身體的崩解被斷滅、消失,死後不存在。」

  「焰摩迦學友!你怎麼想它:色是常的,或是無常的?」

  「無常的,學友!」

  「受……想……諸行……識是常的,或是無常的?」

  「無常的,學友!」

  「……因此,在這裡,這麼看的……(中略)他知道:『……\twnr{不再有此處[輪迴]的狀態}{21.1}。』」

  「焰摩迦學友!你怎麼想它:你\twnr{認為}{964.0}『色是如來。』嗎?」

  「學友!這確實不是。」

  「你認為:『受是如來。』嗎?」

  「學友!這確實不是。」

  「想……諸行……你認為:『識是如來。』嗎?」

  「學友!這確實不是。」

  「焰摩迦學友!你怎麼想它:你認為『如來是在色中。』嗎?」

  「學友!這確實不是。」

  「你認為:『\twnr{除了色外有如來}{x424}。』嗎?」

  「學友!這確實不是。」

  「在受中……除了受外……(中略)在想中……除了想外……在諸行中……除了在諸行外……你認為:『如來在識中。』嗎?」

  「學友!這確實不是。」

  「你認為:『除了識外有如來。』嗎?」

  「學友!這確實不是。」

  「焰摩迦學友!你怎麼想它:你認為『色受想諸行識是如來。』嗎?」

  「學友!這確實不是。」

  「焰摩迦學友!你怎麼想它:你認為『這位那個無色者……無受者……無想者……無行者……無識者是如來。』嗎?」

  「學友!這確實不是。」

  「焰摩迦學友!而在這裡,當在此生中真實的、\twnr{實際的如來未被你得到時}{x425},\twnr{你的那個記說}{x426}:『我像這樣了知被世尊教導的法,如:漏盡比丘,以身體的崩解被斷滅、消失,死後不存在。』是否是適當的呢?」

  「舍利弗學友!之前,無智的我的那個邪惡的惡見存在,但聽聞尊者舍利弗的這個說法後,那個邪惡的惡見就已被捨斷,並且,法被我\twnr{現觀}{53.0}。」

  「焰摩迦學友!如果他們這麼問你:『焰摩迦學友!凡那位諸漏已滅盡的\twnr{阿羅漢}{5.0}比丘,他以身體崩解,死後是怎樣呢?』焰摩迦學友!被這麼問時,你會怎麼回答呢?」

  「學友!如果他們這麼問我:『焰摩迦學友!凡那位諸漏已滅盡的阿羅漢比丘,他以身體崩解,死後是怎樣呢?』學友!被這麼問時,我會這麼回答:『學友們!色是無常的,凡是無常的,那個是苦的,凡是苦的,那個被滅、已消失;受……想……諸行……識是無常的,凡是無常的,那個是苦的,凡是苦的,那個被滅、已消失。』學友!被這麼問時,我會這麼回答。」

  「焰摩迦學友!好!好!焰摩迦學友!那樣的話,就為了這個義理更清晰(更多量的智),我將為你作譬喻。

  焰摩迦學友!猶如富有的、大富的、大財富的\twnr{屋主}{103.0}或屋主之子,他具足守護者。就對他想要無利益,想要不利,想要不\twnr{軛安穩}{192.0},想要奪命的某位男子出現,他這麼想:『這位富有的、大富的、大財富的屋主或屋主之子,他具足守護者,這是不容易的:強行奪命。讓我侵入後奪命。』他去見那位屋主或屋主之子後這麼說:『大德!願我侍候你。』那位屋主或屋主之子使他侍候。他先起床、後就寢、任何行為都順從、合意行為、可愛言語地侍候。那位屋主或屋主之子相信他為他的朋友,也相信他為知己,而且於他來到信任。學友!當那位男子這麼想:『這位屋主或屋主之子對我已安心。』時,那時,知道他已獨處後,以銳利的刀子奪命。

  焰摩迦學友!你怎麼想它:即使當那位男子去見那屋主或屋主之子,對他這麼說:『大德!願我侍候你。』那時,雖然他就是殺害者,而當是殺害者時,他不知道:『是我的殺害者。』即使當他先起床、後就寢、任何行為都順從、合意行為、可愛言語地侍候時,那時,雖然他就是殺害者,而當是殺害者時,他不知道:『是我的殺害者。』即使當知道他已獨處後,以銳利的刀子奪命時,那時,雖然他就是殺害者,而當是殺害者時,他不知道:『是我的殺害者。』嗎?」

  「是的,學友!」

  「同樣的,焰摩迦學友!\twnr{未聽聞的一般人}{74.0}是聖者的未看見者,聖者法的不熟知者,在聖者法上未被教導者;是善人的未看見者,\twnr{善人法}{76.0}的不熟知者,在善人法上未被教導者,\twnr{認為}{964.0}色是我,\twnr{或我擁有色}{13.0},或色在我中,\twnr{或我在色中}{14.0};受……想……諸行……認為識是我,或我擁有識,或識在我中,或我在識中。

  他不如實知道無常的色為『無常的色』;不如實知道無常的受為『無常的受』;不如實知道無常的想為『無常的想』;不如實知道無常的諸行為『無常的諸行』;不如實知道無常的識為『無常的識』。不如實知道苦的色為『苦的色』……苦的受……苦的想……苦的諸行……不如實知道苦的識為『苦的識』。不如實知道無我色為『無我色』……無我受……無我想……無我諸行……不如實知道無我識為『無我識』。不如實知道\twnr{有為的}{90.0}色為『有為的色』……有為的受……有為的想……有為的諸行……不如實知道有為的識為『有為的識』。不如實知道殺害的色為『殺害的色』……殺害的受為『殺害的受』……殺害的想為『殺害的想』;不如實知道殺害的諸行為『殺害的諸行』;不如實知道殺害的識為『殺害的識』。

  他攀取、緊握、固持色為『\twnr{我的真我}{34.0}』……受……想……諸行……攀取、緊握、固持識為『我的真我』。這些五取蘊被攀取、被緊握,對他轉起長久的不利、苦。

  學友!但\twnr{有聽聞的聖弟子}{24.0}是聖者的看見者……(中略)在善人法上被善教導者,認為色不是我,我不擁有色,色不在我中,我不在色中;受不……想不……諸行不……認為識不是我,我擁有識,識在我中,我在識中。

  他如實知道無常的色為『無常的色』……無常的受……無常的想……無常的諸行……如實知道無常的識為『無常的識』。

  他如實知道苦的色為『苦的色』……苦的受……苦的想……苦的諸行……如實知道苦的識為『苦的識』。

  他如實知道無我色為『無我色』……無我受……無我想……無我諸行……如實知道無我識為『無我識』。

  他如實知道有為的色為『有為的色』……有為的受……有為的想……有為的諸行……如實知道有為的識為『有為的識』。

  他如實知道殺害的色為『殺害的色』……殺害的受……殺害的想……如實知道殺害的諸行為『殺害的諸行』;如實知道殺害的識為『殺害的識』。

  他\twnr{不攀取}{717.0}、不緊握、不固持色為『我的真我』……受……想……諸行……不攀取、不緊握、不固持識為『我的真我』。這些五取蘊不被攀取、不被緊握,對他轉起長久的利益、安樂。」

  「這是這樣,舍利弗學友!像你那樣的是對凡\twnr{同梵行的}{138.0}尊者們有憐愍的、想要利益的、教誡的教誡者。而且,聽聞尊者舍利弗的這個說法後,不執取後我的心從諸漏被解脫了。」



\sutta{86}{86}{阿奴羅度經}{https://agama.buddhason.org/SN/sn.php?keyword=22.86}
  \twnr{被我這麼聽聞}{1.0}:

  \twnr{有一次}{2.0},\twnr{世尊}{12.0}住在毘舍離大林\twnr{重閣}{213.0}講堂。

  當時,\twnr{尊者}{200.0}阿奴羅度住在世尊的不遠處的\twnr{林野}{142.0}小屋中。

  那時,眾多其他外道\twnr{遊行者}{79.0}去見尊者阿奴羅度。抵達後,與尊者阿奴羅度一起互相問候。交換應該被互相問候的友好交談後,在一旁坐下。在一旁坐下的那些其他外道遊行者對尊者阿奴羅度說這個:

  「阿奴羅度\twnr{道友}{201.0}!凡當\twnr{安立}{143.0}那位最上人、最高人、得到最高成就的\twnr{如來}{4.0}時,那位如來在這四個地方安立:『死後如來存在』,或『死後如來不存在』,或『\twnr{死後如來存在且不存在}{354.0}』,或『死後如來既非存在也非不存在』。」

  在這麼說時,尊者阿奴羅度對那些其他外道遊行者說這個:

  「道友!凡當安立那位最上人、最高人、得到最高成就的如來時,那位如來\twnr{從除了這四個地方外}{x427}安立:『死後如來存在』,或『死後如來不存在』,或『死後如來存在且不存在』,或『死後如來既非存在也非不存在』。」

  在這麼說時,[那些]其他外道遊行者們對尊者阿奴羅度說這個:

  「這位必將是出家不久的新\twnr{比丘}{31.0},又或他是愚笨的、無能的\twnr{上座}{135.0}。」

  那時,其他外道遊行者們以新的之語以及愚笨的之語貶抑尊者阿奴羅度後,從座位起來後離開。

  那時,在那些其他外道遊行者離開不久,尊者阿奴羅度想這個:「如果那些其他外道遊行者進一步問我問題,當怎樣回答那些其他外道遊行者時,我才\twnr{會是世尊的所說之說者}{115.0},而且不會以不實的誹謗世尊,以及會\twnr{法隨法地回答}{415.0},而任何如法的種種說不會來到應該被呵責處呢?」

  那時,尊者阿奴羅度去見世尊。抵達後……(中略)在一旁坐下的尊者阿奴羅度對世尊說這個:

  「\twnr{大德}{45.0}!這裡,我住在世尊的不遠處的林野小屋中,大德!那時,眾多其他外道遊行者來見我……(中略)對我說這個:『阿奴羅度道友!凡當安立那位最上人、最高人、得到最高成就的如來時,那位如來在這四個地方安立:「死後如來存在」,或不存在……或存在且不存在……或「死後如來既非存在也非不存在」。』大德!在這麼說時,我對那些其他外道遊行者說這個:『道友!凡當安立那位最上人、最高人、得到最高成就的如來時,那位如來從除了這四個地方外安立:「死後如來存在」……(中略)或「死後如來既非存在也非不存在」。』在這麼說時,那些其他外道遊行者對我說這個:『這位必將是出家不久的新比丘,又或他是愚笨的、無能的上座。』大德!那時,那些其他外道遊行者以新的之語以及愚笨的之語貶抑我後,從座位起來後離開。

  大德!在那些其他外道遊行者離開不久,那個我想這個:『如果那些其他外道遊行者進一步問我問題,當怎樣回答那些其他外道遊行者時,我才會是世尊的所說之說者,而且不會以不實的誹謗世尊,以及會法隨法地回答,而任何如法的種種說不會來到應該被呵責處?』」

  「阿奴羅度!你怎麼想它:色是常的,或是無常的?」

  「無常的,大德!」

  「那麼,凡為無常的,那是苦的或樂的?」

  「苦的,大德!」

  「那麼,凡為無常的、苦的、\twnr{變易法}{70.0},適合認為它:『\twnr{這是我的}{32.0},\twnr{我是這個}{33.0},\twnr{這是我的真我}{34.1}。』嗎?」

  「大德!這確實不是。」

  「受……想……諸行……識是常的,或是無常的?」

  「無常的,大德!」……(中略)

  「……因此,在這裡,這麼看的……(中略)他知道:『……\twnr{不再有此處[輪迴]的狀態}{21.1}。」

  「阿奴羅度!你怎麼想它:你認為『色是如來。』嗎?」

  「大德!這確實不是。」

  「受……想……諸行……你認為『識是如來。』嗎?」

  「大德!這確實不是。」

  「阿奴羅度!你怎麼想它:你認為『如來是在色中。』嗎?」

  「大德!這確實不是。」

  「你認為『除了色外有如來。』嗎?」

  「大德!這確實不是。」

  「在受中……(中略)除了受外……(中略)在想中……除了想外……在諸行中……除了在諸行外……在識中……你認為『除了識外有如來。』嗎?」

  「大德!這確實不是。」

  「阿奴羅度!你怎麼想它:你認為『色受想諸行識是如來。』嗎?」

  「大德!這確實不是。」

  「阿奴羅度!你怎麼想它:你認為『這位那個無色者……無受者……無想者……無行者……無識者是如來。』嗎?」

  「大德!這確實不是。」

  「阿奴羅度!而在這裡,當在此生中真實的、實際的如來未被你得到時,你的那個記說:『道友!凡當安立那位最上人、最高人、得到最高成就的如來時,那位如來從除了這四個地方外安立:「死後如來存在」……或「死後如來既非存在也非不存在」。』是否是適當的呢?」

  「大德!這確實不是。」

  「阿奴羅度!\twnr{好}{44.0}!好!阿奴羅度!以前與現在,\twnr{我只告知苦,連同苦的滅}{x428}。」[\suttaref{SN.44.2}]



\sutta{87}{87}{跋迦梨經}{https://agama.buddhason.org/SN/sn.php?keyword=22.87}
  \twnr{有一次}{2.0},\twnr{世尊}{12.0}住在王舍城栗鼠飼養處的竹林中。

  當時,生病的、受苦的、重病的\twnr{尊者}{200.0}跋迦梨住在陶匠住處。

  那時,尊者跋迦梨召喚看護們:

  「來!\twnr{學友}{201.0}們!你們去見世尊。抵達後,請你們以我的名義\twnr{以頭禮拜世尊的足}{40.0}:『\twnr{大德}{45.0}!跋迦梨\twnr{比丘}{31.0}是生病者、受苦者、重病者,他以頭禮拜世尊的足。』以及請你們這麼說:『大德!請世尊\twnr{出自憐愍}{121.0},去見跋迦梨比丘,\twnr{那就好了}{44.0}!』」

  「是的,學友!」那些比丘回答尊者跋迦梨後,去見世尊。抵達後,向世尊\twnr{問訊}{46.0}後,在一旁坐下。在一旁坐下的那些比丘對世尊說這個:「大德!跋迦梨比丘是生病者、受苦者、重病者,他以頭禮拜世尊的足,以及他這麼說:『大德!請世尊出自憐愍,去見跋迦梨比丘,那就好了!』」

  世尊以沈默狀態同意。

  那時,世尊穿衣、拿起衣鉢後,去見尊者跋迦梨。

  尊者跋迦梨看見正從遠處到來的世尊。看見後,\twnr{在臥床上移動}{386.0}。

  那時,世尊對尊者跋迦梨說這個:

  「夠了,跋迦梨!你不要在臥床上移動,有這些設置的座位,我將坐在那裡。」

  世尊在設置的座位坐下。坐下後,世尊對尊者跋迦梨說這個:

  「跋迦梨!是否能被你忍受?\twnr{是否能被[你]維持生活}{137.0}?是否苦的感受減退、不增進,減退的結局被知道,非增進?」

  「大德!不能被我忍受,不能被[我]維持,我強烈苦的感受增進、不減退,增進的結局被知道,非減退。」

  「跋迦梨!是否你沒有任何後悔?沒有任何悔憾?」

  「大德!我當然有不少的後悔,不少的悔憾。」

  「跋迦梨!那麼,是否自己從戒不責備你?」

  「大德!自己從戒不責備我。」

  「跋迦梨!如果自己從戒不責備你,那麼,你有什麼後悔與什麼悔憾呢?」

  「大德!我想去見世尊很久了,但在我的身體上沒有我能去見世尊那個程度的力氣。」

  「夠了,跋迦梨!以見這腐臭之身對你如何呢?跋迦梨!\twnr{凡見法者他見我;凡見我者他見法}{x429},跋迦梨!因為,見法者見我;見我者見法。

  跋迦梨!你怎麼想它:色是常的,或是無常的?」

  「無常的,大德!」

  「那麼,凡為無常的,那是苦的或樂的?」

  「苦的,大德!」

  「那麼,凡為無常的、苦的、\twnr{變易法}{70.0},適合認為它:『\twnr{這是我的}{32.0},\twnr{我是這個}{33.0},\twnr{這是我的真我}{34.1}。』嗎?」

  「大德!這確實不是。」

  「受……想……諸行……識是常的,或是無常的?」

  「無常的,大德!」……(中略)

  「……『……這是我的真我。』嗎?」

  「大德!這確實不是。」

  「……因此,在這裡,這麼看的……(中略)他知道:『……\twnr{不再有此處[輪迴]的狀態}{21.1}。』」

  那時,世尊以這個教誡教誡尊者跋迦梨後,從座位起來後,離開去\twnr{耆闍崛山}{258.0}。

  那時,當世尊離開不久,尊者跋迦梨召喚看護們:

  「來!學友們!使我登上臥床後,去仙吞山坡的黑岩處。像我這樣的,怎麼會想應該被死在住家中呢?」

  「是的,\twnr{學友}{201.0}!」那些比丘回答尊者跋迦梨後,使尊者跋迦梨上臥床後,去仙吞山坡的黑岩處。

  那時,那天剩餘的白天與夜晚,世尊住在耆闍崛山。

  那時,在夜已深時,容色絕佳的兩位天神使整個耆闍崛山發光後,去見世尊。……(中略)在一旁站立。在一旁站立的一位天神對世尊說這個:

  「大德!跋迦梨比丘\twnr{意圖解脫}{x430}。」

  另一位天神對世尊說這個:

  「大德!那位\twnr{善解脫}{28.0}者將解脫。」

  那些天神說這個,說這個後,向世尊問訊、\twnr{作右繞}{47.0}後,就在那裡消失。

  那時,那夜過後,世尊召喚比丘們:

  「來!比丘們!請你們去見跋迦梨比丘。抵達後,請你們對跋迦梨比丘這麼說:『跋迦梨學友!請你聽世尊與兩位天神的言語:學友!這夜,在夜已深時,容色絕佳的兩位天神使整個耆闍崛山發光後,去見世尊。抵達後,向世尊問訊後,在一旁站立。學友!在一旁站立的一位天神對世尊說這個:「大德!跋迦梨比丘意圖解脫。」另一位天神對世尊說這個:「大德!那位善解脫者將解脫。」跋迦梨學友!而世尊對你這麼說:「不要害怕,跋迦梨!不要害怕,跋迦梨!對你來說,死必將是無惡的,無惡的命終。」』」

  「是的,大德!」那些比丘回答世尊後,去見尊者跋迦梨。抵達後,對尊者跋迦梨說這個:

  「跋迦梨學友!請你聽世尊與兩位天神的言語。」

  那時,尊者跋迦梨召喚看護們:

  「來!學友們!請你們使我下臥床,像我這樣的,怎能想坐在高座後,那位世尊的教誡應該被聽聞呢?」

  「是的,學友!」那些比丘回答尊者跋迦梨後,使尊者跋迦梨下臥床。

  「學友!這夜,在夜已深時,容色絕佳的兩位天神……(中略)在一旁站立,學友!在一旁站立的一位天神對世尊說這個:『大德!跋迦梨比丘意圖解脫。』另一位天神對世尊說這個:『大德!那位善解脫者將解脫。』跋迦梨學友!而世尊對你這麼說:「不要害怕,跋迦梨!不要害怕,跋迦梨!對你來說,死必將是無惡的,無惡的命終。』」

  「學友們!那樣的話,請你們以我的名義以頭禮拜世尊的足:『大德!跋迦梨比丘是生病者、受苦者、重病者,他以頭禮拜世尊的足。』以及請你們這麼說:『色是無常的,大德!我對那個不懷疑。「凡是無常的,那個是苦的。」我不懷疑。「凡是無常的、苦的、變易法,在那裡,對我來說沒有意欲,或貪,或情愛。」我不懷疑。受是無常的,大德!我對那個不懷疑。「凡是無常的,那個是苦的。」我不懷疑。「凡是無常的、苦的、變易法,在那裡,對我來說沒有意欲,或貪,或情愛。」我不懷疑。想……諸行是無常的,大德!我對那個不懷疑。「凡是無常的,那個是苦的。」我不懷疑。「凡是無常的、苦的、變易法,在那裡,對我來說沒有意欲,或貪,或情愛。」我不懷疑。識是無常的,大德!我對那個不懷疑。「凡是無常的,那個是苦的。」我不懷疑。「凡是無常的、苦的、變易法,在那裡,對我來說沒有意欲,或貪,或情愛。」我不懷疑。』」

  「是的,學友!」那些比丘回答尊者跋迦梨後離開。

  那時,尊者跋迦梨在那些比丘離開不久,取刀。

  那時,那些比丘去見世尊。抵達後,在一旁坐下。在一旁坐下的那些比丘對世尊說這個:

  「大德!跋迦梨比丘是生病者、受苦者、重病者,他以頭禮拜世尊的足,且這麼說:『色是無常的,大德!我對那個不懷疑。『凡是無常的,那個是苦的。』我不懷疑。『凡是無常的、苦的、變易法,在那裡,對我來說沒有意欲,或貪,或情愛。』我不懷疑。受……想……諸行……識是無常的,大德!我對那個不懷疑。『凡是無常的,那個是苦的。』我不懷疑。『凡是無常的、苦的、變易法,在那裡,對我來說沒有意欲,或貪,或情愛。』我不懷疑。」

  那時,世尊召喚比丘們:

  「比丘們!我們走,我們將去仙吞山坡的黑岩處,在該處\twnr{善男子}{41.0}跋迦梨已取刀。」

  「是的,\twnr{大德}{45.0}!」那些比丘回答世尊。

  那時,世尊與眾多比丘一起去仙吞山坡的黑岩處。

  那時,世尊從遠處看見就在臥床上躺臥的、肩膀轉回(側轉)的尊者跋迦梨。

  當時,黑闇雲煙正走到東方,走到西方;走到北方;走到南方;走到上方;走到下方;走到四方的中間方位。

  那時,世尊召喚比丘們:

  「比丘們!你們看見黑闇雲煙走到東方……(中略)走到四方的中間方位嗎?」

  「是的,大德!」

  「比丘們!這是魔\twnr{波旬}{49.0}探求善男子跋迦梨的識:『善男子跋迦梨的識被住立在哪裡?』比丘們!而以識不被住立,善男子跋迦梨已\twnr{般涅槃}{72.0}。」



\sutta{88}{88}{阿說示經}{https://agama.buddhason.org/SN/sn.php?keyword=22.88}
  \twnr{有一次}{2.0},\twnr{世尊}{12.0}住在王舍城栗鼠飼養處的竹林中。

  當時,生病的、受苦的、重病的\twnr{尊者}{200.0}阿說示住在迦葉園。

  那時,尊者阿說示召喚看護們:

  「來!\twnr{學友}{201.0}!請你們去見世尊。抵達後,請你們以我的名義\twnr{以頭禮拜世尊的足}{40.0}:『\twnr{大德}{45.0}!阿說示\twnr{比丘}{31.0}是生病者、受苦者、重病者,他以頭禮拜世尊的足。』以及請你們這麼說:『大德!請世尊\twnr{出自憐愍}{121.0},去見阿說示比丘,\twnr{那就好了}{44.0}!』」

  「是的,學友!。」那些比丘回答尊者阿說示後,去見世尊。抵達後,向世尊\twnr{問訊}{46.0}後,在一旁坐下。在一旁坐下的那些比丘對世尊說這個:

  「大德!阿說示比丘是生病……(中略)大德!請世尊出自憐愍,去見阿說示比丘,那就好了!」

  世尊以沈默狀態同意。

  那時,世尊傍晚時,從\twnr{獨坐}{92.0}出來,去見尊者阿說示。

  尊者阿說示看見正從遠處到來的世尊。看見後,\twnr{在臥床上移動}{386.0}。

  那時,世尊對尊者阿說示說這個:

  「夠了,阿說示!你不要在臥床上移動,有這些設置的座位,我將坐在那裡。」

  世尊在設置的座位坐下。坐下後,世尊對尊者阿說示說這個:

  「阿說示!是否能被你忍受?\twnr{是否能被[你]維持生活}{137.0}?……(中略)減退的結局被知道,非增進?」

  「大德!不能被我忍受……(中略)增進的結局被知道,非減退。」

  「阿說示!是否你沒有任何後悔?沒有任何悔憾?」

  「大德!我當然有不少的後悔,不少的悔憾。」

  「阿說示!那麼,是否自己從戒不責備你?」

  「大德!自己從戒不責備我。」

  「阿說示!如果自己從戒不責備你,那麼,你有什麼後悔與什麼悔憾呢?」

  「大德!之前,在生病時我住於一再使身行變得寧靜,[現在]那個我沒得到定。大德!那個沒得到那個定的我這麼想:『是否我退失呢?』」

  「阿說示!凡那些定為核心、定為沙門性的沙門、婆羅門,當沒得到那個定時他們這麼想:『我們是否退失呢?』

  阿說示!你怎麼想它:色是常的,或是無常的?」

  「無常的,大德!」……(中略)

  「識……(中略)因此,在這裡……(中略)這麼看的……(中略)他知道:『……\twnr{不再有此處[輪迴]的狀態}{21.1}。』

  他如果感受樂受,知道:『那是無常的。』知道:『是不被固執的。』知道:『是不被歡喜的。』他如果感受苦受,知道:『那是無常的。』知道:『是不被固執的。』知道:『是不被歡喜的。』他如果感受不苦不樂受,知道:『那是無常的。』……(中略)知道:『是不被歡喜的。』他如果感受樂受,離結縛地感受它;他如果感受苦受,離結縛地感受它;他如果感受不苦不樂受,離結縛地感受它。

  如果他\twnr{當感受身體終了的感受時}{720.0},知道:『我感受身體終了的感受。』\twnr{當感受生命終了的感受時}{721.0},知道:『我感受生命終了的感受。』他知道:『以身體的崩解,隨後生命耗盡,就在這裡,一切所感受的、不被歡喜的將成為清涼[,遺骸被留下(剩下)-\suttaref{SN.12.51}]。』

  阿說示!猶如\twnr{緣於}{252.0}油與緣於燈芯,油燈燃燒。就從那個油與燈芯的耗盡,無食物者被熄滅。同樣的,阿說示!比丘當感受身體終了的感受時,知道:『我感受身體終了的感受。』或當感受生命終了的感受時,知道:『我感受生命終了的感受。』他知道:『以身體的崩解,隨後生命耗盡,就在這裡,一切所感受的、不被歡喜的將成為清涼[,遺骸被留下(剩下)]。』」



\sutta{89}{89}{差摩經}{https://agama.buddhason.org/SN/sn.php?keyword=22.89}
  \twnr{有一次}{2.0},眾多\twnr{上座}{135.0}\twnr{比丘}{31.0}住在\twnr{憍賞彌}{140.0}瞿師羅園。

  當時,生病的、受苦的、重病的\twnr{尊者}{200.0}差摩住在棗樹園。

  那時,上座比丘們傍晚時,從\twnr{獨坐}{92.0}出來,召喚尊者陀裟:

  「來!陀裟\twnr{學友}{201.0}!請你去見差摩比丘。抵達後,請你對差摩比丘這麼說:『差摩學友!上座們對你這麼說:「學友!是否能被你忍受?\twnr{是否能被[你]維持生活}{137.0}?是否苦的感受減退、不增進,減退的結局被知道,非增進?」』」

  「是的,學友們!」尊者陀裟回答上座比丘們後,去見尊者差摩。抵達後,對尊者差摩說這個:「差摩學友!上座們對你這麼說:『學友!是否能被你忍受?……(中略)非增進?』」

  「學友!不能被我忍受,不能被[我]維持……(中略)增進的結局被知道,非減退。」

  那時,尊者陀裟去見上座比丘們。抵達後,對上座比丘們說這個:「學友們!差摩比丘這麼說:『學友!不能被我忍受……(中略)增進的結局被知道,非減退。』」

  「來!陀裟學友!請你去見差摩比丘。抵達後,請你對差摩比丘這麼說:『差摩學友!上座們對你這麼說:「學友!被\twnr{世尊}{12.0}說的這些\twnr{五取蘊}{36.0},即:色取蘊、受取蘊、想取蘊、行取蘊、識取蘊,在這些五取蘊中有任何尊者差摩認為是我,或屬於我的嗎?」』」

  「是的,學友們!」尊者陀裟回答上座比丘們後,去見尊者差摩。抵達後……(中略)差摩學友!上座們對你這麼說:「學友!被世尊說的這些五取蘊,即:色取蘊……(中略)識取蘊,在這些五取蘊中有任何尊者差摩認為是我,或屬於我的嗎?」

  「學友!被世尊說的這些五取蘊,即:色取蘊……(中略)識取蘊,學友!在這些五取蘊中沒有任何我認為是我,或屬於我的。」

  那時,尊者陀裟去見上座比丘們。抵達後,對上座比丘們說這個:「學友們!差摩比丘這麼說:『學友!被世尊說的這些五取蘊,即:色取蘊……(中略)識取蘊,學友!在這些五取蘊中沒有任何我認為是我,或屬於我的。』」

  「來!陀裟學友!請你去見差摩比丘。抵達後,請你對差摩比丘這麼說:『差摩學友!上座們對你這麼說:「學友!被世尊說的這些五取蘊,即:色取蘊……(中略)識取蘊,如果在這些五取蘊中確實沒有任何尊者差摩認為是我,或屬於我的,那樣的話,尊者差摩是諸漏已滅盡的\twnr{阿羅漢}{5.0}。」』」

  「是的,學友們!」尊者陀裟回答上座比丘們後,去見尊者差摩……(中略)差摩學友!上座們對你這麼說:「學友!被世尊說的這些五取蘊,即:色取蘊……(中略)識取蘊,如果在這些五取蘊中確實沒有任何尊者差摩認為是我,或屬於我的,那樣的話,那尊者差摩是諸漏已滅盡的阿羅漢。」

  「學友!被世尊說的這些五取蘊,即:色取蘊……(中略)識取蘊,學友!在這些五取蘊中沒有任何我認為是我,或屬於我的,但我不是諸漏已滅盡的阿羅漢。學友!此外,在五取蘊上『\twnr{我是}{894.0}』被我\twnr{到達}{x431},但我不認為『\twnr{我是這個}{33.0}』。」

  那時,尊者陀裟去見上座比丘們……(中略)對上座比丘們說這個:「學友們!差摩比丘這麼說:『學友!被世尊說的這些五取蘊,即:色取蘊……(中略)識取蘊,學友!在這些五取蘊中沒有任何我認為是我,或屬於我的,但我不是諸漏已滅盡的阿羅漢。學友!此外,在五取蘊上「我是」被我到達,但我不認為「我是這個。」』」

  「來!陀裟學友!請你去見差摩比丘。抵達後,請你對差摩比丘這麼說:『差摩學友!上座們對你這麼說:「差摩學友!你說這個『我是』,什麼是你說的這個『我是』呢?你說色是『我是』嗎?你說除了色外是『我是』嗎?受……想……諸行……你說識是『我是』嗎?你說除了識外是『我是』嗎?差摩學友!你說這個『我是』,什麼是你說的這個『我是』呢?」』」

  「是的,學友們!」尊者陀裟回答上座比丘們後,去見尊者差摩。抵達後,對尊者差摩說這個:「差摩學友!上座們對你這麼說:『差摩學友!你說這個「我是」,什麼是你說的這個「我是」呢?你說色是「我是」嗎?你說除了色外是「我是」嗎?受……想……諸行……你說識是「我是」嗎?你說除了識外是「我是」嗎?差摩學友!你說這個「我是」,什麼是你說的這個「我是」呢?』」

  「夠了,陀裟學友!為何為這個來回走,學友!請你取拐杖來,我將去見上座比丘們。」

  那時,尊者差摩拄著拐杖後,去見上座比丘們。抵達後,與上座比丘們一起互相問候。交換應該被互相問候的友好交談後,在一旁坐下。上座比丘們對在一旁坐下的尊者差摩說這個:

  「差摩學友!你說這個『我是』,什麼是你說的這個『我是』呢?你說色是『我是』嗎?你說除了色外是『我是』嗎?受……想……諸行……你說識是『我是』嗎?你說除了識外是『我是』嗎?差摩學友!你說這個『我是』,什麼是你說的這個『我是』呢?」

  「學友們!我既不說色是『我是』,也不說除了色外是『我是』;不說受……不說想……不說諸行……我既不說識是『我是』,也不說除了識外是『我是』。學友們!此外,在五取蘊上『我是』被我到達,但我不認為『我是這個。』

  學友們!猶如青蓮,或紅蓮,或白蓮的香,凡這麼說,是『葉的香』,或『\twnr{容色的香}{x432}』,或『花蕊絲的香』,那位說者正確地說嗎?」

  「學友!這確實不是。」

  「學友們!那麼,如怎樣回答者會正確地回答?」

  「學友!是『花的香』,回答者會正確地回答。」

  「同樣的,學友們!我既不說色是『我是』,也不說除了色外是『我是』;不說受……不說想……不說諸行……我既不說識是『我是』,也不說除了識外是『我是』。學友們!此外,在五取蘊上『我是』被我到達,但我不認為『我是這個。』

  學友們!即使聖弟子的\twnr{五下分結}{134.0}被捨斷,那時,他有:凡在五取蘊上殘留的『我是』之慢、『我是』之意欲、『我是』之\twnr{煩惱潛在趨勢}{253.1}未被根除。過些時候,他在五取蘊上住於\twnr{隨看著生滅}{427.0}:『像這樣是色,像這樣是色的\twnr{集起}{67.0},像這樣是色的滅沒;像這樣是受……像這樣是想……像這樣是諸行……像這樣是識,像這樣是識的集起,這樣是識的滅沒。』當他在這些五取蘊上住於隨看著生滅時,凡他還有在五取蘊上殘留的『我是』之慢、『我是』之意欲、『我是』之煩惱潛在趨勢未被根除者,它也走到根除。

  學友們!猶如被污染、著垢的衣服,主人們將它交給洗濯者。那位洗濯者在鹽中,或在鹼中,或在牛糞中踩踏後,在清水中清洗。即使那件衣服變成遍純淨的、皎潔的,那時,它有:凡正是殘留的鹽味,或鹼味,或牛糞味未被根除。洗濯者將它給主人。主人們將它放在香味遍滿的箱中,凡它還有殘留的鹽味,或鹼味,或牛糞味未被根除者,它也走到根除。

  同樣的,學友們!即使聖弟子的五下分結被捨斷,那時,他有:凡在五取蘊上殘留的『我是』之慢、『我是』之意欲、『我是』之煩惱潛在趨勢未被根除。過些時候,他在五取蘊上住於隨看著生滅:『像這樣是色,像這樣是色的集起,像這樣是色的滅沒;像這樣是受……像這樣是想……像這樣是諸行……像這樣是識,像這樣是識的集起,這樣是識的滅沒。』當他在這些五取蘊上住於隨看著生滅時,凡他還有在五取蘊上殘留的『我是』之慢、『我是』之意欲、『我是』之煩惱潛在趨勢未被根除者,它也走到根除。」

  在這麼說時,上座比丘們對尊者差摩說這個:

  「我們詢問尊者差摩,非惱害之期待,而是尊者差摩能詳細地(廣說地)告知、教導、\twnr{使知}{143.0}、建立、開顯、解析、闡明那位世尊的教說。這件事,尊者差摩已詳細地告知、教導、使知、建立、開顯、解析、闡明那位世尊的教說。」

  尊者差摩說這個,悅意的上座比丘們歡喜尊者差摩所說。

  還有,\twnr{在當這個解說被說時}{136.0},約六十位上座比丘與尊者差摩的心,不執取後從諸\twnr{漏}{188.0}被解脫。



\sutta{90}{90}{闡陀經}{https://agama.buddhason.org/SN/sn.php?keyword=22.90}
  \twnr{有一次}{2.0},眾多\twnr{上座}{135.0}\twnr{比丘}{31.0}住在波羅奈仙人墜落處的鹿林。

  那時,\twnr{尊者}{200.0}闡陀傍晚時,從\twnr{獨坐}{92.0}出來,取\twnr{鑰匙}{x433}後,從僧房去僧房後,對上座比丘們說這個:

  「尊者上座們!請教誡我,尊者上座們!請訓誡我,尊者上座們!請為我作法說,依之我能見法。」

  在這麼說時,上座比丘們對尊者闡陀說這個:

  「闡陀\twnr{學友}{201.0}!色是無常的,受是無常的,想是無常的,諸行是無常的,識是無常的;色是無我,受……想……諸行……識是無我;\twnr{一切行是無常的}{628.0},\twnr{一切法是無我}{23.1}。」

  那時,尊者闡陀想這個:

  「這個,我也這麼想:『色是無常的,受……想……諸行……識是無常的;色是無我,受……想……諸行……識是無我;一切行是無常的,一切法是無我。』然而,在\twnr{一切行的止}{55.0},一切\twnr{依著}{198.0}的\twnr{斷念}{211.0},渴愛的滅盡、\twnr{離貪}{77.0}、\twnr{滅}{68.0}、涅槃上我的心不躍入、不明淨、不\twnr{住立}{424.0}、不\twnr{勝解}{257.0},生起戰慄、執取,心意回轉:『那麼,那樣的話,誰是我的真我?』然而,這樣不是見法者。誰會為我教導法,以便我會見法呢?」

  那時,尊者闡陀想這個:

  「這位尊者阿難住在\twnr{憍賞彌}{140.0}瞿師羅園,是大師的稱讚者,同時也是同梵行智者的敬重者,且尊者阿難能夠為我教導法,以便我會見法。又,在尊者阿難上我有那樣程度的信賴,讓我去見尊者阿難。」

  那時,尊者闡陀收起臥坐具、拿起衣鉢後,去憍賞彌瞿師羅園見尊者阿難。抵達後,與尊者阿難互相歡迎……(中略)在一旁坐下的尊者闡陀對尊者阿難說這個:

  「阿難學友!有這一次,我住在波羅奈仙人墜落處的鹿林。學友!那時,我傍晚時,從獨坐出來,取鑰匙後,從僧房去僧房後,對上座比丘們說這個:『尊者上座們!請教誡我,尊者上座們!請訓誡我,尊者上座們!請為我作法說,依之我會見法。』學友!在這麼說時,上座比丘們對我說這個:『闡陀學友!色是無常的,受……想……諸行……識是無常的;色是無我……(中略)識是無我;一切行是無常的,一切法是無我。』

  學友!那個我這麼想:『這個,我也這麼想:「色是無常的……(中略)識是無常的;色是無我,受……想……諸行……識是無我;一切行是無常的,一切法是無我。」然而,在一切行的止,一切依著的斷念,渴愛的滅盡、離貪、滅、涅槃上我的心不躍入、不明淨、不住立、不勝解,生起戰慄、執取,心意回轉:『那麼,那樣的話,誰是我的真我?」然而,這樣不是見法者。誰會為我教導法,以便我會見法呢?』

  學友!那個我這麼想:『這位尊者阿難住在憍賞彌瞿師羅園,是大師的稱讚者,同時也是同梵行智者的敬重者,且尊者阿難能夠為我教導法,以便我會見法。又,在尊者阿難上我有那樣程度的信賴,讓我去見尊者阿難。』請尊者阿難教誡我,請尊者阿難訓誡我,請尊者阿難為我作法說,依之我會見法。」

  「以只這樣的我們確實也對尊者闡陀是滿意的:那位尊者闡陀公開地作了,切斷樁(障礙)。請闡陀學友傾耳,你是能夠了知法者。」

  那時,尊者闡陀的\twnr{廣大喜、欣悅立刻生起}{x434}:「聽說我是能夠了知法者。」

  「闡陀學友!這個被我當面聽聞,當被迦旃延氏比丘當面領受\twnr{世尊}{12.0}教誡時:『迦旃延!這世間多數是依止兩種者:實有性(實有的觀念)與虛無性。迦旃延!對以正確之慧如實看見世間集者,凡關於世間的虛無性它不存在(對世間沒有虛無的觀念);迦旃延!對以正確之慧如實看見世間滅者,凡關於世間的實有性它不存在。

  迦旃延!這世間多數有攀住、執取、執持的束縛,但對那個攀住、執取、心的依處、執持、\twnr{煩惱潛在趨勢}{253.1},這位\twnr{不攀取}{717.0}、不執取,他不執持『我的真我』,『當生起時僅苦生起;當被滅時苦被滅。』他不疑惑、不懷疑,不從緣於他人就在這裡有他的智(他的智存在),迦旃延!這個情形是正見。

  迦旃延!『一切存在』,這是一邊(極端);『一切不存在』,這是第二邊,迦旃延!不走入這些那些兩個邊後,如來以中間教導法:『以無明\twnr{為緣}{180.0}有諸行(而諸行存在);以行為緣有識……(中略)這樣是這整個\twnr{苦蘊}{83.0}的\twnr{集}{67.0}。但就以無明的\twnr{無餘褪去與滅}{491.0}有行滅(而行滅存在)……(中略)這樣是這整個苦蘊的滅。」[\suttaref{SN.12.15}]』」

  「這是這樣,阿難學友!像你那樣的是對凡同梵行的尊者們有憐愍的、想要利益的、教誡的教誡者。而且,聽聞尊者阿難的這個說法後,法已被我\twnr{現觀}{53.0}了。[≃\suttaref{SN.22.85}]」



\sutta{91}{91}{羅侯羅經}{https://agama.buddhason.org/SN/sn.php?keyword=22.91}
  起源於舍衛城。

  那時,\twnr{尊者}{200.0}羅侯羅去見\twnr{世尊}{12.0}。抵達後……(中略)在一旁坐下的尊者羅侯羅對世尊說這個:

  「\twnr{大德}{45.0}!當怎樣知、當怎樣見時,在這個有識之身上\twnr{與在一切外部諸相上}{559.0}沒有\twnr{我作}{22.0}、\twnr{我所作}{25.0}、\twnr{慢煩惱潛在趨勢}{26.0}?」

  「羅侯羅!凡任何色:過去、未來、現在,或內、或外,或粗、或細,或下劣、或勝妙,或凡在遠處、在近處,所有色:『\twnr{這不是我的}{32.1},\twnr{我不是這個}{33.1},\twnr{這不是我的真我}{34.2}。』以正確之慧這樣如實看見這個。

  凡任何受……凡任何想……凡任何諸行……凡任何識:過去、未來、現在,或內、或外……(中略)所有識:『這不是我的,我不是這個,這不是我的真我。』以正確之慧這樣如實看見這個。

  羅侯羅!當這樣知、這樣見時,在這個有識之身上與在一切外部諸相上沒有我作、我所作、慢煩惱潛在趨勢。」[\suttaref{SN.18.21}]



\sutta{92}{92}{羅侯羅經第二}{https://agama.buddhason.org/SN/sn.php?keyword=22.92}
  起源於舍衛城。

  在一旁坐下的\twnr{尊者}{200.0}羅侯羅對\twnr{世尊}{12.0}說這個:

  「\twnr{大德}{45.0}!當怎樣知、當怎樣見時,在這個有識之身上\twnr{與在一切外部諸相上}{559.0}離\twnr{我作}{22.0}、\twnr{我所作}{25.0}、慢,成為心意超越\twnr{慢類}{647.0}者、寂靜者、\twnr{善解脫}{28.0}者呢?」

  「羅侯羅!凡任何色:過去、未來、現在,或內、或外……(中略)或凡在遠處、在近處,所有色:『\twnr{這不是我的}{32.1},\twnr{我不是這個}{33.1},\twnr{這不是我的真我}{34.2}。』以正確之慧這樣如實看見這個後,不執取後成為解脫者。

  凡任何受……(中略)凡任何想……凡任何諸行……凡任何識:過去、未來、現在,或內、或外,或粗、或細,或下劣、或勝妙,或凡在遠處、在近處,所有識:『這不是我的,我不是這個,這不是我的真我。』以正確之慧這樣如實看見這個後,不執取後成為解脫者。

  羅侯羅!當這樣知、這樣見時,在這個有識之身上與在一切外部諸相上離我作、我所作、慢,成為心意超越慢類者、寂靜者、善解脫者。」[\suttaref{SN.18.22}]

  上座品第九,其\twnr{攝頌}{35.0}:

  「阿難、低舍、焰摩迦,阿奴羅度、跋迦梨,

   阿說示、差摩、闡陀,羅侯羅二則在後。」





\pin{花品}{93}{102}
\sutta{93}{93}{河經}{https://agama.buddhason.org/SN/sn.php?keyword=22.93}
  起源於舍衛城。

  「\twnr{比丘}{31.0}們!猶如發源於山上、向下流向遠處、急流的河,在它的兩岸處如果有已生的葦草,它們懸掛它(懸掛在河岸);如果也有已生的茅草,它們懸掛它;如果也有已生的燈心草,它們懸掛它;如果也有已生的香草,它們懸掛它;如果也有已生的樹木,它們懸掛它。當男子被它的水流沖走時,即使抓住葦草,它們被破壞(被折斷),他從那個因由來到不幸、災難;即使抓住茅草;即使抓住燈心草;即使抓住香草;即使抓住樹木,它們被破壞,他從那個因由來到不幸、災難。同樣的,比丘們!\twnr{未聽聞的一般人}{74.0}是聖者的未看見者,聖者法的不熟知者,在聖者法上未被教導者;是善人的未看見者,\twnr{善人法}{76.0}的不熟知者,在善人法上未被教導者,他\twnr{認為}{964.0}色是我,\twnr{或我擁有色}{13.0},或色在我中,\twnr{或我在色中}{14.0},他的那個色被破壞,他從那個因由來到不幸、災難;受……想……諸行……他認為識是我,或我擁有識,或識在我中,或我在識中,他的那個識被破壞,他從那個因由來到不幸、災難。

  比丘們!你們怎麼想它:色是常的,或是無常的?」

  「無常的,\twnr{大德}{45.0}!」

  「受……想……諸行……識是常的,或是無常的?」

  「無常的,大德!」

  「因此,在這裡……(中略)這麼看的……(中略)他知道:『……\twnr{不再有此處[輪迴]的狀態}{21.1}。』」



\sutta{94}{94}{花經}{https://agama.buddhason.org/SN/sn.php?keyword=22.94}
  起源於舍衛城。

  「\twnr{比丘}{31.0}們!我不與世間爭論,僅世間與我爭論。比丘們!\twnr{如法之說者}{648.0}不與世間中任何人爭論:比丘們!凡被世間中賢智者們認同為不存在,我也說它『不存在』;比丘們!凡\twnr{被世間中賢智者們認同為存在}{x435},我也說它『存在』。

  比丘們!而什麼是被世間中賢智者們認同為不存在,凡我說『不存在』呢?比丘們!常的、堅固的、恆的、不\twnr{變易法}{70.0}的色,是屬於被世間中賢智者們認同為不存在的,我也說它『不存在』;受……想……諸行……常的、堅固的、恆的、不變易法的識,是屬於被世間中賢智者們認同為不存在的,我也說它『不存在』。比丘們!這是屬於被世間中賢智者們認同為不存在的,我也說它『不存在』。

  比丘們!而什麼是被世間中賢智者們認同為存在,凡我說『存在』呢?比丘們!無常的、苦的、變易法的色,是被世間中賢智者們認同為存在,我也說它『存在』。無常的受……(中略)無常的、苦的、變易法的識,是被世間中賢智者們認同為存在,我也說它『存在』。比丘們!這是被世間中的賢智者們認為存在的,我也說它『存在』。

  比丘們!有\twnr{世間中的世間法}{x436},\twnr{如來}{4.0}\twnr{現正覺}{75.0}、\twnr{現觀}{53.0}它;現正覺、現觀它後,告知、教導、\twnr{告知}{143.0}、建立、開顯、\twnr{解析}{x437}、闡明。比丘們!而什麼是世間中的世間法,如來現正覺、現觀它;現正覺、現觀它後,告知、教導、使知、建立、開顯、解析、闡明呢?比丘們!色是世間中的世間法,如來現正覺、現觀它;現正覺、現觀它後,告知、教導、使知、建立、開顯、解析、闡明。比丘們!凡當在被如來這樣告知、教導、使知、建立、開顯、解析、闡明時,他不知不見,比丘們!我對那位愚癡、無知、盲目、無眼、無知、無見的\twnr{一般人}{x438}要作什麼呢!

  比丘們!受是世間中的世間法……(中略)比丘們!想……比丘們!諸行……比丘們!識是世間中的世間法,如來現正覺、現觀它;現正覺、現觀它後,告知、教導、使知、建立、開顯、解析、闡明。比丘們!凡當在被如來這樣告知、教導、使知、建立、開顯、解析、闡明時,他不知不見,比丘們!我對那位愚癡、無知、盲目、無眼、無知、無見的一般人要作什麼呢!

  比丘們!\twnr{猶如青蓮}{x439},或紅蓮,或白蓮被生於水中,被長於水中,從水中上升後,不被水附著(染著)地住立。同樣的,比丘們!如來被生於世間中,被長於世間中,征服世間後,不被世間附著地居住。」



\sutta{95}{95}{如泡沫團經}{https://agama.buddhason.org/SN/sn.php?keyword=22.95}
  \twnr{有一次}{2.0},\twnr{世尊}{12.0}住在阿毘陀的恒河邊。

  在那裡,世尊召喚\twnr{比丘}{31.0}們:

  「比丘們!猶如這恒河帶來大泡沫團,有眼的男子看、靜觀、如理觀察它;當看、靜觀、如理觀察它時,它看起來像僅空無的,看起來像僅空虛的,看起來像僅不堅實的。比丘們!在泡沫團中會有什麼堅實呢?同樣的,比丘們!凡任何色:過去、未來、現在……(中略)或凡在遠處、在近處,比丘看、靜觀、如理觀察它;當看、靜觀、如理觀察它時,它看起來像僅空無的,看起來像僅空虛的,看起來像僅不堅實的。比丘們!在色中會有什麼堅實呢?

  比丘們!猶如在秋天時期天下著大雨時,水泡在水上生起同時也被滅,有眼的男子看、靜觀、如理觀察它;當看、靜觀、如理觀察它時,它看起來像僅空無的,看起來像僅空虛的,看起來像僅不堅實的。比丘們!在水泡中會有什麼堅實呢?同樣的,比丘們!凡任何受:過去、未來、現在……(中略)凡或凡在遠處、在近處,比丘看、靜觀、如理觀察它;當看、靜觀、如理觀察它時,它看起來像僅空無的,看起來像僅空虛的,看起來像僅不堅實的。比丘們!在受中會有什麼堅實呢?

  比丘們!猶如在夏天的最後一個月正中午(住立的中午)時,陽燄搖動,有眼的男子看、靜觀、如理觀察它;當看、靜觀、如理觀察它時,它看起來像僅空無的,看起來像僅空虛的……(中略)。比丘們!在陽燄中會有什麼堅實呢?同樣的,比丘們!凡任何想……(中略)?

  比丘們!猶如欲求\twnr{心材}{356.0}、尋求心材、進行心材之遍求的男子拿起銳利的斧頭後,進入樹林,在那裡,他看見筆直、新長的、未抽芽結果實的大芭蕉樹幹,他隨即在根處切斷,在根處切斷後在頂端切斷,在頂端切斷後分開芭蕉葉鞘。當分開它的葉鞘時,他連\twnr{膚材}{356.1}也沒得到,從哪裡有心材!有眼的男子看、靜觀、如理觀察它;當看、靜觀、如理觀察它時,它看起來像僅空無的,看起來像僅空虛的,看起來像僅不堅實的。比丘們!在芭蕉樹幹中會有什麼堅實呢?同樣的,比丘們!凡任何諸行:過去、未來、現在……(中略)凡或凡在遠處、在近處,比丘看、靜觀、如理觀察它;當看、靜觀、如理觀察它時,它看起來像僅空無的,看起來像僅空虛的,看起來像僅不堅實的。比丘們!在諸行中會有什麼堅實呢?

  比丘們!猶如\twnr{幻術師}{x440}或幻術師的徒弟在十字路口表演幻術,有眼的男子看、靜觀、如理觀察它;當看、靜觀、如理觀察它時,它看起來像僅空無的,看起來像僅空虛的,看起來像僅不堅實的。比丘們!在幻術中會有什麼堅實呢?同樣的,比丘們!凡任何識:過去、未來、現在……(中略)凡或凡在遠處、在近處,比丘看、靜觀、如理觀察它;當看、靜觀、如理觀察它時,它看起來像僅空無的,看起來像僅空虛的,看起來像僅不堅實的。比丘們!在識中會有什麼堅實呢?

  比丘們!這麼看的\twnr{有聽聞的聖弟子}{24.0}在色上\twnr{厭}{15.0},也在受上……也在想上……也在諸行上……也在識上厭。厭者\twnr{離染}{558.0},從\twnr{離貪}{77.0}被解脫,在已解脫時,\twnr{有『[這是]解脫』之智}{27.0},他知道:『\twnr{出生已盡}{18.0},\twnr{梵行已完成}{19.0},\twnr{應該被作的已作}{20.0},\twnr{不再有此處[輪迴]的狀態}{21.1}。』」

  世尊說這個,說這個後,\twnr{善逝}{8.0}、\twnr{大師}{145.0}更進一步說這個:

  「色如泡沫團,受如水泡,

   想如陽燄,諸行如芭蕉,

   識如幻術,\twnr{被太陽族人教導}{662.1}。

   如是如是靜觀,如理觀察,

   是空無的、空虛的:凡如理看它者。

   而關於這個身體,被廣慧者教導:

   三法的捨斷,你們看見色被捨棄。

   壽、\twnr{暖與識}{x441},當它們捨棄這個身體時,

   那時被拋棄者躺臥,成為無思者、\twnr{其牠者之食物}{x442}。

   像這樣的相續,這是幻術、這是愚癡的虛談者,

   被告知這是殺害者,在這裡堅實不被發現。

   在諸蘊上應該這樣觀察:活力已被發動的比丘,

   不論日或夜:正知、朝向念者。

   應該捨棄所有的結,應該作自己的\twnr{歸依}{284.0},

   應該如頭被燃燒般地行(過生活):\twnr{不死足跡}{x443}的希求者。」



\sutta{96}{96}{牛糞團經}{https://agama.buddhason.org/SN/sn.php?keyword=22.96}
  起源於舍衛城。

  在一旁坐下的那位\twnr{比丘}{31.0}對\twnr{世尊}{12.0}說這個:

  「\twnr{大德}{45.0}!有任何色:常的、堅固的、常恆的、不\twnr{變易法}{70.0}的色,\twnr{將就像那樣永久地住立}{552.0}嗎?

  大德!有任何受:常的、堅固的、常恆的、不變易法的受,將就像那樣永久地住立嗎?大德!有任何想……(中略)大德!有任何諸行:常的、堅固的、常恆的、不變易法的諸行,將就像那樣永久地住立嗎?大德!有任何識:常的、堅固的、常恆的、不變易法的識,將就像那樣永久地住立嗎?」

  「比丘!沒有任何色:常的、堅固的、常恆的、不變易法的色,將就像那樣永久地住立。比丘!沒有任何受……任何想……任何諸行……任何識:常的、堅固的、常恆的、不變易法的識,將就像那樣永久地住立。」

  那時,世尊以手取小牛糞團後,對那位比丘說這個:

  「比丘!連只這樣的\twnr{自體的獲得}{661.0}也沒有常的、堅固的、常恆的、不變易法的,將就像那樣永久地住立。比丘!如果只這樣自體的獲得也有常的、堅固的、常恆的、不變易法的,這為了苦的完全滅盡之梵行生活不被知道。比丘!但因為連只這樣自體的獲得也沒有常的、堅固的、常恆的、不變易法的,因此,這為了苦的完全滅盡之梵行生活被知道。

  比丘!從前,我為剎帝利灌頂王。比丘!當那個我是剎帝利灌頂王時,有八萬四千城市,咕薩瓦帝王都為上首[\ccchref{DA.17}{https://agama.buddhason.org/DA/dm.php?keyword=17}, 263段]。比丘!當那個我是剎帝利灌頂王時,有八萬四千宮殿,達摩宮殿為上首。比丘!當那個我是剎帝利灌頂王時,有八萬四千\twnr{重閣}{213.0},\twnr{大陣列}{787.0}重閣為上首。比丘!當那個我是剎帝利灌頂王時,有八萬四千床座:象牙製的、木\twnr{心材}{356.0}製的、金製的[、銀製的],長羊毛覆蓋的、白羊毛布覆蓋的、繡花毛織布覆蓋的、\twnr{頂級羚鹿皮覆蓋的}{465.0}、有頂篷的,\twnr{兩端有紅色枕墊}{463.0}。比丘!當那個我是剎帝利灌頂王時,有八萬四千頭象:黃金裝飾,黃金旗幟,被金絲網覆蓋的,布薩象王為上首。比丘!當那個我是剎帝利灌頂王時,有八萬四千匹馬,黃金裝飾,黃金旗幟,被金絲網覆蓋的,雷雲馬王為上首。比丘!當那個我是剎帝利灌頂王時,有八萬四千輛車,黃金裝飾,黃金旗幟,被金絲網覆蓋的,最勝車為上首。比丘!當那個我是剎帝利灌頂王時,有八萬四千寶珠,寶珠寶石為上首。比丘!我為……(中略)有八萬四千女人,善吉祥皇后為上首。比丘!我為……(中略)有八萬四千剎帝利隨從,主兵臣寶為上首。比丘!我為……(中略)有八萬四千乳牛:[配戴]黃麻繫繩,青銅牛奶桶。比丘!我為……(中略)有八萬四千\twnr{俱胝}{x444}衣服:\twnr{精緻的亞麻衣}{466.0}、精緻的絲綢衣、\twnr{精緻的毛衣}{464.0}、精緻的木綿衣。比丘!我為……(中略)有八萬四千鍋煮好的食物,早上、傍晚供養食物被帶來。

  比丘!然而那些八萬四千城市中,只有那一個城市凡我那時居住:咕薩瓦帝王都。比丘!然而那些八萬四千宮殿中,只有那一個宮殿凡我那時居住:達摩宮殿。比丘!然而那些八萬四千重閣中,只有那一個重閣凡我那時居住:大陣列重閣。比丘!然而那些八萬四千床座中,只有那一個床座凡我那時受用:或象牙製的,或木心材製的,或金製的,或銀製的。比丘!然而那些八萬四千頭象中,只有那一頭象凡我那時登上:布薩象王。比丘!然而那些八萬四千匹馬中,只有那一匹馬凡我那時登上:雷雲馬王。比丘!然而那些八萬四千輛馬車中,只有那一輛馬車凡我那時登上:最勝車。比丘!然而那些八萬四千位女子中,只有那一位女子凡那時侍候我:或剎帝利女,或\twnr{偉拉米迦女}{x445}。比丘!然而那些八萬四千俱胝衣服中,只有那一對衣服凡我那時裹上:或精緻的亞麻衣,或精緻的絲綢衣,或精緻的毛衣,或精緻的木綿衣。比丘!然而那些八萬四千鍋煮好的食物中,只有那一鍋煮好的食物,從那裡我吃最多一拿哩的飯與放進那裡的咖哩。

  比丘!像這樣那\twnr{一切諸行}{x446}已過去;已被滅;已變易。比丘!諸行是這麼無常的;比丘!諸行是這麼不堅固的;比丘!諸行是這麼不安的。比丘!到那個程度,這就足以在一切諸行上\twnr{厭}{15.0},足以\twnr{離染}{558.0},足以解脫。」



\sutta{97}{97}{指甲尖經}{https://agama.buddhason.org/SN/sn.php?keyword=22.97}
  起源於舍衛城。

  在一旁坐下的那位\twnr{比丘}{31.0}對\twnr{世尊}{12.0}說這個:

  「\twnr{大德}{45.0}!有任何色:常的、堅固的、常恆的、不\twnr{變易法}{70.0}的色,\twnr{將就像那樣永久地住立}{552.0}嗎?

  大德!有任何受:常的、堅固的、常恆的、不變易法的受,將就像那樣永久地住立嗎?大德!有任何想……(中略)大德!有任何諸行:常的、堅固的、常恆的、不變易法的諸行,將就像那樣永久地住立嗎?大德!有任何識:常的、堅固的、常恆的、不變易法的識,將就像那樣永久地住立嗎?」

  「比丘!沒有任何色:常的、堅固的、常恆的、不變易法的色,將就像那樣永久地住立。比丘!沒有任何受……任何想……任何諸行……任何識:常的、堅固的、常恆的、不變易法的識,將就像那樣永久地住立。」

  那時,世尊使微少塵土沾在指甲尖後,對那位比丘說這個:

  「比丘!連只這樣的色也沒有常的、堅固的、常恆的、不變易法的,將就像那樣永久地住立。比丘!如果只這樣的色也有常的、堅固的、常恆的、不變易法的,這為了苦的完全滅盡之梵行生活不被知道。比丘!但因為連只這樣的色也沒有常的、堅固的、常恆的、不變易法的,因此,這為了苦的完全滅盡之梵行生活被知道。

  比丘!連只這樣的受也沒有常的、堅固的、常恆的、不變易法的,將就像那樣永久地住立,比丘!如果連只這樣的受也有常的、堅固的、常恆的、不變易法的,這為了苦的完全滅盡之梵行生活不被知道。比丘!但因為連只這樣的受也沒有常的、堅固的、常恆的、不變易法的,因此,這為了苦的完全滅盡之梵行生活被知道。

  比丘!連只這樣的想也沒有……(中略)比丘!連只這樣的諸行也沒有常的、堅固的、常恆的、不變易法的,將就像那樣永久地住立,比丘!如果連只這樣的諸行也有常的、堅固的、常恆的、不變易法的,這為了苦的完全滅盡之梵行生活不被知道,比丘!但因為連只這樣的諸行也沒有常的、堅固的、常恆的、不變易法的,因此,這為了苦的完全滅盡之梵行生活被知道。

  比丘!連只這樣的識也沒有常的、堅固的、常恆的、不變易法的,將就像那樣永久地住立,比丘!如果連只這樣的識也有常的、堅固的、常恆的、不變易法的,這為了苦的完全滅盡之梵行生活不被知道,比丘!但因為連只這樣的識也沒有常的、堅固的、常恆的、不變易法的,因此,這為了苦的完全滅盡之梵行生活被知道。

  比丘!你怎麼想它:色是常的,或是無常的?」

  「無常的,大德!」

  「受……想……諸行……識是常的,或是無常的?」

  「無常的,大德!」……(中略)

  「因此,在這裡……(中略)這麼看的……(中略)他知道:『……\twnr{不再有此處[輪迴]的狀態}{21.1}。』」



\sutta{98}{98}{概要經}{https://agama.buddhason.org/SN/sn.php?keyword=22.98}
  起源於舍衛城。

  在一旁坐下的那位\twnr{比丘}{31.0}對\twnr{世尊}{12.0}說這個:

  「\twnr{大德}{45.0}!有任何色:常的、堅固的、常恆的、不\twnr{變易法}{70.0}的色,\twnr{將就像那樣永久地住立}{552.0}嗎?

  大德!有任何受……(中略)任何想……任何諸行……大德!有任何識:常的、堅固的、常恆的、不變易法的識,將就像那樣永久地住立嗎?」

  「比丘!沒有任何色:常的、堅固的、常恆的、不變易法的色,將就像那樣永久地住立。比丘!沒有任何受……任何想……任何諸行……任何識:常的、堅固的、常恆的、不變易法的識,將就像那樣永久地住立。」



\sutta{99}{99}{被皮帶繫縛的經}{https://agama.buddhason.org/SN/sn.php?keyword=22.99}
  起源於舍衛城。

  「\twnr{比丘}{31.0}們!這輪迴是無始的,\twnr{無明蓋}{158.0}、渴愛結眾生的流轉的、輪迴的\twnr{起始點}{639.0}不被知道。

  比丘們!有那個時期:大海乾涸、乾枯、不存在,比丘們!然而,我說,[那]不是無明蓋、渴愛結眾生的流轉著、輪迴著\twnr{作苦的終結}{54.0}。

  比丘們!有那個時期:\twnr{須彌山山王}{272.0}被燃燒、滅亡、不存在,比丘們!然而,我說,[那]不是無明蓋、渴愛結眾生的流轉著、輪迴著作苦的終結。

  比丘們!有那個時期:大地被燃燒、滅亡、不存在,比丘們!然而,我說,[那]不是無明蓋、渴愛結眾生的流轉著、輪迴著作苦的終結。

  比丘們!猶如被皮帶繫縛、被綁在堅固的樁或柱上的狗,牠只對樁或柱繞著跑、隨著轉。同樣的,比丘們!\twnr{未聽聞的一般人}{74.0}是聖者的未看見者……(中略)在\twnr{善人法}{76.0}上未被教導者,他\twnr{認為}{964.0}色是我……(中略)認為受是我……認為想是我……認為行是我……認為識是我,或我擁有識,或識在我中,或我在識中,他只對色繞著跑、隨著轉;只對受……(中略)只對想……只對諸行……只對識繞著跑、隨著轉。

  當他對色繞著跑、隨著轉;對受……(中略)對想……對諸行……對識繞著跑、隨著轉時,他不從色被釋放,不從受被釋放,不從想被釋放,不從諸行被釋放,不從識被釋放,不被生、老、死、愁、悲、苦、憂、\twnr{絕望}{342.0}被釋放,我說:『不從苦被釋放。』

  比丘們!\twnr{有聽聞的聖弟子}{24.0}是聖者的看見者……(中略)在善人法上被善教導者,認為色不是我……(中略)受不……想不……諸行不……認為識不是我,或我擁有識,或識在我中,或我在識中,他對色不繞著跑、不隨著轉;對受……對想……對諸行……對識不繞著跑、不隨著轉。

  當他對色不繞著跑、不隨著轉;對受……(中略)對想……對諸行……對識不繞著跑、不隨著轉時,他從色被釋放,從受被釋放,從想被釋放,從諸行被釋放,從識被釋放,被生、老、死、愁、悲、苦、憂、絕望被釋放,我說:『從苦被釋放。』」



\sutta{100}{100}{被皮帶繫縛的經第二}{https://agama.buddhason.org/SN/sn.php?keyword=22.100}
  起源於舍衛城。

  「\twnr{比丘}{31.0}們!這輪迴是無始的,\twnr{無明蓋}{158.0}、渴愛結眾生的流轉的、輪迴的\twnr{起始點}{639.0}不被知道。

  比丘們!猶如被皮帶繫縛、被綁在堅固的樁或柱上的狗,如果牠行走,也都靠近那個樁或柱;如果牠站立,也都靠近那個樁或柱站立;如果牠坐下,也都靠近那個樁或柱坐下;如果牠躺下,也都靠近那個樁或柱躺下。同樣的,比丘們!\twnr{未聽聞的一般人}{74.0}認為色:『\twnr{這是我的}{32.0},\twnr{我是這個}{33.0},\twnr{這是我的真我}{34.1}。』受……想……諸行……認為識:『\twnr{這是我的}{32.0},\twnr{我是這個}{33.0},這是\twnr{我的真我}{34.0}。』如果他行走,都靠近這些五取蘊行走;如果他站立,都靠近這些五取蘊站立;如果他坐下,都靠近這些五取蘊坐下;如果他躺下,都靠近這些五取蘊躺下。

  比丘們!因此,在這裡,自己的心應該經常被省察:『長時間這個心被貪、瞋、癡污染。』比丘們!從心的污染,眾生成為污染;從心的清淨,眾生變成清淨。

  比丘們!\twnr{名為行的畫}{x447}被你們看見嗎?」

  「是的,\twnr{大德}{45.0}!」

  「比丘們!即使那個名為行的畫正被心畫,比丘們!但心還比那個[名為]行的畫更多樣。比丘們!因此,在這裡,自己的心應該經常被省察:『長時間這個心被貪、瞋、癡污染。』比丘們!從心的污染,眾生成為污染;從心的清淨,眾生變成清淨。

  比丘們!我不認為還有其他一類這麼多樣,比丘們!如這諸落入畜生生物,比丘們!即使那些落入畜生生物也以心被多樣,比丘們!但心還比那些落入畜生生物都更多樣。比丘們!因此,在這裡,自己的心應該經常被省察:『長時間這個心被貪、瞋、癡污染。』比丘們!從心的污染,眾生成為污染;從心的清淨,眾生變成清淨。

  比丘們!猶如染工或畫家,以染料、胭脂紅、鬱金黃、靛藍、深紅,\twnr{在磨得很細緻的}{x448}板或壁或白布上,創作一切肢節的男人形色或女人形色。同樣的,比丘們!未聽聞的一般人當使生起時,只使色生起;只使受……(中略)只使想……只使諸行……當使生起時,只使識生起。

  比丘們!你們怎麼想它:色是常的,或是無常的?」

  「無常的,大德!」

  「受……想……諸行……識……(中略)比丘們!因此,在這裡,這麼看的……(中略)他知道:『……\twnr{不再有此處[輪迴]的狀態}{21.1}。』」



\sutta{101}{101}{斧頭柄經}{https://agama.buddhason.org/SN/sn.php?keyword=22.101}
  起源於舍衛城。

  「\twnr{比丘}{31.0}們!我說知者、見者有諸\twnr{漏}{188.0}的滅盡,非對不知者、不見者。比丘們!而知什麼者、見什麼者有諸漏的滅盡?『像這樣是色,像這樣是色的\twnr{集起}{67.0},像這樣是色的滅沒;像這樣是受……像這樣是想……像這樣是諸行……像這樣是識,像這樣是識的集起,像這樣是識的滅沒。』比丘們!這樣知者、這樣見者有諸漏的滅盡。

  比丘們!當比丘住於未實踐\twnr{修習}{94.0}之實踐時,即使這樣的欲求生起:『啊!願不執取後我的心從諸漏被解脫。』但他的心仍未不執取後從諸\twnr{漏}{188.0}被解脫,那是什麼原因?『未自我修習(未修習的狀態)』應該被回答。什麼的未自我修習呢?\twnr{四念住}{286.0}的未自我修習、\twnr{四正勤}{292.0}的未自我修習、\twnr{四神足}{503.1}的未自我修習、五根的未自我修習、五力的未自我修習、\twnr{七覺支}{524.0}的未自我修習、\twnr{八支聖道}{525.0}的未自我修習。

  比丘們!猶如八個,或十個,或十二個母雞的蛋,它們是被母雞不正確坐在上面的,不正確孵熱的,不正確孵化的,即使那隻母雞這樣的欲求生起:『啊!願我的雛雞們以足爪尖或以嘴尖啄破蛋殼後,平安地破殼!』但那些雛雞仍不能以足爪尖或以嘴尖啄破蛋殼後,平安地破殼,那是什麼原因?比丘們!因為,像那樣,八個,或十個,或十二個母雞的蛋,它們是被母雞不正確坐在上面的,不正確孵熱的,不正確孵化的。同樣的,比丘們!當比丘住於未實踐修習之實踐時,即使這樣的欲求生起:『啊!願不執取後我的心從諸漏被解脫。』但他的心仍未不執取後從諸\twnr{漏}{188.0}被解脫,那是什麼原因?『未自我修習』應該被回答。什麼的未自我修習呢?四念住的未自我修習……(中略)八支聖道的未自我修習。

  比丘們!當比丘住於已實踐修習之實踐時,即使這樣的欲求沒生起:『啊!願不執取後我的心從諸漏被解脫。』但他的心仍不執取後從諸\twnr{漏}{188.0}被解脫,那是什麼原因?『\twnr{已自我修習}{658.0}』應該被回答。什麼的已自我修習呢?四念住的已自我修習、四正勤的已自我修習、四神足的已自我修習、五根的已自我修習、五力的已自我修習、七覺支的已自我修習、八支聖道的已自我修習。

  比丘們!猶如八個,或十個,十二個母雞的蛋,它們是被母雞正確坐在上面的,正確孵熱的,正確孵化的,即使那隻母雞這樣的欲求沒生起:『啊!願我的雛雞們以足爪尖或以嘴尖啄破蛋殼後,平安地破殼!』但那些雛雞仍能以足爪尖或以嘴尖啄破蛋殼後,平安地破殼,那是什麼原因?比丘們!因為,像那樣,八個,或十個,或十二個母雞的蛋,它們是被母雞正確坐在上面的,正確孵熱的,正確孵化的。同樣的,比丘們!當比丘住於已實踐修習之實踐時,即使這樣的欲求沒生起:『啊!願不執取後我的心從諸漏被解脫。』但他的心仍不執取後從諸\twnr{漏}{188.0}被解脫,那是什麼原因?『已自我修習』應該被回答。什麼的已自我修習呢?四念住的已自我修習……(中略)八支聖道的已自我修習。

  比丘們!猶如在石匠或石匠徒弟的手斧柄上,諸指痕被看見,拇指痕被看見,但沒有他這樣的智:『啊!我的手斧柄今天有這麼多被滅盡,昨天有這麼多,在其他時有這麼多。』在被滅盡時,就像這樣他有滅盡的智。同樣的,比丘們!當比丘住於已實踐修習之實踐時,即使沒有這樣的智:『啊!我的諸\twnr{漏}{188.0}今天有這麼多被滅盡,昨天有這麼多,在其他時有這麼多。』在被滅盡時,就像這樣他有滅盡的智。

  比丘們!猶如當航海船被繫縛的籐索繫縛時,雨季月在水中消耗後,冬天被拉上陸地,被風、陽光影響,被雨雲下大雨的藤索,它們就少困難地止息,成為腐爛。同樣的,比丘們!當比丘住於已實踐修習之實踐時,諸結就少困難地止息,成為腐爛。」[\ccchref{AN.7.71}{https://agama.buddhason.org/AN/an.php?keyword=7.71}]



\sutta{102}{102}{無常想經}{https://agama.buddhason.org/SN/sn.php?keyword=22.102}
  起源於舍衛城。

  「\twnr{比丘}{31.0}們!無常想已\twnr{修習}{94.0}、已\twnr{多作}{95.0},終結一切欲貪,終結一切色貪,終結一切有貪,終結一切無明,根除一切\twnr{我是之慢}{400.0}。

  比丘們!猶如在秋天時,以大犁耕作的農夫耕作,切斷著一切蔓延的根。同樣的,比丘們!無常想已修習、已多作,終結一切欲貪,終結一切色貪,終結一切有貪,終結一切無明,根除一切我是之慢。

  比丘們!猶如割燈心草者割燈心草後,握住頂端處後抖、甩、搖。同樣的,比丘們!無常想已修習、已多作,終結一切欲貪……(中略)根除一切我是之慢。

  比丘們!猶如在芒果團簇莖處被切斷時,在那裡,凡芒果莖關連的一切,那些都成為隨行那個者。同樣的,比丘們!無常想已修習……(中略)根除一切我是之慢。

  比丘們!猶如凡任何重閣的\twnr{椽}{663.0},那些全部是走到屋頂的、傾向屋頂的、屋頂為會合,屋頂被告知為它們中第一的。同樣的,比丘們!無常想已修習……(中略)根除一切我是之慢。

  比丘們!猶如凡任何香根,黑鳶尾草被告知為它們中第一的。同樣的,比丘們!無常想已修習……(中略)根除一切我是之慢。

  比丘們!猶如凡任何香樹心,紫檀被告知為它們中第一的。同樣的,比丘們!無常想已修習……(中略)根除一切我是之慢。

  比丘們!猶如凡任何香花,茉莉花被告知為它們中第一的。同樣的,比丘們!無常想已修習……(中略)根除一切我是之慢。

  比丘們!猶如凡任何小王,他們全部是\twnr{轉輪王}{278.0}的跟隨者,轉輪王被告知為他們中第一的。同樣的,比丘們!當無常想……(中略)根除一切我是之慢。

  比丘們!猶如凡任何星光的光明,那些全都不及月亮光明的十六分之一,月亮光明被告知為它們中第一的。同樣的,比丘們!當無常想……(中略)根除一切我是之慢。

  比丘們!猶如在秋天時晴朗無雲的天空,當太陽上升到天空時,擊破一切來到天空的黑闇後,輝耀、照亮、照耀。同樣的,比丘們!無常想已修習、已多作,終結一切欲貪,終結一切色貪,終結一切有貪,終結一切無明,根除一切我是之慢。

  比丘們!無常想怎樣已修習、怎樣已多作,終結一切欲貪……(中略)根除一切我是之慢呢?『像這樣是色,像這樣是色的\twnr{集起}{67.0},像這樣是色的滅沒;像這樣是受……像這樣是想……像這樣是諸行……像這樣是識,像這樣是識的集起,像這樣是識的滅沒。』比丘們!當無常想這樣已修習、這麼已多作時,終結一切欲貪,終結一切色貪,終結一切有貪,終結一切無明,根除一切我是之慢。」

  花品第十,其\twnr{攝頌}{35.0}:

  「河、花與泡沬,牛糞與指甲尖,

   概要與二則皮帶,手斧柄、無常性。」

  中間五十則完成。

  那個中間五十則的品的攝頌:

  「攀住與阿羅漢,被食、有名稱的上座,

   以花品為五十,以那個被稱為第二。」





\pin{邊品}{103}{112}
\sutta{103}{103}{邊經}{https://agama.buddhason.org/SN/sn.php?keyword=22.103}
  起源於舍衛城。

  「\twnr{比丘}{31.0}們!有這四個\twnr{邊}{x449},哪四個?\twnr{有身}{93.0}邊、有身的\twnr{集}{67.0}邊、有身的\twnr{滅}{68.0}邊、導向有身的\twnr{滅道跡}{69.0}邊。

  比丘們!而什麼是有身邊?『\twnr{五取蘊}{36.0}』應該被回答。哪五個?即:色取蘊、受取蘊、想取蘊、行取蘊、識取蘊,比丘們!這被稱為有身邊。

  比丘們!而什麼是有身的集邊?凡這個導致再有的、與歡喜及貪俱行的、\twnr{到處歡喜的}{96.2}渴愛,即:欲的渴愛、有的渴愛、\twnr{虛無的渴愛}{244.0},比丘們!這被稱為有身的集邊。

  比丘們!而什麼是有身的滅邊?凡正是那個渴愛的\twnr{無餘褪去與滅}{491.0}、捨棄、\twnr{斷念}{211.0}、解脫、無\twnr{阿賴耶}{391.0},比丘們!這被稱為有身的滅邊。

  比丘們!而什麼是導向有身的滅道跡邊?就是這\twnr{八支聖道}{525.0},即:正見、正志、正語、正業、正命、正精進、正念、正定,比丘們!這被稱為導向有身的滅道跡邊。

  比丘們!這是四個邊。」[≃\suttaref{SN.22.22}]



\sutta{104}{104}{苦經}{https://agama.buddhason.org/SN/sn.php?keyword=22.104}
  起源於舍衛城。

  「\twnr{比丘}{31.0}們!我將為你們教導苦、苦\twnr{集}{67.0}、苦\twnr{滅}{68.0}、導向苦\twnr{滅道跡}{69.0},\twnr{你們要聽}{43.0}它!

  比丘們!而什麼是苦?『\twnr{五取蘊}{36.0}』應該被回答。哪五個?即:色取蘊……(中略)識取蘊,比丘們!這被稱為苦。

  比丘們!而什麼是苦集?凡這渴愛:導致再生的……(中略)欲的渴愛、有的渴愛、\twnr{虛無的渴愛}{244.0},比丘們!這被稱為苦集。

  比丘們!而什麼是苦滅?凡正是那個渴愛的\twnr{無餘褪去與滅}{491.0}、捨棄、\twnr{斷念}{211.0}、解脫、無\twnr{阿賴耶}{391.0},比丘們!這被稱為苦滅。

  比丘們!而什麼是導向苦滅道跡?就是這\twnr{八支聖道}{525.0},即:正見……(中略)正定,比丘們!這被稱為導向苦滅道跡。」



\sutta{105}{105}{有身經}{https://agama.buddhason.org/SN/sn.php?keyword=22.105}
  起源於舍衛城。

  「\twnr{比丘}{31.0}們!我將為你們教導\twnr{有身}{93.0}、有身的\twnr{集}{67.0}、有身的\twnr{滅}{68.0}、導向有身的\twnr{滅道跡}{69.0},\twnr{你們要聽}{43.0}它!

  比丘們!而什麼是有身?『\twnr{五取蘊}{36.0}』應該被回答。哪五個?即:色取蘊、受取蘊、想取蘊、行取蘊、識取蘊,比丘們!這被稱為有身。

  比丘們!而什麼是有身的集?凡這渴愛:\twnr{導致再生}{96.0}……(中略)欲的渴愛、有的渴愛、\twnr{虛無的渴愛}{244.0},比丘們!這被稱為有身的集。

  比丘們!而什麼是有身的滅?凡正是那個渴愛的……(中略)比丘們!這被稱為有身的滅。

  比丘們!而什麼是導向有身的滅道跡?就是這\twnr{八支聖道}{525.0},即:正見……(中略)正定,比丘們!這被稱為導向有身的滅道跡。」



\sutta{106}{106}{應該被遍知的經}{https://agama.buddhason.org/SN/sn.php?keyword=22.106}
  起源於舍衛城。

  「\twnr{比丘}{31.0}們!我將教導應該被遍知的法,以及\twnr{遍知}{154.0},與有遍知的人,\twnr{你們要聽}{43.0}它!

  比丘們!而什麼是應該被遍知的法?比丘們!色是應該被遍知的法,受……(中略)想……諸行……識是應該被遍知的法,比丘們!這些被稱為應該被遍知的法。

  比丘們!而什麼是遍知?貪的滅盡、瞋的滅盡、癡的滅盡,比丘們!這被稱為遍知。

  比丘們!而什麼是有遍知的人?『\twnr{阿羅漢}{5.0}』應該被回答。凡這樣名、這樣姓的這位\twnr{尊者}{200.0},比丘們!這被稱為有遍知的人。」[\suttaref{SN.23.4}]



\sutta{107}{107}{沙門經}{https://agama.buddhason.org/SN/sn.php?keyword=22.107}
  起源於舍衛城。

  「\twnr{比丘}{31.0}們!有這些\twnr{五取蘊}{36.0},哪五個?即:色取蘊……(中略)識取蘊,比丘們!凡任何沙門或婆羅門不如實知道這些五取蘊的\twnr{樂味}{295.0}、\twnr{過患}{293.0}、\twnr{出離}{294.0}者……(中略[按:如\suttaref{SN.12.13}])以證智自作證後,在當生中\twnr{進入後住於}{66.0}\twnr{沙門義}{327.0}或婆羅門義。」 



\sutta{108}{108}{沙門經第二}{https://agama.buddhason.org/SN/sn.php?keyword=22.108}
  起源於舍衛城。

  「\twnr{比丘}{31.0}們!有這些\twnr{五取蘊}{36.0},哪五個?即:色取蘊、受取蘊、想取蘊、行取蘊、識取蘊,比丘們!凡任何沙門或婆羅門不如實知道這些五取蘊的\twnr{集起}{67.0}、滅沒、\twnr{樂味}{295.0}、\twnr{過患}{293.0}、\twnr{出離}{294.0}者……(中略[按:如\suttaref{SN.12.13}])以證智自作證後,在當生中\twnr{進入後住於}{66.0}\twnr{沙門義}{327.0}或婆羅門義。」 



\sutta{109}{109}{入流者經}{https://agama.buddhason.org/SN/sn.php?keyword=22.109}
  起源於舍衛城。

  「\twnr{比丘}{31.0}們!有這些\twnr{五取蘊}{36.0},哪五個?即:色取蘊……(中略)識取蘊,比丘們!當聖弟子如實知道這些五取蘊的\twnr{集起}{67.0}、滅沒、\twnr{樂味}{295.0}、\twnr{過患}{293.0}、\twnr{出離}{294.0},比丘們!這位聖弟子被稱為\twnr{入流者}{165.0}、不墮惡趣法者、\twnr{決定者}{159.0}、\twnr{正覺為彼岸者}{160.0}。」 



\sutta{110}{110}{阿羅漢經}{https://agama.buddhason.org/SN/sn.php?keyword=22.110}
  起源於舍衛城。

  「\twnr{比丘}{31.0}們!有這些\twnr{五取蘊}{36.0},哪五個?即:色取蘊……(中略)識取蘊,比丘們!當比丘如實知道這些五取蘊的\twnr{集起}{67.0}、滅沒、\twnr{樂味}{295.0}、\twnr{過患}{293.0}、\twnr{出離}{294.0}後,不執取後成為解脫者,比丘們!這被稱為漏已滅盡的、已完成的、\twnr{應該被作的已作的}{20.0}、負擔已卸的、\twnr{自己的利益已達成的}{189.0}、\twnr{有之結已滅盡的}{190.0}、以\twnr{究竟智}{191.0}解脫的\twnr{阿羅漢}{5.0}比丘。」 



\sutta{111}{111}{意欲的捨斷經}{https://agama.buddhason.org/SN/sn.php?keyword=22.111}
  起源於舍衛城。 

  「\twnr{比丘}{31.0}們!在色上凡意欲,凡貪,凡歡喜,凡渴愛,你們要捨斷它,這樣,那個色必將被捨斷,根被切斷,\twnr{[如]已斷根的棕櫚樹}{147.1},\twnr{成為非有}{408.0},\twnr{為未來不生之物}{229.0}。

  在受上……(中略)在想上……在諸行上……在識上凡意欲,凡貪,凡歡喜,凡渴愛,你們要捨斷它,這樣,那個識必將被捨斷,根被切斷,[如]已斷根的棕櫚樹,成為非有,為未來不生之物。」[\suttaref{SN.23.9}]



\sutta{112}{112}{意欲的捨斷經第二}{https://agama.buddhason.org/SN/sn.php?keyword=22.112}
  起源於舍衛城。 

  「\twnr{比丘}{31.0}們!在色上凡意欲,凡貪,凡歡喜,凡渴愛,凡攀住、執取、心的依處、執持、\twnr{煩惱潛在趨勢}{253.1},你們要捨斷那些,這樣,那個色必將被捨斷,根被切斷……在受上……(中略)在想上……在諸行上凡意欲……這樣,那些行必將被捨斷,根被切斷,\twnr{[如]已斷根的棕櫚樹}{147.1},\twnr{成為非有}{408.0},\twnr{為未來不生之物}{229.0}。在識上凡意欲,凡貪,凡歡喜,凡渴愛,凡攀住、執取、心的依處、執持、煩惱潛在趨勢,你們要捨斷那些,這樣,那個識必將被捨斷,根被切斷,[如]已斷根的棕櫚樹,成為非有,為未來不生之物。」[\suttaref{SN.23.10}]

  邊品第十一,其\twnr{攝頌}{35.0}:

  「邊、苦與有身,應該被遍知的、沙門二則,

   入流者與阿羅漢,二則意欲的捨斷。」





\pin{說法者品}{113}{125}
\sutta{113}{113}{無明經}{https://agama.buddhason.org/SN/sn.php?keyword=22.113}
  起源於舍衛城。

  那時,\twnr{某位比丘}{39.0}去見\twnr{世尊}{12.0}。……(中略)在一旁坐下的那位比丘對世尊說這個:

  「\twnr{大德}{45.0}!被稱為『\twnr{無明}{207.0}、無明』,大德!什麼是無明?而什麼情形是\twnr{進入無明者}{645.0}?」

  「比丘!這裡,\twnr{未聽聞的一般人}{74.0}不知道色,不知道色\twnr{集}{67.0},不知道色\twnr{滅}{68.0},不知道導向色\twnr{滅道跡}{69.0}。不知道受……(中略)想……不知道諸行……(中略)不知道導向識滅道跡,比丘!這被稱為無明,而這個情形是進入無明者。」[\suttaref{SN.22.135}]



\sutta{114}{114}{明經}{https://agama.buddhason.org/SN/sn.php?keyword=22.114}
  起源於舍衛城。

  在一旁坐下的那位\twnr{比丘}{31.0}對\twnr{世尊}{12.0}說這個:

  「\twnr{大德}{45.0}!被稱為『明、明』,大德!什麼是明?而什麼情形是進入了\twnr{明}{207.0}?」

  「比丘!這裡,\twnr{有聽聞的聖弟子}{24.0}知道色,知道色集,知道色滅,知道導向色\twnr{滅道跡}{69.0}。受……想……知道諸行……(中略)知道導向識滅道跡,比丘!這被稱為明,而這個情形是進入明者。」[\suttaref{SN.22.135}]



\sutta{115}{115}{說法者經}{https://agama.buddhason.org/SN/sn.php?keyword=22.115}
  起源於舍衛城。

  在一旁坐下的那位\twnr{比丘}{31.0}對\twnr{世尊}{12.0}說這個:

  「\twnr{大德}{45.0}!被稱為『\twnr{說法者}{x450},說法者』,大德!什麼情形是說法者呢?」

  「比丘!如果對色是為了\twnr{厭}{15.0}、\twnr{離貪}{77.0}、\twnr{滅}{68.0}而教導法,『說法者比丘』是適當的言語。

  比丘!如果對色是為了厭、離貪、\twnr{滅的行者}{519.0},『\twnr{法、隨法行者}{58.0}比丘』是適當的言語。

  比丘!如果對色從厭、離貪、滅,不執取後成為解脫者,『得\twnr{當生}{42.0}涅槃比丘』是適當的言語。

  比丘!如果對受……(中略)比丘!如果對想……比丘!如果對諸行……比丘!如果對識是為了厭、離貪、滅而教導法,『說法者比丘』是適當的言語。比丘!如果對識是為了厭、離貪、滅的行者,『法、隨法行者比丘』是適當的言語。比丘!如果對識從厭、離貪、滅,不執取後成為解脫者,『得當生涅槃者比丘』是適當的言語。」[≃\suttaref{SN.12.16}]



\sutta{116}{116}{說法者經第二}{https://agama.buddhason.org/SN/sn.php?keyword=22.116}
  起源於舍衛城。

  在一旁坐下的那位\twnr{比丘}{31.0}對\twnr{世尊}{12.0}說這個:

  「\twnr{大德}{45.0}!被稱為『\twnr{說法者}{x451},說法者』,大德!什麼情形是說法者呢?大德!什麼情形是\twnr{法、隨法行}{58.0}呢?大德!什麼情形是得\twnr{當生}{42.0}涅槃呢?」

  「比丘!如果對色是為了\twnr{厭}{15.0}、\twnr{離貪}{77.0}、\twnr{滅}{68.0}而教導法,『說法者比丘』是適當的言語。

  比丘!如果對色是為了\twnr{厭}{15.0}、\twnr{離貪}{77.0}、滅的行者,『法、隨法行者比丘』是適當的言語。

  比丘!如果對色從厭、離貪、滅,不執取後成為解脫者,『得當生涅槃者比丘』是適當的言語。

  比丘!如果對受……(中略)比丘!如果對想……比丘!如果對行……比丘!如果對識是為了厭、離貪、滅而教導法,『說法者比丘』是適當的言語。比丘!如果對識是為了厭、離貪、滅的行者,『法、隨法行者比丘』是適當的言語。比丘!如果是對識從厭、離貪、滅,不執取後成為解脫者,『得當生涅槃者比丘』是適當的言語。」[≃\suttaref{SN.12.16}] 



\sutta{117}{117}{捕縛經}{https://agama.buddhason.org/SN/sn.php?keyword=22.117}
  起源於舍衛城。

  「\twnr{比丘}{31.0}們!這裡,\twnr{未聽聞的一般人}{74.0}是聖者的未看見者……(中略)在\twnr{善人法}{76.0}上未被教導者,他\twnr{認為}{964.0}色是我,\twnr{或我擁有色}{13.0},或色在我中,\twnr{或我在色中}{14.0},比丘們!這被稱為被色之繫縛繫縛的、\twnr{被內部外部之繫縛繫縛的}{x452}、不見此岸不見\twnr{彼岸}{226.0}的未聽聞的一般人。\twnr{被繫縛者衰老}{x453},\twnr{被繫縛者死亡}{x454},被繫縛者從此世到他世。

  認為受是我……(中略)或我在受中,比丘們!這被稱為被受之繫縛繫縛的、被內部外部之繫縛繫縛的、不見此岸不見彼岸的未聽聞的一般人,被繫縛者衰老,被繫縛者死亡,被繫縛者從此世到他世。想……諸行……認為識是我……(中略)比丘們!這被稱為被識之繫縛繫縛的、被內部外部之繫縛繫縛的、不見此岸不見彼岸的未聽聞的一般人。被繫縛者衰老,被繫縛者死亡,被繫縛者從此世到他世。

  比丘們!而\twnr{有聽聞的聖弟子}{24.0}是聖者的看見者……(中略)在善人法上被善教導者,他認為色不是我,或我不擁有色,或色不在我中,或我不在色中,比丘們!這被稱為不被色之繫縛繫縛的、不被內部外部之繫縛繫縛的、見此岸見彼岸的有聽聞的聖弟子,我說:『他從苦被解脫。』

  認為受不是我……(中略)認為想不是我……(中略)認為諸行不是我……(中略)認為識不是我……(中略)比丘們!這被稱為不被識之繫縛繫縛的、不被內部外部之繫縛繫縛的、見此岸見彼岸的有聽聞的聖弟子,我說:『他從苦被解脫。』」



\sutta{118}{118}{遍問經}{https://agama.buddhason.org/SN/sn.php?keyword=22.118}
  起源於舍衛城。

  「\twnr{比丘}{31.0}們!你們怎麼想它,你們認為色:『\twnr{這是我的}{32.0},\twnr{我是這個}{33.0},這是\twnr{我的真我}{34.0}。』嗎?」

  「\twnr{大德}{45.0}!這確實不是。」

  「比丘們!\twnr{好}{44.0}!比丘們!色:『\twnr{這不是我的}{32.1},\twnr{我不是這個}{33.1},\twnr{這不是我的真我}{34.2}。』這樣,這個應該以正確之慧如實被看見。

  受……想……諸行……你們認為識:『這是我的,我是這個,這是我的真我。』嗎?」

  「大德!這確實不是。」

  「比丘們!好!比丘們!識:『這不是我的,我不是這個,這不是我的真我。』這樣,這個應該以正確之慧如實被看見。

  ……(中略)這麼看的……他知道:『……\twnr{應該被作的已作}{20.0},\twnr{不再有此處[輪迴]的狀態}{21.1}。』」



\sutta{119}{119}{遍問經第二}{https://agama.buddhason.org/SN/sn.php?keyword=22.119}
  起源於舍衛城。

  「\twnr{比丘}{31.0}們!你們怎麼想它,你們認為色:『\twnr{這不是我的}{32.1},\twnr{我不是這個}{33.1},\twnr{這不是我的真我}{34.2}。』嗎?」

  「是的,\twnr{大德}{45.0}!」

  「比丘們!\twnr{好}{44.0}!比丘們!色:『這不是我的,我不是這個,這不是我的真我。』這樣,這個應該以正確之慧如實被看見。

  受……想……諸行……你們認為識:『這不是我的,我不是這個,這不是我的真我。』呢?」

  「是的,大德!」

  「比丘們!好!比丘們!識:『這不是我的,我不是這個,這不是我的真我。』這樣,這個應該以正確之慧如實被看見。

  ……(中略)這麼……他知道:『……\twnr{不再有此處[輪迴]的狀態}{21.1}。』」



\sutta{120}{120}{會被結縛的經}{https://agama.buddhason.org/SN/sn.php?keyword=22.120}
  起源於舍衛城。  

  「\twnr{比丘}{31.0}們!我將教導\twnr{會被結縛的諸法}{666.0}與結,\twnr{你們要聽}{43.0}它!

     比丘們!而什麼是會被結縛的諸法,哪個是結呢?比丘們!色是會被結縛的諸法,凡在那裡有意欲貪者,在那裡有結;受……(中略)想……諸行……識是會被結縛的諸法,凡在那裡有意欲貪者,在那裡有結,比丘們!這些被稱為會被結縛的法,這個是結。」[≃\suttaref{SN.35.109}, \suttaref{SN.35.122}]



\sutta{121}{121}{與執取有關的經}{https://agama.buddhason.org/SN/sn.php?keyword=22.121}
  起源於舍衛城。  

  「\twnr{比丘}{31.0}們!我將教導\twnr{與執取有關的}{551.0}諸法與執取,\twnr{你們要聽}{43.0}它!

     比丘們!什麼是與執取有關的諸法?以及什麼是執取?比丘們!色是與執取有關的法,凡在那裡有意欲貪者,在那裡有執取;受……(中略)想……諸行……識是與執取有關的法,凡在那裡有意欲貪者,在那裡有執取,比丘們!這些被稱為與執取有關的法,這個是執取。」[≃\suttaref{SN.35.110}, \suttaref{SN.35.123}] 



\sutta{122}{122}{持戒者經}{https://agama.buddhason.org/SN/sn.php?keyword=22.122}
  \twnr{有一次}{2.0},\twnr{尊者}{200.0}舍利弗與尊者摩訶拘絺羅住在波羅奈仙人墜落處的鹿林。

  那時,尊者摩訶拘絺羅傍晚時,從\twnr{獨坐}{92.0}出來,去見尊者舍利弗……(中略)說這個:

  「舍利弗\twnr{學友}{201.0}!哪些法應該被持戒的\twnr{比丘}{31.0}\twnr{如理作意}{114.0}?」

  「拘絺羅學友!五取蘊為無常的、苦的、病的、腫瘤的、箭的、禍的、疾病的、\twnr{另一邊的}{431.0}、敗壞的、空的、無我的應該被持戒的比丘如理作意。哪五個?即:色取蘊、受取蘊、想取蘊、行取蘊、識取蘊。拘絺羅學友!這些五取蘊為無常的、苦的、病的、腫瘤的、箭的、禍的、疾病的、另一邊的、敗壞的、空的、無我的應該被持戒的比丘如理作意。學友!又,這存在可能性:凡如理作意這些五取蘊為無常的……(中略)無我的之持戒的比丘作證入流果。」

  「舍利弗學友!那麼,哪些法應該被\twnr{入流者}{165.0}比丘如理作意?」

  「拘絺羅學友!這些五取蘊為無常的……(中略)無我的也應該被入流者比丘如理作意。學友!又,這存在可能性:凡如理作意這些五取蘊為無常的……(中略)無我的之入流者比丘作證\twnr{一來}{208.0}果。」

  「舍利弗學友!那麼,哪些法應該被\twnr{一來}{208.0}者比丘如理作意?」

  「拘絺羅學友!這些五取蘊為無常的……(中略)無我的也應該被一來者比丘如理作意。學友!又,這存在可能性:凡如理作意這些五取蘊為無常的……(中略)無我的之一來者比丘作證不還果。」

  「舍利弗學友!那麼,哪些法應該被\twnr{不還者}{209.0}比丘如理作意?」

  「拘絺羅學友!這些五取蘊為無常的……(中略)無我的也應該被不還者比丘如理作意。學友!又,這存在可能性:凡如理作意這些五取蘊為無常的……(中略)無我的之不還者比丘作證\twnr{阿羅漢}{5.0}。」

  「舍利弗學友!那麼,哪些法應該被阿羅漢比丘如理作意?」

  「拘絺羅學友!這些五取蘊為無常的、苦的、病的、腫瘤的、箭的、禍的、疾病的、另一邊的、敗壞的、空的、無我的也應該被阿羅漢比丘如理作意。學友!對阿羅漢來說,沒有更進一步應該被作的,或對已作的之增加,但這些法已\twnr{修習}{94.0}、已\twnr{多作}{95.0},轉起當生樂的住處,連同念、正知。」



\sutta{123}{123}{多聞者經}{https://agama.buddhason.org/SN/sn.php?keyword=22.123}
  \twnr{有一次}{2.0},\twnr{尊者}{200.0}舍利弗與尊者摩訶拘絺羅住在波羅奈仙人墜落處的鹿林。

  那時,尊者摩訶拘絺羅傍晚時,從\twnr{獨坐}{92.0}出來,去見尊者舍利弗。抵達後……(中略)說這個:

  「舍利弗\twnr{學友}{201.0}!哪些法應該被多聞的\twnr{比丘}{31.0}\twnr{如理作意}{114.0}?」

  「拘絺羅學友!五取蘊為無常的……(中略)無我的應該被多聞的比丘如理作意。哪五個?即:色取蘊……(中略)識取蘊。拘絺羅學友!這些五取蘊為無常的……(中略)無我的應該被多聞的比丘如理作意。學友!又,這存在可能性:凡如理作意這些五取蘊為無常的……(中略)無我的之多聞的比丘作證入流果。」

  「舍利弗學友!那麼,哪些法應該被\twnr{入流者}{165.0}比丘如理作意?」

  「拘絺羅學友!這些五取蘊為無常的……(中略)無我的也應該被入流者比丘如理作意。學友!又,這存在可能性:凡如理作意這些五取蘊為無常的……(中略)無我的之入流者比丘作證\twnr{一來果}{208.2}……(中略)\twnr{不還果}{209.1}……(中略)作證\twnr{阿羅漢}{5.0}果。」

  「舍利弗學友!那麼,哪些法應該被阿羅漢比丘如理作意?」

  「拘絺羅學友!這些五取蘊為無常的、苦的、病的、腫瘤的、箭的、禍的、疾病的、另一邊的、敗壞的、空的、無我的應該被阿羅漢比丘如理作意。學友!對阿羅漢來說,沒有更進一步應該被作的,或對已作的之增加,但這些法已\twnr{修習}{94.0}、已\twnr{多作}{95.0},轉起當生樂的住處,連同念、正知。」



\sutta{124}{124}{葛波經}{https://agama.buddhason.org/SN/sn.php?keyword=22.124}
  起源於舍衛城。

  那時,\twnr{尊者}{200.0}\twnr{葛波}{x455}去見\twnr{世尊}{12.0}……(中略)在一旁坐下的尊者葛波對世尊說這個:

  「\twnr{大德}{45.0}!當怎樣知、當怎樣見時,在這個有識之身上\twnr{與在一切外部諸相上}{559.0}沒有\twnr{我作}{22.0}、\twnr{我所作}{25.0}、\twnr{慢煩惱潛在趨勢}{26.0}?」

  「葛波!凡任何色:過去、未來、現在,或內、或外,或粗、或細,或下劣、或勝妙,或凡在遠處、在近處,所有色:『\twnr{這不是我的}{32.1},\twnr{我不是這個}{33.1},\twnr{這不是我的真我}{34.2}。』以正確之慧這樣如實看見這個。

  凡任何受……(中略)凡任何想……凡任何諸行……凡任何識:過去、未來、現在,或內、或外,或粗、或細,或下劣、或勝妙,或凡在遠處、在近處,所有識:『這不是我的,我不是這個,這不是我的真我。』以正確之慧這樣如實看見這個。

  葛波!當這樣知、這樣見時,在這個有識之身上與在一切外部諸相上沒有我作、我所作、慢煩惱潛在趨勢。」[\suttaref{SN.18.21}]



\sutta{125}{125}{葛波經第二}{https://agama.buddhason.org/SN/sn.php?keyword=22.125}
  起源於舍衛城。

  在一旁坐下的\twnr{尊者}{200.0}葛波對\twnr{世尊}{12.0}說這個:

  「\twnr{大德}{45.0}!當怎樣知、當怎樣見時,在這個有識之身上\twnr{與在一切外部諸相上}{559.0}離\twnr{我作}{22.0}、\twnr{我所作}{25.0}、慢,成為心意超越\twnr{慢類}{647.0}者、寂靜者、\twnr{善解脫}{28.0}者呢?」

  「葛波!凡任何色:過去、未來、現在……(中略)所有色:『\twnr{這不是我的}{32.1},\twnr{我不是這個}{33.1},\twnr{這不是我的真我}{34.2}。』以正確之慧這樣如實看見這個後,不執取後成為解脫者。

  凡任何受……(中略)凡任何想……凡任何諸行……凡任何識:過去、未來、現在,或內、或外,或粗、或細,或下劣、或勝妙,或凡在遠處、在近處,所有識:『這不是我的,我不是這個,這不是我的真我。』以正確之慧這樣如實看見這個後,不執取後成為解脫者。

  葛波!當這樣知、這樣見時,在這個有識之身上與在一切外部諸相上離我作、我所作、慢,成為心意超越慢類者、寂靜者、善解脫者。」[\suttaref{SN.18.22}]

  說法者品第十二,其\twnr{攝頌}{35.0}:

  「無明、明、二則說法者,繫縛、遍問二則,

   結縛、執取,戒、多聞者、二則葛波。」





\pin{無明品}{126}{135}
\sutta{126}{126}{集法經}{https://agama.buddhason.org/SN/sn.php?keyword=22.126}
  起源於舍衛城。

  那時,\twnr{某位比丘}{39.0}去見\twnr{世尊}{12.0}。抵達後……(中略)在一旁坐下的那位比丘對世尊說這個:

  「\twnr{大德}{45.0}!被稱為『\twnr{無明}{207.0}、無明』,大德!什麼是無明?而什麼情形是\twnr{進入無明者}{645.0}?」

  「比丘!這裡,\twnr{未聽聞的一般人}{74.0}不如實知道\twnr{集法}{67.1}之色為『集法之色』;不如實知道\twnr{消散法}{155.0}之色為『消散法之色』;不如實知道集與消散法之色為『集與消散法之色』。不如實知道集法之受為『集法之受』;不如實知道消散法之受為『消散法之受』;不如實知道集與消散法之受為『集與消散法之受』。集法之想……(中略)不如實知道集法之諸行為『集法之諸行』;不如實知道消散法之諸行為『消散法之諸行』;不如實知道集與消散法之諸行為『集與消散法之諸行』。不如實知道集法之識為『集法之識』;不如實知道消散法之識為『消散法之識』;不如實知道集與消散法之識為『集與消散法之識』,比丘!這被稱為無明,而這個情形是進入無明者。」

  在這麼說時,那位比丘對世尊說這個: 

  「大德!被稱為『明、明』,大德!什麼是明?而什麼情形是進入明者?」 

  「比丘!這裡,\twnr{有聽聞的聖弟子}{24.0}如實知道集法之色為『集法之色』;如實知道消散法之色為『消散法之色』;如實知道集與消散法之色為『集與消散法之色』。如實知道集法之受為『集法之受』;如實知道消散法之受為『消散法之受』;如實知道集與消散法之受為『集與消散法之受』。集法之想……(中略)如實知道集法之諸行為『集法之諸行』;如實知道消散法之諸行為『消散法之諸行』;如實知道集與消散法之諸行為『集與消散法之諸行』。如實知道集法之識為『集法之識』;如實知道消散法之識為『消散法之識』;如實知道集與消散法之識為『集與消散法之識』,比丘!這被稱為明,而這個情形是進入明者。」 



\sutta{127}{127}{集法經第二}{https://agama.buddhason.org/SN/sn.php?keyword=22.127}
  \twnr{有一次}{2.0},\twnr{尊者}{200.0}舍利弗與尊者摩訶拘絺羅住在波羅奈仙人墜落處的鹿林。

  那時,尊者摩訶拘絺羅傍晚時,從\twnr{獨坐}{92.0}出來……(中略)。在一旁坐下的尊者摩訶拘絺羅對尊者舍利弗說這個:

  「舍利弗\twnr{學友}{201.0}!被稱為『\twnr{無明}{207.0}、無明』,學友!什麼是無明?而什麼情形是\twnr{進入無明者}{645.0}?」

  「學友!這裡,\twnr{未聽聞的一般人}{74.0}不如實知道\twnr{集法}{67.1}之色為『集法之色』;\twnr{消散法}{155.0}之色……(中略)不如實知道集與消散法之色為『集與消散法之色』。集法之受……(中略)消散法之受……(中略)不如實知道集與消散法之受為『集與消散法之受』;集法之想……(中略)集法之諸行……(中略)消散法之諸行……(中略)不如實知道集與消散法之諸行為『集與消散法之諸行』;集法之識……(中略)不如實知道集與消散法之識為『集與消散法之識』,學友!這被稱為無明,而這個情形是進入無明者。」



\sutta{128}{128}{集法經第三}{https://agama.buddhason.org/SN/sn.php?keyword=22.128}
  \twnr{有一次}{2.0},\twnr{尊者}{200.0}舍利弗與尊者摩訶拘絺羅住在波羅奈仙人墜落處的鹿林。

  ……(中略)在一旁坐下的尊者摩訶拘絺羅對尊者舍利弗說這個:

  「舍利弗\twnr{學友}{201.0}!被稱為『\twnr{明}{207.0}、明』,學友!什麼是明?而什麼情形是進入明者?」

  「學友!這裡,\twnr{有聽聞的聖弟子}{24.0}如實知道\twnr{集法}{67.1}之色為『集法之色』;\twnr{消散法}{155.0}之色……(中略)如實知道集與消散法之色為『集與消散法之色』;集法之受……(中略)集與消散法之受,[……(中略)]集法之想……(中略)集法之諸行……消散法之諸行……如實知道集與消散法之諸行為『集與消散法之諸行』;集法之識……消散法之識……如實知道集與消散法之識為『集與消散法之識』,學友!這被稱為明,而這個情形是進入明者。」



\sutta{129}{129}{樂味經}{https://agama.buddhason.org/SN/sn.php?keyword=22.129}
  住在波羅奈仙人墜落處的鹿林。……(中略)  

  在一旁坐下的\twnr{尊者}{200.0}摩訶拘絺羅對尊者舍利弗說這個:

  「舍利弗\twnr{學友}{201.0}!被稱為『\twnr{無明}{207.0}、無明』,學友!什麼是無明?而什麼情形是\twnr{進入無明者}{645.0}?」

  「學友!這裡,\twnr{未聽聞的一般人}{74.0}不如實知道色的\twnr{樂味}{295.0}、\twnr{過患}{293.0}、\twnr{出離}{294.0};受的……(中略)想的……諸行的……不如實知道識的樂味、過患、出離,學友!這被稱為無明,而這個情形是進入無明者。」



\sutta{130}{130}{樂味經第二}{https://agama.buddhason.org/SN/sn.php?keyword=22.130}
  住在波羅奈仙人墜落處的鹿林。……(中略)  

  「舍利弗\twnr{學友}{201.0}!被稱為『\twnr{明}{207.0}、明』,學友!什麼是明?而什麼情形是進入明者?」

  「學友!這裡,\twnr{有聽聞的聖弟子}{24.0}如實知道色的\twnr{樂味}{295.0}、\twnr{過患}{293.0}、\twnr{出離}{294.0};受的……(中略)想的……諸行的……如實知道識的樂味、過患、出離,學友!這被稱為明,而這個情形是進入明者。」



\sutta{131}{131}{集起經}{https://agama.buddhason.org/SN/sn.php?keyword=22.131}
  住在波羅奈仙人墜落處的鹿林。……(中略)  

  「舍利弗\twnr{學友}{201.0}!被稱為『\twnr{無明}{207.0}、無明』,學友!什麼是無明?而什麼情形是\twnr{進入無明者}{645.0}?」

  「學友!這裡,\twnr{未聽聞的一般人}{74.0}不如實知道色的\twnr{集起}{67.0}、滅沒、\twnr{樂味}{295.0}、\twnr{過患}{293.0}、\twnr{出離}{294.0};受的……(中略)想的……諸行的……不如實知道識的集起、滅沒、樂味、過患、出離,學友!這被稱為無明,而這個情形是進入無明者。」



\sutta{132}{132}{集起經第二}{https://agama.buddhason.org/SN/sn.php?keyword=22.132}
  住在波羅奈仙人墜落處的鹿林。……(中略)  

  在一旁坐下的\twnr{尊者}{200.0}摩訶拘絺羅對尊者舍利弗說這個:

  「舍利弗\twnr{學友}{201.0}!被稱為『\twnr{明}{207.0}、明』,學友!什麼是明?而什麼情形是進入明者?」

  「學友!這裡,\twnr{有聽聞的聖弟子}{24.0}如實知道色的\twnr{集起}{67.0}、滅沒、\twnr{樂味}{295.0}、\twnr{過患}{293.0}、\twnr{出離}{294.0};受的……(中略)想的……諸行的……如實知道識的集起、滅沒、樂味、過患、出離,學友!這被稱為明,而這個情形是進入明者。」



\sutta{133}{133}{拘絺羅經}{https://agama.buddhason.org/SN/sn.php?keyword=22.133}
  住在波羅奈仙人墜落處的鹿林。那時,\twnr{尊者}{200.0}舍利弗傍晚時……(中略)在一旁坐下的尊者舍利弗對尊者摩訶拘絺羅說這個:

  「拘絺羅\twnr{學友}{201.0}!被稱為『\twnr{無明}{207.0}、無明』,學友!什麼是無明?而什麼情形是\twnr{進入無明者}{645.0}?」

  「學友!這裡,\twnr{未聽聞的一般人}{74.0}不如實知道色的\twnr{樂味}{295.0}、\twnr{過患}{293.0}、\twnr{出離}{294.0}。受的……(中略)想的……諸行的……不如實知道識的樂味、過患、出離,學友!這被稱為無明,而這個情形是進入無明者。」

  在這麼說時,尊者舍利弗對尊者摩訶拘絺羅說這個:

  「拘絺羅學友!被稱為『明、明』,學友!什麼是明?而什麼情形是進入明者?」

  「學友!這裡,\twnr{有聽聞的聖弟子}{24.0}如實知道色的樂味、過患、出離。受的……(中略)想的……諸行的……如實知道識的樂味、過患、出離,學友!這被稱為明,而這個情形是進入明者。」



\sutta{134}{134}{拘絺羅經第二}{https://agama.buddhason.org/SN/sn.php?keyword=22.134}
  住在波羅奈仙人墜落處的鹿林……(中略)。

  「拘絺羅\twnr{學友}{201.0}!被稱為『\twnr{無明}{207.0}、無明』,學友!什麼是無明?而什麼情形是\twnr{進入無明者}{645.0}?」

  「學友!這裡,\twnr{未聽聞的一般人}{74.0}不如實知道色的\twnr{集起}{67.0}、滅沒、\twnr{樂味}{295.0}、\twnr{過患}{293.0}、\twnr{出離}{294.0}。受的……(中略)想的……諸行的……不如實知道識的集起、滅沒、樂味、過患、出離,學友!這被稱為無明,而這個情形是進入無明者。」

  在這麼說時,\twnr{尊者}{200.0}舍利弗對尊者摩訶拘絺羅說這個:

  「拘絺羅學友!被稱為『明、明』,學友!什麼是明?而什麼情形是進入明者?」

  「學友!這裡,\twnr{有聽聞的聖弟子}{24.0}如實知道色的集起、滅沒、樂味、過患、出離。受的……(中略)想的……諸行的……如實知道識的集起、滅沒、樂味、過患、出離,學友!這被稱為明,而這個情形是進入明者。」



\sutta{135}{135}{拘絺羅經第三}{https://agama.buddhason.org/SN/sn.php?keyword=22.135}
  相同的起源。

  在一旁坐下的\twnr{尊者}{200.0}舍利弗對尊者摩訶拘絺羅說這個:

  「拘絺羅\twnr{學友}{201.0}!被稱為『\twnr{無明}{207.0}、無明』,學友!什麼是無明?而什麼情形是\twnr{進入無明者}{645.0}?」

  「學友!這裡,\twnr{未聽聞的一般人}{74.0}不知道色,不知道色\twnr{集}{67.0},不知道色\twnr{滅}{68.0},不知道導向色\twnr{滅道跡}{69.0}。不知道受……(中略)想……諸行……不知道識,不知道識集,不知道識滅,不知道導向識滅道跡,學友!這被稱為無明,而這個情形是進入無明者。」[\suttaref{SN.22.113}]

  在這麼說時,尊者舍利弗對尊者摩訶拘絺羅說這個:

  「拘絺羅學友!被稱為『明、明』,學友!什麼是明?而什麼情形是進入明者?」

  「學友!這裡,\twnr{有聽聞的聖弟子}{24.0}知道色,知道色集,知道色滅,知道導向色滅道跡;受……想……諸行……知道識,知道識集,知道識滅,知道導向識滅道跡,學友!這被稱為明,而這個情形是進入明者。」[\suttaref{SN.22.114}]

  無明品第十三,其\twnr{攝頌}{35.0}:

  「集法三則,樂味二則在後,

   集起二說,拘絺羅三則在後。」





\pin{熱灰燼品}{136}{149}
\sutta{136}{136}{熱灰燼經}{https://agama.buddhason.org/SN/sn.php?keyword=22.136}
  起源於舍衛城。  

  「\twnr{比丘}{31.0}們!色是熱灰燼;受是熱灰燼;想是熱灰燼;諸行是熱灰燼;識是熱灰燼,比丘們!這麼看的\twnr{有聽聞的聖弟子}{24.0}在色上\twnr{厭}{15.0},也在受上厭,也在想上厭,也在諸行上厭,也在識上厭。厭者\twnr{離染}{558.0},從\twnr{離貪}{77.0}被解脫,在已解脫時,\twnr{有『[這是]解脫』之智}{27.0},他知道:『\twnr{出生已盡}{18.0},\twnr{梵行已完成}{19.0},\twnr{應該被作的已作}{20.0},\twnr{不再有此處[輪迴]的狀態}{21.1}。』」



\sutta{137}{137}{無常經}{https://agama.buddhason.org/SN/sn.php?keyword=22.137}
  起源於舍衛城。

  「\twnr{比丘}{31.0}們!凡是無常的,在那裡意欲應該被你們捨斷。比丘們!而什麼是無常的呢?比丘們!色是無常的,在那裡意欲應該被你們捨斷;受是無常的……(中略)想……諸行……識是無常的,在那裡意欲應該被你們捨斷。比丘們!凡是無常的,在那裡意欲應該被你們捨斷。」



\sutta{138}{138}{無常經第二}{https://agama.buddhason.org/SN/sn.php?keyword=22.138}
  起源於舍衛城。

  「\twnr{比丘}{31.0}們!凡是無常的,在那裡,貪應該被你們捨斷。比丘們!而什麼是無常的呢?比丘們!色是無常的,在那裡,貪應該被你們捨斷;受是無常的……想……諸行……識是無常的,在那裡,貪應該被你們捨斷。比丘們!凡是無常的,在那裡,貪應該被你們捨斷。」



\sutta{139}{139}{無常經第三}{https://agama.buddhason.org/SN/sn.php?keyword=22.139}
  起源於舍衛城。

  「\twnr{比丘}{31.0}們!凡是無常的,在那裡,意欲貪應該被你們捨斷。比丘們!而什麼是無常的呢?比丘們!色是無常的,在那裡,意欲貪應該被你們捨斷;受是無常的……想……諸行……識是無常的,在那裡,意欲貪應該被你們捨斷。比丘們!凡是無常的,在那裡,意欲貪應該被你們捨斷。」



\sutta{140}{140}{苦經}{https://agama.buddhason.org/SN/sn.php?keyword=22.140}
  起源於舍衛城。

  「\twnr{比丘}{31.0}們!凡是苦的,在那裡意欲應該被你們捨斷……(中略)比丘們!凡是苦的,在那裡意欲應該被你們捨斷。」



\sutta{141}{141}{苦經第二}{https://agama.buddhason.org/SN/sn.php?keyword=22.141}
  起源於舍衛城。

  「\twnr{比丘}{31.0}們!凡是苦的,在那裡,貪應該被你們捨斷……(中略)比丘們!凡是苦的,在那裡,貪應該被你們捨斷。」



\sutta{142}{142}{苦經第三}{https://agama.buddhason.org/SN/sn.php?keyword=22.142}
  起源於舍衛城。

  「\twnr{比丘}{31.0}們!凡是苦的,在那裡,意欲貪應該被你們捨斷……(中略)比丘們!凡是苦的,在那裡,意欲貪應該被你們捨斷。」



\sutta{143}{143}{無我經}{https://agama.buddhason.org/SN/sn.php?keyword=22.143}
  起源於舍衛城。

  「\twnr{比丘}{31.0}們!凡\twnr{無我}{23.0}的,在那裡意欲應該被你們捨斷。比丘們!而什麼是無我的呢?比丘們!色是無我的,在那裡意欲應該被你們捨斷;受是無我的……想……諸行……識是無我的,在那裡意欲應該被你們捨斷,比丘們!凡無我的,在那裡意欲應該被你們捨斷。」



\sutta{144}{144}{無我經第二}{https://agama.buddhason.org/SN/sn.php?keyword=22.144}
  起源於舍衛城。

  「\twnr{比丘}{31.0}們!凡\twnr{無我}{23.0}的,在那裡,貪應該被你們捨斷。比丘們!而什麼是無我的呢?比丘們!色是無我的,在那裡,貪應該被你們捨斷;受是無我的……想……諸行……識是無我的,在那裡,貪應該被你們捨斷,比丘們!凡無我的,在那裡,貪應該被你們捨斷。」



\sutta{145}{145}{無我經第三}{https://agama.buddhason.org/SN/sn.php?keyword=22.145}
  起源於舍衛城。

  「\twnr{比丘}{31.0}們!凡\twnr{無我}{23.0}的,在那裡,意欲貪應該被你們捨斷。比丘們!而什麼是無我的呢?比丘們!色是無我的,在那裡,意欲貪應該被你們捨斷;受是無我的……想……諸行……識是無我的,在那裡,意欲貪應該被你們捨斷。比丘們!凡無我的,在那裡,意欲貪應該被你們捨斷。」



\sutta{146}{146}{多厭經}{https://agama.buddhason.org/SN/sn.php?keyword=22.146}
  起源於舍衛城。

  「\twnr{比丘}{31.0}們!這是從信出家\twnr{善男子}{41.0}的\twnr{隨法}{58.0}:凡在色上應該住於\twnr{多厭}{x456},在受上……(中略)在想上……在諸行上……在識上應該住於多\twnr{厭}{15.0}。

  凡在色上住於多厭者;在受上……在想上……在諸行上……在識上住於多厭者\twnr{遍知}{154.0}色……受……想……諸行……遍知識,那位遍知色者、遍知受者、遍知想者、遍知諸行者、遍知識者從色被釋放、從受被釋放、從想被釋放、從諸行被釋放,從識被釋放,從生、老、死、愁、悲、苦、憂、\twnr{絕望}{342.0}被釋放,我說:『從苦被釋放。』」[\suttaref{SN.22.39}]



\sutta{147}{147}{隨看著無常經}{https://agama.buddhason.org/SN/sn.php?keyword=22.147}
  起源於舍衛城。

  「\twnr{比丘}{31.0}們!這是從信心出家\twnr{善男子}{41.0}的\twnr{隨法}{58.0}:應該住於在色上\twnr{隨看著}{59.0}無常;在受上……在想上……在諸行上……應該住於在識上隨看著無常。……(中略)我說:『從苦被釋放。』」



\sutta{148}{148}{隨看苦經}{https://agama.buddhason.org/SN/sn.php?keyword=22.148}
  起源於舍衛城。

  「\twnr{比丘}{31.0}們!這是從信心出家\twnr{善男子}{41.0}的\twnr{隨法}{58.0}:應該住於在色上\twnr{隨看著}{59.0}苦;在受上……在想上……在諸行上……應該住於在識上隨看苦。……(中略)我說:『從苦被釋放。』」



\sutta{149}{149}{隨看無我經}{https://agama.buddhason.org/SN/sn.php?keyword=22.149}
  起源於舍衛城。

  「\twnr{比丘}{31.0}們!這是從信出家\twnr{善男子}{41.0}的\twnr{隨法}{58.0}:應該住於在色上\twnr{隨看著}{59.0}無我,在受上……在想上……在諸行上……應該住於在識上隨看無我。

  當住於[在色上]隨看無我時;在受上……在想上……在諸行上……當住於在識上隨看無我時,他\twnr{遍知}{154.0}色、遍知受、遍知想、遍知諸行、遍知識;當遍知色、遍知受、遍知想、遍知諸行、遍知識時,他從色被釋放、從受被釋放、從想被釋放、從諸行被釋放,從識被釋放,從生、老、死、愁、悲、苦、憂、\twnr{絕望}{342.0}被釋放,我說:『從苦被釋放。』」

  熱灰燼品第十四,其\twnr{攝頌}{35.0}:

  「熱灰燼與三則無常,苦三則在後,

   無我三說,與善男子兩對。」





\pin{見品}{150}{159}
\sutta{150}{150}{自身內的經}{https://agama.buddhason.org/SN/sn.php?keyword=22.150}
  起源於舍衛城。

  「\twnr{比丘}{31.0}們!在有什麼時,執取什麼後,自身內的樂、苦生起呢?」

  「\twnr{大德}{45.0}!我們的法以\twnr{世尊}{12.0}為根本……(中略)。」

  「比丘們!在有色時,執取色後,自身內的樂、苦生起。

  在有受時……(中略)在有想時……在有諸行時……在有識時,執取識後,自身內的樂、苦生起。

  比丘們!你們怎麼想它:色是常的,或是無常的?」

  「無常的,大德!」

  「那麼,凡為無常的,那是苦的或樂的?」

  「苦的,大德!」

  「而凡為無常、苦、\twnr{變易法}{70.0},不執取它後,自身內的樂、苦會生起嗎?」

  「大德!這確實不是。」

  「受……想……諸行……識是常的,或是無常的?」

  「無常的,大德!」

  「那麼,凡為無常的,那是苦的或樂的?」

  「苦的,大德!」

  「而凡為無常、苦、變易法,不執取它後,自身內的樂、苦會生起嗎?」

  「大德!這確實不是。」

  「這麼看的……(中略)他知道:『……\twnr{不再有此處[輪迴]的狀態}{21.1}。』」[≃\suttaref{SN.35.105}]



\sutta{151}{151}{這是我的經}{https://agama.buddhason.org/SN/sn.php?keyword=22.151}
  起源於舍衛城。

  「\twnr{比丘}{31.0}們!在有什麼時,執取什麼後,執著什麼後\twnr{認為}{964.0}:『\twnr{這是我的}{32.0},\twnr{我是這個}{33.0},\twnr{這是我的真我}{34.1}。』嗎?」

  「\twnr{大德}{45.0}!我們的法以\twnr{世尊}{12.0}為根本……(中略)。」

  「比丘們!在有色時,執取色後,執著色後……(中略)在有識時,執取識後,執著識後認為:『\twnr{這是我的}{32.0},\twnr{我是這個}{33.0},這是\twnr{我的真我}{34.0}。』

  比丘們!你們怎麼想它:色是常的,或是無常的?」

  「無常的,大德!」……

  「……(中略)\twnr{變易法}{70.0},不執取它後,會認為:『這是我的,我是這個,這是我的真我。』嗎?」

  「大德!這確實不是。」

  「受……想……諸行……識是常的,或是無常的?」

  「無常的,大德!」……

  「……(中略)變易法,不執取它後,會認為:『這是我的,我是這個,這是我的真我。』嗎?」

  「大德!這確實不是。」

  「這麼看的……(中略)他知道:『……\twnr{不再有此處[輪迴]的狀態}{21.1}。』」



\sutta{152}{152}{彼-我經}{https://agama.buddhason.org/SN/sn.php?keyword=22.152}
  起源於舍衛城。

  「\twnr{比丘}{31.0}們!在有什麼時,執取什麼後,執著什麼後,這樣的見生起:『\twnr{彼是我者彼即是世間}{887.0},那個我死後將成為常的、堅固的、常恆的、不\twnr{變易法}{70.0}。』呢?」

  「\twnr{大德}{45.0}!我們的法以\twnr{世尊}{12.0}為根本……(中略)。」

  「比丘們!在有色時,執取色後,執著色後,這樣的見生起:『彼是我者彼即是世間,那個我死後將成為常的、堅固的、常恆的、不變易法。』

  在有受時……(中略)在有想時……在有諸行時……(中略)在有識時,執取識後,執著識後,這樣的見生起:『彼是我者彼即是世間,那個我死後將成為常的、堅固的、常恆的、不變易法。』

  比丘們!你們怎麼想它:色是常的,或是無常的?」

  「無常的,大德!」

  「那麼,凡為無常的,那是苦的或樂的?」

  「苦的,大德!」

  「而凡為無常、苦、變易法,是否不執取它後,這樣的見會生起:『彼是我者彼即是世間,那個我死後將成為常的、堅固的、常恆的、不變易法。』呢?」

  「大德!這確實不是。」

  「受……想……諸行……識是常的,或是無常的?」

  「無常的,大德!」

  「那麼,凡為無常的,那是苦的或樂的?」

  「苦的,大德!」

  「而凡為無常、苦、變易法,是否不執取它後,這樣的見會生起:『彼是我者彼即是世間,那個我死後將成為常的、堅固的、常恆的、不變易法。』呢?」

  「大德!這確實不是。」

  「這麼看的……(中略)他知道:『……\twnr{不再有此處[輪迴]的狀態}{21.1}。』」[\suttaref{SN.24.3}]



\sutta{153}{153}{我的不會存在經}{https://agama.buddhason.org/SN/sn.php?keyword=22.153}
  起源於舍衛城。

  「\twnr{比丘}{31.0}們!在有什麼時,執取什麼後,執著什麼後,這樣的見生起:『我不會存在以及我的不會存在,如果我不存在,\twnr{我的將不存在}{616.0}。』呢?」

  「\twnr{大德}{45.0}!我們的法以\twnr{世尊}{12.0}為根本……(中略)。」

  「比丘們!在有色時,執取色後,執著色後,這樣的見生起:『我不會存在以及我的不會存在,如果我不存在,我的將不存在。』

  在有受時……在有想時……在有諸行時……在有識時,執取識後,執著識後,這樣的見生起:『我不會存在以及我的不會存在,如果我不存在,我的將不存在。』

  比丘們!你們怎麼想它:色是常的,或是無常的?」

  「無常的,大德!」

  「那麼,凡為無常的,那是苦的或樂的?」

  「苦的,大德!」

  「而凡為無常、苦、\twnr{變易法}{70.0},是否不執取它後,這樣的見會生起:『我不會存在以及我的不會存在,如果我不存在,我的將不存在。』呢?」

  「大德!這確實不是。」

  「受……想……諸行……識是常的,或是無常的?」

  「無常的,大德!」

  「那麼,凡為無常的,那是苦的或樂的?」

  「苦的,大德!」

  「而凡為無常、苦、變易法,是否不執取它後,這樣的見會生起:『我不會存在以及我的不會存在,如果我不存在,我的將不存在。』呢?」

  「大德!這確實不是。」

  「這麼看的……(中略)他知道:『……\twnr{不再有此處[輪迴]的狀態}{21.1}。』」[≃\suttaref{SN.24.4}]



\sutta{154}{154}{邪見經}{https://agama.buddhason.org/SN/sn.php?keyword=22.154}
  起源於舍衛城。

  「\twnr{比丘}{31.0}們!在有什麼時,執取什麼後,執著什麼後,邪見生起呢?」

  「\twnr{大德}{45.0}!我們的法以\twnr{世尊}{12.0}為根本……(中略)。」

  「比丘們!在有色時,執取色後,執著色後,邪見生起。

  在有受時……在有想時……在有諸行時……在有識時,執取識後,執著識後,邪見生起。

  比丘們!你們怎麼想它:色是常的,或是無常的?」

  「無常的,大德!」

  「而凡為無常的……(中略)不執取它後,邪見會生起嗎?」

  「大德!這確實不是。」

  「受……想……諸行……識是常的,或是無常的?」

  「無常的,大德!」

  「那麼,凡為無常的,那是苦的或樂的?」

  「苦的,大德!」

  「而凡為無常、苦、\twnr{變易法}{70.0},不執取它後,邪見會生起嗎?」

  「大德!這確實不是。」

  「這麼看的……(中略)他知道:『……\twnr{不再有此處[輪迴]的狀態}{21.1}。』」



\sutta{155}{155}{有身見經}{https://agama.buddhason.org/SN/sn.php?keyword=22.155}
  起源於舍衛城。

  「\twnr{比丘}{31.0}們!在有什麼時,執取什麼後,執著什麼後,\twnr{有身見}{93.1}生起呢?」

  「\twnr{大德}{45.0}!我們的法以\twnr{世尊}{12.0}為根本……(中略)。」

  「比丘們!在有色時,執取色後,執著色後,有身見生起。

  在有受時……在有想時……在有諸行時……在有識時,執取識後,執著識後,有身見生起。

  比丘們!你們怎麼想它:色是常的,或是無常的?」

  「無常的,大德!」

  「而凡為無常的……(中略)不執取它後,有身見會生起嗎?」

  「大德!這確實不是。」

  「受……想……諸行……識是常的,或是無常的?」

  「無常的,大德!」

  「而凡為無常的……(中略)不執取它後,有身見會生起嗎?」

  「大德!這確實不是。」

  「這麼看的……(中略)他知道:『……\twnr{不再有此處[輪迴]的狀態}{21.1}。』」



\sutta{156}{156}{我隨見經}{https://agama.buddhason.org/SN/sn.php?keyword=22.156}
  起源於舍衛城。

  「\twnr{比丘}{31.0}們!在有什麼時,執取什麼後,執著什麼後我隨見生起呢?」

  「\twnr{大德}{45.0}!我們的法以\twnr{世尊}{12.0}為根本……(中略)。」

  「比丘們!在有色時,執取色後,執著色後,我隨見生起。

  在有受時……在有想時……在有諸行時……在有識時,執取識後,執著識後,我隨見生起。

  比丘們!你們怎麼想它:色是常的,或是無常的?」

  「無常的,大德!」

  「而凡為無常的……(中略)不執取它後,我隨見會生起嗎?」

  「大德!這確實不是。」

  「受……想……諸行……識是常的,或是無常的?」

  「無常的,大德!」

  「而凡為無常的……(中略)不執取它後,我隨見會生起嗎?」

  「大德!這確實不是。」

  「這麼看的……(中略)他知道:『……\twnr{不再有此處[輪迴]的狀態}{21.1}。』」



\sutta{157}{157}{執持經}{https://agama.buddhason.org/SN/sn.php?keyword=22.157}
  起源於舍衛城。

  「\twnr{比丘}{31.0}們!在有什麼時,執取什麼後,執著什麼後,結、執持、繫縛生起呢?」

  「\twnr{大德}{45.0}!我們的法以\twnr{世尊}{12.0}為根本……(中略)。」

  「比丘們!在有色時,執取色後,執著色後,結、執持、繫縛生起。

  在有受時……在有想時……在有行時……在有識時,執取識後,執著識後,結、執持、繫縛生起。

  比丘們!你們怎麼想它:色是常的,或是無常的?」

  「無常的,大德!」

  「而凡為無常的……(中略)不執取它後,結、執持、繫縛會生起嗎?」

  「大德!這確實不是。」

  「受……想……諸行……識是常的,或是無常的?」

  「無常的,大德!」

  「而凡為無常的……(中略)不執取它後,結、執持、繫縛會生起嗎?」

  「大德!這確實不是。」

  「這麼看的……(中略)他知道:『……\twnr{不再有此處[輪迴]的狀態}{21.1}。』」



\sutta{158}{158}{執持經第二}{https://agama.buddhason.org/SN/sn.php?keyword=22.158}
  起源於舍衛城。

  「\twnr{比丘}{31.0}們!在有什麼時,執取什麼後,執著什麼後,結、執持、繫縛、取著生起呢?」

  「\twnr{大德}{45.0}!我們的法以\twnr{世尊}{12.0}為根本……(中略)。」

  「比丘們!在有色時,執取色後,對色執著,結、執持、繫縛、取著生起。

  在有受時……在有想時……在有行時……在有識時,執取識後,執著識後,結、執持、繫縛、取著生起。

  比丘們!你們怎麼想它:色是常的,或是無常的?」

  「無常的,大德!」

  「而凡為無常的……(中略)不執取它後,結、執持、繫縛、取著會生起嗎?」

  「大德!這確實不是。」……(中略)

  「這麼看的……(中略)他知道:『……\twnr{不再有此處[輪迴]的狀態}{21.1}。』」



\sutta{159}{159}{阿難經}{https://agama.buddhason.org/SN/sn.php?keyword=22.159}
  起源於舍衛城。  

  那時,\twnr{尊者}{200.0}阿難去見\twnr{世尊}{12.0}。抵達後……(中略)對世尊說這個:

  「\twnr{大德}{45.0}!請世尊為我簡要地教導法,凡我聽聞世尊的法後,會住於單獨的、隱離的、不放逸的、熱心的、自我努力的,\twnr{那就好了}{44.0}!」

  「阿難!你怎麼想它:色是常的,或是無常的?」

  「無常的,大德!」

  「那麼,凡為無常的,那是苦的或樂的?」

  「苦的,大德!」

  「那麼,凡為無常的、苦的、\twnr{變易法}{70.0},適合認為它:『\twnr{這是我的}{32.0},\twnr{我是這個}{33.0},\twnr{這是我的真我}{34.1}。』嗎?」

  「大德!這確實不是。」

  「受……想……諸行……識是常的,或是無常的?」

  「無常的,大德!」

  「那麼,凡為無常的,那是苦的或樂的?」

  「苦的,大德!」

  「那麼,凡為無常的、苦的、變易法,適合認為它:『\twnr{這是我的}{32.0},\twnr{我是這個}{33.0},這是\twnr{我的真我}{34.0}。』嗎?」

  「大德!這確實不是。」

  「這麼看的……(中略)他知道:『……\twnr{不再有此處[輪迴]的狀態}{21.1}。』」

  見品第十五,其\twnr{攝頌}{35.0}:

  「自身內的、這是我的,彼-我、我的不會存在,

   邪、有身、我隨[見],二則執持、阿難。」

  後五十則終了。

  那個後五十則的品之攝頌:

  「邊、說法者、無明,熱灰燼、見為第五,

   三個五十說,被稱為『集篇』。」

  蘊相應完成。





\page

\xiangying{23}{羅陀相應}
\pin{第一品}{1}{10}
\sutta{1}{1}{魔經}{https://agama.buddhason.org/SN/sn.php?keyword=23.1}
  起源於舍衛城。

  那時,\twnr{尊者}{200.0}羅陀去見世尊。抵達後,向世尊\twnr{問訊}{46.0}後,在一旁坐下。在一旁坐下的尊者羅陀對世尊說這個:

  「\twnr{大德}{45.0}!被稱為『魔、魔』,大德!什麼情形是魔?」

  「羅陀!在色存在時會有魔,或殺者,又或凡它死亡者。

  羅陀!因此,在這裡,對色,你要看為『魔』,你要看為『殺者』,你要看為『它死亡』,你要看為『病』,你要看為『腫瘤』,你要看為『箭』,你要看為『禍』,你要看為『禍之類的』。凡這樣看它者,他們正確地看見。

  在受存在時……在想存在時……在諸行存在時……在識存在時會有魔,或殺者,又或凡它死亡者。羅陀!因此,在這裡,你要看識為『魔』,你要看為『殺者』,你要看為『死者』,你要看為『病』,你要看為『腫瘤』,你要看為『箭』,你要看為『禍』,你要看為『禍之類的』。凡這樣看它者,他們正確地看見。」

  「大德!那麼,正確之看見的目的是什麼?」

  「羅陀!正確之看見的目的是\twnr{厭}{15.0}。」

  「大德!那麼,厭的目的是什麼?」

  「羅陀!厭的目的是\twnr{離貪}{77.0}。」

  「大德!那麼,離貪的目的是什麼?」

  「羅陀!離貪的目的是解脫。」

  「大德!那麼,解脫的目的是什麼?」

  「羅陀!解脫的目的是涅槃。」

  「大德!那麼,涅槃的目的是什麼?」

  「羅陀!你越過問題,不能夠把握問題的極限,羅陀!因為\twnr{梵行}{381.0}被住於涅槃為立足處、\twnr{涅槃為彼岸}{226.2}、涅槃為完結。[\ccchref{MN.44}{https://agama.buddhason.org/MN/dm.php?keyword=44}]」



\sutta{2}{2}{眾生經}{https://agama.buddhason.org/SN/sn.php?keyword=23.2}
  起源於舍衛城。

  在一旁坐下的\twnr{尊者}{200.0}羅陀對\twnr{世尊}{12.0}說這個:

  「\twnr{大德}{45.0}!被稱為『\twnr{眾生}{x457}、眾生』,大德!什麼情形被稱為『眾生』?」

  「羅陀!在色上凡意欲,凡貪,凡歡喜,凡渴愛,在那裡執著者,在那裡強力執著者,因此被稱為『眾生』;在受上……在想上……在行上……在識上凡意欲,凡貪,凡歡喜,凡渴愛,在那裡執著者,在那裡強力執著者,因此被稱為『眾生』。

  羅陀!猶如男孩們或女孩們以泥土屋玩,只要在那些泥土屋上未離貪、未離意欲、未離情愛、未離渴望、未離熱惱、未離渴愛,他們就對那些泥土屋黏著、愛惜、視為自己的財富、當做自己的,羅陀!但當男孩們或女孩們在那些泥土屋上已離貪、已離意欲、已離情愛、已離渴望、已離熱惱、已離渴愛,那時,他們對那些泥土屋以手或以腳分散、破壞、碎破,作無樂之物。同樣的,羅陀!你們也要對色分散、破壞、碎破,作無樂之物,走上為了渴愛的滅盡之路;對受分散、破壞、碎破,作無樂之物,走上為了渴愛的滅盡之路;對想……對諸行分散、破壞、碎破,作無樂之物,走上為了渴愛的滅盡之路;對識分散、破壞、碎破,作無樂之物,走上為了渴愛的滅盡之路。

  羅陀!因為,渴愛的滅盡是涅槃。」



\sutta{3}{3}{有之管道經}{https://agama.buddhason.org/SN/sn.php?keyword=23.3}
  起源於舍衛城。

  在一旁坐下的\twnr{尊者}{200.0}羅陀對\twnr{世尊}{12.0}說這個:

  「\twnr{大德}{45.0}!被稱為『\twnr{有之管道}{279.0}的滅,有之管道的滅』,大德!什麼是有之管道?什麼是有之管道的滅?」

  「羅陀!在色上凡意欲,凡貪,凡歡喜,凡渴愛,凡攀住、執取、心的依處、執持、煩惱潛在趨勢,這被稱為『有之管道』,它們的滅為有之管道的滅。

  在受上……在想上……在諸行上……在識上凡意欲……(中略)依處、執持、煩惱潛在趨勢,這被稱為『有之管道』,它們的滅為有之管道的滅。」



\sutta{4}{4}{應該被遍知的經}{https://agama.buddhason.org/SN/sn.php?keyword=23.4}
  起源於舍衛城。

  \twnr{尊者}{200.0}羅陀去見\twnr{世尊}{12.0}。抵達後,向世尊\twnr{問訊}{46.0}後,在一旁坐下。世尊對在一旁坐下的尊者羅陀說這個:

  「羅陀!我將教導應該被遍知的法,以及\twnr{遍知}{154.0},與有遍知的人,你要聽它!你要\twnr{好好作意}{43.1}!我將說。」

  「是的,\twnr{大德}{45.0}!」尊者羅陀回答世尊。

  世尊說這個:

  「羅陀!而什麼是應該被遍知的法?羅陀!色是應該被遍知的法,受是應該被遍知之法,想是應該被遍知之法,諸行是應該被遍知之法,識是應該被遍知的法,羅陀!這些被稱為應該被遍知的法。

  羅陀!而什麼是遍知?羅陀!凡貪的滅盡、瞋的滅盡、癡的滅盡,羅陀!這被稱為遍知。

  羅陀!而什麼是有遍知的人?『\twnr{阿羅漢}{5.0}』應該被回答。凡這樣名、這樣姓的這位\twnr{尊者}{200.0},羅陀!這被稱為有遍知的人。」[\suttaref{SN.22.106}]



\sutta{5}{5}{沙門經}{https://agama.buddhason.org/SN/sn.php?keyword=23.5}
  起源於舍衛城。

  \twnr{世尊}{12.0}對在一旁坐下的\twnr{尊者}{200.0}羅陀說這個:

  「羅陀!有這些\twnr{五取蘊}{36.0},哪五個?色取蘊、受取蘊、想取蘊、行取蘊、識取蘊,羅陀!凡任何\twnr{沙門}{29.0}或\twnr{婆羅門}{17.0}不如實知道這些五取蘊的\twnr{樂味}{295.0}、\twnr{過患}{293.0}、\twnr{出離}{294.0}者,羅陀!那些沙門或婆羅門不被我認同為\twnr{沙門中的沙門}{560.0},或婆羅門中的婆羅門,而且,那些\twnr{尊者}{200.0}也不以證智自作證後,在當生中\twnr{進入後住於}{66.0}\twnr{沙門義}{327.0}或婆羅門義。

  羅陀!而凡任何沙門或婆羅門如實知道這些五取蘊的樂味、過患、出離者,羅陀!那些沙門或婆羅門被我認同為沙門中的沙門,或婆羅門中的婆羅門,而且,那些尊者也以證智自作證後,在當生中進入後住於沙門義或婆羅門義。」



\sutta{6}{6}{沙門經第二}{https://agama.buddhason.org/SN/sn.php?keyword=23.6}
  起源於舍衛城。

  \twnr{世尊}{12.0}對在一旁坐下的\twnr{尊者}{200.0}羅陀說這個:

  「羅陀!有這些\twnr{五取蘊}{36.0},哪五個?色取蘊……(中略)識取蘊,羅陀!凡任何\twnr{沙門}{29.0}或\twnr{婆羅門}{17.0}不如實知道這些五取蘊的\twnr{集起}{67.0}、滅沒、\twnr{樂味}{295.0}、\twnr{過患}{293.0}、\twnr{出離}{294.0}者……(中略)以證智自作證後,\twnr{在當生中}{42.0}\twnr{進入後住於}{66.0}……。」



\sutta{7}{7}{入流者經}{https://agama.buddhason.org/SN/sn.php?keyword=23.7}
  起源於舍衛城。

  \twnr{世尊}{12.0}對在一旁坐下的\twnr{尊者}{200.0}羅陀說這個:

  「羅陀!有這些\twnr{五取蘊}{36.0},哪五個?色取蘊……(中略)識取蘊,羅陀!當聖弟子如實知道這些五取蘊的\twnr{集起}{67.0}、滅沒、\twnr{樂味}{295.0}、\twnr{過患}{293.0}、\twnr{出離}{294.0},羅陀!這位聖弟子被稱為\twnr{入流者}{165.0}、不墮惡趣法者、\twnr{決定者}{159.0}、\twnr{正覺為彼岸者}{160.0}。」 



\sutta{8}{8}{阿羅漢經}{https://agama.buddhason.org/SN/sn.php?keyword=23.8}
  起源於舍衛城。

  \twnr{世尊}{12.0}對在一旁坐下的\twnr{尊者}{200.0}羅陀說這個:

  「羅陀!有這些\twnr{五取蘊}{36.0},哪五個?色取蘊……(中略)識取蘊,羅陀!當\twnr{比丘}{31.0}如實知道這些五取蘊的\twnr{集起}{67.0}、滅沒、\twnr{樂味}{295.0}、\twnr{過患}{293.0}、\twnr{出離}{294.0}後,不執取後成為解脫者,羅陀!這被稱為漏已滅盡的、已完成的、\twnr{應該被作的已作的}{20.0}、負擔已卸的、\twnr{自己的利益已達成的}{189.0}、\twnr{有之結已滅盡的}{190.0}、以\twnr{究竟智}{191.0}解脫的\twnr{阿羅漢}{5.0}比丘。」 



\sutta{9}{9}{意欲貪經}{https://agama.buddhason.org/SN/sn.php?keyword=23.9}
  起源於舍衛城。 

  \twnr{世尊}{12.0}對在一旁坐下的\twnr{尊者}{200.0}羅陀說這個:

  「羅陀!在色上凡意欲,凡貪,凡歡喜,凡渴愛,你們要捨斷它,這樣,那個色必將被捨斷,根被切斷,\twnr{[如]已斷根的棕櫚樹}{147.1},\twnr{成為非有}{408.0},\twnr{為未來不生之物}{229.0}。

  在受上凡意欲,凡貪,凡歡喜,凡渴愛,你們要捨斷它,這樣,那個受必將被捨斷,根被切斷,[如]已斷根的棕櫚樹,成為非有,為未來不生之物。在想上……在諸行上凡意欲,凡貪,凡歡喜,凡渴愛,你們要捨斷它,這樣,那些行必將被捨斷,根被切斷,[如]已斷根的棕櫚樹,成為非有,為未來不生之物。在識上凡意欲,凡貪,凡歡喜,凡渴愛,你們要捨斷它,這樣,那個識必將被捨斷……(中略)為未來不生之物。」[\suttaref{SN.22.111}]



\sutta{10}{10}{意欲貪經第二}{https://agama.buddhason.org/SN/sn.php?keyword=23.10}
  起源於舍衛城。 

  \twnr{世尊}{12.0}對在一旁坐下的\twnr{尊者}{200.0}羅陀說這個:

  「羅陀!在色上凡意欲,凡貪,凡歡喜,凡渴愛,凡攀住、執取、心的依處、執持、\twnr{煩惱潛在趨勢}{253.1},你們要捨斷那些,這樣,那個色必將被捨斷,根被切斷,\twnr{[如]已斷根的棕櫚樹}{147.1},\twnr{成為非有}{408.0},\twnr{為未來不生之物}{229.0}。

  在受上凡意欲,凡貪,凡歡喜,凡渴愛,凡攀住、執取、心的依處、執持、煩惱潛在趨勢,你們要捨斷它,這樣,那個受必將被捨斷,根被切斷,[如]已斷根的棕櫚樹,成為非有,為未來不生之物。在想上……在諸行上凡意欲,凡貪,凡歡喜,凡渴愛,凡攀住、執取、心的依處、執持、煩惱潛在趨勢,你們要捨斷它,這樣,那些行必將被捨斷,根被切斷,[如]已斷根的棕櫚樹,成為非有,為未來不生之物。在識上凡意欲,凡貪,凡歡喜,凡渴愛,凡攀住、執取、心的依處、執持、煩惱潛在趨勢,你們要捨斷那些,這樣,那個識必將被捨斷,根被切斷,[如]已斷根的棕櫚樹,成為非有,為未來不生之物。」[\suttaref{SN.22.112}]

  第一品,其\twnr{攝頌}{35.0}:

  「魔、眾生、有之管道,應該被遍知的、沙門二則,

   入流者與阿羅漢,意欲貪隨後的二則。」





\pin{第二品}{11}{22}
\sutta{11}{11}{魔經}{https://agama.buddhason.org/SN/sn.php?keyword=23.11}
  起源於舍衛城。

  在一旁坐下的\twnr{尊者}{200.0}羅陀對\twnr{世尊}{12.0}說這個:

  「\twnr{大德}{45.0}!被稱為『魔、魔』,大德!什麼是魔?」

  「羅陀!色是魔,受是魔,想是魔,諸行是魔,識是魔。

  羅陀!這麼看的有聽聞的聖弟子在色上\twnr{厭}{15.0},也在受上厭,也在想上厭,也在諸行上厭,也在識上厭。厭者\twnr{離染}{558.0},從\twnr{離貪}{77.0}被解脫,在已解脫時,\twnr{有『[這是]解脫』之智}{27.0},他知道:『\twnr{出生已盡}{18.0},\twnr{梵行已完成}{19.0},\twnr{應該被作的已作}{20.0},\twnr{不再有此處[輪迴]的狀態}{21.1}。』」



\sutta{12}{12}{魔法經}{https://agama.buddhason.org/SN/sn.php?keyword=23.12}
  起源於舍衛城。

  在一旁坐下的\twnr{尊者}{200.0}羅陀對\twnr{世尊}{12.0}說這個:

  「\twnr{大德}{45.0}!被稱為『\twnr{魔法}{x458}、魔法』,大德!什麼是魔法?」

  「羅陀!色是魔法,受是魔法,想是魔法,諸行是魔法,識是魔法。

  這麼看的……(中略)他知道:『……\twnr{不再有此處[輪迴]的狀態}{21.1}。』」



\sutta{13}{13}{無常經}{https://agama.buddhason.org/SN/sn.php?keyword=23.13}
  起源於舍衛城。

  在一旁坐下的\twnr{尊者}{200.0}羅陀對\twnr{世尊}{12.0}說這個:

  「\twnr{大德}{45.0}!被稱為『無常、無常』,大德!什麼是無常?」

  「羅陀!色是無常,受是無常,想是無常,諸行是無常,識是無常。

  這麼看的……(中略)他知道:『……\twnr{不再有此處[輪迴]的狀態}{21.1}。』」



\sutta{14}{14}{無常法經}{https://agama.buddhason.org/SN/sn.php?keyword=23.14}
  起源於舍衛城。

  在一旁坐下的\twnr{尊者}{200.0}羅陀對\twnr{世尊}{12.0}說這個:

  「\twnr{大德}{45.0}!被稱為『無常法、無常法』,大德!什麼是無常法?」

  「羅陀!色是無常法,受是無常法,想是無常法,諸行是無常法,識是無常法。

  這麼看的……(中略)他知道:『……\twnr{不再有此處[輪迴]的狀態}{21.1}。』」



\sutta{15}{15}{苦經}{https://agama.buddhason.org/SN/sn.php?keyword=23.15}
  起源於舍衛城。

  在一旁坐下的\twnr{尊者}{200.0}羅陀對\twnr{世尊}{12.0}說這個:

  「\twnr{大德}{45.0}!被稱為『苦、苦』,大德!什麼是苦?」

  「羅陀!色是苦,受是苦,想是苦,諸行是苦,識是苦。

  這麼看的……(中略)他知道:『……\twnr{不再有此處[輪迴]的狀態}{21.1}。』」



\sutta{16}{16}{苦法經}{https://agama.buddhason.org/SN/sn.php?keyword=23.16}
  起源於舍衛城。

  在一旁坐下的\twnr{尊者}{200.0}羅陀對\twnr{世尊}{12.0}說這個:

  「\twnr{大德}{45.0}!被稱為『苦法、苦法』,大德!什麼是苦法?」

  「羅陀!色是苦法,受是苦法,想是苦法,諸行是苦法,識是苦法。

  這麼看的……(中略)他知道:『……\twnr{不再有此處[輪迴]的狀態}{21.1}。』」



\sutta{17}{17}{無我經}{https://agama.buddhason.org/SN/sn.php?keyword=23.17}
  起源於舍衛城。

  在一旁坐下的\twnr{尊者}{200.0}羅陀對\twnr{世尊}{12.0}說這個:

  「\twnr{大德}{45.0}!被稱為『\twnr{無我}{23.0}、無我』,大德!什麼是無我?」

  「羅陀!色是無我,受是無我,想是無我,諸行是無我,識是無我。

  這麼看的……(中略)他知道:『……\twnr{不再有此處[輪迴]的狀態}{21.1}。』」



\sutta{18}{18}{無我法經}{https://agama.buddhason.org/SN/sn.php?keyword=23.18}
  起源於舍衛城。

  在一旁坐下的\twnr{尊者}{200.0}羅陀對\twnr{世尊}{12.0}說這個:

  「\twnr{大德}{45.0}!被稱為『\twnr{無我}{23.0}法、無我法』,大德!什麼是無我法?」

  「羅陀!色是無我法,受是無我法,想是無我法,諸行是無我法,識是無我法。

  這麼看的……(中略)他知道:『……\twnr{不再有此處[輪迴]的狀態}{21.1}。』」



\sutta{19}{19}{滅盡法經}{https://agama.buddhason.org/SN/sn.php?keyword=23.19}
  起源於舍衛城。

  在一旁坐下的\twnr{尊者}{200.0}羅陀對\twnr{世尊}{12.0}說這個:

  「\twnr{大德}{45.0}!被稱為『\twnr{滅盡法}{273.1}、滅盡法』,大德!什麼是滅盡法?」

  「羅陀!色是滅盡法,受是滅盡法,想是滅盡法,諸行是滅盡法,識是滅盡法。

  這麼看的……(中略)他知道:『……\twnr{不再有此處[輪迴]的狀態}{21.1}。』」



\sutta{20}{20}{消散法經}{https://agama.buddhason.org/SN/sn.php?keyword=23.20}
  起源於舍衛城。

  在一旁坐下的\twnr{尊者}{200.0}羅陀對\twnr{世尊}{12.0}說這個:

  「\twnr{大德}{45.0}!被稱為『\twnr{消散法}{155.0}、消散法』,大德!什麼是消散法?」

  「羅陀!色是消散法,受是消散法,想是消散法,諸行是消散法,識是消散法。

  這麼看的……(中略)他知道:『……\twnr{不再有此處[輪迴]的狀態}{21.1}。』」



\sutta{21}{21}{集法經}{https://agama.buddhason.org/SN/sn.php?keyword=23.21}
  起源於舍衛城。

  在一旁坐下的\twnr{尊者}{200.0}羅陀對\twnr{世尊}{12.0}說這個:

  「\twnr{大德}{45.0}!被稱為『\twnr{集法}{67.1}、集法』,大德!什麼是集法?」

  「羅陀!色是集法,受是集法,想是集法,諸行是集法,識是集法。

  這麼看的……(中略)他知道:『……\twnr{不再有此處[輪迴]的狀態}{21.1}。』」



\sutta{22}{22}{滅法經}{https://agama.buddhason.org/SN/sn.php?keyword=23.22}
  起源於舍衛城。

  在一旁坐下的\twnr{尊者}{200.0}羅陀對\twnr{世尊}{12.0}說這個:

  「\twnr{大德}{45.0}!被稱為『\twnr{滅法}{68.1}、滅法』,大德!什麼是滅法?」

  「羅陀!色是滅法,受是滅法,想是滅法,諸行是滅法,識是滅法。

  這麼看的……(中略)他知道:『……\twnr{不再有此處[輪迴]的狀態}{21.1}。』」

  羅陀相應的第二品,其\twnr{攝頌}{35.0}:

  「魔與魔法,以無常二則在後,

   以苦二說,與同樣地以無我,

   滅盡、消散、集,滅法為十二。」





\pin{勸請品}{23}{34}
\sutta{23}{33}{魔經等十一則}{https://agama.buddhason.org/SN/sn.php?keyword=23.23}
  起源於舍衛城。

  在一旁坐下的\twnr{尊者}{200.0}羅陀對\twnr{世尊}{12.0}說這個:

  「\twnr{大德}{45.0}!請世尊為我簡要地教導法,凡我聽聞世尊的法後,會住於單獨的、隱離的、不放逸的、熱心的、自我努力的,\twnr{那就好了}{44.0}!」

  「羅陀!凡為魔者,在那裡意欲應該被你捨斷、貪應該被捨斷、意欲貪應該被捨斷。

  而,羅陀!什麼是魔?羅陀!色是魔,在那裡意欲應該被你捨斷、貪應該被捨斷、意欲貪應該被捨斷;受是魔,在那裡意欲應該被你捨斷……(中略)想是魔,在那裡意欲應該被你捨斷……(中略)諸行是魔,在那裡意欲應該被你捨斷……(中略)識是魔,在那裡意欲應該被你捨斷……(中略)羅陀!凡為魔者,在那裡意欲應該被你捨斷、貪應該被捨斷、意欲貪應該被捨斷。

  羅陀!凡為魔法者,在那裡意欲應該被你捨斷、貪應該被捨斷、意欲貪應該被捨斷。……(中略)。」

  「羅陀!凡為無常者……(中略)。」

  「羅陀!凡為無常法者……(中略)。」

  「羅陀!凡為苦者……(中略)。」

  「羅陀!凡為苦法者……(中略)。」

  「羅陀!凡為無我者……(中略)。」

  「羅陀!凡為無我法者……(中略)。」

  「羅陀!凡為\twnr{滅盡法}{273.1}者……(中略)。」

  「羅陀!凡為\twnr{消散法}{155.0}者……(中略)。」

  「羅陀!凡為\twnr{集法}{67.1}者,在那裡意欲應該被你捨斷、貪應該被捨斷、意欲貪應該被捨斷。……(中略)。」



\sutta{34}{34}{滅法經}{https://agama.buddhason.org/SN/sn.php?keyword=23.34}
  起源於舍衛城。

  在一旁坐下的\twnr{尊者}{200.0}羅陀對\twnr{世尊}{12.0}說這個:

  「\twnr{大德}{45.0}!請世尊為我簡要地教導法,凡我聽聞世尊的法後,會住於單獨的、隱離的、不放逸的、熱心的、自我努力的,\twnr{那就好了}{44.0}!」

  「羅陀!凡為\twnr{滅法}{68.1}者,在那裡意欲應該被你捨斷、貪應該被捨斷、意欲貪應該被捨斷。

  而,羅陀!什麼是滅法?羅陀!色是滅法,在那裡意欲應該被你捨斷……(中略)受是滅法,在那裡意欲應該被你捨斷……(中略)想是滅法,在那裡意欲應該被你捨斷……(中略)諸行是滅法,在那裡意欲應該被你捨斷……(中略)識是滅法,在那裡意欲應該被你捨斷……(中略)羅陀!凡為滅法者,在那裡意欲應該被你捨斷、貪應該被捨斷、意欲貪應該被捨斷。」

  勸請品第三,其\twnr{攝頌}{35.0}:

  「魔與魔法,以無常二則在後,

   以苦二說,與同樣地以無我,

   滅盡、消散、集,滅法為十二。」 





\pin{近坐品}{35}{46}
\sutta{35}{45}{魔經等十一則}{https://agama.buddhason.org/SN/sn.php?keyword=23.35}
  起源於舍衛城。

  \twnr{世尊}{12.0}對在一旁坐下的\twnr{尊者}{200.0}羅陀說這個:

  「羅陀!凡為魔者,在那裡意欲應該被你捨斷、貪應該被捨斷、意欲貪應該被捨斷。

  羅陀!而什麼是魔?

  羅陀!色是魔,在那裡意欲應該被你捨斷……(中略)識是魔,在那裡意欲應該被你捨斷……(中略),羅陀!凡為魔者,在那裡意欲應該被你捨斷、貪應該被捨斷、意欲貪應該被捨斷。」

  「羅陀!凡為魔法者,在那裡意欲應該被你捨斷、貪應該被捨斷、意欲貪應該被捨斷。……(中略)。」

  「羅陀!凡為無常者……(中略)。」

  「羅陀!凡為無常法者……(中略)。」

  「羅陀!凡為苦者……(中略)。」

  「羅陀!凡為苦法者……(中略)。」

  「羅陀!凡為無我者……(中略)。」

  「羅陀!凡為無我法者……(中略)。」

  「羅陀!凡為\twnr{滅盡法}{273.1}者……(中略)。」

  「羅陀!凡為\twnr{消散法}{155.0}者……(中略)。」

  「羅陀!凡為\twnr{集法}{67.1}者,在那裡意欲應該被你捨斷、貪應該被捨斷、意欲貪應該被捨斷。……(中略)。」



\sutta{46}{46}{滅法經}{https://agama.buddhason.org/SN/sn.php?keyword=23.46}
  起源於舍衛城。

  \twnr{世尊}{12.0}對在一旁坐下的\twnr{尊者}{200.0}羅陀說這個:

  「羅陀!凡為\twnr{滅法}{68.1}者,在那裡意欲應該被你捨斷、貪應該被捨斷、意欲貪應該被捨斷。

  而,羅陀!什麼是滅法?羅陀!色是滅法,在那裡意欲應該被你捨斷、貪應該被捨斷、意欲貪應該被捨斷;受……(中略)想……(中略)諸行……(中略)識是滅法,在那裡意欲應該被你捨斷、貪應該被捨斷、意欲貪應該被捨斷,羅陀!凡為滅法者,在那裡意欲應該被你捨斷、貪應該被捨斷、意欲貪應該被捨斷。」

  近坐品第四,其\twnr{攝頌}{35.0}:

  「魔與魔法,以無常二則在後,

   以苦二說,與同樣地以無我,

   滅盡、消散、集,以滅法為十二。」

  羅陀相應完成。





\page

\xiangying{24}{見相應}
\pin{入流品}{1}{18}
\sutta{1}{1}{風經}{https://agama.buddhason.org/SN/sn.php?keyword=24.1}
  \twnr{有一次}{2.0},\twnr{世尊}{12.0}住在舍衛城祇樹林。

  世尊說這個:

  「\twnr{比丘}{31.0}們!在有什麼時,執取什麼後,執著什麼後,這樣的見生起:『風沒吹,河沒流,孕婦沒生,日月沒起落,\twnr{如直立不動的石柱}{x459}。』呢?」

  「\twnr{大德}{45.0}!我們的法是世尊為根本的、\twnr{世尊為導引的}{56.0}、世尊為依歸的,大德!請世尊澄清所說的義理,\twnr{那就好了}{44.0}!聽聞世尊的[教說]後,比丘們將會\twnr{憶持}{57.0}。」

  「比丘們!那樣的話,你們要聽!你們要\twnr{好好作意}{43.1}!我將說。」

  「是的,\twnr{大德}{45.0}!」那些比丘回答世尊。

  世尊說這個:

  「比丘們!在有色時,執取色後,執著色後,這樣的見生起:『風沒吹,河沒流,孕婦沒生,日月沒起落,如直立不動的石柱。』

  在有受時……(中略)在有想時……在有諸行時……在有識時,執取識後,執著識後,這樣的見生起:『風沒吹,河沒流,孕婦沒生,日月沒起落,如直立不動的石柱。』

  比丘們!你們怎麼想它:色是常的,或是無常的?」

  「無常的,大德!」

  「那麼,凡為無常的,那是苦的或樂的?」

  「苦的,大德!」

  「而凡為無常、苦、\twnr{變易法}{70.0},是否不執取它後,這樣的見會生起:『風沒吹,河沒流,孕婦沒生,日月沒起落,如直立不動的石柱。』呢?」

  「大德!這確實不是。」

  「受是常的,或是無常的?」……

  「想……諸行……識是常的,或是無常的?」

  「無常的,大德!」

  「那麼,凡為無常的,那是苦的或樂的?」

  「苦的,大德!」

  「而凡為無常、苦、變易法,是否不執取它後,這樣的見會生起:『風沒吹,河沒流,孕婦沒生,日月沒起落,如直立不動的石柱。』呢?」

  「大德!這確實不是。」

  「凡這個所見、所聞、所覺、所識、所得、所遍求、\twnr{被意所隨行}{396.0}都是常的,或是無常的?」

  「無常的,大德!」

  「那麼,凡為無常的,那是苦的或樂的?」

  「苦的,大德!」

  「而凡為無常、苦、變易法,是否不執取它後,這樣的見會生起:『風沒吹,河沒流,孕婦沒生,日月沒起落,如直立不動的石柱。』呢?」

  「大德!這確實不是。」

  「比丘們!當\twnr{聖弟子}{24.0}\twnr{在這些地方}{997.0}捨斷懷疑,捨斷苦的懷疑,捨斷苦\twnr{集}{67.0}懷疑,捨斷苦\twnr{滅}{68.0}的懷疑,捨斷導向苦\twnr{滅道跡}{69.0}的懷疑,比丘們!這位聖弟子被稱為\twnr{入流者}{165.0}、不墮惡趣法者、\twnr{決定者}{159.0}、\twnr{正覺為彼岸者}{160.0}。」



\sutta{2}{2}{這是我的經}{https://agama.buddhason.org/SN/sn.php?keyword=24.2}
  起源於舍衛城。

  「\twnr{比丘}{31.0}們!在有什麼時,執取什麼後,執著什麼後,這樣的見生起:『\twnr{這是我的}{32.0},\twnr{我是這個}{33.0},\twnr{這是我的真我}{34.1}。』嗎?」

  「\twnr{大德}{45.0}!我們的法以\twnr{世尊}{12.0}為根本……(中略)。」

  「比丘們!在有色時,執取色後,執著色後,這樣的見生起:『\twnr{這是我的}{32.0},\twnr{我是這個}{33.0},這是\twnr{我的真我}{34.0}。』

  在有受時……(中略)在有想時……在有諸行時……在有識時,執取識後,執著識後,這樣的見生起:『這是我的,我是這個,這是我的真我。』

  比丘們!你們怎麼想它:色是常的,或是無常的?」

  「無常的,大德!」……

  「受……想……諸行……識是常的,或是無常的?」

  「無常的,大德!」……

  「……(中略)是否不執取它後,這樣的見會生起:『這是我的,我是這個,這是我的真我。』嗎?」

  「大德!這確實不是。」

  「凡這個所見、所聞、所覺、所識、所得、所遍求、\twnr{被意所隨行}{396.0}都是常的,或是無常的?」

  「無常的,大德!」

  「那麼,凡為無常的,那是苦的或樂的?」

  「苦的,大德!」

  「而凡為無常、苦、\twnr{變易法}{70.0},是否不執取它後,這樣的見會生起:『這是我的,我是這個,這是我的真我。』嗎?」

  「大德!這確實不是。」

  「比丘們!當聖弟子\twnr{在這些地方}{997.0}捨斷懷疑,捨斷苦的懷疑……(中略)捨斷導向苦\twnr{滅道跡}{69.0}的懷疑,比丘們!這位聖弟子被稱為\twnr{入流者}{165.0}、不墮惡趣法者、\twnr{決定者}{159.0}、\twnr{正覺為彼岸者}{160.0}。」



\sutta{3}{3}{彼-我經}{https://agama.buddhason.org/SN/sn.php?keyword=24.3}
  起源於舍衛城。

  「\twnr{比丘}{31.0}們!在有什麼時,執取什麼後,執著什麼後,這樣的見生起:『\twnr{彼是我者彼即是世間}{887.0},那個我死後將成為常的、堅固的、常恆的、不\twnr{變易法}{70.0}。』呢?」

  「\twnr{大德}{45.0}!我們的法以\twnr{世尊}{12.0}為根本……(中略)。」

  「比丘們!在有色時,執取色後,執著色後,這樣的見生起:『彼是我者彼即是世間,那個我死後將成為常的、堅固的、常恆的、不變易法。』

  在有受時……(中略)在有想時……在有諸行時……在有識時,執取識後,執著識後,這樣的見生起:『彼是我者彼即是世間,那個我死後將成為常的、堅固的、常恆的、不變易法。』

  比丘們!你們怎麼想它:色是常的,或是無常的?」

  「無常的,大德!」……

  「……(中略)是否不執取它後,這樣的見會生起:『彼是我者……(中略)不變易法。』呢?」

  「大德!這確實不是。」

  「受……想……諸行……識是常的,或是無常的?」

  「無常的,大德!」……

  「……(中略)是否不執取它後,這樣的見會生起:『彼是我者……(中略)不變易法。』呢?」

  「大德!這確實不是。」

  「凡這個所見、所聞、所覺、所識、所得、所遍求、\twnr{被意所隨行}{396.0}都是常的,或是無常的?」

  「無常的,大德!」……

  「……(中略)是否不執取它後,這樣的見會生起:『彼是我者彼即是世間,那個我死後將成為常的、堅固的、常恆的、不變易法。』呢?」

  「大德!這確實不是。」

  「比丘們!當\twnr{聖弟子}{24.0}\twnr{在這些地方}{997.0}捨斷懷疑,捨斷苦的懷疑……(中略)捨斷導向苦\twnr{滅道跡}{69.0}的懷疑,比丘們!這位聖弟子被稱為\twnr{入流者}{165.0}、不墮惡趣法者、\twnr{決定者}{159.0}、\twnr{正覺為彼岸者}{160.0}。」[\suttaref{SN.22.152}]



\sutta{4}{4}{我的不會存在經}{https://agama.buddhason.org/SN/sn.php?keyword=24.4}
  起源於舍衛城。

  「\twnr{比丘}{31.0}們!在有什麼時,執取什麼後,執著什麼後,這樣的見生起:『我不會存在以及我的不會存在,如果我不存在,\twnr{我的將不存在}{616.0}。』呢?」

  「\twnr{大德}{45.0}!我們的法以\twnr{世尊}{12.0}為根本……(中略)。」

  「比丘們!在有色時,執取色後,執著色後,這樣的見生起:『我不會存在以及我的不會存在,如果我不存在,我的將不存在。』

  在有受時……在有想時……在有諸行時……在有識時,執取識後,執著識後,這樣的見生起:『我不會存在以及我的不會存在,如果我不存在,我的將不存在。』

  比丘們!你們怎麼想它:色是常的,或是無常的?」

  「無常的,大德!」……(中略)是否不執取它後,這樣的見會生起:『我不會存在以及我的不會存在,如果我不存在,我的將不存在。』呢?」

  「大德!這確實不是。」

  「受……想……諸行……識是常的,或是無常的?」

  「無常的,大德!」……(中略)是否不執取它後,這樣的見會生起:『我不會存在以及我的不會存在,如果我不存在,我的將不存在。』呢?」

  「大德!這確實不是。」

  「凡這個所見、所聞、所覺、所識、所得、所遍求、\twnr{被意所隨行}{396.0}都是常的,或是無常的?」

  「無常的,大德!」……(中略)是否不執取它後,這樣的見會生起:『我不會存在以及我的不會存在,如果我不存在,我的將不存在。』呢?」

  「大德!這確實不是。」

  「比丘們!當\twnr{聖弟子}{24.0}\twnr{在這些地方}{997.0}捨斷懷疑,捨斷苦的懷疑,捨斷苦\twnr{集}{67.0}懷疑,捨斷苦\twnr{滅}{68.0}的懷疑,捨斷導向苦\twnr{滅道跡}{69.0}的懷疑,比丘們!這位聖弟子被稱為\twnr{入流者}{165.0}、不墮惡趣法者、\twnr{決定者}{159.0}、\twnr{正覺為彼岸者}{160.0}。」[≃\suttaref{SN.22.153}]



\sutta{5}{5}{無布施經}{https://agama.buddhason.org/SN/sn.php?keyword=24.5}
  起源於舍衛城。

  「\twnr{比丘}{31.0}們!在有什麼時,執取什麼後,執著什麼後,這樣的見生起:『沒有被施與的,沒有被祭祀的,沒有被供養的,沒有善作惡作業的果、果報,沒有這個世間,沒有其他世間,沒有母親,沒有父親,沒有\twnr{化生}{346.0}眾生,在世間中沒有\twnr{正行的}{441.0}、正行道的\twnr{沙門}{29.0}、\twnr{婆羅門}{17.0}以證智自作證後告知這個世間與其他世間。\twnr{四大}{646.0}所成的這位男子當命終時,\twnr{地沒入、隨行地身}{x460},水沒入、隨行水身,火沒入、隨行火身,風沒入、隨行風身,\twnr{諸根轉移到虛空}{x461},\twnr{[四]人、長椅為第五拿取死者後走去}{795.0},\twnr{直到墓地為止[哀悼]諸句被知道}{x462},\twnr{骨頭成為灰白色}{x463},\twnr{祭品成為落下的}{796.0},布施即被愚者安立的,凡任何說有[布施]之論者,他們的[言論]全是空虛的、\twnr{虛妄的、無價值的話}{797.0},愚者與賢智者以身體的崩解被斷滅、消失;死後不存在。[\ccchref{DN.2}{https://agama.buddhason.org/DN/dm.php?keyword=2}, 171段]』呢?」

  「\twnr{大德}{45.0}!我們的法以\twnr{世尊}{12.0}為根本……(中略)。」

  「比丘們!在有色時,執取色後,執著色後,這樣的見生起:『沒有被施與的,沒有被祭祀的……(中略)以身體的崩解被斷滅、消失;死後不存在。』

  在有受時……(中略)在有想時……在有諸行時……在有識時,執取識後,執著識後,這樣的見生起:『沒有被施與的,沒有被祭祀的……(中略)以身體的崩解被斷滅、消失;死後不存在。』

  比丘們!你們怎麼想它:色是常的,或是無常的?」

  「無常的,大德!」……

  「……(中略)是否不執取它後,這樣的見會生起:『沒有被施與的,沒有被祭祀的……(中略)以身體的崩解被斷滅、消失;死後不存在。』呢?」

  「大德!這確實不是。」

  「受……想……諸行……識是常的,或是無常的?」

  「無常的,大德!」……

  「……(中略)是否不執取它後,這樣的見會生起:『沒有被施與的,沒有被祭祀的……(中略)以身體的崩解被斷滅、消失;死後不存在。』呢?」

  「大德!這確實不是。」

  「凡這個所見、所聞、所覺、所識、所得、所遍求、\twnr{被意所隨行}{396.0}都是常的,或是無常的?」

  「無常的,大德!」……

  「……(中略)是否不執取它後,這樣的見會生起:『沒有被施與的,沒有被祭祀的……(中略)他們的[言論]全是空虛的、虛妄的、無價值的話,愚者與賢智者以身體的崩解被斷滅、消失;死後不存在。』呢?」

  「大德!這確實不是。」

  「比丘們!當\twnr{聖弟子}{24.0}\twnr{在這些地方}{997.0}捨斷懷疑,捨斷苦的懷疑……(中略)捨斷導向苦\twnr{滅道跡}{69.0}的懷疑,比丘們!這位聖弟子被稱為\twnr{入流者}{165.0}、不墮惡趣法者、\twnr{決定者}{159.0}、\twnr{正覺為彼岸者}{160.0}。」



\sutta{6}{6}{作者經}{https://agama.buddhason.org/SN/sn.php?keyword=24.6}
  起源於舍衛城。

  「\twnr{比丘}{31.0}們!在有什麼時,執取什麼後,執著什麼後,這樣的見生起:『作者、\twnr{使他作者}{791.0},切斷者、\twnr{使他切斷者}{792.0},折磨者、\twnr{使他折磨者}{793.0},造成悲傷者、使他造成悲傷者,造成疲勞者、使他造成疲勞者,造成悸動者、使他造成悸動者,殺生者,\twnr{未給予而取}{104.0}者、\twnr{入侵人家者}{794.0}、奪取(搬運)掠奪物者、作盜匪者、\twnr{攔路搶劫}{988.0}者,通姦(走入)他人的妻子者,虛妄地說者:無惡被作,如果以剃刀輪周邊使在這大地上的生類轉成一肉聚、一肉堆,\twnr{從那個因由沒有惡的}{x464},\twnr{沒有惡的傳來}{x465};如果走在恒河南岸,殺者、使他殺者,切斷者、使他切斷者,折磨者、使他折磨者,從那個因由沒有惡的,沒有惡的傳來;如果走在恒河北岸,施與者、使他施與者,祭祀者、使他祭祀者,從那個因由沒有福德,沒有福德的傳來;以布施,以調御,\twnr{以抑制}{x466},以真實所言的,從那個因由沒有福德,沒有福德的傳來。[\ccchref{DN.2}{https://agama.buddhason.org/DN/dm.php?keyword=2}, 166段]』呢?」

  「\twnr{大德}{45.0}!我們的法以\twnr{世尊}{12.0}為根本……(中略)。」

  「比丘們!在有色時,執取色後,執著色後,這樣的見生起:『作者、使他作者……(中略)沒有福德,沒有福德的傳來。』

  在有受時……(中略)在有想時……在有諸行時……在有識時,執取識後,執著識後,這樣的見生起:『作者、使他作者……(中略)沒有福德,沒有福德的傳來。』

  比丘們!你們怎麼想它:色是常的,或是無常的?」

  「無常的,大德!」……

  「……(中略)是否不執取它後,這樣的見會生起:『作者……(中略)沒有福德,沒有福德的傳來。』呢?」

  「大德!這確實不是。」

  「受……想……諸行……識是常的,或是無常的?」

  「無常的,大德!」……

  「……(中略)是否不執取它後,這樣的見會生起:『作者……(中略)沒有福德,沒有福德的傳來。』呢?」

  「大德!這確實不是。」

  「凡這個所見、所聞、所覺、所識、所得、所遍求、\twnr{被意所隨行}{396.0}都是常的,或是無常的?」

  「無常的,大德!」……

  「……(中略)是否不執取它後,這樣的見會生起:『作者……(中略)沒有福德,沒有福德的傳來。』呢?」

  「大德!這確實不是。」

  「比丘們!當\twnr{聖弟子}{24.0}\twnr{在這些地方}{997.0}捨斷懷疑,捨斷苦的懷疑……(中略)捨斷導向苦\twnr{滅道跡}{69.0}的懷疑,比丘們!這位聖弟子被稱為\twnr{入流者}{165.0}、不墮惡趣法者、\twnr{決定者}{159.0}、\twnr{正覺為彼岸者}{160.0}。」



\sutta{7}{7}{因經}{https://agama.buddhason.org/SN/sn.php?keyword=24.7}
  起源於舍衛城。

  「\twnr{比丘}{31.0}們!在有什麼時,執取什麼後,執著什麼後,這樣的見生起:『對眾生的污染,沒有因沒有\twnr{緣}{180.0},眾生們無因無緣地被污染;對眾生的清淨,沒有因沒有緣,眾生們無因無緣地變成清淨;沒有力,沒有活力,沒有人的力量,沒有人的努力;一切眾生、一切生物類、一切生存類、\twnr{一切生命}{164.0}無自在力,無力,無活力,被命運、意外、\twnr{本性變化}{960.0}而\twnr{在六等級中}{899.0}感受苦樂。[\ccchref{DN.2}{https://agama.buddhason.org/DN/dm.php?keyword=2}, 168段]』呢?」

  「\twnr{大德}{45.0}!我們的法以\twnr{世尊}{12.0}為根本……(中略)。」

  「比丘們!在有色時,執取色後,執著色後,這樣的見生起:『沒有因沒有緣……(中略)感受苦樂。』

  在有受時……(中略)在有想時……在有諸行時……在有識時,執取識後,執著識後,這樣的見生起:『沒有因沒有緣……(中略)感受苦樂。』

  比丘們!你們怎麼想它:色是常的,或是無常的?」

  「無常的,大德!」……

  「……(中略)\twnr{變易法}{70.0},是否不執取它後,這樣的見會生起:『沒有因沒有緣……(中略)感受苦樂。』呢?」

  「大德!這確實不是。」

  「受……想……諸行……識是常的,或是無常的?」

  「無常的,大德!」……

  「……(中略)是否不執取它後,這樣的見會生起:『沒有因沒有緣……(中略)感受苦樂。』呢?」

  「大德!這確實不是。」

  「凡這個所見、所聞、所覺、所識、所得、所遍求、\twnr{被意所隨行}{396.0}都是常的,或是無常的?」

  「無常的,大德!」……

  「……(中略)是否不執取它後,這樣的見會生起:『沒有因沒有緣……(中略)感受苦樂。』呢?」

  「大德!這確實不是。」

  「比丘們!當\twnr{聖弟子}{24.0}\twnr{在這些地方}{997.0}捨斷懷疑,捨斷苦的懷疑……(中略)捨斷導向苦\twnr{滅道跡}{69.0}的懷疑,比丘們!這位聖弟子被稱為\twnr{入流者}{165.0}、不墮惡趣法者、\twnr{決定者}{159.0}、\twnr{正覺為彼岸者}{160.0}。」



\sutta{8}{8}{大見經}{https://agama.buddhason.org/SN/sn.php?keyword=24.8}
  起源於舍衛城。

  「\twnr{比丘}{31.0}們!在有什麼時,執取什麼後,執著什麼後,這樣的見生起:『有這七身,是非被作的、非被作之種類的、非被創造的、無創造者、\twnr{不孕的}{334.0}、如山頂站立的、如石柱狀態住立的,它們不搖動、不變易、不互相加害、不足以互相[起]或樂或苦或苦樂,哪七個?地身、水身、火身、風身、樂、苦,命為第七,這七身是非被作的、非被作之種類的、非被創造的、無創造者、不孕的、如山頂站立的、如石柱狀態住立的,它們不搖動、不變易、不互相加害、不足以互相[起]或樂或苦或苦樂,凡即使以銳利的刀切斷頭,他也沒奪取任何生命,刀但就經七身的中間隨進入[D.2, 174段]。\twnr{又有這一百四十萬六千六百最上首之胎}{798.0}、五百種業、五種業、三種業、業、半業,有六十二道、六十二中間劫、\twnr{六等級}{899.0}、\twnr{人之八地}{800.0}、四千九百種邪命外道、四千九百種\twnr{遊行者}{79.0}、四千九百種龍之住所、二千根、三千地獄、三十六塵界、七有想胎、七無想胎、\twnr{七無結胎}{900.0}、七種天、七種人、七種惡鬼、七座湖、七種結節、七座斷崖又七百座斷崖、七種夢、七百種夢、八百四十萬大劫,凡愚者與賢智者們\twnr{流轉輪迴後}{901.0},將\twnr{作苦的終結}{54.0},在那裡,沒有:「我將以這個戒,或\twnr{禁戒}{799.0},或苦行,或梵行使未遍熟之業遍熟,或將以一再觸達使已遍熟之業作終結。」確實不是這樣,對一桶量[固定]的苦樂,對作限制的輪迴,沒有減退、增加,沒有優秀、貶抑,猶如在線球被投出時,\twnr{當被解開時它就逃走}{902.0}[\ccchref{DN.2}{https://agama.buddhason.org/DN/dm.php?keyword=2}, 169段]。同樣的,愚者與賢智者當被解開時他們逃離苦樂。』呢?」

  「\twnr{大德}{45.0}!我們的法以\twnr{世尊}{12.0}為根本……(中略)。」

  「比丘們!在有色時,執取色後,執著色後,這樣的見生起:『有這七身,是非被作的、非被作之種類的……(中略)他們逃離苦樂。』

  在有受時……(中略)在有想時……在有諸行時……在有識時,執取識後,執著識後,這樣的見生起:『有這七身,是非被作的、非被作之種類的……(中略)他們逃離苦樂。』

  比丘們!你們怎麼想它:色是常的,或是無常的?」

  「無常的,大德!」……(中略)

  「而凡為無常、苦、\twnr{變易法}{70.0},是否不執取它後,這樣的見會生起:『有這七身,是非被作的、非被作之種類的……(中略)他們逃離苦樂。』呢?」

  「大德!這確實不是。」

  「凡這個所見、所聞、所覺、所識、所得、所遍求、\twnr{被意所隨行}{396.0}都是常的,或是無常的?」

  「無常的,大德!」……

  「……(中略)是否不執取它後,這樣的見會生起:『有這七身,是非被作的、非被作之種類的……(中略)他們逃離苦樂。』呢?」

  「大德!這確實不是。」

  「比丘們!當\twnr{聖弟子}{24.0}\twnr{在這些地方}{997.0}捨斷懷疑,捨斷苦的懷疑……(中略)捨斷導向苦\twnr{滅道跡}{69.0}的懷疑,比丘們!這位聖弟子被稱為\twnr{入流者}{165.0}、不墮惡趣法者、\twnr{決定者}{159.0}、\twnr{正覺為彼岸者}{160.0}。」



\sutta{9}{9}{常恆之見經}{https://agama.buddhason.org/SN/sn.php?keyword=24.9}
  起源於舍衛城。

  「\twnr{比丘}{31.0}們!在有什麼時,執取什麼後,執著什麼後,這樣的見生起:『世界是常恆的。』呢?」

  「\twnr{大德}{45.0}!我們的法以\twnr{世尊}{12.0}為根本……(中略)。」 

  「比丘們!在有色時,執取色後,執著色後,這樣的見生起:『世界是常恆的。』

  在有受時……(中略)在有想時……在有諸行時……在有識時,執取識後,執著識後,這樣的見生起:『世界是常恆的。』

  比丘們!你們怎麼想它:色是常的,或是無常的?」

  「無常的,大德!」……

  「……(中略)\twnr{變易法}{70.0},是否不執取它後,這樣的見會生起:『世界是常恆的。』呢?」

  「大德!這確實不是。」

  「受……想……諸行……識是常的,或是無常的?」

  「無常的,大德!」……

  「……(中略)是否不執取它後,這樣的見會生起:『世界是常恆的。』呢?」

  「大德!這確實不是。」

  「凡這個所見、所聞、所覺、所識、所得、所遍求、\twnr{被意所隨行}{396.0}都是常的,或是無常的?」

  「無常的,大德!」

  「那麼,凡為無常的,那是苦的或樂的?」

  「苦的,大德!」

  「而凡為無常、苦、變易法,是否不執取它後,這樣的見會生起:『世界是常恆的。』呢?」

  「大德!這確實不是。」

  「比丘們!當\twnr{聖弟子}{24.0}\twnr{在這些地方}{997.0}捨斷懷疑,捨斷苦的懷疑……(中略)捨斷導向苦\twnr{滅道跡}{69.0}的懷疑,比丘們!這位聖弟子被稱為\twnr{入流者}{165.0}、不墮惡趣法者、\twnr{決定者}{159.0}、\twnr{正覺為彼岸者}{160.0}。」



\sutta{10}{10}{非常恆之見經}{https://agama.buddhason.org/SN/sn.php?keyword=24.10}
  起源於舍衛城。

  「\twnr{比丘}{31.0}們!在有什麼時,執取什麼後,執著什麼後,這樣的見生起:『\twnr{世界是非常恆的}{170.0}。』呢?」

  「\twnr{大德}{45.0}!我們的法以\twnr{世尊}{12.0}為根本……(中略)。」 

  「比丘們!當有色時……(中略)。」……

  「識是常的,或是無常的?」

  「無常的,大德!」……

  「……(中略)是否不執取它後,這樣的見會生起:『世界是非常恆的。』呢?」

  「大德!這確實不是。」

  「凡這個所見、所聞、所覺、所識、所得、所遍求、\twnr{被意所隨行}{396.0}都是常的,或是無常的?」

  「無常的,大德!」……

  「……(中略)是否不執取它後,這樣的見會生起:『世界是非常恆的。』呢?」

  「大德!這確實不是。」

  「比丘們!當聖弟子\twnr{在這些地方}{997.0}捨斷困惑,捨斷苦的困惑……(中略)捨斷導向苦滅道跡的困惑,比丘們!這位聖弟子被稱為\twnr{入流者}{165.0}、不墮惡趣法者、\twnr{決定者}{159.0}、\twnr{正覺為彼岸者}{160.0}。」



\sutta{11}{11}{有邊經}{https://agama.buddhason.org/SN/sn.php?keyword=24.11}
  起源於舍衛城。

  「\twnr{比丘}{31.0}們!在有什麼時,執取什麼後,執著什麼後,這樣的見生起:『世界是有邊的。』呢?」

  「\twnr{大德}{45.0}!我們的法以\twnr{世尊}{12.0}為根本……(中略)。」……

  「……\twnr{決定}{159.0}、\twnr{以正覺為彼岸}{160.0}的\twnr{入流者}{165.0}聖弟子。」



\sutta{12}{12}{無邊經}{https://agama.buddhason.org/SN/sn.php?keyword=24.12}
  起源於舍衛城。

  「\twnr{比丘}{31.0}們!在有什麼時,執取什麼後,執著什麼後,這樣的見生起:『世界是無邊的。』呢?」

  「\twnr{大德}{45.0}!我們的法以\twnr{世尊}{12.0}為根本……(中略)。」……

  「……\twnr{決定}{159.0}、\twnr{以正覺為彼岸}{160.0}的\twnr{入流者}{165.0}聖弟子。」



\sutta{13}{13}{命即是身體經}{https://agama.buddhason.org/SN/sn.php?keyword=24.13}
  起源於舍衛城。

  「\twnr{比丘}{31.0}們!在有什麼時,執取什麼後,執著什麼後,這樣的見生起:『命即是身體。』呢?」

  「\twnr{大德}{45.0}!我們的法以\twnr{世尊}{12.0}為根本……(中略)。」……

  「……\twnr{決定}{159.0}、\twnr{以正覺為彼岸}{160.0}的\twnr{入流者}{165.0}聖弟子。」



\sutta{14}{14}{命是一身體是另一經}{https://agama.buddhason.org/SN/sn.php?keyword=24.14}
  起源於舍衛城。

  「\twnr{比丘}{31.0}們!在有什麼時,執取什麼後,執著什麼後,這樣的見生起:『\twnr{命是一身體是另一}{169.0}。』呢?」

  「\twnr{大德}{45.0}!我們的法以\twnr{世尊}{12.0}為根本……(中略)。」……

  「……\twnr{決定}{159.0}、\twnr{以正覺為彼岸}{160.0}的\twnr{入流者}{165.0}聖弟子。」



\sutta{15}{15}{如來存在經}{https://agama.buddhason.org/SN/sn.php?keyword=24.15}
  起源於舍衛城。

  「\twnr{比丘}{31.0}們!在有什麼時,執取什麼後,執著什麼後,這樣的見生起:『死後如來存在。』呢?」

  「\twnr{大德}{45.0}!我們的法以\twnr{世尊}{12.0}為根本……(中略)。」……

  「……\twnr{決定}{159.0}、\twnr{以正覺為彼岸}{160.0}的\twnr{入流者}{165.0}聖弟子。」





\sutta{16}{16}{死後如來不存在經}{https://agama.buddhason.org/SN/sn.php?keyword=24.16}
  起源於舍衛城。

  「\twnr{比丘}{31.0}們!在有什麼時,執取什麼後,執著什麼後,這樣的見生起:『死後如來不存在。』呢?」

  「\twnr{大德}{45.0}!我們的法以\twnr{世尊}{12.0}為根本……(中略)。」……

  「……\twnr{決定}{159.0}、\twnr{以正覺為彼岸}{160.0}的\twnr{入流者}{165.0}聖弟子。」





\sutta{17}{17}{死後如來存在且不存在經}{https://agama.buddhason.org/SN/sn.php?keyword=24.17}
  起源於舍衛城。

  「\twnr{比丘}{31.0}們!在有什麼時,執取什麼後,執著什麼後,這樣的見生起:『\twnr{死後如來存在且不存在}{354.0}。』呢?」

  「\twnr{大德}{45.0}!我們的法以\twnr{世尊}{12.0}為根本……(中略)。」……

  「……\twnr{決定}{159.0}、\twnr{以正覺為彼岸}{160.0}的\twnr{入流者}{165.0}聖弟子。」



\sutta{18}{18}{死後如來既非存在也非不存在經}{https://agama.buddhason.org/SN/sn.php?keyword=24.18}
  起源於舍衛城。

  「\twnr{比丘}{31.0}們!在有什麼時,執取什麼後,執著什麼後,這樣的見生起:『死後如來既非存在也非不存在。』呢?」

  「\twnr{大德}{45.0}!我們的法以\twnr{世尊}{12.0}為根本……(中略)。」 

  「比丘們!在有色時,執取色後,執著色後,這樣的見生起:『死後如來既非存在也非不存在。』……(中略)

  比丘們!你們怎麼想它:色是常的,或是無常的?」

  「無常的,大德!」……

  「……(中略)\twnr{變易法}{70.0},是否不執取它後,這樣的見會生起:『死後如來既非存在也非不存在。』呢?」

  「大德!這確實不是。」

  「凡這個所見、所聞、所覺、所識、所得、所遍求、\twnr{被意所隨行}{396.0}都是常的,或是無常的?」

  「無常的,大德!」

  「那麼,凡為無常的,那是苦的或樂的?」 

  「苦的,大德!」 

  「而凡為無常、苦、變易法,是否不執取它後,這樣的見會生起:『死後如來既非存在也非不存在。』呢?」

  「大德!這確實不是。」

  「比丘們!當聖弟子\twnr{在這些地方}{997.0}捨斷困惑,捨斷苦的困惑,捨斷苦集的困惑,捨斷苦滅的困惑,捨斷導向苦滅道跡的困惑,比丘們!這位聖弟子被稱為\twnr{入流者}{165.0}、不墮惡趣法者、\twnr{決定者}{159.0}、\twnr{正覺為彼岸者}{160.0}。」

  入流品十八則解說終了,其\twnr{攝頌}{35.0}:

  「風、這是我的,彼-我、我的不會存在,

   沒有、作者與因,以大見為第八。

   世界是常恆的,以及非常恆的與有邊,

   無邊與命即是身體,以及命是一身體是另一。

   死後如來存在,死後如來不存在,

   [死後如來存在且不存在,]死後如來既非存在也非不存在。」





\pin{第二行品}{19}{44}
\sutta{19}{19}{風經}{https://agama.buddhason.org/SN/sn.php?keyword=24.19}
  起源於舍衛城。

  「\twnr{比丘}{31.0}們!在有什麼時,執取什麼後,執著什麼後,這樣的見生起:『風沒吹,河沒流,孕婦沒生,日月沒起落,如直立不動的石柱。』呢?」

  「\twnr{大德}{45.0}!我們的法以\twnr{世尊}{12.0}為根本……(中略)。」 

  「比丘們!在有色時,執取色後,執著色後,這樣的見生起:『風沒吹……(中略)如直立不動的石柱。』

  在有受時……(中略)在有想時……(中略)在有諸行時……在有識時,執取識後,執著識後,這樣的見生起:『風沒吹……(中略)如直立不動的石柱。』

  比丘們!你們怎麼想它:色是常的,或是無常的?」

  「無常的,大德!」……(中略)

  「……\twnr{變易法}{70.0},是否不執取它後,這樣的見會生起:『風沒吹……(中略)如直立不動的石柱。』呢?」

  「大德!這確實不是。」

  「比丘們!像這樣,在有苦時,執取苦後,執著苦後,這樣的見生起:『風沒吹……(中略)如直立不動的石柱。』」

  「受……想……諸行……識是常的,或是無常的?」

  「無常的,大德!」……(中略)

  「……變易法,是否不執取它後,這樣的見會生起:『風沒吹……(中略)如直立不動的石柱。』呢?」

  「大德!這確實不是。」

  「比丘們!像這樣,在有苦時,執取苦後,執著苦後,這樣的見生起:『風沒吹,河沒流,孕婦沒生,日月沒起落,如直立不動的石柱。』」



\sutta{20}{36}{既非存在也非不存在經}{https://agama.buddhason.org/SN/sn.php?keyword=24.20}
  ([十七經]應該如前品十八經解說使之被細說)

  起源於舍衛城。

  「\twnr{比丘}{31.0}們!在有什麼時,執取什麼後,執著什麼後,這樣的見生起:『死後如來既非存在也非不存在。』呢?」

  「\twnr{大德}{45.0}!我們的法以\twnr{世尊}{12.0}為根本……(中略)。」 

  「比丘們!在有色時,執取色後,執著色後,這樣的見生起:『死後如來既非存在也非不存在。』

  在有受時……(中略)在有想時……(中略)在有諸行時……在有識時,執取識後,執著識後,這樣的見生起:『死後如來既非存在也非不存在。』

  比丘們!你們怎麼想它:色是常的,或是無常的?」

  「無常的,大德!」……(中略)

  「……\twnr{變易法}{70.0},是否不執取它後,這樣的見會生起:『死後如來既非存在也非不存在。』呢?」

  「大德!這確實不是。」

  「比丘們!像這樣,在有苦時,執取苦後,執著苦後,這樣的見生起:『死後如來既非存在也非不存在。』」

  「受……想……諸行……識是常的,或是無常的?」

  「無常的,大德!」……(中略)

  「……變易法,是否不執取它後,這樣的見會生起:『死後如來既非存在也非不存在。』呢?」

  「大德!這確實不是。」

  「比丘們!像這樣,在有苦時,執取苦後,執著苦後,這樣的見生起:『死後如來既非存在也非不存在。』」



\sutta{37}{37}{有色的-我經}{https://agama.buddhason.org/SN/sn.php?keyword=24.37}
  起源於舍衛城。

  「\twnr{比丘}{31.0}們!在有什麼時,執取什麼後,執著什麼後,這樣的見生起:『\twnr{我是有色的}{x467},\twnr{死後無病}{863.0}。』呢?」

  「\twnr{大德}{45.0}!我們的法以\twnr{世尊}{12.0}為根本……(中略)。」 

  「比丘們!在有色時,執取色後,執著色後,這樣的見生起:『我是有色的,死後無病。』

  在有受時……(中略)在有想時……在有諸行時……在有識時,執取識後,執著識後,這樣的見生起:『我是有色的,死後無病。』

  比丘們!你們怎麼想它:色是常的,或是無常的?」

  「無常的,大德!」……

  「……(中略)\twnr{變易法}{70.0},是否不執取它後,這樣的見會生起:『我是有色的,死後無病。』呢?」

  「大德!這確實不是。」

  「比丘們!像這樣,在有苦時,執取苦後,執著苦後,這樣的見生起:『我是有色的,死後無病。』」

  「受……(中略)。」……

  「大德!這確實不是。」

  「比丘們!像這樣,在有苦時,執取苦後,執著苦後,這樣的見生起:『我是有色的,死後無病。』」



\sutta{38}{38}{無色的-我經}{https://agama.buddhason.org/SN/sn.php?keyword=24.38}
  起源於舍衛城。

  「\twnr{比丘}{31.0}們!在有什麼時,執取什麼後,執著什麼後,這樣的見生起:『我是無色的,\twnr{死後無病}{863.0}。』呢?」……(中略)



\sutta{39}{39}{有色的與-無色的-我經}{https://agama.buddhason.org/SN/sn.php?keyword=24.39}
  起源於舍衛城。

  「……『我是有色的與無色的,\twnr{死後無病}{863.0}。』呢?」……(中略)



\sutta{40}{40}{非有色的非無色的-我經}{https://agama.buddhason.org/SN/sn.php?keyword=24.40}
  「……『我是非有色的非無色的,\twnr{死後無病}{863.0}。』呢?」……(中略)



\sutta{41}{41}{一向樂的經}{https://agama.buddhason.org/SN/sn.php?keyword=24.41}
  「……『我是\twnr{一向}{168.0}樂的,\twnr{死後無病}{863.0}。』呢?」……(中略)



\sutta{42}{42}{一向苦的經}{https://agama.buddhason.org/SN/sn.php?keyword=24.42}
  「……『我是\twnr{一向}{168.0}苦的,\twnr{死後無病}{863.0}。』呢?」……(中略)



\sutta{43}{43}{苦樂的經}{https://agama.buddhason.org/SN/sn.php?keyword=24.43}
  「……『我是苦樂的,\twnr{死後無病}{863.0}。』呢?」……(中略)



\sutta{44}{44}{不苦不樂經}{https://agama.buddhason.org/SN/sn.php?keyword=24.44}
  「……『我是不苦不樂的,\twnr{死後無病}{863.0}。』呢?」

  「\twnr{大德}{45.0}!我們的法以\twnr{世尊}{12.0}為根本……(中略)。」 

  「\twnr{比丘}{31.0}們!在有色時,執取色後,執著色後,這樣的見生起:『我是不苦不樂的,死後無病。』

  在有受時……(中略)在有想時……在有諸行時……在有識時,執取識後,執著識後,這樣的見生起:『我是不苦不樂的,死後無病。』

  比丘們!你們怎麼想它:色是常的,或是無常的?」

  「無常的,大德!」……

  「……(中略)\twnr{變易法}{70.0},是否不執取它後,這樣的見會生起:『我是不苦不樂的,死後無病。』呢?」

  「大德!這確實不是。」

  「比丘們!像這樣,在有苦時,執取苦後,執著苦後,這樣的見生起:『我是不苦不樂的,死後無病。』」

  「受……想……諸行……識是常的,或是無常的?」

  「無常的,大德!」……

  「……(中略)變易法,是否不執取它後,這樣的見會生起:『我是不苦不樂的,死後無病。』呢?」

  「大德!這確實不是。」

  「比丘們!像這樣,在有苦時,執取苦後,執著苦後,這樣的見生起:『我是不苦不樂的,死後無病。』」

  中略第二[第二行品],其\twnr{攝頌}{35.0}:

  「風、這是我的,彼-我、我的不會存在,

   沒有、作者與因,以大見為第八。

   恆連同非恆,以及被稱為有邊或無邊,

   命即與命異,以如來四則,

   我是有色與我是無色,我是有色與無色,

   我是非有色非無色,我是一向樂的,

   我是一向苦的,我是苦樂的,

   我是不苦不樂的,死後無病,

   這二十六經,被第二章教導。」





\pin{第三行品}{45}{70}
\sutta{45}{45}{無風經}{https://agama.buddhason.org/SN/sn.php?keyword=24.45}
  起源於舍衛城。

  「\twnr{比丘}{31.0}們!在有什麼時,執取什麼後,執著什麼後,這樣的見生起:『風沒吹,河沒流,孕婦沒生,日月沒起落,如直立不動的石柱。』呢?」

  「\twnr{大德}{45.0}!我們的法以\twnr{世尊}{12.0}為根本……(中略)。」 

  「比丘們!在有色時,執取色後,執著色後,這樣的見生起:『風沒吹……(中略)』在有受時……(中略)在有想時……(中略)在有諸行時……在有識時,執取識後,執著識後,這樣的見生起:『風沒吹……(中略)如直立不動的石柱。』

  比丘們!你們怎麼想它:色是常的,或是無常的?」

  「無常的,大德!」……(中略)

  「……\twnr{變易法}{70.0},是否不執取它後,這樣的見會生起:『風沒吹……(中略)如直立不動的石柱。』呢?」

  「大德!這確實不是。」

  「比丘們!像這樣,凡為無常的它是苦的,在有那個時,執取那個後,這樣的見生起:『風沒吹,河沒流,孕婦沒生,日月沒起落,如直立不動的石柱。』」

  「受……想……諸行……識是常的,或是無常的?」

  「無常的,大德!」……(中略)

  「……變易法,是否不執取它後,這樣的見會生起:『風沒吹……(中略)如直立不動的石柱。』呢?」

  「大德!這確實不是。」

  「比丘們!像這樣,凡為無常的它是苦的,在有那個時,執取那個後,這樣的見生起:『風沒吹……(中略)如直立不動的石柱。』」



\sutta{46}{70}{不苦不樂經}{https://agama.buddhason.org/SN/sn.php?keyword=24.46}
  (應該如第二品二十四經解說使之完成)

  起源於舍衛城。

  「\twnr{比丘}{31.0}們!在有什麼時,執取什麼後,執著什麼後,這樣的見生起:『我是不苦不樂的,\twnr{死後無病}{863.0}。』呢?」

  「\twnr{大德}{45.0}!我們的法以\twnr{世尊}{12.0}為根本……(中略)。」 

  「比丘們!在有色時,執取色後,執著色後,這樣的見生起:『我是不苦不樂的,死後無病。』

  在有受時……(中略)在有想時……(中略)在有諸行時……在有識時,執取識後,執著識後,這樣的見生起:『我是不苦不樂的,死後無病。』

  比丘們!你們怎麼想它:色是常的,或是無常的?」

  「無常的,大德!」……(中略)

  「……\twnr{變易法}{70.0},是否不執取它後,這樣的見會生起:『我是不苦不樂的,死後無病。』呢?」

  「大德!這確實不是。」

  「比丘們!像這樣,凡為無常的它是苦的,在有那個時,執取那個後,這樣的見生起:『我是不苦不樂的,死後無病。』」

  「受……想……諸行……識是常的,或是無常的?」

  「無常的,大德!」……(中略)

  「……變易法,是否不執取它後,這樣的見會生起:『我是不苦不樂的,死後無病。』呢?」

  「大德!這確實不是。」

  「比丘們!像這樣,凡為無常的它是苦的,在有那個時,執取那個後,這樣的見生起:『我是不苦不樂的,死後無病。』」

  第三中略。





\pin{第四行品}{71}{96}
\sutta{71}{71}{無風經}{https://agama.buddhason.org/SN/sn.php?keyword=24.71}
  起源於舍衛城。

  「\twnr{比丘}{31.0}們!在有什麼時,執取什麼後,執著什麼後,這樣的見生起:『風沒吹,河沒流,孕婦沒生,日月沒起落,如直立不動的石柱。』呢?」

  「\twnr{大德}{45.0}!我們的法以\twnr{世尊}{12.0}為根本……(中略)。」 

  「比丘們!在有色時,執取色後,執著色後,這樣的見生起:『風沒吹……(中略)如直立不動的石柱。』

  在有受時……(中略)在有想時……(中略)在有諸行時……在有識時,執取識後,執著識後,這樣的見生起:『風沒吹……(中略)如直立不動的石柱。』

  比丘們!你們怎麼想它:色是常的,或是無常的?」

  「無常的,大德!」

  「那麼,凡為無常的,那是苦的或樂的?」

  「苦的,大德!」

  「而凡為無常、苦、\twnr{變易法}{70.0},適合認為它:『\twnr{這是我的}{32.0},\twnr{我是這個}{33.0},這是\twnr{我的真我}{34.0}。』嗎?」

  「大德!這確實不是。」

  「受……想……諸行……識是常的,或是無常的?」

  「無常的,大德!」

  「那麼,凡為無常的,那是苦的或樂的?」

  「苦的,大德!」

  「而凡為無常、苦、變易法,你們認為:『這是我的,我是這個,這是我的真我。』嗎?」

  「大德!這確實不是。」

  「比丘們!因此,在這裡,凡任何色:過去、未來、現在,或內、或外,或粗、或細,或下劣、或勝妙,或凡在遠處、在近處,所有色:『\twnr{這不是我的}{32.1},\twnr{我不是這個}{33.1},\twnr{這不是我的真我}{34.2}。』這樣,這個應該以正確之慧如實被看見。

  凡任何受……凡任何想……凡任何諸行……凡任何識:過去、未來、現在,或內、或外,或粗、或細,或下劣、或勝妙,或凡在遠處、在近處,所有識:『這不是我的,我不是這個,這不是我的真我。』這樣,這個應該以正確之慧如實被看見。

  [比丘們!]這麼看的……(中略)他知道:『……\twnr{不再有此處[輪迴]的狀態}{21.1}。』」



\sutta{72}{96}{不苦不樂經}{https://agama.buddhason.org/SN/sn.php?keyword=24.72}
  (應該如第二品二十四經解說使之完成)

  起源於舍衛城。

  「\twnr{比丘}{31.0}們!在有什麼時,執取什麼後,執著什麼後,這樣的見生起:『我是不苦不樂的,\twnr{死後無病}{863.0}。』呢?」

  「\twnr{大德}{45.0}!我們的法以\twnr{世尊}{12.0}為根本……(中略)。」 

  「比丘們!在有色時,執取色後,執著色後,這樣的見生起:『我是不苦不樂的,死後無病。』

  在有受時……(中略)在有想時……(中略)在有諸行時……在有識時,執取識後,執著識後,這樣的見生起:『我是不苦不樂的,死後無病。』

  比丘們!你們怎麼想它:色是常的,或是無常的?」

  「無常的,大德!」

  「那麼,凡為無常的,那是苦的或樂的?」

  「苦的,大德!」

  「而凡為無常、苦、\twnr{變易法}{70.0},適合認為它:『\twnr{這是我的}{32.0},\twnr{我是這個}{33.0},這是\twnr{我的真我}{34.0}。』嗎?」

  「大德!這確實不是。」

  「受……想……諸行……識是常的,或是無常的?」

  「無常的,大德!」

  「那麼,凡為無常的,那是苦的或樂的?」

  「苦的,大德!」

  「而凡為無常、苦、變易法,你們認為:『這是我的,我是這個,這是我的真我。』嗎?」

  「大德!這確實不是。」

  「比丘們!因此,在這裡,凡任何色:過去、未來、現在,或內、或外,或粗、或細,或下劣、或勝妙,或凡在遠處、在近處,所有色:『\twnr{這不是我的}{32.1},\twnr{我不是這個}{33.1},\twnr{這不是我的真我}{34.2}。』這樣,這個應該以正確之慧如實被看見。

  凡任何受……凡任何想……凡任何諸行……凡任何識:過去、未來、現在,或內、或外,或粗、或細,或下劣、或勝妙,或凡在遠處、在近處,所有識:『這不是我的,我不是這個,這不是我的真我。』這樣,這個應該以正確之慧如實被看見。

  比丘們!這麼看的有聽聞的聖弟子在色上\twnr{厭}{15.0},也在受上厭,也在想上厭,也在諸行上厭,也在識上厭。厭者\twnr{離染}{558.0},從\twnr{離貪}{77.0}被解脫,在已解脫時,\twnr{有『[這是]解脫』之智}{27.0},他知道:『\twnr{出生已盡}{18.0},\twnr{梵行已完成}{19.0},\twnr{應該被作的已作}{20.0},\twnr{不再有此處[輪迴]的狀態}{21.1}。』」

  [見相應,]其\twnr{攝頌}{35.0}:

  「在第一行[品]中有十八則解說,

   在第二行[品]中有二十六則解說,

   在第三行[品]中有二十六則解說,

   在第四行[品]中有二十六則解說。」

  見相應完成。





\page

\xiangying{25}{入相應}
\sutta{1}{1}{眼經}{https://agama.buddhason.org/SN/sn.php?keyword=25.1}
  起源於舍衛城。

  「\twnr{比丘}{31.0}們!眼是無常的、變易的、變異的;耳是無常的、變易的、變異的;鼻是無常的、變易的、變異的;舌是無常的、變易的、變異的;身是無常的、變易的、變異的;意是無常的、變易的、變異的。比丘們!凡\twnr{這樣信、勝解這些法}{x468},這位被稱為已進入\twnr{正性決定}{588.0}、已進入善士地、\twnr{超越凡夫地}{x469}的\twnr{隨信行}{166.0}者,他不可能作那個業:作該業後會往生地獄,或畜生界,或\twnr{餓鬼}{362.0}界,且不可能到死時還不證\twnr{入流果}{165.1}。

  比丘們!凡\twnr{這些法以慧這樣足夠沉思地接受}{x470},這位被稱為已進入正性決定、已進入善士地、超越凡夫地的\twnr{隨法行者}{167.0},他不可能作那個業:作該業後會往生地獄,或畜生界,或餓鬼界,且不可能到死時還不證入流果。比丘們!凡這麼知、這麼見這些法,這位被稱為入流者、不墮惡趣法者、\twnr{決定者}{159.0}、\twnr{正覺為彼岸者}{160.0}。」



\sutta{2}{2}{色經}{https://agama.buddhason.org/SN/sn.php?keyword=25.2}
  起源於舍衛城。

  「\twnr{比丘}{31.0}們!色是無常的、變易的、變異的;聲是無常的、變易的、變異的;氣味是無常的、變易的、變異的;諸味道是無常的、變易的、變異的;\twnr{所觸}{220.2}是無常的、變易的、變異的;諸法是無常的、變易的、變異的。比丘們!凡這麼信、勝解這些法,這位被稱為已進入\twnr{正性決定}{588.0}、已進入善士地、超越凡夫地的\twnr{隨信行者}{166.0},他不可能作那個業:作該業後會往生地獄,或畜生界,或餓鬼界,且不可能到死時還不證\twnr{入流果}{165.1}。

  比丘們!凡這些法以慧這樣足夠沉思地接受,這位被稱為已進入正性決定、已進入善士地、超越凡夫地的\twnr{隨法行者}{167.0},他不可能作那個業:作該業後會往生地獄,或畜生界,或餓鬼界,且不可能到死時還不證入流果。比丘們!凡這麼知、這麼見這些法,這位被稱為不墮惡趣法、決定、\twnr{以正覺為彼岸}{160.0}的入流者。」



\sutta{3}{3}{識經}{https://agama.buddhason.org/SN/sn.php?keyword=25.3}
  起源於舍衛城。

  「\twnr{比丘}{31.0}們!眼識是無常的、變易的、變異的;耳識……鼻識……舌識……身識……意識是無常的、變易的、變異的。比丘們!凡……(中略)\twnr{以正覺為彼岸}{160.0}的[\twnr{入流者}{165.0}]。」



\sutta{4}{4}{觸經}{https://agama.buddhason.org/SN/sn.php?keyword=25.4}
  起源於舍衛城。

  「\twnr{比丘}{31.0}們!眼觸是無常的、變易的、變異的;耳觸……鼻觸……舌觸……身觸……意觸是無常的、變易的、變異的。比丘們!凡這麼信、勝解這些法,這位被稱為……\twnr{隨信行者}{166.0}……(中略)\twnr{以正覺為彼岸}{160.0}的[\twnr{入流者}{165.0}]。」



\sutta{5}{5}{觸所生經}{https://agama.buddhason.org/SN/sn.php?keyword=25.5}
  起源於舍衛城。

  「\twnr{比丘}{31.0}們!眼觸所生受是無常的、變易的、變異的;耳觸所生受……(中略)鼻觸所生受……(中略)舌觸所生受……(中略)身觸所生受……(中略)意觸所生受是無常的、變易的、變異的。比丘們!凡這麼信、勝解這些法,這位被稱為……\twnr{隨信行者}{166.0}……(中略)\twnr{以正覺為彼岸}{160.0}的[\twnr{入流者}{165.0}]。」



\sutta{6}{6}{色之想經}{https://agama.buddhason.org/SN/sn.php?keyword=25.6}
  起源於舍衛城。

  「\twnr{比丘}{31.0}們!色之想是無常的、變易的、變異的;聲之想……氣味之想……味道之想……\twnr{所觸}{220.2}之想……法之想是無常的、變易的、變異的。比丘們!凡這麼信、勝解這些法,這位被稱為……\twnr{隨信行者}{166.0}……(中略)\twnr{以正覺為彼岸}{160.0}的[\twnr{入流者}{165.0}]。」



\sutta{7}{7}{色之思經}{https://agama.buddhason.org/SN/sn.php?keyword=25.7}
  起源於舍衛城。

  「\twnr{比丘}{31.0}們!\twnr{色之思}{872.0}是無常的、變易的、變異的;聲之思……氣味之思……味道之思……\twnr{所觸}{220.2}之思……法之思是無常的、變易的、變異的。比丘們!凡這麼信、勝解這些法,這位被稱為……\twnr{隨信行者}{166.0}……(中略)\twnr{以正覺為彼岸}{160.0}的[\twnr{入流者}{165.0}]。」



\sutta{8}{8}{色之渴愛經}{https://agama.buddhason.org/SN/sn.php?keyword=25.8}
  起源於舍衛城。

  「\twnr{比丘}{31.0}們!色之渴愛是無常的、變易的、變異的;聲之渴愛……氣味之渴愛……味道之渴愛……\twnr{所觸}{220.2}之渴愛……法之渴愛是無常的、變易的、變異的。比丘們!凡這麼信、勝解這些法,這位被稱為……\twnr{隨信行者}{166.0}……(中略)\twnr{以正覺為彼岸}{160.0}的[\twnr{入流者}{165.0}]。」



\sutta{9}{9}{地界經}{https://agama.buddhason.org/SN/sn.php?keyword=25.9}
  起源於舍衛城。

  「\twnr{比丘}{31.0}們!地界是無常的、變易的、變異的;水界……火界……風界……虛空界……識界是無常的、變易的、變異的。比丘們!凡這麼信、勝解這些法,這位被稱為……\twnr{隨信行者}{166.0}……(中略)\twnr{以正覺為彼岸}{160.0}的[\twnr{入流者}{165.0}]。」



\sutta{10}{10}{蘊經}{https://agama.buddhason.org/SN/sn.php?keyword=25.10}
  起源於舍衛城。

  「\twnr{比丘}{31.0}們!色是無常的、變易的、變異的;受是無常的、變易的、變異的;想……諸行是無常的、變易的、變異的;識是無常的、變易的、變異的。比丘們!凡這樣信、勝解這些法,這位被稱為已進入\twnr{正性決定}{588.0}、已進入善士地、超越凡夫地的\twnr{隨信行}{166.0}者,他不可能作那個業:作該業後會往生地獄,或畜生界,或\twnr{餓鬼}{362.0}界,且不可能到死時還不證\twnr{入流果}{165.1}。

  比丘們!凡這些法以慧這樣足夠沉思地接受,這位被稱為已進入正性決定、已進入善士地、超越凡夫地的\twnr{隨法行者}{167.0},他不可能作那個業:作該業後會往生地獄,或畜生界,或餓鬼界,且不可能到死時還不證入流果。比丘們!凡這麼知、這麼見這些法者,這位被稱為入流、不墮惡趣法者、\twnr{決定者}{159.0}、\twnr{正覺為彼岸者}{160.0}。」

  入相應完成,其\twnr{攝頌}{35.0}:

  「眼、色與識,觸與受,

   想與思、渴愛,界與蘊它們為十。」





\page

\xiangying{26}{生相應}
\sutta{1}{1}{眼經}{https://agama.buddhason.org/SN/sn.php?keyword=26.1}
  起源於舍衛城。

  「\twnr{比丘}{31.0}們!凡眼的生起、存續、生出、顯現,這是苦的生起、諸病的存續、老死的顯現。

  凡耳的生起、存續……(中略)凡鼻的生起、存續……(中略)凡舌的生起、存續……(中略)凡身的生起、存續……(中略)凡意的生起、存續、生出、顯現,這是苦的生起、諸病的存續,老死的顯現。

  比丘們!而凡眼的\twnr{滅}{68.0}、平息、滅沒,這是苦的滅、諸病的平息、老死的滅沒。

  凡耳的滅……(中略)凡鼻的滅……(中略)凡舌的滅……(中略)凡身的滅……(中略)凡意的滅、平息、滅沒,這是苦的滅、諸病的平息、老死的滅沒。」[\suttaref{SN.35.21}, ≃\suttaref{SN.22.30}]



\sutta{2}{2}{色經}{https://agama.buddhason.org/SN/sn.php?keyword=26.2}
  起源於舍衛城。

  「\twnr{比丘}{31.0}們!凡諸色的生起、存續、生出、顯現,這是苦的生起、諸病的存續、老死的顯現。

  凡諸聲音的……凡諸氣味的……凡諸味道的……凡諸\twnr{所觸}{220.2}的……凡諸法的生起、存續、生出、顯現,這是苦的生起、諸病的存續,老死的顯現。

  比丘們!而凡諸色的\twnr{滅}{68.0}、平息、滅沒,這是苦的滅、諸病的平息、老死的滅沒。

  凡諸聲音的……凡諸氣味的……凡諸味道的……凡諸所觸的……凡諸法的滅、平息、滅沒,這是苦的滅、諸病的平息、老死的滅沒。」[\suttaref{SN.35.22}, ≃\suttaref{SN.22.30}]



\sutta{3}{3}{識經}{https://agama.buddhason.org/SN/sn.php?keyword=26.3}
  起源於舍衛城。

  「\twnr{比丘}{31.0}們!凡眼識的生起、存續……(中略)老死的顯現。……(中略)凡意識的生起、存續……(中略)老死的顯現。

  比丘們!而凡眼識的滅……(中略)老死的滅沒。……(中略)凡意識的滅……(中略)老死的滅沒。」



\sutta{4}{4}{觸經}{https://agama.buddhason.org/SN/sn.php?keyword=26.4}
  起源於舍衛城。

  「\twnr{比丘}{31.0}們!凡眼觸的生起、存續……(中略)老死的顯現。……(中略)凡意觸的生起、存續……(中略)老死的顯現。

  比丘們!而凡眼觸的滅……(中略)老死的滅沒。……(中略)凡意觸的滅……(中略)老死的滅沒。」



\sutta{5}{5}{觸所生經}{https://agama.buddhason.org/SN/sn.php?keyword=26.5}
  起源於舍衛城。

  「\twnr{比丘}{31.0}們!凡眼觸所生受的生起、存續……(中略)老死的顯現。……(中略)凡意觸所生受的生起、存續……(中略)老死的顯現。

  比丘們!而凡眼觸所生受的滅、平息……(中略)老死的滅沒。……(中略)凡意觸所生受的\twnr{滅}{68.0}、平息、滅沒,這是苦的滅、諸病的平息、老死的滅沒。」



\sutta{6}{6}{想經}{https://agama.buddhason.org/SN/sn.php?keyword=26.6}
  起源於舍衛城。

  「\twnr{比丘}{31.0}們!凡色之想的生起、存續……(中略)老死的顯現。……(中略)凡法之想的生起、存續、生出、顯現,這是苦的生起、諸病的存續、老死的顯現。

  比丘們!而凡色之想的滅……(中略)老死的滅沒。……(中略)凡法之想的\twnr{滅}{68.0}、平息、滅沒,這是苦的滅、諸病的平息、老死的滅沒。」



\sutta{7}{7}{思經}{https://agama.buddhason.org/SN/sn.php?keyword=26.7}
  起源於舍衛城。

  「\twnr{比丘}{31.0}們!凡\twnr{色之思}{872.0}的生起、存續……(中略)老死的顯現。……(中略)凡法之思的生起、存續、生出、顯現,這是苦的生起、諸病的存續、老死的顯現。

  比丘們!而凡色之思的滅……(中略)老死的滅沒。……(中略)凡法之思的\twnr{滅}{68.0}、平息、滅沒,這是苦的滅、諸病的平息、老死的滅沒。」



\sutta{8}{8}{渴愛經}{https://agama.buddhason.org/SN/sn.php?keyword=26.8}
  起源於舍衛城。

  「\twnr{比丘}{31.0}們!凡色之渴愛的生起、存續……(中略)老死的顯現。……(中略)凡法之渴愛的生起、存續、生出、顯現,這是苦的生起、諸病的存續、老死的顯現。

  比丘們!而凡色之渴愛的滅……(中略)老死的滅沒。……(中略)凡法之渴愛的\twnr{滅}{68.0}、平息、滅沒,這是苦的滅、諸病的平息、老死的滅沒。」



\sutta{9}{9}{界經}{https://agama.buddhason.org/SN/sn.php?keyword=26.9}
  起源於舍衛城。

  「\twnr{比丘}{31.0}們!凡地界的生起、存續、生出、顯現……(中略)老死的顯現。凡水界的……(中略)凡火界的……(中略)凡風界的……(中略)凡虛空界的……(中略)凡識界的生起、存續、再生、顯現,這是苦的生起、諸病的存續,老死的顯現。

  比丘們!而凡地界的滅……(中略)老死的滅沒。凡水界的滅……(中略)凡火界的滅……(中略)凡風界的滅……(中略)凡虛空界的滅……(中略)凡識界的\twnr{滅}{68.0}、平息、滅沒,這是苦的滅、諸病的平息、老死的滅沒。」



\sutta{10}{10}{蘊經}{https://agama.buddhason.org/SN/sn.php?keyword=26.10}
  起源於舍衛城。

  「\twnr{比丘}{31.0}們!凡色的生起、存續、生出、顯現,這是苦的生起、諸病的存續、老死的顯現。

  凡受的……(中略)凡想的……(中略)凡諸行的……(中略)凡識的生起、存續、再生、顯現,這是苦的生起、諸病的存續,老死的顯現。

  比丘們!而凡色的\twnr{滅}{68.0}、平息、滅沒,這是苦的滅、諸病的平息、老死的滅沒。

  凡受的……(中略)凡想的……(中略)凡諸行的……(中略)凡識的滅、平息、滅沒,這是苦的滅、諸病的平息、老死的滅沒。」[≃\suttaref{SN.22.30}]

  生相應完成,其\twnr{攝頌}{35.0}:

  「眼、色與識,觸與受,

   想與思、渴愛,界與蘊它們為十。」





\page

\xiangying{27}{污染相應}
\sutta{1}{1}{眼經}{https://agama.buddhason.org/SN/sn.php?keyword=27.1}
  起源於舍衛城。

  「\twnr{比丘}{31.0}們!凡在眼上的意欲貪,這是心的\twnr{隨雜染}{288.0};凡在耳上的意欲貪,這是心的隨雜染;凡在鼻上的意欲貪,這是心的隨雜染;凡在舌上的意欲貪,這是心的隨雜染;凡在身上的意欲貪,這是心的隨雜染;凡在意上的意欲貪,這是心的隨雜染。

  比丘們!當在這六處上比丘心的隨雜染被捨斷,他的心是傾向離欲的;已遍\twnr{修習}{94.0}離欲的心可以認為在應該被證智作證的諸法上,是\twnr{適合作業的}{412.0}。」



\sutta{2}{2}{色經}{https://agama.buddhason.org/SN/sn.php?keyword=27.2}
  起源於舍衛城。

  「\twnr{比丘}{31.0}們!凡在諸色上的意欲貪,這是心的\twnr{隨雜染}{288.0};凡在諸聲音上……(中略)凡在諸氣味上……(中略)凡在諸味道上……(中略)凡在諸\twnr{所觸}{220.2}上……(中略)凡在諸法上的意欲貪,這是心的隨雜染。

  比丘們!當在這六處上比丘心的隨雜染被捨斷,他的心是傾向離欲的;已遍\twnr{修習}{94.0}離欲的心可以認為在應該被證智作證的諸法上,是\twnr{適合作業的}{412.0}。」



\sutta{3}{3}{識經}{https://agama.buddhason.org/SN/sn.php?keyword=27.3}
  起源於舍衛城。

  「\twnr{比丘}{31.0}們!凡在眼識上的意欲貪,這是心的\twnr{隨雜染}{288.0};凡在耳識上……(中略)凡在鼻識上……(中略)凡在舌識上……(中略)凡在身識上……(中略)凡在意識上的意欲貪,這是心的隨雜染。

  比丘們!當在這六處上比丘心的隨雜染被捨斷,他的心是傾向離欲的;已遍\twnr{修習}{94.0}離欲的心可以認為在應該被證智作證的諸法上,是\twnr{適合作業的}{412.0}。」



\sutta{4}{4}{觸經}{https://agama.buddhason.org/SN/sn.php?keyword=27.4}
  起源於舍衛城。

  「\twnr{比丘}{31.0}們!凡在眼觸上的意欲貪,這是心的\twnr{隨雜染}{288.0};凡在耳觸上……(中略)凡在鼻觸上……(中略)凡在舌觸上……(中略)凡在身觸上……(中略)凡在意觸上的意欲貪,這是心的隨雜染。

  比丘們!當比丘……(中略)在應該被證智作證的諸法上,是\twnr{適合作業的}{412.0}。」



\sutta{5}{5}{觸所生經}{https://agama.buddhason.org/SN/sn.php?keyword=27.5}
  起源於舍衛城。

  「\twnr{比丘}{31.0}們!凡在眼觸所生受上的意欲貪,這是心的\twnr{隨雜染}{288.0};凡在耳觸所生受上……(中略)凡在鼻觸所生受上……(中略)凡在舌觸所生受上……(中略)凡在身觸所生受上……(中略)凡在意觸所生受上的意欲貪,這是心的隨雜染。

  比丘們!當比丘……(中略)在應該被證智作證的諸法上,是\twnr{適合作業的}{412.0}。」



\sutta{6}{6}{想經}{https://agama.buddhason.org/SN/sn.php?keyword=27.6}
  起源於舍衛城。

  「\twnr{比丘}{31.0}們!凡在色之想上的意欲貪,這是心的\twnr{隨雜染}{288.0};凡在聲音之想上……(中略)凡在氣味之想上……(中略)凡在味道之想上……(中略)凡在\twnr{所觸}{220.2}之想上……(中略)凡在法想上的意欲貪,這是心的隨雜染。

  比丘們!當比丘……(中略)在應該被證智作證的諸法上,是\twnr{適合作業的}{412.0}。」



\sutta{7}{7}{思經}{https://agama.buddhason.org/SN/sn.php?keyword=27.7}
  起源於舍衛城。

  「\twnr{比丘}{31.0}們!凡在\twnr{色之思}{872.0}上的意欲貪是心的\twnr{隨雜染}{288.0};凡在聲音之思上……(中略)凡在氣味之思上……(中略)凡在味道之思上……(中略)凡在\twnr{所觸}{220.2}之思上……(中略)凡在法思上的意欲貪,這是心的隨雜染。

  比丘們!當比丘……(中略)在應該被證智作證的諸法上,是\twnr{適合作業的}{412.0}。」



\sutta{8}{8}{渴愛經}{https://agama.buddhason.org/SN/sn.php?keyword=27.8}
  起源於舍衛城。

  「\twnr{比丘}{31.0}們!凡在色之渴愛上的意欲貪,這是心的\twnr{隨雜染}{288.0};凡在聲之渴愛上……(中略)凡在氣味之渴愛上……(中略)凡在味道之渴愛上……(中略)凡在\twnr{所觸}{220.2}之渴愛上……(中略)凡在法之渴愛上的意欲貪,這是心的隨雜染。

  比丘們!當比丘……(中略)在應該被證智作證的諸法上,是\twnr{適合作業的}{412.0}。」



\sutta{9}{9}{界經}{https://agama.buddhason.org/SN/sn.php?keyword=27.9}
  起源於舍衛城。

  「\twnr{比丘}{31.0}們!凡在地界上的意欲貪,這是心的\twnr{隨雜染}{288.0};凡在水界上……(中略)凡在火界上……(中略)凡在風界上……(中略)凡在虛空界上……(中略)凡在識界上的意欲貪,這是心的隨雜染。

  比丘們!當在這六處上比丘心的隨雜染被捨斷,他的心是傾向離欲的;已遍\twnr{修習}{94.0}離欲的心可以認為在應該被證智作證的諸法上,是\twnr{適合作業的}{412.0}。」



\sutta{10}{10}{蘊經}{https://agama.buddhason.org/SN/sn.php?keyword=27.10}
  起源於舍衛城。

  「\twnr{比丘}{31.0}們!凡在色上的意欲貪,這是心的\twnr{隨雜染}{288.0}……(中略)凡在識上的意欲貪,這是心的隨雜染。

  比丘們!當在這五處上比丘心的隨雜染被捨斷,他的心是傾向離欲的;已遍\twnr{修習}{94.0}離欲的心可以認為在應該被證智作證的諸法上,是\twnr{適合作業的}{412.0}。」

  污染相應完成,其\twnr{攝頌}{35.0}:

  「眼、色與識,觸與受,

   想與思、渴愛,界與蘊它們為十。」





\page

\xiangying{28}{舍利弗相應}
\sutta{1}{1}{離而生經}{https://agama.buddhason.org/SN/sn.php?keyword=28.1}
  \twnr{有一次}{2.0},\twnr{尊者}{200.0}舍利弗住在舍衛城祇樹林給孤獨園。

  那時,尊者舍利弗午前時穿衣、拿起衣鉢後,\twnr{為了托鉢}{87.0}進入舍衛城。在舍衛城為了托鉢行走後,\twnr{餐後已從施食返回}{512.0},\twnr{為了白天的住處}{128.0}去\twnr{盲者的樹林}{88.0},進入盲者的樹林後,坐在某顆樹下為白天的住處。

  那時,尊者舍利弗傍晚時,從\twnr{獨坐}{92.0}出來,去祇樹給孤獨園。尊者阿難看見正從遠處到來的尊者舍利弗。看見後,對尊者舍利弗說這個:

  「舍利弗\twnr{學友}{201.0}!你的諸根明淨,臉色清淨、皎潔,今日尊者舍利弗以什麼住處而住呢?」 

  「學友!這裡,就從離諸欲後,從離諸不善法後,我\twnr{進入後住於}{66.0}有尋、\twnr{有伺}{175.0},\twnr{離而生喜、樂}{174.0}的初禪,學友!那個我不這麼想:『我進入初禪。』或『我已進入初禪。』或『我已從初禪出來。』」

  「那麼,像這樣是因為對尊者舍利弗來說,已\twnr{長久}{51.0}善根除\twnr{我作}{22.0}、\twnr{我所作}{25.0}、\twnr{慢煩惱潛在趨勢}{26.0},因此,尊者舍利弗不這麼想:『我進入初禪。』或『我已進入初禪。』或『我已從初禪出來。』」



\sutta{2}{2}{無尋經}{https://agama.buddhason.org/SN/sn.php?keyword=28.2}
  起源於舍衛城。

  \twnr{尊者}{200.0}阿難看見……(中略)對尊者舍利弗說這個:

  「舍利弗\twnr{學友}{201.0}!你的諸根明淨,臉色清淨、皎潔,今日尊者舍利弗以什麼住處而住呢?」 

  「學友!這裡,從尋與伺的平息,\twnr{自身內的明淨}{256.0},\twnr{心的專一性}{255.0},我\twnr{進入後住於}{66.0}無尋、無伺,定而生喜、樂的第二禪,學友!那個我不這麼想:『我進入第二禪。』或『我已進入第二禪。』或『我已從第二禪出來。』」

  「那麼,像這樣是因為對尊者舍利弗來說,已\twnr{長久}{51.0}善根除\twnr{我作}{22.0}、\twnr{我所作}{25.0}、\twnr{慢煩惱潛在趨勢}{26.0},因此,尊者舍利弗不這麼想:『我進入第二禪。』或『我已進入第二禪。』或『我已從第二禪出來。』」



\sutta{3}{3}{喜經}{https://agama.buddhason.org/SN/sn.php?keyword=28.3}
  起源於舍衛城。

  \twnr{尊者}{200.0}阿難看見……(中略)「舍利弗\twnr{學友}{201.0}!你的諸根明淨,臉色清淨、皎潔,今日尊者舍利弗以什麼住處而住呢?」 

  「學友!這裡,我從喜的\twnr{褪去}{77.0}、住於\twnr{平靜}{228.0}、有念正知、以身體感受樂,我\twnr{進入後住於}{66.0}凡聖者們告知『他是平靜者、具念者、\twnr{安樂住者}{317.0}』的第三禪,學友!那個我不這麼想:『我進入第三禪。』或『我已進入第三禪。』或『我已從第三禪出來。』」

  「那麼,像這樣是因為對尊者舍利弗來說,已\twnr{長久}{51.0}善根除\twnr{我作}{22.0}、\twnr{我所作}{25.0}、\twnr{慢煩惱潛在趨勢}{26.0},因此,尊者舍利弗不這麼想:『我進入第三禪。』或『我已進入第三禪。』或『我已從第三禪出來。』」



\sutta{4}{4}{平靜經}{https://agama.buddhason.org/SN/sn.php?keyword=28.4}
  起源於舍衛城。

  \twnr{尊者}{200.0}阿難看見……(中略)「舍利弗\twnr{學友}{201.0}!你的諸根明淨,臉色清淨、皎潔,今日尊者舍利弗以什麼住處而住呢?」 

  「學友!這裡,從樂的捨斷與從苦的捨斷,就在之前諸喜悅、憂的滅沒,我\twnr{進入後住於}{66.0}不苦不樂,\twnr{平靜、念遍純淨}{494.0}的第四禪,學友!那個我不這麼想:『我進入第四禪。』或『我已進入第四禪。』或『我已從第四禪出來。』」

  「那麼,像這樣是因為對尊者舍利弗來說,已\twnr{長久}{51.0}善根除\twnr{我作}{22.0}、\twnr{我所作}{25.0}、\twnr{慢煩惱潛在趨勢}{26.0},因此,尊者舍利弗不這麼想:『我進入第四禪。』或『我已進入第四禪。』或『我已從第四禪出來。』」



\sutta{5}{5}{虛空無邊處經}{https://agama.buddhason.org/SN/sn.php?keyword=28.5}
  起源於舍衛城。

  \twnr{尊者}{200.0}阿難看見……(中略)。 

  「\twnr{學友}{201.0}!這裡,\twnr{從一切色想的超越}{490.0},從\twnr{有對想}{331.0}的滅沒,從不作意種種想[而知]:『虛空是無邊的』,我\twnr{進入後住於}{66.0}虛空無邊處……(中略)。」

  「……(中略)[或『我已從虛空無邊處]出來。』」



\sutta{6}{6}{識無邊處經}{https://agama.buddhason.org/SN/sn.php?keyword=28.6}
  起源於舍衛城。

  \twnr{尊者}{200.0}阿難看見……(中略)。 

  「\twnr{學友}{201.0}!這裡,超越一切虛空無邊處後[而知]:『識是無邊的』,我\twnr{進入後住於}{66.0}識無邊處……(中略)。」

  「……(中略)[或『我已從識無邊處]出來。』」



\sutta{7}{7}{無所有處經}{https://agama.buddhason.org/SN/sn.php?keyword=28.7}
  起源於舍衛城。

  \twnr{尊者}{200.0}舍利弗……(中略)。 

  「\twnr{學友}{201.0}!這裡,超越一切識無邊處後[而知]:『什麼都沒有』,我\twnr{進入後住於}{66.0}\twnr{無所有處}{533.0}……(中略)。」

  「……(中略)[或『我已從無所有處]出來。』」



\sutta{8}{8}{非想非非想處經}{https://agama.buddhason.org/SN/sn.php?keyword=28.8}
  起源於舍衛城。

  \twnr{尊者}{200.0}舍利弗……(中略)。 

  「\twnr{學友}{201.0}!這裡,超越一切\twnr{無所有處}{533.0}後,我\twnr{進入後住於}{66.0}\twnr{非想非非想處}{534.0}……(中略)。」

  「……(中略)[或『我已從非想非非想處]出來。』」



\sutta{9}{9}{滅等至經}{https://agama.buddhason.org/SN/sn.php?keyword=28.9}
  起源於舍衛城。

  \twnr{尊者}{200.0}舍利弗……(中略)。 

  「\twnr{學友}{201.0}!這裡,超越一切\twnr{非想非非想處}{534.0}後,我\twnr{進入後住於}{66.0}\twnr{想受滅}{416.0},學友!那個我不這麼想:『我進入想受滅。』或『我已進入想受滅。』或『我已從想受滅出來。』」

  「那麼,像這樣是因為對尊者舍利弗來說,已\twnr{長久}{51.0}善根除\twnr{我作}{22.0}、\twnr{我所作}{25.0}、\twnr{慢煩惱潛在趨勢}{26.0},因此,尊者舍利弗不這麼想:『我進入想受滅。』或『我已進入想受滅。』或『我已從想受滅出來。』」



\sutta{10}{10}{淨臉經}{https://agama.buddhason.org/SN/sn.php?keyword=28.10}
  \twnr{有一次}{2.0},\twnr{尊者}{200.0}舍利弗住在王舍城栗鼠飼養處的竹林中。

  那時,尊者舍利弗午前時穿衣、拿起衣鉢後,\twnr{為了托鉢}{87.0}進入王舍城。

  在王舍城\twnr{為了托鉢次第地行走著}{127.0}後,靠著某個牆下受用\twnr{施食}{196.0}。

  那時,女遊行者\twnr{淨臉}{x471}去見尊者舍利弗。抵達後,對尊者舍利弗說這個:

  「\twnr{沙門}{29.0}!是否你臉向下吃呢?」

  「姊妹!我不臉向下吃。」

  「沙門!那樣的話,是否你臉向上吃呢?」

  「姊妹!我不臉向上吃。」

  「沙門!那樣的話,是否你\twnr{臉向四方吃}{x472}呢?」

  「姊妹!我不臉向四方吃。」

  「沙門!那樣的話,是否你臉向四方的中間方吃呢?」

  「姊妹!我不臉向四方的中間方吃。」

  「當被像這樣問:『沙門!是否你臉向下吃呢?』你說:『姊妹!我不臉向下吃。』當被像這樣問:『沙門!那樣的話,是否你臉向上吃呢?』你說:『姊妹!我不臉向上吃。』當被像這樣問:『沙門!那樣的話,是否你臉向四方吃呢?』你說:『姊妹!我不臉向四方吃。』當被像這樣問:『沙門!那樣的話,是否你臉向四方的中間方吃呢?』你說:『姊妹!我不臉向四方的中間方吃。』沙門!那樣的話你如何吃?」

  「姊妹!凡任何沙門或婆羅門以宅地明的\twnr{畜生明}{955.1}邪命謀生者,姊妹!這些沙門婆羅門被稱為『臉向下吃』。

  姊妹!凡任何沙門或婆羅門以占星術的畜生明邪命謀生者,姊妹!這些沙門婆羅門被稱為『臉向上吃』。

  姊妹!凡任何沙門或婆羅門以\twnr{遣使行走的實行}{x473}邪命謀生者,姊妹!這些沙門婆羅門被稱為『臉向四方吃』。

  姊妹!凡任何沙門或婆羅門以手足占相的畜生明邪命謀生者,姊妹!這些沙門婆羅門被稱為『臉向四方的中間方吃』。

  姊妹!那個我不以宅地明的畜生明邪命謀生;不以占星術的畜生明邪命謀生;不以遣使行走的實行邪命謀生;不以手足占相的畜生明邪命謀生,我依法遍求施食,依法遍求施食後,我吃。」

  那時,女遊行者淨臉去王舍城後,從街道到街道;從十字路口到十字路口這麼告知:

  「\twnr{釋迦之徒的}{262.1}沙門們吃如法的食物,釋迦之徒的沙門們吃無過失的食物,請你們對釋迦之徒的沙門們施與食物。」

  舍利弗相應完成,其\twnr{攝頌}{35.0}:

  「離而生、無尋,喜、第四則平靜,

   空連同識,無所有、非非想,

   滅為第九說,以及第十則淨臉。」





\page

\xiangying{29}{龍相應}
\sutta{1}{1}{概要經}{https://agama.buddhason.org/SN/sn.php?keyword=29.1}
  起源於舍衛城。

  「\twnr{比丘}{31.0}們!有這四種龍的出生(胎),哪四種?卵生龍、胎生龍、濕生龍、\twnr{化生}{346.0}龍,比丘們!這是四種龍的出生(胎)。」



\sutta{2}{2}{更勝妙經}{https://agama.buddhason.org/SN/sn.php?keyword=29.2}
  起源於舍衛城。

  「\twnr{比丘}{31.0}們!有這四種龍的出生(胎),哪四種?卵生龍、胎生龍、濕生龍、\twnr{化生}{346.0}龍。比丘們!在那裡,胎生、濕生、化生龍比卵生龍更勝妙;比丘們!在那裡,濕生、化生龍比胎生、卵生龍更勝妙;比丘們!在那裡,化生龍比卵生、胎生、濕生龍更勝妙,比丘們!這是四種龍的出生(胎)。」 



\sutta{3}{3}{布薩經}{https://agama.buddhason.org/SN/sn.php?keyword=29.3}
  \twnr{被我這麼聽聞}{1.0}:

  \twnr{有一次}{2.0},\twnr{世尊}{12.0}住在舍衛城祇樹林給孤獨園。

  那時,\twnr{某位比丘}{39.0}去見世尊。抵達後,向世尊\twnr{問訊}{46.0}後,在一旁坐下。在一旁坐下的那位比丘對世尊說這個:

  「\twnr{大德}{45.0}!什麼因、什麼\twnr{緣}{180.0},以那個,這裡一些卵生龍\twnr{入布薩}{246.0},而成為\twnr{捨棄身體者}{x474}呢?」

  「比丘!這裡,一些卵生龍這麼想:『以前,我們是以身\twnr{作善惡二重者}{561.0},以語作善惡二重者,以意作善惡二重者。那些以身作善惡二重、以語作善惡二重、以意作善惡二重的我們,以身體的崩解,死後被往生卵生龍們的共住狀態,今日,如果我們以身行善行,以語行善行,以意行善行,這樣,我們以身體的崩解,死後往生\twnr{善趣}{112.0}、天界。來吧!現在,我們以身行善行,以語行善行,以意行善行。』比丘!這是因、這是緣,以那個,這裡一些卵生龍入布薩,而成為捨棄身體者。」



\sutta{4}{4}{布薩經第二}{https://agama.buddhason.org/SN/sn.php?keyword=29.4}
  起源於舍衛城。

  那時,\twnr{某位比丘}{39.0}去見\twnr{世尊}{12.0}。……(中略)在一旁坐下的那位比丘對世尊說這個:

  「\twnr{大德}{45.0}!什麼因、什麼\twnr{緣}{180.0},以那個,這裡一些胎生龍\twnr{入布薩}{246.0},而成為\twnr{捨棄身體者}{x475}呢?」

  「比丘!這裡……(中略)比丘!這是因、這是緣,以那個,這裡一些胎生龍入布薩,而成為捨棄身體者。」



\sutta{5}{5}{布薩經第三}{https://agama.buddhason.org/SN/sn.php?keyword=29.5}
  起源於舍衛城。

  在一旁坐下的那位\twnr{比丘}{31.0}對\twnr{世尊}{12.0}說這個:

  「\twnr{大德}{45.0}!什麼因、什麼\twnr{緣}{180.0},以那個,這裡一些濕生龍\twnr{入布薩}{246.0},而成為\twnr{捨棄身體者}{x475}呢?」

  「比丘!這裡……(中略)比丘!這是因、這是緣,以那個,這裡一些濕生龍入布薩,而成為捨棄身體者。」



\sutta{6}{6}{布薩經第四}{https://agama.buddhason.org/SN/sn.php?keyword=29.6}
  起源於舍衛城。

  在一旁坐下的那位\twnr{比丘}{31.0}對\twnr{世尊}{12.0}說這個:

  「\twnr{大德}{45.0}!什麼因、什麼\twnr{緣}{180.0},以那個,這裡一些\twnr{化生}{346.0}龍\twnr{入布薩}{246.0},而成為\twnr{捨棄身體者}{x475}呢?」

  「比丘!這裡,一些化生龍這麼想:『以前,我們是以身\twnr{作善惡二重者}{561.0},以語作善惡二重者,以意作善惡二重者。那些以身作善惡二重、以語作善惡二重、以意作善惡二重的我們,以身體的崩解,死後被往生化生龍們的共住狀態,今日,如果我們以身行善行,以語行善行,以意行善行,這樣,我們以身體的崩解,死後往生\twnr{善趣}{112.0}、天界。來吧!現在,我們以身行善行,以語行善行,以意行善行。』比丘!這是因、這是緣,以那個,這裡一些化生龍入布薩,而成為捨棄身體者。」



\sutta{7}{7}{所聞經}{https://agama.buddhason.org/SN/sn.php?keyword=29.7}
  起源於舍衛城。

  在一旁坐下的那位\twnr{比丘}{31.0}對\twnr{世尊}{12.0}說這個:

  「\twnr{大德}{45.0}!什麼因、什麼\twnr{緣}{180.0},以那個,這裡某人以身體的崩解,死後往生卵生龍們的共住狀態呢?」

  「比丘!這裡,某人是以身\twnr{作善惡二重者}{561.0},以語作善惡二重者,以意作善惡二重者,有他的所聞:『卵生龍是長壽者、美貌者、多樂者。』他這麼想:『啊!願我以身體的崩解,死後往生卵生龍們的共住狀態。』他以身體的崩解,死後往生卵生龍們的共住狀態。比丘!這是因、這是緣,以那個,這裡某人以身體的崩解,死後往生卵生龍們的共住狀態。」



\sutta{8}{8}{所聞經第二}{https://agama.buddhason.org/SN/sn.php?keyword=29.8}
  起源於舍衛城。

  在一旁坐下的那位\twnr{比丘}{31.0}對\twnr{世尊}{12.0}說這個:

  「\twnr{大德}{45.0}!什麼因、什麼\twnr{緣}{180.0},以那個,這裡某人以身體的崩解,死後往生胎生龍們的共住狀態呢?」

  「……(中略)比丘!這是因、這是緣,以那個,這裡某人以身體的崩解,死後往生胎生龍們的共住狀態。」



\sutta{9}{9}{所聞經第三}{https://agama.buddhason.org/SN/sn.php?keyword=29.9}
  起源於舍衛城。

  在一旁坐下的那位\twnr{比丘}{31.0}對\twnr{世尊}{12.0}說這個:

  「\twnr{大德}{45.0}!什麼因、什麼\twnr{緣}{180.0},以那個,這裡某人以身體的崩解,死後往生濕生龍們的共住狀態呢?」

  「……(中略)比丘!這是因、這是緣,以那個,這裡某人以身體的崩解,死後往生濕生龍們的共住狀態。」



\sutta{10}{10}{所聞經第四}{https://agama.buddhason.org/SN/sn.php?keyword=29.10}
  起源於舍衛城。

  在一旁坐下的那位\twnr{比丘}{31.0}對\twnr{世尊}{12.0}說這個:

  「\twnr{大德}{45.0}!這裡,什麼因、什麼\twnr{緣}{180.0},以那個某人以身體的崩解,死後往生\twnr{化生}{346.0}龍們的共住狀態呢?」

  「比丘!這裡,某人是以身\twnr{作善惡二重者}{561.0},以語作善惡二重者,以意作善惡二重者,有他的所聞:『化生龍是長壽者、美貌者、多樂者。』他這麼想:『啊!願我以身體的崩解,死後往生化生龍們的共住狀態。』他以身體的崩解,死後往生化生龍們的共住狀態。比丘!這是因、這是緣,以那個,這裡某人以身體的崩解,死後往生化生龍們的共住狀態。」



\sutta{11}{20}{卵生布施之資助經十則}{https://agama.buddhason.org/SN/sn.php?keyword=29.11}
  起源於舍衛城。

  在一旁坐下的那位\twnr{比丘}{31.0}對\twnr{世尊}{12.0}說這個:

  「\twnr{大德}{45.0}!什麼因、什麼\twnr{緣}{180.0},以那個,這裡某人以身體的崩解,死後往生卵生龍們的共住狀態呢?」

  「比丘!這裡,某人是以身\twnr{作善惡二重者}{561.0},以語作善惡二重者,以意作善惡二重者,有他的所聞:『卵生龍是長壽者、美貌者、多樂者。』他這麼想:『啊!願我以身體的崩解,死後往生卵生龍們的共住狀態。』他施與食物,他以身體的崩解,死後往生卵生龍們的共住狀態。比丘!這是因……(中略)往生……。」「……(中略)他施與飲料……(中略)他施與衣服……(中略)他施與車乘……(中略)他施與花環……(中略)他施與香料……(中略)他施與塗油……(中略)他施與臥床……(中略)他施與住處……(中略)他施與燈燭,他以身體的崩解,死後往生卵生龍們的共住狀態。比丘!這是因、這是緣,以那個,這裡某人以身體的崩解,死後往生卵生龍們的共住狀態。」



\sutta{21}{50}{胎生等布施之資助經三十則}{https://agama.buddhason.org/SN/sn.php?keyword=29.21}
  起源於舍衛城。

  在一旁坐下的那位\twnr{比丘}{31.0}對\twnr{世尊}{12.0}說這個:

  「\twnr{大德}{45.0}!什麼因、什麼\twnr{緣}{180.0},以那個,這裡某人以身體的崩解,死後往生胎生龍……(中略)濕生龍……(中略)\twnr{化生}{346.0}龍們的共住狀態呢?」

  「比丘!這裡,某人是以身\twnr{作善惡二重者}{561.0},以語作善惡二重者,以意作善惡二重者,有他的所聞:『化生龍是長壽者、美貌者、多樂者。』他這麼想:『啊!願我以身體的崩解,死後往生化生龍們的共住狀態。』他施與食物……(中略)他施與飲料……(中略)他施與燈燭,他以身體的崩解,死後往生卵生龍們的共住狀態。比丘!這是因、這是緣,以那個,這裡某人以身體的崩解,死後往生化生龍們的共住狀態。」

  (「十經、十經應該被這些中略完成,這樣,在四種胎上有四十則解說,與最初的十經一起有五十經。」)

  龍相應完成,其\twnr{攝頌}{35.0}:

  「概要、更勝妙,以及四則布薩,

   他的所聞四則,以及布施之資助四十則,

   五十則集團經,在龍上被善知道。」





\page

\xiangying{30}{金翅鳥相應}
\sutta{1}{1}{概要經}{https://agama.buddhason.org/SN/sn.php?keyword=30.1}
  起源於舍衛城。

  「\twnr{比丘}{31.0}們!有這四種金翅鳥的出生(胎),哪四種?卵生金翅鳥、胎生金翅鳥、濕生金翅鳥、\twnr{化生}{346.0}金翅鳥,比丘們!這是四種金翅鳥的出生(胎)。」



\sutta{2}{2}{帶走經}{https://agama.buddhason.org/SN/sn.php?keyword=30.2}
  起源於舍衛城。

  「\twnr{比丘}{31.0}們!有這四種金翅鳥的出生(胎),哪四種?卵生……(中略)比丘們!這是四種金翅鳥的出生(胎)。比丘們!在那裡,卵生金翅鳥帶走卵生龍[吃掉],非胎生、非濕生、非\twnr{化生}{346.0}[龍];比丘們!在那裡,胎生金翅鳥帶走卵生與胎生龍,非濕生、非化生;比丘們!在那裡,濕生金翅鳥帶走卵生、胎生、濕生龍,非化生;比丘們!在那裡,化生金翅鳥帶走卵生、胎生、濕生、化生龍,比丘們!這是四種金翅鳥的出生。」



\sutta{3}{3}{作善惡二重者經}{https://agama.buddhason.org/SN/sn.php?keyword=30.3}
  起源於舍衛城。

  某位比丘去見世尊。抵達後,向世尊\twnr{問訊}{46.0}後,在一旁坐下。在一旁坐下的那位\twnr{比丘}{31.0}對\twnr{世尊}{12.0}說這個:

  「\twnr{大德}{45.0}!什麼因、什麼\twnr{緣}{180.0},以那個,這裡某人以身體的崩解,死後往生卵生金翅鳥們的共住狀態呢?」

  「比丘!這裡,某人是以身\twnr{作善惡二重者}{561.0},以語作善惡二重者,以意作善惡二重者,有他的所聞:『卵生金翅鳥是長壽者、美貌者、多樂者。』他這麼想:『啊!願我以身體的崩解,死後往生卵生金翅鳥們的共住狀態。』他以身體的崩解,死後往生卵生金翅鳥們的共住狀態。比丘!這是因、這是緣,以那個,這裡某人以身體的崩解,死後往生卵生金翅鳥們的共住狀態。」



\sutta{4}{6}{作善惡二重者經第二等三則}{https://agama.buddhason.org/SN/sn.php?keyword=30.4}
  起源於舍衛城。

  在一旁坐下的那位\twnr{比丘}{31.0}對\twnr{世尊}{12.0}說這個:

  「\twnr{大德}{45.0}!什麼因、什麼\twnr{緣}{180.0},以那個,這裡某人以身體的崩解,死後往生胎生金翅鳥……(中略)濕生金翅鳥……(中略)\twnr{化生}{346.0}金翅鳥們的共住狀態呢?」

  「比丘!這裡,某人是以身\twnr{作善惡二重者}{561.0},以語作善惡二重者,以意作善惡二重者,有他的所聞:『化生金翅鳥是長壽者、美貌者、多樂者。』他這麼想:『啊!願我以身體的崩解,死後往生化生金翅鳥們的共住狀態。』他以身體的崩解,死後往生化生金翅鳥們的共住狀態。比丘!這是因、這是緣,以那個,這裡某人以身體的崩解,死後往生化生金翅鳥們的共住狀態。」



\sutta{7}{16}{卵生布施之資助經十則}{https://agama.buddhason.org/SN/sn.php?keyword=30.7}
  起源於舍衛城。

  在一旁坐下的那位\twnr{比丘}{31.0}對\twnr{世尊}{12.0}說這個:

  「\twnr{大德}{45.0}!什麼因、什麼\twnr{緣}{180.0},以那個,這裡某人以身體的崩解,死後往生卵生金翅鳥們的共住狀態呢?」

  「比丘!這裡,某人是以身\twnr{作善惡二重者}{561.0},以語作善惡二重者,以意作善惡二重者,有他的所聞:『卵生金翅鳥是長壽者、美貌者、多樂者。』他這麼想:『啊!願我以身體的崩解,死後往生卵生金翅鳥們的共住狀態。』他施與食物……(中略)他施與飲料……(中略)他施與衣服……(中略)他施與車乘……(中略)他施與花環……(中略)他施與香料……(中略)他施與塗油……(中略)他施與臥床……(中略)他施與住處……(中略)他施與燈燭,他以身體的崩解,死後往生卵生金翅鳥們的共住狀態。比丘!這是因、這是緣,以那個,這裡某人以身體的崩解,死後往生卵生金翅鳥們的共住狀態。」



\sutta{17}{46}{胎生等布施之資助經三十則}{https://agama.buddhason.org/SN/sn.php?keyword=30.17}
  起源於舍衛城。

  在一旁坐下的那位\twnr{比丘}{31.0}對\twnr{世尊}{12.0}說這個:

  「\twnr{大德}{45.0}!什麼因、什麼\twnr{緣}{180.0},以那個,這裡某人以身體的崩解,死後往生胎生金翅鳥……(中略)濕生金翅鳥……(中略)\twnr{化生}{346.0}金翅鳥們的共住狀態呢?」

  「比丘!這裡,某人是以身\twnr{作善惡二重者}{561.0},以語作善惡二重者,以意作善惡二重者,有他的所聞:『化生金翅鳥是長壽者、美貌者、多樂者。』他這麼想:『啊!願我以身體的崩解,死後往生化生金翅鳥們的共住狀態。』他施與食物……(中略)他施與飲料……(中略)他施與燈燭,他以身體的崩解,死後往生卵生金翅鳥們的共住狀態。比丘!這是因、這是緣,以那個,這裡某人以身體的崩解,死後往生化生金翅鳥們的共住狀態。」

  (「這樣,以集團而有四十六經。」)

  金翅鳥相應完成,其\twnr{攝頌}{35.0}:

  「概要、帶走,連同作善惡二重者四則,

   布施之資助[四]十則,在金翅鳥上被善知道。」





\page

\xiangying{31}{乾達婆眾相應}
\sutta{1}{1}{概要經}{https://agama.buddhason.org/SN/sn.php?keyword=31.1}
  \twnr{有一次}{2.0},\twnr{世尊}{12.0}住在舍衛城祇樹林給孤獨園。……(中略)

  世尊說這個:

  「\twnr{比丘}{31.0}們!我將為你們教導乾達婆眾天神,\twnr{你們要聽}{43.0}它!

  比丘們!而什麼是乾達婆眾天神?比丘們!有\twnr{在香樹根}{x476}居住的天神;比丘們!有在香樹\twnr{心材}{356.0}居住的天神;比丘們!有在香樹膚材居住的天神;比丘們!有在香樹內皮居住的天神;比丘們!有在香樹外皮居住的天神;比丘們!有在香葉居住的天神;比丘們!有在香花居住的天神;比丘們!有在香果實居住的天神;比丘們!有在香樹汁居住的天神;比丘們!有在香氣味居住的天神,比丘們!這些被稱為乾達婆眾天神。」



\sutta{2}{2}{善行經}{https://agama.buddhason.org/SN/sn.php?keyword=31.2}
  起源於舍衛城。

  在一旁坐下的那位\twnr{比丘}{31.0}對\twnr{世尊}{12.0}說這個:

  「\twnr{大德}{45.0}!什麼因、什麼\twnr{緣}{180.0},以那個,這裡某人以身體的崩解,死後往生乾達婆眾天神們的共住狀態呢?」

  「比丘!這裡,某人以身行善行、以語行善行、以意行善行,有他的所聞:『乾達婆眾的天神是長壽者、美貌者、多樂者。』他這麼想:『啊!願我以身體的崩解,死後往生乾達婆眾天神們的共住狀態。』他以身體的崩解,死後往生乾達婆眾天神們的共住狀態。比丘!這是因、這是緣,以那個,這裡某人以身體的崩解,死後往生乾達婆眾天神們的共住狀態。」



\sutta{3}{3}{香樹根的施與者經}{https://agama.buddhason.org/SN/sn.php?keyword=31.3}
  起源於舍衛城。

  在一旁坐下的那位\twnr{比丘}{31.0}對\twnr{世尊}{12.0}說這個:

  「\twnr{大德}{45.0}!什麼因、什麼\twnr{緣}{180.0},以那個,這裡某人以身體的崩解,死後往生在香樹根居住天神們的共住狀態呢?」

  「比丘!這裡,某人以身行善行、以語行善行、以意行善行,有他的所聞:『在香樹根居住的天神是長壽者、美貌者、多樂者。』他這麼想:『啊!願我以身體的崩解,死後往生在香樹根居住天神們的共住狀態。』他是香樹根的施與者,他以身體的崩解,死後往生在香樹根居住天神們的共住狀態。比丘!這是因、這是緣,以那個,這裡某人以身體的崩解,死後往生在香樹根居住天神們的共住狀態。」



\sutta{4}{12}{香樹心材的施與者經九則}{https://agama.buddhason.org/SN/sn.php?keyword=31.4}
  起源於舍衛城。

  在一旁坐下的那位\twnr{比丘}{31.0}對\twnr{世尊}{12.0}說這個:

  「\twnr{大德}{45.0}!什麼因、什麼\twnr{緣}{180.0},以那個,這裡某人以身體的崩解,死後往生在香樹\twnr{心材}{356.0}居住天神們的……(中略)與在香樹膚材居住天神們的……與在香樹內皮居住天神們的……與在香樹外皮居住天神們的……與在香葉居住天神們的……與在香花居住天神們的……與在香果實居住天神們的……與在香樹汁居住天神們的……與在香氣味居住天神們的共住狀態呢?」

  「比丘!這裡,某人以身行善行、以語行善行、以意行善行,有他的所聞:『在香樹心材居住天神們是長壽者、美貌者、多樂者。』他這麼想:『啊!願我以身體的崩解,死後往生在香樹心材居住天神們的……(中略)與在香樹膚材居住天神們的……與在香樹內皮居住天神們的……與在香樹外皮居住天神們的……與在香葉居住天神們的……與在香花居住天神們的……與在香果實居住天神們的……與在香樹汁居住天神們的……與在香氣味居住天神們的們的共住狀態。』他是香樹心材的施與者……(中略)他是香樹膚材的施與者……他是香樹內皮的施與者……他是香樹外皮的施與者……他是香葉的施與者……他是香花的施與者……他是香果實的施與者……他是香樹汁的施與者……他是香氣味的施與者,他以身體的崩解,死後往生在香氣味居住天神們的們的共住狀態。比丘!這是因、這是緣,以那個,這裡某人以身體的崩解,死後往生在香氣味居住天神們的們的共住狀態。」



\sutta{13}{22}{香樹根布施之資助經十則}{https://agama.buddhason.org/SN/sn.php?keyword=31.13}
  起源於舍衛城。

  在一旁坐下的那位\twnr{比丘}{31.0}對\twnr{世尊}{12.0}說這個:

  「\twnr{大德}{45.0}!什麼因、什麼\twnr{緣}{180.0},以那個,這裡某人以身體的崩解,死後往生在香樹根居住天神們的共住狀態呢?」

  「比丘!這裡,某人以身行善行、以語行善行、以意行善行,有他的所聞:『在香樹根居住的天神是長壽者、美貌者、多樂者。』他這麼想:『啊!願我以身體的崩解,死後往生在香樹根居住天神們的共住狀態。』他施與食物……(中略)他施與飲料……他施與衣服……他施與車乘……他施與花環……他施與香料……他施與塗油……他施與臥床……他施與住處……他施與燈燭,他以身體的崩解,死後往生在香樹根居住天神們的共住狀態。比丘!這是因、這是緣,以那個,這裡某人以身體的崩解,死後往生在香樹根居住天神們的共住狀態。」



\sutta{23}{112}{香樹心材布施之資助經九十則}{https://agama.buddhason.org/SN/sn.php?keyword=31.23}
  起源於舍衛城。

  在一旁坐下的那位\twnr{比丘}{31.0}對\twnr{世尊}{12.0}說這個:

  「\twnr{大德}{45.0}!什麼因、什麼\twnr{緣}{180.0},以那個,這裡某人以身體的崩解,死後往生在香樹\twnr{心材}{356.0}居住的天神……(中略)與在香樹膚材居住的天神……與在香樹內皮居住的天神……與在香樹外皮居住的天神……與在香葉居住的天神……與在香花居住的天神……與在香果實居住的天神……與在香樹汁居住的天神……與在香氣味居住天神們的共住狀態呢?」

  「比丘!這裡,某人以身行善行、以語行善行、以意行善行,有他的所聞:『在香氣味居住的天神是長壽者、美貌者、多樂者。』他這麼想:『啊!願我以身體的崩解,死後往生在香氣味居住天神們的共住狀態。』他施與食物……(中略)他施與飲料……他施與衣服……他施與車乘……他施與花環……他施與香料……他施與塗油……他施與臥床……他施與住處……他施與燈燭,他以身體的崩解,死後往生在香氣味居住天神們的共住狀態。比丘!這是因、這是緣,以那個,這裡某人以身體的崩解,死後往生在香氣味居住天神們的共住狀態。」

  (「這樣,以集團而有一一二經。」)

  乾達婆眾相應完成,其\twnr{攝頌}{35.0}:

  「概要與善行,施與者在後十則,

   布施之資助百則,在乾達婆上被善知道。」





\page

\xiangying{32}{雲相應}
\sutta{1}{1}{概要經}{https://agama.buddhason.org/SN/sn.php?keyword=32.1}
  起源於舍衛城。

  「\twnr{比丘}{31.0}們!我將為你們教導雲眾天神,\twnr{你們要聽}{43.0}它!

  比丘們!而什麼是雲眾天神?比丘們!有寒雲天神,有熱雲天神,有黑雲天神,有風雲天神,有雨雲天神,比丘們!這些被稱為雲眾天神。」



\sutta{2}{2}{善行經}{https://agama.buddhason.org/SN/sn.php?keyword=32.2}
  起源於舍衛城。

  在一旁坐下的那位\twnr{比丘}{31.0}對\twnr{世尊}{12.0}說這個:

  「\twnr{大德}{45.0}!什麼因、什麼\twnr{緣}{180.0},以那個,這裡某人以身體的崩解,死後往生雲眾的天神們的共住狀態呢?」

  「比丘!這裡,某人以身行善行、以語行善行、以意行善行,有他的所聞:『雲眾的天神是長壽者、美貌者、多樂者。』他這麼想:『啊!願我以身體的崩解,死後往生雲眾的天神們的共住狀態。』他以身體的崩解,死後往生雲眾的天神們的共住狀態。比丘!這是因、這是緣,以那個,這裡某人以身體的崩解,死後往生雲眾的天神們的共住狀態。」



\sutta{3}{12}{寒雲布施之資助經十則經}{https://agama.buddhason.org/SN/sn.php?keyword=32.3}
  起源於舍衛城。

  在一旁坐下的那位\twnr{比丘}{31.0}對\twnr{世尊}{12.0}說這個:

  「\twnr{大德}{45.0}!什麼因、什麼\twnr{緣}{180.0},以那個,這裡某人以身體的崩解,死後往生寒雲天神們的共住狀態呢?」

  「比丘!這裡,某人以身行善行、以語行善行、以意行善行,有他的所聞:『寒雲天神是長壽者、美貌者、多樂者。』他這麼想:『啊!願我以身體的崩解,死後往生寒雲天神們的共住狀態。』他施與食物……(中略)他施與燈燭,他以身體的崩解,死後往生寒雲天神們的共住狀態。比丘!這是因、這是緣,以那個,這裡某人以身體的崩解,死後往生寒雲天神們的共住狀態。」



\sutta{13}{52}{熱雲布施之資助經四十則}{https://agama.buddhason.org/SN/sn.php?keyword=32.13}
  起源於舍衛城。

  在一旁坐下的那位\twnr{比丘}{31.0}對\twnr{世尊}{12.0}說這個:

  「\twnr{大德}{45.0}!什麼因、什麼\twnr{緣}{180.0},以那個,這裡某人以身體的崩解,死後往生熱雲天神……(中略)黑雲天神……(中略)風雲天神……(中略)往生雨雲天神們的共住狀態呢?」

  「比丘!這裡,某人以身行善行、以語行善行、以意行善行,有他的所聞:『雨雲天神是長壽者、美貌者、多樂者。』他這麼想:『啊!願我以身體的崩解,死後往生雨雲天神們的共住狀態。』他施與食物……(中略)他施與燈燭,他以身體的崩解,死後往生雨雲天神們的共住狀態。比丘!這是因、這是緣,以那個,這裡某人以身體的崩解,死後往生雨雲天神們的共住狀態。」



\sutta{53}{53}{寒雲經}{https://agama.buddhason.org/SN/sn.php?keyword=32.53}
  起源於舍衛城。

  在一旁坐下的那位\twnr{比丘}{31.0}對\twnr{世尊}{12.0}說這個:

  「\twnr{大德}{45.0}!什麼因、什麼\twnr{緣}{180.0},以那個有時變寒呢?」

  「比丘!有名叫寒雲的天神,每當祂們想這個:『\twnr{讓我們住於自己的喜樂}{928.0}。』時,隨從祂們的心願後變寒。比丘!這是因、這是緣,以那個有時變寒。」 



\sutta{54}{54}{熱雲經}{https://agama.buddhason.org/SN/sn.php?keyword=32.54}
  起源於舍衛城。

  在一旁坐下的那位\twnr{比丘}{31.0}對\twnr{世尊}{12.0}說這個:

  「\twnr{大德}{45.0}!什麼因、什麼\twnr{緣}{180.0},以那個有時變熱呢?」

  「比丘!有名叫熱雲的天神,每當祂們想這個:『\twnr{讓我們住於自己的喜樂}{928.0}。』時,隨從祂們的心願後變熱。比丘!這是因、這是緣,以那個有時變熱。」 



\sutta{55}{55}{黑雲經}{https://agama.buddhason.org/SN/sn.php?keyword=32.55}
  起源於舍衛城。

  在一旁坐下的那位\twnr{比丘}{31.0}對\twnr{世尊}{12.0}說這個:

  「\twnr{大德}{45.0}!什麼因、什麼\twnr{緣}{180.0},以那個有時變黑呢?」

  「比丘!有名叫黑雲的天神,每當祂們這麼想:『\twnr{讓我們住於自己的喜樂}{928.0}。』時,隨從祂們的心願後變黑。比丘!這是因、這是緣,以那個有時變黑。」 



\sutta{56}{56}{風雲經}{https://agama.buddhason.org/SN/sn.php?keyword=32.56}
  起源於舍衛城。

  在一旁坐下的那位\twnr{比丘}{31.0}對\twnr{世尊}{12.0}說這個:

  「\twnr{大德}{45.0}!什麼因、什麼\twnr{緣}{180.0},以那個有時有風呢?」

  「比丘!有名叫\twnr{風雲的天神}{x477},當祂們這麼想:『\twnr{讓我們住於自己的喜樂}{928.0}。』時,隨從祂們的心願後有風。比丘!這是因、這是緣,以那個有時有風。」 



\sutta{57}{57}{雨雲經}{https://agama.buddhason.org/SN/sn.php?keyword=32.57}
  起源於舍衛城。

  在一旁坐下的那位\twnr{比丘}{31.0}對\twnr{世尊}{12.0}說這個:

  「\twnr{大德}{45.0}!什麼因、什麼\twnr{緣}{180.0},以那個有時有雨呢?」

  「比丘!有名叫雨雲的天神,每當祂們這麼想:『\twnr{讓我們住於自己的喜樂}{928.0}。』時,隨從祂們的心願後有雨。比丘!這是因、這是緣,以那個有時有雨。」 

  五十七經終了。

  雲相應完成,其\twnr{攝頌}{35.0}:

  「概要與善行,布施之資助五十則,

   寒、熱與黑,風、雨雲。」





\page

\xiangying{33}{婆蹉氏相應}
\sutta{1}{1}{色-無知經}{https://agama.buddhason.org/SN/sn.php?keyword=33.1}
  \twnr{有一次}{2.0},\twnr{世尊}{12.0}住在舍衛城祇樹林給孤獨園。

  那時,\twnr{遊行者}{79.0}婆蹉氏去見世尊。抵達後,與世尊一起互相問候。交換應該被互相問候的友好交談後,在一旁坐下。在一旁坐下的遊行者婆蹉氏對世尊說這個:

  「\twnr{喬達摩}{80.0}尊師!什麼是因,什麼是緣:凡在世間中這些種種\twnr{惡見}{722.0}生起:『世界是常恆的,或\twnr{世界是非常恆的}{170.0},或世界是有邊的,或世界是無邊的,或命即是身體,或命是一身體是另一,或死後如來存在,或死後如來不存在、\twnr{死後如來存在且不存在}{354.0},或死後如來既非存在也非不存在?』」

  「婆蹉!以在色上的無知,以在色\twnr{集}{67.0}上的無知,以在色\twnr{滅}{68.0}上的無知,以在導向色\twnr{滅道跡}{69.0}上的無知,這樣,在世間中這些種種惡見生起:『世界是常恆的,或世界是非常恆的……(中略)或死後如來既非存在也非不存在。』婆蹉!這是因,這是緣:凡在世間中這些種種惡見生起:『世界是常恆的,或世界是非常恆的……(中略)或死後如來既非存在也非不存在。』」



\sutta{2}{2}{受-無知經}{https://agama.buddhason.org/SN/sn.php?keyword=33.2}
  起源於舍衛城。

  在一旁坐下的\twnr{遊行者}{79.0}婆蹉氏對\twnr{世尊}{12.0}說這個:

  「\twnr{喬達摩}{80.0}尊師!什麼是因,什麼是緣:凡在世間中這些種種\twnr{惡見}{722.0}生起:『世界是常恆的,或\twnr{世界是非常恆的}{170.0}……(中略)或世界是非常恆的死後如來既非存在也非不存在?』」

  「婆蹉!以在受上的無知,以在受\twnr{集}{67.0}上的無知,以在受\twnr{滅}{68.0}上的無知,以在導向受\twnr{滅道跡}{69.0}上的無知,這樣,在世間中這些種種惡見生起:『世界是常恆的,或世界是非常恆的……(中略)或死後如來既非存在也非不存在。』婆蹉!這是因,這是緣:凡在世間中這些種種惡見生起:『世界是常恆的,或世界是非常恆的……(中略)或死後如來既非存在也非不存在。』」



\sutta{3}{3}{想-無知經}{https://agama.buddhason.org/SN/sn.php?keyword=33.3}
  起源於舍衛城。

  在一旁坐下的\twnr{遊行者}{79.0}婆蹉氏對\twnr{世尊}{12.0}說這個:

  「\twnr{喬達摩}{80.0}尊師!什麼是因,什麼是緣:凡在世間中這些種種\twnr{惡見}{722.0}生起:『世界是常恆的,或\twnr{世界是非常恆的}{170.0}……(中略)或死後如來既非存在也非不存在?』」

  「婆蹉!以在想上的無知,以在想\twnr{集}{67.0}上的無知,以在想\twnr{滅}{68.0}上的無知,以在導向想\twnr{滅道跡}{69.0}上的無知,這樣,在世間中這些種種惡見生起:『世界是常恆的,或世界是非常恆的……(中略)或死後如來既非存在也非不存在。』婆蹉!這是因,這是緣:凡在世間中這些種種惡見生起:『世界是常恆的,或世界是非常恆的……(中略)或死後如來既非存在也非不存在。』」



\sutta{4}{4}{行-無知經}{https://agama.buddhason.org/SN/sn.php?keyword=33.4}
  起源於舍衛城。

  在一旁坐下的\twnr{遊行者}{79.0}婆蹉氏對\twnr{世尊}{12.0}說這個:

  「\twnr{喬達摩}{80.0}尊師!什麼是因,什麼是緣:凡在世間中這些種種\twnr{惡見}{722.0}生起:『世界是常恆的,或\twnr{世界是非常恆的}{170.0}……(中略)或死後如來既非存在也非不存在?』」

  「婆蹉!以在諸行上的無知,以在行\twnr{集}{67.0}上的無知,以在行\twnr{滅}{68.0}上的無知,以在導向行\twnr{滅道跡}{69.0}上的無知,這樣,在世間中這些種種惡見生起:『世界是常恆的,或世界是非常恆的……(中略)或死後如來既非存在也非不存在。』婆蹉!這是因,這是緣:凡在世間中這些種種惡見生起:『世界是常恆的,或世界是非常恆的……(中略)或死後如來既非存在也非不存在。』」



\sutta{5}{5}{識-無知經}{https://agama.buddhason.org/SN/sn.php?keyword=33.5}
  起源於舍衛城。

  在一旁坐下的\twnr{遊行者}{79.0}婆蹉氏對\twnr{世尊}{12.0}說這個:

  「\twnr{喬達摩}{80.0}尊師!什麼是因,什麼是緣:凡在世間中這些種種\twnr{惡見}{722.0}生起:『世界是常恆的,或\twnr{世界是非常恆的}{170.0}……(中略)或死後如來既非存在也非不存在?』」

  「婆蹉!以在識上的無知,以在識\twnr{集}{67.0}上的無知,以在識\twnr{滅}{68.0}上的無知,以在導向識\twnr{滅道跡}{69.0}上的無知,這樣,在世間中這些種種惡見生起:『世界是常恆的,或世界是非常恆的……(中略)或死後如來既非存在也非不存在。』婆蹉!這是因,這是緣:凡在世間中這些種種惡見生起:『世界是常恆的,或世界是非常恆的……(中略)或死後如來既非存在也非不存在。』」



\sutta{6}{10}{色-不見等經五則}{https://agama.buddhason.org/SN/sn.php?keyword=33.6}
  起源於舍衛城。

  在一旁坐下的\twnr{遊行者}{79.0}婆蹉氏對\twnr{世尊}{12.0}說這個:

  「\twnr{喬達摩}{80.0}尊師!什麼是因,什麼是緣:凡在世間中這些種種\twnr{惡見}{722.0}生起:『世界是常恆的,或\twnr{世界是非常恆的}{170.0}……(中略)或死後如來既非存在也非不存在?』」

  「婆蹉!以在色上的不見……(中略)以在導向色\twnr{滅道跡}{69.0}上的不見;以在受上……以在想上……婆蹉!以在諸行上的不見……(中略)婆蹉!以在識上的不見……(中略)以在導向識滅道跡上的不見……(中略)。」



\sutta{11}{15}{色-不現觀等經五則}{https://agama.buddhason.org/SN/sn.php?keyword=33.11}
  起源於舍衛城。

  「婆蹉!以在色上的不\twnr{現觀}{53.0}……(中略)以在導向色\twnr{滅道跡}{69.0}上的不現觀……(中略)。」

  起源於舍衛城。

  「婆蹉!以在受上的不現觀……(中略)以在導向受滅道跡上的不現觀……(中略)。」

  起源於舍衛城。

  「婆蹉!以在想上的不現觀……(中略)以在導向想滅道跡上的不現觀……(中略)。」

  起源於舍衛城。

  「婆蹉!以在諸行上的不現觀……(中略)以在導向行滅道跡上的不現觀……(中略)。」

  起源於舍衛城。

  「婆蹉!以在識上的不現觀……(中略)以在導向識滅道跡上的不現觀……(中略)。」



\sutta{16}{20}{色-不隨覺等經五則}{https://agama.buddhason.org/SN/sn.php?keyword=33.16}
  起源於舍衛城。

  在一旁坐下的\twnr{遊行者}{79.0}婆蹉氏對\twnr{世尊}{12.0}說這個:

  「\twnr{喬達摩}{80.0}尊師!什麼因、什麼\twnr{緣}{180.0}……(中略)。」

  「婆蹉!以在色上的不\twnr{隨覺}{355.0}……(中略)以在導向色\twnr{滅道跡}{69.0}上的不隨覺……(中略)。」

  起源於舍衛城。「婆蹉!以在受上的不隨覺……(中略)。」

  起源於舍衛城。「婆蹉!以在想上的不隨覺……(中略)。」

  起源於舍衛城。「婆蹉!以在行上的不隨覺……(中略)。」

  起源於舍衛城。「婆蹉!以在識上的不隨覺……(中略)以在導向識滅道跡上的不隨覺……(中略)。」



\sutta{21}{25}{色-不通達等經五則}{https://agama.buddhason.org/SN/sn.php?keyword=33.21}
  起源於舍衛城。

  「\twnr{喬達摩}{80.0}尊師!什麼因、什麼\twnr{緣}{180.0}……(中略)。」

  「婆蹉!以在色上的不通達……(中略)婆蹉!以在識上的不通達……(中略)。」



\sutta{26}{30}{色-不識別等經五則}{https://agama.buddhason.org/SN/sn.php?keyword=33.26}
  起源於舍衛城。

  「婆蹉!以在色上的不識別……(中略)婆蹉!以在識上的不識別……(中略)。」



\sutta{31}{35}{色-不辨別等經五則}{https://agama.buddhason.org/SN/sn.php?keyword=33.31}
  起源於舍衛城。

  「婆蹉!以在色上的不辨別……(中略)婆蹉!以在識上的不辨別……(中略)。」



\sutta{36}{40}{色-不精察等經五則}{https://agama.buddhason.org/SN/sn.php?keyword=33.36}
  起源於舍衛城。

  「婆蹉!以在色上的不精察……(中略)婆蹉!以在識上的不精察……(中略)。」



\sutta{41}{45}{色-不等察等經五則}{https://agama.buddhason.org/SN/sn.php?keyword=33.41}
  起源於舍衛城。

  「婆蹉!以在色上的不等察……(中略)婆蹉!以在識上的不等察……(中略)。」



\sutta{46}{50}{色-不觀察等經五則}{https://agama.buddhason.org/SN/sn.php?keyword=33.46}
  起源於舍衛城。

  「婆蹉!以在色上的不觀察……(中略)婆蹉!以在識上的不觀察……(中略)。」



\sutta{51}{54}{色-不領會等經四則}{https://agama.buddhason.org/SN/sn.php?keyword=33.51}
  起源於舍衛城。

  那時,\twnr{遊行者}{79.0}婆蹉氏去見\twnr{世尊}{12.0}。抵達後,與世尊一起互相問候。交換應該被互相問候的友好交談後,在一旁坐下。在一旁坐下的遊行者婆蹉氏對世尊說這個:

  「\twnr{喬達摩}{80.0}尊師!什麼是因,什麼是緣:凡在世間中這些種種\twnr{惡見}{722.0}生起:『世界是常恆的……(中略)或死後如來既非存在也非不存在?』」

  「婆蹉!以在色上的不領會,以在色\twnr{集}{67.0}上的不領會,以在色\twnr{滅}{68.0}上的不領會,以在導向色\twnr{滅道跡}{69.0}上的不領會……(中略)。」

  起源於舍衛城。

  「婆蹉!以在受上的不領會……(中略)以在導向受滅道跡上的不領會……(中略)。」

  起源於舍衛城。

  「婆蹉!以在想上的不領會……(中略)以在導向想滅道跡上的不領會……(中略)。」

  起源於舍衛城。

  「婆蹉!以在諸行上的不領會……(中略)以在導向行滅道跡上的不領會……(中略)。」



\sutta{55}{55}{識-不領會經}{https://agama.buddhason.org/SN/sn.php?keyword=33.55}
  起源於舍衛城。

  「婆蹉!以在識上的不領會,以在識\twnr{集}{67.0}上的不領會,以在識\twnr{滅}{68.0}上的不領會、以在導向識\twnr{滅道跡}{69.0}上的不領會,這樣,在世間中這些種種\twnr{惡見}{722.0}生起:『世界是常恆的,或世界是非常恆的……(中略)或死後如來既非存在也非不存在。』婆蹉!這是因,這是緣:凡在世間中這些種種惡見生起:『世界是常恆的,或\twnr{世界是非常恆的}{170.0},或世界是有邊的,或世界是無邊的,或命即是身體,或命是一,\twnr{身體是另一}{169.0},或死後如來存在,或死後如來不存在,或\twnr{死後如來存在且不存在}{354.0},或死後如來既非存在也非不存在。』」

  婆蹉氏相應完成,其\twnr{攝頌}{35.0}:

  「以無知連同以不見,以不現觀、以不隨覺,

   以不通達、以不識別,以不辨別、以不精察,

   以不等察、以不觀察,不領會。」





\page

\xiangying{34}{禪相應}
\sutta{1}{1}{根本定-等至經}{https://agama.buddhason.org/SN/sn.php?keyword=34.1}
  起源於舍衛城。

  「\twnr{比丘}{31.0}們!有這四種禪修者,哪四種?

  比丘們!這裡,某一類禪修者是在定上\twnr{定善巧者}{723.0},但不是在定上\twnr{等至善巧者}{724.0}。

  比丘們!又,這裡,某一類禪修者是在定上等至善巧者,但不是在定上定善巧者。

  比丘們!又,這裡,某一類禪修者既不是在定上定善巧者,也不是在定上等至善巧者。

  比丘們!又,這裡,某一類禪修者是在定上定善巧者與在定上等至善巧者。

  比丘們!在那裡,凡這位在定上定善巧者與在定上等至善巧者的禪修者,在這四類禪修者中是第一的、最上的、最勝的、最高的、最頂的。

  比丘們!猶如從牛有牛乳;從牛乳有凝乳;從凝乳有生酥;從生酥有\twnr{熟酥}{402.2};從熟酥有\twnr{熟酥醍醐}{361.2},在那裡,被告知為它們中第一的。同樣的,比丘們!凡這位在定上定善巧者與在定上等至善巧者的禪修者,在這四類禪修者中是第一的、最上的、最勝的、最高的、最頂的。」



\sutta{2}{2}{根本定-持續經}{https://agama.buddhason.org/SN/sn.php?keyword=34.2}
  起源於舍衛城。

  「\twnr{比丘}{31.0}們!有這四種禪修者,哪四種?

  比丘們!這裡,某一類禪修者是在定上\twnr{定善巧者}{723.0},但不是在定上\twnr{持續善巧者}{725.0}。

  比丘們!又,這裡,某一類禪修者是在定上持續善巧者,但不是在定上定善巧者。

  比丘們!又,這裡,某一類禪修者既不是在定上定善巧者,也不是在定上持續善巧者。

  比丘們!又,這裡,某一類禪修者是在定上定善巧者與在定上持續善巧者。

  比丘們!在那裡,凡這位在定上定善巧者與在定上持續善巧者的禪修者,在這四類禪修者中是第一的、最上的、最勝的、最高的、最頂的。

  比丘們!猶如從牛有牛乳;從牛乳有凝乳;從凝乳有生酥;從生酥有\twnr{熟酥}{402.2};從熟酥有\twnr{熟酥醍醐}{361.2},在那裡,被告知為它們中第一的。同樣的,比丘們!凡這位在定上定善巧者與在定上持續善巧者的禪修者,在這四類禪修者中是第一的、最上的、最勝的、最高的、最頂的。」



\sutta{3}{3}{根本定-出定經}{https://agama.buddhason.org/SN/sn.php?keyword=34.3}
  起源於舍衛城。

  「\twnr{比丘}{31.0}們!有這四種禪修者,哪四種?

  比丘們!這裡,某一類禪修者是在定上\twnr{定善巧者}{723.0},但不是在定上\twnr{出定善巧者}{726.0}。

  比丘們!又,這裡,某一類禪修者是在定上出定善巧者,但不是在定上定善巧者。

  比丘們!又,這裡,某一類禪修者既不是在定上定善巧者,也不是在定上出定善巧者。

  比丘們!又,這裡,某一類禪修者是在定上定善巧者與在定上出定善巧者。

  比丘們!在那裡,凡這位在定上定善巧者與在定上出定善巧者的禪修者,在這四類禪修者中是第一的、最上的、最勝的、最高的、最頂的。

  比丘們!猶如從牛有牛乳……(中略)最頂的。」



\sutta{4}{4}{根本定-順意經}{https://agama.buddhason.org/SN/sn.php?keyword=34.4}
  起源於舍衛城。

  「\twnr{比丘}{31.0}們!有這四種禪修者,哪四種?

  比丘們!這裡,某一類禪修者是在定上\twnr{定善巧者}{723.0},但不是在定上\twnr{順意善巧者}{727.0}。

  比丘們!又,這裡,某一類禪修者是在定上順意善巧者,但不是在定上定善巧者。

  比丘們!又,這裡,某一類禪修者既不是在定上定善巧者,也不是在定上順意善巧者。

  比丘們!又,這裡,某一類禪修者是在定上定善巧者與在定上順意善巧者。

  比丘們!在那裡,凡這位在定上定善巧者與在定上順意善巧者的禪修者,在這四類禪修者中是第一的、最上的、最勝的、最高的、最頂的。

  比丘們!猶如從牛有牛乳……(中略)最頂的。」



\sutta{5}{5}{根本定-所緣經}{https://agama.buddhason.org/SN/sn.php?keyword=34.5}
  起源於舍衛城。

  「\twnr{比丘}{31.0}們!有這四種禪修者,哪四種?

  比丘們!這裡,某一類禪修者是在定上\twnr{定善巧者}{723.0},但不是在定上\twnr{所緣善巧者}{728.0}。

  比丘們!又,這裡,某一類禪修者是在定上所緣善巧者,但不是在定上定善巧者。

  比丘們!又,這裡,某一類禪修者既不是在定上定善巧者,也不是在定上所緣善巧者。

  比丘們!又,這裡,某一類禪修者是在定上定善巧者與在定上所緣善巧者。

  比丘們!在那裡,凡這位在定上定善巧者與在定上所緣善巧者的禪修者,在這四類禪修者中是第一的、最上的、最勝的、最高的、最頂的。

  比丘們!猶如從牛有牛乳……(中略)最頂的。」



\sutta{6}{6}{根本定-行境經}{https://agama.buddhason.org/SN/sn.php?keyword=34.6}
  起源於舍衛城。

  「\twnr{比丘}{31.0}們!有這四種禪修者,哪四種?

  比丘們!這裡,某一類禪修者是在定上\twnr{定善巧者}{723.0},但不是在定上\twnr{行境善巧者}{729.0}。

  比丘們!又,這裡,某一類禪修者是在定上行境善巧者,但不是在定上定善巧者。

  比丘們!又,這裡,某一類禪修者既不是在定上定善巧者,也不是在定上行境善巧者。

  比丘們!又,這裡,某一類禪修者是在定上定善巧者與在定上行境善巧者。

  比丘們!在那裡,凡這位在定上定善巧者與在定上行境善巧者的禪修者,在這四類禪修者中是第一的、最上的、最勝的、最高的、最頂的。

  比丘們!猶如從牛有牛乳……(中略)最頂的。」



\sutta{7}{7}{根本定-決意經}{https://agama.buddhason.org/SN/sn.php?keyword=34.7}
  起源於舍衛城。

  「\twnr{比丘}{31.0}們!有這四種禪修者,哪四種?

  比丘們!這裡,某一類禪修者是在定上\twnr{定善巧者}{723.0},但不是在定上\twnr{決意善巧者}{730.0}。

  比丘們!又,這裡,某一類禪修者是在定上決意善巧者,但不是在定上定善巧者。

  比丘們!又,這裡,某一類禪修者既不是在定上定善巧者,也不是在定上決意善巧者。

  比丘們!又,這裡,某一類禪修者是在定上定善巧者與在定上決意善巧者。

  比丘們!在那裡,凡這位在定上定善巧者與在定上決意善巧者的禪修者,在這四類禪修者中是第一的、最上的、最勝的、最高的、最頂的。

  比丘們!猶如從牛有牛乳……(中略)最頂的。」



\sutta{8}{8}{根本定-恭敬作者經}{https://agama.buddhason.org/SN/sn.php?keyword=34.8}
  起源於舍衛城。

  「\twnr{比丘}{31.0}們!有這四種禪修者,哪四種?

  比丘們!這裡,某一類禪修者是在定上\twnr{定善巧者}{723.0},但不是在定上\twnr{恭敬作者}{731.0}。

  比丘們!又,這裡,某一類禪修者是在定上恭敬作者,但不是在定上定善巧者。

  比丘們!又,這裡,某一類禪修者既不是在定上定善巧者,也不是在定上恭敬作者。

  比丘們!又,這裡,某一類禪修者是在定上定善巧者也是在定上恭敬作者。

  比丘們!在那裡,凡這位在定上定善巧者與在定上恭敬作者的禪修者,在這四類禪修者中是第一的、最上的、最勝的、最高的、最頂的。

  比丘們!猶如從牛有牛乳……(中略)最頂的。」



\sutta{9}{9}{根本定-常作者經}{https://agama.buddhason.org/SN/sn.php?keyword=34.9}
  起源於舍衛城。

  「\twnr{比丘}{31.0}們!有這四種禪修者,哪四種?

  比丘們!這裡,某一類禪修者是在定上\twnr{定善巧者}{723.0},但不是在定上\twnr{常作者}{732.0}。

  比丘們!又,這裡,某一類禪修者是在定上常作者,但不是在定上定善巧者。

  比丘們!又,這裡,某一類禪修者既不是在定上定善巧者,也不是在定上常作者。

  比丘們!又,這裡,某一類禪修者是在定上定善巧者,也是在定上常作者。

  比丘們!在那裡,凡這位在定上定善巧者與在定上常作者的禪修者,在這四類禪修者中是第一的、最上的、最勝的、最高的、最頂的。

  比丘們!猶如從牛有牛乳……(中略)最頂的。」



\sutta{10}{10}{根本定-適當作者經}{https://agama.buddhason.org/SN/sn.php?keyword=34.10}
  起源於舍衛城。

  「\twnr{比丘}{31.0}們!有這四種禪修者,哪四種?

  比丘們!這裡,某一類禪修者是在定上\twnr{定善巧者}{723.0},但不是在定上\twnr{適當作者}{733.0}。

  比丘們!又,這裡,某一類禪修者是在定上適當作者,但不是在定上定善巧者。

  比丘們!又,這裡,某一類禪修者既不是在定上定善巧者,也不是在定上適當作者。

  比丘們!又,這裡,某一類禪修者是在定上定善巧者,也是在定上適當作者。

  比丘們!在那裡,凡這位在定上定善巧者與在定上適當作者的禪修者,在這四類禪修者中是第一的、最上的、最勝的、最高的、最頂的。

  比丘們!猶如從牛有牛乳……(中略)最頂的。」(根本定)



\sutta{11}{11}{根本等至-持續經}{https://agama.buddhason.org/SN/sn.php?keyword=34.11}
  起源於舍衛城。

  「\twnr{比丘}{31.0}們!有這四種禪修者,哪四種?

  比丘們!這裡,某一類禪修者是在定上\twnr{等至善巧者}{724.0},但不是在定上\twnr{持續善巧者}{725.0}。

  比丘們!又,這裡,某一類禪修者是在定上持續善巧者,但不是在定上等至善巧者。

  比丘們!又,這裡,某一類禪修者既不是在定上等至善巧者,也不是在定上持續善巧者。

  比丘們!又,這裡,某一類禪修者是在定上等至善巧者與在定上持續善巧者。

  比丘們!在那裡,凡這位在定上等至善巧者與在定上持續善巧者的禪修者,在這四類禪修者中是第一的、最上的、最勝的、最高的、最頂的。

  比丘們!猶如從牛有牛乳……(中略)最頂的。」



\sutta{12}{12}{根本等至-出定經}{https://agama.buddhason.org/SN/sn.php?keyword=34.12}
  起源於舍衛城。

  「\twnr{比丘}{31.0}們!有這四種禪修者,哪四種?

  比丘們!這裡,某一類禪修者是在定上\twnr{等至善巧者}{724.0},但不是在定上\twnr{出定善巧者}{726.0}。

  比丘們!又,這裡,某一類禪修者是在定上出定善巧者,但不是在定上等至善巧者。

  比丘們!又,這裡,某一類禪修者既不是在定上等至善巧者,也不是在定上出定善巧者。

  比丘們!又,這裡,某一類禪修者是在定上等至善巧者與在定上出定善巧者。

  比丘們!在那裡,凡這位……禪修者……(中略)最高的、最頂的。」



\sutta{13}{13}{根本等至-順意經}{https://agama.buddhason.org/SN/sn.php?keyword=34.13}
  起源於舍衛城。

  「\twnr{比丘}{31.0}們!有這四種禪修者,哪四種?

  比丘們!這裡,某一類禪修者是在定上\twnr{等至善巧者}{724.0},但不是在定上\twnr{順意善巧者}{727.0}。

  比丘們!又,這裡,某一類禪修者是在定上順意善巧者,但不是在定上等至善巧者。

  比丘們!又,這裡,某一類禪修者既不是在定上等至善巧者,也不是在定上順意善巧者。

  比丘們!又,這裡,某一類禪修者是在定上等至善巧者與在定上順意善巧者。

  比丘們!在那裡……(中略)最頂的。」



\sutta{14}{14}{根本等至-所緣經}{https://agama.buddhason.org/SN/sn.php?keyword=34.14}
  起源於舍衛城。

  「\twnr{比丘}{31.0}們!有這四種禪修者,哪四種?

  比丘們!這裡,某一類禪修者是在定上\twnr{等至善巧者}{724.0},但不是在定上\twnr{所緣善巧者}{728.0}。

  比丘們!又,這裡,某一類禪修者是在定上所緣善巧者,但不是在定上等至善巧者。

  比丘們!又,這裡,某一類禪修者既不是在定上等至善巧者,也不是在定上所緣善巧者。

  比丘們!又,這裡,某一類禪修者是在定上等至善巧者與在定上所緣善巧者。

  比丘們!在那裡……(中略)最頂的。」



\sutta{15}{15}{根本等至-行境經}{https://agama.buddhason.org/SN/sn.php?keyword=34.15}
  起源於舍衛城。

  「\twnr{比丘}{31.0}們!有這四種禪修者,哪四種?

  比丘們!這裡,某一類禪修者是在定上\twnr{等至善巧者}{724.0},但不是在定上\twnr{行境善巧者}{729.0}。

  比丘們!又,這裡,某一類禪修者是在定上行境善巧者,但不是在定上等至善巧者。

  比丘們!又,這裡,某一類禪修者既不是在定上等至善巧者,也不是在定上行境善巧者。

  比丘們!又,這裡,某一類禪修者是在定上等至善巧者與在定上行境善巧者。

  比丘們!在那裡……(中略)最頂的。」



\sutta{16}{16}{根本等至-決意經}{https://agama.buddhason.org/SN/sn.php?keyword=34.16}
  起源於舍衛城。

  「\twnr{比丘}{31.0}們!有這四種禪修者,哪四種?

  比丘們!這裡,某一類禪修者是在定上\twnr{等至善巧者}{724.0},但不是在定上\twnr{決意善巧者}{730.0}。

  比丘們!又,這裡,某一類禪修者是在定上決意善巧者,但不是在定上等至善巧者。

  比丘們!又,這裡,某一類禪修者既不是在定上等至善巧者,也不是在定上決意善巧者。

  比丘們!又,這裡,某一類禪修者是在定上等至善巧者與在定上決意善巧者。

  比丘們!在那裡……(中略)最頂的。」



\sutta{17}{17}{根本等至-恭敬作者經}{https://agama.buddhason.org/SN/sn.php?keyword=34.17}
  起源於舍衛城。

  「\twnr{比丘}{31.0}們!有這四種禪修者,哪四種?

  比丘們!這裡,某一類禪修者是在定上\twnr{等至善巧者}{724.0},但不是在定上\twnr{恭敬作者}{731.0}。

  比丘們!又,這裡,某一類禪修者是在定上恭敬作者,但不是在定上等至善巧者。

  比丘們!又,這裡,某一類禪修者既不是在定上等至善巧者,也不是在定上恭敬作者。

  比丘們!又,這裡,某一類禪修者是在定上等至善巧者也是在定上恭敬作者。

  比丘們!在那裡……(中略)最頂的。」



\sutta{18}{18}{根本等至-常作者經}{https://agama.buddhason.org/SN/sn.php?keyword=34.18}
  起源於舍衛城。

  「\twnr{比丘}{31.0}們!有這四種禪修者,哪四種?

  比丘們!這裡,某一類禪修者是在定上\twnr{等至善巧者}{724.0},但不是在定上\twnr{常作者}{732.0}。

  比丘們!又,這裡,某一類禪修者是在定上常作者,但不是在定上等至善巧者。

  比丘們!又,這裡,某一類禪修者既不是在定上等至善巧者,也不是在定上常作者。

  比丘們!又,這裡,某一類禪修者是在定上等至善巧者,也是在定上常作者。

  比丘們!在那裡……(中略)最頂的。」



\sutta{19}{19}{根本等至-適當作者經}{https://agama.buddhason.org/SN/sn.php?keyword=34.19}
  起源於舍衛城。

  「\twnr{比丘}{31.0}們!有這四種禪修者,哪四種?

  比丘們!這裡,某一類禪修者是在定上\twnr{等至善巧者}{724.0},但不是在定上\twnr{適當作者}{733.0}。

  比丘們!又,這裡,某一類禪修者是在定上適當作者,但不是在定上等至善巧者。

  比丘們!又,這裡,某一類禪修者既不是在定上等至善巧者,也不是在定上適當作者。

  比丘們!又,這裡,某一類禪修者是在定上等至善巧者,也是在定上適當作者。

  比丘們!在那裡,凡這位在定上等至善巧者與在定上適當作者的禪修者,在這四類禪修者中是第一的、最上的、最勝的、最高的、最頂的。

  比丘們!猶如從牛有牛乳;從牛乳有凝乳;從凝乳有生酥;從生酥有\twnr{熟酥}{402.2};從熟酥有\twnr{熟酥醍醐}{361.2},在那裡,被告知為它們中第一的。同樣的,比丘們!凡這位在定上等至善巧者與在定上適當作者的禪修者,在這四類禪修者中是第一的、最上的、最勝的、最高的、最頂的。」(根本等至)



\sutta{20}{27}{根本持續-出定經等八則}{https://agama.buddhason.org/SN/sn.php?keyword=34.20}
  起源於舍衛城。

  「\twnr{比丘}{31.0}們!有這四種禪修者,哪四種?

  比丘們!這裡,某一類禪修者是在定上\twnr{持續善巧者}{725.0},但不是在定上\twnr{出定善巧者}{726.0}。

  比丘們!又,這裡,某一類禪修者是在定上出定善巧者,但不是在定上持續善巧者。

  比丘們!又,這裡,某一類禪修者既不是在定上持續善巧者,也不是在定上出定善巧者。

  比丘們!又,這裡,某一類禪修者是在定上持續善巧者與在定上出定善巧者。

  比丘們!在那裡,凡這位……禪修者……(中略)最高的、最頂的。」

  (應該如前面根本[等至]到第二十七經根本持續的適當作者使八經完成)(根本持續)



\sutta{28}{34}{根本出定-順意經等七則}{https://agama.buddhason.org/SN/sn.php?keyword=34.28}
  起源於舍衛城。

  「\twnr{比丘}{31.0}們!有這四種禪修者,哪四種?

  比丘們!這裡,某一類禪修者是在定上\twnr{出定善巧者}{726.0},但不是在定上\twnr{順意善巧者}{727.0}。……是在定上順意善巧者,但不是在定上出定善巧者。……既不是在定上順意善巧者,也不是在定上出定善巧者。……是在定上順意善巧者與在定上出定善巧者。

  比丘們!在那裡,凡這位……禪修者……(中略)最高的、最頂的。」

  (應該如前面根本[持續]到第三十四經根本出定的適當作者使七經完成)(根本出定)



\sutta{35}{40}{根本順意-所緣經等六則}{https://agama.buddhason.org/SN/sn.php?keyword=34.35}
  起源於舍衛城。

  「是在定上\twnr{順意善巧者}{727.0},但不是在定上\twnr{所緣善巧者}{728.0}。……是在定上所緣善巧者,但不是在定上順意善巧者。……既不是在定上順意善巧者,也不是在定上所緣善巧者。……是在定上順意善巧者與在定上所緣善巧者。

  \twnr{比丘}{31.0}們!在那裡,凡這位……禪修者……(中略)最高的、最頂的。」

  (應該如前面根本[出定]到第四十經根本順意的適當作者使六經完成)(根本順意)



\sutta{41}{45}{根本所緣-行境經等五則}{https://agama.buddhason.org/SN/sn.php?keyword=34.41}
  起源於舍衛城。

  「是在定上\twnr{所緣善巧者}{728.0},但不是在定上\twnr{行境善巧者}{729.0}。……是在定上行境善巧者,但不是在定上所緣善巧者。……既不是在定上所緣善巧者,也不是在定上行境善巧者。……是在定上所緣善巧者與在定上行境善巧者。

  \twnr{比丘}{31.0}們!在那裡,凡這位……禪修者……(中略)最高的、最頂的。」

  (應該如前面根本[順意]到第四十五經根本所緣的適當作者使五經完)(根本所緣)



\sutta{46}{49}{根本行境-決意經等四則}{https://agama.buddhason.org/SN/sn.php?keyword=34.46}
  起源於舍衛城。

  「是在定上\twnr{行境善巧者}{729.0},但不是在定上\twnr{決意善巧者}{730.0}。……是在定上決意善巧者,但不是在定上行境善巧者。……既不是在定上行境善巧者,也不是在定上決意善巧者。……是在定上行境善巧者與在定上決意善巧者。

  \twnr{比丘}{31.0}們!猶如從牛有牛乳;從牛乳有凝乳;從凝乳有生酥;從生酥有\twnr{熟酥}{402.2};從熟酥有\twnr{熟酥醍醐}{361.2},在那裡,被告知為它們中第一的。同樣的,比丘們!凡這位在定上行境善巧者與在定上決意善巧者的禪修者,在這四類禪修者中是……(中略)最高的、最頂的。」

  是在定上行境善巧者,但不是在定上\twnr{恭敬作者}{731.0}。……(中略)。(應該使之被細說)

  是在定上行境善巧者,但不是在定上\twnr{常作者}{732.0}。……(中略)。

  是在定上行境善巧者,但不是在定上\twnr{適當作者}{733.0}。……(中略)。

  (根本行境)



\sutta{50}{52}{根本決意-恭敬經等三則}{https://agama.buddhason.org/SN/sn.php?keyword=34.50}
  起源於舍衛城。

  「是在定上\twnr{決意善巧者}{730.0},但不是在定上\twnr{恭敬作者}{731.0}。……是在定上恭敬作者,但不是在定上決意善巧者。……既不是在定上決意善巧者,也不是在定上恭敬作者。……是在定上決意善巧者也是在定上恭敬作者。……\twnr{比丘}{31.0}們!在那裡,凡這位……禪修者……(中略)最高的、最頂的。」

  是在定上決意善巧者,但不是在定上\twnr{常作者}{732.0}。……(中略)。

  是在定上決意善巧者,但不是在定上\twnr{適當作者}{733.0}。……(中略)。

  (根本決意)



\sutta{53}{54}{根本恭敬-常作者經等二則}{https://agama.buddhason.org/SN/sn.php?keyword=34.53}
  起源於舍衛城。

  「他是在定上\twnr{恭敬作者}{731.0},但不是在定上\twnr{常作者}{732.0}。……他是在定上常作者,但不是在定上恭敬作者。……既不是在定上常作者,也不是在定上恭敬作者。……是在定上常作者與在定上恭敬作者。……\twnr{比丘}{31.0}們!在那裡,凡這位……(中略)最高的、最頂的。」 

  他是在定上恭敬作者,但不是在定上\twnr{適當作者}{733.0}。……(中略)。 



\sutta{55}{55}{根本常-適當作者經}{https://agama.buddhason.org/SN/sn.php?keyword=34.55}
  起源於舍衛城。

  「\twnr{比丘}{31.0}們!有這四種禪修者,哪四種?

  比丘們!這裡,某一類禪修者是在定上\twnr{常作者}{732.0},但不是在定上\twnr{適當作者}{733.0}。

  比丘們!又,這裡,某一類禪修者是在定上適當作者,但不是在定上常作者。

  比丘們!又,這裡,某一類禪修者既不是在定上常作者,也不是在定上適當作者。

  比丘們!又,這裡,某一類禪修者是在定上常作者,也是在定上適當作者。

  比丘們!在那裡,凡這位在定上常作者與在定上適當作者的禪修者,在這四類禪修者中是第一的、最上的、最勝的、最高的、最頂的。

  比丘們!猶如從牛有牛乳;從牛乳有凝乳;從凝乳有生酥;從生酥有\twnr{熟酥}{402.2};從熟酥有\twnr{熟酥醍醐}{361.2},在那裡,被告知為它們中第一的。同樣的,比丘們!凡這位在定上常作者與在定上適當作者的禪修者,在這四類禪修者中是第一的、最上的、最勝的、最高的、最頂的。」

  \twnr{世尊}{12.0}說這個,那些悅意的比丘歡喜世尊的所說。

  (有五十五則解說都應該像那樣使之被細說)

  禪相應完成,其\twnr{攝頌}{35.0}:

  「定、等至、持續,出定、順意、所緣,

   行境、決意、恭敬,常又適當。」

  蘊篇第三,其\twnr{攝頌}{35.0}:

  「蘊、羅陀相應,見、入、生,

   雜染與舍利弗,龍、金翅鳥、乾達婆,

   雲、婆蹉、禪,蘊篇有十三。」

  蘊篇相應經典終了。





\page

\pian{六處篇}{35}{44}
\xiangying{35}{六處相應}
\pin{無常品}{1}{12}
\sutta{1}{1}{自身內無常經}{https://agama.buddhason.org/SN/sn.php?keyword=35.1}
  \twnr{被我這麼聽聞}{1.0}:

  \twnr{有一次}{2.0},\twnr{世尊}{12.0}住在舍衛城祇樹林給孤獨園。

  在那裡,世尊召喚\twnr{比丘}{31.0}們:「比丘們!」

  「\twnr{尊師}{480.0}!」那些比丘回答世尊。

  世尊說這個:

  「比丘們!眼是無常的,凡是無常的,那個是苦的,凡是苦的,那個是無我,凡是無我,那個:『\twnr{這不是我的}{32.1},\twnr{我不是這個}{33.1},\twnr{這不是我的真我}{34.2}。』這樣,這個應該以正確之慧如實被看見。

  耳是無常的,凡是無常的……(中略)鼻是無常的,凡是無常的……(中略)舌是無常的,凡是無常的,那個是苦的,凡是苦的,那個是無我,凡是無我,那個:『這不是我的,我不是這個,這不是我的真我。』這樣,這個應該以正確之慧如實被看見。身是無常的,凡是無常的……(中略)意是無常的,凡是無常的,那個是苦的,凡是苦的,那個是無我,凡是無我,那個:『這不是我的,我不是這個,這不是我的真我。』這樣,這個應該以正確之慧如實被看見。

  比丘們!這麼看的\twnr{有聽聞的聖弟子}{24.0}在眼上\twnr{厭}{15.0},也在耳上厭,在鼻上厭,在舌上厭,在身上厭,也在意上厭。厭者\twnr{離染}{558.0},從\twnr{離貪}{77.0}被解脫,在已解脫時,\twnr{有『[這是]解脫』之智}{27.0},他知道:『\twnr{出生已盡}{18.0},\twnr{梵行已完成}{19.0},\twnr{應該被作的已作}{20.0},\twnr{不再有此處[輪迴]的狀態}{21.1}。』」



\sutta{2}{2}{自身內苦經}{https://agama.buddhason.org/SN/sn.php?keyword=35.2}
  「\twnr{比丘}{31.0}們!眼是苦的,凡是苦的,那個是無我,凡是無我,那個:『\twnr{這不是我的}{32.1},\twnr{我不是這個}{33.1},\twnr{這不是我的真我}{34.2}。』這樣,這個應該以正確之慧如實被看見。

  耳是苦的……(中略)鼻是苦的……舌是苦的……身是苦的……意是苦的,凡是苦的,那個是無我,凡是無我,那個:『這不是我的,我不是這個,這不是我的真我。』這樣,這個應該以正確之慧如實被看見。這麼看的……(中略)他知道:『……\twnr{不再有此處[輪迴]的狀態}{21.1}。』」



\sutta{3}{3}{自身內無我經}{https://agama.buddhason.org/SN/sn.php?keyword=35.3}
  「\twnr{比丘}{31.0}們!眼是無我,凡是無我,那個:『\twnr{這不是我的}{32.1},\twnr{我不是這個}{33.1},\twnr{這不是我的真我}{34.2}。』這樣,這個應該以正確之慧如實被看見。

  耳是無我……(中略)鼻是無我……舌是無我……身是無我……意是無我,凡是無我,那個:『這不是我的,我不是這個,這不是我的真我。』這樣,這個應該以正確之慧如實被看見。這麼看的……(中略)他知道:『……\twnr{不再有此處[輪迴]的狀態}{21.1}。』」



\sutta{4}{4}{外部無常經}{https://agama.buddhason.org/SN/sn.php?keyword=35.4}
  「\twnr{比丘}{31.0}們!諸色是無常的,凡是無常的,那個是苦的,凡是苦的,那個是無我,凡是無我,那個:『\twnr{這不是我的}{32.1},\twnr{我不是這個}{33.1},\twnr{這不是我的真我}{34.2}。』這樣,這個應該以正確之慧如實被看見。

  諸聲音……諸氣味……諸味道……諸\twnr{所觸}{220.2}……諸法是無常的,凡是無常的,那個是苦的,凡是苦的,那個是無我,凡是無我,那個:『這不是我的,我不是這個,這不是我的真我。』這樣,這個應該以正確之慧如實被看見。

  比丘們!這麼看的\twnr{有聽聞的聖弟子}{24.0}在諸色上\twnr{厭}{15.0},在諸聲音上也厭,在諸氣味上也厭,在諸味道上也厭,在諸所觸上也厭,在諸法上也厭。厭者\twnr{離染}{558.0},從\twnr{離貪}{77.0}被解脫,在已解脫時,\twnr{有『[這是]解脫』之智}{27.0},他知道:『\twnr{出生已盡}{18.0},\twnr{梵行已完成}{19.0},\twnr{應該被作的已作}{20.0},\twnr{不再有此處[輪迴]的狀態}{21.1}。』」



\sutta{5}{5}{外部苦經}{https://agama.buddhason.org/SN/sn.php?keyword=35.5}
  「\twnr{比丘}{31.0}們!諸色是苦的,凡是苦的,那是無我,凡是無我,那個:『\twnr{這不是我的}{32.1},\twnr{我不是這個}{33.1},\twnr{這不是我的真我}{34.2}。』這樣,這個應該以正確之慧如實被看見。

  諸聲音……諸氣味……諸味道……諸\twnr{所觸}{220.2}……諸法是苦的,凡是苦的,那是無我,凡是無我,那個:『這不是我的,我不是這個,這不是我的真我。』這樣,這個應該以正確之慧如實被看見。這麼看的……(中略)他知道:『……\twnr{不再有此處[輪迴]的狀態}{21.1}。』」



\sutta{6}{6}{外部無我經}{https://agama.buddhason.org/SN/sn.php?keyword=35.6}
  「\twnr{比丘}{31.0}們!諸色是無我,凡是無我,那個:『\twnr{這不是我的}{32.1},\twnr{我不是這個}{33.1},\twnr{這不是我的真我}{34.2}。』這樣,這個應該以正確之慧如實被看見。

  諸聲音……諸氣味……諸味道……諸\twnr{所觸}{220.2}……諸法是無我,凡是無我,那個:『這不是我的,我不是這個,這不是我的真我。』這樣,這個應該以正確之慧如實被看見。這麼看的……(中略)他知道:『……\twnr{不再有此處[輪迴]的狀態}{21.1}。』」



\sutta{7}{7}{自身內的過去未來無常經}{https://agama.buddhason.org/SN/sn.php?keyword=35.7}
  「\twnr{比丘}{31.0}們!過去、未來的眼是無常的,更不用說現在!

  比丘們!這麼看的\twnr{有聽聞的聖弟子}{24.0}在過去眼上是無期待者,他不歡喜未來眼,對現在眼是為了\twnr{厭}{15.0}、\twnr{離貪}{77.0}、\twnr{滅的行者}{519.0}。

  過去、未來的耳是無常的……過去、未來的鼻是無常的……過去、未來的舌是無常的,更不用說現在!

  比丘們!這麼看的有聽聞的聖弟子在過去舌上是無期待者,他不歡喜未來舌,對現在舌是為了\twnr{厭}{15.0}、\twnr{離貪}{77.0}、\twnr{滅}{68.0}的行者。

  過去、未來的身是無常的……(中略)過去、未來的意是無常的,更不用說現在!

  比丘們!這麼看的有聽聞的聖弟子在過去意上是無期待者,他不歡喜未來意,對現在意是為了\twnr{厭}{15.0}、\twnr{離貪}{77.0}、滅的行者。」



\sutta{8}{8}{自身內的過去未來苦經}{https://agama.buddhason.org/SN/sn.php?keyword=35.8}
  「\twnr{比丘}{31.0}們!過去、未來的眼是苦的,更不用說現在!

  比丘們!這麼看的\twnr{有聽聞的聖弟子}{24.0}在過去眼上是無期待者,他不歡喜未來眼,對現在眼是為了\twnr{厭}{15.0}、\twnr{離貪}{77.0}、\twnr{滅的行者}{519.0}。

  過去、未來的耳是苦的……(中略)過去、未來的鼻是苦的……(中略)過去、未來的舌是苦的,更不用說現在!

  比丘們!這麼看的有聽聞的聖弟子在過去舌上是無期待者,他不歡喜未來舌,對現在舌是為了\twnr{厭}{15.0}、\twnr{離貪}{77.0}、\twnr{滅}{68.0}的行者。

  過去、未來的身是苦的……(中略)過去、未來的意是苦的,更不用說現在!

  比丘們!這麼看的有聽聞的聖弟子在過去意上是無期待者,他不歡喜未來意,對現在意是為了\twnr{厭}{15.0}、\twnr{離貪}{77.0}、滅的行者。」



\sutta{9}{9}{自身內的過去未來無我經}{https://agama.buddhason.org/SN/sn.php?keyword=35.9}
  「\twnr{比丘}{31.0}們!過去、未來的眼是無我,更不用說現在!

  比丘們!這麼看的\twnr{有聽聞的聖弟子}{24.0}在過去眼上是無期待者,他不歡喜未來眼,對現在眼是為了\twnr{厭}{15.0}、\twnr{離貪}{77.0}、\twnr{滅的行者}{519.0}。

  過去、未來的耳是無我……(中略)過去、未來的鼻是無我……(中略)過去、未來的舌是無我,更不用說現在!

  比丘們!這麼看的有聽聞的聖弟子在過去舌上是無期待者,他不歡喜未來舌,對現在舌是為了\twnr{厭}{15.0}、\twnr{離貪}{77.0}、\twnr{滅}{68.0}的行者。

  過去、未來的身是無我……(中略)過去、未來的意是無我,更不用說現在!

  比丘們!這麼看的有聽聞的聖弟子在過去意上是無期待者,他不歡喜未來意,對現在意是為了厭、離貪、滅的行者。」



\sutta{10}{10}{外部的過去未來無常經}{https://agama.buddhason.org/SN/sn.php?keyword=35.10}
  「\twnr{比丘}{31.0}們!過去、未來諸色是無常的,更不用說現在!

  比丘們!這麼看的\twnr{有聽聞的聖弟子}{24.0}在過去諸色上是無期待者,在未來諸色上不歡喜,對現在諸色是為了\twnr{厭}{15.0}、\twnr{離貪}{77.0}、\twnr{滅的行者}{519.0}。

  諸聲音……諸氣味……諸味道……諸\twnr{所觸}{220.2}……過去、未來諸法是無常的,更不用說現在!

  比丘們!這麼看的有聽聞的聖弟子在過去諸法上是無期待者,在未來諸法不歡喜,對現在諸法是為了厭、離貪、滅的行者。」



\sutta{11}{11}{外部的過去未來苦經}{https://agama.buddhason.org/SN/sn.php?keyword=35.11}
  「\twnr{比丘}{31.0}們!過去、未來諸色是苦的,更不用說現在!

  比丘們!這麼看的\twnr{有聽聞的聖弟子}{24.0}在過去諸色上是無期待者,在未來諸色上不歡喜,對現在諸色是為了\twnr{厭}{15.0}、\twnr{離貪}{77.0}、\twnr{滅的行者}{519.0}。……(中略)。」



\sutta{12}{12}{外部的過去未來無我經}{https://agama.buddhason.org/SN/sn.php?keyword=35.12}
  「\twnr{比丘}{31.0}們!過去、未來諸色是無我,更不用說現在!

  比丘們!這麼看的\twnr{有聽聞的聖弟子}{24.0}在過去諸色上是無期待者,在未來諸色上不歡喜,對現在諸色是為了\twnr{厭}{15.0}、\twnr{離貪}{77.0}、\twnr{滅}{68.0}的修行者。

  諸聲音……諸氣味……諸味道……諸\twnr{所觸}{220.2}……過去、未來的諸法是無我,更不用說現在!

  比丘們!這麼看的有聽聞的聖弟子在過去諸法上是無期待者,在未來諸法不歡喜,對現在諸法是為了厭、離貪、滅的行者。」

  無常品第一,其\twnr{攝頌}{35.0}:

  「無常、苦與無我,自身內外三則,

   以凡無常三說,個個分自身內外。」





\pin{雙品}{13}{22}
\sutta{13}{13}{正覺以前經第一}{https://agama.buddhason.org/SN/sn.php?keyword=35.13}
  起源於舍衛城。  

  「\twnr{比丘}{31.0}們!當就在我\twnr{正覺}{185.1}以前,還是未\twnr{現正覺}{75.0}的\twnr{菩薩}{186.0}時想這個:『什麼是眼的\twnr{樂味}{295.0},什麼是\twnr{過患}{293.0},什麼是\twnr{出離}{294.0}?什麼是耳的……(中略)?什麼是鼻的……?什麼是舌的……?什麼是身的……什麼是意的樂味,什麼是過患,什麼是出離?』

  比丘們!那個我想這個:『凡\twnr{緣於}{252.0}眼生起樂、喜悅,這是眼的樂味;凡眼是無常的、苦的、\twnr{變易法}{70.0},這是眼的過患;凡在眼上意欲貪的調伏、意欲貪的捨斷,這是眼的出離。

  凡耳……(中略)凡鼻……(中略)凡緣於舌生起樂、喜悅,這是舌的樂味;凡舌是無常的、苦的、變易法,這是舌的過患;凡在舌上意欲貪的調伏、意欲貪的捨斷,這是舌的出離。凡身……(中略)凡緣於意生起樂、喜悅,這是意的樂味;凡意是無常的、苦的、變易法,這是意的過患;凡在意上意欲貪的調伏、意欲貪的捨斷,這是意的出離。』

  比丘們!只要我不這樣如實證知這六內處的樂味為樂味、過患為過患、出離為出離,比丘們!我在包括天,在包括魔,在包括梵的世間;在包括沙門婆羅門,在包括天-人的\twnr{世代}{38.0}中,就不自稱『\twnr{已現正覺}{75.0}\twnr{無上遍正覺}{37.0}』。

  比丘們!但當我這樣如實證知這六內處的樂味為樂味、過患為過患、出離為出離,比丘們!那時,我在包括天,在包括魔,在包括梵的世間;在包括沙門婆羅門,在包括天-人的世代中,才自稱『已現正覺無上遍正覺』。而且,我的\twnr{智與見}{433.0}生起:『我的解脫是不動搖的,這是最後的出生,現在,沒有\twnr{再有}{21.0}。』」



\sutta{14}{14}{正覺以前經第二}{https://agama.buddhason.org/SN/sn.php?keyword=35.14}
  「\twnr{比丘}{31.0}們!當就在我\twnr{正覺}{185.1}以前,還是未\twnr{現正覺}{75.0}的\twnr{菩薩}{186.0}時想這個:『什麼是諸色的\twnr{樂味}{295.0}、什麼是\twnr{過患}{293.0}、什麼是\twnr{出離}{294.0}?什麼是諸聲音的……(中略)?什麼是諸氣味的……?什麼是諸味道的……?什麼是諸\twnr{所觸}{220.2}的……什麼是諸法的樂味、什麼是過患、什麼是出離?』

  比丘們!那個我想這個:『凡\twnr{緣於}{252.0}諸色生起樂、喜悅,這是諸色的樂味;凡諸色是無常的、苦的、\twnr{變易法}{70.0},這是諸色的過患;凡在諸色上意欲貪的調伏、意欲貪的捨斷,這是諸色的出離。

  凡諸聲音……諸氣味……諸味道……諸所觸……凡緣於諸法生起樂、喜悅,這是諸法的樂味;凡諸法是無常的、苦的、變易法,這是諸法的過患;凡在諸法上意欲貪的調伏、意欲貪的捨斷,這是諸法的出離。』

  比丘們!只要我不這樣如實證知這六外處的樂味為樂味、過患為過患、出離為出離,比丘們!我在包括天,在包括魔,在包括梵的世間;在包括沙門婆羅門,在包括天-人的\twnr{世代}{38.0}中,就不自稱『\twnr{已現正覺}{75.0}\twnr{無上遍正覺}{37.0}』。

  比丘們!但當我這樣如實證知這六外處的樂味為樂味、過患為過患、出離為出離,比丘們!那時,我在包括天,在包括魔,在包括梵的世間;在包括沙門婆羅門,在包括天-人的世代中,才自稱『已現正覺無上遍正覺』。而且,我的\twnr{智與見}{433.0}生起:『我的解脫是不動搖的,這是最後的出生,現在,沒有\twnr{再有}{21.0}。』」



\sutta{15}{15}{樂味的遍求經}{https://agama.buddhason.org/SN/sn.php?keyword=35.15}
  「\twnr{比丘}{31.0}們!\twnr{我曾行}{954.0}眼的\twnr{樂味}{295.0}之遍求,凡眼的樂味,我曾到達那個。眼的樂味之所及,那個被我以慧善見。

  比丘們!我曾行眼的\twnr{過患}{293.0}之遍求,凡眼的過患,我曾到達那個。眼的過患之所及,那個被我以慧善見。

  比丘們!我曾行眼的\twnr{出離}{294.0}之遍求,凡眼的出離,我曾到達那個。眼的出離之所及,那個被我以慧善見。

  比丘們!我[曾行]耳的……比丘們!我[曾行]鼻的……比丘們!我曾行舌的樂味之遍求,凡舌的樂味,我曾到達那個。舌的樂味之所及,那個被我以慧善見。比丘們!我曾行舌的過患之遍求,凡舌的過患,我曾到達那個。舌的過患之所及,那個被我以慧善見。比丘們!我曾行舌的出離之遍求,我曾到達那個。凡舌的出離,舌的出離之所及,那個被我以慧善見。……比丘們!我曾行意的樂味之遍求,凡意的樂味,我曾到達那個。意的樂味之所及,那個被我以慧善見。比丘們!我曾行意的過患之遍求,凡意的過患,我曾到達那個。意的過患之所及,那個被我以慧善見。比丘們!我曾行意的出離之遍求,凡意的出離,我曾到達那個。意的出離之所及,那個被我以慧善見。

  比丘們!只要我不如實證知這六內處的樂味為樂味、過患為過患、出離為出離……(中略)自稱……。而且,我的\twnr{智與見}{433.0}生起:『我的解脫是不動搖的,這是最後的出生,現在,沒有\twnr{再有}{21.0}。』」



\sutta{16}{16}{樂味的遍求經第二}{https://agama.buddhason.org/SN/sn.php?keyword=35.16}
  「\twnr{比丘}{31.0}們!\twnr{我曾行}{954.0}諸色的\twnr{樂味}{295.0}之遍求,凡諸色的樂味,我曾到達那個。諸色的樂味之所及,那個被我以慧善見。

  比丘們!我曾行諸色的\twnr{過患}{293.0}之遍求,凡諸色的過患,我曾到達那個。諸色的過患之所及,那個被我以慧善見。

  比丘們!我曾行諸諸色的\twnr{出離}{294.0}之遍求,凡諸色的出離,我曾到達那個。諸色的出離之所及,那個被我以慧善見。

  比丘們!我[曾行]諸聲音的……比丘們!我[曾行]諸氣味的……比丘們!我[曾行]諸味道的……比丘們!我[曾行]諸\twnr{所觸}{220.2}的……比丘們!我曾行諸法的樂味之遍求,凡諸法的樂味,我曾到達那個。諸法的樂味之所及,那個被我以慧善見。比丘們!我曾行諸法的過患之遍求,凡諸法的過患,我曾到達那個。諸法的過患之所及,那個被我以慧善見。比丘們!我曾行諸法的出離之遍求,凡諸法的出離,我曾到達那個。諸法的出離之所及,那個被我以慧善見。

  比丘們!只要我不如實證知這六外處的樂味為樂味、過患為過患、出離為出離……(中略)自稱……。而且,我的\twnr{智與見}{433.0}生起:『我的解脫是不動搖的,這是最後的出生,現在,沒有\twnr{再有}{21.0}。』」



\sutta{17}{17}{如果沒有樂味經第一}{https://agama.buddhason.org/SN/sn.php?keyword=35.17}
  「\twnr{比丘}{31.0}們!如果沒有眼的\twnr{樂味}{295.0},眾生不在眼上\twnr{貪著}{x297}。比丘們!但因為有眼的樂味,因此眾生在眼上貪著。

  比丘們!如果沒有眼的\twnr{過患}{293.0},眾生不在眼上\twnr{厭}{15.0}。比丘們!但因為有眼的過患,因此眾生在眼上厭。

  比丘們!如果沒有眼的\twnr{出離}{294.0},眾生不從眼出離。比丘們!但因為有眼的出離,因此眾生從眼出離。

  比丘們!如果沒有耳的樂味……比丘們!如果沒有鼻的樂味……。比丘們!如果沒有舌的樂味,眾生不在舌上貪著。比丘們!但因為有舌的樂味,因此眾生在舌上貪著。比丘們!如果沒有舌的過患,眾生不在舌上厭。比丘們!但因為有舌的過患,因此眾生在舌上厭。比丘們!如果沒有舌的出離,眾生不從舌出離。比丘們!但因為有舌的出離,因此眾生從舌出離。比丘們!如果沒有身的樂味……比丘們!如果沒有意的樂味,眾生不在意上貪著。比丘們!但因為有意的樂味,因此眾生在意上貪著。比丘們!如果沒有意的過患,眾生不在意上厭。比丘們!但因為有意的過患,因此眾生在意上厭。比丘們!如果沒有意的出離,眾生不從意出離。比丘們!但因為有意的出離,因此眾生從意出離。

  比丘們!只要眾生不如實證知這六內處的樂味為樂味、過患為過患、出離為出離,比丘們!這些眾生就還未從這包括天、魔、梵的世間;包括沙門婆羅門,包括天-人的\twnr{世代}{38.0}出離、離繫縛、脫離,\twnr{以離被限制之心}{555.0}而住。

  比丘們!但當眾生如實證知這六內處的樂味為樂味、過患為過患、出離為出離,比丘們!那時,眾生已從這包括天、魔、梵的世間;包括沙門婆羅門,包括天-人的世代中出離、離繫縛、脫離,以離被限制之心而住。」[≃\suttaref{SN.14.33}, \suttaref{SN.22.28}, \suttaref{SN.35.18}]



\sutta{18}{18}{如果沒有樂味經第二}{https://agama.buddhason.org/SN/sn.php?keyword=35.18}
  「\twnr{比丘}{31.0}們!如果沒有諸色的\twnr{樂味}{295.0},眾生不在諸色上貪著。比丘們!但因為有諸色的樂味,因此眾生在諸色上貪著。

  比丘們!如果沒有諸色的\twnr{過患}{293.0},眾生不在諸色上\twnr{厭}{15.0}。比丘們!但因為有諸色的過患,因此眾生在諸色上\twnr{厭}{15.0}。

  比丘們!如果沒有諸色的\twnr{出離}{294.0},眾生不從諸色上出離。比丘們!但因為有諸色的出離,因此眾生從諸色上出離。

  比丘們!如果沒有諸聲音的……諸氣味的……諸味道的……諸\twnr{所觸}{220.2}的……如果沒有諸法的樂味,眾生不在諸法上貪著。比丘們!但因為有諸法的樂味,因此眾生在諸法上貪著。比丘們!如果沒有諸法的過患,眾生不在諸法上厭。比丘們!但因為有諸法的過患,因此眾生在諸法上厭。比丘們!如果沒有諸法的出離,眾生不從諸法上出離。比丘們!但因為有諸法的出離,因此眾生從諸法上出離。

  比丘們!只要眾生不如實證知這六外處的樂味為樂味、過患為過患、出離為出離,比丘們!這些眾生就還未從這包括天、魔、梵的世間;包括沙門婆羅門,包括天-人的\twnr{世代}{38.0}出離、離繫縛、脫離,\twnr{以離被限制之心}{555.0}而住。

  比丘們!但當眾生如實證知這六外處的樂味為樂味、過患為過患、出離為出離,比丘們!那時,眾生已從這包括天、魔、梵的世間;包括沙門婆羅門,包括天-人的世代中出離、離繫縛、脫離,以離被限制之心而住。」[≃\suttaref{SN.14.13}, \suttaref{SN.22.28}, \suttaref{SN.35.17}]



\sutta{19}{19}{歡喜經第一}{https://agama.buddhason.org/SN/sn.php?keyword=35.19}
  「\twnr{比丘}{31.0}們!凡歡喜眼者,他歡喜苦;凡歡喜苦者,我說:『他不從苦被解脫。』

  耳……(中略)鼻……(中略)凡歡喜舌者,他歡喜苦;凡歡喜苦者,我說:『他不從苦被解脫。』身……(中略)凡歡喜意者,他歡喜苦;凡歡喜苦者,我說:『他不從苦被解脫。』

  比丘們!凡不歡喜眼者,他不歡喜苦;凡不歡喜苦者,我說:『他從苦被解脫。』

  耳……(中略)鼻……(中略)凡不歡喜舌者,他不歡喜苦;凡不歡喜苦者,我說:『他從苦被解脫。』身……(中略)凡不歡喜意者,他不歡喜苦;凡不歡喜苦者,我說:『他從苦被解脫。』」[≃\suttaref{SN.14.35}, \suttaref{SN.22.29}, \suttaref{SN.35.20}]



\sutta{20}{20}{歡喜經第二}{https://agama.buddhason.org/SN/sn.php?keyword=35.20}
  「\twnr{比丘}{31.0}們!凡歡喜諸色者,他歡喜苦;凡歡喜苦者,我說:『他不從苦被解脫。』諸聲音……(中略)諸氣味……(中略)諸味道……(中略)諸\twnr{所觸}{220.2}……(中略)凡歡喜諸法者,他歡喜苦;凡歡喜苦者,我說『他不從苦被解脫』。

  比丘們!凡不歡喜諸色者,他不歡喜苦;凡不歡喜苦者,我說:『他從苦被解脫。』諸聲音……(中略)諸氣味……(中略)諸味道……(中略)諸所觸……(中略)凡不歡喜諸法者,他不歡喜苦;凡不歡喜苦者,我說:『他從苦被解脫。』」[≃\suttaref{SN.14.35}, \suttaref{SN.22.29}, \suttaref{SN.35.19}]



\sutta{21}{21}{苦的生起經第一}{https://agama.buddhason.org/SN/sn.php?keyword=35.21}
  「\twnr{比丘}{31.0}們!凡眼的\twnr{生起}{x478}、存續、生出、顯現,這是苦的生起、諸病的存續、老死的顯現。

  凡耳的……(中略)凡鼻的……凡舌的……凡身的……凡意的生起、存續、生出、顯現,這是苦的生起、諸病的存續,老死的顯現。

  比丘們!而凡眼的\twnr{滅}{68.0}、平息、滅沒,這是苦的滅、諸病的平息、老死的滅沒。

  凡耳的……凡鼻的……凡舌的……凡身的……凡意的滅、平息、滅沒,這是苦的滅、諸病的平息、老死的滅沒。」[\suttaref{SN.26.1}]



\sutta{22}{22}{苦的生起經第二}{https://agama.buddhason.org/SN/sn.php?keyword=35.22}
  「\twnr{比丘}{31.0}們!凡諸色的\twnr{生起}{x478}、存續、生出、顯現,這是苦的生起、諸病的存續、老死的顯現。

  凡諸聲音的……凡諸氣味的……凡諸味道的……凡諸\twnr{所觸}{220.2}的……凡諸法的生起、存續、生出、顯現,這是苦的生起、諸病的存續,老死的顯現。

  比丘們!而凡諸色的\twnr{滅}{68.0}、平息、滅沒,這是苦的滅、諸病的平息、老死的滅沒。

  凡諸聲音的……凡諸氣味的……凡諸味道的……凡諸所觸的……凡諸法的滅、平息、滅沒,這是苦的滅、諸病的平息、老死的滅沒。」[\suttaref{SN.26.2}]

  雙第二,其\twnr{攝頌}{35.0}:

  「以正覺二說,以樂味二則在後,

   以如果沒有那個二說,以歡喜二則在後,

   以生起二說,以那個被稱為品。」





\pin{一切品}{23}{32}
\sutta{23}{23}{一切經}{https://agama.buddhason.org/SN/sn.php?keyword=35.23}
  起源於舍衛城。

  「\twnr{比丘}{31.0}們!我將為你們教導\twnr{一切}{x479},\twnr{你們要聽}{43.0}它!

  比丘們!而什麼是一切?

  即是眼與諸色、耳與諸聲、鼻與諸氣味、舌與諸味道、身與諸\twnr{所觸}{220.2}、意與諸法,比丘們!這被稱為一切。

  比丘們!凡如果這麼說:『拒絕這個一切後,我將\twnr{安立}{143.0}另一個一切。』他的言語會成為無根據的,而當被詢問時不會解答,且更會來到惱害,那是什麼原因?比丘們!那個正如\twnr{不在[感官的]境域中}{783.0}那樣。」[≃\suttaref{SN.35.92}]



\sutta{24}{24}{捨斷經}{https://agama.buddhason.org/SN/sn.php?keyword=35.24}
  「\twnr{比丘}{31.0}們!我將為你們教導一切的捨斷之法,\twnr{你們要聽}{43.0}它!

  比丘們!而什麼是一切的捨斷之法呢?

  比丘們!眼應該被捨斷,諸色應該被捨斷,眼識應該被捨斷,眼觸應該被捨斷,又凡以這眼觸\twnr{為緣}{180.0}生起感受的樂,或苦,或不苦不樂,那也應該被捨斷。

  ……(中略)又凡以這耳觸為緣生起感受的樂,或苦,或不苦不樂,那也應該被捨斷。……又凡以這鼻觸為緣生起感受的樂,或苦,或不苦不樂,那也應該被捨斷。舌應該被捨斷,諸味道應該被捨斷,舌識應該被捨斷,舌觸應該被捨斷,又凡以這舌觸為緣生起感受的樂,或苦,或不苦不樂,那也應該被捨斷。身應該被捨斷……意應該被捨斷,諸法應該被捨斷,意識應該被捨斷,意觸應該被捨斷,又凡以這意觸為緣生起感受的樂,或苦,或不苦不樂,那也應該被捨斷。

  比丘們!這是一切的捨斷之法。」



\sutta{25}{25}{以證知以遍知之捨斷經}{https://agama.buddhason.org/SN/sn.php?keyword=35.25}
  「\twnr{比丘}{31.0}們!我將為你們教導以\twnr{證知}{242.0}以\twnr{遍知}{154.0}(證知後遍知後)一切的捨斷之法,\twnr{你們要聽}{43.0}它!

  比丘們!而什麼是以證知以遍知一切的捨斷之法呢?

  比丘們!眼應該以證知以遍知被捨斷,諸色應該以證知以遍知被捨斷,眼識應該以證知以遍知被捨斷,眼觸應該以證知以遍知被捨斷,又凡以這眼觸\twnr{為緣}{180.0}生起感受的樂,或苦,或不苦不樂,那也應該以證知以遍知被捨斷。

  ……(中略)舌應該以證知以遍知被捨斷,諸味道應該以證知以遍知被捨斷,舌識應該以證知以遍知被捨斷,舌觸應該以證知以遍知被捨斷,又凡以這舌觸為緣生起感受的樂,或苦,或不苦不樂,那也應該以證知以遍知被捨斷。身應該以證知以遍知被捨斷……意應該以證知以遍知被捨斷,諸法應該以證知以遍知被捨斷,意識應該以證知以遍知被捨斷,意觸應該以證知以遍知被捨斷,又凡以這意觸為緣生起感受的樂,或苦,或不苦不樂,那也應該以證知以遍知被捨斷。

  比丘們!這是以證知以遍知一切的捨斷之法。」



\sutta{26}{26}{不遍知經第一}{https://agama.buddhason.org/SN/sn.php?keyword=35.26}
  「\twnr{比丘}{31.0}們!不\twnr{證知}{242.0}、不\twnr{遍知}{154.0}、\twnr{不離貪}{77.1}、不捨斷一切者是對苦滅盡的不可能者。

  比丘們!而不證知、不遍知、不離貪、不捨斷什麼是對苦滅盡的不可能者?

  比丘們!不證知、不遍知、不離貪、不捨斷眼者是對苦滅盡的不可能者。

  不證知、不遍知、不離貪、不捨斷諸色者是對苦滅盡的不可能者。

  不證知、不遍知、不離貪、不捨斷眼識者是對苦滅盡的不可能者。

  不證知、不遍知、不離貪、不捨斷眼觸者是對苦滅盡的不可能者。

  又凡以這眼觸\twnr{為緣}{180.0}生起感受的樂,或苦,或不苦不樂,也不證知、不遍知、不離貪、不捨斷那個者是對苦滅盡的不可能者。

  ……(中略)不證知、不遍知、不離貪、不捨斷舌者是對苦滅盡的不可能者。諸味道……(中略)舌識……(中略)舌觸……(中略)又凡以這舌觸為緣生起感受的樂,或苦,或不苦不樂,也不證知、不遍知、不離貪、不捨斷那個者是對苦滅盡的不可能者。身……(中略)不證知、不遍知、不離貪、不捨斷意者是對苦滅盡的不可能者。諸法……(中略)意識……(中略)意觸……(中略)又凡以這意觸為緣生起感受的樂,或苦,或不苦不樂,也不證知、不遍知、不離貪、不捨斷那個者是對苦滅盡的不可能者。

  比丘們!這是不證知、不遍知、不離貪、不捨斷一切者是對苦滅盡的不可能者。

  比丘們!這是證知、遍知、離貪、捨斷一切者是對苦滅盡的可能者。

  比丘們!而證知、遍知、離貪、捨斷什麼者是對苦滅盡的可能者?

  比丘們!證知、遍知、離貪、捨斷眼者是對苦滅盡的可能者。

  證知、遍知、離貪、捨斷諸色者是對苦滅盡的可能者。

  證知、遍知、離貪、捨斷眼識者是對苦滅盡的可能者。

  證知、遍知、離貪、捨斷眼觸者是對苦滅盡的可能者。

  又凡以這眼觸為緣生起感受的樂,或苦,或不苦不樂,也證知、遍知、離貪、捨斷那個者是對苦滅盡的可能者。

  ……(中略)證知、遍知、離貪、捨斷舌者是對苦滅盡的可能者。諸味道……(中略)舌識……(中略)舌觸……(中略)又凡以這舌觸為緣生起感受的樂,或苦,或不苦不樂,也證知、遍知、離貪、捨斷那個者是對苦滅盡的可能者。身……(中略)證知、遍知、離貪、捨斷意者是對苦滅盡的可能者。諸法……(中略)意識……(中略)意觸……(中略)又凡以這意觸為緣生起感受的樂,或苦,或不苦不樂,也證知、遍知、離貪、捨斷那個者是對苦滅盡的可能者。

  比丘們!這是證知、遍知、離貪、捨斷一切者是對苦滅盡的可能者。」[≃\suttaref{SN.22.24}, \suttaref{SN.35.27}, \suttaref{SN.35.111}, \suttaref{SN.35.112}]



\sutta{27}{27}{不遍知經第二}{https://agama.buddhason.org/SN/sn.php?keyword=35.27}
  「\twnr{比丘}{31.0}們!不\twnr{證知}{242.0}、不\twnr{遍知}{154.0}、\twnr{不離貪}{77.1}、不捨斷一切者是對苦滅盡的不可能者。

  比丘們!而不證知、不遍知、不離貪、不捨斷什麼是對苦滅盡的不可能者?

  比丘們!凡眼、凡諸色、凡眼識、凡能被眼識所識知的諸法……(中略)凡舌、凡諸味道、凡舌識、凡能被舌識所識知的諸法,凡身、凡諸\twnr{所觸}{220.2}、凡身識、凡能被身識所識知的諸法,凡意、凡諸法、凡意識、凡能被意識所識知的諸法,比丘們!這是不證知、不遍知、不離貪、不捨斷一切者是對苦滅盡的不可能者。

  比丘們!而證知、遍知、離貪、捨斷什麼者是對苦滅盡的可能者?

  比丘們!凡眼、凡諸色、凡眼識、凡能被眼識所識知的諸法……(中略)凡舌、凡諸味道、凡舌識、凡能被舌識所識知的諸法,凡身、凡諸所觸、凡身識、凡能被身識所識知的諸法,凡意、凡諸法、凡意識、凡能被意識所識知的諸法,比丘們!這是證知、遍知、離貪、捨斷一切者是對苦滅盡的可能者。」[≃\suttaref{SN.22.24}, \suttaref{SN.35.26}, \suttaref{SN.35.111}, \suttaref{SN.35.112}] 



\sutta{28}{28}{已燃燒經}{https://agama.buddhason.org/SN/sn.php?keyword=35.28}
  \twnr{有一次}{2.0},\twnr{世尊}{12.0}與千位\twnr{比丘}{31.0}同住在伽耶的伽耶山頂。

  在那裡,世尊召喚比丘們:

  「比丘們!一切已燃燒。比丘們!而什麼是一切已燃燒?比丘們!眼已燃燒,諸色已燃燒,眼識已燃燒,眼觸已燃燒,又凡以這眼觸\twnr{為緣}{180.0}生起感受的樂,或苦,或不苦不樂,那也已燃燒。以什麼已燃燒呢?我說『以貪火、瞋火、癡火已燃燒,以生、老、死、愁、悲、苦、憂、\twnr{絕望}{342.0}已燃燒』。……(中略)舌已燃燒,諸味道已燃燒,舌識已燃燒,舌觸已燃燒,又凡以這舌觸為緣生起感受的樂,或苦,或不苦不樂,那也已燃燒。以什麼已燃燒呢?我說『以貪火、瞋火、癡火燃燒,以生、老、死、愁、悲、苦、憂、絕望已燃燒』。……(中略)意已燃燒,諸法已燃燒,意識已燃燒,意觸已燃燒,又凡以這意觸為緣生起感受的樂,或苦,或不苦不樂,那也已燃燒。以什麼已燃燒呢?我說『以貪火、瞋火、癡火燃燒,以生、老、死、愁、悲、苦、憂、絕望已燃燒』。

  比丘們!這麼看的\twnr{有聽聞的聖弟子}{24.0}在眼上\twnr{厭}{15.0},也在諸色上厭,也在眼識上厭,也在眼觸上厭,又凡以這眼觸為緣生起感受的樂,或苦,或不苦不樂,在那個上也厭。……(中略)又凡以這意觸為緣生起感受的樂,或苦,或不苦不樂,在那個上也厭。厭者\twnr{離染}{558.0},從\twnr{離貪}{77.0}被解脫,在已解脫時,\twnr{有『[這是]解脫』之智}{27.0},他知道:『\twnr{出生已盡}{18.0},\twnr{梵行已完成}{19.0},\twnr{應該被作的已作}{20.0},\twnr{不再有此處[輪迴]的狀態}{21.1}。』」

  世尊說這個,那些悅意的比丘歡喜世尊的所說。

  還有,\twnr{在當這個解說被說時}{136.0},這千位不執取後比丘的心從諸\twnr{漏}{188.0}被解脫。



\sutta{29}{29}{被征服經}{https://agama.buddhason.org/SN/sn.php?keyword=35.29}
  \twnr{被我這麼聽聞}{1.0}:

  \twnr{有一次}{2.0},\twnr{世尊}{12.0}住在王舍城栗鼠飼養處的竹林中。

  在那裡,世尊召喚\twnr{比丘}{31.0}們:

  「比丘們!一切被征服,比丘們!而什麼是一切被征服?比丘們!眼被征服,諸色被征服,眼識被征服,眼觸被征服,又凡以這眼觸為緣生起感受的樂,或苦,或不苦不樂,那也被征服。被什麼征服呢?我說『被生、老、死、愁、悲、苦、憂、\twnr{絕望}{342.0}征服』。……(中略)舌被征服,諸味道被征服,舌識被征服,舌觸被征服,又凡以這舌觸為緣生起感受的樂,或苦,或不苦不樂,那也被征服。被什麼征服呢?我說『被生、老、死、愁、悲、苦、憂、絕望征服』。身被征服……(中略)意被征服,諸法被征服,意識被征服,意觸被征服,又凡以這意觸為緣生起感受的樂,或苦,或不苦不樂,那也被征服。以什麼征服呢?我說『被生、老、死、愁、悲、苦、憂、絕望被征服』。

  比丘們!這麼看的\twnr{有聽聞的聖弟子}{24.0}在眼上\twnr{厭}{15.0},也在諸色上厭,也在眼識上厭,也在眼觸上厭……(中略)又凡以這意觸為緣生起感受的樂,或苦,或不苦不樂,在那個上也厭。厭者\twnr{離染}{558.0},從\twnr{離貪}{77.0}被解脫,在已解脫時,\twnr{有『[這是]解脫』之智}{27.0},他知道:『\twnr{出生已盡}{18.0},\twnr{梵行已完成}{19.0},\twnr{應該被作的已作}{20.0},\twnr{不再有此處[輪迴]的狀態}{21.1}。』」



\sutta{30}{30}{適合根除的經}{https://agama.buddhason.org/SN/sn.php?keyword=35.30}
  「\twnr{比丘}{31.0}們!我將為你們教導一切\twnr{思量}{963.0}之適合根除的道跡,你們要聽它!你們\twnr{要好好作意}{43.1}!我將說。

  比丘們!而哪個是那個一切思量之適合根除的道跡?比丘們!這裡,比丘不思量眼,不在眼中思量,\twnr{不從眼思量}{966.0},不思量『眼是我的』。不思量諸色,不在諸色中思量,不從諸色思量,不思量『諸色是我的』。不思量眼識,不在眼識中思量,不從眼識思量,不思量『眼識是我的』。不思量眼觸,不在眼觸中思量,不從眼觸思量,不思量『眼觸是我的』。又凡以這眼觸為緣生起感受的樂,或苦,或不苦不樂,那也不思量,也不在其中思量,也不從其思量,也不思量『那是我的』。

  ……(中略)不思量舌,不在舌中思量,不從舌思量,不思量『舌是我的』。不思量諸味道,不在諸味道中思量,不從諸味道思量,不思量『諸味道是我的』。不思量舌識,不在舌識中思量,不從舌識思量,不思量『舌識是我的』。不思量舌觸,不在舌觸中思量,不從舌觸思量,不思量『舌觸是我的』。又凡以這舌觸為緣生起感受的樂,或苦,或不苦不樂,那也不思量,也不在其中思量,也不從其思量,也不思量『那是我的』。……(中略)不思量意,不在意中思量,不從意思量,不思量『意是我的』。不思量諸法,不在諸法中思量,不從諸法思量,不思量『諸法是我的』。不思量意識,不在意識中思量,不從意識思量,不思量『意識是我的』。不思量意觸,不在意觸中思量,不從意觸思量,不思量『意觸是我的』。又凡以這意觸為緣生起感受的樂,或苦,或不苦不樂,那也不思量,也不在其中思量,也不從其思量,也不思量『那是我的』。不思量一切,不在一切中思量,不從一切思量,不思量『一切是我的』。當這麼不思量時,不執取世間中任何事物。不執取者不\twnr{戰慄}{436.0},不戰慄者\twnr{就自己證涅槃}{71.0},他知道:『\twnr{出生已盡}{18.0},\twnr{梵行已完成}{19.0},\twnr{應該被作的已作}{20.0},\twnr{不再有此處[輪迴]的狀態}{21.1}。』

  比丘們!這是那個一切思量之適合根除的道跡。」 



\sutta{31}{31}{適當根除的經第一}{https://agama.buddhason.org/SN/sn.php?keyword=35.31}
  「\twnr{比丘}{31.0}們!我將為你們教導一切\twnr{思量}{963.0}之適當根除的道跡,你們要聽它!

  比丘們!而哪個是那個一切思量之適當根除的道跡?比丘們!這裡,比丘不思量眼,不在眼中思量,\twnr{不從眼思量}{966.0},不思量『眼是我的』。不思量諸色……(中略)不思量眼識……不思量眼觸……又凡以這眼觸為緣生起感受的樂,或苦,或不苦不樂,那也不思量,也不在其中思量,也不從其思量,也不思量『那是我的』。比丘們!因為,凡思量,凡在其中思量,凡從其思量,凡思量『那是我的』者,從那裡它相異地存在,\twnr{成為相異的}{x480}、執著存在的世間只歡喜存在。

  ……(中略)不思量舌,不在舌中思量,不從舌思量,不思量『舌是我的』。不思量諸味道……(中略)不思量舌識……不思量舌觸……又凡以這舌觸為緣生起感受的樂,或苦,或不苦不樂,那也不思量,也不在其中思量,也不從其思量,也不思量『那是我的』。比丘們!因為,凡思量,凡在其中思量,凡從其思量,凡思量『那是我的』者,從那裡它相異地存在,成為相異的、執著存在的世間只歡喜存在者。……(中略)不思量意,不在意中思量,不從意思量,不思量『意是我的』。不思量諸法……(中略)不思量意識……不思量意觸……又凡以這意觸為緣生起感受的樂,或苦,或不苦不樂,那也不思量,也不在其中思量,也不從其思量,也不思量『那是我的』。比丘們!因為,凡思量,凡在其中思量,凡從其思量,凡思量『那是我的』者,從那裡它相異地存在,成為相異的、執著存在的世間只歡喜存在者。

  比丘們!凡蘊、界、處之所及,他不思量,不在其中思量,不從其思量,不思量『那是我的』。當這麼不思量時,不執取世間中任何事物。不執取者不\twnr{戰慄}{436.0},不戰慄者\twnr{就自己證涅槃}{71.0},他知道:『\twnr{出生已盡}{18.0},\twnr{梵行已完成}{19.0},\twnr{應該被作的已作}{20.0},\twnr{不再有此處[輪迴]的狀態}{21.1}。』

  比丘們!這是那一切思量之適當根除的道跡。」 



\sutta{32}{32}{適當根除的經第二}{https://agama.buddhason.org/SN/sn.php?keyword=35.32}
  「\twnr{比丘}{31.0}們!我將為你們教導一切\twnr{思量}{963.0}之適當根除的道跡,你們要聽它!比丘們!而哪個是那個一切思量之適當根除的道跡?

  比丘們!你們怎麼想它:眼是常的,或是無常的?」

  「\twnr{大德}{45.0}!是無常的。」

  「那麼,凡為無常的,那是苦的或樂的?」

  「大德!是苦的。」

  「而凡為無常、苦、\twnr{變易法}{70.0},適合認為它:『\twnr{這是我的}{32.0},\twnr{我是這個}{33.0},\twnr{這是我的真我}{34.1}。』嗎?」

  「大德!這確實不是。」

  「諸色……(中略)眼識……眼觸是常的,或是無常的?」

  「大德!是無常的。」

  「又凡以這眼觸為緣生起感受的樂,或苦,或不苦不樂,那也是常的,或是無常的?」

  「大德!是無常的。」

  「那麼,凡為無常的,那是苦的或樂的?」

  「大德!是苦的。」

  「而凡為無常、苦、變易法,適合認為它:『\twnr{這是我的}{32.0},\twnr{我是這個}{33.0},這是\twnr{我的真我}{34.0}。』嗎?」

  「大德!這確實不是。」……(中略)

  「舌是常的,或是無常的?」

  「大德!是無常的。」

  「諸味道……舌識……舌觸……又凡以這舌觸為緣生起感受的樂,或苦,或不苦不樂,那也是常的,或是無常的?」

  「大德!是無常的。」……(中略)

  「諸法……意識……意觸是常的,或是無常的?」

  「大德!是無常的。」

  「又凡以這意觸為緣生起感受的樂,或苦,或不苦不樂,那也是常的,或是無常的?」

  「大德!是無常的。」

  「那麼,凡為無常的,那是苦的或樂的?」

  「大德!是苦的。」

  「而凡為無常、苦、變易法,你們認為:『這是我的,我是這個,這是我的真我。』嗎?」

  「大德!這確實不是。」

  「比丘們!這麼看的\twnr{有聽聞的聖弟子}{24.0}在眼上\twnr{厭}{15.0},也在諸色上厭,也在眼識上厭,也在眼觸上厭,又凡以這眼觸為緣生起感受的樂,或苦,或不苦不樂,在那個上也厭。……(中略)也在舌上厭,也在諸味道上……(中略)又凡以這舌觸為緣生起感受的樂,或苦,或不苦不樂,在那個上也厭。……(中略)也在意上厭,也在諸法上厭,也在意識上厭,也在意觸上厭,又凡以這意觸為緣生起感受的樂,或苦,或不苦不樂,在那個上也厭。厭者\twnr{離染}{558.0},從\twnr{離貪}{77.0}被解脫,在已解脫時,\twnr{有『[這是]解脫』之智}{27.0},他知道:『\twnr{出生已盡}{18.0},\twnr{梵行已完成}{19.0},\twnr{應該被作的已作}{20.0},\twnr{不再有此處[輪迴]的狀態}{21.1}。』

  比丘們!這是那個一切思量之適當根除的道跡。」 

  一切品第三,其\twnr{攝頌}{35.0}:

  「一切與二則捨斷,遍知在後二則,

   已燃燒與被征服,適合的與適當的二則,

   以那個被稱為品。」





\pin{生法品}{33}{42}
\sutta{33}{42}{生法等經十則}{https://agama.buddhason.org/SN/sn.php?keyword=35.33}
  起源於舍衛城。

  在那裡……(中略)。

  「\twnr{比丘}{31.0}們!一切是\twnr{生法}{587.0}。比丘們!而什麼是一切是生法呢?比丘們!眼是生法,諸色……眼識……眼觸是生法,又凡以這眼觸\twnr{為緣}{180.0}生起感受的樂,或苦,或不苦不樂,那也是生法。……(中略)舌……諸味道……舌識……舌觸……又凡以這舌觸為緣生起感受的樂,或苦,或不苦不樂,那也是生法。身……(中略)意是生法,諸法是生法,意識是生法,意觸是生法,又凡以這意觸為緣生起感受的樂,或苦,或不苦不樂,那也是生法。

  比丘們!這麼看的\twnr{有聽聞的聖弟子}{24.0}在眼上\twnr{厭}{15.0},也在諸色上……也在眼識上……也在眼觸上……(中略)他知道:『……\twnr{不再有此處[輪迴]的狀態}{21.1}。』」第一

  比丘們!一切是老法。……(中略)簡要的。第二

  比丘們!一切是病法。……(中略)。第三

  比丘們!一切是\twnr{死法}{587.3}。……(中略)。第四

  比丘們!一切是愁法。……(中略)。第五

  比丘們!一切是污染法。……(中略)。第六

  比丘們!一切是\twnr{滅盡法}{273.1}。……(中略)。第七

  比丘們!一切是\twnr{消散法}{155.0}。……(中略)。第八

  比丘們!一切是集法。……(中略)。第九

  比丘們!一切是\twnr{滅法}{68.1}。……(中略)。第十

  生法品第四,其\twnr{攝頌}{35.0}:

  「生、老、病、死,愁與污染,

   滅盡、消散、集,以滅法它們為十。」





\pin{一切無常品}{43}{52}
\sutta{43}{51}{無常等經九則}{https://agama.buddhason.org/SN/sn.php?keyword=35.43}
  起源於舍衛城。

  在那裡……(中略)。

  「\twnr{比丘}{31.0}們!一切是無常的。比丘們!而什麼是一切是無常的呢?比丘們!眼是無常的,諸色是無常的,眼識是無常的,眼觸是無常的,又凡以這眼觸為緣生起感受的樂,或苦,或不苦不樂,那也是無常的。

  ……(中略)舌是無常的,味道是無常的,舌識是無常的,舌觸是無常的,又凡以這舌觸為緣生起感受的樂,或苦,或不苦不樂,那也是無常的。身是無常的……(中略)意是無常的,諸法是無常的,意識是無常的,意觸是無常的,又凡以這意觸為緣生起感受的樂,或苦,或不苦不樂,那也是無常的。

  比丘們!這麼看的\twnr{有聽聞的聖弟子}{24.0}在眼上\twnr{厭}{15.0},也在諸色上厭,也在眼識上厭,也在眼觸上厭,又凡以這眼觸為緣生起感受的樂,或苦,或不苦不樂,在那個上也厭。……(中略)也在意上厭,也在諸法上厭,也在意識上厭,也在意觸上厭,又凡以這意觸為緣生起感受的樂,或苦,或不苦不樂,在那個上也厭。厭者\twnr{離染}{558.0},從\twnr{離貪}{77.0}被解脫,在已解脫時,\twnr{有『[這是]解脫』之智}{27.0},他知道:『\twnr{出生已盡}{18.0},\twnr{梵行已完成}{19.0},\twnr{應該被作的已作}{20.0},\twnr{不再有此處[輪迴]的狀態}{21.1}。』」第一

  比丘們!一切是苦的。……(中略)。第二

  比丘們!一切是無我。……(中略)。第三

  比丘們!一切應該被證知。……(中略)。第四

  比丘們!一切應該被\twnr{遍知}{154.0}。……(中略)。第五

  比丘們!一切應該被捨斷。……(中略)。第六

  比丘們!一切應該被作證。……(中略)。第七

  比丘們!一切應該被證知遍知。……(中略)。第八

  比丘們!一切是困厄的。……(中略)。第九



\sutta{52}{52}{被逼惱經}{https://agama.buddhason.org/SN/sn.php?keyword=35.52}
  「\twnr{比丘}{31.0}們!一切是被逼惱的。比丘們!而什麼是一切是被逼惱的呢?比丘們!眼是被逼惱的,諸色是被逼惱的,眼識是被逼惱的,眼觸是被逼惱的,又凡以這眼觸為緣生起感受的樂,或苦,或不苦不樂,那也是被逼惱的。……(中略)舌是被逼惱的,諸味道是被逼惱的,舌識是被逼惱的,舌觸是被逼惱的,又凡以這舌觸為緣生起感受的樂,或苦,或不苦不樂,那也是被逼惱的。身是被逼惱的……(中略)意是被逼惱的,諸法是被逼惱的,意識是被逼惱的,意觸是被逼惱的,又凡以這意觸為緣生起感受的樂,或苦,或不苦不樂,那也是被逼惱的。

  比丘們!這麼看的\twnr{有聽聞的聖弟子}{24.0}在眼上\twnr{厭}{15.0},也在諸色上厭,也在眼識上厭,也在眼觸上厭,又凡以這眼觸為緣生起感受的樂,或苦,或不苦不樂,在那個上也厭。……(中略)也在意上厭,也在諸法上厭,也在意識上厭,也在意觸上厭,又凡以這意觸為緣生起感受的樂,或苦,或不苦不樂,在那個上也厭。厭者\twnr{離染}{558.0},從\twnr{離貪}{77.0}被解脫,在已解脫時,\twnr{有『[這是]解脫』之智}{27.0},他知道:『\twnr{出生已盡}{18.0},\twnr{梵行已完成}{19.0},\twnr{應該被作的已作}{20.0},\twnr{不再有此處[輪迴]的狀態}{21.1}。』」

  一切無常的品第五,其\twnr{攝頌}{35.0}:

  「無常、苦、無我,應該被證知、應該被遍知,

   應該被捨斷、應該被作證,應該被證知遍知,

   困厄的、被逼惱的,以那個被稱為品。」

  六處篇第一個五十則完成,其品的攝頌:

  「無常、雙品,一切品、生法,

   以無常品為五十則,以那個被稱為第五。」





\pin{無明品}{53}{62}
\sutta{53}{53}{無明的捨斷經}{https://agama.buddhason.org/SN/sn.php?keyword=35.53}
  起源於舍衛城。

  那時,\twnr{某位比丘}{39.0}去見世尊。抵達後,向世尊\twnr{問訊}{46.0}後,在一旁坐下。在一旁坐下的那位比丘對世尊說這個:

  「\twnr{大德}{45.0}!當怎樣知、當怎樣見時,\twnr{無明}{207.0}被捨斷,生起明?」

  「比丘!當知、當見眼為無常的時,無明被捨斷,生起明。當知、當見諸色為無常的時,無明被捨斷,生起明。眼識……又凡以這眼觸\twnr{為緣}{180.0}生起感受的樂,或苦,或不苦不樂,當知、當見那也是無常的時,無明被捨斷,生起明。

  耳……鼻……舌……身……當知、當見意為無常的時,無明被捨斷,生起明。諸法……意識……意觸……又凡以這意觸為緣生起感受的樂,或苦,或不苦不樂,當知、當見那也是無常的時,無明被捨斷,生起明。

  比丘!這樣知者、這樣見者的無明被捨斷,生起明。」



\sutta{54}{54}{結的捨斷經}{https://agama.buddhason.org/SN/sn.php?keyword=35.54}
  「\twnr{大德}{45.0}!當怎樣知、當怎樣見時,諸結被捨斷?」

  「\twnr{比丘}{31.0}!當知、當見眼為無常的時,諸結被捨斷。諸色……眼識……眼觸……又凡以這眼觸為緣生起感受的樂,或苦,或不苦不樂,當知、當見那也是無常的時,諸結被捨斷。耳……鼻……舌……身……意……諸法……意識……意觸……又凡以這意觸為緣生起感受的樂,或苦,或不苦不樂,當知、當見那也是無常的時,諸結被捨斷。

  比丘!當這樣知、當這樣見時,諸結被捨斷。」



\sutta{55}{55}{結的根除經}{https://agama.buddhason.org/SN/sn.php?keyword=35.55}
  「\twnr{大德}{45.0}!當怎樣知、當怎樣見時,諸結走到根除?」

  「\twnr{比丘}{31.0}!當知、當見眼是無我時,諸結走到根除。諸色……眼識……眼觸……又凡以這眼觸為緣生起感受的樂,或苦,或不苦不樂,當知、當見那也是無我時,諸結走到根除。耳……鼻……舌……身……意……諸法……意識……意觸……又凡以這意觸為緣生起感受的樂,或苦,或不苦不樂,當知、當見那也是無我時,諸結走到根除。

  比丘!當這樣知、當這樣見時,諸結走到根除。」



\sutta{56}{56}{漏的捨斷經}{https://agama.buddhason.org/SN/sn.php?keyword=35.56}
  「\twnr{大德}{45.0}!當怎樣知、當怎樣見時,諸\twnr{漏}{188.0}被捨斷?」……(中略)。



\sutta{57}{57}{漏的根除經}{https://agama.buddhason.org/SN/sn.php?keyword=35.57}
  「\twnr{大德}{45.0}!當怎樣知、當怎樣見時,諸\twnr{漏}{188.0}走到根除?」……(中略)。



\sutta{58}{58}{煩惱潛在趨勢的捨斷經}{https://agama.buddhason.org/SN/sn.php?keyword=35.58}
  「\twnr{大德}{45.0}!當怎樣知、當怎樣見時,諸\twnr{煩惱潛在趨勢}{253.1}被捨斷?」……(中略)。



\sutta{59}{59}{煩惱潛在趨勢的根除經}{https://agama.buddhason.org/SN/sn.php?keyword=35.59}
  「\twnr{大德}{45.0}!……怎樣……諸\twnr{煩惱潛在趨勢}{253.1}走到根除?」

  「\twnr{比丘}{31.0}!當知、當見眼是無我時,諸煩惱潛在趨勢走到根除。……(中略)耳……鼻……舌……身……意……諸法……意識……意觸……又凡以這意觸為緣生起感受的樂,或苦,或不苦不樂,當知、當見那也是無我時,諸煩惱潛在趨勢走到根除。

  比丘!當這樣知、當這樣見時,諸漏潛在趨勢走到根除。」



\sutta{60}{60}{一切取的遍知經}{https://agama.buddhason.org/SN/sn.php?keyword=35.60}
  「\twnr{比丘}{31.0}們!我將為你們教導為了一切取的\twnr{遍知}{154.0}之法,\twnr{你們要聽}{43.0}它!  

  比丘們!而什麼是為了一切取的遍知之法呢?\twnr{緣於}{252.0}眼與諸色眼識生起,\twnr{三者的會合有觸}{195.0},以觸\twnr{為緣}{180.0}有受。比丘們!這麼看的\twnr{有聽聞的聖弟子}{24.0}在眼上\twnr{厭}{15.0},也在諸色上厭,也在眼識上厭,也在眼觸上厭,也在受上厭。厭者\twnr{離染}{558.0},從\twnr{離貪}{77.0}被解脫,從解脫他知道:『取被我遍知。』

  緣於耳與諸聲音生起……緣於鼻與諸氣味……緣於舌與諸味道……緣於身與諸\twnr{所觸}{220.2}……緣於意與諸法意識生起,三者的會合有觸,以觸為緣有受。比丘們!這麼看的有聽聞的聖弟子也在意上厭,也在諸法上厭,也在意識上厭,也在意觸上厭,也在受上厭。厭者\twnr{離染}{558.0},從\twnr{離貪}{77.0}被解脫,從解脫他知道:『取被我遍知。』

  比丘們!這是為了一切取的遍知之法。」 



\sutta{61}{61}{一切取的耗盡經第一}{https://agama.buddhason.org/SN/sn.php?keyword=35.61}
  「\twnr{比丘}{31.0}們!我將為你們教導為了一切取的耗盡之法,\twnr{你們要聽}{43.0}它!  

  比丘們!而什麼是為了一切取的耗盡之法呢?\twnr{緣於}{252.0}眼與諸色眼識生起,\twnr{三者的會合有觸}{195.0},以觸\twnr{為緣}{180.0}有受。比丘們!這麼看的\twnr{有聽聞的聖弟子}{24.0}在眼上\twnr{厭}{15.0},也在諸色上厭,也在眼識上厭,也在眼觸上厭,也在受上厭。厭者\twnr{離染}{558.0},從\twnr{離貪}{77.0}被解脫,從解脫他知道:『取被我耗盡。』

  ……(中略)緣於舌與諸味道後舌識生起……(中略)緣於意與諸法意識生起,三者的會合有觸,以觸為緣有受。比丘們!這麼看的有聽聞的聖弟子也在意上厭,也在諸法上厭,也在意識上厭,也在意觸上厭,也在受上厭。厭者\twnr{離染}{558.0},從\twnr{離貪}{77.0}被解脫,從解脫他知道:『取被我耗盡。』

  比丘們!這是為了一切取的耗盡之法。」 



\sutta{62}{62}{一切取的耗盡經第二}{https://agama.buddhason.org/SN/sn.php?keyword=35.62}
  「\twnr{比丘}{31.0}們!我將為你們教導為了一切取的耗盡之法,\twnr{你們要聽}{43.0}它!  

  比丘們!而什麼是為了一切取的耗盡之法呢?

  比丘們!你們怎麼想它:眼是常的,或是無常的?」

  「無常的,\twnr{大德}{45.0}!」

  「那麼,凡為無常的,那是苦的或樂的?」

  「苦的,大德!」

  「而凡為無常、苦、\twnr{變易法}{70.0},適合認為它:『\twnr{這是我的}{32.0},\twnr{我是這個}{33.0},\twnr{這是我的真我}{34.1}。』嗎?」

  「大德!這確實不是。」

  「諸色……(中略)眼識是常的,或是無常的?」

  「無常的,大德!」……(中略)。

  「眼觸是常的,或是無常的?」

  「無常的,大德!」

  「又凡以這眼觸為緣生起感受的樂,或苦,或不苦不樂,那也是常的,或是無常的?」

  「無常的,大德!」……(中略)。

  「耳……鼻……舌……身……意……諸法……意識……意觸……又凡以這意觸為緣生起感受的樂,或苦,或不苦不樂,那也是常的,或是無常的?」

  「無常的,大德!」

  「那麼,凡為無常的,那是苦的或樂的?」

  「苦的,大德!」

  「而凡為無常、苦、變易法,適合認為它:『\twnr{這是我的}{32.0},\twnr{我是這個}{33.0},這是\twnr{我的真我}{34.0}。』嗎?」

  「大德!這確實不是。」

  「比丘們!這麼看的\twnr{有聽聞的聖弟子}{24.0}在眼上\twnr{厭}{15.0},也在諸色上厭,也在眼識上厭,也在眼觸上厭,又凡以這眼觸為緣生起感受的樂,或苦,或不苦不樂,在那個上也厭。……(中略)也在舌上厭,也在諸味道上厭,也在舌識上厭,也在舌觸上厭,又凡以這舌觸為緣生起感受的樂,或苦,或不苦不樂,在那個上也厭。……(中略)也在意上厭,也在諸法上厭,也在意識上厭,也在意觸上厭,又凡以這意觸為緣生起感受的樂,或苦,或不苦不樂,在那個上也厭。厭者\twnr{離染}{558.0},從\twnr{離貪}{77.0}被解脫,在已解脫時,\twnr{有『[這是]解脫』之智}{27.0},他知道:『\twnr{出生已盡}{18.0},\twnr{梵行已完成}{19.0},\twnr{應該被作的已作}{20.0},\twnr{不再有此處[輪迴]的狀態}{21.1}。』

  比丘們!這是為了一切取的耗盡之法。」 

  無明品第四,其\twnr{攝頌}{35.0}:

  「無明、諸結二則,以諸漏二說,

   諸煩惱潛在趨勢在後二則,遍知、二則耗盡,

   以那個被稱為品。」





\pin{鹿網品}{63}{73}
\sutta{63}{63}{鹿網經第一}{https://agama.buddhason.org/SN/sn.php?keyword=35.63}
  起源於舍衛城。

  那時,\twnr{尊者}{200.0}鹿網去見\twnr{世尊}{12.0}。……(中略)在一旁坐下的尊者鹿網對世尊說這個:

  「\twnr{大德}{45.0}!被稱為『\twnr{獨住}{x481}、獨住』,大德!什麼情形是獨住?什麼情形是\twnr{有伴同住}{x482}?」

  「鹿網!有能被眼識知的、想要的、所愛的、合意的、可愛形色的、伴隨欲的、誘人的諸色,如果\twnr{比丘}{31.0}歡喜、歡迎、持續固持那個,那位歡喜、歡迎、持續固持那個者的歡喜生起;當有歡喜時,有貪著;當有貪著時,有繫縛,鹿網!被歡喜結結縛的比丘被稱為『有伴同住』。有……(中略)鹿網!有能被舌識知的、想要的、所愛的、合意的、可愛形色的、伴隨欲的、誘人的諸味道,如果比丘歡喜、歡迎、持續固持那個,那位歡喜、歡迎、持續固持那個者的歡喜生起;當有歡喜時,有貪著;當有貪著時,有結縛,鹿網!被歡喜結結縛的比丘被稱為『有伴同住』。……鹿網!而這樣住的比丘:即使受用諸少聲音的、安靜的、\twnr{離人之氛圍的}{493.0}、\twnr{人獨住的}{x483}、適合\twnr{獨坐}{92.0}的\twnr{林野}{142.0}、荒林、\twnr{邊地臥坐處}{30.0},那時被稱為『有伴同住』。那是什麼原因?因為渴愛是他的同伴,他未捨斷它,因此被稱為『有伴同住』。

  鹿網!有能被眼識知的、想要的、所愛的、合意的、可愛形色的、伴隨欲的、誘人的諸色,如果比丘不歡喜、不歡迎、不持續固持那個,對那位不歡喜、不歡迎、不持續固持那個者來說歡喜被滅;當沒有歡喜時,沒有貪著;當沒有貪著時,沒有結縛,鹿網!離被歡喜結結縛的比丘被稱為『獨住』。……(中略)鹿網!有能被舌識知……的諸味道……(中略)鹿網!有能被意識知的、想要的、所愛的、合意的、可愛形色的、伴隨欲的、誘人的諸法,如果比丘不歡喜、不歡迎、不持續固持那個,對那位不歡喜、不歡迎、不持續固持那個者來說歡喜被滅;當沒有歡喜時,沒有貪著;當沒有貪著時,沒有結縛,鹿網!離被歡喜結結縛的比丘被稱為『獨住』。鹿網!而這樣住的比丘:即使住在村落邊,與比丘、比丘尼、\twnr{優婆塞}{98.0}、\twnr{優婆夷}{99.0}、國王、國王的大臣、外道、外道弟子們混雜者,那時被稱為『獨住』。

  那是什麼原因?因為渴愛是他的同伴,他已捨斷它,因此被稱為『獨住』。」



\sutta{64}{64}{鹿網經第二}{https://agama.buddhason.org/SN/sn.php?keyword=35.64}
  那時,\twnr{尊者}{200.0}鹿網去見世尊。……(中略)在一旁坐下的尊者鹿網對世尊說這個:

  「\twnr{大德}{45.0}!請世尊為我簡要地教導法,凡我聽聞世尊的法後,會住於單獨的、隱離的、不放逸的、熱心的、自我努力的,\twnr{那就好了}{44.0}!」

  「鹿網!有能被眼識知的、想要的、所愛的、合意的、可愛形色的、伴隨欲的、誘人的諸色,如果\twnr{比丘}{31.0}歡喜、歡迎、持續固持那個,那位歡喜、歡迎、持續固持那個者的歡喜生起,我說:『鹿網!以歡喜\twnr{集}{67.0}而有苦集。』……(中略)鹿網!有能被舌識知的、想要的、所愛的……諸味道……(中略)鹿網!有能被意識知的、想要的、所愛的、合意的、可愛形色的、伴隨欲的、誘人的諸法,如果比丘歡喜、歡迎、持續固持那個,那位歡喜、歡迎、持續固持那個者的歡喜生起,我說:『鹿網!以歡喜集而有苦集。』

  鹿網!有能被眼識知的、想要的、所愛的、合意的、可愛形色的、伴隨欲的、誘人的諸色,如果比丘不歡喜、不歡迎、不持續固持那個,對那位不歡喜、不歡迎、不持續固持那個者來說歡喜被滅,我說:『鹿網!以歡喜\twnr{滅}{68.0}有苦滅。』……(中略)鹿網!有能被舌識知的、想要的、所愛的……諸味道……(中略)鹿網!有能被意識知的、想要的、所愛的、合意的、可愛形色的、伴隨欲的、誘人的諸法,如果比丘不歡喜、不歡迎、不持續固持那個,對那位不歡喜、不歡迎、不持續固持那個者來說歡喜被滅,我說:『鹿網!以歡喜滅有苦滅。』」

  那時,尊者鹿網歡喜、隨喜世尊所說後,從座位起來、向世尊\twnr{問訊}{46.0}、\twnr{作右繞}{47.0}後,離開。

  那時,當尊者鹿網住於單獨的、隱離的、不放逸的、熱心的、自我努力的時,就不久,以證智自作證後,在當生中\twnr{進入後住於}{66.0}凡\twnr{善男子}{41.0}們為了利益正確地\twnr{從在家出家成為無家者}{48.0}的那個無上梵行結尾,他證知:「\twnr{出生已盡}{18.0},\twnr{梵行已完成}{19.0},\twnr{應該被作的已作}{20.0},\twnr{不再有此處[輪迴]的狀態}{21.1}。」然後尊者鹿網成為眾\twnr{阿羅漢}{5.0}之一。



\sutta{65}{65}{三彌提問魔經第一}{https://agama.buddhason.org/SN/sn.php?keyword=35.65}
  \twnr{有一次}{2.0},\twnr{世尊}{12.0}住在王舍城栗鼠飼養處的竹林中。

  那時,\twnr{尊者}{200.0}三彌提去見世尊……(中略)對世尊說這個:

  「\twnr{大德}{45.0}!被稱為『魔、魔』,大德!什麼情形會有魔或魔的名字(\twnr{安立}{143.0})?」

  「三彌提!在有眼,有諸色,有眼識,有能被眼識識知諸法之處,在那裡有魔或魔的名字;在有耳,有諸聲音,有耳識,有能被耳識識知諸法之處,在那裡有魔或魔的名字;在有鼻,有諸氣味,有鼻識,有能被鼻識識知諸法之處,在那裡有魔或魔的名字;在有舌,有諸味道,有舌識,有能被舌識識知諸法之處,在那裡有魔或魔的名字;在有身,有諸\twnr{所觸}{220.2},有身識,有能被身識識知諸法之處,在那裡有魔或魔的名字;在有意,有諸法,有意識,有能被意識識知諸法之處,在那裡有魔或魔的名字。

  三彌提!在沒有眼,沒有諸色,沒有眼識,沒有能被眼識識知諸法之處,在那裡沒有魔或魔的名字;沒有耳……(中略)沒有鼻……(中略)沒有舌,沒有諸味道,沒有舌識,沒有能被舌識所識之法的地方,就沒有魔或魔的名字;沒有身……(中略)在沒有意,沒有諸法,沒有意識,沒有能被意識識知諸法之處,在那裡沒有魔或魔的名字。」



\sutta{66}{66}{三彌提問眾生經}{https://agama.buddhason.org/SN/sn.php?keyword=35.66}
  「\twnr{大德}{45.0}!被稱為『眾生、眾生』,大德!什麼情形會有眾生或眾生的名字(\twnr{安立}{143.0})?」……(中略)。



\sutta{67}{67}{三彌提問苦經}{https://agama.buddhason.org/SN/sn.php?keyword=35.67}
  「\twnr{大德}{45.0}!被稱為『苦、苦』,大德!什麼情形會有苦或苦的名字(\twnr{安立}{143.0})?」……(中略)。



\sutta{68}{68}{三彌提問世間經}{https://agama.buddhason.org/SN/sn.php?keyword=35.68}
  「\twnr{大德}{45.0}!被稱為『世間、世間』,大德!什麼情形會有世間或世間的名字(\twnr{安立}{143.0})?」

  「三彌提!在有眼,有諸色,有眼識,有能被眼識識知諸法之處,在那裡有世間或世間的名字……(中略)有舌……(中略)在有意,有諸法,有意識,有能被意識識知諸法之處,在那裡有世間或世間的名字。

  三彌提!在沒有眼,沒有諸色,沒有眼識,沒有能被眼識識知諸法之處,在那裡沒有世間或世間的名字……(中略)沒有舌……(中略)在沒有意,沒有諸法,沒有意識,沒有能被意識識知諸法之處,在那裡沒有世間或世間的名字。」





\sutta{69}{69}{優波先那-毒蛇經}{https://agama.buddhason.org/SN/sn.php?keyword=35.69}
  \twnr{有一次}{2.0},\twnr{尊者}{200.0}舍利弗與尊者\twnr{優波先那}{x484}住在王舍城寒林蛇頭岩洞窟處。

  當時,有毒蛇落在尊者優波先那的身體上。

  那時,尊者優波先那召喚\twnr{比丘}{31.0}們:

  「來!學友們!請你們將我的這個身體放上床後,在身體就在這裡猶如拳中的粗糠般散掉之前,移到外面。」

  在這麼說時,尊者舍利弗對尊者優波先那說這個:

  「但我們沒看見尊者優波先那的身體變異,或諸根變易,然而,尊者優波先那這麼說:『來!學友們!請你們將我的這個身體放上床後,在身體就在這裡猶如拳中的粗糠散掉之前,移到外面。』」

  「舍利弗\twnr{學友}{201.0}!確實,凡如果這麼想:『我是眼』或『眼是我的』……(中略)『我是舌』或『舌是我的』……『我是意』或『意是我的』者,舍利弗學友!對他來說會有身體的變異,或諸根的變易。舍利弗學友!但我不這麼想:『我是眼』或『眼是我的』……(中略)『我是舌』或『舌是我的』……『我是意』或『意是我的』,舍利弗學友!那個我為何將有身體的變異,或諸根的變易呢?」

  「那麼,因為對尊者優波先那來說,長久以來\twnr{我作}{22.0}、\twnr{我所作}{25.0}、\twnr{慢煩惱潛在趨勢}{26.0}像這樣被徹底根除(善根除),因此尊者優波先那不這麼想:『我是眼』或『眼是我的』……(中略)『我是舌』或『舌是我的』……『我是意』或『意是我的』。」

  那時,那些比丘將尊者優波先那的身體放上床後,移到外面。

  那時,尊者優波先那的身體就在那裡猶如拳中的粗糠般散掉。



\sutta{70}{70}{優波哇那直接可見的經}{https://agama.buddhason.org/SN/sn.php?keyword=35.70}
  那時,\twnr{尊者}{200.0}優波哇那去見\twnr{世尊}{12.0}……(中略)在一旁坐下的尊者優波哇那對世尊說這個:

  「\twnr{大德}{45.0}!被稱為『直接可見的法、直接可見的法』,大德!什麼情形法是直接可見的、即時的、請你來看的、能引導的、應該被智者各自經驗的?」

  「優波哇那!那麼,這裡,\twnr{比丘}{31.0}以眼見色後,有色的感受與色貪的感受,而自身內在諸色上存在貪時他知道:『我的自身內在諸色上有貪。』優波哇那!凡那位比丘以眼見色後,有色的感受與色貪的感受,而當自身內在諸色上存在貪時他知道:『我的自身內在諸色上有貪。』優波哇那!這樣,『法是直接可見的、即時的、請你來看的、能引導的、應該被智者各自經驗的。』……(中略)。

  再者,優波哇那!比丘以舌嚐味道後,有味道的感受與味道貪的感受,而當自身內在諸味道上存在貪時他知道:『我的自身內在諸味道上有貪。』優波哇那!凡那位比丘以舌嚐味道後,有味道的感受與味道貪的感受,而當自身內在諸味道上存在貪時他知道:『我的自身內在諸味道上有貪。』優波哇那!這樣也是『法是直接可見的、即時的、請你來看的、能引導的、應該被智者各自經驗的。』……(中略)。

  再者,優波哇那!比丘以意識知法後,有法的感受與法貪的感受,而當自身內在諸法上存在貪時他知道:『我的自身內在諸法上有貪。』優波哇那!凡那位比丘以意識知法後,有法的感受與法貪的感受,而當自身內在諸法上存在貪時他知道:『我的自身內在諸法上有貪。』優波哇那!這樣也是『法是直接可見的……(中略)應該被智者各自經驗的。』{……(中略)}

  優波哇那!這裡,比丘以眼見色後,有色的感受而無色貪的感受,而當自身內在諸色上不存在貪時他知道:『我的自身內在諸色上沒有貪。』優波哇那!凡那位比丘以眼見色後,有色的感受而無色貪的感受,而當在諸色上自身內不存在貪時他知道:『我的自身內在諸色上沒有貪。』優波哇那!這樣也是『法是直接可見的、即時的、請你來看的、能引導的、應該被智者各自經驗的。』……(中略)。

  再者,優波哇那!比丘以舌嚐味道後,有味道的感受而無味道貪的感受,而當自身內在諸味道上不存在貪時他知道:『我的自身內在諸味道上沒有貪。』……(中略)。

  再者,優波哇那!比丘以意識知法後,有法的感受而無法貪的感受,而當自身內在諸法上不存在貪時他知道:『我的自身內在諸法上沒有貪。』優波哇那!凡那位比丘以意識知法後,有法的感受而無法貪的感受,而當自身內在諸法上不存在貪時他知道:『我的自身內在諸法上沒有貪。』優波哇那!這樣也是『法是直接可見的、即時的、請你來看的、能引導的、應該被智者各自經驗的。』」



  ※附記:關於「法」的定型句,北傳作:「說現法,說滅熾然,說不待時,說正向,說即此見,說緣自覺」(「說現法」的「說」是指「世尊所說的法」),另譯作「離諸熱惱,非時通達,即於現法,緣自覺悟」,南傳作「直接可見的、即時的、請你來看的、能引導的、應該被智者各自經驗的」,也是定型句,但多數經文在「直接可見的」前面還多一項「被世尊善說的」:

  「被世尊善說的」(svākkhāto bhagavatā),菩提比丘長老英譯為「被幸福者很好地解說的」(is well expounded by the Blessed One)。按「善」(su-)至少還有「適切地;巧妙的;完全的」等廣泛的含意。

  「直接可見的」(sandiṭṭhiko,另譯為「現見的;現證的、自見的、現世的」),相當於「現法」、「即於現法」,菩提比丘長老英譯為「直接可見的」(directly visible)。

  「即時的」(akāliko,另譯為「非時的」),相當於「不待時」、「非時」,菩提比丘長老英譯為「即時的」(immediate)。

  「請你來看的」(ehipassiko,另譯為「來見」),相當於「即此見」,菩提比丘長老英譯為「勸誘人來並且看」(inviting one to come and see)。

  「能引導的」(opaneyyiko),相當於「正向」、「通達」,菩提比丘長老英譯為「能應用的;能引導的」(applicable; worthly of application)。

  「應該被智者各自經驗的」(paccattaṃ veditabbo viññūhī),相當於「緣自覺」、「緣自覺悟」,菩提比丘長老英譯為「被智者親自體驗的」(to be personally experienced by the wise)。按:「智者」(viññū),另譯為「有知;有智;識者」。



\sutta{71}{71}{六觸處經第一}{https://agama.buddhason.org/SN/sn.php?keyword=35.71}
  「\twnr{比丘}{31.0}們!凡任何比丘不如實知道\twnr{六觸處}{78.0}的\twnr{集起}{67.0}、滅沒、\twnr{樂味}{295.0}、\twnr{過患}{293.0}、\twnr{出離}{294.0}者,梵行沒被他完成(住),他從這法與律遠離。」

  在這麼說時,某位比丘對\twnr{世尊}{12.0}說這個:

  「\twnr{大德}{45.0}!在這裡,我滅亡了,大德!因為我不如實知道六觸處的集起、滅沒、樂味、過患、出離。」

  「比丘!你怎麼想它:你認為眼『\twnr{這是我的}{32.0},\twnr{我是這個}{33.0},這是\twnr{我的真我}{34.0}。』嗎?」

  「大德!這確實不是。」

  「比丘!\twnr{好}{44.0}!比丘!而在這裡,眼:『\twnr{這不是我的}{32.1},\twnr{我不是這個}{33.1},\twnr{這不是我的真我}{34.2}。』這樣這個將被你以正確之慧如實善見,這就是苦的結束。……(中略)。

  你認為舌:『這是我的,我是這個,這是我的真我。』嗎?」

  「大德!這確實不是。」

  「比丘!好!比丘!而在這裡,舌:『這不是我的,我不是這個,這不是我的真我。』這樣,這將被你以正確之慧如實善見,這就是苦的結束。……(中略)。

  你認為意:『這是我的,我是這個,這是我的真我。』嗎?」

  「大德!這確實不是。」

  「比丘!好!比丘!而在這裡,意:『這不是我的,我不是這個,這不是我的真我。』這樣,這將被你以正確之慧如實善見,這就是苦的結束。」



\sutta{72}{72}{六觸處經第二}{https://agama.buddhason.org/SN/sn.php?keyword=35.72}
  「\twnr{比丘}{31.0}們!凡任何比丘不如實知道\twnr{六觸處}{78.0}的\twnr{集起}{67.0}、滅沒、\twnr{樂味}{295.0}、\twnr{過患}{293.0}、\twnr{出離}{294.0}者,梵行沒被他完成(住),他從這法與律遠離。」

  在這麼說時,某位比丘對\twnr{世尊}{12.0}說這個:

  「\twnr{大德}{45.0}!在這裡,我滅亡了、毀滅了,大德!因為我不如實知道六觸處的集起、滅沒、樂味、過患、出離。」

  「比丘!你怎麼想它:你認為眼『\twnr{這不是我的}{32.1},\twnr{我不是這個}{33.1},\twnr{這不是我的真我}{34.2}。』嗎?」

  「是的,大德!」

  「比丘!\twnr{好}{44.0}!比丘!而在這裡,眼:『這不是我的,我不是這個,這不是我的真我。』這樣,這將被你以正確之慧如實善見,這樣,這將是為了未來不再有被你捨斷的第一個觸處。……(中略)。

  你認為舌:『這不是我的,我不是這個,這不是我的真我。』嗎?」

  「是的,大德!」

  「比丘!好!比丘!而在這裡,舌:『這不是我的,我不是這個,這不是我的真我。』這樣,這將被你以正確之慧如實善見,這樣,這將是為了未來不再有被你捨斷的第四個觸處。……(中略)。

  你認為意:『這不是我的,我不是這個,這不是我的真我。』嗎?」

  「是的,大德!」

  「比丘!好!比丘!而在這裡,意:『這不是我的,我不是這個,這不是我的真我。』這樣,這將被你以正確之慧如實善見,這樣,這將是為了未來不再有而被你捨斷的第六個觸處。」



\sutta{73}{73}{六觸處經第三}{https://agama.buddhason.org/SN/sn.php?keyword=35.73}
  「\twnr{比丘}{31.0}們!凡任何比丘不如實知道\twnr{六觸處}{78.0}的\twnr{集起}{67.0}、滅沒、\twnr{樂味}{295.0}、\twnr{過患}{293.0}、\twnr{出離}{294.0}者,梵行沒被他完成(住),他從這法與律遠離。」

  在這麼說時,某位比丘對\twnr{世尊}{12.0}說這個:

  「\twnr{大德}{45.0}!在這裡,我滅亡了、毀滅了,大德!因為我不如實知道六觸處的集起、滅沒、樂味、過患、出離。」

  「比丘!你怎麼想它:眼是常的,或是無常的?」

  「無常的,大德!」

  「而凡為無常的,是苦的還是樂的呢?」

  「苦的,大德!」

  「那麼,凡為無常的、苦的、\twnr{變易法}{70.0},適合認為它:『\twnr{這是我的}{32.0},\twnr{我是這個}{33.0},\twnr{這是我的真我}{34.1}。』嗎?」

  「大德!這確實不是。」

  耳……鼻……舌……身……。

  「意是常的,或是無常的?」

  「無常的,大德!」

  「而凡為無常的,是苦的還是樂的呢?」

  「苦的,大德!」

  「那麼,凡為無常的、苦的、變易法,適合認為它:『\twnr{這是我的}{32.0},\twnr{我是這個}{33.0},這是\twnr{我的真我}{34.0}。』嗎?」

  「大德!這確實不是。」

  「比丘!這麼看的\twnr{有聽聞的聖弟子}{24.0}在眼上\twnr{厭}{15.0},也在耳上厭,在鼻上厭,在舌上厭,在身上厭,也在意上厭。厭者\twnr{離染}{558.0},從\twnr{離貪}{77.0}被解脫,在已解脫時,\twnr{有『[這是]解脫』之智}{27.0},他知道:『\twnr{出生已盡}{18.0},\twnr{梵行已完成}{19.0},\twnr{應該被作的已作}{20.0},\twnr{不再有此處[輪迴]的狀態}{21.1}。』」

  鹿網品第七,其\twnr{攝頌}{35.0}:

  「鹿網二說,以及三彌提四則,

   優波先那、優波哇那,六觸處三則。」





\pin{病人品}{74}{83}
\sutta{74}{74}{病人經第一}{https://agama.buddhason.org/SN/sn.php?keyword=35.74}
  起源於舍衛城。

  那時,\twnr{某位比丘}{39.0}去見世尊。……(中略)

  在一旁坐下的那位比丘對世尊說這個:

  「\twnr{大德}{45.0}!在那樣的住處,有某位新進、少被知道、生病的、受苦的、重病的比丘,大德!請世尊\twnr{出自憐愍}{121.0},去見那位比丘,\twnr{那就好了}{44.0}!」

  那時,世尊聽聞新進之語與生病之語後,像這樣,知道「是少被知道的比丘」後,去見那位比丘。

  那位比丘看見正從遠處到來的世尊。看見後,\twnr{在臥床上移動}{386.0}。

  那時,世尊對那位比丘說這個:

  「夠了,比丘!你不要在臥床上移動,有這些設置的座位,我將坐在那裡。」

  世尊在設置的座位坐下。坐下後,世尊對那位比丘說這個:

  「比丘!是否能被你忍受?\twnr{是否能被[你]維持生活}{137.0}?是否苦的感受減退、不增進,減退的結局被知道,非增進?」

  「大德!不能被我忍受,不能被[我]維持,我強烈苦的感受增進、不減退,增進的結局被知道,非減退。」

  「比丘!是否你沒有任何後悔?沒有任何悔憾?」

  「大德!我當然有不少的後悔,不少的悔憾。」

  「比丘!那麼,是否自己從戒[不-\suttaref{SN.22.87}]責備你?」

  「大德!自己從戒不責備我。」

  「比丘!如果自己從戒不責備你,那麼,你有什麼後悔與什麼悔憾呢?」

  「大德!我了知被世尊教導的法不是戒清淨之目的。」

  「比丘!如果你確實了知被我教導的法不是戒清淨之目的,比丘!那麼,那樣的話,你了知被我教導的法是什麼目的?」

  「大德!我了知被世尊教導的法是為了貪的\twnr{褪去}{77.0}。」

  「比丘!\twnr{好}{44.0}!好!比丘!你了知被我教導的法是為了貪的褪去,好!比丘!因為被我教導的法是為了貪的褪去。比丘!你怎麼想它:眼是常的,或是無常的?」

  「無常的,大德!」

  「凡……(中略)耳……鼻……舌……身……意是常的,或是無常的?」

  「無常的,大德!」

  「那麼,凡為無常的,那是苦的或樂的?」

  「苦的,大德!」

  「那麼,凡為無常的、苦的、\twnr{變易法}{70.0},適合認為它:『\twnr{這是我的}{32.0},\twnr{我是這個}{33.0},\twnr{這是我的真我}{34.1}。』嗎?」

  「大德!這確實不是。」

  「比丘!這麼看的\twnr{有聽聞的聖弟子}{24.0}在眼上\twnr{厭}{15.0},也在耳上厭……(中略)在意上厭。厭者\twnr{離染}{558.0},從\twnr{離貪}{77.0}被解脫,在已解脫時,\twnr{有『[這是]解脫』之智}{27.0},他知道:『\twnr{出生已盡}{18.0},\twnr{梵行已完成}{19.0},\twnr{應該被作的已作}{20.0},\twnr{不再有此處[輪迴]的狀態}{21.1}。』」

  世尊說這個,那位悅意的比丘歡喜世尊的所說。

  還有,\twnr{在當這個解說被說時}{136.0},那位比丘的\twnr{遠塵、離垢之法眼}{62.0}生起:

  「凡任何\twnr{集法}{67.1}那個全部是\twnr{滅法}{68.1}。」



\sutta{75}{75}{病人經第二}{https://agama.buddhason.org/SN/sn.php?keyword=35.75}
  那時,\twnr{某位比丘}{39.0}……(中略)那位比丘對世尊說這個:

  「\twnr{大德}{45.0}!在那樣的住處,有某位新進、少被知道、生病的、受苦的、重病的比丘,大德!請世尊\twnr{出自憐愍}{121.0},去見那位比丘,\twnr{那就好了}{44.0}!」

  那時,世尊聽聞新進之語與生病之語後,像這樣,知道「是少被知道的比丘」後,去見那位比丘。

  那位比丘看見正從遠處到來的世尊。看見後,\twnr{在臥床上移動}{386.0}。

  那時,世尊對那位比丘說這個:

  「夠了,比丘!你不要在臥床上移動,有這些設置的座位,我將坐在那裡。」

  世尊在設置的座位坐下。坐下後,世尊對那位比丘說這個:

  「比丘!是否能被你忍受?\twnr{是否能被[你]維持生活}{137.0}?是否苦的感受減退、不增進,減退的結局被知道,非增進?」

  「大德!不能被我忍受,不能被[我]維持……(中略)。」……

  「大德!自己從戒不責備我。」

  「比丘!如果自己從戒不責備你,那麼,你有什麼後悔與什麼悔憾呢?」

  「大德!我了知被世尊教導的法不是戒清淨之目的。」

  「比丘!如果你確實了知被我教導的法不是戒清淨之目的,比丘!那麼,那樣的話,你了知被我教導的法是什麼目的?」

  「大德!我了知被世尊教導的法是不執取後\twnr{般涅槃}{72.0}之目的。」

  「比丘!\twnr{好}{44.0}!好!比丘!你了知被我教導的法是不執取後般涅槃之目的,好!比丘!因為被我教導的法是不執取後般涅槃之目的。比丘!你怎麼想它:眼是常的,或是無常的?」

  「無常的,大德!」

  「凡……(中略)耳……鼻……舌……身……意……意識……意觸……又凡以這意觸\twnr{為緣}{180.0}生起感受的樂,或苦,或不苦不樂,那也是常的,或是無常的?」

  「無常的,大德!」

  「那麼,凡為無常的,那是苦的或樂的?」

  「苦的,大德!」

  「那麼,凡為無常的、苦的、\twnr{變易法}{70.0},適合認為它:『\twnr{這是我的}{32.0},\twnr{我是這個}{33.0},\twnr{這是我的真我}{34.1}。』嗎?」

  「大德!這確實不是。」

  「比丘!這麼看的\twnr{有聽聞的聖弟子}{24.0}在眼上\twnr{厭}{15.0}……(中略)在意上……在意識上……在意觸上厭,又凡以這意觸為緣生起感受的樂,或苦,或不苦不樂,在那個上也厭。厭者\twnr{離染}{558.0},從\twnr{離貪}{77.0}被解脫,在已解脫時,\twnr{有『[這是]解脫』之智}{27.0},他知道:『\twnr{出生已盡}{18.0},\twnr{梵行已完成}{19.0},\twnr{應該被作的已作}{20.0},\twnr{不再有此處[輪迴]的狀態}{21.1}。』」

  世尊說這個,那位悅意的比丘歡喜世尊的所說。

  還有,\twnr{在當這個解說被說時}{136.0},不執取後那位比丘的心從諸\twnr{漏}{188.0}被解脫。



\sutta{76}{76}{羅陀-無常經}{https://agama.buddhason.org/SN/sn.php?keyword=35.76}
  那時,\twnr{尊者}{200.0}羅陀……(中略)在一旁坐下的尊者羅陀對世尊說這個:

  「\twnr{大德}{45.0}!請世尊為我簡要地教導法,凡我聽聞世尊的法後,會住於單獨的、隱離的、不放逸的、熱心的、自我努力的,\twnr{那就好了}{44.0}!」

  「羅陀!凡是無常的,在那裡意欲應該被你捨斷。羅陀!而什麼是無常的,在那裡意欲應該被你捨斷呢?眼是無常的……諸色是無常的……眼識……眼觸……又凡以這眼觸\twnr{為緣}{180.0}生起感受的樂,或苦,或不苦不樂那也是無常的,在那裡意欲應該被你捨斷……(中略)舌……身……意是無常的,在那裡意欲應該被你捨斷……諸法……意識……意觸……又凡以這意觸為緣生起感受的樂,或苦,或不苦不樂那也是無常的,在那裡意欲應該被你捨斷,羅陀!凡是無常的,在那裡意欲應該被你捨斷。」



\sutta{77}{77}{羅陀-苦經}{https://agama.buddhason.org/SN/sn.php?keyword=35.77}
  「羅陀!凡是苦的,在那裡意欲應該被你捨斷。羅陀!而什麼是苦的?羅陀!眼是苦的,在那裡意欲應該被你捨斷;諸色……眼識……眼觸……又凡以這眼觸……(中略)或不苦不樂受那也是苦的,在那裡意欲應該被你捨斷……(中略)意是苦的……諸法……意識……意觸……又凡以這意觸為緣生起感受的樂,或苦,或不苦不樂那也是苦的,在那裡意欲應該被你捨斷,羅陀!凡是苦的,在那裡意欲應該被你捨斷。」



\sutta{78}{78}{羅陀-無我經}{https://agama.buddhason.org/SN/sn.php?keyword=35.78}
  「羅陀!凡是無我,在那裡意欲應該被你捨斷。羅陀!而什麼是無我呢?羅陀!眼是無我,在那裡意欲應該被你捨斷;諸色……眼識……眼觸……又凡以這眼觸\twnr{為緣}{180.0}……(中略)意是無我……諸法……意識……意觸……又凡以這意觸為緣生起感受的樂,或苦,或不苦不樂那也是無我,在那裡意欲應該被你捨斷,羅陀!凡是無我,在那裡意欲應該被你捨斷。」



\sutta{79}{79}{無明的捨斷經第一}{https://agama.buddhason.org/SN/sn.php?keyword=35.79}
  那時,\twnr{某位比丘}{39.0}去見\twnr{世尊}{12.0}。……(中略)在一旁坐下的那位比丘對世尊說這個:

  「\twnr{大德}{45.0}!有\twnr{一法}{522.0},從對它的捨斷,比丘的\twnr{無明}{207.0}被捨斷,明生起嗎?」

  「\twnr{比丘}{31.0}!有一法,從對它的捨斷,比丘的無明被捨斷,明生起。」

  「大德!那麼,哪一法,從對它的捨斷,比丘的無明被捨斷,明生起?」

  「比丘!無明是一法,從對它的捨斷,比丘的無明被捨斷,明生起。」

  「大德!那麼,當比丘怎樣知、怎樣見時,無明被捨斷,明生起?」

  「比丘!當比丘知、見眼為無常的時,無明被捨斷,明生起。諸色……眼識……眼觸……又凡以這眼觸\twnr{為緣}{180.0}生起感受的樂,或苦,或不苦不樂,當比丘知、見那也是無常的時,無明被捨斷,明生起。……(中略)當比丘知、見意是無常的時,無明被捨斷,明生起。諸法……意識……意觸……又凡以這意觸為緣生起感受的樂,或苦,或不苦不樂,當比丘知、見那也是無常的時,無明被捨斷,明生起。

  比丘!當比丘這麼知、這麼見時,無明被捨斷,明生起。」



\sutta{80}{80}{無明的捨斷經第二}{https://agama.buddhason.org/SN/sn.php?keyword=35.80}
  那時,\twnr{某位比丘}{39.0}……(中略)說這個:

  「\twnr{大德}{45.0}!有\twnr{一法}{522.0},從對它的捨斷,比丘的\twnr{無明}{207.0}被捨斷,明生起嗎?」

  「\twnr{比丘}{31.0}!有一法,從對它的捨斷,比丘的無明被捨斷,明生起。」

  「大德!那麼,哪一法,從對它的捨斷,比丘的無明被捨斷,明生起?」

  「比丘!無明是一法,從對它的捨斷,比丘的無明被捨斷,明生起。」

  「大德!那麼,當比丘怎樣知、怎樣見時,無明被捨斷,明生起?」

  「比丘!這裡,被比丘聽聞:『一切法都是不足以為了執持的。』比丘!且這樣,這個被比丘聽聞:『一切法都是不足以為了執持的。』他\twnr{證知}{242.0}一切法,證知一切法後,\twnr{遍知}{154.0}一切法,遍知一切法後[\ccchref{MN.37}{https://agama.buddhason.org/MN/dm.php?keyword=37}],從其它的[觀點]看見一切相:從其它的看見眼,諸色……眼識……眼觸……又凡以這眼觸為緣生起感受的樂,或苦,或不苦不樂,也從其他的看見那個。……(中略)從其他的看見意,諸法……意識……意觸……又凡以這意觸為緣生起感受的樂,或苦,或不苦不樂,也從其他的看見那個。

  比丘!當比丘這麼知、這麼見時,無明被捨斷,明生起。」



\sutta{81}{81}{眾多比丘經}{https://agama.buddhason.org/SN/sn.php?keyword=35.81}
  那時,眾多比丘去見\twnr{世尊}{12.0}……(中略)在一旁坐下的那些比丘對世尊說這個:

  「\twnr{大德}{45.0}!這裡,其他外道\twnr{遊行者}{79.0}們這麼問我們:『\twnr{道友}{201.0}們!為了什麼目的在\twnr{沙門}{29.0}\twnr{喬達摩}{80.0}處梵行被住?』

  大德!被這麼問時,我們這麼回答那些其他外道遊行者:『道友們!為了苦的\twnr{遍知}{154.0}在世尊處梵行被住。』

  大德!被這麼問時,當這麼回答時,是否我們就\twnr{是世尊的所說之說者}{115.0},而且不會以不實的誹謗世尊,以及會\twnr{法隨法地回答}{415.0},而任何如法的種種說不會來到應該被呵責處?」

  「比丘們!當被這麼問,你們這麼回答時,確實就是我的所說之說,不以不實的誹謗我,法、隨法地解說,而任何如法的種種說不來到應該被呵責處,比丘們!因為,為了苦的遍知之目的在我這裡(處)梵行被住。

  比丘們!如果其他外道遊行者們再這麼問你們:『道友們!那麼,哪個是苦的,為了那個的遍知在沙門喬達摩處梵行被住?』

  比丘們!被這麼問時,你們應該這麼回答那些其他外道遊行者:『道友們!眼是苦的,為了那個的遍知在世尊處梵行被住;諸色……(中略)又凡以這眼觸\twnr{為緣}{180.0}生起感受的樂,或苦,或不苦不樂,那個也是苦的,為了那個的遍知在世尊處梵行被住……(中略)意是苦的……(中略)又凡以這意觸為緣生起感受的樂,或苦,或不苦不樂,那個也是苦的,為了那個的遍知在世尊處梵行被住,道友們!為了遍知這個苦在世尊處梵行被住。』比丘們!被這麼問,你們應該這麼回答那些其他外道遊行者。」[≃\suttaref{SN.35.152}]



\sutta{82}{82}{世間之問經}{https://agama.buddhason.org/SN/sn.php?keyword=35.82}
  那時,\twnr{某位比丘}{39.0}去見\twnr{世尊}{12.0}……(中略)在一旁坐下的那位比丘對世尊說這個:

  「\twnr{大德}{45.0}!被稱為『世間、世間』,大德!什麼情形被稱為『世間』?」

  「比丘!『被破壞』,因此被稱為『世間』。

  而什麼被破壞?比丘!眼被破壞,諸色被破壞,眼識被破壞,眼觸被破壞,又凡以這眼觸\twnr{為緣}{180.0}生起感受的樂,或苦,或不苦不樂,那也被破壞。……(中略)舌被破壞……(中略)意被破壞,諸法被破壞,意識被破壞,意觸被破壞,又凡以這意觸為緣生起感受的樂,或苦,或不苦不樂,那也被破壞。

  比丘!『被破壞』,因此被稱為『世間』。」



\sutta{83}{83}{帕辜那之問經}{https://agama.buddhason.org/SN/sn.php?keyword=35.83}
  那時,\twnr{尊者}{200.0}帕辜那……(中略)在一旁坐下的尊者帕辜那對\twnr{世尊}{12.0}說這個:

  「\twnr{大德}{45.0}!有那個眼,當在過去已般涅槃、已切斷\twnr{虛妄}{952.0}、\twnr{已切斷路徑}{891.0}、已終結輪迴、已超越一切苦的佛上\twnr{安立}{143.0}時,能以該眼安立嗎?……(中略)大德!有那個舌,當在過去已般涅槃、已切斷虛妄、已切斷路徑、已終結輪迴、已超越一切苦的佛上安立時,能以該舌安立嗎?……(中略)大德!有那個意,當在過去已般涅槃、已切斷虛妄、已切斷路徑、已終結輪迴、已超越一切苦的佛上安立時,能以該意安立嗎?」

  「帕辜那!沒有那個眼,當在過去已般涅槃、已切斷虛妄、已切斷路徑、已終結輪迴、已超越一切苦的佛上安立時,能以該眼安立……(中略)帕辜那!沒有那個舌,當在過去已般涅槃、已切斷虛妄、已切斷路徑、已終結輪迴、已超越一切苦的佛上安立時,能以該舌安立……(中略)帕辜那!沒有那個意,當在過去已般涅槃、已切斷虛妄、已切斷路徑、已終結輪迴、已超越一切苦的佛上安立時,能以該意安立。」

  病人品第八,其\twnr{攝頌}{35.0}:

  「以病人二說,以羅陀三則在後,

   無明二說,以及比丘、世間、帕辜那。」





\pin{闡陀品}{84}{93}
\sutta{84}{84}{壞散法經}{https://agama.buddhason.org/SN/sn.php?keyword=35.84}
  起緣於舍衛城。

  那時,\twnr{尊者}{200.0}阿難去見\twnr{世尊}{12.0}……(中略)在一旁坐下的尊者阿難對世尊說這個:

  「\twnr{大德}{45.0}!被稱為『世間、世間』,大德!什麼情形被稱為『世間』?」

  「阿難!凡壞散法,這在聖者之律中被稱為世間。

  阿難!而什麼是壞散法?阿難!眼是壞散法,諸色是壞散法,眼識是壞散法,眼觸是壞散法,又凡以這眼觸\twnr{為緣}{180.0}……(中略)那也是壞散法。……(中略)舌是壞散法,諸味道是壞散法,舌識是壞散法,舌觸是壞散法,又凡以這舌觸為緣……(中略)那也是壞散法。……(中略)意是壞散法,諸法是壞散法,意識是壞散法,意觸是壞散法,又凡以這意觸為緣生起感受的樂,或苦,或不苦不樂,那也是壞散法。

  阿難!凡壞散法,這在聖者之律中被稱為世間。」



\sutta{85}{85}{世間是空經}{https://agama.buddhason.org/SN/sn.php?keyword=35.85}
  那時,\twnr{尊者}{200.0}阿難……(中略)對\twnr{世尊}{12.0}說這個:

  「\twnr{大德}{45.0}!被稱為『\twnr{世間是空}{x485}、世間是空』,大德!什麼情形被稱為『世間是空』?」

  「阿難!因為\twnr{以我或我所是空}{916.0},因此被稱為『世間是空』。

  阿難!而什麼是以我或我所是空呢?阿難!眼以我或我所是空,諸色以我或我所是空,眼識以我或我所是空,眼觸以我或我所是空……(中略)又凡以意觸\twnr{為緣}{180.0}生起感受的樂,或苦,或不苦不樂,那也以我或我所是空。

  阿難!因為以我或我所是空,因此被稱為『世間是空』。」



\sutta{86}{86}{簡要法經}{https://agama.buddhason.org/SN/sn.php?keyword=35.86}
  在一旁坐下的\twnr{尊者}{200.0}阿難對\twnr{世尊}{12.0}說這個:

  「\twnr{大德}{45.0}!請世尊為我簡要地教導法,凡我聽聞世尊的法後,會住於單獨的、隱離的、不放逸的、熱心的、自我努力的,\twnr{那就好了}{44.0}!」

  「阿難!你怎麼想它:眼是常的,或是無常的?」

  「無常的,大德!」

  「那麼,凡為無常的,那是苦的或樂的?」

  「苦的,大德!」

  「而凡為無常、苦、\twnr{變易法}{70.0},適合認為它:『\twnr{這是我的}{32.0},\twnr{我是這個}{33.0},\twnr{這是我的真我}{34.1}。』嗎?」

  「大德!這確實不是。」

  「諸色是常的,或是無常的?」

  「無常的,大德!」……(中略)

  「眼識……(中略)又凡以這眼觸\twnr{為緣}{180.0}生起感受的樂,或苦,或不苦不樂,那也是常的,或是無常的?」

  「無常的,大德!」

  「那麼,凡為無常的,那是苦的或樂的?」

  「苦的,大德!」

  「而凡為無常、苦、變易法,適合認為它:『\twnr{這是我的}{32.0},\twnr{我是這個}{33.0},這是\twnr{我的真我}{34.0}。』嗎?」

  「大德!這確實不是。」……(中略)。

  「舌是常的,或是無常的?」

  「無常的,大德!」……(中略)

  「舌識……舌觸……(中略)又凡以這意觸為緣生起感受的樂,或苦,或不苦不樂,那也是常的,或是無常的?」

  「無常的,大德!」

  「那麼,凡為無常的,那是苦的或樂的?」

  「苦的,大德!」

  「而凡為無常、苦、變易法,你們認為:『這是我的,我是這個,這是我的真我。』嗎?」

  「大德!這確實不是。」……(中略)。

  「阿難!這麼看的\twnr{有聽聞的聖弟子}{24.0}在眼上\twnr{厭}{15.0}……(中略)在眼觸上也厭……(中略)又凡以這意觸為緣生起感受的樂,或苦,或不苦不樂,在那個上也厭。厭者\twnr{離染}{558.0},從\twnr{離貪}{77.0}被解脫,在已解脫時,\twnr{有『[這是]解脫』之智}{27.0},他知道:『\twnr{出生已盡}{18.0},\twnr{梵行已完成}{19.0},\twnr{應該被作的已作}{20.0},\twnr{不再有此處[輪迴]的狀態}{21.1}。』」 



\sutta{87}{87}{闡陀經}{https://agama.buddhason.org/SN/sn.php?keyword=35.87}
  \twnr{有一次}{2.0},\twnr{世尊}{12.0}住在王舍城栗鼠飼養處的竹林中。

  當時,\twnr{尊者}{200.0}舍利弗與尊者摩訶純陀,以及尊者闡陀住在\twnr{耆闍崛山}{258.0}。

  當時,尊者闡陀在那裡是生病者、受苦者、重病者。

  那時,尊者舍利弗傍晚時,從\twnr{獨坐}{92.0}出來,去見尊者摩訶純陀。抵達後,對尊者摩訶純陀說這個:

  「純陀\twnr{學友}{201.0}!我們走,我們將去見尊者闡陀,為探病者。」「是的,學友!」尊者摩訶純陀回答尊者舍利弗。

  那時,尊者舍利弗與尊者摩訶純陀去見尊者闡陀。抵達後,在設置的座位坐下。坐下後,尊者舍利弗對尊者闡陀說這個:

  「闡陀學友!是否能被你忍受?\twnr{是否能被[你]維持生活}{137.0}?是否苦的感受減退、不增進,減退的結局被知道,非增進?」

  「舍利弗學友!不能被我忍受,不能被[我]維持,我強烈苦的感受增進、不減退,增進的結局被知道,非減退,學友!猶如有力氣的男子以銳利的刀刃在頭上切割。同樣的,學友!激烈的風在頭上切割。

  學友!不能被我忍受,不能被[我]維持……(中略)非減退,學友!猶如有力氣的男子以堅固的皮繩在頭上\twnr{給與纏頭巾}{611.0}。同樣的,學友!在頭上有激烈的頭痛。

  學友!不能被我忍受,不能被[我]維持……(中略)非減退,學友!猶如熟練的屠牛夫或屠牛夫的徒弟,以銳利的牛刀切開腹部。同樣的,學友!激烈的風切開腹部。

  學友!不能被我忍受,不能被[我]維持……(中略)非減退,學友!猶如兩位有力氣的男子對較弱男子在不同的手臂上捉住後,在炭火坑上燒、遍燒。同樣的,學友!在身體中有激烈的熱病。

  學友!不能被我忍受,不能被[我]維持,我強烈苦的感受增進、不減退,增進的結局被知道,非減退,舍利弗學友!我將取刀[自殺],我不期待活命。」

  「尊者闡陀不要取刀,請尊者闡陀存活,我們想要存活的尊者闡陀,如果尊者闡陀沒有適當食物,我將為尊者闡陀遍求適當食物;如果尊者闡陀沒有適當醫藥,我將為尊者闡陀遍求適當醫藥;如果尊者闡陀沒有適當看護者,我將看護尊者闡陀,尊者闡陀不要取刀,請尊者闡陀存活,我們想要存活的尊者闡陀。」

  「舍利弗學友!非沒有我的適當食物,有我的適當食物;非沒有我的適當醫藥,有我的適當醫藥;非沒有我的適當看護,有我的適當看護,學友!此外,大師長久都被我合意、非不合意地\twnr{尊敬}{971.0};學友!因為這對弟子來說是適當的:凡都應該合意、非不合意地尊敬大師,舍利弗學友!『闡陀\twnr{比丘}{31.0}將\twnr{無應該被責備的}{x486}取刀。』請你這麼\twnr{憶持}{57.0}它。」

  「願我們就某點詢問尊者闡陀,如果尊者闡陀為問題之解答給機會。」

  「舍利弗學友!請你問,聽聞後我們將知道。」

  「闡陀學友!你認為眼、眼識、能被眼識所識知的諸法:『\twnr{這是我的}{32.0},\twnr{我是這個}{33.0},這是\twnr{我的真我}{34.0}。』……(中略)闡陀學友!你認為舌、舌識、能被舌識所識知的諸法:『這是我的,我是這個,這是我的真我。』……(中略)闡陀學友!你認為意、意識、能被意識所識知的諸法:『這是我的,我是這個,這是我的真我。』嗎?」

  「舍利弗學友!我認為眼、眼識、能被眼識所識知的諸法:『\twnr{這不是我的}{32.1},\twnr{我不是這個}{33.1},\twnr{這不是我的真我}{34.2}。』……(中略)舍利弗學友!我認為舌、舌識、能被舌識所識知的諸法:『這不是我的,我不是這個,這不是我的真我。』……(中略)舍利弗學友!我認為意、意識、能被意識所識知的諸法:『這不是我的,我不是這個,這不是我的真我。』」

  「闡陀學友!在眼、眼識、能被眼識所識知的諸法上看見什麼、證知什麼後,你認為眼、眼識、能被眼識所識知的諸法:『這不是我的,我不是這個,這不是我的真我。』……(中略)闡陀學友!在舌、舌識、能被舌識所識知的諸法上看見什麼、證知什麼後,你認為舌、舌識、能被舌識所識知的諸法:『這不是我的,我不是這個,這不是我的真我。』……(中略)闡陀學友!在意、意識、能被意識所識知的諸法上看見什麼、證知什麼後,你認為意、意識、能被意識所識知的諸法:『這不是我的,我不是這個,這不是我的真我。』呢?」

  「舍利弗學友!在眼、眼識、能被眼識所識知的諸法上看見\twnr{滅}{68.0}、證知滅後,我認為眼、眼識、能被眼識所識知的諸法:『這不是我的,我不是這個,這不是我的真我。』……(中略)舍利弗學友!在舌、舌識、能被舌識所識知的諸法上看見滅、證知滅後,我認為舌、舌識、能被舌識所識知的諸法:『這不是我的,我不是這個,這不是我的真我。』……(中略)舍利弗學友!在意、意識、能被意識所識知的諸法上看見滅、證知滅後,我認為意、意識、能被意識所識知的諸法:『這不是我的,我不是這個,這不是我的真我。』」

  在這麼說時,尊者摩訶純陀對尊者闡陀說這個:

  「闡陀學友!因此,在這裡,這個那位世尊的教說應該被經常\twnr{好好作意}{43.1}:

  『\twnr{對依止者來說}{x487}有\twnr{搖動}{x488},對無依止者來說沒有搖動,

   在沒有搖動時有\twnr{寧靜}{313.0},在有寧靜時沒有\twnr{傾斜}{x489},

   在沒有傾斜時沒有來去,在沒有來去時沒有死亡與往生,

   在沒有死亡與往生時就不在此世不在他世不在兩者的中間,這就是苦的結束。』」[\ccchref{Ud.74}{https://agama.buddhason.org/Ud/dm.php?keyword=74}]

  那時,尊者舍利弗與尊者摩訶純陀以這個勸誡勸誡尊者闡陀後,從座位起來後離開。

  那時,尊者闡陀在那些(兩位)尊者離開不久,取刀。

  那時,尊者舍利弗去見世尊。抵達後,向世尊\twnr{問訊}{46.0}後,在一旁坐下。在一旁坐下的尊者舍利弗對世尊說這個:

  「\twnr{大德}{45.0}!尊者闡陀已取刀,他的趣處是什麼?來世是什麼?」

  「舍利弗!無應該被責備的被闡陀比丘就在你的面前記說,不是嗎?」

  「大德!有名叫晡玻跋若的跋耆村落,在那裡,尊者闡陀有諸朋友俗家、諸親友俗家、諸應該被斥責的俗家。」

  「舍利弗!確實有闡陀比丘的這些諸朋友俗家、諸親友俗家、諸應該被斥責的俗家,但,舍利弗!我不說:『這個情形有應該被責備的。』

  舍利弗!凡那個身體倒下,他執取另一個身體者,我說:『有應該被責備的。』對闡陀比丘來說,那個不存在。『無應該被責備的被闡陀比丘取刀。』舍利弗!請你這麼憶持它。」[\ccchref{MN.144}{https://agama.buddhason.org/MN/dm.php?keyword=144}]



\sutta{88}{88}{富樓那經}{https://agama.buddhason.org/SN/sn.php?keyword=35.88}
  那時,\twnr{尊者}{200.0}富樓那去見世尊,抵達後……(中略)在一旁坐下的尊者富樓那對世尊說這個:

  「\twnr{大德}{45.0}!請世尊為我簡要地教導法,凡我聽聞世尊的法後,會住於單獨的、隱離的、不放逸的、熱心的、自我努力的,\twnr{那就好了}{44.0}!」

  「富樓那!有能被眼識知的、想要的、所愛的、合意的、可愛形色的、伴隨欲的、誘人的諸色,如果\twnr{比丘}{31.0}歡喜、歡迎、持續固持那個,那位歡喜、歡迎、持續固持那個者的歡喜生起,我說:『富樓那!以歡喜\twnr{集}{67.0}而有苦集。』……(中略)富樓那!有能被舌識知的……(中略)的諸味道富樓那!有能被意識知的、想要的、所愛的、合意的、可愛形色的、伴隨欲的、誘人的諸法,如果比丘歡喜、歡迎、持續固持那個,那位歡喜、歡迎、持續固持那個者的歡喜生起,我說:『富樓那!以歡喜集而有苦集。』

  富樓那!有能被眼識知的、想要的、所愛的、合意的、可愛形色的、伴隨欲的、誘人的諸色,如果比丘不歡喜、不歡迎、不持續固持那個,對那位不歡喜、不歡迎、不持續固持那個者來說歡喜被滅,我說:『富樓那!以歡喜\twnr{滅}{68.0}有苦滅。』……(中略)富樓那!有能被意識知的、想要的、所愛的、合意的、可愛形色的、伴隨欲的、誘人的諸法,如果比丘不歡喜、不歡迎、不持續固持那個,對那位不歡喜、不歡迎、不持續固持那個者來說歡喜被滅,我說:『富樓那!以歡喜滅有苦滅。』

  富樓那!被我以這個簡要的教誡教誡,你將住在什麼地方?」

  「大德!有個叫輸南巴蘭陀的地方,我將住在那裡。」

  「富樓那!輸南巴蘭陀人凶惡,富樓那!輸南巴蘭陀人粗暴,富樓那!如果輸南巴蘭陀人將辱罵、誹謗你,富樓那!在那裡,你將怎麼想?」

  「大德!如果輸南巴蘭陀人將辱罵、誹謗我,在那裡,我將想這個:『這些輸南巴蘭陀人確實是善的,這些輸南巴蘭陀人確實是極善的:凡這些沒以拳對我施與攻擊。』世尊!在那裡,將是這樣,\twnr{善逝}{8.0}!在那裡,將是這樣。」

  「富樓那!但如果輸南巴蘭陀人將以拳對你施與攻擊,富樓那!在那裡,你將怎麼想?」

  「大德!如果輸南巴蘭陀人將以拳對我施與攻擊,在那裡,我將想這個:『這些輸南巴蘭陀人確實是善的,這些輸南巴蘭陀人確實是極善的:凡這些沒以土塊對我施與攻擊。』世尊!在那裡,將是這樣,善逝!在那裡,將是這樣。」

  「富樓那!但如果輸南巴蘭陀人將以土塊對你施與攻擊,富樓那!在那裡,你將怎麼想?」

  「大德!如果輸南巴蘭陀人將以土塊對我施與攻擊,在那裡,我將想這個:『這些輸南巴蘭陀人確實是善的,這些輸南巴蘭陀人確實是極善的:凡這些沒以棍棒對我施與攻擊。』世尊!在那裡,將是這樣,善逝!在那裡,將是這樣。」

  「富樓那!但如果輸南巴蘭陀人將以棍棒對你施與攻擊,富樓那!在那裡,你將怎麼想?」

  「大德!如果輸南巴蘭陀人將以棍棒對我施與攻擊,在那裡,我將想這個:『這些輸南巴蘭陀人確實是善的,這些輸南巴蘭陀人確實是極善的:凡這些沒以刀對我施與攻擊。』世尊!在那裡,將是這樣,善逝!在那裡,將是這樣。」

  「富樓那!但如果輸南巴蘭陀人將以刀對你施與攻擊,富樓那!在那裡,你將怎麼想?」

  「大德!如果輸南巴蘭陀人將以刀對我施與攻擊,在那裡,我將想這個:『這些輸南巴蘭陀人確實是善的,這些輸南巴蘭陀人確實是極善的:凡這些沒以利刃奪我性命。』世尊!在那裡,將是這樣,善逝!在那裡,將是這樣。」

  「富樓那!但如果輸南巴蘭陀人將以利刃奪你性命,富樓那!在那裡,你將怎麼想?」

  「大德!如果輸南巴蘭陀人將以利刃奪我性命,在那裡,將是這樣:『有那位世尊的弟子們以身體與生命為厭惡的、慚恥的、嫌惡的,他們遍求殺手,對我來說這位殺手未被遍求就得到他。』世尊!在那裡,將是這樣,善逝!在那裡,將是這樣。」

  「富樓那!\twnr{好}{44.0}!好!富樓那!具備這調御與寂靜,你將能夠住在輸南巴蘭陀地方。富樓那!現在是那個\twnr{你考量的時間}{84.0}。」

  那時,尊者富樓那歡喜、隨喜世尊所說後,從座位起來、向世尊\twnr{問訊}{46.0}、\twnr{作右繞}{47.0}後,收起臥坐具、拿起衣鉢後,向輸南巴蘭陀地方出發遊行。\twnr{次第地進行著遊行}{61.1},抵達輸南巴蘭陀地方,在那裡,尊者富樓那就住在輸南巴蘭陀地方。

  那時,尊者富樓那就在那個雨季中,使約五百位\twnr{優婆塞}{98.0}知道;就在那個雨季中,使約五百位\twnr{優婆夷}{99.0}知道;就在那個雨季中作證三明;就在那個雨季中\twnr{般涅槃}{72.0}。

  那時,眾多比丘去見世尊……(中略)。

  在一旁坐下的那些比丘對世尊說這個:

  「大德!那位被世尊簡要教誡教誡,名叫富樓那的\twnr{善男子}{41.0}死了,他的趣處是什麼?來世是什麼?」

  「比丘們!富樓那善男子是賢智者,實行法的隨法,且不因為法困擾我。比丘們!善男子富樓那已般涅槃了。」[\ccchref{MN.145}{https://agama.buddhason.org/MN/dm.php?keyword=145}]



\sutta{89}{89}{婆醯雅經}{https://agama.buddhason.org/SN/sn.php?keyword=35.89}
  那時,\twnr{尊者}{200.0}婆醯雅去見\twnr{世尊}{12.0}……(中略)。

  在一旁坐下的尊者婆醯雅對世尊說這個:

  「\twnr{大德}{45.0}!請世尊為我簡要地教導法,凡我聽聞世尊的法後,會住於單獨的、隱離的、不放逸的、熱心的、自我努力的,\twnr{那就好了}{44.0}!」

  「婆醯雅!你怎麼想它:眼是常的,或是無常的?」

  「無常的,大德!」

  「那麼,凡為無常的,那是苦的或樂的?」

  「苦的,大德!」

  「而凡為無常、苦、\twnr{變易法}{70.0},適合認為它:『\twnr{這是我的}{32.0},\twnr{我是這個}{33.0},\twnr{這是我的真我}{34.1}。』嗎?」

  「大德!這確實不是。」

  「諸色是常的,或是無常的?」

  「無常的,大德!」……(中略)眼識……(中略)眼觸……(中略)「又凡以這意觸\twnr{為緣}{180.0}生起感受的樂,或苦,或不苦不樂,那也是常的,或是無常的?」

  「無常的,大德!」

  「那麼,凡為無常的,那是苦的或樂的?」

  「苦的,大德!」

  「而凡為無常、苦、變易法,適合認為它:『\twnr{這是我的}{32.0},\twnr{我是這個}{33.0},這是\twnr{我的真我}{34.0}。』嗎?」

  「大德!這確實不是。」

  「婆醯雅!這麼看的\twnr{有聽聞的聖弟子}{24.0}在眼上\twnr{厭}{15.0},也在諸色上厭,也在眼識上厭,也在眼觸上厭……(中略)又凡以這意觸為緣生起感受的樂,或苦,或不苦不樂,在那個上也厭。厭者\twnr{離染}{558.0},從\twnr{離貪}{77.0}被解脫,在已解脫時,\twnr{有『[這是]解脫』之智}{27.0},他知道:『\twnr{出生已盡}{18.0},\twnr{梵行已完成}{19.0},\twnr{應該被作的已作}{20.0},\twnr{不再有此處[輪迴]的狀態}{21.1}。』」

  那時,尊者婆醯雅歡喜、隨喜世尊所說後,從座位起來、向世尊\twnr{問訊}{46.0}、\twnr{作右繞}{47.0}後離開。

  那時,住於單獨的、隱離的、不放逸的、熱心的、自我努力的尊者婆醯雅不久就以證智自作證後,在當生中\twnr{進入後住於}{66.0}凡\twnr{善男子}{41.0}們為了利益正確地\twnr{從在家出家成為無家者}{48.0}的那個無上梵行結尾,他證知:「\twnr{出生已盡}{18.0},\twnr{梵行已完成}{19.0},\twnr{應該被作的已作}{20.0},\twnr{不再有此處[輪迴]的狀態}{21.1}。」然後尊者婆醯雅成為眾\twnr{阿羅漢}{5.0}之一。



\sutta{90}{90}{擾動經第一}{https://agama.buddhason.org/SN/sn.php?keyword=35.90}
  「\twnr{比丘}{31.0}們!\twnr{擾動}{965.0}是病,擾動是腫瘤,擾動是箭。比丘們!因此,在這裡,\twnr{如來}{4.0}住於不擾動、離箭。

  比丘們!因此,在這裡,如果比丘也希望:『願住於不擾動、離箭。』他不應該\twnr{思量}{963.0}眼,不應該在眼中思量,不應該\twnr{從眼思量}{966.0},不應該思量『眼是我的』;不應該思量諸色,不應該在諸色中思量,不應該從諸色思量,不應該思量『諸色是我的』;不應該思量眼識,不應該在眼識中思量,不應該從眼識思量,不應該思量『眼識是我的』;不應該思量眼觸,不應該在眼觸中思量,不應該從眼觸思量,不應該思量『眼觸是我的』;又凡以這眼觸\twnr{為緣}{180.0}生起感受的樂,或苦,或不苦不樂,那也不應該思量,也不應該在其中思量,也不應該從其思量,不應該思量『那是我的』。

  不應該思量耳……(中略)不應該思量鼻……(中略)不應該思量舌,不應該在舌中思量,不應該從舌思量,不應該思量『舌是我的』;不應該思量諸味道……(中略)不應該思量舌識……(中略)不應該思量舌觸……(中略)又凡以這舌觸為緣生起感受的樂,或苦,或不苦不樂,那也不應該思量,也不應該在其中思量,也不應該從其思量,不應該思量『那是我的』。不應該思量身……(中略)不應該思量意,不應該在意中思量,不應該從意思量,不應該思量『意是我的』;不應該思量諸法……(中略)意識……(中略)意觸……(中略)又凡以這意觸為緣生起感受的樂,或苦,或不苦不樂,那也不應該思量,也不應該在其中思量,也不應該從其思量,不應該思量『那是我的』。

  不應該思量一切,不應該在一切中思量,不應該從一切思量,不應該思量『一切是我的』。

  當這麼不思量時,他在世間中都不執取任何事物;不執取者不\twnr{戰慄}{436.0},不戰慄者\twnr{就自己證涅槃}{71.0},他知道:『\twnr{出生已盡}{18.0},\twnr{梵行已完成}{19.0},\twnr{應該被作的已作}{20.0},\twnr{不再有此處[輪迴]的狀態}{21.1}。』」



\sutta{91}{91}{擾動經第二}{https://agama.buddhason.org/SN/sn.php?keyword=35.91}
  「\twnr{比丘}{31.0}們!\twnr{擾動}{965.0}是病,擾動是腫瘤,擾動是箭。比丘們!因此,在這裡,\twnr{如來}{4.0}住於不擾動、離箭。

  比丘們!因此,在這裡,如果比丘也希望:『願住於不擾動、離箭。』他不應該\twnr{思量}{963.0}眼,不應該在眼中思量,不應該\twnr{從眼思量}{966.0},不應該思量『眼是我的』;不應該思量諸色……眼識……眼觸……又凡以這眼觸為緣生起感受的樂,或苦,或不苦不樂,那也不應該思量,也不應該在其中思量,也不應該從其思量,不應該思量『那是我的』。

  比丘們!因為,凡思量,凡在其中思量,凡從其思量,凡思量『那是我的』者,從那裡它相異地存在,成為相異的、執著存在的世間只歡喜存在者。……(中略)

  不應該思量舌,不應該在舌中思量,不應該從舌思量,不應該思量『舌是我的』;不應該思量諸味道……舌識……舌觸……又凡以這舌觸為緣生起感受的樂,或苦,或不苦不樂,那也不應該思量,也不應該在其中思量,也不應該從其思量,不應該思量『那是我的』。

  比丘們!因為,凡思量,凡在其中思量,凡從其思量,凡思量『那是我的』者,從那裡它相異地存在,成為相異的、執著存在的世間只歡喜存在者。……(中略)

  不應該思量意,不應該在意中思量,不應該從意思量,不應該思量『意是我的』……意識……意觸……又凡以這意觸為緣生起感受的樂,或苦,或不苦不樂,那也不應該思量,也不應該在其中思量,也不應該從其思量,不應該思量『那是我的』。

  比丘們!因為,凡思量,凡在其中思量,凡從其思量,凡思量『那是我的』者,從那裡它相異地存在,成為相異的、執著存在的世間只歡喜存在者。

  比丘們!凡蘊、界、處之所及也都不應該思量,也不應該在其中思量,也不應該從其思量,不應該思量『那是我的』。

  當這麼不思量時,他在世間中都不執取任何事物。不執取者不\twnr{戰慄}{436.0},不戰慄者\twnr{就自己證涅槃}{71.0},他知道:『\twnr{出生已盡}{18.0},\twnr{梵行已完成}{19.0},\twnr{應該被作的已作}{20.0},\twnr{不再有此處[輪迴]的狀態}{21.1}。』」





\sutta{92}{92}{一對經第一}{https://agama.buddhason.org/SN/sn.php?keyword=35.92}
  「\twnr{比丘}{31.0}們!我將為你們教導\twnr{一對}{x490},\twnr{你們要聽}{43.0}它!

  比丘們!而什麼是一對呢?

  即是眼與諸色、耳與諸聲、鼻與諸氣味、舌與諸味道、身與諸\twnr{所觸}{220.2}、意與諸法,比丘們!這被稱為一對。

  比丘們!如果這麼說:『拒絕這一對後,我將\twnr{安立}{143.0}另一個一對。』他的言語會成為無根據的,而當被詢問時不會解答,且更會來到惱害,那是什麼原因?比丘們!那個正如\twnr{不在[感官的]境域中}{783.0}那樣。」[≃\suttaref{SN.35.23}]



\sutta{93}{93}{一對經第二}{https://agama.buddhason.org/SN/sn.php?keyword=35.93}
  「\twnr{比丘}{31.0}們!\twnr{緣於}{252.0}\twnr{一對}{x491}後識生成。比丘們!而怎樣緣一對後識生成?

  緣於眼與諸色後眼識生起,眼是無常的、變易的、變異的;諸色是無常的、變易的、變異的,在這裡,這一對就是動的、搖擺的、無常的、變易的、變異的,眼識是無常的、變易的、變異的,為了眼識生起的該因及該緣,那個因及那個緣也是無常的、變易的、變異的。比丘們!而\twnr{緣於}{252.0}無常\twnr{緣}{180.0}生起的眼識,將從哪裡有常的?

  比丘們!凡這三法的會合、集合、結合,這被稱為眼觸,眼觸也是無常的、變易的、變異的,為了眼觸生起的該因及該緣,那個因及那個緣也是無常的、變易的、變異的。比丘們!而緣於無常緣生起的眼觸,將從哪裡有常的?

  比丘們!被接觸者感受,被接觸者意圖,\twnr{被接觸者認知}{x492},在這裡,這些法就也是動的、搖擺的、無常的、變易的、變異的。

  耳……(中略)緣於舌與諸味道後舌識生起,舌是無常的、變易的、變異的;諸味道是無常的、變易的、變異的,在這裡,這一對就是動的、搖擺的、無常的、變易的、變異的;舌識是無常的、變易的、變異的,為了舌識生起的該因及該緣,那個因及那個緣也是無常的、變易的、變異的。比丘們!而緣於無常緣生起的舌識,將從哪裡有常的?比丘們!凡這三法的會合、集合、結合,這被稱為舌觸,舌觸也是無常的、變易的、變異的,為了舌觸生起的該因及該緣,那個因及那個緣也是無常的、變易的、變異的。比丘們!而緣於無常緣生起的舌觸,將從哪裡有常的?比丘們!被接觸者感受,被接觸者意圖,被接觸者\twnr{認知}{583.0},在這裡,這些法就也是動的、搖擺的、無常的、變易的、變異的。身……(中略)緣於意與諸法後意識生起,意是無常的、變易的、變異的;諸法是無常的、變易的、變異的,在這裡,這一對就是動的、搖擺的、無常的、變易的、變異的;意識是無常的、變易的、變異的,為了意識生起的該因及該緣,那個因及那個緣也是無常的、變易的、變異的。比丘們!而緣於無常緣生起的意識,將從哪裡有常的?比丘們!凡這三法的會合、集合、結合,這被稱為意觸,意觸也是無常的、變易的、變異的。為了意觸生起的該因及該緣,那個因及那個緣也是無常的、變易的、變異的。比丘們!而緣於無常緣生起的意觸,將從哪裡有常的?比丘們!被接觸者感受,被接觸者意圖,被接觸者認知,在這裡,這些法就也是動的、搖擺的、無常的、變易的、變異的。

  比丘們!這樣,緣一對後識生成。」

  闡陀品第九其\twnr{攝頌}{35.0}:

  「壞散、空、簡要,闡陀、富樓那與婆醯雅,

   以及擾動兩說,以一對二則在後。」





\pin{六品}{94}{103}
\sutta{94}{94}{不調御-不護經}{https://agama.buddhason.org/SN/sn.php?keyword=35.94}
  起源於舍衛城。

  「\twnr{比丘}{31.0}們!這些\twnr{六觸處}{78.0}的不調御者、不守護者、不保護者、不自制者,是苦的帶來者,哪六個?

  比丘們!眼觸處不調御者、不守護者、不保護者、不自制者,是苦的帶來者……(中略)比丘們!舌觸處不調御者、不守護者、不保護者、不自制者,是苦的帶來者。……(中略)比丘們!意觸處不調御者、不守護者、不保護者、不自制者,是苦的帶來者。

  比丘們!這些六觸處不調御者、不守護者、不保護者、不自制者,是苦的帶來者。

  比丘們!這些六觸處的善調御者、善守護者、善保護者、善自制者,是樂的帶來者,哪六個?

  比丘們!眼觸處的善調御者、善守護者、善保護者、善自制者,是樂的帶來者……(中略)比丘們!舌觸處的善調御者、善守護者、善保護者、善自制者,是樂的帶來者。……(中略)比丘們!意觸處的善調御者、善守護者、善保護者、善自制者,是樂的帶來者。

  比丘們!這些六觸處的善調御者、善守護者、善保護者、善自制者,是樂的帶來者。」

  \twnr{世尊}{12.0}說這個……(中略)\twnr{大師}{145.0}[又更進一步]說這個:

  「比丘們!正有六觸處,不防護者在該處遭受苦,

   但凡知道對它們自制者,住於\twnr{以信為伴侶}{x493}、無\twnr{流漏的}{188.0}。

   看見悅意的諸色後,還是看見不悅意的後,

   在悅意的上應該除去貪之路,且不應該使『對我不可愛的』沾污意(心)。

   聽聞可愛的與不愛的兩者的聲音後,在可愛的聲音上不應該被迷昏,

   在不可愛的上應該除去來到瞋的,且不應該使『對我不可愛的』沾污意。

   嗅聞芳香的、悅意的氣味後,還是嗅聞不淨、不愉快的後,

   在不愉快的上應該除去嫌惡,且在愉快的上不應該被意欲引誘。

   吃有樂味、美味的味道後,還是有時吃不美味的後,

   對美味的味道不應該染著地吃,在不美味的上不應該顯示嫌惡。

   被樂觸接觸不應該沈醉,被苦的接觸也不應該大動搖,

   在苦樂兩觸上應該變成無關心,不被任何者喜樂、\twnr{妨礙}{x494}。

   任何有\twnr{虛妄}{952.0}想的人,持續虛妄的有想者們進入[輪迴],

   一切\twnr{意所生的}{755.0}與掛慮家的,排除後\twnr{依存於離欲}{825.1}行動。

   當意在六[處]上這麼善\twnr{修習}{94.0}時,對被接觸者來說不管在哪裡心不動搖,

   比丘們!征服那些貪瞋後,令你們成為去生死的彼岸者。」



\sutta{95}{95}{瑪魯迦之子經}{https://agama.buddhason.org/SN/sn.php?keyword=35.95}
  那時,\twnr{尊者}{200.0}瑪魯迦之子去見世尊……(中略)在一旁坐下的尊者瑪魯迦之子對世尊說這個:

  「\twnr{大德}{45.0}!請世尊為我簡要地教導法,凡我聽聞世尊的法後,會住於單獨的、隱離的、不放逸的、熱心的、自我努力的,\twnr{那就好了}{44.0}!」

  「瑪魯迦之子!在這裡,現在,我們將對年輕的\twnr{比丘}{31.0}們說什麼?瑪魯迦之子!確實是因為衰老的、年老的、高齡的、老年的、到達老年的你要求以簡要的教誡。」

  「大德!雖然我是衰老的、年老的、高齡的、老年的、到達老年的,大德!請世尊為我簡要地教導法,請\twnr{善逝}{8.0}簡要地教導法,也許我會了知世尊所說的義理,也許我會成為世尊所說的繼承人。[\ccchref{AN.4.257}{https://agama.buddhason.org/AN/an.php?keyword=4.257}]」

  「瑪魯迦之子!你怎麼想它:凡那些能被眼識知而未看見的、以前未看見的諸色,你未看見與不能被你看見,在那裡,有你的意欲,或貪,或情愛嗎?」

  「大德!這確實不是。」

  「凡那些能被耳識知而未聽聞的、以前未聽聞的諸聲音,你未聽聞與不能被你聽聞,在那裡,有你的意欲,或貪,或情愛嗎?」

  「大德!這確實不是。」

  「凡那些能被鼻識知而未嗅聞的、以前未嗅聞的諸氣味,你未嗅聞與不能被你嗅聞,在那裡,有你的意欲,或貪,或情愛嗎?」

  「大德!這確實不是。」

  「凡那些能被舌識知而未嚐的、以前未嚐的諸味道,你未嚐與不能被你嚐,在那裡,有你的意欲,或貪,或情愛嗎?」

  「大德!這確實不是。」

  「凡那些能被身識知而未接觸的、以前未接觸的諸\twnr{所觸}{220.2},你未接觸與不能被你接觸,在那裡,有你的意欲,或貪,或情愛嗎?」

  「大德!這確實不是。」

  「凡那些能被意識知而未識知的、以前未識知的諸法,你未識知(了知)與不能被你識知,在那裡,有你的意欲,或貪,或情愛嗎?」

  「大德!這確實不是。」

  「瑪魯迦之子!而在這裡,在被你所見、所聞、所覺,所識諸法上,\twnr{在所見的中將只有所見的這麼多}{x495};在所聽聞的中將只有所聽聞的這麼多;在所覺知的中將只有所覺知的這麼多;在所識知的中將只有所識知的這麼多,瑪魯迦之子!當你所見的、所聽聞的、所覺知的,在能被識知的諸法上,在所見的中將只有所見的這麼多;在所聽聞的中將只有所聽聞的這麼多;在所覺知的中將只有所覺知的這麼多;在所識知的中將只有所識知的這麼多,瑪魯迦之子!從那裡,你\twnr{不以那個}{x496},瑪魯迦之子!當你不以那個,瑪魯迦之子!從那裡,你\twnr{不在那裡}{x497},瑪魯迦之子!當你不在那裡,瑪魯迦之子!從那裡,你就不在此世(這裡)、不在他世、不在兩者的中間,這就是苦的結束。」

  「大德!我對這個被世尊簡要地說的義理,詳細地了知:

  『看見色後\twnr{念已忘失者}{216.0},對作意可愛相者,

   心被染著地感受,且持續固持它。

   他的種種感受增長:從色生成,

   \twnr{貪婪}{435.0}與惱害,他的心被傷害,

   對這樣累積苦者,被稱為遠離涅槃。

   聽聞聲音後念已忘失者,對作意可愛相者,

   心被染著地感受,且持續固持它。

   他的種種感受增長:從聲音生成,

   貪婪與惱害,他的心被傷害,

   對這樣累積苦者,被稱為遠離涅槃。

   嗅聞氣味後念已忘失者,對作意可愛相者,

   心被染著地感受,且持續固持它。

   他的種種感受增長:從氣味生成,

   貪婪與惱害,他的心被傷害,

   對這樣累積苦者,被稱為遠離涅槃。

   受用味道後念已忘失者,對作意可愛相者,

   心被染著地感受,且持續固持它。

   他的種種感受增長:從味道生成,

   貪婪與惱害,他的心被傷害,

   對這樣累積苦者,被稱為遠離涅槃。

   \twnr{接觸觸後}{x498},對作意可愛相者,

   心被染著地感受,且持續固持它。

   他的種種感受增長:從觸生成,

   貪婪與惱害,他的心被傷害,

   對這樣累積苦者,被稱為遠離涅槃。

   知道法後念已忘失者,對作意可愛相者,

   心被染著地感受,且持續固持它。

   他的種種感受增長:從法生成,

   貪婪與惱害,他的心被傷害,

   對這樣累積苦者,被稱為遠離涅槃。

   他在色上不被染著:朝向念者看見色後,

   心離染著地感受,且不持續固持它。

   他如是看見色,而且經歷受,

   被滅盡不被堆積,\twnr{他這樣進行念}{x499},

   對這樣減少苦者,被稱為涅槃在面前。

   他在聲音上不被染著:朝向念者聽聲音後,

   心離染著地感受,且不持續固持它。

   他如是聽聞聲音,而且經歷受,

   被滅盡不被堆積,他這樣進行念,

   對這樣減少苦者,被稱為涅槃在面前。

   他在氣味上不被染著:朝向念者嗅聞氣味後,

   心離染著地感受,且不持續固持它。

   他如是嗅氣味,而且經歷受,

   被滅盡不被堆積,他這樣進行念,

   對這樣減少苦者,被稱為涅槃在面前。

   他在味道上不被染著:朝向念者受用味道後,

   心離染著地感受,且不持續固持它。

   他如是享用味道,而且經歷受,

   被滅盡不被堆積,他這樣進行念,

   對這樣減少苦者,被稱為涅槃在面前。

   他在觸上不被染著:朝向念者接觸觸後,

   心離染著地感受,且不持續固持它。

   他如是接觸了觸,而且經歷受,

   被滅盡不被堆積,他這樣進行念,

   對這樣減少苦者,被稱為涅槃在面前。

   他在法上不被染著:朝向念者知道法後,

   心離染著地感受,且不持續固持它。

   他如是知道法,而且經歷受,

   被滅盡不被堆積,他這樣進行念,

   對這樣減少苦者,被稱為涅槃在面前。』

  大德!我對這個被世尊簡要地說的義理,這樣詳細地了知。」

  「瑪魯迦之子!好!好!瑪魯迦之子!好!你了知被我簡要說的詳細義理:

  『看見色後念已忘失者,對作意可愛相者,

   心被染著地感受,且持續固持它。

   他的種種感受增長:從色生成,

   貪婪與惱害,他的心被傷害,

   對這樣累積苦者,被稱為遠離涅槃。

   ……(中略)

   他在法上不被染著:朝向念者知道法後,

   心離染著地感受,且不持續固持它。

   他如是知道法,而且經歷受,

   被滅盡不被堆積,他這樣進行念,

   對這樣減少苦者,被稱為涅槃在面前。』

  瑪魯迦之子!對這個被我簡要地說的,義理應該這樣被詳細地看見。」

  那時,尊者瑪魯迦之子歡喜、隨喜世尊所說後,從座位起來、向世尊\twnr{問訊}{46.0}、\twnr{作右繞}{47.0}後,離開。

  那時,住於單獨的、隱離的、不放逸的、熱心的、自我努力的尊者瑪魯迦之子不久就以證智自作證後,在當生中\twnr{進入後住於}{66.0}凡\twnr{善男子}{41.0}們為了利益正確地\twnr{從在家出家成為無家者}{48.0}的那個無上梵行結尾,他證知:「\twnr{出生已盡}{18.0},\twnr{梵行已完成}{19.0},\twnr{應該被作的已作}{20.0},\twnr{不再有此處[輪迴]的狀態}{21.1}。」然後尊者瑪魯迦之子成為眾\twnr{阿羅漢}{5.0}之一。



\sutta{96}{96}{退失法經}{https://agama.buddhason.org/SN/sn.php?keyword=35.96}
  「\twnr{比丘}{31.0}們!我將為你們教導退失法、不退失法、六\twnr{勝處}{617.0},\twnr{你們要聽}{43.0}它!

  比丘們!而什麼是退失法?

  比丘們!這裡,對比丘來說,以眼見色後,惡不善的、會被結縛的念、意向生起,對它如果比丘容忍、不捨斷、不驅離、不作終結、不使之走到不存在,比丘們!這應該被比丘知道:『我從諸善法退失,因為這被\twnr{世尊}{12.0}稱為退失。』……(中略)。

  再者,比丘們!對比丘來說,以舌嚐味道後,生起……(中略)。

  再者,比丘們!對比丘來說,以意識知法後,惡不善的、會被結縛的念、意向生起,對它如果比丘容忍、不捨斷、不驅離、不作終結、不使之走到不存在,比丘們!這應該被比丘知道:『我從諸善法退失,因為這被世尊稱為退失法。』比丘們!這樣是退失法。

  比丘們!而什麼是不退失法?

  比丘們!這裡,對比丘來說,以眼見色後,惡不善的、會被結縛的念、意向生起,對它如果比丘不容忍、捨斷、驅離、作終結、使之走到不存在,比丘們!這應該被比丘知道:『我不從諸善法上退失,因為這被世尊稱為不退失。』……(中略)。

  再者,比丘們!對比丘來說,以舌嚐味道後,生起……(中略)。

  再者,比丘們!對比丘來說,以意識知法後,惡不善的、會被結縛的念、意向生起,對它如果比丘不容忍、捨斷、驅離、作終結、使之走到不存在,比丘們!這應該被比丘知道:『我不從諸善法上退失,因為這被世尊稱為不退失。』比丘們!這樣是不退失法。

  比丘們!而什麼是六勝處?

  比丘們!這裡,對比丘來說,以眼見色後,惡不善的、會被結縛的念、意向不生起,比丘們!這應該被比丘知道:『這處已打勝,因為這被世尊稱為勝處。』……(中略)。

  再者,比丘們!對比丘來說,以意識知法後,惡不善的、會被結縛的念、意向不生起,比丘們!這應該被比丘知道:『這處已打勝,因為這被世尊稱為勝處。』比丘們!這被稱為六勝處。」



\sutta{97}{97}{住放逸者經}{https://agama.buddhason.org/SN/sn.php?keyword=35.97}
  「\twnr{比丘}{31.0}們!我將為你們教導住放逸者與住不放逸者,\twnr{你們要聽}{43.0}它!

  比丘們!而怎樣是住放逸者?

  比丘們!對眼根住於不\twnr{自制}{217.0}者,在能被眼識知的諸色上心墮落;對那位心墮落者,欣悅不存在;在欣悅不存在時,\twnr{喜}{428.0}不存在;在喜不存在時,\twnr{寧靜}{313.0}不存在;在寧靜不存在時,苦存在;對心苦者,不入定;在心不得定時,諸法不變成明顯;以諸法的不明顯,就名為(就走到稱呼)『住放逸者』。……(中略)比丘們!對舌根住於不自制者,在能被舌識知的諸味道上心墮落;對那位心墮落者……(中略)就名為『住放逸者』。……(中略)比丘們!對意根住於不自制者,在能被意識知的諸法上心墮落;對那位心墮落者,欣悅不存在;在欣悅不存在時,喜不存在;在喜不存在時,\twnr{寧靜}{313.0}不存在;當在寧靜不存在時,苦存在(有苦);對心苦者,不入定;在心不得定時,諸法不變成明顯;以諸法的不明顯,就名為『住放逸者』。比丘們!這樣是住放逸者。

  比丘們!而怎樣是住不放逸者?

  比丘們!對眼根住於自制者,在能被眼識知的諸色上心不墮落;對那位心不墮落者,欣悅被生起;對喜悅者,喜被生起;對\twnr{意喜}{320.0}者,身變得寧靜;\twnr{身已寧靜}{318.0}者\twnr{住於樂}{317.0};對有樂者,心入定;在心得定時,\twnr{諸法變成明顯}{x500};以諸法的明顯,就名為『住不放逸者』。……(中略)比丘們!對舌根住於自制者……心不墮落……(中略)就名為『住不放逸者』。[……(中略)]比丘們!對意根住於自制者,在能被意識知的諸法上心不墮落;對那位心不墮落者,欣悅被生起;對喜悅者,喜被生起;對意喜者,身變得寧靜;身已寧靜者住於樂;對有樂者,心入定;在心得定時,諸法變成明顯;以諸法的明顯,就名為『住不放逸者』。比丘們!這樣是住不放逸者。」



\sutta{98}{98}{自制經}{https://agama.buddhason.org/SN/sn.php?keyword=35.98}
  「\twnr{比丘}{31.0}們!我將為你們教導\twnr{自制}{217.0}與不自制,\twnr{你們要聽}{43.0}它!

  比丘們!而怎樣是不自制?

  比丘們!有能被眼識知的、想要的、所愛的、合意的、可愛形色的、伴隨欲的、誘人的諸色,如果比丘歡喜、歡迎、持續固持那個,比丘們!這應該被比丘知道:『我從諸善法退失,因為這被\twnr{世尊}{12.0}稱為退失。』

  ……(中略)比丘們!有能被舌識知的[、想要的、所愛的、合意的、可愛形色的、伴隨欲的、誘人的]諸味道……(中略)比丘們!有能被意識知的、想要的、所愛的、合意的、可愛形色的、伴隨欲的、誘人的諸法,如果比丘歡喜、歡迎、持續固持那個,比丘們!這應該被比丘知道:『我從諸善法退失,因為這被世尊稱為退失。』比丘們!這樣是不自制。

  比丘們!而怎樣是自制?

  比丘們!有能被眼識知的、想要的、所愛的、合意的、可愛形色的、伴隨欲的、誘人的諸色,如果比丘不歡喜、不歡迎、不持續固持那個,比丘們!這應該被比丘知道:『我不從諸善法上退失,因為這被世尊稱為不退失。』……(中略)

  比丘們!有能被舌識知……的諸味道……(中略)比丘們!有能被意識知的、想要的、所愛的、合意的、可愛形色的、伴隨欲的、誘人的諸法,如果比丘不歡喜、不歡迎、不持續固持那個,比丘們!這應該被比丘知道:『我不從諸善法上退失,因為這被世尊稱為不退失。』比丘們!這樣是自制。」



\sutta{99}{99}{定經}{https://agama.buddhason.org/SN/sn.php?keyword=35.99}
  「\twnr{比丘}{31.0}們!你們要\twnr{修習}{94.0}\twnr{定}{182.0},比丘們!得定的比丘\twnr{如實知道}{x501}。如實知道什麼?

  如實知道『眼是無常的』,如實知道『諸色是無常的』,如實知道『眼識是無常的』,如實知道『眼觸是無常的』,如實知道『又凡以這眼觸為緣生起感受的樂,或苦,或不苦不樂,那也是無常的』。……(中略)如實知道『意是無常的』,諸法……意識……意觸……如實知道『又凡以這意觸為緣生起感受的樂,或苦,或不苦不樂,那也是無常的』。

  比丘們!你們要修習定,比丘們!得定的比丘如實知道。」



\sutta{100}{100}{獨坐經}{https://agama.buddhason.org/SN/sn.php?keyword=35.100}
  「\twnr{比丘}{31.0}們!你們要在\twnr{獨坐}{92.0}上來到努力,比丘們!獨坐的比丘\twnr{如實知道}{x501}。如實知道什麼?

  如實知道『眼是無常的』,如實知道『諸色是無常的』,如實知道『眼識是無常的』,如實知道『眼觸是無常的』,如實知道『又凡以這眼觸為緣生起感受的樂,或苦,或不苦不樂,那也是無常的』。……(中略)如實知道『又凡以這意觸為緣生起感受的樂,或苦,或不苦不樂,那也是無常的』。

  比丘們!你們在獨坐上來到努力,比丘們!獨坐的比丘如實知道。」



\sutta{101}{101}{非你們的經第一}{https://agama.buddhason.org/SN/sn.php?keyword=35.101}
  「\twnr{比丘}{31.0}們!凡非你們的,你們要捨斷它!它被捨斷,對你們將有利益、安樂。比丘們!而什麼是非你們的?

  比丘們!眼是非你們的,你們要捨斷它!它被捨斷,對你們將有利益、安樂;諸色是非你們的,你們要捨斷它們!它們被捨斷,對你們有長久的利益、安樂;眼識是非你們的,你們要捨斷它!它被捨斷,對你們將有利益、安樂;眼觸是非你們的,你們要捨斷它!它被捨斷,對你們將有利益、安樂;又凡以這眼觸\twnr{為緣}{180.0}生起感受的樂,或苦,或不苦不樂,那也不是你們的,你們要捨斷它!它被捨斷,對你們將有利益、安樂。

  耳是非你們的,你們要捨斷它!它被捨斷,對你們將有利益、安樂;諸聲音是非你們的,你們要捨斷它們!它們被捨斷,對你們有長久的利益、安樂;耳識是非你們的,你們要捨斷它!它被捨斷,對你們將有利益、安樂;耳觸是非你們的,你們要捨斷它!它被捨斷,對你們將有利益、安樂;又凡以這耳觸為緣生起感受的樂,或苦,或不苦不樂,那也不是你們的,你們要捨斷它!它被捨斷,對你們將有利益、安樂。

  鼻是非你們的,你們要捨斷它!它被捨斷,對你們將有利益、安樂;諸氣味是非你們的,你們要捨斷它們!它們被捨斷,對你們有長久的利益、安樂;鼻識是非你們的,你們要捨斷它!它被捨斷,對你們將有利益、安樂;鼻觸是非你們的,你們要捨斷它!它被捨斷,對你們將有利益、安樂;又凡以這鼻觸為緣生起感受的樂,或苦,或不苦不樂,那也不是你們的,你們要捨斷它!它被捨斷,對你們將有利益、安樂。

  舌是非你們的,你們要捨斷它!它被捨斷,對你們將有利益、安樂;諸味道是非你們的,你們要捨斷它們!它們被捨斷,對你們有長久的利益、安樂;舌識是非你們的,你們要捨斷它!它被捨斷,對你們將有利益、安樂;舌觸是非你們的,你們要捨斷它!它被捨斷,對你們將有利益、安樂;又凡以這舌觸為緣生起感受的樂,或苦,或不苦不樂,那也不是你們的,你們要捨斷它!它被捨斷,對你們將有利益、安樂。

  ……(中略)意是非你們的,你們要捨斷它!它被捨斷,對你們將有利益、安樂;諸法是非你們的,你們要捨斷它們!它們被捨斷,對你們有長久的利益、安樂;意識是非你們的,你們要捨斷它!它被捨斷,對你們將有利益、安樂;意觸是非你們的,你們要捨斷它!它被捨斷,對你們將有利益、安樂;又凡以這意觸為緣生起感受的樂,或苦,或不苦不樂,那也不是你們的,你們要捨斷它!它被捨斷,對你們將有利益、安樂。

  比丘們!猶如凡在這祇樹林中的草、薪木、枝條、樹葉,[某]人帶走它,或燃燒,或如需要做,是否你們這麼想:『[某]人帶走我們,或燃燒,或如需要做。』呢?」

  「\twnr{大德}{45.0}!這確實不是,那是什麼原因?大德!因為對我們這不是自己,或自己的。」

  「同樣的,比丘們!眼是非你們的,你們要捨斷它!它被捨斷,對你們將有利益、安樂;諸色非你們的……眼識……眼觸……(中略)又凡以這意觸為緣生起感受的樂,或苦,或不苦不樂,那也不是你們的,你們要捨斷它!它被捨斷,對你們將有利益、安樂。」[≃\suttaref{SN.22.33}]



\sutta{102}{102}{非你們的經第二}{https://agama.buddhason.org/SN/sn.php?keyword=35.102}
  「\twnr{比丘}{31.0}們!凡非你們的,你們要捨斷它!它被捨斷,對你們將有利益、安樂。比丘們!而什麼是非你們的?

  比丘們!眼是非你們的,你們要捨斷它!它被捨斷,對你們將有利益、安樂;諸色是非你們的,你們要捨斷它們!它們被捨斷,對你們有長久的利益、安樂;眼識是非你們的,你們要捨斷它!它被捨斷,對你們將有利益、安樂;眼觸是非你們的,你們要捨斷它!它被捨斷,對你們將有利益、安樂……(中略)又凡以這意觸\twnr{為緣}{180.0}生起感受的樂,或苦,或不苦不樂,那也不是你們的,你們要捨斷它!它被捨斷,對你們將有利益、安樂。

  比丘們!凡非你們的,你們要捨斷它!它被捨斷,對你們將有利益、安樂。」



\sutta{103}{103}{優陀羅經}{https://agama.buddhason.org/SN/sn.php?keyword=35.103}
  「\twnr{比丘}{31.0}們!優陀羅羅摩子說這樣的話:『這確實是通曉吠陀者,這確實是戰勝一切者,這確實是未根除腫瘤根的根除者。』比丘們!但當優陀羅羅摩子就是非通曉吠陀者時,他說:『我是通曉吠陀者。』當就是非戰勝一切者時,他說:『我是戰勝一切者。』就是腫瘤根的未根除者,他說:『我的腫瘤根已根除。』比丘們!這裡,那是比丘,當正確說時,應該說:『這確實是通曉吠陀者,這確實是戰勝一切者,這確實是未根除腫瘤根的根除者。』

  比丘們!而怎樣是通曉吠陀者呢?比丘們!當比丘如實知道\twnr{六觸處}{78.0}的\twnr{集起}{67.0}、滅沒、\twnr{樂味}{295.0}、\twnr{過患}{293.0}、\twnr{出離}{294.0},比丘們!這樣的比丘是通曉吠陀者。

  比丘們!而怎樣的比丘是戰勝一切者呢?比丘們!當比丘如實知道六觸處的集起、滅沒、樂味、過患、出離後,不執取後成為解脫者,比丘們!這樣的比丘是戰勝一切者。

  比丘們!而對怎樣的比丘來說是未根除腫瘤根的根除者呢?比丘們!『腫瘤』,這是對這父母生成的、米粥積聚的、無常-塗身-\twnr{按摩}{967.0}-破壞-分散法的\twnr{四大}{646.0}身的同義語,比丘們!『腫瘤根』,這是渴愛的同義語。比丘們!當比丘的渴愛已被捨斷,根已被切斷,\twnr{[如]已斷根的棕櫚樹}{147.1},\twnr{成為非有}{408.0},\twnr{為未來不生之物}{229.0}時,比丘們!對這樣的比丘來說是未根除腫瘤根的根除者。

  比丘們!優陀羅羅摩子說這樣的話:『這確實是通曉吠陀者,這確實是戰勝一切者,這確實是未根除腫瘤根的根除者。』比丘們!但當優陀羅羅摩子就是非通曉吠陀者時,他說:『我是通曉吠陀者。』當就是非戰勝一切者時,他說:『我是戰勝一切者。』就是腫瘤根的未根除者,他說:『我的腫瘤根已根除。』比丘們!這裡,那是比丘,當正確說時,應該說:『這確實是通曉吠陀者,這確實是戰勝一切者,這確實是未根除腫瘤根的根除者。』」

  六品第十,其\twnr{攝頌}{35.0}:

  「二則執著、退失,住放逸者與自制,

   定、獨坐,二則非你們的與優陀羅。」

  六處篇第二個五十則完成,其品的攝頌:

  「無明與鹿網,病人、闡陀為第四,

   以六品為五十,這是第二個五十則。」

  第一個一百則。





\pin{軛安穩者品}{104}{113}
\sutta{104}{104}{軛安穩者經}{https://agama.buddhason.org/SN/sn.php?keyword=35.104}
  起源於舍衛城。 

  「\twnr{比丘}{31.0}們!我將為你們教導\twnr{軛安穩者}{192.0}法門\twnr{法的教說}{562.1},\twnr{你們要聽}{43.0}它!

  比丘們!而什麼是軛安穩者法門法的教說呢?

  比丘們!有能被眼識知的、想要的、所愛的、合意的、可愛形色的、伴隨欲的、誘人的諸色,那些被如來捨斷,根已被切斷,\twnr{[如]已斷根的棕櫚樹}{147.1},\twnr{成為非有}{408.0},\twnr{為未來不生之物}{229.0},他告知為了那些的捨斷之軛(修行),因此,如來被稱為『軛安穩者』。……(中略)

  比丘們!有能被意識知的、想要的、所愛的、合意的、可愛形色的、伴隨欲的、誘人的諸法,那些被如來捨斷,根已被切斷,[如]已斷根的棕櫚樹,成為非有,為未來不生之物,他告知為了那些的捨斷之軛,因此,如來被稱為『軛安穩者』。

  比丘們!這是軛安穩者法門法的教說。」



\sutta{105}{105}{執取經}{https://agama.buddhason.org/SN/sn.php?keyword=35.105}
  「\twnr{比丘}{31.0}們!在有什麼時,執取什麼後,自身內的樂、苦生起呢?」

  「\twnr{大德}{45.0}!我們的法以\twnr{世尊}{12.0}為根本……(中略)。」

  「比丘們!當有眼時,執取眼後,自身內的樂、苦生起。

  ……(中略)當有意時,執取意後,自身內的樂、苦生起。

  比丘們!你們怎麼想它:眼是常的,或是無常的?」

  「無常的,大德!」

  「那麼,凡為無常的,那是苦的或樂的?」

  「苦的,大德!」

  「而凡為無常、苦、\twnr{變易法}{70.0},不執取它後,自身內的樂、苦會生起嗎?」

  「大德!這確實不是。」……(中略)。

  「舌是常的,或是無常的?」

  「無常的,大德!」

  「那麼,凡為無常的,那是苦的或樂的?」

  「苦的,大德!」

  「而凡為無常、苦、變易法,不執取它後,自身內的樂、苦會生起嗎?」

  「大德!這確實不是。」……(中略)。

  「意是常的,或是無常的?」

  「無常的,大德!」

  「那麼,凡為無常的,那是苦的或樂的?」

  「苦的,大德!」

  「而凡為無常、苦、變易法,不執取它後,自身內的樂、苦會生起嗎?」

  「大德!這確實不是。」

  「比丘們!這麼看的\twnr{有聽聞的聖弟子}{24.0}在眼上\twnr{厭}{15.0}……(中略)在意上厭。厭者\twnr{離染}{558.0},從\twnr{離貪}{77.0}被解脫,在已解脫時,\twnr{有『[這是]解脫』之智}{27.0},他知道:『\twnr{出生已盡}{18.0},\twnr{梵行已完成}{19.0},\twnr{應該被作的已作}{20.0},\twnr{不再有此處[輪迴]的狀態}{21.1}。』」[≃\suttaref{SN.22.150}]



\sutta{106}{106}{苦的集起經}{https://agama.buddhason.org/SN/sn.php?keyword=35.106}
  「\twnr{比丘}{31.0}們!我將教導苦的\twnr{集起}{67.0}與滅沒,\twnr{你們要聽}{43.0}它!

  比丘們!而什麼是苦的集起?\twnr{緣於}{252.0}眼與諸色眼識生起,三者的會合有觸,以觸\twnr{為緣}{180.0}有受(而受存在),以受為緣有渴愛,這是苦的集起。……(中略)

  緣於舌與諸味道後舌識生起,三者的會合有觸,以觸為緣有受,以受為緣有渴愛,這是苦的集起。……(中略)緣於意與諸法生起意識,三者的會合為觸,以觸為緣有受,以受為緣有渴愛,這是苦的集起。

  比丘們!而什麼是苦的滅沒?緣於眼與諸色眼識生起,三者的會合有觸,以觸為緣有受,以受為緣有渴愛,就以那個渴愛的\twnr{無餘褪去與滅}{491.0}有取\twnr{滅}{68.0}(而取滅存在),以取滅有有滅,以有滅有生滅,以生滅而老、死、愁、悲、苦、憂、\twnr{絕望}{342.0}被滅,這樣是這整個苦蘊的滅,這是苦的滅沒。……(中略)

  緣於舌與諸味道後舌識生起……(中略)緣於意與諸法生起意識,三者的會合為觸,以觸為緣有受,以受為緣有渴愛,就以那個渴愛的無餘褪去與滅有取滅,以取滅有有滅,以有滅有生滅,以生滅而老、死、愁、悲、苦、憂、絕望被滅,這樣是這整個苦蘊的滅,比丘們!這是苦的滅沒。」[\suttaref{SN.12.43}]



\sutta{107}{107}{世間的集起經}{https://agama.buddhason.org/SN/sn.php?keyword=35.107}
  「\twnr{比丘}{31.0}們!我將教導世間的\twnr{集起}{67.0}與滅沒,\twnr{你們要聽}{43.0}它!

  比丘們!而什麼是世間的集起?\twnr{緣於}{252.0}眼與諸色眼識生起,三者的會合有觸,以觸\twnr{為緣}{180.0}有受(而受存在),以受為緣有渴愛,以渴愛為緣有取,以取為緣有有,以有為緣有生,以生為緣而老、死、愁、悲、苦、憂、\twnr{絕望}{342.0}生成,比丘們!這是世間的集起。……(中略)

  緣於舌與諸味道後舌識生起……(中略)緣於意與諸法生起意識,三者的會合有觸,以觸為緣有受,以受為緣有渴愛,以渴愛為緣有取,以取為緣有有,以有為緣有生,以生為緣而老、死、愁、悲、苦、憂、絕望生成,比丘們!這是世間的集起。

  比丘們!而什麼是世間的滅沒?緣於眼與諸色眼識生起,三者的會合有觸,以觸為緣有受,以受為緣有渴愛,就以那個渴愛的\twnr{無餘褪去與滅}{491.0}有取\twnr{滅}{68.0}(而取滅存在),以取滅有有滅,以有滅有生滅,以生滅而老、死、愁、悲、苦、憂、絕望被滅,這樣是這整個苦蘊的滅,比丘們!這是世間的滅沒。……(中略)緣於舌與諸味道生起……(中略)緣於意與諸法生起意識,三者的會合為觸,以觸為緣有受,以受為緣有渴愛,就以那個渴愛的無餘褪去與滅有取滅……(中略)這樣是這整個苦蘊的滅,比丘們!這是世間的滅沒。」[\suttaref{SN.12.41}]



\sutta{108}{108}{我是優勝者經}{https://agama.buddhason.org/SN/sn.php?keyword=35.108}
  「\twnr{比丘}{31.0}們!在有什麼時,執取什麼後,執著什麼後,有『我是優勝者』,或有『我是同等者』,或有『我是下劣者』呢?」

  「\twnr{大德}{45.0}!我們的法以\twnr{世尊}{12.0}為根本[……(中略)]。」

  「比丘們!在有眼時,執取眼後,執著眼後,有『我是優勝者』,或有『我是同等者』,或有『我是下劣者』。

  ……(中略)在有舌時……(中略)在有意時,執取意後,執著意後,有『我是優勝者』,或有『我是同等者』,或有『我是下劣者』。

  比丘們!你們怎麼想它:眼是常的,或是無常的?」

  「無常的,大德!」

  「那麼,凡為無常的,那是苦的或樂的?」

  「苦的,大德!」

  「而凡為無常、苦、\twnr{變易法}{70.0},不執取它後,是否有『我是優勝者』,或有『我是同等者』,或有『我是下劣者』呢?」

  「大德!這確實不是。」……(中略)。

  「舌……身是常的,或是無常的?」

  「無常的,大德!」……(中略)。

  「意是常的,或是無常的?」

  「無常的,大德!」

  「那麼,凡為無常的,那是苦的或樂的?」

  「苦的,大德!」

  「而凡為無常、苦、變易法,不執取它後,是否有『我是優勝者』,或有『我是同等者』,或有『我是下劣者』呢?」

  「大德!這確實不是。」

  「比丘們!這麼看的\twnr{有聽聞的聖弟子}{24.0}在眼上\twnr{厭}{15.0}……(中略)在意上厭。厭者\twnr{離染}{558.0},從\twnr{離貪}{77.0}被解脫,在已解脫時,\twnr{有『[這是]解脫』之智}{27.0},他知道:『\twnr{出生已盡}{18.0},\twnr{梵行已完成}{19.0},\twnr{應該被作的已作}{20.0},\twnr{不再有此處[輪迴]的狀態}{21.1}。』」



\sutta{109}{109}{會被結縛經}{https://agama.buddhason.org/SN/sn.php?keyword=35.109}
  「\twnr{比丘}{31.0}們!我將教導\twnr{會被結縛的諸法}{666.0}與結,\twnr{你們要聽}{43.0}它!

  比丘們!而什麼是會被結縛的諸法?而什麼是結呢?

  比丘們!眼是會被結縛的法;凡在那裡有意欲貪者,在那裡有結。……(中略)舌是會被結縛的法……(中略)意是會被結縛的法;凡在那裡有意欲貪者,在那裡有結。

  比丘們!這些被稱為會被結縛的法,這個是結。」[≃\suttaref{SN.22.120}, \suttaref{SN.35.122}]



\sutta{110}{110}{與執取有關的經}{https://agama.buddhason.org/SN/sn.php?keyword=35.110}
  「\twnr{比丘}{31.0}們!我將教導\twnr{與執取有關的}{551.0}諸法與執取,\twnr{你們要聽}{43.0}它!

  比丘們!什麼是與執取有關的諸法?以及什麼是執取?

  比丘們!眼是與執取有關的法;凡在那裡有意欲貪者,在那裡有執取。……(中略)舌是與執取有關的法……(中略)意是與執取有關的法;凡在那裡有意欲貪者,在那裡有執取。

  比丘們!這些被稱為與執取有關的法,這個是執取。」[≃\suttaref{SN.22.121}, \suttaref{SN.35.123}]



\sutta{111}{111}{內處遍知經}{https://agama.buddhason.org/SN/sn.php?keyword=35.111}
  「\twnr{比丘}{31.0}們!不\twnr{證知}{242.0}、不\twnr{遍知}{154.0}、\twnr{不離貪}{77.1}、不捨斷眼者是對苦滅盡的不可能者。

  耳……鼻……舌……身……不證知、不遍知、不離貪、不捨斷意者是對苦滅盡的不可能者。

  比丘們!證知、遍知、離貪、捨斷眼者是對苦滅盡的可能者。

  耳……鼻……舌……身……證知、遍知、離貪、捨斷意者是對苦滅盡的可能者。」[≃\suttaref{SN.22.24}, \suttaref{SN.35.26}, \suttaref{SN.35.27}, \suttaref{SN.35.112}]



\sutta{112}{112}{外處遍知經}{https://agama.buddhason.org/SN/sn.php?keyword=35.112}
  「比丘們!不\twnr{證知}{242.0}、不\twnr{遍知}{154.0}、\twnr{不離貪}{77.1}、不捨斷諸色者是對苦滅盡的不可能者。

  諸聲音……諸氣味……諸味道……諸\twnr{所觸}{220.2}……不證知、不遍知、不離貪、不捨斷諸法者是對苦滅盡的不可能者。

  比丘們!證知、遍知、離貪、斷諸色者是對苦滅盡的可能者。

  諸聲音……諸氣味……諸味道……諸\twnr{所觸}{220.2}……證知、遍知、離貪、斷諸法者是對苦滅盡的可能者。」[≃\suttaref{SN.22.24}, \suttaref{SN.35.26}, \suttaref{SN.35.27}, \suttaref{SN.35.112}]



\sutta{113}{113}{屏息側聽經}{https://agama.buddhason.org/SN/sn.php?keyword=35.113}
  \twnr{有一次}{2.0},\twnr{世尊}{12.0}住在\twnr{親戚村}{709.0}的磚屋中。

  那時,獨處、\twnr{獨坐}{92.0}的世尊說這\twnr{法的教說}{562.1}:  

  「\twnr{緣於}{252.0}眼與諸色眼識生起,三者的會合有觸,以觸\twnr{為緣}{180.0}有受(而受存在),以受為緣有渴愛,以渴愛為緣有取,以取為緣有有,以有為緣有生;以生為緣而老、死、愁、悲、苦、憂、\twnr{絕望}{342.0}生成,這樣是這整個\twnr{苦蘊}{83.0}的\twnr{集}{67.0}。……(中略)

  緣於舌與諸味道生起……(中略)緣於意與諸法生起意識,三者的會合為觸,以觸為緣有受,以受為緣有渴愛,以渴愛為緣有取,以取為緣有有,以有為緣有生;以生為緣而老、死、愁、悲、苦、憂、絕望生成,這樣是這整個\twnr{苦蘊}{83.0}的\twnr{集}{67.0}。

  緣於眼與諸色眼識生起,三者的會合有觸,以觸為緣有受,以受為緣有渴愛,就以那個渴愛的\twnr{無餘褪去與滅}{491.0}而取\twnr{滅}{68.0}(而取滅存在),以取滅有有滅,以有滅有生滅,以生滅而老、死、愁、悲、苦、憂、絕望被滅,這樣是這整個苦蘊的滅。……(中略)

  緣於舌與諸味道生起……(中略)緣於意與諸法生起意識,三者的會合為觸,以觸為緣有受,以受為緣有渴愛,就以那個渴愛的無餘褪去與滅有取滅……(中略)這樣是這整個苦蘊的滅。」

  當時,\twnr{某位比丘}{39.0}站立對世尊屏息側聽。世尊看見那位站立屏息側聽的比丘。看見後,對那位比丘說這個:

  「比丘!你聽見這個法的教說了嗎?」

  「是的,\twnr{大德}{45.0}!」

  「比丘!你要學習這個法的教說,比丘!你要學得這個法的教說,比丘!你要\twnr{憶持}{57.0}這個法的教說,比丘!這個法的教說是\twnr{伴隨利益的}{50.0},是\twnr{梵行基礎的}{446.0}。」[\suttaref{SN.12.45}]

  軛安穩品第十一,其\twnr{攝頌}{35.0}:

  「軛安穩、執取,苦、世間、優勝者,

   結、執取,遍知二則、屏息側聽。」





\pin{世間與欲的種類品}{114}{123}
\sutta{114}{114}{魔網經第一}{https://agama.buddhason.org/SN/sn.php?keyword=35.114}
  「\twnr{比丘}{31.0}們!有能被眼識知的、想要的、所愛的、合意的、可愛形色的、伴隨欲的、誘人的諸色,如果比丘歡喜、歡迎、持續固持那個,比丘們!這被稱為進入魔的住處的、進入魔的控制的、被魔捕網緊綁的比丘,他被魔繫縛繫縛、被\twnr{波旬}{49.0}為所欲為。……(中略)比丘們!有能被舌識知的、想要的、所愛的、合意的、可愛形色的、伴隨欲的、誘人的諸味道,如果比丘歡喜、歡迎、持續固持那個,比丘們!這被稱為進入魔的住處的、進入魔的控制的、被魔捕網緊綁的比丘,他被魔繫縛繫縛、被波旬為所欲為。……(中略)比丘們!有能被意識知的、想要的、所愛的、合意的、可愛形色的、伴隨欲的、誘人的諸法,如果比丘歡喜、歡迎、持續固持那個,比丘們!這被稱為進入魔的住處的、進入魔的控制的、被魔捕網緊綁的比丘,他被魔繫縛繫縛、被波旬為所欲為。

  比丘們!而有能被眼識知的、想要的、所愛的、合意的、可愛形色的、伴隨欲的、誘人的諸色,如果比丘不歡喜、不歡迎、不持續固持那個,比丘們!這被稱為不進入魔的住處的、不進入魔的控制的、被魔捕網鬆開的比丘,他被魔繫縛釋放、不被波旬為所欲為。……(中略)比丘們!有能被舌識知的、想要的、所愛的、合意的、可愛形色的、伴隨欲的、誘人的諸味道,如果比丘不歡喜、不歡迎、不持續固持那個,比丘們!這被稱為不進入魔的住處的、不進入魔的控制的、被魔捕網鬆開的比丘,他被魔繫縛釋放、不被波旬為所欲為。……(中略)比丘們!有能被意識知的、想要的、所愛的、合意的、可愛形色的、伴隨欲的、誘人的諸法,如果比丘不歡喜、不歡迎、不持續固持那個,比丘們!這被稱為不進入魔的住處的、不進入魔的控制的、被魔捕網鬆開的比丘,他被魔繫縛釋放、不被波旬為所欲為。」



\sutta{115}{115}{魔網經第二}{https://agama.buddhason.org/SN/sn.php?keyword=35.115}
  「\twnr{比丘}{31.0}們!有能被眼識知的、想要的、所愛的、合意的、可愛形色的、伴隨欲的、誘人的諸色,如果比丘歡喜、歡迎、持續固持那個,比丘們!這被稱為在能被眼識知的諸色上繫縛的、進入魔的住處的、進入魔的控制的、被魔捕網緊綁的比丘,他被魔繫縛繫縛、被\twnr{波旬}{49.0}為所欲為。……(中略)比丘們!有能被舌識知……的諸味道……(中略)比丘們!有能被意識知的、想要的、所愛的、合意的、可愛形色的、伴隨欲的、誘人的諸法,如果比丘歡喜、歡迎、持續固持那個,比丘們!這被稱為在能被意識知的諸法上繫縛的、進入魔的住處的、進入魔的控制的、被魔捕網緊綁的比丘,他被魔繫縛繫縛、被波旬為所欲為。{……(中略)。}

  比丘們!而有能被眼識知的、想要的、所愛的、合意的、可愛形色的、伴隨欲的、誘人的諸色,如果比丘不歡喜、不歡迎、不持續固持那個,比丘們!這被稱為在能被眼識知的諸色上釋放的、不進入魔的住處的、不進入魔的控制的、被魔捕網鬆開的比丘,他被魔繫縛釋放、不被波旬為所欲為。……(中略)比丘們!有能被舌識知……的諸味道……(中略)比丘們!有能被意識知的、想要的、所愛的、合意的、可愛形色的、伴隨欲的、誘人的諸法,如果比丘不歡喜、不歡迎、不持續固持那個,比丘們!這被稱為在能被意識知的諸法上釋放的、不進入魔的住處的、不進入魔的控制的、被魔捕網鬆開的比丘,他被魔繫縛釋放、不被波旬為所欲為。」



\sutta{116}{116}{世間邊之行經}{https://agama.buddhason.org/SN/sn.php?keyword=35.116}
  「\twnr{比丘}{31.0}們!我不說:『世界的邊以行走能被知道、能被看見、能被到達。』比丘們!而且我也不說:『未到達世界的邊後,有\twnr{苦的作終結}{54.0}。』」[\suttaref{SN.2.26}]

  \twnr{世尊}{12.0}說這個後,從座位起來後進入住處。

  那時,當世尊離開不久,那些比丘想這個:

  「學友們!世尊為我們簡要地誦說這個總說,未詳細地解析義理後,從座位起來後已進入住處:『比丘們!我不說:「世界的邊以行走能被知道、能被看見、能被到達。」比丘們!而且我也不說:「未到達世界的邊後,有苦的作終結。」』誰對被世尊以簡要誦說的這個總說、未詳細地解析的義理,會詳細地解析義理呢?」

  那時,那些比丘想這個:

  「這位\twnr{尊者}{200.0}阿難正是\twnr{大師}{145.0}稱讚者,以及有智的同梵行者們的尊重者,且尊者阿難能夠對被世尊以簡要誦說的這個總說、未被詳細地解析的義理,詳細地解析義理,讓我們去見尊者阿難。抵達後,詢問尊者阿難這個義理。」

  那時,那些比丘去見尊者阿難。抵達後,與尊者阿難一起互相問候。交換應該被互相問候的友好交談後,在一旁坐下。在一旁坐下的那些比丘對尊者阿難說這個:

  「阿難\twnr{學友}{201.0}!世尊為我們簡要地誦說這個總說,未詳細地解析義理後,從座位起來後已進入住處:『比丘們!我不說:「世界的邊以行走能被知道、能被看見、能被到達。」比丘們!而且我也不說:「未到達世界的邊後,有苦的作終結。」』學友!當世尊離開不久,那些我們想這個:『學友們!世尊為我們簡要地誦說這個總說,未詳細地解析義理後,從座位起來後已進入住處:「比丘們!我不說:『世界的邊以行走能被知道、能被看見、能被到達。』比丘們!而且我也不說:『未到達世界的邊後,有苦的作終結。』」誰對被世尊以簡要誦說的這個總說、未詳細地解析的義理,會詳細地解析義理呢?』學友!我們想這個:『這位尊者阿難正是大師稱讚者,以及有智的同梵行者們的尊重者,且尊者阿難能夠對被世尊以簡要誦說的這個總說、未被詳細地解析的義理,詳細地解析義理,讓我們去見尊者阿難。抵達後,詢問尊者阿難這個義理。』請尊者阿難解析。」

  「學友們!猶如欲求\twnr{心材}{356.0},找尋心材,遍求心材的男子,當走到有心材住立的大樹時,就越過根後,越過樹幹後,想在枝葉中心材應該被遍求,尊者們就像這樣,在大師是面對者時,越過那位世尊後,你們想這個義理我們應該被詢問。學友們!因為,那位世尊是知道者,他知道、是見者,\twnr{他看見}{660.0}、是\twnr{眼已生者}{455.0}、\twnr{智已生者}{456.0}、\twnr{法已生者}{457.0}、\twnr{梵已生者}{458.0}、解說者、宣說者、義理的引導者、\twnr{不死}{123.0}的施與者、\twnr{法主}{139.0}、\twnr{如來}{4.0}。而那正是為了這個的時機:你們正應該詢問世尊這個義理,你們應該如世尊會為你們解說那樣憶持。」

  「阿難學友!確實,那位世尊是知道者,他知道、是見者,他看見、是眼已生者、智已生者、法已生者、梵已生者、解說者、宣說者、義理的引導者、不死的施與者、法主、如來。而那正是為了這個的時機:我們正應該詢問世尊這個義理,如世尊會為我們解說,像那樣應該使我們憶持它。但,尊者阿難正是大師稱讚者,以及有智的同梵行者們的尊重者,且尊者阿難能夠對被世尊以簡要誦說的這個總說、未被詳細地解析的義理,詳細地解析義理,請尊者阿難不辭麻煩地解析。」

  「學友們!那樣的話,請你們聽!請你們\twnr{好好作意}{43.1}!我將說。」

  「是的,學友!」那些比丘們回答尊者阿難。

  尊者阿難說這個:

  「學友們!凡世尊為你們簡要地誦說這個總說,未詳細地解析義理後,從座位起來後已進入住處:『比丘們!我不說:『世界的邊以行走能被知道、能被看見、能被到達。』比丘們!而且我也不說:『未到達世界的邊後,有苦的作終結。』』學友們!我對被世尊以簡要誦說的這個總說、未詳細地解析的義理,這樣詳細地了知義理:

  學友們!凡以之為世間中之世間想者、世間思考者,這被稱為聖者之律中的世間。學友們!而以什麼為世間中之世間想者、世間思考者呢?學友們!以眼為世間中之世間想者、世間思考者。學友們!以耳……學友們!以鼻……學友們!以舌為世間中之世間想者、世間思考者。學友們!以身……學友們!以意為世間中之世間想者、世間思考者。學友們!凡以之為世間中之世間想者、世間思考者,這被稱為聖者之律中的世間。

  學友們!凡世尊為你們簡要地誦說這個總說,未詳細地解析義理後,從座位起來後已進入住處:『比丘們!我不說:『世界的邊以行走能被知道、能被看見、能被到達。』比丘們!而且我也不說:『未到達世界的邊後,有苦的作終結。』我對被世尊以簡要誦說的這個總說、未詳細地解析的義理,這樣詳細地了知義理。\twnr{尊者}{200.0}們!還有,當希望時,你們就去見世尊後應該詢問這個義理,你們應該如世尊為你們解說那樣憶持它。」

  「是的,學友!」那些比丘回答尊者阿難後,從座位起來後去見世尊。抵達後,向世尊\twnr{問訊}{46.0}後,在一旁坐下。在一旁坐下的那些比丘對世尊說這個:

  「\twnr{大德}{45.0}!世尊為我們簡要地誦說這個總說,未詳細地解析義理後,從座位起來後已進入住處:『比丘們!我不說:「世界的邊以行走能被知道、能被看見、能被到達。」比丘們!而且我也不說:「未到達世界的邊後,有苦的作終結。」』大德!當世尊離開不久,那些我們想這個:『學友們!這裡,世尊為我們簡要地誦說這個總說,未詳細地解析義理後,從座位起來後已進入住處:「比丘們!我不說:『世界的邊以行走能被知道、能被看見、能被到達。』比丘們!而且我也不說:『未到達世界的邊後,有苦的作終結。』」誰對被世尊以簡要誦說的這個總說、未詳細地解析的義理,會詳細地解析義理呢?』大德!我們想這個:『這位尊者阿難正是大師稱讚者,以及有智的同梵行者們的尊重者,且尊者阿難能夠對被世尊以簡要誦說的這個總說、未被詳細地解析的義理,詳細地解析義理,讓我們去見尊者阿難。抵達後,詢問尊者阿難這個義理。』大德!那時,我們去見尊者阿難。抵達後,我們詢問尊者阿難這個義理。

  大德!以這些方式(行相)、以這些語詞、以這些文句被尊者阿難為那些我們解析義理。」

  「比丘們!阿難是賢智者;比丘們!阿難是大智慧者。比丘們!如果你們也詢問我這個義理,我也正是這樣解說它,如它被阿難解說。這就是這個的義理,你們應該這樣憶持它。」



\sutta{117}{117}{欲種類經}{https://agama.buddhason.org/SN/sn.php?keyword=35.117}
  「\twnr{比丘}{31.0}們!當就在我\twnr{正覺}{185.1}以前,還是未\twnr{現正覺}{75.0}的\twnr{菩薩}{186.0}時想這個:『凡我的心在以前所觸的、已過去的、被滅的、變易的\twnr{五種欲}{187.0},在那裡,當我的心走入時,會多走入,或在現在的,或少量在未來的。』

  比丘們!那個我想這個:『凡我的心在以前所觸的、已過去的、被滅的、變易的五種欲,在那裡,以為了我自己,我的心的不放逸、念、守護應該被建立(作)。』

  比丘們!因此,在這裡,凡你們的心在以前所觸的、已過去的、被滅的、變易的那些五種欲,在那裡,當你們的心走入時,也會多走入,或在現在的,或少量在未來的。

  比丘們!因此,在這裡,凡你們的心在以前所觸的、已過去的、被滅的、變易的那些五種欲,在那裡,以為了你們自己,你們心的不放逸、念、守護應該被建立。

  比丘們!因此,在這裡,那些處應該被知道(感受):眼被滅與色想被滅之處,那些處應該被知道……(中略)舌被滅與味道想被滅之處,那些處應該被知道……(中略)意被滅與法想被滅之處,那些處應該被知道。」

  \twnr{世尊}{12.0}說這個後,從座位起來後進入住處。

  那時,當世尊離開不久,那些比丘想這個:

  「學友們!世尊為我們簡要地誦說這個總說,未詳細地解析義理後,從座位起來後已進入住處:『比丘們!因此,在這裡,那些處應該被知道:眼被滅與色想被滅之處,那些處應該被知道……(中略)舌被滅與味道想被滅之處,那些處應該被知道……(中略)意被滅與法想被滅之處,那些處應該被知道。』誰對被世尊以簡要誦說的這個總說、未詳細地解析的義理,會詳細地解析義理呢?」

  那時,那些比丘想這個:

  「這位\twnr{尊者}{200.0}阿難正是大師稱讚者,以及有智的同梵行者們的尊重者,且尊者阿難能夠對被世尊以簡要誦說的這個總說、未被詳細地解析的義理,詳細地解析義理。讓我們去見尊者阿難。抵達後,詢問尊者阿難這個義理。」

  那時,那些比丘去見尊者阿難。抵達後,與尊者阿難一起互相問候。交換應該被互相問候的友好交談後,在一旁坐下。在一旁坐下的那些比丘對尊者阿難說這個:

  「阿難\twnr{學友}{201.0}!這裡,世尊為我們簡要地誦說這個總說,未詳細地解析義理後,從座位起來後已進入住處:『比丘們!因此,在這裡,那些處應該被知道:眼被滅與色想被滅之處,那些處應該被知道……(中略)舌被滅與味道想被滅之處,那些處應該被知道……(中略)意被滅與法想被滅之處,那些處應該被知道。』學友!當世尊離開不久,那些我們想這個:『學友們!世尊為我們簡要地誦說這個總說,未詳細地解析義理後,從座位起來後已進入住處:「比丘們!因此,在這裡,那些處應該被知道:眼被滅與色想被滅之處,那些處應該被知道……(中略)舌被滅與味道想被滅之處,那些處應該被知道……(中略)意被滅與法想被滅之處,那些處應該被知道。」誰對被世尊以簡要誦說的這個總說、未詳細地解析的義理,會詳細地解析義理呢?』學友!我們想這個:『這位尊者阿難正是大師稱讚者,以及有智的同梵行者們的尊重者,且尊者阿難能夠對被世尊以簡要誦說的這個總說、未被詳細地解析的義理,詳細地解析義理。讓我們去見尊者阿難。抵達後,詢問尊者阿難這個義理。』請尊者阿難解說。」

  「學友們!猶如欲求\twnr{心材}{356.0},找尋心材,遍求心材的男子,當走到有心材[住立的]大樹時……(中略)。」

  「……請尊者阿難不辭麻煩地解析。」

  「學友們!那樣的話,你們要聽!你們要\twnr{好好作意}{43.1}!我將說。」

  「是的,學友!」那些比丘回答尊者阿難。

  尊者阿難說這個:

  「學友們!凡世尊為你們簡要地誦說這個總說,未詳細地解析義理後,從座位起來後已進入住處:『比丘們!因此,在這裡,那些處應該被知道:眼被滅與色想被滅之處,那些處應該被知道……(中略)意被滅與法想被滅之處,那些處應該被知道。』學友們!我對被世尊以簡要誦說的這個總說、未詳細地解析的義理,這樣詳細地了知義理:

  學友們!確實關於六處被滅,這被世尊說:『比丘們!因此,在這裡,那些處應該被知道:眼被滅與色想被滅之處,那些處應該被知道……(中略)意被滅與法想被滅之處,那些處應該被知道。』

  學友們!這位世尊簡要地誦說總說、未詳細地解析義理後,從座位起來後已進入住處:『比丘們!因此,在這裡,那些處應該被知道:眼被滅與色想被滅之處,那些處應該被知道……(中略)意被滅與法想被滅之處,那些處應該被知道。』我對被世尊以簡要誦說的這個總說、未詳細地解析的義理,這樣詳細地了知義理。\twnr{尊者}{200.0}們!還有,當希望時,請你們就去見世尊,抵達後應該詢問這個義理,你們應該如世尊為你們解說那樣憶持它。」

  「是的,學友!」那些比丘回答尊者阿難後,從座位起來後去見世尊。抵達後,向世尊\twnr{問訊}{46.0}後,在一旁坐下。在一旁坐下的那些比丘對世尊說這個:

  「\twnr{大德}{45.0}!世尊為我們簡要地誦說這個總說,未詳細地解析義理後,從座位起來後已進入住處:『比丘們!因此,在這裡,那些處應該被知道:眼被滅與色想被滅之處,那些處應該被知道……(中略)舌被滅與味道想被滅之處,那些處應該被知道……(中略)意被滅與法想被滅之處,那些處應該被知道。』

  大德!當世尊離開不久,那些我們想這個:『學友們!這裡,世尊為我們簡要地誦說這個總說,未詳細地解析義理後,從座位起來後已進入住處:「比丘們!因此,在這裡,那些處應該被知道:眼被滅與色想被滅之處,那些處應該被知道……(中略)意被滅與法想被滅之處。那些處應該被知道。」誰對被世尊以簡要誦說的這個總說、未詳細地解析的義理,會詳細地解析義理呢?』

  大德!我們想這個:『這位尊者阿難正是大師稱讚者,以及有智的同梵行者們的尊重者,且尊者阿難能夠對被世尊以簡要誦說的這個總說、未被詳細地解析的義理,詳細地解析義理,讓我們去見尊者阿難。抵達後,詢問尊者阿難這個義理。』

  大德!那時,我們去見尊者阿難。抵達後,詢問尊者阿難這個義理。大德!以這些方式(行相)、以這些語詞、以這些文句被尊者阿難為那些我們解析義理。」

  「比丘們!阿難是賢智者;比丘們!阿難是大慧者。比丘們!如果你們也詢問我這個義理,我也正是這樣解說它,如它被阿難解說。這就是這個的義理,你們應該這樣憶持它。」



\sutta{118}{118}{帝釋之問經}{https://agama.buddhason.org/SN/sn.php?keyword=35.118}
  \twnr{有一次}{2.0},\twnr{世尊}{12.0}住在王舍城\twnr{耆闍崛山}{258.0}中。

  那時,\twnr{天帝釋}{263.0}去見世尊。抵達後,向世尊\twnr{問訊}{46.0}後,在一旁站立。在一旁站立的天帝釋對世尊說這個:

  「\twnr{大德}{45.0}!什麼因、什麼\twnr{緣}{180.0},以那個,這裡一些眾生\twnr{當生}{42.0}不\twnr{證涅槃}{71.0}呢?大德!而什麼因、什麼緣,以那個,這裡一些眾生當生證涅槃呢?」

  「天帝!有能被眼識知的、想要的、所愛的、合意的、可愛形色的、伴隨欲的、誘人的諸色,如果\twnr{比丘}{31.0}歡喜、歡迎、持續固持那個,對那位歡喜、歡迎、持續固持那個者來說識是依止那個的、執取那個的,天帝!有執取的比丘不證涅槃。……(中略)

  天帝!有能被舌識知……的諸味道……(中略)天帝!有能被意識知的、想要的、所愛的、合意的、可愛形色的、伴隨欲的、誘人的諸法,如果比丘歡喜、歡迎、持續固持那個,對那位歡喜、歡迎、持續固持那個者來說識是依止那個的、執取那個的,天帝!有執取的比丘不證涅槃。天帝!這是因、這是緣,這裡一些眾生當生不證涅槃。

  天帝!有能被眼識知的、想要的、所愛的、合意的、可愛形色的、伴隨欲的、誘人的諸色,如果比丘不歡喜、不歡迎、不持續固持那個,對那位不歡喜、不歡迎、不固持那個者來說識是不依止那個的、不執取那個的,天帝!無執取的比丘證涅槃。……(中略)

  天帝!有能被舌識知……的諸味道……(中略)天帝!有能被意識知的、想要的、所愛的、合意的、可愛形色的、伴隨欲的、誘人的諸法,如果比丘不歡喜、不歡迎、不持續固持那個,對那位不歡喜、不歡迎、不固持那個者來說識是不依止那個的、不執取那個的,天帝!無執取的比丘證涅槃。天帝!這是因、這是緣,這裡一些眾生當生證涅槃。」



\sutta{119}{119}{五髻經}{https://agama.buddhason.org/SN/sn.php?keyword=35.119}
  \twnr{有一次}{2.0},\twnr{世尊}{12.0}住在王舍城\twnr{耆闍崛山}{258.0}中。

  那時,乾達婆天神之子五髻去見世尊。抵達後,向世尊\twnr{問訊}{46.0}後,在一旁站立。在一旁站立的乾達婆天神之子五髻對世尊說這個:

  「\twnr{大德}{45.0}!什麼因、什麼\twnr{緣}{180.0},以那個,這裡一些眾生\twnr{當生}{42.0}不\twnr{證涅槃}{71.0}呢?大德!而什麼因、什麼緣,以那個,這裡一些眾生當生證涅槃呢?」

  「五髻!有能被眼識知的……諸色……(中略)五髻!有能被意識知的、想要的、所愛的、合意的、可愛形色的、伴隨欲的、誘人的諸法,如果\twnr{比丘}{31.0}歡喜、歡迎、持續固持那個,對那位歡喜、歡迎、持續固持那個者來說識是依止那個的、執取那個的,五髻!有執取的比丘不證涅槃。五髻!這是因、這是緣,這裡一些眾生當生不證涅槃。

  五髻!有能被眼識知的、想要的、所愛的、合意的……諸色……(中略)五髻!有能被意識知的、想要的、所愛的、合意的、可愛形色的、伴隨欲的、誘人的諸法,如果比丘不歡喜、不歡迎、不持續固持那個,對那位不歡喜、不歡迎、不固持那個者來說識是不依止那個的、不執取那個的,五髻!無執取的比丘證涅槃。五髻!這是因、這是緣,這裡一些眾生當生證涅槃。」



\sutta{120}{120}{舍利弗-共住者經}{https://agama.buddhason.org/SN/sn.php?keyword=35.120}
  \twnr{有一次}{2.0},\twnr{尊者}{200.0}舍利弗住在舍衛城祇樹林給孤獨園。

  那時,\twnr{某位比丘}{39.0}去見尊者舍利弗。抵達後,與尊者舍利弗一起互相問候。交換應該被互相問候的友好交談後,在一旁坐下。在一旁坐下的那位比丘對尊者舍利弗說這個:

  「舍利弗\twnr{學友}{201.0}!一位共住比丘放棄學後還俗了。」

  「這是這樣,學友!對不\twnr{在諸根上守護門者}{468.0}、不在飲食上知適量者、不專修清醒者來說。『學友!那位不在諸根上守護門、不在飲食上知適量、不專修清醒的比丘確實終生將維持圓滿、遍純淨梵行。』\twnr{這不存在可能性}{650.0}。『學友!那位在諸根上守護門、在飲食上知適量者、專修清醒的比丘確實終生將維持圓滿、遍純淨梵行。』這存在可能性。

  學友!而怎樣是在諸根上守護門者?學友!這裡,比丘以眼見色後,不成為相的執取者、\twnr{細相}{197.0}的執取者,因那個理由,\twnr{貪婪}{435.0}、憂諸惡不善法會\twnr{流入}{224.0}那位住於眼根不自制者。他走向為了那個的\twnr{自制}{217.0},守護眼根,在眼根上來到自制;以耳聽聲音後……以鼻聞氣味後……以舌嚐味道後……以身觸\twnr{所觸}{220.2}後……以意識知法後,不成為相的執取者、細相的執取者,因那個理由,貪婪、憂諸惡不善法會流入那位住於意根不防護者,他走上為了那個自制之路,他守護意根,他在意根上來到自制,學友!這樣是在諸根上守護門者。

  學友!而怎樣是在飲食上知適量者?學友!這裡,比丘如理省察後吃食物:『既不為了娛樂,也不為了自豪,也不為了裝飾,也\twnr{不為了莊嚴}{520.0},最多為了這個身體的存續、生存,為了止息傷害,為了資助梵行。像這樣,我將擊退\twnr{之前的感受}{536.0},與不使新的感受生起,將有我的生存,與無過失狀態,以及\twnr{安樂住}{156.0}。』學友!這樣是在飲食上知適量者。

  學友!怎樣是專修清醒者呢?學友!這裡,比丘白天以經行、安坐,使心\twnr{從障礙法}{990.1}淨化。在\twnr{初夜}{214.0},以經行、安坐,使心從障礙法淨化。在中夜,[左]腳放在[右]腳上、\twnr{作意起來想後}{502.0},具念正知地\twnr{以右脅作獅子臥}{367.0}。在後夜,起來後以經行、安坐,使心從障礙法淨化,學友!這樣是專修清醒者。

  學友!因此,在這裡,應該被這麼學:『我們要成為在諸根上守護門者、在飲食上知適量者、專修清醒者。』學友!應該被你們這麼學。」



\sutta{121}{121}{教誡羅侯羅經}{https://agama.buddhason.org/SN/sn.php?keyword=35.121}
  \twnr{有一次}{2.0},\twnr{世尊}{12.0}住在舍衛城祇樹林給孤獨園。

  那時,當世尊獨處、\twnr{獨坐}{92.0}時,這樣心的深思生起:

  「羅侯羅的\twnr{解脫圓熟法}{x502}已遍熟,讓我更進一步在諸\twnr{漏}{188.0}的滅盡上教導(調伏)羅侯羅。」

  那時,世尊午前時穿衣、拿起衣鉢後,在舍衛城\twnr{為了托鉢}{87.0}行走後,\twnr{餐後已從施食返回}{512.0},召喚\twnr{尊者}{200.0}羅侯羅:

  「羅侯羅!取坐墊布,\twnr{為了白天的住處}{128.0}我們將去\twnr{盲者的樹林}{88.0}。」

  「是的,\twnr{大德}{45.0}!」尊者羅侯羅回答世尊後,取坐墊布,在世尊後面緊跟隨。

  當時,有好幾千位跟隨世尊的天眾們:「今天,世尊將更進一步在諸漏的滅盡上教導羅侯羅。」

  那時,世尊進入盲者的樹林後,坐在某棵樹下設置的座位。

  尊者羅侯羅向世尊\twnr{問訊}{46.0}後,也在一旁坐下。世尊對在一旁坐下的尊者羅侯羅說這個:

  「羅侯羅!你怎麼想它:眼是常的,或是無常的?」

  「無常的,大德!」

  「那麼,凡為無常的,那是苦的或樂的?」

  「苦的,大德!」

  「那麼,凡為無常的、苦的、\twnr{變易法}{70.0},適合認為它:『\twnr{這是我的}{32.0},\twnr{我是這個}{33.0},\twnr{這是我的真我}{34.1}。』嗎?」

  「大德!這確實不是。」

  「諸色是常的,或是無常的?」

  「無常的,大德!」……(中略)。

  「眼識是常的,或是無常的?」

  「無常的,大德!」……(中略)。

  「眼觸是常的,或是無常的?」

  「無常的,大德!」……(中略)。

  「又凡以這眼觸\twnr{為緣}{180.0}生起受之類的、想之類的、行之類的、\twnr{識之類的}{923.0},是常的,或是無常的?」

  「無常的,大德!」

  「那麼,凡為無常的,那是苦的或樂的?」

  「苦的,大德!」

  「那麼,凡為無常的、苦的、變易法,適合認為它:『\twnr{這是我的}{32.0},\twnr{我是這個}{33.0},這是\twnr{我的真我}{34.0}。』嗎?」

  「大德!這確實不是。」……(中略)。

  「舌是常的,或是無常的?」

  「無常的,大德!」……(中略)。

  「舌識是常的,或是無常的?」

  「無常的,大德!」……(中略)。

  「舌觸是常的,或是無常的?」

  「無常的,大德!」……(中略)。

  「又凡以這舌觸為緣生起受之類的、想之類的、行之類的、識之類的,是常的,或是無常的?」

  「無常的,大德!」

  「那麼,凡為無常的,那是苦的或樂的?」

  「苦的,大德!」

  「那麼,凡為無常的、苦的、變易法,適合認為它:『這是我的,我是這個,這是我的真我。』嗎?」

  「大德!這確實不是。」……(中略)。

  「意是常的,或是無常的?」

  「無常的,大德!」

  「那麼,凡為無常的,那是苦的或樂的?」

  「苦的,大德!」

  「那麼,凡為無常的、苦的、變易法,適合認為它:『這是我的,我是這個,這是我的真我。』嗎?」

  「大德!這確實不是。」

  「諸法是常的,或是無常的?」

  「無常的,大德!」……(中略)。

  「意識是常的,或是無常的?」

  「無常的,大德!」……(中略)。

  「意觸是常的,或是無常的?」

  「無常的,大德!」……(中略)。

  「又凡以這意觸為緣生起受之類的、想之類的、行之類的、識之類的,是常的,或是無常的?」

  「無常的,大德!」

  「那麼,凡為無常的,那是苦的或樂的?」

  「苦的,大德!」

  「那麼,凡為無常的、苦的、變易法,適合認為它:『這是我的,我是這個,這是我的真我。』嗎?」

  「大德!這確實不是。」

  「羅侯羅!這麼看的\twnr{有聽聞的聖弟子}{24.0}在眼上\twnr{厭}{15.0},也在諸色上厭,也在眼識上厭,也在眼觸上厭,又凡以這眼觸為緣生起受之類的、想之類的、行之類的、識之類的,也在那上面厭。……(中略)也在舌上厭,也在諸味道上厭,在舌識上厭,在舌觸上厭,又凡以這舌觸為緣生起受之類的、想之類的、行之類的、識之類的,也在那上面厭。……(中略)也在意上厭,也在諸法上厭,也在意識上厭,也在意觸上厭,凡以這意觸為緣生起受之類的、想之類的、行之類的、識之類的,也在那上面厭。厭者\twnr{離染}{558.0},從\twnr{離貪}{77.0}被解脫,在已解脫時,\twnr{有『[這是]解脫』之智}{27.0},他知道:『\twnr{出生已盡}{18.0},\twnr{梵行已完成}{19.0},\twnr{應該被作的已作}{20.0},\twnr{不再有此處[輪迴]的狀態}{21.1}。』」

  世尊說這個,悅意的尊者羅侯羅歡喜世尊的所說。

  還有,\twnr{在當這個解說被說時}{136.0},不執取後尊者羅侯羅的心從諸\twnr{漏}{188.0}被解脫。

  且好幾千位天神的遠塵、\twnr{離垢之法眼}{62.0}生起:

  「凡任何\twnr{集法}{67.1}那個全部是\twnr{滅法}{68.1}。」[\ccchref{MN.147}{https://agama.buddhason.org/MN/dm.php?keyword=147}]



\sutta{122}{122}{會被結縛的法經}{https://agama.buddhason.org/SN/sn.php?keyword=35.122}
  「\twnr{比丘}{31.0}們!我將教導\twnr{會被結縛的諸法}{666.0}與結,\twnr{你們要聽}{43.0}它!

  比丘們!而什麼是會被結縛的諸法?什麼是結呢?

  比丘們!有能被眼識知的、想要的、所愛的、合意的、可愛形色的、伴隨欲的、誘人的諸色,比丘們!這些被稱為會被結縛的法,凡在那裡有意欲貪者,在那裡有結……(中略)比丘們!有能被舌識知……的諸味道……(中略)比丘們!有能被意識知的、想要的、所愛的、合意的、可愛形色的、伴隨欲的、誘人的諸法,比丘們!這些被稱為會被結縛的法,凡在那裡有意欲貪者,在那裡有結。」[≃\suttaref{SN.22.120}, \suttaref{SN.35.109}]



\sutta{123}{123}{與執取有關的法經}{https://agama.buddhason.org/SN/sn.php?keyword=35.123}
  「\twnr{比丘}{31.0}們!我將教導\twnr{與執取有關的}{551.0}諸法與執取,\twnr{你們要聽}{43.0}它!

  比丘們!什麼是與執取有關的諸法?以及什麼是執取?比丘們!有能被眼識知的、想要的、所愛的、合意的、可愛形色的、伴隨欲的、誘人的諸色,比丘們!這些被稱為與執取有關的法,凡在那裡有意欲貪者,在那裡有執取……(中略)比丘們!有能被舌識知……的諸味道……(中略)比丘們!有能被意識知的、想要的、所愛的、合意的、可愛形色的、伴隨欲的、誘人的諸法,比丘們!這些被稱為與執取有關的法,凡在那裡有意欲貪者,在那裡有執取。」[≃\suttaref{SN.22.121}, \suttaref{SN.35.110}]

  世間欲種類品第十二,其\twnr{攝頌}{35.0}:

  「以魔網二說,世間與欲種類,

   帝釋連同五髻,舍利弗與羅侯羅,

   結縛、執取,以那個被稱為品。」





\pin{屋主品}{124}{133}
\sutta{124}{124}{毘舍離經}{https://agama.buddhason.org/SN/sn.php?keyword=35.124}
  \twnr{有一次}{2.0},\twnr{世尊}{12.0}住在毘舍離大林重閣講堂。

  那時,毘舍離人\twnr{屋主}{103.0}郁伽去見世尊。抵達後,在一旁坐下。在一旁坐下的毘舍離人屋主郁伽對世尊說這個:

  「\twnr{大德}{45.0}!什麼因、什麼\twnr{緣}{180.0},以那個,這裡一些眾生\twnr{當生}{42.0}不\twnr{證涅槃}{71.0}呢?大德!而什麼因、什麼緣,以那個,這裡一些眾生當生證涅槃呢?」

  「屋主!有能被眼識知的、想要的、所愛的、合意的、可愛形色的、伴隨欲的、誘人的諸色,如果\twnr{比丘}{31.0}歡喜、歡迎、持續固持那個,對那位歡喜、歡迎、持續固持那個者來說識是依止那個的、執取那個的。屋主!有執取的比丘不證涅槃。

  ……(中略)屋主!有能被舌識知……的諸味道……(中略)屋主!有能被意識知的、想要的、所愛的、合意的、可愛形色的、伴隨欲的、誘人的諸法,如果比丘歡喜、歡迎、持續固持那個,對那位歡喜、歡迎、持續固持那個者來說識是依止那個的、執取那個的。屋主!有執取的比丘不證涅槃。屋主!這是因、這是緣,這裡一些眾生當生不證涅槃。

  屋主!有能被眼識知的、想要的、所愛的、合意的、可愛形色的、伴隨欲的、誘人的諸色,如果比丘不歡喜、不歡迎、不持續固持那個,對那位不歡喜、不歡迎、不固持那個者來說識是不依止那個的、不執取那個的。屋主!無執取的比丘證涅槃。

  ……(中略)屋主!有能被舌識知……的諸味道……(中略)屋主!有能被意識知的、想要的、所愛的、合意的、可愛形色的、伴隨欲的、誘人的諸法,如果比丘不歡喜、不歡迎、不持續固持那個,對那位不歡喜、不歡迎、不固持那個者來說識是不依止那個的、不執取那個的。屋主!無執取的比丘證涅槃。屋主!這是因、這是緣,這裡一些眾生當生證涅槃。」[≃\suttaref{SN.35.118}, 119, \suttaref{SN.35.125}-126, 128, 131]



\sutta{125}{125}{跋耆經}{https://agama.buddhason.org/SN/sn.php?keyword=35.125}
  \twnr{有一次}{2.0},\twnr{世尊}{12.0}住在跋耆的象村。

  那時,象村人\twnr{屋主}{103.0}郁伽去見世尊。抵達後,在一旁坐下。在一旁坐下的象村人屋主郁伽對世尊說這個:

  「\twnr{大德}{45.0}!什麼因、什麼\twnr{緣}{180.0},以那個,這裡一些眾生\twnr{當生}{42.0}不\twnr{證涅槃}{71.0}呢?大德!而什麼因、什麼緣,以那個,這裡一些眾生當生證涅槃呢?」

  (應該依前經那樣使之被細說)

  屋主!這是因、這是緣,這裡一些眾生當生證涅槃。」



\sutta{126}{126}{那難陀經}{https://agama.buddhason.org/SN/sn.php?keyword=35.126}
  \twnr{有一次}{2.0},\twnr{世尊}{12.0}住在那難陀賣衣者的芒果園中。

  那時,\twnr{屋主}{103.0}優波離去見世尊。……(中略)在一旁坐下的屋主優波離對世尊說這個:

  「\twnr{大德}{45.0}!什麼因、什麼\twnr{緣}{180.0},以那個,這裡一些眾生\twnr{當生}{42.0}不\twnr{證涅槃}{71.0}呢?大德!而什麼因、什麼緣,以那個,這裡一些眾生當生證涅槃呢?」

  (應該依前經那樣使之被細說)

  屋主!這是因、這是緣,這裡一些眾生當生證涅槃。」



\sutta{127}{127}{婆羅墮若經}{https://agama.buddhason.org/SN/sn.php?keyword=35.127}
  \twnr{有一次}{2.0},\twnr{尊者}{200.0}賓頭盧婆羅墮若住在\twnr{憍賞彌}{140.0}瞿師羅園。

  那時,\twnr{優填那王}{x503}去見尊者賓頭盧婆羅墮若。抵達後,與尊者賓頭盧婆羅墮若一起互相問候。交換應該被互相問候的友好交談後,就在一旁坐下。在一旁坐下的優填那王對尊者賓頭盧婆羅墮若說這個:

  「婆羅墮若尊師!什麼因、什麼\twnr{緣}{180.0},以那個這些年輕的\twnr{比丘}{31.0}:黑髮青年、具備青春的幸福者,在人生初期不在諸欲中玩樂,終生行圓滿、遍純淨梵行而度過時間?」

  「大王!這被那位\twnr{有知、有見}{383.0}的世尊、阿羅漢、遍正覺者說:『來!比丘們!在母親程度者(母親年紀的女人)上你們要使[她是]母親的心現起;在姊妹程度者上你們要使姊妹的心現起;在女兒程度者上你們要使女兒的心現起。』大王!這是因、這是緣,以那個這些年輕的比丘:黑髮青年、具備青春的幸福者,在人生初期不在諸欲中玩樂,終生行圓滿、遍純淨梵行而度過時間。」

  「婆羅墮若尊師!心是搖動的(動貪的),有時,在母親程度者上諸貪法也生起;在姊妹程度者上諸貪法也生起;在女兒程度者上諸貪法也生起,婆羅墮若尊師!有其他因、其他緣,以那個這些年輕的比丘:黑髮青年……(中略)而度過時間?」

  「大王!這被那位有知、有見的世尊、阿羅漢、遍正覺者說:『來!比丘們!你們要就這個身體從腳掌之上,從髮梢之下,皮膚為邊界,有種種種類不淨充滿的,省察:「在這個身體中有頭髮、體毛、指甲、牙齒、皮膚、肌肉、筋腱、骨骼、骨髓、腎臟、心臟、肝臟、肋膜、脾臟、肺臟、腸子、腸間膜、胃、糞便、膽汁、痰、膿、血、汗、脂肪、眼淚、油脂、唾液、鼻涕、關節液、尿。」』大王!這也是因、這是緣,以那個這些年輕的的比丘:黑髮青年……(中略)而度過時間。」

  「婆羅墮若尊師!凡那些身已\twnr{修習}{94.0}、戒已修習、心已修習、慧已修習的比丘們,對他們來說那是容易的,婆羅墮若尊師!但凡那些\twnr{身未修習}{x504}、戒未修習、心未修習、慧未修習的比丘們,對他們來說,那是困難的,婆羅墮若尊師!有時:『我要作意為不淨的。』仍來到為淨的。婆羅墮若尊師!有其他因、其他緣,以那個這些年輕的比丘:黑髮青年……(中略)終生行圓滿、遍純淨梵行而度過時間?」

  「大王!這被那位有知、有見的世尊、阿羅漢、遍正覺者說:『來!比丘們!你們要住於\twnr{在諸根上守護門}{468.0}:以眼見色後,你們不要成為相的執取者、\twnr{細相}{197.0}的執取者,因那個理由,\twnr{貪婪}{435.0}、憂諸惡不善法會\twnr{流入}{224.0}那位住於眼根不防護者,你們要走上為了那個的\twnr{自制}{217.0}之路,你們要守護眼根,你們要在眼根上來到自制;以耳聽聲音後……(中略)以鼻嗅氣味後……以舌嚐味道後……以身觸\twnr{所觸}{220.2}後……以意識知法後,你們不要成為相的執取者、細相的執取者,因那個理由,貪婪、憂諸惡不善法會流入那位住於意根不防護者,你們要走上為了那個的自制之路,你們要守護意根,你們要在意根上來到自制。』大王!這也是因、這是緣,以那個這些年輕的比丘:黑髮青年、具備青春的幸福者,在人生初期不在諸欲中玩樂,終生行圓滿、遍純淨梵行而度過時間。」

  「\twnr{不可思議}{206.0}啊,婆羅墮若尊師!\twnr{未曾有}{206.0}啊,婆羅墮若尊師!

  婆羅墮若尊師!而這被那位有知、有見的世尊、阿羅漢、遍正覺者多麼善說:『婆羅墮若尊師!這就是因、這就是緣,以那個這些年輕的比丘:黑髮青年、具備青春的幸福者,在人生初期不在諸欲中玩樂,終生行圓滿、遍純淨梵行而度過時間。』

  婆羅墮若尊師!我也凡在就以身未守護、以語未守護、以心未守護,以念未現起,以諸根未自制時進入內宮,那時,諸貪法極度地遍征服我,婆羅墮若尊師!但凡在我以身已守護、以語已守護、以心已守護,以念已現起,以諸根已自制時進入內宮,那時,諸貪法不像這樣遍征服我。

  太偉大了,婆羅墮若尊師!太偉大了,婆羅墮若尊師!婆羅墮若尊師!猶如扶正顛倒的,或揭開隱藏的,或告知迷路者的道路,或在黑暗中持燈火:『有眼者們看見諸色。』同樣的,法被婆羅墮若\twnr{尊師}{203.0}以種種\twnr{法門}{562.0}說明。婆羅墮若尊師!這個我\twnr{歸依}{284.0}世尊、法、\twnr{比丘僧團}{65.0},請婆羅墮若尊師記得我為\twnr{優婆塞}{98.0},從今天起\twnr{已終生歸依}{64.0}。」



\sutta{128}{128}{輸屢那經}{https://agama.buddhason.org/SN/sn.php?keyword=35.128}
  \twnr{有一次}{2.0},\twnr{世尊}{12.0}住在王舍城栗鼠飼養處的竹林中。

  那時,\twnr{屋主}{103.0}之子輸屢那去見世尊。抵達後,向世尊\twnr{問訊}{46.0}後,在一旁坐下。在一旁坐下的屋主之子輸屢那對世尊說這個:

  「\twnr{大德}{45.0}!什麼因、什麼\twnr{緣}{180.0},以那個,這裡一些眾生\twnr{當生}{42.0}不\twnr{證涅槃}{71.0}呢?大德!而什麼因、什麼緣,以那個,這裡一些眾生當生證涅槃呢?」

  (應該依前經那樣使之被細說)

  輸屢那!這是因、這是緣,這裡一些眾生當生證涅槃。」



\sutta{129}{129}{瞿師羅經}{https://agama.buddhason.org/SN/sn.php?keyword=35.129}
  \twnr{有一次}{2.0},\twnr{尊者}{200.0}阿難住在\twnr{憍賞彌}{140.0}瞿師羅園。

  那時,瞿師羅\twnr{屋主}{103.0}去見尊者阿難……(中略)在一旁坐下的瞿師羅屋主對尊者阿難說這個:

  「\twnr{大德}{45.0}阿難!被稱為『種種界(界種種性),種種界』,大德!什麼情形被\twnr{世尊}{12.0}稱為種種界?」

  「屋主!存在眼界、合意的諸色、眼識、\twnr{緣於}{252.0}樂能被感受的觸,樂受生起,屋主!存在眼界、不合意的諸色、眼識、緣於苦能被感受的觸,苦受生起,屋主!存在眼界、\twnr{平靜}{228.0}能被感受的{合意}諸色、眼識、緣於不苦不樂能被感受的觸,不苦不樂受生起。

  ……(中略)屋主!存在舌界、合意的諸味道、舌識、緣於樂能被感受的觸,樂受生起,屋主!存在舌界、不合意的諸味道、舌識、緣於苦能被感受的觸,苦受生起,屋主!存在舌界、平靜能被感受的諸味道、舌識、緣於不苦不樂能被感受的觸,不苦不樂受生起。……(中略)屋主!存在\twnr{意界}{337.0}、合意的諸法、意識、緣於樂能被感受的觸,樂受生起,屋主!存在意界、不合意的諸法、意識、緣於苦能被感受的觸,苦受生起,屋主!存在有意界、平靜被能感受的諸法、意識、緣於不苦不樂能被感受的觸,不苦不樂受生起。

  屋主!這個情形被世尊稱為種種界。」



\sutta{130}{130}{訶梨迪迦尼經}{https://agama.buddhason.org/SN/sn.php?keyword=35.130}
  \twnr{有一次}{2.0},\twnr{尊者}{200.0}大迦旃延住在拘拉拉迦拉之斷崖山,阿槃提中。

  那時,\twnr{屋主}{103.0}訶梨迪迦尼去見尊者大迦旃延。……(中略)在一旁坐下的屋主訶梨迪迦尼對尊者大迦旃延說這個:

  「\twnr{大德}{45.0}!這被\twnr{世尊}{12.0}說:『\twnr{緣於}{252.0}種種界(界種種性)種種觸生起;緣於種種觸種種受生起。』大德!怎樣緣於種種界種種觸生起;緣於種種觸種種受生起?」

  「屋主!這裡,\twnr{比丘}{31.0}以眼見色後知道:『在這裡,這是合意的。』以及,有眼識、緣於樂能被感受的觸,樂受生起。還有,比丘以眼見色後知道:『在這裡,這是不合意的。』以及,有眼識、緣於苦能被感受的觸,苦受生起。還有,比丘以眼見色後知道:『在這裡,這是平靜情況(處)。』以及,有眼識,緣於不苦不樂能被感受的觸,不苦不樂受生起。

  再者,屋主!這裡,比丘以耳聽聲音後……(中略)以鼻聞氣味後……(中略)以舌嚐味道後……(中略)以身觸\twnr{所觸}{220.2}後……(中略)以意識知法後知道:『在這裡,這是合意的。』以及,有意識,緣於樂能被感受的觸,樂受生起。還有,比丘以意識知法後知道:『在這裡,這是不合意的。』以及,有意識,緣於苦能被感受的觸,苦受生起。還有,比丘以意識知法後知道:『在這裡,這是平靜情況。』以及,有意識,緣於不苦不樂能被感受的觸,不苦不樂受生起。屋主!這樣,緣於種種界種種觸生起;緣於種種觸種種受生起。」



\sutta{131}{131}{那拘羅的父親經}{https://agama.buddhason.org/SN/sn.php?keyword=35.131}
  \twnr{有一次}{2.0},\twnr{世尊}{12.0}住在婆祇國蘇蘇馬拉山之配沙卡拉林的鹿林。

  那時,\twnr{屋主}{103.0}那拘羅的父親去見世尊。……(中略)在一旁坐下的屋主那拘羅的父親對世尊說這個:

  「\twnr{大德}{45.0}!什麼因、什麼\twnr{緣}{180.0},以那個,這裡一些眾生\twnr{當生}{42.0}不\twnr{證涅槃}{71.0}呢?大德!而什麼因、什麼緣,以那個,這裡一些眾生當生證涅槃呢?」

  「屋主!有能被眼識知的、想要的、所愛的、合意的、可愛形色的、伴隨欲的、誘人的諸色,如果\twnr{比丘}{31.0}歡喜、歡迎、持續固持那個,對那位歡喜、歡迎、持續固持那個者來說識是依止那個的、執取那個的,屋主!有執取的比丘不證涅槃。……(中略)

  屋主!有能被舌識知……的諸味道……(中略)屋主!有能被意識知的、想要的、所愛的、合意的、可愛形色的、伴隨欲的、誘人的諸法,如果比丘歡喜、歡迎、持續固持那個,對那位歡喜、歡迎、持續固持那個者來說識是依止那個的、執取那個的,屋主!有執取的比丘不證涅槃。屋主!這是因、這是緣,這裡一些眾生當生不證涅槃。

  屋主!有能被眼識知的、想要的、所愛的、合意的、可愛形色的、伴隨欲的、誘人的諸色,如果比丘不歡喜、不歡迎、不持續固持那個,對那位不歡喜、不歡迎、不固持那個者來說識是不依止那個的、不執取那個的,屋主!無執取的比丘證涅槃。……(中略)

  屋主!有能被舌識知……的諸味道……(中略)屋主!有能被意識知的、想要的、所愛的、合意的、可愛形色的、伴隨欲的、誘人的諸法,如果比丘不歡喜、不歡迎、不持續固持那個,對那位不歡喜、不歡迎、不固持那個者來說識是不依止那個的、不執取那個的,屋主!無執取的比丘證涅槃。屋主!這是因、這是緣,這裡一些眾生當生證涅槃。」



\sutta{132}{132}{魯西遮經}{https://agama.buddhason.org/SN/sn.php?keyword=35.132}
  \twnr{有一次}{2.0},\twnr{尊者}{200.0}大迦旃延住在阿槃提國馬迦拉迦的\twnr{林野}{142.0}小屋。

  那時,眾多魯西遮婆羅門的打柴\twnr{學生婆羅門}{102.0}徒弟,去尊者迦旃延的林野小屋處。抵達後,在小屋的四周圍散步、漫步,高聲、大聲地\twnr{玩(作)種種跳戲}{x505}:「然而這些卑俗的、黑的、\twnr{親族腳子孫}{613.0}的\twnr{禿頭假沙門}{439.0},被他們的維持者恭敬、尊重、尊敬、禮拜、崇拜。」

  那時,尊者大迦旃延從住處出來後,對那些學生婆羅門們說這個:

  「學生婆羅門們!請你們不要吵(作聲音),我將為你們說法。」

  在這麼說時,那些學生婆羅門保持沈默。

  那時,尊者大迦旃延以\twnr{偈頌}{281.0}對那些學生婆羅門說:

  「在更早以前有最上戒行者:那些婆羅門凡他們記得古傳者,

   他們是守護諸[根]門者、善守護者:克服他們的憤怒後,

   他們是在法與禪上的愛好者:那些婆羅門凡他們記得古傳者。

   但這些墮落後[說]『我們誦讀』,被種姓陶醉(驕慢)者們行不正,

   被憤怒征服者們取各種棍棒,{在有渴愛無渴愛上離被染者}[\twnr{在懦弱者與堅強者上違犯者}{x506}],

   對不守護門者們來說是無用的,像在夢中所得到的人的財產。

   斷食者們與露地臥者們,\twnr{清早沐浴與三吠陀}{x507},

   粗的獸皮衣、結髮、污泥,咒語、戒與禁制、苦行,

   詭詐的彎曲拐杖,以及清洗水,

   這些婆羅門的容色(象徵),是\twnr{些細的修習所做}{x508}。

   而心善得定的,明淨無混濁的,

   在一切生物類上不頑固的,那是為了到達梵天之道。」

  那時,那些憤怒、不悅的學生婆羅門去見魯西遮婆羅門。抵達後,對魯西遮婆羅門說這個:

  「真的,\twnr{尊師}{203.0}應該知道,沙門大迦旃延\twnr{一向}{168.0}地斥責、呵責婆羅門的經典。」

  在這麼說時,魯西遮婆羅門變得憤怒、不悅。

  那時,魯西遮婆羅門想這個:

  「然而,這對我是不適當的:凡我全然只聽學生婆羅門的後,應該辱罵、應該誹謗沙門大迦旃延。讓我抵達後詢問。」

  那時,魯西遮婆羅門與那些學生婆羅門一同去見尊者大迦旃延。抵達後,與尊者大迦旃延一起互相問候。交換應該被互相問候的友好交談後,在一旁坐下。在一旁坐下的魯西遮婆羅門對尊者大迦旃延說這個:

  「迦旃延尊師!這裡,我的眾多打柴學生婆羅門徒弟曾來到嗎?」

  「婆羅門!這裡,你的眾多打柴學生婆羅門徒弟曾來到。」

  「那麼,迦旃延先生與那些學生婆羅門一起有就某種交談?」

  「婆羅門!我與那些學生婆羅門一起有就某種交談。」

  「那麼,如怎樣迦旃延先生與那些學生婆羅門一起有就某種交談?」

  「婆羅門!我與那些學生婆羅門一起有這樣的交談:

  『在更早以前有最上戒行者:那些婆羅門凡他們記得古傳者……(中略)

   在一切生物類上不頑固的,那是為了到達梵天之道。』

  婆羅門!我與那些學生婆羅門一起有這樣的交談。」

  「迦旃延尊師說『不守護門』,迦旃延尊師!什麼情形是不守護門呢?」

  「婆羅門!這裡,某一類人以眼見色後,\twnr{志向}{257.0}可愛形色的諸色,排拒不可愛形色的諸色,住於身念未現起的、\twnr{少心的}{x509},以及不如實知道那個\twnr{心解脫}{16.0}、\twnr{慧解脫}{539.0},於該處他的那些生起的惡不善法無殘餘地被滅;以耳聽聲音後……以鼻聞氣味後……以舌嚐味道後……以身觸\twnr{所觸}{220.2}後……以意識知法後,志向可愛形色的諸法,排拒不可愛形色的諸法,住於身念未現起的、少心的,以及不如實知道那個心解脫、慧解脫,於該處他的那些生起的惡不善法無殘餘地被滅,婆羅門!這樣是不守護門。」

  「\twnr{不可思議}{206.0}啊,迦旃延尊師!未曾有啊,迦旃延尊師!這位就不守護門者多麼被迦旃延尊師告知為『不守護門者』。

  迦旃延尊師說『守護門』,迦旃延尊師!什麼情形是守護門?」

  「婆羅門!這裡,\twnr{比丘}{31.0}以眼見色後,不志向可愛形色的諸色,不排拒不可愛形色的諸色,住於身念已現起的、無量心的,以及如實知道那個心解脫、慧解脫,於該處他的那些生起的惡不善法無殘餘地被滅;以耳聽聲音後……以鼻聞氣味後……以舌嚐味道後……以身觸所觸後……以意識知法後,不志向諸可愛法,不排拒不可愛形色的諸法,住於身念已現起的、無量心的,以及如實知道那個心解脫、慧解脫,於該處他的那些生起的惡不善法無殘餘地被滅,婆羅門先生!這樣是守護門。」

  「不可思議啊,迦旃延尊師!未曾有啊,迦旃延尊師!這位就守護門者多麼被迦旃延尊師告知為『守護門者』。

  太偉大了,迦旃延尊師!太偉大了,迦旃延尊師!迦旃延尊師!猶如扶正顛倒的,或揭開隱藏的,或告知迷路者的道路,或在黑暗中持燈火:『有眼者們看見諸色。』同樣的,法被迦旃延尊師以種種\twnr{法門}{562.0}說明。迦旃延尊師!這個我\twnr{歸依}{284.0}\twnr{世尊}{12.0}、法、\twnr{比丘僧團}{65.0},請迦旃延尊師記得我為\twnr{優婆塞}{98.0},從今天起\twnr{已終生歸依}{64.0}。請迦旃延尊師去魯西遮家,就像迦旃延尊師去在馬迦拉迦的諸優婆塞家一樣,在那裡,凡學生婆羅門,或女學生婆羅門們將會\twnr{問訊}{46.0}、起立迎接,將會給與座位,或水,那對他們將有長久的利益、安樂。」



\sutta{133}{133}{韋拉哈迦尼經}{https://agama.buddhason.org/SN/sn.php?keyword=35.133}
  \twnr{有一次}{2.0},\twnr{尊者}{200.0}優陀夷住在迦嗎達的兜泥訝婆羅門芒果林中。

  那時,韋拉哈迦尼氏族的婆羅門尼學生婆羅門徒弟去見尊者優陀夷。抵達後,與尊者優陀夷一起互相問候。交換應該被互相問候的友好交談後,在一旁坐下。尊者優陀夷對在一旁坐下的那位學生婆羅門以法說開示、勸導、鼓勵、\twnr{使歡喜}{86.0}。

  那時,被尊者優陀夷以法說開示、勸導、鼓勵、使歡喜的那位學生婆羅門從座位起來後,去見韋拉哈迦尼氏族的婆羅門尼。抵達後,對韋拉哈迦尼氏族的婆羅門尼說這個:

  「真的,\twnr{尊師}{203.0}!你應該知道,\twnr{沙門}{29.0}優陀夷教導開頭是善的、中間是善的、結尾是善的;\twnr{有意義的}{81.0}、\twnr{有文字的}{82.0}法,他說明完全圓滿、\twnr{遍純淨的梵行}{483.0}。」

  「學生婆羅門!那樣的話,請你以我的名義,以明天的食事邀請沙門優陀夷。」

  「是的,尊師!」

  那位學生婆羅門回答韋拉哈迦尼氏族的婆羅門尼後,去見尊者優陀夷。抵達後,對尊者優陀夷說這個:

  「優陀夷尊師!請你同意我們女老師韋拉哈迦尼氏族婆羅門尼的明天的食事。」

  尊者優陀夷以沈默狀態同意。

  那時,那夜過後,尊者優陀夷午前時穿衣、拿起衣鉢後,去韋拉哈迦尼氏族的婆羅門尼住處。抵達後,在設置的座位坐下。

  那時,韋拉哈迦尼氏族的婆羅門尼以勝妙的\twnr{硬食、軟食}{153.0}親手款待尊者優陀夷,使之滿足。

  那時,韋拉哈迦尼氏族的婆羅門尼對已食、手離鉢的尊者優陀夷穿上鞋、坐在高的座位上、覆蓋頭後,對尊者優陀夷說這個:「沙門!請你說法。」

  說:「姊妹!將會有機會。」後,從座位起來離開。

  第二次,那位學生婆羅門還去見尊者優陀夷。抵達後,與尊者優陀夷一起互相問候。交換應該被互相問候的友好交談後,在一旁坐下。尊者優陀夷對在一旁坐下的那位學生婆羅門以法說開示、勸導、鼓勵、使歡喜。

  第二次,也被尊者優陀夷以法說開示、勸導、鼓勵、使歡喜的那位學生婆羅門從座位起來後,去見韋拉哈迦尼氏族的婆羅門尼。抵達後,對韋拉哈迦尼氏族的婆羅門尼說這個:

  「真的,尊師!你應該知道,沙門優陀夷教導開頭是善的、中間是善的、結尾是善的;有意義的、有文字的法,他說明完全圓滿、遍純淨的梵行。」

  「學生婆羅門!你又同樣地稱讚沙門優陀夷,但當沙門優陀夷被說:『沙門!請你說法。』時,說:『姊妹!將會有機會。』後,從座位起來已離開。」

  「尊師!可是因為你像那樣穿上鞋、坐在高的座位上、覆蓋頭後,對尊者優陀夷說這個:『沙門!請你說法。』因為那些重法的尊師是法的尊重者。」

  「學生婆羅門!那樣的話,請你以我的名義,以明天的食事邀請沙門優陀夷。」

  「是的,尊師!」

  那位學生婆羅門回答韋拉哈迦尼氏族的婆羅門尼後,去見尊者優陀夷。抵達後,對尊者優陀夷說這個:

  「優陀夷尊師!請你同意我們女老師韋拉哈迦尼氏族婆羅門尼的明天的食事。」

  尊者優陀夷以沈默狀態同意。

  那時,那夜過後,尊者優陀夷午前時穿衣、拿起衣鉢後,去韋拉哈迦尼氏族的婆羅門尼住處。抵達後,在設置的座位坐下。

  那時,韋拉哈迦尼氏族的婆羅門尼以勝妙的硬食、軟食親手款待尊者優陀夷,使之滿足。

  那時,韋拉哈迦尼氏族的婆羅門尼對已食、手離鉢的尊者優陀夷脫下鞋、坐在低的座位上、敞開頭後,對尊者優陀夷說這個:

  「\twnr{大德}{45.0}!在什麼存在時,\twnr{阿羅漢}{5.0}們\twnr{安立}{143.0}苦樂?在什麼不存在時,阿羅漢們不安立苦樂?」

  「姊妹!在眼存在時,阿羅漢們安立苦樂;在眼不存在時,阿羅漢們不安立苦樂……(中略)在舌存在時,阿羅漢們安立苦樂;在舌不存在時,阿羅漢們不安立苦樂……(中略)在意存在時,阿羅漢們安立苦樂;在意不存在時,阿羅漢們不安立苦樂。」

  在這麼說時,韋拉哈迦尼氏族的婆羅門尼對尊者優陀夷說這個:

  「大德!太偉大了,大德!太偉大了,大德!猶如扶正顛倒的,或揭開隱藏的,或告知迷路者的道路,或在黑暗中持燈火:『有眼者們看見諸色。』同樣的,法被\twnr{聖}{612.1}優陀夷以種種\twnr{法門}{562.0}說明。聖優陀夷!這個我歸依\twnr{世尊}{12.0}、法、\twnr{比丘僧團}{65.0},請聖優陀夷記得我為\twnr{優婆夷}{99.0},從今天起\twnr{已終生歸依}{64.0}。」

  屋主品第十三,其\twnr{攝頌}{35.0}:

  「毘舍離、跋耆、那爛陀,婆羅墮若、輸屢那、瞿師羅,

   訶梨迪迦尼、那拘羅的父親,魯西遮、韋拉哈迦尼。」





\pin{天臂品}{134}{145}
\sutta{134}{134}{天臂經}{https://agama.buddhason.org/SN/sn.php?keyword=35.134}
  \twnr{有一次}{2.0},\twnr{世尊}{12.0}住在釋迦族人中,名叫天臂的釋迦族城鎮。

  在那裡,世尊召喚\twnr{比丘}{31.0}們:

  「比丘們!我不說:『對所有比丘來說在\twnr{六觸處}{78.0}上都有以不放逸應該被作的。』比丘們!而且,我也不說:『對所有比丘來說在六觸處上都沒有以不放逸應該被作的。』

  比丘們!凡那些漏已滅盡的、已完成的、\twnr{應該被作的已作的}{20.0}、負擔已卸的、\twnr{自己的利益已達成}{189.0},\twnr{有之結已滅盡的}{190.0}、以\twnr{究竟智}{191.0}解脫的\twnr{阿羅漢}{5.0}比丘,比丘們!我說:『對那些比丘來說在六觸處上沒有以不放逸應該被作的。』那是什麼原因?對他們來說以不放逸已作,他們不可能放逸。

  比丘們!但凡那些心意未達成、住於希求著無上\twnr{軛安穩}{192.0}的\twnr{有學}{193.0}比丘,比丘們!我說:『對那些比丘來說在六觸處上有以不放逸應該被作的。』那是什麼原因?比丘們!有能被眼識知,悅意及不悅意的諸色,他們一再接觸那些後心不\twnr{持續遍取}{530.0},從心的不遍取,活力已被發動成為不退縮的,\twnr{念已現起}{341.0}成為不忘失的,\twnr{身已寧靜成為無激情的}{330.0},已得定成為\twnr{心一境的}{255.0},比丘們!當考慮這不放逸結果時,我說:『對那些比丘來說在六觸處上有以不放逸應該被作的。』……(中略)比丘們!有能被意識知,悅意及不悅意的諸法,他們一再接觸那些後心不持續遍取,從心的不遍取,活力已被發動成為不退縮的,念已現起成為不忘失的,身已寧靜成為無激情的,心一境性成為得定的,比丘們!當考慮這不放逸結果時,我說:『對那些比丘來說在六觸處上有以不放逸應該被作的。』」



\sutta{135}{135}{機會經}{https://agama.buddhason.org/SN/sn.php?keyword=35.135}
  「\twnr{比丘}{31.0}們!對你們來說是諸利得,比丘們!對你們來說是好的獲得:對你們來說為了梵行生活的機會已得到。

  比丘們!\twnr{名叫六觸處的地獄}{x510}被我看見,在那裡,凡任何者以眼看見色,只看見不想要的色、非想要的色;只看見不可愛的色、非可愛的色;只看見不合意的色、非合意的色。

  凡任何者以耳聽聞聲音……(中略)凡任何者以鼻嗅聞氣味……(中略)凡任何者以舌嚐味道……(中略)凡任何者以身觸\twnr{所觸}{220.2}……(中略)凡任何者以意識知法,只識知不想要的法、非想要的法;只看見不可愛的法、非可愛的法;只看見不合意的色、非合意的法。

  比丘們!對你們來說是諸利得,比丘們!對你們來說是好的獲得:對你們來說為了梵行生活的機會已得到。

  比丘們!\twnr{名叫六觸處的天界}{x511}被我看見,在那裡,凡任何者以眼看見色,只看見令人滿意的色、非不令人滿意的色;只看見可愛的色、非不可愛的色;只看見合意的色、非不合意的色。……(中略)凡任何者以舌嚐味道……(中略)凡任何者以意識知法,只識知令人滿意的法、非不令人滿意的法;只看見可愛的法、非不可愛的法;只看見合意的法、非不合意的法。

  比丘們!對你們來說是諸利得,比丘們!對你們來說是好的獲得:對你們來說為了梵行生活的機會已得到。」



\sutta{136}{136}{色的快樂經第一}{https://agama.buddhason.org/SN/sn.php?keyword=35.136}
  「\twnr{比丘}{31.0}們!天-人們有色的快樂,樂於色,喜於色,比丘們!以色的變易、\twnr{褪去}{77.0}、滅,天-人們住於苦。

  比丘們!天-人們有聲音的快樂,樂於聲音,喜於聲音,比丘們!以聲音的變易、褪去、滅,天-人們住於苦。有氣味的快樂……有味道的快樂……有\twnr{所觸}{220.2}的快樂……比丘們!天-人們有法的快樂,樂於法,喜於法,比丘們!以法的變易、褪去、滅,天-人們住於苦。

  比丘們!但\twnr{如來}{4.0}、\twnr{阿羅漢}{5.0}、\twnr{遍正覺者}{6.0}如實知道諸色的\twnr{集起}{67.0}、滅沒、\twnr{樂味}{295.0}、\twnr{過患}{293.0}、\twnr{出離}{294.0}後,沒有色的快樂,不樂於色,不喜於色,比丘們!以色的變易、褪去、滅,如來\twnr{住於樂}{317.0}。

  諸聲音的……諸氣味的……諸味道的……諸所觸的……如實知道諸法的集起、滅沒、樂味、過患、出離後,沒有法的快樂,不樂於法,不喜於法,比丘們!以法的變易、褪去、滅,如來住於樂。」

  \twnr{世尊}{12.0}說這個,說這個後,\twnr{善逝}{8.0}、\twnr{大師}{145.0}更進一步說這個:

  「諸色、諸聲音、諸氣味、諸味道,諸所觸與全部的法,

   諸令人滿意的、諸可愛的與諸合意的:只要被稱為『它存在』。

   對包括天的世間,這些確實被認定為樂,

   但這些被滅之處,那個被他們認定為苦。

   被聖者們看作樂:\twnr{有身}{93.0}的滅,

   這是相反的:與全世間看見者們的。

   凡其他人們說為樂者,聖者們說它為苦,

   凡其他人們說為苦者,聖者們已知道它為樂。

   請你們看難了知的法,在這裡無智者被迷惑,

   對被覆蓋的沒看見者,是闇黑、黑暗。

   但對善的看見者,如被揭開的光明,

   在面前他們不了知:愚人們法的不熟知者。

   從被有貪征服者,\twnr{從被有貪隨行者}{x512},

   從到達魔領域者,這個法不容易被正覺。

   除了聖者之外,誰適合能正覺語句呢?

   凡當正確地了知語句後,無\twnr{漏}{188.0}者\twnr{般涅槃}{72.0}。」



\sutta{137}{137}{色的快樂經第二}{https://agama.buddhason.org/SN/sn.php?keyword=35.137}
  「\twnr{比丘}{31.0}們!天-人們有色的快樂,樂於色,喜於色;比丘們!以色的變易、\twnr{褪去}{77.0}、滅,天-人們住於苦。

  比丘們!天-人們有聲音的快樂……有氣味的快樂……有味道的快樂……有\twnr{所觸}{220.2}的快樂……比丘們!天-人們有法的快樂,樂於法,喜於法;比丘們!以法的變易、褪去、滅,天-人們住於苦。

  比丘們!但\twnr{如來}{4.0}、\twnr{阿羅漢}{5.0}、\twnr{遍正覺者}{6.0}如實知道諸色的\twnr{集起}{67.0}、滅沒、\twnr{樂味}{295.0}、\twnr{過患}{293.0}、\twnr{出離}{294.0}後,沒有色的快樂,不樂於色,不喜於色;比丘們!以色的變易、褪去、滅,如來\twnr{住於樂}{317.0}。

  諸聲音的……諸氣味的……諸味道的……諸所觸的……如實知道諸法的集起、滅沒、樂味、過患、出離後,沒有法的快樂,不樂於法,不喜於法,比丘們!以法的變易、褪去、滅,如來住於樂。」



\sutta{138}{138}{非你們的經第一}{https://agama.buddhason.org/SN/sn.php?keyword=35.138}
  「\twnr{比丘}{31.0}們!凡非你們的,你們要捨斷它!它被捨斷,對你們將有利益、安樂。比丘們!而什麼是非你們的?

  比丘們!眼非你們的,你們要捨斷它!它被捨斷,對你們將有利益、安樂。

  ……(中略)舌非你們的,你們要捨斷它!它被捨斷,對你們將有利益、安樂。……(中略)意非你們的,你們要捨斷它!它被捨斷,對你們將有利益、安樂。

  比丘們!猶如凡在這祇樹林中的草、薪木、枝條、樹葉,[某]人帶走它,或燃燒,或如需要做,是否你們這麼想:『[某]人帶走我們,或燃燒,或如需要做。』呢?」

  「\twnr{大德}{45.0}!這確實不是,那是什麼原因?大德!因為對我們這不是自己,或自己的。」

  「同樣的,比丘們!眼非你們的,你們要捨斷它!它被捨斷,對你們將有利益、安樂。……(中略)舌非你們的,你們要捨斷它!它被捨斷,對你們將有利益、安樂。……(中略)意非你們的,你們要捨斷它!它被捨斷,對你們將有利益、安樂。」[\suttaref{SN.35.101}]



\sutta{139}{139}{非你們的經第二}{https://agama.buddhason.org/SN/sn.php?keyword=35.139}
  「\twnr{比丘}{31.0}們!凡非你們的,你們要捨斷它!它被捨斷,對你們將有利益、安樂。比丘們!而什麼是非你們的?

  比丘們!諸色非你們的,你們要捨斷它們!它們被捨斷,對你們將有利益、安樂。

  諸聲音……諸氣味……諸味道……諸\twnr{所觸}{220.2}……諸法非你們的,你們要捨斷它們!它們被捨斷,對你們將有利益、安樂。

  比丘們!猶如在這祇樹林中……(中略)。同樣的,比丘們!諸色非你們的,你們要捨斷它們!它們被捨斷,對你們將有利益、安樂。」



\sutta{140}{140}{自身內有因的無常經}{https://agama.buddhason.org/SN/sn.php?keyword=35.140}
  「\twnr{比丘}{31.0}們!眼是無常的,為了眼生起的該因及該\twnr{緣}{180.0},那個也是無常的,比丘們!無常所生成的眼,將從哪裡有常的?

  ……(中略)舌是無常的,為了舌生起的該因及該緣,那個也是無常的,比丘們!無常所生成的舌,將從哪裡有常的?……(中略)意是無常的,為了意生起的該因及該緣,那個也是無常的,比丘們!無常所生成的意,將從哪裡有常的?

  比丘們!這麼看的\twnr{有聽聞的聖弟子}{24.0}在眼上\twnr{厭}{15.0}……(中略)在舌上厭……(中略)。厭者\twnr{離染}{558.0},從\twnr{離貪}{77.0}被解脫,在已解脫時,\twnr{有『[這是]解脫』之智}{27.0},他知道:『\twnr{出生已盡}{18.0},\twnr{梵行已完成}{19.0},\twnr{應該被作的已作}{20.0},\twnr{不再有此處[輪迴]的狀態}{21.1}。』」



\sutta{141}{141}{自身內有因的苦經}{https://agama.buddhason.org/SN/sn.php?keyword=35.141}
  「\twnr{比丘}{31.0}們!眼是苦的,為了眼生起的該因及該\twnr{緣}{180.0},那個也是苦的,比丘們!苦所生成的眼,將從哪裡有樂的?

  ……(中略)舌是苦的,為了舌生起的該因及該緣,那個也是苦的,比丘們!苦所生成的舌,將從哪裡有樂的?……(中略)意是苦的,為了意生起的該因及該緣,那個也是苦的,比丘們!苦所生成的意,將從哪裡有樂的?

  這麼看的……(中略)他知道:『\twnr{出生已盡}{18.0},\twnr{梵行已完成}{19.0},\twnr{應該被作的已作}{20.0},\twnr{不再有此處[輪迴]的狀態}{21.1}。』」



\sutta{142}{142}{自身內有因的無我經}{https://agama.buddhason.org/SN/sn.php?keyword=35.142}
  「\twnr{比丘}{31.0}們!眼是無我,為了眼生起的該因及該\twnr{緣}{180.0},那個也是無我,比丘們!無我所生成的眼,將從哪裡有我?

  ……(中略)舌是無我,為了舌生起的該因及該緣,那個也是無我,比丘們!無我所生成的舌,將從哪裡有我?……(中略)意是無我,為了意生起的該因及該緣,那個也是無我,比丘們!無我所生成的意,將從哪裡有我?

  這麼看的……(中略)他知道:『\twnr{出生已盡}{18.0},\twnr{梵行已完成}{19.0},\twnr{應該被作的已作}{20.0},\twnr{不再有此處[輪迴]的狀態}{21.1}。』」



\sutta{143}{143}{外部有因的無常經}{https://agama.buddhason.org/SN/sn.php?keyword=35.143}
  「\twnr{比丘}{31.0}們!諸色是無常的,為了諸色生起的該因及該\twnr{緣}{180.0},那個也是無常的,比丘們!無常所生成的諸色,將從哪裡有常的?

  諸聲音……諸氣味……諸味道……諸\twnr{所觸}{220.2}……諸法是無常的,為了諸法生起的該因及該緣,那個也是無常的,比丘們!無常所生成的諸法,將從哪裡有常的?

  這麼看的……(中略)他知道:『\twnr{出生已盡}{18.0},\twnr{梵行已完成}{19.0},\twnr{應該被作的已作}{20.0},\twnr{不再有此處[輪迴]的狀態}{21.1}。』」



\sutta{144}{144}{外部有因的苦經}{https://agama.buddhason.org/SN/sn.php?keyword=35.144}
  「\twnr{比丘}{31.0}們!諸色是苦的,為了諸色生起的該因及該\twnr{緣}{180.0},那個也是苦的,比丘們!苦所生成的諸色,將從哪裡有樂的?

  諸聲音……諸氣味……諸味道……諸\twnr{所觸}{220.2}……諸法是苦的,為了諸法生起的該因及該緣,那個也是苦的,比丘們!苦所生成的諸法,將從哪裡有樂的?

  這麼看的……(中略)他知道:『\twnr{出生已盡}{18.0},\twnr{梵行已完成}{19.0},\twnr{應該被作的已作}{20.0},\twnr{不再有此處[輪迴]的狀態}{21.1}。』」



\sutta{145}{145}{外部有因的無我經}{https://agama.buddhason.org/SN/sn.php?keyword=35.145}
  「\twnr{比丘}{31.0}們!諸色是無我,為了諸色生起的該因及該\twnr{緣}{180.0},那個也是無我,比丘們!無我所生成的色,將從哪裡有我?

  諸聲音……諸氣味……諸味道……諸\twnr{所觸}{220.2}……諸法是無我,為了諸法生起的該因及該緣,那個也是無我,比丘們!無我所生成的諸法,將從哪裡有我?

  這麼看的有聽聞的聖弟子在諸色上厭,也在諸聲音上厭,也在氣味上……也在諸味道上……也在諸所觸上……也在諸法上厭。厭者離染,從\twnr{離貪}{77.0}被解脫,在已解脫時,\twnr{有『[這是]解脫』之智}{27.0},他知道:『\twnr{出生已盡}{18.0},\twnr{梵行已完成}{19.0},\twnr{應該被作的已作}{20.0},\twnr{不再有此處[輪迴]的狀態}{21.1}。』」

  天臂品第十四,其\twnr{攝頌}{35.0}:

  「天臂、機會、諸色,與非你們的二則,

   有因三說,自身內外二種。」





\pin{新與舊品}{146}{155}
\sutta{146}{146}{業滅經}{https://agama.buddhason.org/SN/sn.php?keyword=35.146}
  「\twnr{比丘}{31.0}們!我將教導新舊業、業滅,以及導向業\twnr{滅道跡}{69.0},你們要聽它!你們\twnr{要好好作意}{43.1}!我將說。

  比丘們!而什麼是舊業呢?比丘們!眼是被造作的、\twnr{被思惟的}{827.0}舊業,應該被看作能被感受的[依處]。……(中略)舌是被造作的、被思惟的舊業,應該被看作能被感受的。……(中略)意是被造作的、被思惟的舊業,應該被看作能被感受的,比丘們!這被稱為舊業。

  比丘們!而什麼是新業呢?比丘們!凡現在以身、語、意做業,比丘們!這被稱為新業。

  比丘們!而什麼是業滅呢?比丘們!凡以身業、語業、意業的滅觸達解脫,比丘們!這被稱為業滅。

  比丘們!而什麼是導向業滅道跡呢?就是這\twnr{八支聖道}{525.0},即:正見、正志、正語、正業、正命、正精進、正念、正定,比丘們!這被稱為導向業滅道跡。

  比丘們!像這樣,舊業被我教導,新業被教導,業滅被教導,導向業滅道跡被教導。比丘們!凡\twnr{出自憐愍}{121.0}應該被老師、利益者、憐愍者為了弟子作的,那個被我為你們做了。比丘們!有這些樹下、這些空屋,比丘們!你們要修禪,不要放逸,不要以後成為後悔者,這是我們為你們的教誡。」



\sutta{147}{147}{無常-有益涅槃經}{https://agama.buddhason.org/SN/sn.php?keyword=35.147}
  「\twnr{比丘}{31.0}們!我將為你們教導有益涅槃道跡,\twnr{你們要聽}{43.0}它!……(中略)。

  比丘們!而哪個是那個有益涅槃道跡?

  比丘們!這裡,比丘看『眼是無常的。』看『諸色是無常的。』看『眼識是無常的。』看『眼觸是無常的。』看『又凡以這眼觸\twnr{為緣}{180.0}生起感受的樂,或苦,或不苦不樂那也是無常的。』

  ……(中略)看『舌是無常的。』看『諸味道是無常的。』看『舌識是無常的。』看『舌觸是無常的。』看『又凡以這舌觸為緣生起感受的樂,或苦,或不苦不樂那也是無常的。』……(中略)看『意是無常的。』看『諸法是無常的。』看『意識是無常的。』看『意觸是無常的。』看『又凡以這意觸為緣生起感受的樂,或苦,或不苦不樂那也是無常的。』

  比丘們!這是那個有益涅槃道跡。」



\sutta{148}{148}{苦-有益涅槃經}{https://agama.buddhason.org/SN/sn.php?keyword=35.148}
  「\twnr{比丘}{31.0}們!我將為你們教導有益涅槃道跡,\twnr{你們要聽}{43.0}它!……(中略)。

  比丘們!而哪個是那個有益涅槃道跡?

  比丘們!這裡,比丘看『眼是苦的。』看『諸色是苦的。』看『眼識是苦的。』看『眼觸是苦的。』看『又凡以這眼觸\twnr{為緣}{180.0}生起感受的樂,或苦,或不苦不樂那也是苦的。』

  ……(中略)看『舌是苦的。』……(中略)看『意是苦的。』看『諸法是苦的。』看『意識是苦的。』看『意觸是苦的。』看『又凡以這意觸為緣生起感受的樂,或苦,或不苦不樂那也是苦的。』

  比丘們!這是那個有益涅槃道跡。」



\sutta{149}{149}{無我-有益涅槃經}{https://agama.buddhason.org/SN/sn.php?keyword=35.149}
  「\twnr{比丘}{31.0}們!我將為你們教導有益涅槃道跡,\twnr{你們要聽}{43.0}它!……(中略)。

  比丘們!而哪個是那個有益涅槃道跡?

  比丘們!這裡,比丘看『眼是無我。』看『諸色是無我。』看『眼識是無我。』看『眼觸是無我。』看『又凡以這眼觸\twnr{為緣}{180.0}生起感受的樂,或苦,或不苦不樂那也是無我。』

  ……(中略)看『意是無我。』看『諸法是無我。』看『意識是無我。』看『意觸是無我。』看『又凡以這意觸為緣生起感受的樂,或苦,或不苦不樂那也是無我。』

  比丘們!這是那個有益涅槃道跡。」



\sutta{150}{150}{有益涅槃道跡經}{https://agama.buddhason.org/SN/sn.php?keyword=35.150}
  「\twnr{比丘}{31.0}們!我將為你們教導有益涅槃道跡,\twnr{你們要聽}{43.0}它!……(中略)。

  比丘們!而哪個是那個有益涅槃道跡?

  比丘們!你們怎麼想它:眼是常的,或是無常的?」

  「無常的,\twnr{大德}{45.0}!」

  「那麼,凡為無常的,那是苦的或樂的?」

  「苦的,大德!」

  「而凡為無常、苦、\twnr{變易法}{70.0},適合認為它:『\twnr{這是我的}{32.0},\twnr{我是這個}{33.0},\twnr{這是我的真我}{34.1}。』嗎?」

  「大德!這確實不是。」

  「諸色是常的,或是無常的?」

  「無常的,大德!」……(中略)

  「眼識……眼觸……又凡以這意觸為緣生起感受的樂,或苦,或不苦不樂,那也是常的,或是無常的?」

  「無常的,大德!」

  「那麼,凡為無常的,那是苦的或樂的?」

  「苦的,大德!」

  「而凡為無常、苦、變易法,適合認為它:『\twnr{這是我的}{32.0},\twnr{我是這個}{33.0},這是\twnr{我的真我}{34.0}。』嗎?」

  「大德!這確實不是。」

  比丘們!這麼看的\twnr{有聽聞的聖弟子}{24.0}在眼上\twnr{厭}{15.0},也在諸色上厭,也在眼識上厭,也在眼觸上厭……(中略)又凡以這意觸為緣生起感受的樂,或苦,或不苦不樂,在那個上也厭。厭者\twnr{離染}{558.0},從\twnr{離貪}{77.0}被解脫……(中略)他知道:『……\twnr{不再有此處[輪迴]的狀態}{21.1}。』」

  比丘們!這是那個有益涅槃道跡。」



\sutta{151}{151}{徒弟經}{https://agama.buddhason.org/SN/sn.php?keyword=35.151}
  「\twnr{比丘}{31.0}們!這梵行被住於無\twnr{徒弟}{x513}與無師父。比丘們!有徒弟、有師父的比丘住於苦,是不安樂的。比丘們!無徒弟、無師父的比丘\twnr{住於樂}{317.0},是安樂的。而怎樣是有徒弟、有師父的比丘住於苦,是不安樂的?比丘們!這裡,以眼見色後,比丘的諸惡不善法生起:諸隨順結的憶念、意向。『\twnr{它們住於他之內}{x514};諸惡不善法住於他之內。』因此,他被稱為『有徒弟』。『\twnr{它們征服他}{x515};惡不善法征服他。』因此,他被稱為『有師父』。……(中略)

  再者,比丘們!以舌嚐味道後,比丘的諸惡不善法生起:諸隨順結的憶念、意向。『它們住於他之內;惡不善法住於他之內。』因此,他被稱為『有徒弟』。『它們征服他;惡不善法征服他。』因此,他被稱為『有師父』。……(中略)

  再者,比丘們!比丘以意識知法後,比丘的諸惡不善法生起:諸隨順結的憶念、意向。『它們住於他之內;惡不善法住於他之內。』因此,他被稱為『有徒弟』。『它們征服他;惡不善法征服他。』因此,他被稱為『有師父』。比丘們!這樣是有徒弟、有師父的比丘住於苦,是不安樂的。

  比丘們!而怎樣是無徒弟、無師父的比丘住於樂,是安樂的?比丘們!這裡,以眼見色後,比丘的諸惡不善法不生起:諸隨順結的憶念、意向。『它們不住於他之內;惡不善法不住於他之內。』因此,他被稱為『無徒弟』。『它們不征服他;惡不善法不征服他。』因此,他被稱為『無師父』。……(中略)再者,比丘們!以舌嚐味道後,比丘的諸惡不善法不生起:諸隨順結的憶念、意向。『它們不住於他之內;惡不善法不住於他之內。』因此,他被稱為『無徒弟』。『它們不征服他;惡不善法不征服他。』因此,他被稱為『無師父』。……(中略)再者,比丘們!比丘以意識知法後,比丘的諸惡不善法不生起:諸隨順結的憶念、意向。『它們不住於他之內;惡不善法不住於他之內。』因此,他被稱為『無徒弟』。『它們不征服他;惡不善法不征服他。』因此,他被稱為『無師父』。比丘們!這樣是無徒弟、無師父的比丘住於樂,是安樂的。比丘們!這梵行被住於無徒弟、無師父。比丘們!有徒弟、有師父的比丘住於苦,是不安樂的。比丘們!無徒弟、無師父的比丘住於樂,是安樂的。」



\sutta{152}{152}{為了什麼目的梵行經}{https://agama.buddhason.org/SN/sn.php?keyword=35.152}
  「\twnr{比丘}{31.0}們!如果其他外道\twnr{遊行者}{79.0}們這麼問你們:『\twnr{道友}{201.0}們!為了什麼目的在\twnr{沙門}{29.0}\twnr{喬達摩}{80.0}處梵行被住?』比丘們!被這麼問,你們應該這麼回答那些其他外道遊行者:『道友們!為了苦的\twnr{遍知}{154.0}在\twnr{世尊}{12.0}處梵行被住。』

  比丘們!如果其他外道遊行者們再這麼問你們:『道友們!那麼,哪個是苦的,為了那個的遍知在沙門喬達摩處梵行被住?』比丘們!被這麼問,你們應該這麼回答那些其他外道遊行者:『道友們!眼是苦的,為了那個的遍知在世尊處梵行被住;諸色是苦的,為了那個的遍知在世尊處梵行被住;眼識是苦的,為了那個的遍知在世尊處梵行被住;眼觸是苦的,為了那個的遍知在世尊處梵行被住;又凡以這眼觸\twnr{為緣}{180.0}生起感受的樂,或苦,或不苦不樂,那個也是苦的,為了那個的遍知在世尊處梵行被住……(中略)舌是苦的……;意是苦的,為了那個的遍知在世尊處梵行被住……(中略);又凡以這意觸為緣生起感受的樂,或苦,或不苦不樂,那個也是苦的,為了那個的遍知在世尊處梵行被住。道友們!這是苦的,為了那個的遍知在世尊處梵行被住。』比丘們!被這麼問,你們應該這麼回答那些其他外道遊行者。」[≃\suttaref{SN.35.81}]



\sutta{153}{153}{有法門嗎經}{https://agama.buddhason.org/SN/sn.php?keyword=35.153}
  「\twnr{比丘}{31.0}們!有\twnr{法門}{562.0},由於該法門,比丘就\twnr{除了}{908.0}從信外,除了從愛好外,除了從口傳外,除了從理由的遍尋思外,除了\twnr{從見解的審慮接受}{609.0}外,記說\twnr{完全智}{489.0}:『我知道:「\twnr{出生已盡}{18.0},\twnr{梵行已完成}{19.0},\twnr{應該被作的已作}{20.0},\twnr{不再有此處[輪迴]的狀態}{21.1}。」』嗎?」「\twnr{大德}{45.0}!我們的法是\twnr{世尊}{12.0}為根本的、\twnr{世尊為導引的}{56.0}、世尊為依歸的,大德!就請世尊說明這個所說的義理,\twnr{那就好了}{44.0}!聽聞世尊的[教說]後,比丘們將會\twnr{憶持}{57.0}。」「比丘們!那樣的話,你們要聽!你們要\twnr{好好作意}{43.1}!我將說。」「是的,大德!」那些比丘回答世尊。世尊說這個:「比丘們!有法門,由於該法門,比丘就除了從信外,除了從愛好外,除了從口傳外,除了從理由的遍尋思外,除了從見解的審慮接受外,記說完全智:『我知道:「出生已盡,梵行已完成,應該被作的已作,不再有此處[輪迴]的狀態。」』

  比丘們!而什麼法門,由於該法門,比丘就除了從信外……(中略)除了從見解的審慮接受外,記說完全智:『我知道:「出生已盡,梵行已完成,應該被作的已作,不再有此處[輪迴]的狀態。」』呢?

  比丘們!這裡,比丘以眼見色後,當自身內有貪、瞋、癡時,知道:『我自身內有貪、瞋、癡。』或當自身內沒有貪、瞋、癡時,知道:『我自身內沒有貪、瞋、癡。』比丘們!凡那位比丘以眼見色後,當自身內有貪、瞋、癡時,知道:『我自身內有貪、瞋、癡。』或當自身內沒有貪、瞋、癡時,知道:『我自身內沒有貪、瞋、癡。』我自身,比丘們!是否這些法能被信感知(經驗),或能被愛好感知,或能被口傳感知,或能被理由的遍尋思感知,或能被見解的審慮接受感知?」「大德!這確實不是。」「比丘們!這些法以慧看見後能被感知,不是嗎?」「是的,大德!」「比丘們!這是法門,由於該法門,比丘就除了從信外,除了從愛好外,除了從口傳外,除了從理由的遍尋思外,除了從見解的審慮接受外,記說完全智:『我知道:「出生已盡,梵行已完成,應該被作的已作,不再有此處[輪迴]的狀態。」』……(中略)。

  再者,比丘們!比丘以舌嚐味道後,當自身內……(中略),知道:『……貪、瞋、癡。』或當自身內沒有貪、瞋、癡時,知道:『我自身內沒有貪、瞋、癡。』比丘們!凡這位比丘以舌嚐味道後,當自身內有貪、瞋、癡時,知道:『我自身內有貪、瞋、癡。』或當自身內沒有貪、瞋、癡時,知道:『我自身內沒有貪、瞋、癡。』我自身,比丘們!是否這些法能被信感知(經驗),或能被愛好感知,或能被口傳感知,或能被理由的遍尋思感知,或能被見解的審慮接受感知?」「大德!這確實不是。」「比丘們!這些法以慧看見後能被感知,不是嗎?」「是的,大德!」「比丘們!這也是法門,由於該法門,比丘就除了從信外,除了從愛好外,除了從口傳外,除了從理由的遍尋思外,除了從見解的審慮接受外,記說完全智:『我知道:「出生已盡,梵行已完成,應該被作的已作,不再有此處[輪迴]的狀態。」』

  再者,比丘們!比丘以意識知法後,當自身內有貪、瞋、癡時,知道:『我自身內有貪、瞋、癡。』或當自身內沒有貪、瞋、癡時,知道:『我自身內沒有貪、瞋、癡。』比丘們!凡這位比丘以意識知法後,當自身內有貪、瞋、癡時,知道:『我自身內有貪、瞋、癡。』或當自身內沒有貪、瞋、癡時,知道:『我自身內沒有貪、瞋、癡。』,比丘們!是否這些法能被信感知(經驗),或能被愛好感知,或能被口傳感知,或能被理由的遍尋思感知,或能被見解的審慮接受感知?」「大德!這確實不是。」「比丘們!這些法以慧看見後能被感知,不是嗎?」「是的,大德!」「比丘們!這也是法門,由於該法門,比丘就除了從信外,除了從愛好外,除了從口傳外,除了從理由的遍尋思外,除了從見解的審慮接受外,記說完全智:『我知道:「出生已盡,梵行已完成,應該被作的已作,不再有此處[輪迴]的狀態。」』」



\sutta{154}{154}{根具足者經}{https://agama.buddhason.org/SN/sn.php?keyword=35.154}
  那時,\twnr{某位比丘}{39.0}去見\twnr{世尊}{12.0}。……(中略)在一旁坐下的那位比丘對世尊說這個:  

  「\twnr{大德}{45.0}!被稱為『根具足者、根具足者』大德!什麼情形是根具足者呢?」

   「比丘!如果當在眼根上住於\twnr{隨看著生滅}{427.0}時,他在眼根上\twnr{厭}{15.0}……(中略)比丘!如果當在舌根上住於隨看著生滅時,他在舌根上厭……(中略)比丘!如果當在意根上住於隨看著生滅時,他在意根上厭……(中略)厭者\twnr{離染}{558.0}……(中略)在已解脫時,\twnr{有『[這是]解脫』之智}{27.0},他知道:『\twnr{出生已盡}{18.0},\twnr{梵行已完成}{19.0},\twnr{應該被作的已作}{20.0},\twnr{不再有此處[輪迴]的狀態}{21.1}。』比丘!這個情形是根具足者。」



\sutta{155}{155}{說法者的詢問經}{https://agama.buddhason.org/SN/sn.php?keyword=35.155}
  那時,\twnr{某位比丘}{39.0}去見\twnr{世尊}{12.0}……(中略)在一旁坐下的那位比丘對世尊說這個:

  「\twnr{大德}{45.0}!被稱為『說法者,說法者』,大德!那麼,什麼情形才是說法者?」

  「比丘!如果對眼為了\twnr{厭}{15.0}、\twnr{離貪}{77.0}、\twnr{滅}{68.0}教導法,『說法者比丘』是適當的言語。

  比丘!如果對眼是為了厭、離貪、滅的行者,『\twnr{法、隨法行者}{58.0}比丘』是適當的言語。

  比丘!如果對眼從厭、離貪、滅,不執取後成為解脫者,『得\twnr{當生}{42.0}涅槃比丘』是適當的言語。

  ……(中略)比丘!如果對舌是為了厭、離貪、滅而教導法,『說法者比丘』是適當的言語。……(中略)。比丘!如果對意是為了厭、離貪、滅而教導法,『說法者比丘』是適當的言語。比丘!如果對意是為了厭、離貪、滅的行者,『法、隨法行者比丘』是適當的言語。比丘!如果對意從厭、離貪、滅,不執取後成為解脫者,『得當生涅槃者比丘』是適當的言語。」[≃\suttaref{SN.12.16}]

  新舊品第十五,其\twnr{攝頌}{35.0}:

  「業、四則有益的、無徒弟、為了什麼目,

   有法門嗎,以及根與說[法]者。」

  六處篇第三個五十則完成,其品的攝頌:

  「軛安穩與世間,屋主與以天臂,

   以新舊五十則,以那個被稱為第三。」





\pin{歡喜的滅盡品}{156}{167}
\sutta{156}{156}{自身內之歡喜的滅盡經}{https://agama.buddhason.org/SN/sn.php?keyword=35.156}
  「\twnr{比丘}{31.0}們!比丘看『無常的眼』只是無常的,那是他的正見。正確看見者\twnr{厭}{15.0},從歡喜的\twnr{滅盡}{273.0}有貪的滅盡;從貪的滅盡有歡喜的滅盡。從歡喜、貪的滅盡被稱為『心\twnr{善解脫}{28.0}』[\suttaref{SN.22.51}:心被解脫,被稱為『善解脫』]……(中略)比丘們!比丘看『無常的舌』只是無常的,那是他的正見。正確看見者厭,從歡喜的滅盡有貪的滅盡;從貪的滅盡……(中略)被稱為『心善解脫。』……(中略)比丘們!比丘看『無常的意』只是無常的:那是他的正見。正確看見者厭,從歡喜的滅盡有貪的滅盡;從貪的滅盡有歡喜的滅盡。從歡喜、貪的滅盡被稱為『心善解脫』。」



\sutta{157}{157}{自身外之歡喜的滅盡經}{https://agama.buddhason.org/SN/sn.php?keyword=35.157}
  「\twnr{比丘}{31.0}們!比丘看『無常的諸色』只是無常的,那是他的正見。正確看見者\twnr{厭}{15.0},從歡喜的\twnr{滅盡}{273.0}有貪的滅盡;從貪的滅盡有歡喜的滅盡。從歡喜、貪的滅盡被稱為『心\twnr{善解脫}{28.0}』[\suttaref{SN.22.51}:心被解脫,被稱為『善解脫』]。

  比丘們!比丘看『無常的諸聲音』只是無常的……諸氣味……諸味道……諸\twnr{所觸}{220.2}……看『無常的諸法』只是無常的,那是他的正見。正確看見者厭,從歡喜的滅盡有貪的滅盡;從貪的滅盡有歡喜的滅盡。從歡喜、貪的滅盡被稱為『心善解脫』。」



\sutta{158}{158}{自身內無常之歡喜的滅盡經}{https://agama.buddhason.org/SN/sn.php?keyword=35.158}
  「\twnr{比丘}{31.0}們!你們要\twnr{如理作意}{114.0}眼,你們要如實察覺眼的無常性。比丘們!當比丘如理作意眼、如實察覺眼的無常性時,在眼上\twnr{厭}{15.0},從歡喜的\twnr{滅盡}{273.0}有貪的滅盡;從貪的滅盡有歡喜的滅盡。從歡喜、貪的滅盡被稱為『心\twnr{善解脫}{28.0}』[\suttaref{SN.22.52}:心被解脫,被稱為『善解脫』]

  比丘們!你們要如理作意耳……鼻……比丘們!你們要如理作意舌,你們要如實察覺舌的無常性。比丘們!當比丘如理作意舌、如實察覺舌的無常性時,在舌上厭,從歡喜的滅盡有貪的滅盡;從貪的滅盡有歡喜的滅盡。從歡喜、貪的滅盡被稱為心善解脫。身……比丘們!你們要如理作意意,你們要如實察覺意的無常性。比丘們!當比丘如理作意意、如實察覺意的無常性時,也在意上厭,從歡喜的滅盡有貪的滅盡;從貪的滅盡有歡喜的滅盡。從歡喜、貪的滅盡被稱為『心善解脫』。」



\sutta{159}{159}{自身外無常之歡喜的滅盡經}{https://agama.buddhason.org/SN/sn.php?keyword=35.159}
  「\twnr{比丘}{31.0}們!你們要如理作意諸色,你們要如實察覺色的無常性。比丘們!當比丘如理作意諸色、如實察覺色的無常性時,在色上\twnr{厭}{15.0},從歡喜的\twnr{滅盡}{273.0}有貪的滅盡;從貪的滅盡有歡喜的滅盡。從歡喜、貪的滅盡被稱為『心\twnr{善解脫}{28.0}』[\suttaref{SN.22.52}:心被解脫,被稱為『善解脫』]。

  諸聲音……諸氣味……諸味道……諸\twnr{所觸}{220.2}……比丘們!你們要如理作意諸法,你們要如實察覺法的無常性。比丘們!當比丘如理作意法,如實察覺法的無常性時,在法上厭,從歡喜的滅盡有貪的滅盡;從貪的滅盡有歡喜的滅盡。從歡喜、貪的滅盡被稱為『心善解脫』。」



\sutta{160}{160}{耆婆的芒果園定經}{https://agama.buddhason.org/SN/sn.php?keyword=35.160}
  \twnr{有一次}{2.0},\twnr{世尊}{12.0}住在王舍城耆婆的芒果園。

  在那裡,世尊召喚\twnr{比丘}{31.0}們:「比丘們!」……(中略)。

  「比丘們!你們要\twnr{修習}{94.0}\twnr{定}{182.0},比丘們!得定的比丘\twnr{如實明瞭}{x501},而如實明瞭什麼?

  如實明瞭『眼是無常的』,如實明瞭『諸色是無常的』,如實明瞭『眼識是無常的』,如實明瞭『眼觸是無常的』,如實明瞭『又凡以這眼觸\twnr{為緣}{180.0}生起感受的樂,或苦,或不苦不樂,那也是無常的』。……(中略)如實明瞭『舌是無常的』……(中略)如實明瞭『意是無常的』,如實明瞭『諸法是無常的』……(中略)如實明瞭『又凡以這意觸為緣生起感受的樂,或苦,或不苦不樂,那也是無常的』。

  比丘們!你們要修習定,比丘們!得定的比丘如實明瞭。」



\sutta{161}{161}{耆婆的芒果園獨坐經}{https://agama.buddhason.org/SN/sn.php?keyword=35.161}
  \twnr{有一次}{2.0},\twnr{世尊}{12.0}住在王舍城耆婆的芒果園。

  在那裡,世尊召喚\twnr{比丘}{31.0}們:……(中略)。

  「比丘們!你們要著手努力於\twnr{獨坐}{92.0},比丘們!獨坐的比丘如實明瞭,而如實明瞭什麼?

  如實明瞭『眼是無常的』,如實明瞭『諸色是無常的』,如實明瞭『眼識是無常的』,如實明瞭『眼觸是無常的』,如實明瞭『又凡以這眼觸\twnr{為緣}{180.0}生起感受的樂,或苦,或不苦不樂,那也是無常的』。……(中略)如實明瞭『意是無常的』,諸法……意識……意觸……如實明瞭『又凡以這意觸為緣生起感受的樂,或苦,或不苦不樂,那也是無常的』。

  比丘們!你們要著手努力於獨坐,比丘們!獨坐的比丘如實明瞭。」



\sutta{162}{162}{拘絺羅-無常經}{https://agama.buddhason.org/SN/sn.php?keyword=35.162}
  那時,\twnr{尊者}{200.0}摩訶拘絺羅去見\twnr{世尊}{12.0}。……(中略)在一旁坐下的\twnr{尊者}{200.0}拘絺羅對世尊說這個:  

  「\twnr{大德}{45.0}!請世尊為我簡要地教導法,凡我聽聞世尊的法後,會住於單獨的、隱離的、不放逸的、熱心的、自我努力的,\twnr{那就好了}{44.0}!」

  「拘絺羅!凡是無常的,在那裡意欲應該被你捨斷。拘絺羅!而什麼是無常的?眼是無常的,在那裡意欲應該被你捨斷;諸色是無常的,在那裡意欲應該被你捨斷;眼識是無常的,在那裡意欲應該被你捨斷;眼觸是無常的,在那裡意欲應該被你捨斷;又凡以這眼觸\twnr{為緣}{180.0}生起感受的樂,或苦,或不苦不樂那也是無常的,在那裡意欲應該被你捨斷……(中略)舌是無常的,在那裡意欲應該被你捨斷;諸味道是無常的,在那裡意欲應該被你捨斷;舌識是無常的,在那裡意欲應該被你捨斷;舌觸是無常的,在那裡意欲應該被你捨斷;又凡以這舌觸為緣生起感受的樂,或苦,或不苦不樂那也是無常的,在那裡意欲應該被你捨斷……(中略)意是無常的,在那裡意欲應該被你捨斷;諸法是無常的,在那裡意欲應該被你捨斷;意識是無常的,在那裡意欲應該被你捨斷;意觸是無常的,在那裡意欲應該被你捨斷;又凡以這意觸為緣生起感受的樂,或苦,或不苦不樂那也是無常的,在那裡意欲應該被你捨斷,拘絺羅!凡是無常的,在那裡意欲應該被你捨斷。」



\sutta{163}{163}{拘絺羅-苦經}{https://agama.buddhason.org/SN/sn.php?keyword=35.163}
  那時,\twnr{尊者}{200.0}摩訶拘絺羅……(中略)對\twnr{世尊}{12.0}說這個:  

  「\twnr{大德}{45.0}!……為我……(中略)\twnr{那就好了}{44.0}!」

  「拘絺羅!凡是苦的,在那裡意欲應該被你捨斷。拘絺羅!而什麼是苦的?拘絺羅!眼是苦的,在那裡意欲應該被你捨斷;諸色是苦的,在那裡意欲應該被你捨斷;眼識是苦的,在那裡意欲應該被你捨斷;眼觸是苦的,在那裡意欲應該被你捨斷;又凡以這眼觸\twnr{為緣}{180.0}生起感受的樂,或苦,或不苦不樂那也是苦的,在那裡意欲應該被你捨斷……(中略)舌是苦的,在那裡意欲應該被你捨斷……(中略)意是苦的,在那裡意欲應該被你捨斷;諸法是苦的,在那裡意欲應該被你捨斷;意識是苦的,在那裡意欲應該被你捨斷;意觸是苦的,在那裡意欲應該被你捨斷;又凡以這意觸為緣生起感受的樂,或苦,或不苦不樂那也是苦的,在那裡意欲應該被你捨斷,拘絺羅!凡是苦的,在那裡意欲應該被你捨斷。」



\sutta{164}{164}{拘絺羅-無我經}{https://agama.buddhason.org/SN/sn.php?keyword=35.164}
  在一旁坐好後……(中略)能住於……。  

  「拘絺羅!凡是無我者,在那裡意欲應該被你捨斷。拘絺羅!什麼是無我,在那裡意欲應該被你捨斷呢?眼是無我,在那裡意欲應該被你捨斷;諸色是無我,在那裡意欲應該被你捨斷;眼識是無我,在那裡意欲應該被你捨斷;眼觸是無我,在那裡意欲應該被你捨斷;又凡以這眼觸\twnr{為緣}{180.0}生起感受的樂,或苦,或不苦不樂那也是無我,在那裡意欲應該被你捨斷……(中略)舌是無我,在那裡意欲應該被你捨斷……(中略)意是無我,在那裡意欲應該被你捨斷;諸法是無我,在那裡意欲應該被你捨斷;意識……意觸……又凡以這意觸為緣生起感受的樂,或苦,或不苦不樂那也是無我,在那裡意欲應該被你捨斷,拘絺羅!凡是無我,在那裡意欲應該被你捨斷。」



\sutta{165}{165}{邪見的捨斷經}{https://agama.buddhason.org/SN/sn.php?keyword=35.165}
  那時,\twnr{某位比丘}{39.0}去見\twnr{世尊}{12.0}。……(中略)在一旁坐下的那位比丘對世尊說這個:  

  「\twnr{大德}{45.0}!當怎樣知、當怎樣見時,邪見被捨斷呢?」

  「比丘!當知、當見眼是無常的時,邪見被捨斷;當知、當見諸色是無常的時,邪見被捨斷;當知、當見眼識是無常的時,邪見被捨斷;當知、當見眼觸是無常的時,邪見被捨斷……(中略)又凡以這意觸為緣生起感受的樂,或苦,或不苦不樂,當知、當見那也是無常的時,邪見被捨斷,比丘!這樣知者、這樣見者的邪見被捨斷。」



\sutta{166}{166}{有身見的捨斷經}{https://agama.buddhason.org/SN/sn.php?keyword=35.166}
  那時,\twnr{某位比丘}{39.0}……(中略)說這個:  

  「\twnr{大德}{45.0}!當怎樣知、當怎樣見時,\twnr{有身見}{93.1}被捨斷呢?」

  「比丘!當知、當見眼是苦的時,有身見被捨斷;當知、當見諸色是苦的時,有身見被捨斷;當知、當見眼識是苦的時,有身見被捨斷;當知、當見眼觸是苦的時,有身見被捨斷……(中略)又凡以這意觸為緣生起感受的樂,或苦,或不苦不樂,當知、當見那也是苦的時,有身見被捨斷,比丘!這樣知者、這樣見者有身見被捨斷。」



\sutta{167}{167}{我隨見的捨斷經}{https://agama.buddhason.org/SN/sn.php?keyword=35.167}
  那時,\twnr{某位比丘}{39.0}……(中略)說這個:

  「\twnr{大德}{45.0}!當怎樣知、當怎樣見時,\twnr{我隨見}{x516}被捨斷呢?」

  「比丘!當知、當見眼是無我時,我隨見被捨斷;當知、當見色為無我者來說我隨見被捨斷;當知、當見眼識為無我者來說我隨見被捨斷;當知、當見眼觸為無我者來說我隨見被捨斷;又凡以這眼觸\twnr{為緣}{180.0}生起感受的樂,或苦,或不苦不樂,當知、當見那也是無我時,我隨見被捨斷……(中略)當知、當見舌為無我者來說我隨見被捨斷……(中略)當知、當見意為無我者來說我隨見被捨斷;諸法……意識……意觸……又凡以這意觸為緣生起感受的樂,或苦,或不苦不樂,當知、當見那也是無我時,我隨見被捨斷。」

  歡喜的滅盡品第十六,其\twnr{攝頌}{35.0}:

  「以歡喜的滅盡四則,耆婆的芒果園二則,

   以拘絺羅三說,邪、有身、我。」





\pin{六十中略品}{168}{227}
\sutta{168}{168}{自身內-無常-意欲經}{https://agama.buddhason.org/SN/sn.php?keyword=35.168}
  「\twnr{比丘}{31.0}們!凡是無常的,在那裡意欲應該被你們捨斷。比丘們!而什麼是無常的?比丘們!眼是無常的,在那裡意欲應該被你們捨斷……(中略)舌是無常的,在那裡意欲應該被你們捨斷……(中略)意是無常的,在那裡意欲應該被你們捨斷,比丘們!凡是無常的,在那裡意欲應該被你們捨斷。」



\sutta{169}{169}{自身內-無常-貪經}{https://agama.buddhason.org/SN/sn.php?keyword=35.169}
  「\twnr{比丘}{31.0}們!凡是無常的,在那裡貪應該被你們捨斷。比丘們!而什麼是無常的?比丘們!眼是無常的,在那裡貪應該被你們捨斷……(中略)舌是無常的,在那裡貪應該被你們捨斷……(中略)意是無常的,在那裡貪應該被你們捨斷,比丘們!凡是無常的,在那裡貪應該被你們捨斷。」



\sutta{170}{170}{自身內-無常-意欲貪經}{https://agama.buddhason.org/SN/sn.php?keyword=35.170}
  「\twnr{比丘}{31.0}們!凡是無常的,在那裡意欲貪應該被你們捨斷。比丘們!而什麼是無常的?比丘們!眼是無常的,在那裡意欲貪應該被你們捨斷……(中略)舌是無常的,在那裡意欲貪應該被你們捨斷……(中略)意是無常的,在那裡意欲貪應該被你們捨斷,比丘們!凡是無常的,在那裡意欲貪應該被你們捨斷。」



\sutta{171}{173}{苦-意欲等經}{https://agama.buddhason.org/SN/sn.php?keyword=35.171}
  「\twnr{比丘}{31.0}們!凡是苦的,在那裡意欲應該被你們捨斷、貪應該被捨斷、意欲貪應該被捨斷。比丘們!而什麼是苦的?比丘們!眼是苦的,在那裡意欲應該被你們捨斷、貪應該被捨斷、意欲貪應該被捨斷……(中略)舌是苦的,在那裡意欲應該被你捨斷、貪……(中略)意是苦的,在那裡意欲應該被你們捨斷、貪應該被捨斷、意欲貪應該被捨斷。比丘們!凡是苦的,在那裡意欲應該被你們捨斷、貪應該被捨斷、意欲貪應該被捨斷。」



\sutta{174}{176}{無我-意欲等經}{https://agama.buddhason.org/SN/sn.php?keyword=35.174}
  「\twnr{比丘}{31.0}們!凡是無我,在那裡意欲應該被你們捨斷、貪應該被捨斷、意欲貪應該被捨斷。比丘們!而什麼是無我?比丘們!眼是無我,在那裡意欲應該被你們捨斷、貪應該被捨斷、意欲貪應該被捨斷……(中略)舌是無我,在那裡意欲應該被你捨斷、貪……(中略)意是無我,在那裡意欲應該被你們捨斷、貪應該被捨斷、意欲貪應該被捨斷。比丘們!凡是無我,在那裡意欲應該被你們捨斷、貪應該被捨斷、意欲貪應該被捨斷。」



\sutta{177}{179}{外部-無常-意欲等經}{https://agama.buddhason.org/SN/sn.php?keyword=35.177}
  「\twnr{比丘}{31.0}們!凡是無常的,在那裡意欲應該被你們捨斷、貪應該被捨斷、意欲貪應該被捨斷。比丘們!而什麼是無常的?比丘們!諸色是無常的,在那裡意欲應該被你們捨斷、貪應該被捨斷、意欲貪應該被捨斷;諸聲音是無常的,在那裡意欲應該被你們捨斷、貪應該被捨斷、意欲貪應該被捨斷;諸氣味是無常的,在那裡意欲應該被你們捨斷、貪應該被捨斷、意欲貪應該被捨斷;諸味道是無常的,在那裡意欲應該被你們捨斷、貪應該被捨斷、意欲貪應該被捨斷;諸\twnr{所觸}{220.2}是無常的,在那裡意欲應該被你們捨斷、貪應該被捨斷、意欲貪應該被捨斷;諸法是無常的,在那裡意欲應該被你們捨斷、貪應該被捨斷、意欲貪應該被捨斷。比丘們!凡是無常的,在那裡意欲應該被你們捨斷、貪應該被捨斷、意欲貪應該被捨斷。」



\sutta{180}{182}{外部-苦-意欲等經}{https://agama.buddhason.org/SN/sn.php?keyword=35.180}
  「\twnr{比丘}{31.0}們!凡是苦的,在那裡意欲應該被你們捨斷、貪應該被捨斷、意欲貪應該被捨斷。比丘們!而什麼是苦的?比丘們!諸色是苦的,在那裡意欲應該被你們捨斷、貪應該被捨斷、意欲貪應該被捨斷;諸聲音……諸氣味……諸味道……諸\twnr{所觸}{220.2}……諸法是苦的,在那裡意欲應該被你們捨斷、貪應該被捨斷、意欲貪應該被捨斷。比丘們!凡是苦的,在那裡意欲應該被你們捨斷、貪應該被捨斷、意欲貪應該被捨斷。」



\sutta{183}{185}{外部-無我-意欲等經}{https://agama.buddhason.org/SN/sn.php?keyword=35.183}
  「\twnr{比丘}{31.0}們!凡是無我,在那裡意欲應該被你們捨斷、貪應該被捨斷、意欲貪應該被捨斷。比丘們!而什麼是無我?比丘們!諸色是無我,在那裡意欲應該被你們捨斷、貪應該被捨斷、意欲貪應該被捨斷;諸聲音……諸氣味……諸味道……諸\twnr{所觸}{220.2}……諸法是無我,在那裡意欲應該被你們捨斷、貪應該被捨斷、意欲貪應該被捨斷。比丘們!凡是無我,在那裡意欲應該被你們捨斷、貪應該被捨斷、意欲貪應該被捨斷。」



\sutta{186}{186}{過去自身內無常經}{https://agama.buddhason.org/SN/sn.php?keyword=35.186}
  「\twnr{比丘}{31.0}們!過去眼是無常的……(中略)過去舌是無常的……(中略)過去意是無常的,比丘們!這麼看的\twnr{有聽聞的聖弟子}{24.0}在眼上\twnr{厭}{15.0}……(中略)也在舌上厭……(中略)也在意上厭。厭者\twnr{離染}{558.0},從\twnr{離貪}{77.0}被解脫,在已解脫時,\twnr{有『[這是]解脫』之智}{27.0},他知道:『\twnr{出生已盡}{18.0},\twnr{梵行已完成}{19.0},\twnr{應該被作的已作}{20.0},\twnr{不再有此處[輪迴]的狀態}{21.1}。』」



\sutta{187}{187}{未來自身內無常經}{https://agama.buddhason.org/SN/sn.php?keyword=35.187}
  「\twnr{比丘}{31.0}們!未來眼是無常的……(中略)未來舌是無常的……(中略)未來意是無常的……這麼看的……(中略)他知道:『……\twnr{不再有此處[輪迴]的狀態}{21.1}。』」



\sutta{188}{188}{現在自身內無常經}{https://agama.buddhason.org/SN/sn.php?keyword=35.188}
  「\twnr{比丘}{31.0}們!現在眼是無常的……(中略)現在舌是無常的……(中略)現在意是無常的……這麼看的……(中略)他知道:『……\twnr{不再有此處[輪迴]的狀態}{21.1}。』」



\sutta{189}{191}{過去等自身內苦經}{https://agama.buddhason.org/SN/sn.php?keyword=35.189}
  「\twnr{比丘}{31.0}們!過去、未來、現在眼是苦的……(中略)過去、未來、現在舌是苦的……(中略)過去、未來、現在意是苦的,比丘們!這麼看的……(中略)他知道:『……\twnr{不再有此處[輪迴]的狀態}{21.1}。』」



\sutta{192}{194}{過去等自身內無我經}{https://agama.buddhason.org/SN/sn.php?keyword=35.192}
  「\twnr{比丘}{31.0}們!過去、未來、現在眼是無我……(中略)舌是無我……(中略)過去、未來、現在意是無我……這麼看的……(中略)他知道:『……\twnr{不再有此處[輪迴]的狀態}{21.1}。』」



\sutta{195}{197}{過去等外部無常經}{https://agama.buddhason.org/SN/sn.php?keyword=35.195}
  「\twnr{比丘}{31.0}們!過去、未來、現在諸色是無常的,諸聲音……諸氣味……諸味道……諸\twnr{所觸}{220.2}……過去、未來、現在諸法是無常的……這麼看的……(中略)他知道:『……\twnr{不再有此處[輪迴]的狀態}{21.1}。』」 



\sutta{198}{200}{過去等外部苦經}{https://agama.buddhason.org/SN/sn.php?keyword=35.198}
  「\twnr{比丘}{31.0}們!過去、未來、現在諸色是苦的,諸聲音……諸氣味……諸味道……諸\twnr{所觸}{220.2}……過去、未來、現在諸法是苦的……這麼看的……(中略)他知道:『……\twnr{不再有此處[輪迴]的狀態}{21.1}。』」 



\sutta{201}{203}{過去等外部無我經}{https://agama.buddhason.org/SN/sn.php?keyword=35.201}
  「\twnr{比丘}{31.0}們!過去、未來、現在諸色是無我,諸聲音……諸氣味……諸味道……諸\twnr{所觸}{220.2}……過去、未來、現在諸法是無我……這麼看的……(中略)他知道:『……\twnr{不再有此處[輪迴]的狀態}{21.1}。』」 



\sutta{204}{204}{過去自身內凡無常經}{https://agama.buddhason.org/SN/sn.php?keyword=35.204}
  「\twnr{比丘}{31.0}們!過去眼是無常的,凡是無常的,那個是苦的,凡是苦的,那個是無我,凡是無我,那個:『\twnr{這不是我的}{32.1},\twnr{我不是這個}{33.1},\twnr{這不是我的真我}{34.2}。』這樣,這個應該以正確之慧如實被看見。

  ……(中略)過去舌是無常的,凡是無常的,那個是苦的,凡是苦的,那個是無我,凡是無我,那個:『這不是我的,我不是這個,這不是我的真我。』這樣,這個應該以正確之慧如實被看見。……(中略)過去意是無常的,凡是無常的,那個是苦的,凡是苦的,那個是無我,凡是無我,那個:『這不是我的,我不是這個,這不是我的真我。』這樣,這個應該以正確之慧如實被看見。這麼看的……(中略)他知道:『……\twnr{不再有此處[輪迴]的狀態}{21.1}。』」



\sutta{205}{205}{未來自身內凡無常經}{https://agama.buddhason.org/SN/sn.php?keyword=35.205}
  「\twnr{比丘}{31.0}們!未來眼是無常的,凡是無常的,那個是苦的,凡是苦的,那個是無我,凡是無我,那個:『\twnr{這不是我的}{32.1},\twnr{我不是這個}{33.1},\twnr{這不是我的真我}{34.2}。』這樣,這個應該以正確之慧如實被看見。

  ……(中略)未來舌是無常的,凡是無常的,那個是苦的,凡是苦的,那個是無我,凡是無我,那個:『這不是我的,我不是這個,這不是我的真我。』這樣,這個應該以正確之慧如實被看見。……(中略)未來意是無常的,凡是無常的,那個是苦的,凡是苦的,那個是無我,凡是無我,那個:『這不是我的,我不是這個,這不是我的真我。』這樣,這個應該以正確之慧如實被看見。比丘們!這麼看的……(中略)他知道:『……\twnr{不再有此處[輪迴]的狀態}{21.1}。』」



\sutta{206}{206}{現在自身內凡無常經}{https://agama.buddhason.org/SN/sn.php?keyword=35.206}
  「\twnr{比丘}{31.0}們!現在眼是無常的,凡是無常的,那個是苦的,凡是苦的,那個是無我,凡是無我,那個:『\twnr{這不是我的}{32.1},\twnr{我不是這個}{33.1},\twnr{這不是我的真我}{34.2}。』這樣,這個應該以正確之慧如實被看見。

  ……(中略)現在舌是無常的,凡是無常的,那個是苦的,凡是苦的,那個是無我,凡是無我,那個:『這不是我的,我不是這個,這不是我的真我。』這樣,這個應該以正確之慧如實被看見。……(中略)現在意是無常的,凡是無常的,那個是苦的,凡是苦的,那個是無我,凡是無我,那個:『這不是我的,我不是這個,這不是我的真我。』這樣,這個應該以正確之慧如實被看見。這麼看的……(中略)他知道:『……\twnr{不再有此處[輪迴]的狀態}{21.1}。』」



\sutta{207}{209}{過去等自身內凡苦者經}{https://agama.buddhason.org/SN/sn.php?keyword=35.207}
  「\twnr{比丘}{31.0}們!過去、未來、現在眼是苦的,凡是苦的,那是無我,凡是無我,那個:『\twnr{這不是我的}{32.1},\twnr{我不是這個}{33.1},\twnr{這不是我的真我}{34.2}。』這樣,這個應該以正確之慧如實被看見。

  ……(中略)舌是苦的……(中略)過去、未來、現在意是苦的,凡是苦的,那是無我,凡是無我,那個:『這不是我的,我不是這個,這不是我的真我。』這樣,這個應該以正確之慧如實被看見。

  比丘們!這麼看的……(中略)他知道:『……\twnr{不再有此處[輪迴]的狀態}{21.1}。』」



\sutta{210}{212}{過去等自身內凡無我者經}{https://agama.buddhason.org/SN/sn.php?keyword=35.210}
  「\twnr{比丘}{31.0}們!過去、未來、現在眼是無我,凡是無我,那個:『\twnr{這不是我的}{32.1},\twnr{我不是這個}{33.1},\twnr{這不是我的真我}{34.2}。』這樣,這個應該以正確之慧如實被看見。

  ……(中略)舌是無我……(中略)過去、未來、現在意是無我,凡是無我,那個:『這不是我的,我不是這個,這不是我的真我。』這樣,這個應該以正確之慧如實被看見。這麼看的……(中略)他知道:『……\twnr{不再有此處[輪迴]的狀態}{21.1}。』」



\sutta{213}{215}{過去等外部凡無常經}{https://agama.buddhason.org/SN/sn.php?keyword=35.213}
  「\twnr{比丘}{31.0}們!過去、未來、現在諸色是無常的,凡是無常的,那個是苦的,凡是苦的,那個是無我,凡是無我,那個:『\twnr{這不是我的}{32.1},\twnr{我不是這個}{33.1},\twnr{這不是我的真我}{34.2}。』這樣,這個應該以正確之慧如實被看見。

  諸聲音……諸氣味……諸味道……諸\twnr{所觸}{220.2}……過去、未來、現在諸法是無常的,凡是無常的,那個是苦的,凡是苦的,那個是無我,凡是無我,那個:『這不是我的,我不是這個,這不是我的真我。』這樣,這個應該以正確之慧如實被看見。這麼看的……(中略)他知道:『……\twnr{不再有此處[輪迴]的狀態}{21.1}。』」



\sutta{216}{218}{過去等外部凡苦者經}{https://agama.buddhason.org/SN/sn.php?keyword=35.216}
  「\twnr{比丘}{31.0}們!過去、未來、現在諸色是苦的,凡是苦的,那是無我,凡是無我,那個:『\twnr{這不是我的}{32.1},\twnr{我不是這個}{33.1},\twnr{這不是我的真我}{34.2}。』這樣,這個應該以正確之慧如實被看見。

  諸聲音……諸氣味……諸味道……諸\twnr{所觸}{220.2}……過去、未來、現在諸法是苦的,凡是苦的,那是無我,凡是無我,那個:『這不是我的,我不是這個,這不是我的真我。』這樣,這個應該以正確之慧如實被看見。這麼看的……(中略)他知道:『……\twnr{不再有此處[輪迴]的狀態}{21.1}。』」



\sutta{219}{221}{過去等外部凡無我者經}{https://agama.buddhason.org/SN/sn.php?keyword=35.219}
  「\twnr{比丘}{31.0}們!過去、未來、現在諸色是無我,凡是無我,那個:『\twnr{這不是我的}{32.1},\twnr{我不是這個}{33.1},\twnr{這不是我的真我}{34.2}。』這樣,這個應該以正確之慧如實被看見。

  諸聲音……諸氣味……諸味道……諸\twnr{所觸}{220.2}……過去、未來、現在諸法是無我,凡是無我,那個:『這不是我的,我不是這個,這不是我的真我。』這樣,這個應該以正確之慧如實被看見。這麼看的……(中略)他知道:『……\twnr{不再有此處[輪迴]的狀態}{21.1}。』」



\sutta{222}{222}{內處-無常經}{https://agama.buddhason.org/SN/sn.php?keyword=35.222}
  「\twnr{比丘}{31.0}們!眼是無常的……(中略)舌是無常的……(中略)意是無常的。這麼看的……(中略)他知道:『……\twnr{不再有此處[輪迴]的狀態}{21.1}。』」 



\sutta{223}{223}{內處-苦經}{https://agama.buddhason.org/SN/sn.php?keyword=35.223}
  「\twnr{比丘}{31.0}們!眼是苦的……(中略)舌是苦的……(中略)意是苦的。這麼看的……(中略)他知道:『……\twnr{不再有此處[輪迴]的狀態}{21.1}。』」 



\sutta{224}{224}{內處-無我經}{https://agama.buddhason.org/SN/sn.php?keyword=35.224}
  「\twnr{比丘}{31.0}們!眼是無我……(中略)舌是無我……(中略)意是無我。這麼看的……(中略)他知道:『……\twnr{不再有此處[輪迴]的狀態}{21.1}。』」 



\sutta{225}{225}{外處-無常經}{https://agama.buddhason.org/SN/sn.php?keyword=35.225}
  「\twnr{比丘}{31.0}們!諸色是無常的;諸聲音……諸氣味……諸味道……諸\twnr{所觸}{220.2}……諸法是無常的。這麼看的……(中略)他知道:『……\twnr{不再有此處[輪迴]的狀態}{21.1}。』」



\sutta{226}{226}{外處-苦經}{https://agama.buddhason.org/SN/sn.php?keyword=35.226}
  「\twnr{比丘}{31.0}們!諸色是苦的;諸聲音……諸氣味……諸味道……諸\twnr{所觸}{220.2}……諸法是苦的。這麼看的……(中略)他知道:『……\twnr{不再有此處[輪迴]的狀態}{21.1}。』」



\sutta{227}{227}{外處-無我經}{https://agama.buddhason.org/SN/sn.php?keyword=35.227}
  「\twnr{比丘}{31.0}們!諸色是無我;諸聲音……諸氣味……諸味道……諸\twnr{所觸}{220.2}……諸法是無我。這麼看的……(中略)他知道:『……\twnr{不再有此處[輪迴]的狀態}{21.1}。』」

  六十中略品第十七,其\twnr{攝頌}{35.0}:

  「以欲十八則,以過去二者九則,

   凡無常十八說,自身內外三則,

   中略六十則,被太陽族人的佛陀說。」

   六十經[終了]。





\pin{海品}{228}{237}
\sutta{228}{228}{海經第一}{https://agama.buddhason.org/SN/sn.php?keyword=35.228}
  「\twnr{比丘}{31.0}們!\twnr{未聽聞的一般人}{74.0}說『海,海』,比丘們!這不是聖者之律中的海,比丘們!這是大水之積聚,大水之海洋。

  比丘們!眼是人的海,色所成的是它的推動力(急流),凡征服那個色所成的推動力者,比丘們!這被稱為渡過有波浪、有\twnr{漩渦}{993.0}、有水鬼、有\twnr{羅剎}{122.0}的眼之海,已渡、到\twnr{彼岸}{226.0}的婆羅門站在高地上……(中略)比丘們!舌是人的海,味道所成的是它的推動力,凡征服那個味道所成的推動力者,比丘們!這被稱為渡過有波浪、有漩渦、有水鬼、有羅剎的舌之海,已渡、到彼岸的婆羅門站在高地上。……(中略)比丘們!意是人的海,法所成的是它的推動力,凡征服那個法所成的推動力者,比丘們!這被稱為渡過有波浪、有漩渦、有水鬼、有羅剎的意之海,已渡、到彼岸的婆羅門站在高地上。」……說這個……大師……(中略):

  「凡渡過這個有水鬼、有羅剎,有波浪、有漩渦、有恐怖、難渡的海者,

   他是通曉吠陀者、已住於梵行者:到達世界邊者被稱為『到彼岸者』。」



\sutta{229}{229}{海經第二}{https://agama.buddhason.org/SN/sn.php?keyword=35.229}
  「\twnr{比丘}{31.0}們!\twnr{未聽聞的一般人}{74.0}說『海,海』,比丘們!這不是聖者之律中的海,比丘們!這是大水之積聚,大水之海洋。

  比丘們!有能被眼識知的、想要的、所愛的、合意的、可愛形色的、伴隨欲的、誘人的諸色,比丘們!這被稱為聖者之律中的海,在這裡,這包括天、魔、梵的世間;包括沙門婆羅門,包括天-人的\twnr{世代}{38.0},大多數是已沈沒的,\twnr{變成糾纏線軸的}{802.0}、變成打結線球的,\twnr{成為蘆草與燈心草團的}{803.0},不超越\twnr{苦界}{109.0}、\twnr{惡趣}{110.0}、\twnr{下界}{111.0}、輪迴……(中略)比丘們!有能被舌識知……的諸味道……(中略)比丘們!有能被意識知的、想要的、所愛的、合意的、可愛形色的、伴隨欲的、誘人的諸法,比丘們!這被稱為聖者之律中的海,在這裡,這包括天、魔、梵的世間;包括沙門婆羅門,包括天-人的世代,大多數是已沈沒的,變成糾纏線軸的、變成打結線球的,成為蘆草與燈心草團的,不超越苦界、惡趣、下界、輪迴。」

  「凡使其貪與瞋,以及\twnr{無明}{207.0}脫離者,

   他渡過這個有水鬼、\twnr{羅剎}{122.0},有波浪恐怖、難渡的海。

   超越染著者、死神之捨棄者、無\twnr{依著}{198.0}者,為了不再有(再生)捨斷苦,

   \twnr{滅沒的他不再回來}{x517},我說:『他使死神之王迷惑。』」



\sutta{230}{230}{像漁夫那樣經}{https://agama.buddhason.org/SN/sn.php?keyword=35.230}
  「\twnr{比丘}{31.0}們!猶如漁夫在深湖中投入有餌的釣鉤,某隻覓食的魚吞下它,比丘們!這樣,那是吞下漁夫的釣鉤、來到不幸、來到災厄、被漁夫為所欲為的魚。[\suttaref{SN.17.2}]

  同樣的,比丘們!在世間中,為了眾生的不幸,為了生物類的殺害,有這六種釣鉤。哪六種?比丘們!有能被眼識知的、想要的、所愛的、合意的、可愛形色的、伴隨欲的、誘人的諸色,如果比丘歡喜、歡迎、持續固持那個,比丘們!這位被稱為上魔釣鉤、來到不幸、來到災厄、被\twnr{波旬}{49.0}為所欲為的比丘……(中略)比丘們!有能被舌識知……的諸味道……(中略)比丘們!有能被意識知的、想要的、所愛的、合意的、可愛形色的、伴隨欲的、誘人的諸法,如果比丘歡喜、歡迎、持續固持那個,比丘們!這位被稱為上魔釣鉤、來到不幸、來到災厄、被波旬為所欲為的比丘。

  比丘們!而有能被眼識知的、想要的、所愛的、合意的、可愛形色的、伴隨欲的、誘人的諸色,如果比丘不歡喜、不歡迎、不持續固持那個,比丘們!這位被稱為不上魔釣鉤、破壞釣鉤、粉碎釣鉤、不來到不幸、不來到災厄、不被波旬為所欲為的比丘……(中略)比丘們!有能被舌識知……的諸味道……(中略)比丘們!有能被意識知的、想要的、所愛的、合意的、可愛形色的、伴隨欲的、誘人的諸法,如果比丘不歡喜、不歡迎、不持續固持那個,比丘們!這位被稱為不上魔釣鉤、破壞釣鉤、粉碎釣鉤、不來到不幸、不來到災厄、不被波旬為所欲為的比丘。」



\sutta{231}{231}{如乳樹經}{https://agama.buddhason.org/SN/sn.php?keyword=35.231}
  「\twnr{比丘}{31.0}們!凡對任何比丘或比丘尼,在能被眼識知的諸色上凡他有貪,凡他有瞋,凡他有癡,凡他是貪未捨斷者,凡他是瞋未捨斷者,凡他是癡未捨斷者,對他,即使能被眼識知的微少諸色來到眼的範圍,仍佔據他的心,更不用說\twnr{強烈的}{x518},那是什麼原因?比丘們!凡他有貪,凡他有瞋,凡他有癡,凡他是貪未捨斷者,凡他是瞋未捨斷者,凡他是癡未捨斷者……(中略)。

  比丘們!凡對任何比丘或比丘尼,在能被舌識知的諸味道上凡他有貪……(中略)。

  比丘們!凡對任何比丘或比丘尼,在能被意識知的諸法上凡他有貪,凡他有瞋,凡他有癡,凡他是貪未捨斷者,凡他是瞋未捨斷者,凡他是癡未捨斷者,對他,即使能被意識知的微少諸法來到意的範圍,仍佔據他的心,更不用說強烈的,那是什麼原因?比丘們!凡他有貪,凡他有瞋,凡他有癡,凡他是貪未捨斷者,凡他是瞋未捨斷者,凡他是癡未捨斷者。

  比丘們!猶如幼小、幼嫩、年輕的乳樹:菩提樹,或榕樹,或無花果樹,或蘋果樹,男子以銳利的斧頭到處切它,[樹]乳流出?」「是的,\twnr{大德}{45.0}!那是什麼原因?大德!因為它有[樹]乳。」

  「同樣的,比丘們!凡對任何比丘或比丘尼,在能被眼識知的諸色上凡他有貪,凡他有瞋,凡他有癡,凡他是貪未捨斷者,凡他是瞋未捨斷者,凡他是癡未捨斷者,對他,即使能被眼識知的微少諸色來到眼的範圍,仍佔據他的心,更不用說強烈的,那是什麼原因?比丘們!凡他有貪,凡他有瞋,凡他有癡,凡他是貪未捨斷者,凡他是瞋未捨斷者,凡他是癡未捨斷者……(中略)。

  比丘們!凡對任何比丘或比丘尼,在能被舌識知的諸味道上凡他有貪……(中略)。

  比丘們!凡對任何比丘或比丘尼,在能被意識知的諸法上凡他有貪,凡他有瞋,凡他有癡,凡他是貪未捨斷者,凡他是瞋未捨斷者,凡他是癡未捨斷者,對他,即使能被意識知的微少諸法來到意的範圍,仍佔據他的心,更不用說強烈的,那是什麼原因?比丘們!凡他有貪,凡他有瞋,凡他有癡,凡他是貪未捨斷者,凡他是瞋未捨斷者,凡他是癡未捨斷者。

  比丘們!凡對任何比丘或比丘尼,在能被眼識知的諸色上凡他沒有貪,凡他沒有瞋,凡他沒有癡,凡他是貪已捨斷者,凡他是瞋已捨斷者,凡他是癡已捨斷者,對他,即使能被眼識知的強烈諸色來到眼的範圍,仍不佔據他的心,更不用說微少的,那是什麼原因?比丘們!凡他沒有貪,凡他沒有瞋,凡他沒有癡,凡他是貪已捨斷者,凡他是瞋已捨斷者,凡他是癡已捨斷者……(中略)。

  比丘們!凡對任何比丘或比丘尼,在能被舌識知的諸味道上……(中略)在能被意識知的諸法上凡他沒有貪,凡他沒有瞋,凡他沒有癡,凡他是貪已捨斷者,凡他是瞋已捨斷者,凡他是癡已捨斷者,對他,即使能被意識知的強烈諸法來到意的範圍,仍不佔據他的心,更不用說微少的,那是什麼原因?比丘們!凡他沒有貪,凡他沒有瞋,凡他沒有癡,凡他是貪已捨斷者,凡他是瞋已捨斷者,凡他是癡已捨斷者。比丘們!猶如乾枯、枯死、超過一年的乳樹:菩提樹,或榕樹,或無花果樹,或蘋果樹,男子以銳利的斧頭到處切它,[樹]乳流出?」「大德!這確實不是,那是什麼原因?大德!因為它沒有[樹]乳。」

  「同樣的,比丘們!凡對任何比丘或比丘尼,在能被眼識知的諸色上凡他沒有貪,凡他沒有瞋,凡他沒有癡,凡他是貪已捨斷者,凡他是瞋已捨斷者,凡他是癡已捨斷者,對他,即使能被眼識知的強烈諸色來到眼的範圍,仍不佔據他的心,更不用說微少的,那是什麼原因?比丘們!凡他沒有貪,凡他沒有瞋,凡他沒有癡,凡他是貪已捨斷者,凡他是瞋已捨斷者,凡他是癡已捨斷者……(中略)。

  比丘們!凡對任何比丘或比丘尼,在能被舌識知的諸味道上……(中略)。

  比丘們!凡對任何比丘或比丘尼,在能被意識知的諸法上凡他沒有貪,凡他沒有瞋,凡他沒有癡,凡他是貪已捨斷者,凡他是瞋已捨斷者,凡他是癡已捨斷者對他,對他,即使能被意識知的強烈諸法來到意的範圍,仍不佔據他的心,更不用說微少的,那是什麼原因?比丘們!凡他沒有貪,凡他沒有瞋,凡他沒有癡,凡他是貪已捨斷者,凡他是瞋已捨斷者,凡他是癡已捨斷者。」



\sutta{232}{232}{拘絺羅經}{https://agama.buddhason.org/SN/sn.php?keyword=35.232}
  \twnr{有一次}{2.0},\twnr{尊者}{200.0}舍利弗與尊者摩訶拘絺羅住在波羅奈仙人墜落處的鹿林。

  那時,尊者摩訶拘絺羅傍晚時,從\twnr{獨坐}{92.0}出來,去見尊者舍利弗。抵達後,與尊者舍利弗一起互相問候。交換應該被互相問候的友好交談後,在一旁坐下。在一旁坐下的尊者摩訶拘絺羅對尊者舍利弗說這個:

  「舍利弗\twnr{學友}{201.0}!怎麼樣,眼是諸色的結縛?諸色是眼的結縛嗎?……(中略)舌是諸味道的結縛?諸味道是舌的結縛嗎?……(中略)意是諸法的結縛?諸法是意的結縛嗎?」

  「拘絺羅學友!眼不是諸色的結縛;諸色也不是眼的結縛,但凡\twnr{緣於}{252.0}那兩者意欲貪生起處,在那裡那是結縛。……(中略)舌不是諸味道的結縛……(中略)意不是諸法的結縛;諸法也不是意的結縛,但凡緣於那兩者意欲貪生起處,在那裡那是結縛。

  學友!猶如黑牛與白牛,被一條繩子或繫繩連結(綁在一起),凡這麼說:『黑牛是白牛的結縛;白牛是黑牛的結縛。』那位說者正確地說嗎?」

  「學友!這確實不是。」

  「學友!黑牛不是白牛的結縛;白牛不是黑牛的結縛,凡牠們被一條繩子或繫繩連結,在那裡那是結縛。同樣的,學友!眼不是諸色的結縛;諸色也不是眼的結縛,但凡緣於那兩者意欲貪生起處,在那裡那是結縛。……(中略)舌不是諸味道的結縛;諸味道也不是舌的結縛,但凡緣於那兩者意欲貪生起處,在那裡那是結縛。……(中略)意不是諸法的結縛;諸法也不是意的結縛,但凡緣於那兩者意欲貪生起處,在那裡那是結縛。

  學友!如果眼是諸色的結縛,或諸色是眼的結縛,這為了苦的完全滅盡之梵行生活不被知道。學友!但因為眼不是諸色的結縛,諸色不是眼的結縛,但凡緣於那兩者意欲貪生起處,在那裡那是結縛,因此,為了苦的完全滅盡之梵行生活被了知。……(中略)學友!如果舌是諸味道的結縛,或諸味道是舌的結縛,這為了苦的完全滅盡之梵行生活不被知道。學友!但因為舌不是諸味道的結縛,諸味道不是舌的結縛,但凡緣於那兩者意欲貪生起處,在那裡那是結縛,因此,為了苦的完全滅盡之梵行生活被了知。……(中略)學友!如果意是諸法的結縛,或諸法是意的結縛,這為了苦的完全滅盡之梵行生活不被知道。學友!但因為意不是諸法的結縛,諸法不是意的結縛,但凡緣於那兩者意欲貪生起處,在那裡那是結縛,因此,為了苦的完全滅盡之梵行生活被了知。

  學友!以這個法門,這也能被知道,如『眼不是諸色的結縛;諸色不是眼的結縛,但凡緣於那兩者意欲貪生起處,在那裡那是結縛。……(中略)舌不是諸味道的結縛……(中略)意不是諸法的結縛;諸法不是意的結縛,但凡緣於那兩者意欲貪生起處,在那裡那是結縛。』學友!\twnr{世尊}{12.0}的眼被看見,世尊以眼見色,世尊的意欲貪不存在,世尊是心\twnr{善解脫}{28.0}者。學友!世尊的耳被看見,世尊以耳聽聲音,世尊的意欲貪不存在,世尊是心善解脫者。學友!世尊的鼻被看見,世尊以鼻聞氣味,世尊的意欲貪不存在,世尊是心善解脫者。學友!世尊的舌被看見,世尊以舌嚐味道,世尊的意欲貪不存在,世尊是心善解脫者。學友!世尊的身被看見,世尊以身觸\twnr{所觸}{220.2},世尊的意欲貪不存在,世尊是心善解脫者。學友!世尊的意被看見,世尊以意識知法,世尊的意欲貪不存在,世尊是心善解脫者。學友!以這個法門,這能被知道,如『眼不是諸色的結縛;諸色不是眼的結縛,但凡緣於那兩者意欲貪生起處,在那裡那是結縛。耳不是……鼻不是……舌不是諸味道的結縛;諸味道不是舌的結縛,但凡緣於那兩者意欲貪生起處,在那裡那是結縛。身不是……意不是諸法的結縛;諸法不是意的結縛,但凡緣於那兩者意欲貪生起處,在那裡那是結縛。』」



\sutta{233}{233}{葛瑪部經}{https://agama.buddhason.org/SN/sn.php?keyword=35.233}
  \twnr{有一次}{2.0},\twnr{尊者}{200.0}阿難與尊者葛瑪部住在\twnr{憍賞彌}{140.0}瞿師羅園。

  那時,尊者葛瑪部傍晚時,從\twnr{獨坐}{92.0}出來,去見尊者阿難。抵達後,與尊者阿難一起互相問候。交換應該被互相問候的友好交談後,在一旁坐下。在一旁坐下的尊者葛瑪部對尊者阿難說這個:

  「阿難\twnr{學友}{201.0}!怎麼樣,眼是諸色的結縛?諸色是眼的結縛嗎?……(中略)舌是諸味道的結縛?諸味道是舌的結縛嗎?……(中略)意是諸法的結縛?諸法是意的結縛嗎?」

  「葛瑪部學友!眼不是諸色的結縛;諸色也不是眼的結縛,但凡\twnr{緣於}{252.0}那兩者意欲貪生起處,在那裡那是結縛。……(中略)舌不是諸味道的結縛;諸味道也不是舌的結縛……(中略)意不是諸法的結縛;諸法也不是意的結縛,但凡緣於那兩者意欲貪生起處,在那裡那是結縛。

  學友!猶如黑牛與白牛,被一條繩子或繫繩連結(綁在一起),凡這麼說:『黑牛是白牛的結縛;白牛是黑牛的結縛。』那位說者正確地說嗎?」 

  「學友!這確實不是。」

  「學友!黑牛不是白牛的結縛;白牛不是黑牛的結縛,凡牠們被一條繩子或繫繩連結,在那裡那是結縛。同樣的,學友!眼不是諸色的結縛;諸色也不是眼的結縛……(中略)舌不是……(中略)意不是……(中略)但凡緣於那兩者意欲貪生起處,在那裡那是結縛。」



\sutta{234}{234}{優陀夷經}{https://agama.buddhason.org/SN/sn.php?keyword=35.234}
  \twnr{有一次}{2.0},\twnr{尊者}{200.0}阿難與尊者優陀夷住在\twnr{憍賞彌}{140.0}瞿師羅園。

  那時,尊者優陀夷傍晚時,從\twnr{獨坐}{92.0}出來,去見尊者阿難。抵達後,與尊者阿難一起互相問候。交換應該被互相問候的友好交談後,在一旁坐下。在一旁坐下的尊者優陀夷對尊者阿難說這個:

  「阿難\twnr{學友}{201.0}!就如這個身體被\twnr{世尊}{12.0}以種種法門告知、揭開、使之被知道:『像這樣,這個身體是無我。』也能同樣地告知、教導、安立(使知)、建立、揭開、解析、闡明這個識:『像這樣,這個識也是無我。』嗎?」

  「優陀夷學友!就如這個身體被世尊以種種法門告知、揭開、使之被知道:『像這樣,這個身體是無我。』也能同樣地告知、教導、安立(使知)、建立、揭開、解析、闡明這個識:『像這樣,這個識也是無我。』

  學友!\twnr{緣於}{252.0}眼與諸色眼識生起?」

  「是的,學友!」

  「學友!凡眼識之生起的因與\twnr{緣}{180.0},如果那個因與那個緣全部完全地、每一方面完全地、無剩餘地被滅,是否眼識會被知道?」

  「學友!這確實不是。」

  「學友!以這個法門,這被世尊告知、揭開、使之被知道:『像這樣,這個識也是無我。』」……(中略)。

  「學友!緣於舌與諸味道舌識生起嗎?」

  「是的,學友!」

  「學友!凡舌識之生起的因與緣,如果那個因與那個緣全部完全地、每一方面完全地、無剩餘地被滅,是否舌識會被知道?」

  「學友!這確實不是。」

  「學友!以這個法門,這被世尊告知、揭開、使之被知道:『像這樣,這個識也是無我。』」……(中略)。

  「學友!緣於意與諸法生起意識嗎?」

  「是的,學友!」

  「學友!凡意識之生起的因與緣,如果那個因與那個緣全部完全地、每一方面完全地、無剩餘地被滅,是否意識會被知道?」

  「學友!這確實不是。」

  「學友!以這個法門,這被世尊告知、揭開、使之被知道:『像這樣,這個識也是無我。』

  學友!猶如欲求\twnr{心材}{356.0}、尋求心材、進行心材之遍求的男子拿起銳利的斧頭後,進入樹林,在那裡,他看見筆直、新長的、未抽芽結果實的大芭蕉樹幹,他隨即在根處切斷,在根處切斷後在頂端切斷,在頂端切斷後分開芭蕉葉鞘。在那裡[當分開它的葉鞘時, \suttaref{SN.22.95}],他連\twnr{膚材}{356.1}也沒得到,從哪裡有心材!同樣的,學友!\twnr{比丘}{31.0}在\twnr{六觸處}{78.0}上都認為非我、非我所,當不這麼認為時,不執取世間中任何事物。不執取者不\twnr{戰慄}{436.0},不戰慄者\twnr{就自己證涅槃}{71.0},他知道:『\twnr{出生已盡}{18.0},\twnr{梵行已完成}{19.0},\twnr{應該被作的已作}{20.0},\twnr{不再有此處[輪迴]的狀態}{21.1}。』」



\sutta{235}{235}{燃燒法門經}{https://agama.buddhason.org/SN/sn.php?keyword=35.235}
  「\twnr{比丘}{31.0}們!我將為你們教導燃燒\twnr{法門法的教說}{562.0},\twnr{你們要聽}{43.0}它!

  比丘們!而什麼是燃燒法門法的教說呢?

  比丘們!眼根被赤熱的、燃燒的、灼熱的、熾熱的鐵葉片破壞是比較好的,而不在能被眼識知的諸色上從\twnr{細相}{197.0}執取相。

  比丘們!當識住立時,會住立在\twnr{樂味}{295.0}繫縛的相上,或樂味繫縛的細相上。如果在那時候死了,這存在可能性:凡走入兩個趣處中某個趣處:地獄或畜生界。

  比丘們!看到這些\twnr{過患}{293.0}後,我這麼說。

  比丘們!耳根被尖銳的、燃燒的、灼熱的、熾熱的鐵長釘破壞是比較好的,而不在能被耳識知的諸聲音上從細相執取相。比丘們!當識住立時,會住立在樂味繫縛的相上,或樂味繫縛的細相上。如果在那時候死了,這存在可能性:凡走入兩個趣處中某個趣處:地獄或畜生界。比丘們!看到這些過患後,我這麼說。

  比丘們!鼻根被尖銳的、燃燒的、灼熱的、熾熱的指甲剪破壞是比較好的,而不在能被鼻識知的諸氣味上從細相執取相。比丘們!當識住立時,會住立在樂味繫縛的相上,或樂味繫縛的細相上。如果在那時候死了,這存在可能性:凡走入兩個趣處中某個趣處:地獄或畜生界。比丘們!看到這些過患後,我這麼說。

  比丘們!舌根被尖銳的、燃燒的、灼熱的、熾熱的剃刀破壞是比較好的,而不在能被舌識知的諸味道上從細相執取相。比丘們!當識住立時,會住立在樂味繫縛的相上,或樂味繫縛的細相上。如果在那時候死了,這存在可能性:凡走入兩個趣處中某個趣處:地獄或畜生界。比丘們!看到這些過患後,我這麼說。

  比丘們!身根被尖銳的、燃燒的、灼熱的、熾熱的矛破壞是比較好的,而不在能被身識知的諸\twnr{所觸}{220.2}上從細相執取相。比丘們!當識住立時,會住立在樂味繫縛的相上,或樂味繫縛的細相上。如果在那時候死了,這存在可能性:凡走入兩個趣處中某個趣處:地獄或畜生界。比丘們!看到這些過患後,我這麼說。

  比丘們!睡眠是比較好的,比丘們!又,睡眠我說對生命來說是\twnr{徒然白費的}{334.0},我說對生命來說是無果實的,我說對生命來說是愚鈍的,然而,他不會在像那樣的尋上尋思:會來到像這樣諸尋的控制下破僧。

  比丘們!看見這對生命是徒然白費的過患後,我這麼說。

  在那裡,比丘們!\twnr{有聽聞的聖弟子}{24.0}像這樣深慮:『別理會眼根被赤熱的、燃燒的、灼熱的、熾熱的鐵葉片破壞那種程度,來吧,讓我只這麼作意:「像這樣,眼是無常的,諸色是無常的,眼識是無常的,眼觸是無常的,又凡以這眼觸\twnr{為緣}{180.0}生起感受的樂,或苦,或不苦不樂,那也是無常的。」

  別理會耳根被尖銳的、燃燒的、灼熱的、熾熱的鐵釘破壞那種程度,來吧,讓我只這麼作意:「像這樣,耳是無常的,諸聲音是無常的,耳識是無常的,耳觸是無常的,又凡以這耳觸為緣生起感受的樂,或苦,或不苦不樂,那也是無常的。」

  別理會鼻根被尖銳的、燃燒的、灼熱的、熾熱的指甲剪破壞那種程度,來吧,讓我只這麼作意:「像這樣,鼻是無常的,諸氣味是無常的,鼻識是無常的,鼻觸是無常的,又凡以這鼻觸為緣生起感受的……(中略)那也是無常的。」

  別理會舌根被尖銳的、燃燒的、灼熱的、熾熱的剃刀破壞那種程度,來吧,讓我只這麼作意:「像這樣,舌是無常的,諸味道是無常的,舌識是無常的,舌觸是無常的,又凡以這舌觸為緣生起感受的……(中略)那也是無常的。」

  別理會身根被尖銳的、燃燒的、灼熱的、熾熱的矛破壞那種程度,來吧,讓我只這麼作意:「像這樣,身是無常的,諸所觸是無常的,身識是無常的,身觸是無常的,又凡以這身觸為緣生起感受的……(中略)那也是無常的。」

  別理會睡眠那種程度,來吧,讓我只這麼作意:「像這樣,意是無常的,諸法是無常的,意識是無常的,意觸是無常的,又凡以這意觸為緣生起感受的樂,或苦,或不苦不樂,那也是無常的。」』

  比丘們!這麼看的有聽聞的聖弟子在眼上\twnr{厭}{15.0},也在諸色上厭,也在眼識上厭,也在眼觸上厭……(中略)又凡以意觸為緣生起感受的樂,或苦,或不苦不樂也都厭。厭者\twnr{離染}{558.0},從\twnr{離貪}{77.0}被解脫,在已解脫時,\twnr{有『[這是]解脫』之智}{27.0},他知道:『\twnr{出生已盡}{18.0},\twnr{梵行已完成}{19.0},\twnr{應該被作的已作}{20.0},\twnr{不再有此處[輪迴]的狀態}{21.1}。』

  比丘們!這是燃燒法門法的教說。」



\sutta{236}{236}{如手腳經第一}{https://agama.buddhason.org/SN/sn.php?keyword=35.236}
  「\twnr{比丘}{31.0}們!在有諸手時,拿起放下被知道;在有諸腳時,前進後退被知道;在有諸關節時,彎曲伸展被知道;在有肚子時,饑餓口渴被知道。同樣的,比丘們!在有眼時,以眼觸\twnr{為緣}{180.0}自身內的苦樂生起……(中略)在有舌時,以舌觸為緣自身內的苦樂生起……(中略)在有意時,以意觸為緣自身內的苦樂生起。

  比丘們!在沒有諸手時,拿起放下不被知道;在沒有諸腳時,前進後退不被知道;在沒有諸關節時,彎曲伸展不被知道;在沒有肚子時,饑餓口渴不被知道。同樣的,比丘們!在沒有眼時,以眼觸為緣自身內的苦樂不生起……(中略)在沒有舌時,以舌觸為緣自身內的苦樂不生起……(中略)在沒有意時,以意觸為緣自身內的苦樂不生起。」



\sutta{237}{237}{如手腳經第二}{https://agama.buddhason.org/SN/sn.php?keyword=35.237}
  「\twnr{比丘}{31.0}們!在有諸手時,有拿起放下(拿起放下存在);在有諸腳時,有前進後退;在有諸關節時,有彎曲伸展;在有肚子時,有饑餓口渴。同樣的,比丘們!在有眼時,以眼觸為緣自身內的苦樂生起……(中略)在有舌時,以舌觸為緣自身內的苦樂生起……(中略)在有意時,以意觸為緣自身內的苦樂生起。

  比丘們!在沒有諸手時,沒有拿起放下;在沒有諸腳時,沒有前進後退;在沒有諸關節時,沒有彎曲伸展;在沒有肚子時,沒有饑餓口渴不。同樣的,比丘們!在沒有眼時,以眼觸為緣自身內的苦樂不生起……(中略)在沒有舌時,以舌觸為緣自身內的苦樂不生起……(中略)在沒有意時,以意觸為緣自身內的苦樂不生起。」

  海品第十八,其\twnr{攝頌}{35.0}:

  「二則海、漁夫,乳樹與拘絺羅,

   葛瑪部與優陀夷,以及以燃燒為第八,

   『如手腳二則』,以那個被稱為品。」





\pin{毒蛇品}{238}{248}
\sutta{238}{238}{如毒蛇經}{https://agama.buddhason.org/SN/sn.php?keyword=35.238}
  「\twnr{比丘}{31.0}們!猶如有四條威力兇猛、劇毒的毒蛇,那時,想要活命、不想要死,想要樂、厭逆苦的男子到來,他們這麼告訴他:『喂!男子!這四條威力兇猛、劇毒的毒蛇,應該被你經常地使之上升(抬起),應該經常地使之沐浴,應該經常地餵食,應該經常地使之就寢,喂!男子!而當這四種威力兇猛、劇毒毒蛇的其中之一發怒時,喂!男子!從那個因由你遭受死亡,或死亡程度的苦,喂!男子!請你做凡應該被你做的。』

  比丘們!那時,那位害怕四條威力兇猛、劇毒毒蛇的男子到處逃跑。他們這麼告訴他:『喂!男子!這五位殺害者的敵人從後面緊跟隨:「就在我們看見他之處,就在那裡我們將奪取[他的]生命。」喂!男子!請你做凡應該被你做的。』

  比丘們!那時,那位害怕四條威力兇猛劇毒毒蛇、害怕五位殺害者敵人的男子到處逃跑。他們這麼告訴他:『喂!男子!這第六位已拔劍的殺害闖入者緊追在後:「就在我看見他之處,就在那裡我將使[他的]頭落下。」喂!男子!請你做凡應該被你做的。』

  比丘們!那時,那位害怕四條威力兇猛劇毒毒蛇、害怕五位殺害者敵人、害怕第六位已拔劍的殺害闖入者的男子到處逃跑。他看見空的村落,凡任何住家他進入,他就進入空無的,他就進入空虛的,他就進入空的;凡任何器具他碰觸,他就碰觸空無的,他就碰觸空虛的,他就碰觸空的。他們這麼告訴他:『喂!男子!現在,搶劫村落的盜賊們進入這個空村落,喂!男子!請你做凡應該被你做的。』

  比丘們!那時,那位害怕四條威力兇猛劇毒毒蛇、害怕五位殺害者敵人、害怕第六位已拔劍的殺害闖入者、害怕搶劫村落盜賊們的男子到處逃跑。他看見大水河流,此岸是有疑懼的、有所怖畏的,彼岸是安穩的、無所怖畏的,但沒有為了從此岸到彼岸的船或越過的橋,比丘們!那時,那位男子這麼想:『這大水河流的此岸是有疑懼的、有所怖畏的,彼岸是安穩的、無所怖畏的,但沒有為了從此岸到彼岸的船或越過的橋,讓我收集草、木、枝條、樹葉後,捆綁筏後,依靠那個筏後,以手腳努力著,平安地到彼岸。』

  比丘們!那時,那位男子收集草、木、枝條、樹葉後,捆綁筏後,依靠那個筏後,以手腳努力著,平安地到彼岸。已渡已到達彼岸,婆羅門站在高地上。

  比丘們!為了義理的使知這個譬喻被我作:

  比丘們!在這裡,這就是義理:『四條威力兇猛、劇毒的毒蛇』,這是\twnr{四大}{646.0}的同義語,即:地界、水界、火界、風界。

  比丘們!『五位殺害者的敵人』,這是\twnr{五取蘊}{36.0}的同義語,即:色取蘊、受取蘊、想取蘊、行取蘊、識取蘊。

  比丘們!『第六位已拔劍的殺害闖入者』,這是歡喜貪的同義語。

  比丘們!『空的村落』,這是六內處的同義語,比丘們!又,如果賢智者、有能力者、聰明者從眼審察它,看起來只是空無的,看起來只是空虛的,看起來只是空的……(中略)比丘們!又,如果……從舌……(中略)比丘們!又,如果賢智者、有能力者、聰明者從意審察它,看起來只是空無的,看起來只是空虛的,看起來只是空的。

  比丘們!『搶劫村落的盜賊們』,這是六外處的同義語,比丘們!眼在合意不合意的諸色中\twnr{被打}{x519};比丘們!耳……(中略)比丘們!鼻……(中略)比丘們!舌在合意不合意的諸味道中被打;比丘們!身……(中略)比丘們!意在合意不合意的諸法中被打。

  比丘們!『大水河流』,這是四種暴流的同義語:欲的暴流的、\twnr{有的暴流}{369.0}的、見的暴流的、無明的暴流的。

  比丘們!『有疑懼的、有所怖畏的此岸』,這是有身的同義語。

  比丘們!『安穩的、無所怖畏的彼岸』,這是涅槃的同義語。

  比丘們!『筏』,這是\twnr{八支聖道}{525.0}的同義語,即:正見……(中略)正定。

  比丘們!『以他的手腳的努力』,這是活力之激發的同義語。

  比丘們!『已渡已到達彼岸,婆羅門站在高地上』,這是\twnr{阿羅漢}{5.0}的同義語。」



\sutta{239}{239}{如車子經}{https://agama.buddhason.org/SN/sn.php?keyword=35.239}
  「\twnr{比丘}{31.0}們!具備三法的比丘\twnr{當生}{42.0}住於豐富的樂、喜悅,以及為了諸\twnr{漏}{188.0}的滅盡\twnr{他的起源已開始}{581.0},哪三個?他是\twnr{在諸根上守護門者}{468.0}、在飲食上知適量者、專修清醒者。

  比丘們!而怎樣比丘是在諸根上守護門者?比丘們!這裡,比丘以眼見色後,不成為相的執取者、\twnr{細相}{197.0}的執取者,因那個理由,\twnr{貪婪}{435.0}、憂諸惡不善法會\twnr{流入}{224.0}那位住於眼根不防護者。他走向為了那個的\twnr{自制}{217.0},保護眼根,在眼根上來到自制;以耳聽聲音後……以鼻聞氣味後……以舌嚐味道後……以身觸\twnr{所觸}{220.2}後……以意識知法後,不成為相的執取者、細相的執取者,因那個理由,貪婪、憂諸惡不善法會流入那位住於意根不防護者。他走向為了那個的自制,保護意根,在在意根上來到自制。比丘們!猶如在平整地面十字路口處有已套上軛住立的、\twnr{鞭子已放置}{818.0}的駿馬車,熟練的御者、馴馬師登上它後,以左手拿起繮繩、右手拿起鞭子後,去想要之處:凡想要的,使之前去,也使之回來。同樣的,比丘們!比丘對六根為了守護學習、為了抑制學習、為了調御學習、為了寂靜學習,比丘們!這樣,比丘是在諸根上守護門者。

  比丘們!而怎樣比丘是在飲食上知適量者?比丘們!這裡,比丘如理省察後吃食物:『既不為了娛樂,也不為了自豪,也不為了裝飾,也\twnr{不為了莊嚴}{520.0},最多為了這個身體的存續、生存,為了止息傷害,為了資助\twnr{梵行}{381.0}。像這樣,我將擊退\twnr{之前的感受}{536.0},與不使新的感受生起,將有我的生存,與無過失狀態,以及\twnr{安樂住}{156.0}。』比丘們!猶如男子對傷口塗抹,最多為了治癒的目的。又或猶如對車軸塗油,最多為了貨物運送的目的。同樣的,比丘們!比丘如理省察後吃食物:『既不為了娛樂,也不為了自豪,也不為了裝飾,也不為了莊嚴,最多為了這個身體的存續、生存,為了止息傷害,為了資助梵行。像這樣,我將擊退之前的感受,與不使新的感受生起,將有我的生存,與無過失狀態,以及安樂住。』比丘們!這樣,比丘是在飲食上知適量者。

  比丘們!而怎樣比丘是專修清醒者?比丘們!這裡,比丘白天以\twnr{經行}{150.0}、安坐,使心\twnr{從障礙法}{990.1}淨化;在\twnr{初夜}{214.0},以經行、安坐,使心從障礙法淨化。在\twnr{中夜}{214.0},[左]腳放在[右]腳上、\twnr{作意起來想後}{502.0},具念正知地\twnr{以右脅作獅子臥}{367.0}。在後夜,起來後以經行、安坐,使心從障礙法淨化。』比丘們!這樣,比丘是專修清醒者。

  比丘們!具備三法的比丘當生住於豐富的樂、喜悅,以及為了諸漏的滅盡他的起源已開始。」[≃\ccchref{AN.3.16}{https://agama.buddhason.org/AN/an.php?keyword=3.16}]



\sutta{240}{240}{如烏龜經}{https://agama.buddhason.org/SN/sn.php?keyword=35.240}
  「\twnr{比丘}{31.0}們!從前,有隻陸龜烏龜傍晚時沿河岸處找食物。比丘們!有隻狐狼也傍晚時沿河岸處找食物。

  比丘們!陸龜烏龜就從遠處看見找食物的狐狼。看見後,收妥龜頭為第五的肢體到自己的龜殼後,牠\twnr{不活動}{906.0}、沈默地保持靜止。

  比丘們!狐狼就從遠處看見找食物的陸龜烏龜。看見後,去見陸龜烏龜。抵達後,侍候陸龜烏龜:『當這隻陸龜烏龜使龜頭為第五的肢體之任何一肢體轉動(伸出)時,就在那時,捉住、撕裂後我將吃它。』

  比丘們!當陸龜烏龜不使龜頭為第五的肢體之任何一肢體轉動時,那時,當沒得到機會時,厭後狐狼從烏龜離開。同樣的,比丘們!魔\twnr{波旬}{49.0}常恆不斷地侍候你們:『也許我會對他們從眼得到機會……(中略)或從舌得到機會……(中略)或從意得到機會。』

  比丘們!因此,在這裡,你們要住於在諸根上守護門:以眼見色後,你們不要成為相的執取者、\twnr{細相}{197.0}的執取者,因那個理由,\twnr{貪婪}{435.0}、憂諸惡不善法會\twnr{流入}{224.0}那位住於眼根不防護者,你們要走上為了那個的\twnr{自制}{217.0}之路,你們要保護眼根,你們要在眼根上來到自制;以耳聽聲音後……(中略)以鼻嗅氣味後……以舌嚐味道後……以身觸\twnr{所觸}{220.2}後……以意識知法後,你們不要成為相的執取者、細相的執取者,因那個理由,貪婪、憂諸惡不善法會流入那位住於意根不防護者,你們要走上為了那個的自制之路,你們要守護意根,你們要在意根上來到自制。

  比丘們!當你們住於在諸根上守護門,那時,當沒得到機會時,厭後魔波旬也將從你們離開,如狐狼從陸龜烏龜那樣。」

  「[如]收妥肢體在自己龜殼中的烏龜,比丘在意之尋上,

   是不依止者、不惱害其他人者,\twnr{般涅槃}{72.0}者不會非難任何人。」



\sutta{241}{241}{如樹幹經第一}{https://agama.buddhason.org/SN/sn.php?keyword=35.241}
  \twnr{有一次}{2.0},\twnr{世尊}{12.0}住在\twnr{憍賞彌}{140.0}的恒河邊。

  世尊看見正被恒河水流沖走的大樹幹。看見後,召喚\twnr{比丘}{31.0}們:

  「比丘們!你們看見那個正被恒河水流沖走的大樹幹嗎?」

  「是的,\twnr{大德}{45.0}!」

  「比丘們!如果那個樹幹不走到此岸,不走到彼岸,不在中間沈沒,\twnr{不在高地堆積}{x520},人的捕獲不抓住,\twnr{非人}{130.0}的捕獲不抓住,\twnr{漩渦}{993.0}的捕獲不抓住,不成為內部腐爛的,比丘們!這樣,那個樹幹必將是傾向大海的、斜向大海的、坡斜向大海的,那是什麼原因?比丘們!恒河的水流是傾向大海的、斜向大海的、坡斜向大海的。同樣的,比丘們!如果你們也不走到此岸,不走到彼岸,不在中間沈沒,不在高地堆積,人的捕獲不抓住,非人的捕獲不抓住,漩渦的捕獲不抓住,不成為內部腐爛的,比丘們!這樣,你們必將是傾向涅槃的、斜向涅槃的、坡斜向涅槃的,那是什麼原因?比丘們!正見是傾向涅槃的、斜向涅槃的、坡斜向涅槃的。」

  在這麼說時,某位比丘對世尊說這個:

  「大德!什麼是此岸?什麼是彼岸?什麼是在中間沈沒?什麼是在高地堆積?什麼是人的捕獲?什麼是非人的捕獲?什麼是漩渦的捕獲?什麼是內部腐爛的情況?」

  「比丘們!『此岸』,這是六內處的同義語;比丘們!『彼岸』,這是六外處的同義語;比丘們!『在中間沈沒』,這是歡喜貪的同義語;比丘們!『在高地堆積』,這是\twnr{我是之慢}{400.0}的同義語。

  比丘們!而怎樣是人的捕獲?比丘們!這裡,他住於與在家人交際,成為同歡者、同愁者:在受樂者們中為受樂者、在受苦者們中為受苦者、在應該被作的義務生起時以自己在他們中來到努力,比丘們!這被稱為人的捕獲。

  比丘們!而怎樣是非人的捕獲?比丘們!這裡,某一類人志向某個天眾後行梵行:『我將以這個戒,或\twnr{禁戒}{799.0},或苦行,或梵行成為天,或天神之一。』比丘們!這被稱為非人的捕獲。

  比丘們!『漩渦的捕獲』,這是\twnr{五種欲}{187.0}的同義語。

  比丘們!而怎樣是內部腐爛狀態?比丘們!這裡,某一類人是破戒者、\twnr{惡法者}{601.0}、\twnr{不淨的可疑行為者}{602.0}、\twnr{隱密行為者}{426.0}、非\twnr{沙門}{29.0}自稱沙門者、非\twnr{梵行}{381.0}者自稱梵行者、內部腐爛的\twnr{流漏者}{188.0}、壞性格者,比丘們!這被稱為『內部腐爛狀態』。」

  當時,牧牛人難陀站在世尊的不遠處。那時,牧牛人難陀對世尊說這個:

  「大德!我不走到此岸,我不走到彼岸,我將不在中間沈沒,我將不在高地堆積,人的捕獲將不抓住我,非人的捕獲將不抓住我,漩渦的捕獲將不抓住我,我不成為內部腐爛的,大德!願我得到在世尊的面前出家,願我\twnr{得到具足戒}{124.1}。」

  「難陀!那樣的話,請你交還諸母牛給主人。」

  「大德!諸求犢牛的母牛將會走。」

  「難陀!仍請你交還諸母牛給主人。」

  那時,牧牛人難陀交還諸母牛給主人後,去見世尊。抵達後,對世尊說這個:

  「大德!已交還諸母牛給主人,大德!願我得到在世尊的面前出家,願我得到具足戒。」

  那時,牧牛人難陀在世尊的面前出家,得到具足戒。還有,已受具足戒不久,\twnr{尊者}{200.0}難陀住於獨處……(中略)然後尊者難陀成為眾\twnr{阿羅漢}{5.0}之一。



\sutta{242}{242}{如樹幹經第二}{https://agama.buddhason.org/SN/sn.php?keyword=35.242}
  \twnr{有一次}{2.0},\twnr{世尊}{12.0}住在金毘羅的恒河邊。

  世尊看見正被恒河水流沖走的大樹幹。看見後,召喚\twnr{比丘}{31.0}們:

  「比丘們!你們看見那個正被恒河水流沖走的大樹幹嗎?」

  「是的,\twnr{大德}{45.0}!」……(中略)

  在這麼說時,\twnr{尊者}{200.0}金毘羅對世尊說這個:「大德!什麼是此岸?什麼是彼岸?……(中略)金毘羅!而怎樣是內部腐爛狀態?金毘羅!這裡,比丘是犯某種污染的罪者:這樣罪的出罪不被知道,金毘羅!這被稱為『內部腐爛狀態』。」



\sutta{243}{243}{流漏法門經}{https://agama.buddhason.org/SN/sn.php?keyword=35.243}
  \twnr{有一次}{2.0},\twnr{世尊}{12.0}住在釋迦族人的迦毘羅衛城尼拘律園。當時,迦毘羅衛城釋迦族人的新\twnr{集會所}{761.0}被建造不久,未被\twnr{沙門}{29.0}或\twnr{婆羅門}{17.0}或任何生為人的居住。

  那時,迦毘羅衛城的釋迦族人去見世尊。抵達後,向世尊\twnr{問訊}{46.0}後,在一旁坐下。在一旁坐下的迦毘羅衛城的釋迦族人對世尊說這個:「\twnr{大德}{45.0}!這裡,迦毘羅衛城釋迦族人的新集會所被建造不久,未被沙門或婆羅門,或任何生為人的居住,大德!請世尊第一個使用它,世尊第一個使用後,迦毘羅衛城的釋迦族人將使用它,那對迦毘羅衛城釋迦族人會有長久的利益、安樂。」世尊以沈默狀態同意。

  那時,迦毘羅衛城的釋迦族人知道世尊同意後,從座位起來、向世尊問訊、\twnr{作右繞}{47.0}後,去新集會所。抵達後,鋪設集會所的一切鋪設物、設置諸座位、使水瓶設立、懸掛油燈後,去見世尊。抵達後,對世尊說這個:「大德!集會所的一切鋪設物已鋪設,諸座位已設置,已使水瓶設立,油燈已懸掛,大德!現在是那個世尊\twnr{考量的時間}{84.0}。」那時,世尊穿衣、拿起衣鉢後,與\twnr{比丘}{31.0}\twnr{僧團}{375.0}一起去新集會所。抵達後,使腳洗滌、進入集會所後,依止中央柱子面向東坐下,比丘僧團也使腳洗滌、進入集會所後,依止西邊牆壁面向東,置世尊在前面後坐下,迦毘羅衛城的釋迦族人使腳洗滌、進入集會所後,依止東邊牆壁面向西,面對世尊坐下。那時,世尊對迦毘羅衛城的釋迦族人大部分夜晚以法說開示、勸導、鼓勵、\twnr{使歡喜}{86.0}後,使離開:「\twnr{喬達摩}{80.0}們!夜已過,現在是那個你們考量的時間。」「是的,大德!」迦毘羅衛城的釋迦族人回答世尊後,從座位起來、向世尊問訊、作右繞後離開。

  那時,在迦毘羅衛城的釋迦族人離開不久,世尊召喚\twnr{尊者}{200.0}大目揵連:「目揵連!比丘僧團離惛沈睡眠,目揵連!請你為比丘們顯現法的談論,我的背痛,我要伸展它。」「是的,大德!」尊者大目揵連回答世尊。那時,世尊摺大衣成四折後,[左]腳放在[右]腳上、\twnr{作意起來想後}{502.0},具念正知地\twnr{以右脅作獅子臥}{367.0}。在那裡,尊者大目揵連召喚比丘們:「比丘學友們!」「\twnr{學友}{201.0}!」那些比丘們回答尊者大目揵連。尊者大目揵連說這個:

  「學友們!我將為你們教導\twnr{流漏法門}{188.2},以及不流漏法門。你們要聽它!你們要\twnr{好好作意}{43.1}!我將說。」「是的,學友!」那些比丘們回答尊者大目揵連。尊者大目揵連說這個:

  「學友們!而怎樣是流漏者?學友們!這裡,比丘以眼見色後,\twnr{志向}{257.0}可愛形色的諸色,排拒不可愛形色的諸色,住於身念未現起的、少心的,以及不如實知道那個\twnr{心解脫}{16.0}、\twnr{慧解脫}{539.0},於該處他的那些生起的惡不善法無殘餘地被滅……(中略)以舌嚐味道後……(中略)以意識知法後,志向可愛形色的諸法,排拒不可愛形色的諸法,住於身念未現起的、少心的,以及不如實知道那個心解脫、慧解脫,於該處他的那些生起的惡不善法無殘餘地被滅,學友們!這被稱為比丘在能被眼識知的諸色上是流漏者……(中略)在能被舌識知的諸味道上是流漏者……(中略)在能被意識知的諸法上是流漏者。學友們!這麼住的比丘如果魔從眼接近他,魔就得到機會,魔得到對象……(中略)又如果魔從舌接近他,魔就得到機會,魔得到對象……(中略)又如果魔從意接近他,魔就得到機會,魔得到對象。

  學友們!猶如有乾枯超過一年的蘆葦屋或茅草屋,如果男子從東方以燃燒的草炬接近它,火就得到機會,火得到對象;又如果男子從西方以燃燒的草炬接近它……(中略)又如果從北方……(中略)又如果從南方……(中略)又如果從下方……(中略)又如果從上方……(中略)又如果男子從任何處以燃燒的草炬接近它,火就得到機會,火得到對象。同樣的,學友們!這麼住的比丘如果魔從眼接近他,魔就得到機會,魔得到對象……(中略)又如果魔從舌接近他……(中略)又如果魔從意接近他,魔就得到機會,魔得到對象。

  學友們!諸色征服這麼住的比丘,非比丘征服諸色;諸聲音征服比丘,非比丘征服諸聲音;諸氣味征服比丘,非比丘征服諸氣味;諸味道征服比丘,非比丘征服諸味道;諸\twnr{所觸}{220.2}征服比丘,非比丘征服諸所觸;諸法征服比丘,非比丘征服諸法,學友們!這被稱為比丘是被色征服者、被聲音征服者、被氣味征服者、被味道征服者、被所觸征服者、被法征服者、被征服者、非征服者,污染的、\twnr{再有的}{497.0}、有恐懼的、苦果報的、未來生老死的諸惡不善法征服他。學友們!這樣是流漏者。

  學友們!而怎樣是不流漏者?學友們!這裡,比丘以眼見色後,不志向可愛形色的諸色,不排拒不可愛形色的諸色,住於身念已現起的、無量心的,以及如實知道那個心解脫、慧解脫,於該處他的那些生起的惡不善法無殘餘地被滅……(中略)以舌嚐味道後……(中略)以意識知法後,不志向可愛形色的諸法,不排拒不可愛形色的諸法,住於身念已現起的、無量心的,以及如實知道那個心解脫、慧解脫,於該處他的那些生起的惡不善法無殘餘地被滅,學友們!這被稱為比丘在能被眼識知的諸色上是不流漏者……(中略)在能被意識知的諸法上是不流漏者。學友們!這麼住的比丘如果魔從眼接近他,魔既沒得到機會,魔也沒得到對象……(中略)又如果魔從舌接近他……(中略)又如果魔從意接近他,魔既沒得到機會,魔也沒得到對象。

  學友們!猶如有重閣或厚黏土新塗布的講堂,如果男子從東方以燃燒的草炬接近它,火既沒得到機會,火也沒得到對象;又如果從西方……(中略)又如果從北方……(中略)又如果從南方……(中略)又如果從下方……(中略)又如果從上方……(中略)又如果男子從任何處以燃燒的草炬接近它,火既沒得到機會,火也沒得到對象。同樣的,學友們!這麼住的比丘如果魔從眼接近他,魔既沒得到機會,魔也沒得到對象……(中略)又如果魔從意接近他,魔既沒得到機會,魔也沒得到對象。學友們!這麼住的比丘征服諸色,非諸色征服比丘;比丘征服諸聲音,非諸聲音征服比丘;比丘征服諸氣味,非諸氣味征服比丘;比丘征服諸味道,非諸味道征服比丘;比丘征服諸所觸,非諸所觸征服比丘;比丘征服諸法,非諸法征服比丘,學友們!這被稱為比丘是色之征服者、聲音之征服者、氣味之征服者、味道之征服者、所觸之征服者、法之征服者、征服者、非被征服者,他征服那些污染的、再有的、有恐懼的、苦果報的、未來生老死的諸惡不善法。學友們!這樣是不流漏者。」

  那時,世尊起來後召喚尊者大目揵連:「目揵連!\twnr{好}{44.0}!好!目揵連!你為比丘們說流漏法門與不流漏法門,好!」

  尊者大目揵連說這個,\twnr{大師}{145.0}是認可者。那些悅意的比丘們歡喜尊者大目揵連所說。



\sutta{244}{244}{苦法經}{https://agama.buddhason.org/SN/sn.php?keyword=35.244}
  「\twnr{比丘}{31.0}們!當比丘都如實知道一切苦法的\twnr{集起}{67.0}與滅沒,那麼,像這樣他的諸欲被看見,當如是看見他的諸欲時,凡在諸欲上之\twnr{欲的意欲}{118.0}、欲愛、欲迷、欲的熱惱,那個不\twnr{潛伏}{253.0}。又,像這樣他的行與\twnr{住}{x521}被隨覺,當如是行、住時,諸\twnr{貪婪}{435.0}、憂之惡不善法不潛伏(不流入?)。

  比丘們!而怎樣是比丘如實知道一切苦法的集起與滅沒?『像這樣是色,像這樣是色的集起,像這樣是色的滅沒;像這樣是受……像這樣是想……像這樣是行……像這樣是識,像這樣是識的集起,像這樣是識的滅沒。』比丘們!這樣是比丘如實知道一切苦法的集起與滅沒。

  比丘們!而怎樣是比丘的諸欲被看見,當如是看見他的諸欲時,凡在諸欲上之欲的意欲、欲愛、欲迷、欲的熱惱,那個不潛伏?比丘們!猶如有超過一人深的炭火坑被無焰的、無煙的炭火充滿。那時,想要活命、不想要死,想要樂、厭逆苦的男子到來,兩位有力氣的男子隨即在不同手臂處捉住後,拉他向那個炭火坑。他像這樣與像那樣地扭曲身體,那什麼原因呢?比丘們!因為被那位男子知道:『而我將跌落這個炭火坑,從那個因由我將遭受死亡,或死亡程度的苦。』同樣的,比丘們!比丘的諸欲如炭火坑被看見,當如是看見他的諸欲時,凡在諸欲上之欲的意欲、欲愛、欲迷、欲的熱惱,那個不潛伏。

  比丘們!而怎樣是比丘的行與住被隨覺,當如是行、住時,諸貪婪、憂之惡不善法\twnr{不流入}{224.0}?比丘們!猶如男子進入許多荊棘的森林,他的前面是荊棘,後面也是荊棘;左邊是荊棘,右邊也是荊棘;上面是荊棘,下面也是荊棘,他就具念地前進,就具念地後退:『不要荊棘對我。』同樣的,比丘們!凡世間中的可愛形色、\twnr{合意形色}{962.0},這被稱為:『在聖者之律中的荊棘』。像這樣知道後,\twnr{自制}{217.0}與不自制應該被知道(感知)。

  比丘們!而怎樣是不自制?比丘們!這裡,比丘以眼見色後,\twnr{志向}{257.0}可愛形色的諸色,排拒不可愛形色的諸色,住於身念未現起的、少心的,以及不如實知道那個\twnr{心解脫}{16.0}、\twnr{慧解脫}{539.0},於該處他的那些生起的惡不善法無殘餘地被滅……(中略)以舌嚐味道後……(中略)以意識知法後,志向可愛形色的諸法,排拒不可愛形色的諸法,住於身念未現起的、少心的,以及不如實知道那個心解脫、慧解脫,於該處他的那些生起的惡不善法無殘餘地被滅,比丘們!這樣是不自制。

  比丘們!而怎樣是自制?比丘們!這裡,比丘以眼見色後,不志向可愛形色的諸色,不排拒不可愛形色的諸色,住於身念已現起的、無量心的,以及如實知道那個心解脫、慧解脫,於該處他的那些生起的惡不善法無殘餘地被滅……(中略)以舌嚐味道後……(中略)以意識知法後,不志向可愛形色的諸法,不排拒不可愛形色的諸法,住於身念已現起的、無量心的,以及如實知道那個,於該處他的那些生起的惡不善法無殘餘地被滅,比丘們!這樣是自制。

  比丘們!這麼行、這麼住的那位比丘即使偶爾由於念的混亂,隨順結的諸惡不善念的意向生起,比丘們!念的生起[或]是徐緩的,[一生起,]那時,就急速地捨斷、驅離、作終結、使之走到不存在。

  比丘們!猶如男子使二、三滴水滴落在中午被曬熱的鐵盤上,比丘們!水滴的落下[或]是緩慢的,[一落下,]那時,它就急速地走到遍盡、耗盡(遍取)。同樣的,比丘們!這麼行、這麼住的那位比丘即使偶爾由於念的混亂,隨順結的諸惡不善念的意向生起,比丘們!念的生起[或]是徐緩的,[一生起,]那時,就急速地捨斷、驅離、作終結、使之走到不存在。比丘們!這樣是比丘的行與住被隨覺,當如是行、住時,諸貪婪、憂之惡不善法不流入。

  比丘們!如果國王,或國王的大臣,或朋友,或同事,或親族,或血親以財富帶來後邀請這麼行、這麼住的那位比丘:『來!男子\twnr{先生}{202.0}!為何讓這些袈裟耗盡你?為何你實行光頭、鉢?來!還俗後請你在財富上受用與作福德。』比丘們!確實,『這麼行、這麼住的那位比丘放棄學後將還俗。』\twnr{這不存在可能性}{650.0}。

  比丘們!猶如恒河是傾向東的、斜向東的、坡斜向東的,那時,大群人拿鋤頭、籃子後:『我們將轉(作)這恒河成傾向西的、斜向西的、坡斜向西的。』比丘們!你們怎麼想它:是否那個大群人會轉這恒河成傾向西的、斜向西的、坡斜向西的呢?」

  「\twnr{大德}{45.0}!這確實不是,那是什麼原因?大德!恒河是傾向東的、斜向東的、坡斜向東的,不容易轉成傾向西的、斜向西的、坡斜向西的,還有,大群人最終只會是疲勞的、苦惱的\twnr{有分者}{876.0}。」

  「同樣的,比丘們!如果國王,或國王的大臣,或朋友,或同事,或親族,或血親以財富帶來後邀請這麼行、這麼住的那位比丘:『來!男子先生!為何讓這些袈裟耗盡你?為何你實行光頭、鉢?來!還俗後請你在財富上受用與作福德。』比丘們!確實,『這麼行、這麼住的那位比丘放棄學後將還俗。』這不存在可能性,那是什麼原因?比丘們!因為那顆心長久是傾向遠離的、斜向遠離的、坡斜向遠離的,『像那樣他將還俗。』這不存在可能性。」



\sutta{245}{245}{如緊叔迦經}{https://agama.buddhason.org/SN/sn.php?keyword=35.245}
  那時,\twnr{某位比丘}{39.0}去見另一位比丘。抵達後,對那位比丘說這個:「\twnr{學友}{201.0}!什麼情形比丘的見是善清淨的呢?」「學友!當比丘如實知道\twnr{六觸處}{78.0}的\twnr{集起}{67.0}與滅沒,學友!這個情形比丘的見是善清淨的。」

  那時,那位比丘不被那個比丘對問題的解說滿意,去見另一位比丘。抵達後,對那位比丘說這個:「學友!什麼情形比丘的見是善清淨的呢?」「學友!當比丘如實知道五取蘊的集起與滅沒,學友!這個情形比丘的見是善清淨的。」

  那時,那位比丘不被那個比丘對問題的解說滿意,去見另一位比丘。抵達後,對那位比丘說這個:「學友!什麼情形比丘的見是善清淨的呢?」「學友!當比丘如實知道\twnr{四大}{646.0}的集起與滅沒,學友!這個情形比丘的見是善清淨的。」

  那時,那位比丘不被那個比丘對問題的解說滿意,去見另一位比丘。抵達後,對那位比丘說這個:「學友!什麼情形比丘的見是善清淨的呢?」「學友!當比丘如實知道:『凡任何\twnr{集法}{67.1}那個全部是\twnr{滅法}{68.1}。』學友!這個情形比丘的見是善清淨的。」

  那時,那位比丘不被那個比丘對問題的解說滿意,去見\twnr{世尊}{12.0}。抵達後,對世尊說這個:「\twnr{大德}{45.0}!這裡,我去見某位比丘。抵達後,對那位比丘說這個:『學友!什麼情形比丘的見是善清淨的呢?』大德!在這麼說時,那位比丘對我說這個:『學友!當比丘如實知道六觸處的集起與滅沒,學友!這個情形比丘的見是善清淨的。』大德!那時,我不被那個比丘對問題的解說滿意,去見另一位比丘。抵達後,對那位比丘說這個:『學友!什麼情形比丘的見是善清淨的呢?』大德!在這麼說時,那位比丘對我說這個:『學友!當比丘如實知道五取蘊的……(中略)如實知道四大的集起與滅沒……(中略)如實知道:「凡任何集法那個全部是滅法。」學友!這個情形比丘的見是善清淨的。』大德!那時,我不被那個比丘對問題的解說滿意,來見世尊。大德!什麼情形比丘的見是善清淨的呢?」

  「比丘!猶如有以前未看見\twnr{緊叔迦}{x522}的男子,他去見某位緊叔迦看見者的男子。抵達後,對那位男子這麼說:『男子\twnr{先生}{202.0}!緊叔迦是像什麼樣子?』他這麼說:『喂!男子!緊叔迦是黑色的,猶如燃燒[過]的殘株。』比丘!而當時,緊叔迦就是正如那位男子的看見。比丘!那時,那位男子不被那個男子對問題的解說滿意,去見另一位緊叔迦看見者的男子。抵達後,對那位男子這麼說:『男子先生!緊叔迦是像什麼樣子?』他這麼說:『喂!男子!緊叔迦是紅色的,猶如肉片。』比丘!而當時,緊叔迦就是正如那位男子的看見。比丘!那時,那位男子不被那個男子對問題的解說滿意,去見另一位緊叔迦看見者的男子。抵達後,對那位男子這麼說:『男子先生!緊叔迦是像什麼樣子?』他這麼說:『喂!男子!緊叔迦是變成樹皮下垂的、果莢破裂的,猶如\twnr{金合歡樹}{x523}。』比丘!而當時,緊叔迦就是正如那位男子的看見。比丘!那時,那位男子不被那個男子對問題的解說滿意,去見另一位緊叔迦看見者的男子。抵達後,對那位男子這麼說:『男子先生!緊叔迦是像什麼樣子?』他這麼說:『喂!男子!緊叔迦有茂密的樹葉、影子濃密(群集),猶如\twnr{榕樹}{x524}。』比丘!而當時,緊叔迦就是正如那位男子的看見。同樣的,比丘!一一如那些\twnr{勝解}{257.0}善人的善清淨看見,一一那樣地被那些善人回答。

  比丘!猶如國王邊境堅固壁壘、堅固城牆城門的城市,有六道門,在那裡,有賢智的、有能力的、聰明的、未知者(陌生人)之制止的、已知者使進入的守門人,一對急速的使者從東方來到後,對守門人這麼說:『男子先生!這座城的城主在哪裡?』他這麼說:『大德!他坐在中央四衢街道。』那時,那對急速的使者[去見城主,]對城主交與如實話語後,走向如到來的道路。一對急速的使者從西方來到後……(中略)從北方……一對急速的使者從南方來到後,對守門人這麼說:『男子先生!這座城的城主在哪裡?』他會這麼回答:『大德!他坐在中央四衢街道。』那時,那對急速的使者對城主交與如實話語後,走向如到來的道路。

  比丘!為了義理的使知這個譬喻被我作。在這裡,這就是義理:比丘!『城市』,這是這父母生成的、米粥積聚的、無常-塗身-\twnr{按摩}{967.0}-破壞-分散法的四大身的同義語。比丘!『六道門』,這是六內處的同義語。比丘!『守門人』,這是念的同義語。比丘!『一對急速的使者』,這是\twnr{止觀}{178.0}的同義語。比丘!『城主』,這是識的同義語。比丘!『在中央四衢街道』,這是四大的同義語,即:地界的、水界的、火界的、風界的。比丘!『如實話語』,這是涅槃的同義語。比丘!『如到來的道路』,這是\twnr{八支聖道}{525.0}的同義語,即:正見……(中略)正定。」



\sutta{246}{246}{如琵琶琴經}{https://agama.buddhason.org/SN/sn.php?keyword=35.246}
  「\twnr{比丘}{31.0}們!對凡任何比丘或比丘尼,在能被眼識知的諸色上如果生起意欲,或貪,或瞋,或癡,或甚至心的嫌惡者,應該從那裡阻止心:『這是一條有恐怖,有所怖畏,有荊棘,有叢林的道路,是\twnr{歧途}{563.0}、邪道、有劫賊災難的,這是非善人親近的道路、這不是善人親近的道路,你不適合這個。』從那個因由,應該從能被眼識知的諸色阻止心。……(中略)。

  比丘們!對凡任何比丘或比丘尼,在能被舌識知的諸味道上……(中略)在能被意識知的諸法上,如果生起意欲,或貪,或瞋,或癡,或甚至心的嫌惡者,應該從那裡阻止心:『這是一條有恐怖,有所怖畏,有荊棘,有叢林的道路,是歧途、邪道、有劫賊災難的,這是非善人親近的道路,以及這不是善人親近的道路,你不適合這個。』從那個因由,應該從能被意識知的諸法阻止心。

  比丘們!猶如稻田(穀物)已成熟(具足),而稻田守護者是放逸者,又吃穀物的牛進入那個稻田後,盡情地來到陶醉,來到放逸。同樣的,比丘們!\twnr{未聽聞的一般人}{74.0}在\twnr{六觸處}{78.0}上是不\twnr{自制}{217.0}的作者,在\twnr{五種欲}{187.0}上盡情地來到陶醉,來到放逸。

  比丘們!猶如稻田已成熟,而稻田守護者是不放逸者,又吃穀物的牛進入那個稻田,稻田守護者在牛鼻處善捉住地捉住牠;在牛鼻處善捉住地捉住後,在頭上的兩角處善制伏地制伏;在頭上的兩角處善制伏地制伏後,以棍棒善敲打地敲打;以棍棒善敲打地敲打後驅離。

  比丘們!第二次又……(中略)比丘們!第三次,吃穀物的牛又進入那個稻田,稻田守護者在牛鼻處善捉住地捉住牠;在牛鼻處善捉住地捉住後,在頭上的兩角處善制伏地制伏;在頭上的兩角處善制伏地制伏後,以棍棒善敲打地敲打;以棍棒善敲打地敲打後驅離。

  比丘們!這樣,那隻吃穀物的牛來到村落或來到山林,常站的或常坐的,都記得(隨念著)那個先前的棍棒之觸,不再進入那個稻田。同樣的,比丘們!當比丘的心在六觸處上被克制(征服?)、被善克制,自身內在就安頓,平靜,成為專一的,入定。

  比丘們!猶如國王或國王的大臣是以前未聽過琵琶琴聲者,他聽聞琵琶琴聲,他這麼說:『喂!這個這麼誘人的(能被染的)、這麼能被欲求的、這麼能被陶醉的、這麼能被迷戀的、這麼能被繫縛的聲音是哪個的?』他們這麼說它:『\twnr{大德}{45.0}!這是名為琵琶琴:這個這麼誘人的、這麼能被欲求的、這麼能被陶醉的、這麼能被迷戀的、這麼束縛人的聲音。』

  他這麼說:『\twnr{先生}{202.0}!去!請你們為我帶來那個琵琶琴。』

  他們為他帶來那個琵琶琴,他們這麼說它:『大德!這是那個琵琶琴:這個這麼誘人的、這麼能被欲求的、這麼能被陶醉的、這麼能被迷戀的、這麼束縛人的聲音。』

  他這麼說:『先生!以那個琵琶琴對我來說夠了!請你們就為我帶來那個聲音。』

  他們這麼說它:『大德!這名為琵琶琴有種種要素、大要素,被種種要素從事發聲(說),即:\twnr{緣於}{252.0}腹板,緣於腹皮,緣於頸把,緣於頭,緣於絃,緣於弦撥,緣於男子適當的努力,大德!這樣,這名為琵琶琴有種種要素、大要素,被種種要素從事發聲。』

  他使那個琵琶琴破裂成十塊或百塊,使那個琵琶琴破裂成十塊或百塊後,做成碎碎的,做成碎碎的後以火燃燒,以火燃燒後轉成(作)灰,轉成灰後在大風處暴露,或在河中急流處沖走,他這麼說:『先生!這名為琵琶琴確實是像這樣這麼不實的,還有,凡任何名為琵琶琴者,這裡,這人們長時間成為放逸的、沈迷的。』同樣的,比丘們!比丘探求色,直到色的趣處之所及;探求受,直到受的趣處之所及;探求想,直到想的趣處之所及;探求諸行,直到諸行的趣處之所及;探求識,直到識的趣處之所及,當他探求色,直到色的趣處之所及;探求受……(中略)想……諸行……探求識,直到識的趣處之所及時,凡對他來說那個是『我』,或『我的』,或『我是』者,那個對他來說都不存在。」



\sutta{247}{247}{如六種生物經}{https://agama.buddhason.org/SN/sn.php?keyword=35.247}
  「\twnr{比丘}{31.0}們!猶如肢體受傷的、肢體化膿的男子進入蘆葦林,茅草荊棘刺在腳上,同時蘆葦葉刮肢體,比丘們!這樣,那位男子更從那個因由感受苦憂。同樣的,比丘們!這裡,某一類走到村落或山林的比丘\twnr{得到責備(說他人者)}{x525}:『這位這麼作的、這麼行的\twnr{尊者}{200.0}\twnr{是不淨的村落荊棘者}{x526}。』像這樣,知道他是『荊棘』後,\twnr{自制}{217.0}與不自制應該被知道(感知)。

  比丘們!而怎樣是不自制?比丘們!這裡,比丘以眼見色後,\twnr{志向}{257.0}可愛形色的諸色,\twnr{排拒}{x527}不可愛形色的諸色,住於身念未現起的、少心的,以及不如實知道那個\twnr{心解脫}{16.0}、\twnr{慧解脫}{539.0},於該處他的那些生起的惡不善法無殘餘地被滅……以舌嚐味道後……以身觸\twnr{所觸}{220.2}後……以意識知法後,志向可愛形色的諸法,排拒不可愛形色的諸法,住於身念未現起的、少心的,以及不如實知道那個心解脫、慧解脫,於該處他的那些生起的惡不善法無殘餘地被滅。

  比丘們!猶如男子捕捉不同境域、\twnr{不同行境}{897.0}的六種生物後,以堅固的繩索捆綁:捕捉蛇後,以堅固的繩索捆綁;捕捉\twnr{鱷魚}{x528}後,以堅固的繩索捆綁;捕捉鳥後,以堅固的繩索捆綁;捕捉狗後,以堅固的繩索捆綁;捕捉狐狼後,以堅固的繩索捆綁;捕捉猴子後,以堅固的繩索捆綁。以堅固的繩索捆綁後,在中間打結後放手,比丘們!那時,不同境域、不同行境的六種生物拉向各自的境域、行境:蛇拉向『我將入蟻丘。』鱷魚拉向『我將入水。』鳥拉向『我將飛天空。』狗拉向『我將入村落。』狐狼拉向『我將入墓地。』猴子拉向『我將入樹林。』比丘們!當那六種生物被消耗、被疲勞,那時,凡那些生物中成為比較有力者,牠們對牠服從、隨順從、落入控制。同樣的,比丘們!凡任何比丘的\twnr{身至念}{521.0}未\twnr{修習}{94.0}、未\twnr{多作}{95.0}者,他眼拉向諸合意的色上,諸不合意的色成為\twnr{厭逆}{227.0}的……(中略)意拉向諸合意的法上,諸不合意的法成為厭逆的,比丘們!這樣是不自制。

  比丘們!而怎樣是自制?比丘們!這裡,比丘以眼見色後,不志向可愛形色的諸色,不排拒不可愛形色的諸色,住於身念已現起的、無量心的,以及如實知道那個,於該處他的那些生起的惡不善法無殘餘地被滅……以舌嚐味道後……以身觸所觸後……以意識知法後,不志向可愛法,不排拒不可愛形色的諸法,住於身念已現起的、無量心的,以及如實知道那個,於該處他的那些生起的惡不善法無殘餘地被滅。

  比丘們!猶如男子捕捉不同境域、不同行境的六種生物後,以堅固的繩索捆綁:捕捉蛇後,以堅固的繩索捆綁;捕捉鱷魚後,以堅固的繩索捆綁;捕捉鳥後……(中略)捕捉狗後……捕捉狐狼後……捕捉猴子後,以堅固的繩索捆綁。以堅固的繩索捆綁後,緊綁在堅固的樁或柱上,比丘們!那時,不同境域、不同行境的六種生物拉向各自的境域、行境:蛇拉向『我將要入蟻丘。』鱷魚拉向『我將要入水。』鳥拉向『我將要飛天空。』狗拉向『我將要入村落。』狐狼拉向『我將要入墓地。』猴子拉向『我將要入樹林。』比丘們!當那六種生物被消耗、被疲勞,那時,牠們就對那枝樁或柱靠近站立、靠近坐下、靠近躺臥。同樣的,比丘們!凡任何比丘的身至念已修習、已多作者,他眼不拉向諸合意的色上,諸不合意的色不成為厭逆的……(中略)舌不拉向諸合意的味道上……(中略)意不拉向諸合意的法上,諸不合意的法不成為厭逆的,比丘們!這樣是自制。

  比丘們!『堅固的樁或柱』,這是身至念的同義語,比丘們!因此,在這裡,應該被你們這麼學:『我們的身至念將被修習、被多作、被作為車輛、被作為基礎、被實行、被累積、\twnr{被善努力}{682.0}。』比丘們!應該被這麼學。」



\sutta{248}{248}{麥捆經}{https://agama.buddhason.org/SN/sn.php?keyword=35.248}
  「\twnr{比丘}{31.0}們!猶如麥捆被放置在十字路口,那時,六名手持打麥棒的男子到來,他們以六根打麥棒打麥捆。比丘們!這樣,當被六根打麥棒打時,那個麥捆被徹底地打(善打)。那時,第七名手持打麥棒的男子到來,他以第七根打麥棒打那個麥捆。比丘們!這樣,當被第七根打麥棒打時,那個麥捆更被徹底地打。同樣的,比丘們!\twnr{未聽聞的一般人}{74.0}在眼處被合意不合意諸色打……(中略)在舌處被合意不合意諸味道打……(中略)在意處被合意不合意諸法打,比丘們!如果那位未聽聞的一般人意圖未來再有(再生),比丘們!這樣,那位愚鈍的男子更被徹底地打,猶如那個被第七根打麥棒打的麥捆。

  比丘們!從前,天神、阿修羅的戰鬥已群集,比丘們!那時,毘摩質多阿修羅王召喚阿修羅們:『\twnr{親愛的先生}{204.0}們!在天神、阿修羅的戰鬥已群集中,如果阿修羅們勝,天神們敗,那樣,以脖子為第五繫縛[\suttaref{SN.11.4}],繫縛那位天帝釋後,請你們帶來阿修羅城我的面前。』

  比丘們!那時,天帝釋也召喚三十三天天神:『親愛的先生們!在天神、阿修羅的戰鬥已群集中,如果天神們勝,阿修羅們敗,這樣,以脖子為第五繫縛,繫縛那位毘摩質多阿修羅王後,請你們帶來天的善法堂我的面前。』

  比丘們!又,在那場戰鬥中,天神們勝,阿修羅們敗。

  比丘們!那時,三十三天天神以脖子為第五繫縛,繫縛毘摩質多阿修羅王後,帶來天神善法堂天帝釋的面前。

  比丘們!在那裡,毘摩質多阿修羅王以脖子為第五被繫縛。比丘們!當毘摩質多阿修羅王這麼想:『天神是如法的,阿修羅是不如法的,現在,就在這裡,我到天神城市。』那時,他認為(見)自己以脖子為第五繫縛被釋放,且賦有、擁有天的\twnr{五種欲}{187.0}自娛。比丘們!但當毘摩質多阿修羅王這麼想:『阿修羅是如法的,天神是不如法的,現在,就在這裡,我將到阿修羅城。』那時,他認為自己以脖子為第五繫縛被繫縛,且從天的五種欲退失。

  比丘們!毘摩質多的繫縛是這麼微細,魔的繫縛比那個更微細。

  比丘們!\twnr{思量者}{963.0}被魔繫縛;不思量者被\twnr{波旬}{49.0}釋放。[\suttaref{SN.22.64}]

  比丘們!『\twnr{我是}{894.0}』,這是思量;『我是這個』,這是思量;『我將是』,這是思量;『我將不是』,這是思量;『我將是有色者』,這是思量;『我將是無色者』,這是思量;『我將是有想者』,這是思量;『我將是無想者』,這是思量;『我將是非想非非想者』,這是思量。

  比丘們!思量是病,思量是腫瘤,思量是箭。比丘們!因此,在這裡,『我們將以無思量之心居住(生活)。』比丘們!應該被你們這麼學。

  比丘們!『我是』,這是動搖;『我是這個』,這是動搖;『我將是』,這是動搖;『我將不是』,這是動搖;『我將是有色者』,這是動搖;『我將是無色者』,這是動搖;『我將是有想者』,這是動搖;『我將是無想者』,這是動搖;『我將是非想非非想者』,這是動搖。

  比丘們!動搖是病,動搖是腫瘤,動搖是箭。比丘們!因此,在這裡,『我們將以無動搖之心居住。』比丘們!應該被你們這麼學。

  比丘們!『我是』,這是悸動;『我是這個』,這是悸動;『我將是』……(中略)『我將不是』……『我將是有色者』……『我將是無色者』……『我將是有想者』……『我將是無想者』……『我將是非想非非想者』,這是悸動。

  比丘們!悸動是病,悸動是腫瘤,悸動是箭。比丘們!因此,在這裡,『我們將以無悸動之心居住。』比丘們!應該被你們這麼學。

  比丘們!『我是』,這是\twnr{虛妄}{952.0};『我是這個』,這是虛妄;『我將是』……(中略)『我將不是』……『我將是有色者』……『我將是無色者』……『我將是有想者』……『我將是無想者』……『我將是非想非非想者』,這是虛妄。

  比丘們!虛妄是病,虛妄是腫瘤,虛妄是箭。比丘們!因此,在這裡,『我們將以無虛妄之心居住。』比丘們!應該被你們這麼學。

  比丘們!『我是』,這是來到慢的情況;『我是這個』,這是來到慢的情況;『我將是』,這是來到慢的情況;『我將不是』,這是來到慢的情況;『我將是有色的』,這是來到慢的情況;『我將是無色的』,這是來到慢的情況;『我將是有想的』,這是來到慢的情況;『我將是無想的』,這是來到慢的情況;『我將是非想非非想的』,這是來到慢的情況。

  比丘們!來到慢的情況是病,來到慢的情況是腫瘤,來到慢的情況是箭。

  比丘們!因此,在這裡,『我們將以已破慢之心居住。』比丘們!應該被你們這麼學。」

  毒蛇品第十九,其\twnr{攝頌}{35.0}:

  「毒蛇、車子、烏龜,樹幹二則、流漏,

   苦法、緊叔迦、琵琶琴,六種生物、麥捆。」

  六處篇第四個五十則完成,其品的攝頌:

  「歡喜的滅盡、六十理趣,以及海與蛇,

   這些是第四個五十,在集篇中被知道。」

  六處相應完成。





\page

\xiangying{36}{受相應}
\pin{有偈品}{1}{10}
\sutta{1}{1}{定經}{https://agama.buddhason.org/SN/sn.php?keyword=36.1}
  「\twnr{比丘}{31.0}們!有這三受。哪三個?樂受、苦受、不苦不樂受。比丘們!這些是三受。」

  「得定的、正知的,具念的佛弟子,

   他知道受,與受的生成。

   以及凡這些被滅之處,與導向滅盡之道,

   比丘以受的滅盡,成為無飢渴者、般涅槃者。」[\ccchref{It.52}{https://agama.buddhason.org/It/dm.php?keyword=52}]



\sutta{2}{2}{樂經}{https://agama.buddhason.org/SN/sn.php?keyword=36.2}
  「\twnr{比丘}{31.0}們!有這三受。哪三個?樂受、苦受、不苦不樂受。比丘們!這些是三受。」

  「不論樂或苦,連同不苦不樂,

   自身內的與外部的,凡有任何被感受的。

   知道『這是苦的』,是虛假法、壞散的後,

   一再接觸後看見消散,這樣在那裡\twnr{離染}{558.0}。」



\sutta{3}{3}{捨斷經}{https://agama.buddhason.org/SN/sn.php?keyword=36.3}
  「\twnr{比丘}{31.0}們!有這些三受,哪三個?樂受、苦受、不苦不樂受。

  比丘們!對樂受,\twnr{貪煩惱潛在趨勢}{452.0}應該被捨斷,對苦受,\twnr{嫌惡煩惱潛在趨勢}{453.0}應該被捨斷,對不苦不樂受,\twnr{無明煩惱潛在趨勢}{454.0}應該被捨斷。

  比丘們!對比丘來說,當對樂受,貪煩惱潛在趨勢被捨斷;對苦受,嫌惡煩惱潛在趨勢被捨斷;對不苦不樂受,無明煩惱潛在趨勢被捨斷,比丘們!這被稱為:『比丘是無\twnr{煩惱潛在趨勢}{253.1}者、正確看見者,他切斷渴愛,破壞結,\twnr{從慢的完全止滅}{592.0}\twnr{作苦的終結}{54.0}。』」

  「對感受樂時,不知道受者,

   他有貪煩惱潛在趨勢,是不見\twnr{出離}{294.0}者。

   對感受苦時,不知道受者,

   他有嫌惡煩惱潛在趨勢,是不見出離者。

   寂靜的不苦不樂,被廣慧者教導,

   如果他還歡喜它,也不從苦被釋放。

   但當熱心的比丘,不疏忽正知,

   從那裡那位賢智者,\twnr{遍知}{154.0}一切受。

   他遍知受後,\twnr{當生}{42.0}成為無\twnr{漏}{188.0}者,

   住法者以身體的崩解,通曉吠陀者\twnr{不來到稱呼}{254.0}。[\suttaref{SN.36.12}]」



\sutta{4}{4}{深淵經}{https://agama.buddhason.org/SN/sn.php?keyword=36.4}
  「\twnr{比丘}{31.0}們!\twnr{未聽聞的一般人}{74.0}說這個話:『在大海中有深淵。』比丘們!但那位未聽聞的一般人對這個不實的、不存在的說這樣的話:『在大海中有深淵。』

  比丘們!這是身體諸苦受的同義語,即:『深淵』。

  比丘們!當未聽聞的一般人被身體苦受接觸時,他悲傷、疲累、悲泣、搥胸地號哭,來到迷亂,比丘們!這被稱為:『未聽聞的一般人在深淵中未起來,以及未得到立足處。』

  比丘們!而當有聽聞的聖弟子被身體苦受接觸時,他不悲傷、不疲累、不悲泣、不搥胸地號哭,不來到迷亂,比丘們!這被稱為:『有聽聞的聖弟子在深淵中起來,以及得到立足處。』」

  「凡不忍受這個,諸已生起的苦受:

   身體奪命的,被那個接觸他顫抖。

   悲嘆、哭泣,少力的弱者,

   他在深淵中未起來,又也未得到立足處。

   但凡忍受這個,諸已生起的苦受:

   身體奪命的,被那個接觸他不發抖,

   他確實在深淵中起來,又也得到立足處。」



\sutta{5}{5}{應該被看作經}{https://agama.buddhason.org/SN/sn.php?keyword=36.5}
  「\twnr{比丘}{31.0}們!有這些三受,哪三種?樂受、苦受、不苦不樂受。

  比丘們!樂受應該被視(看作)為苦的,苦受應該被視為箭,不苦不樂受應該被視為無常的。

  比丘們!對比丘來說,當樂受被視為苦的,苦受被視為箭,不苦不樂受被視為無常的,比丘們!這被稱為:『比丘是正確看見者,他切斷渴愛,破壞結,\twnr{從慢的完全止滅}{592.0}\twnr{作苦的終結}{54.0}。』」

  「凡見樂的為苦的,見苦的為箭者,

   寂靜的不苦不樂,見它為無常的,

   那位正確看見的比丘確實,\twnr{遍知}{154.0}受。

   他遍知受後,\twnr{當生}{42.0}成為無\twnr{漏}{188.0}者,

   住法者以身體的崩解,通曉吠陀者\twnr{不來到稱呼}{254.0}。」



\sutta{6}{6}{箭經}{https://agama.buddhason.org/SN/sn.php?keyword=36.6}
  「\twnr{比丘}{31.0}們!\twnr{未聽聞的一般人}{74.0}感受樂受,也感受苦受,也感受不苦不樂受,比丘們!有聽聞的聖弟子也感受樂受,也感受苦受,也感受不苦不樂受。比丘們!在那裡,對有聽聞的聖弟子比未聽聞的一般人,什麼是差別,什麼是不同,什麼是區別?」「\twnr{大德}{45.0}!我們的法以\twnr{世尊}{12.0}為根本……(中略)。」

  「比丘們!當未聽聞的一般人被苦受接觸時,悲傷、疲累、悲泣、搥胸地號哭,來到迷亂,他感受二受:身的與心的。

  比丘們!猶如以箭射中男子,以第二支箭接著貫穿地射中他,比丘們!這樣,那位男子以二支箭感受受。同樣的,比丘們!當未聽聞的一般人被苦受接觸時,悲傷、疲累、悲泣、搥胸地號哭,來到迷亂,他感受二受:身的與心的。又,當被苦受接觸時,對那個就成為有嫌惡者,對那個苦受的有嫌惡者,凡苦受的\twnr{嫌惡煩惱潛在趨勢}{453.0}他\twnr{潛伏}{253.0}。當被苦受接觸時,他歡喜欲之樂,那是什麼原因?比丘們!因為未聽聞的一般人不知道欲之樂以外的苦受的\twnr{出離}{294.0}。而對那位歡喜欲之樂者,凡樂受的\twnr{貪煩惱潛在趨勢}{452.0}他潛伏。他不如實知道那些受的\twnr{集起}{67.0}、滅沒、\twnr{樂味}{295.0}、\twnr{過患}{293.0}、\twnr{出離}{294.0},對那位不如實知道那些受的集起、滅沒、樂味、過患、出離者,凡不苦不樂受的\twnr{無明煩惱潛在趨勢}{454.0}他潛伏。他如果感受樂受,被結縛地感受它;如果感受苦受,被結縛地感受它;如果感受不苦不樂受,被結縛地感受它。比丘們!我說:『這被稱為:被生、老、死,愁、悲、苦、憂、\twnr{絕望}{342.0}結縛;被結縛在苦中之未聽聞的一般人。』

  比丘們!而當有聽聞的聖弟子被苦受接觸時,不悲傷、不疲累、不悲泣、不搥胸地號哭,不來到迷亂,他感受一受:身的,非心的。

  比丘們!猶如以箭射中男子,沒以第二支箭接著貫穿地射中他,比丘們!這樣,那位男子以一支箭感受受。同樣的,比丘們!當有聽聞的聖弟子被苦受接觸時,不悲傷、不疲累、不悲泣、不搥胸地號哭,不來到迷亂,他感受一受:身的,非心的。又,當被苦受接觸時,對那個就不成為有嫌惡者,對那個苦受的非有嫌惡者,凡苦受的嫌惡煩惱潛在趨勢他不潛伏。當被苦受接觸時,他不歡喜欲之樂,那是什麼原因?比丘們!因為有聽聞的聖弟子知道欲之樂以外的苦受的出離。而對那位不歡喜欲之樂者,凡樂受的貪煩惱潛在趨勢他不潛伏。他如實知道那些受的集起、滅沒、樂味、過患、出離,對那位如實知道那些受的集起、滅沒、樂味、過患、出離者,凡不苦不樂受的無明煩惱潛在趨勢他不潛伏。他如果感受樂受,離結縛地感受它;如果感受苦受,離結縛地感受它;如果感受不苦不樂受,離結縛地感受它。比丘們!我說:『這被稱為:離被生、老、死,愁、悲、苦、憂、絕望結縛;離被結縛在苦中之有聽聞的聖弟子。』

  比丘們!對有聽聞的聖弟子比未聽聞的一般人,這是差別,這是不同,這是區別。」

  「有慧者及多聞者不感受,苦及樂受,

   而對善巧的明智者與一般人,這是大差別。

   對悟法者、多聞者,對此世與他世(來世)的\twnr{觀者}{178.0}來說,

   想要的諸法不使心攪亂,不想要的不來到敵意。

   他的\twnr{順適}{x529}又或排斥,已破壞已滅沒不存在,

   以及知道遠塵無愁的足跡後,\twnr{已到達有的彼岸者}{785.0}正確地知道。[\ccchref{AN.8.5}{https://agama.buddhason.org/AN/an.php?keyword=8.5}, \ccchref{AN.8.6}{https://agama.buddhason.org/AN/an.php?keyword=8.6}]」



\sutta{7}{7}{生病經第一}{https://agama.buddhason.org/SN/sn.php?keyword=36.7}
  \twnr{有一次}{2.0},\twnr{世尊}{12.0}住在毘舍離大林\twnr{重閣}{213.0}講堂。那時,世尊傍晚時,從\twnr{獨坐}{92.0}出來,去病房。抵達後,在設置的座位坐下。坐下後,世尊召喚\twnr{比丘}{31.0}們:

  「比丘們!比丘應該具念地、正知地等待死時,這是我們為你們的教誡。

  比丘們!而怎樣比丘是具念者?比丘們!這裡,比丘住於\twnr{在身上隨看著身}{176.0}:熱心的、正知的、有念的,調伏世間中的\twnr{貪婪}{435.0}、憂後;住於在諸受上隨看著受……(中略)住於在心上隨看著心……(中略)住於在諸法上隨看著法:熱心的、正知的、有念的,調伏世間中的貪婪、憂後,比丘們!這樣,比丘是具念者。

  比丘們!而怎樣比丘是正知者?比丘們!這裡,比丘在前進後退時是\twnr{正知的行為者}{544.0};在前視環視時是正知的行為者;在[肢體]屈伸時是正知的行為者;在\twnr{大衣}{270.0}、鉢、衣服的受持時是正知的行為者;在食、飲、嚼、嚐時是正知的行為者;在大小便動作時是正知的行為者;在行、住、坐、臥、清醒、語、默狀態時是正知的行為者,比丘們!這樣,比丘是正知{的行為?}者。比丘們!比丘應該具念地、正知地等待死時,這是我們為你們的教誡。

  比丘們!如果當那位比丘住於這樣具念、正知、不放逸、熱心、自我努力時,樂受生起,他這麼知道:『我有這個已生起的樂受,那有\twnr{緣}{252.0},非無緣,\twnr{緣於}{252.0}什麼?就緣於這個身體,但這個身體是無常的、\twnr{有為的}{90.0}、\twnr{緣所生的}{557.0},而緣於無常的、有為的、緣所生的身體生起的樂受,將從哪裡有常的?』他在身上與在樂受上住於隨看著無常、住於隨看著消散、住於隨看著\twnr{離貪}{77.0}、住於\twnr{隨看著滅}{703.0}、住於\twnr{隨看著斷念}{211.1}。當他在身上與在樂受上住於隨看著無常、住於隨看著消散、住於隨看著離貪、住於隨看著滅、住於隨看著斷念時,凡在身上與在樂受上的\twnr{貪煩惱潛在趨勢}{452.0},那個被捨斷。

  比丘們!如果當那位比丘住於這樣具念、正知、不放逸、熱心、自我努力時,苦受生起,他這麼知道:『我有這個已生起的苦受,那有緣,非無緣,緣於什麼?就緣於這個身體,但這個身體是無常的、有為的、緣所生的,而緣於無常的、有為的、緣所生的身體生起的苦受,將從哪裡有常的?』他在身上與在苦受上住於隨看著無常、住於隨看著消散、住於隨看著離貪、住於隨看著滅、住於隨看著斷念。當他在身上與在苦受上住於隨看著無常……(中略)住於隨看著斷念時,凡在身上與在苦受上的\twnr{嫌惡煩惱潛在趨勢}{453.0},那個被捨斷。

  比丘們!如果當那位比丘住於這樣具念、正知、不放逸、熱心、自我努力時,不苦不樂受生起,他這麼知道:『我有這個已生起的不苦不樂受,那有緣,非無緣,緣於什麼?就緣於這個身體,但這個身體是無常的、有為的、緣所生的,而緣於無常的、有為的、緣所生的身體生起的不苦不樂受,將從哪裡有常的?』他在身上與在不苦不樂受上住於隨看著無常、住於隨看著消散、住於隨看著離貪、住於隨看著滅、住於隨看著斷念。當他在身上與在不苦不樂受上住於隨看著無常……(中略)住於隨看著斷念時,凡在身上與在不苦不樂受的\twnr{無明煩惱潛在趨勢}{454.0},那個被捨斷。他如果感受樂受,知道:『那是無常的。』知道:『是不被固執的。』知道:『是不被歡喜的。』如果感受苦受,知道:『那是無常的。』知道:『是不被固執的。』知道:『是不被歡喜的。』如果感受不苦不樂受,知道:『那是無常的。』知道:『是不被固執的。』知道:『是不被歡喜的。』他如果感受樂受,離被結縛地感受它;他如果感受苦受,離被結縛地感受它;他如果感受不苦不樂受,離被結縛地感受它。他\twnr{當感受身體終了的感受時}{720.0},知道:『我感受身體終了的感受。』\twnr{當感受生命終了的感受時}{721.0},知道:『我感受生命終了的感受。』他知道:『以身體的崩解,隨後生命耗盡,就在這裡,一切所感受的、不被歡喜的將成為清涼[,遺骸被留下(剩下)-\suttaref{SN.12.51}]。』

  比丘們!猶如緣於油與緣於燈芯,油燈燃燒。就從那個油與燈芯的耗盡,無食物者被熄滅。同樣的,比丘們!比丘當感受身體終了的感受時,知道:『我感受身體終了的感受。』或當感受生命終了的感受時,知道:『我感受生命終了的感受。』他知道:『以身體的崩解,隨後生命耗盡,就在這裡,一切所感受的、不被歡喜的將成為清涼[,遺骸被留下(剩下)]。』」



\sutta{8}{8}{生病經第二}{https://agama.buddhason.org/SN/sn.php?keyword=36.8}
  \twnr{有一次}{2.0},\twnr{世尊}{12.0}住在毘舍離大林\twnr{重閣}{213.0}講堂。

  那時,世尊傍晚時,從\twnr{獨坐}{92.0}出來,去病房。抵達後,在設置的座位坐下。坐下後,世尊召喚\twnr{比丘}{31.0}們:

  「比丘們!比丘應該具念地、正知地等待死時,這是我們為你們的教誡。

  比丘們!而怎樣比丘是具念者?比丘們!這裡,比丘住於\twnr{在身上隨看著身}{176.0}:熱心的、正知的、有念的,調伏世間中的\twnr{貪婪}{435.0}、憂後;住於在諸受上隨看著受……住於在心上隨看著心……住於在諸法上隨看著法:熱心的、正知的、有念的,調伏世間中的貪婪、憂後,比丘們!這樣,比丘是具念者。

  比丘們!而怎樣比丘是正知者?比丘們!這裡,比丘在前進後退時是\twnr{正知的行為者}{544.0}……(中略)在語、默狀態時是正知的行為者,比丘們!這樣,比丘是正知者。

  比丘們!比丘應該具念地、正知地等待死時,這是我們為你們的教誡。

  比丘們!如果當那位比丘住於這樣具念、正知、不放逸、熱心、自我努力時,樂受生起,他這麼知道:『我有這個已生起的樂受,那有\twnr{緣}{252.0},非無緣,\twnr{緣於}{252.0}什麼?就緣於這個觸,但這個觸是無常的、\twnr{有為的}{90.0}、\twnr{緣所生的}{557.0},而緣於無常的、有為的、緣所生的觸生起的樂受,將從哪裡有常的?』他在觸上與在樂受上住於隨看著無常、住於隨看著消散、住於隨看著\twnr{離貪}{77.0}、住於\twnr{隨看著滅}{703.0}、住於\twnr{隨看著斷念}{211.1}。當他在觸上與在樂受上住於隨看著無常、住於隨看著消散、住於隨看著離貪、住於隨看著滅、住於隨看著斷念時,凡在觸上與在樂受上的\twnr{貪煩惱潛在趨勢}{452.0},那個被捨斷。

  比丘們!如果當那位比丘住於這樣具念……(中略)苦受生起……(中略)不苦不樂受生起,他這麼知道:『我有這個已生起的不苦不樂受,那有緣,非無緣,緣於什麼?就緣於這個觸……(中略)』……知道:『以身體的崩解,隨後生命耗盡,就在這裡,一切所感受的、不被歡喜的將成為清涼[,遺骸被留下(剩下)-\suttaref{SN.12.51}]。』

  比丘們!猶如緣於油與緣於燈芯,油燈燃燒。就從那個油與燈芯的耗盡,無食物者被熄滅。同樣的,比丘們!比丘當感受身體終了的感受時,知道:『我感受身體終了的感受。』或當感受生命終了的感受時,知道:『我感受生命終了的感受。』他知道:『以身體的崩解,隨後生命耗盡,就在這裡,一切所感受的、不被歡喜的將成為清涼[,遺骸被留下(剩下)]。』」



\sutta{9}{9}{無常經}{https://agama.buddhason.org/SN/sn.php?keyword=36.9}
  「 \twnr{比丘}{31.0}們!這三受是無常的、\twnr{有為的}{90.0}、\twnr{緣所生的}{557.0},是\twnr{滅盡法}{273.1}、\twnr{消散法}{155.0}、\twnr{褪去}{77.0}法、\twnr{滅法}{68.1},哪三個?樂受、苦受、不苦不樂受, 比丘們!這三受是無常的、有為的、緣所生的,是滅盡法、消散法、褪去法、滅法。」



\sutta{10}{10}{觸為根本經}{https://agama.buddhason.org/SN/sn.php?keyword=36.10}
  「\twnr{比丘}{31.0}們!這三受是觸生的,觸為根本,觸為因,觸\twnr{為緣}{180.0},哪三個?樂受、苦受、不苦不樂受。

  比丘們!\twnr{緣於}{252.0}能被感受為樂之\twnr{觸}{407.0},樂受生起,就以那個能被感受為樂之觸的\twnr{滅}{68.0},凡對應那個所感受的:緣於能被感受為樂之觸生起的樂受,它被滅,它被平息。

  比丘們!緣於能被感受為苦之觸,苦受生起,就以那個能被感受為苦之觸的滅,凡對應那個所感受的:緣於能被感受為苦之觸生起的苦受,它被滅,它被平息。

  比丘們!緣於能被感受為不苦不樂之觸,不苦不樂受生起,就以那個能被感受為不苦不樂之觸的滅,凡對應那個所感受的:緣於能被感受為不苦不樂之觸生起的不苦不樂受,它被滅,它被平息。

  比丘們!猶如從兩塊柴的磨擦、結合,熱被產生,火生起。就從那兩塊柴的分離分置,凡對應那個的熱,它被滅,它被平息[\suttaref{SN.12.62}, \suttaref{SN.48.39}]。同樣的,比丘們!這三受是觸生的,觸為根本,觸為因,觸為緣,緣於對應那個之觸,對應那個的諸受生起,以對應那個之觸的滅,對應那個的諸受被滅。」

  有偈品第一,其\twnr{攝頌}{35.0}:

  「定、樂與捨斷,深淵與應該被看作,

   箭與生病,無常、根源於觸。」





\pin{獨處品}{11}{20}
\sutta{11}{11}{獨處經}{https://agama.buddhason.org/SN/sn.php?keyword=36.11}
  那時,\twnr{某位比丘}{39.0}去見世尊。抵達後,向世尊\twnr{問訊}{46.0}後,在一旁坐下。在一旁坐下的那位比丘對世尊說這個:「\twnr{大德}{45.0}!這裡,當我獨處、\twnr{獨坐}{92.0}時,這樣心的深思生起:『三受被世尊說:樂受、苦受、不苦不樂受,這三受被世尊說。又,這被世尊說:「凡任何被感受的,它是在苦中。」關於什麼這被世尊說:「凡任何被感受的,它是在苦中。」呢?』」

  「比丘!\twnr{好}{44.0}!好!比丘!這三受被我說:樂受、苦受、不苦不樂受,這三受被我說。比丘!又,這被我說:『凡任何被感受的,它是在苦中。』比丘!又,那就是關於諸行的無常性,這被我說:『凡任何被感受的,它是在苦中。』比丘!又,那就是關於諸行的\twnr{滅盡法}{273.1}性,這被我說:『凡任何被感受的,它是在苦中。』……(中略)\twnr{消散法}{155.0}性……(中略)\twnr{褪去}{77.0}法性……(中略)\twnr{滅法}{68.1}性……(中略)那就是關於諸行是變易法性,這被我說:『凡任何被感受的,它是在苦中。』比丘!又,諸行的次第滅被我告知:對入初禪者,言語被滅;對入第二禪者,尋伺被滅;對入第三禪者,喜被滅;對入第四禪者,入息出息被滅;對入虛空無邊處者,色想被滅;對入識無邊處者,虛空無邊處想被滅;對入無所有處者,識無邊處想被滅;對入非想非非想處者,無所有處想被滅;對入想受滅者,想與受被滅;對漏已滅盡比丘,貪被滅,瞋被滅,癡被滅。

  比丘!又,諸行的次第平息被我告知:對入初禪者,言語被平息;對入第二禪者,尋伺被平息……(中略)對入想受滅者,想與受被平息;對漏已滅盡比丘,貪被平息,瞋被平息,癡被平息。比丘!有這六種安息(\twnr{寧靜}{313.0}):對入初禪者,言語被安息;對入第二禪者,尋伺被安息;對入第三禪者,喜被安息;對入第四禪者,入息出息被安息;對入想受滅者,想與受被安息;對漏已滅盡比丘,貪被安息,瞋被安息,癡被安息。」



\sutta{12}{12}{虛空經第一}{https://agama.buddhason.org/SN/sn.php?keyword=36.12}
  「\twnr{比丘}{31.0}們!猶如在虛空中種種風吹:東風吹,西風也吹,北風也吹,南風也吹,有塵風也吹,無塵風也吹,冷風也吹,熱風也吹,微風也吹,強風也吹。同樣的,比丘們!在這身體中種種受生起:樂受生起,苦受也生起,不苦不樂受也生起。」

  「正如在虛空中,種種風個個吹起,

   東方的還有西方的,北方的又南方的。

   有塵的還有無塵的,有時冷的與熱的,

   強烈的與微的,個個風吹。

   像這樣就在這身體中,諸受生起,

   樂的苦的生起,以及凡不苦不樂的。

   但當熱心的比丘,不疏忽正知,

   從那裡那位賢智者,\twnr{遍知}{154.0}一切受。

   他遍知受後,\twnr{當生}{42.0}成為無\twnr{漏}{188.0}者,

   住法者以身體的崩解,通曉吠陀者\twnr{不來到稱呼}{254.0}。[\suttaref{SN.36.3}]」



\sutta{13}{13}{虛空經第二}{https://agama.buddhason.org/SN/sn.php?keyword=36.13}
  「\twnr{比丘}{31.0}們!猶如在虛空中種種風吹:東風吹……(中略)強風也吹。同樣的,比丘們!在這身體中種種受生起:樂受生起,苦受也生起,不苦不樂受也生起。」



\sutta{14}{14}{屋舍經}{https://agama.buddhason.org/SN/sn.php?keyword=36.14}
  「\twnr{比丘}{31.0}們!猶如來客之屋舍,在那裡,他們從東方到來後作住所,他們從西方到來後也作住所,他們從北方到來後也作住所,他們從南方到來後也作住所;\twnr{剎帝利}{116.0}們到來後也作住所,婆羅門們到來後也作住所,毘舍們到來後也作住所,首陀羅們到來後也作住所。同樣的,比丘們!在這身體中種種受生起:樂受生起,苦受也生起,不苦不樂受也生起;\twnr{肉體的樂受}{603.0}也生起,肉體的苦受也生起,肉體的不苦不樂受也生起;精神的樂受也生起,精神的苦受也生起,精神的不苦不樂受也生起。」



\sutta{15}{15}{阿難經第一}{https://agama.buddhason.org/SN/sn.php?keyword=36.15}
  那時,\twnr{尊者}{200.0}阿難去見\twnr{世尊}{12.0}。抵達後,在一旁坐下。在一旁坐下的尊者阿難對世尊說這個:

  「\twnr{大德}{45.0}!什麼是受?什麼是受\twnr{集}{67.0}?什麼是受\twnr{滅}{68.0}?什麼是導向受\twnr{滅道跡}{69.0}?什麼是受的\twnr{樂味}{295.0}?什麼是\twnr{過患}{293.0}?什麼是\twnr{出離}{294.0}?」

  「阿難!有這三受:樂受、苦受、不苦不樂受,阿難!這些被稱為受。

  以觸集而有受集;以觸滅有受滅。

  這\twnr{八支聖道}{525.0}就是導向受滅道跡,即:正見……(中略)正定。

  凡\twnr{緣於}{252.0}受,樂、喜悅生起,這是受的樂味。

  凡受是無常的、苦的、變易法,這是受的過患。

  凡在受上意欲貪的調伏、意欲貪的捨斷,這是受的出離。

  阿難!又,諸行的次第滅被我告知:對入初禪者,言語被滅……(中略)對入想受滅者,想與受被滅;對漏已滅盡\twnr{比丘}{31.0},貪被滅,瞋被滅,癡被滅。

  阿難!又,諸行的次第平息被我告知:對入初禪者,言語被平息……(中略)對入想受滅者,想與受被平息;對漏盡比丘,貪被平息,瞋被平息,癡被平息。

  阿難!又,諸行次第安息(\twnr{寧靜}{313.0})被我告知:對入初禪者,言語被安息……(中略)對入虛空無邊處者,色想被安息;對入識無邊處者,虛空無邊處想被安息;對入無所有處者,識無邊處想被安息;對入非想非非想處者,無所有處想被安息;對入想受滅者,想與受被安息;對漏已滅盡比丘,貪被安息,瞋被安息,癡被安息。」[\suttaref{SN.36.17}]



\sutta{16}{16}{阿難經第二}{https://agama.buddhason.org/SN/sn.php?keyword=36.16}
  那時,\twnr{尊者}{200.0}阿難去見\twnr{世尊}{12.0}。抵達後,向世尊\twnr{問訊}{46.0}後,在一旁坐下。世尊對在一旁坐下的尊者阿難說這個:

  「阿難!什麼是受?什麼是受\twnr{集}{67.0}?什麼是受\twnr{滅}{68.0}?什麼是導向受\twnr{滅道跡}{69.0}?什麼是受的\twnr{樂味}{295.0}?什麼是\twnr{過患}{293.0}?什麼是\twnr{出離}{294.0}?」

  「\twnr{大德}{45.0}!我們的法是世尊為根本的、\twnr{世尊為導引的}{56.0}、世尊為依歸的,大德!就請世尊說明這個所說的義理,那就好了!聽聞世尊的[教說]後,\twnr{比丘}{31.0}們將會\twnr{憶持}{57.0}。」

  「阿難!那樣的話,你要聽!你要\twnr{好好作意}{43.1}!我將說。」 

  「是的,大德!」尊者阿難回答世尊。 

  世尊說這個:

  「阿難!有這三受:樂受、苦受、不苦不樂受,阿難!這些被稱為受。

  ……(中略)觸集……對漏已滅盡比丘,貪被安息,瞋被安息,癡被安息。」



\sutta{17}{17}{眾多經第一}{https://agama.buddhason.org/SN/sn.php?keyword=36.17}
  那時,眾多\twnr{比丘}{31.0}去見世尊。抵達後,向世尊\twnr{問訊}{46.0}後,在一旁坐下。在一旁坐下的眾多比丘對世尊說這個:

  「\twnr{大德}{45.0}!什麼是受?什麼是受\twnr{集}{67.0}?什麼是受\twnr{滅}{68.0}?什麼是導向受\twnr{滅道跡}{69.0}?什麼是受的\twnr{樂味}{295.0}?什麼是\twnr{過患}{293.0}?什麼是\twnr{出離}{294.0}?」

  「比丘們!有這三受:樂受、苦受、不苦不樂受,比丘們!這些被稱為受。

  以觸集而有受集;以觸滅有受滅。

  這\twnr{八支聖道}{525.0}就是導向受滅道跡,即:正見……(中略)正定。

  凡\twnr{緣於}{252.0}受,樂、喜悅生起,這是受的樂味。

  凡受是無常的、苦的、變易法,這是受的過患。

  凡在受上意欲貪的調伏、意欲貪的捨斷,這是受的出離。

  比丘們!又,諸行的次第滅被我告知:對入初禪者,言語被滅……(中略)對入想受滅者,想與受被滅;對漏已滅盡比丘,貪被滅,瞋被滅,癡被滅。

  比丘們!又,諸行的次第平息被我告知:對入初禪者,言語被平息……(中略)對入想受滅者,想與受被平息;對漏已滅盡比丘,貪被平息,瞋被平息,癡被平息。

  比丘們!又,諸行次第安息(\twnr{寧靜}{313.0})被我告知:對入初禪者,言語被安息……(中略)對入虛空無邊處者,色想被安息;對入識無邊處者,虛空無邊處想被安息;對入無所有處者,識無邊處想被安息;對入非想非非想處者,無所有處想被安息;對入想受滅者,想與受被安息;對漏已滅盡比丘,貪被安息,瞋被安息,癡被安息。[\suttaref{SN.36.15}]」



\sutta{18}{18}{眾多經第二}{https://agama.buddhason.org/SN/sn.php?keyword=36.18}
  那時,眾多\twnr{比丘}{31.0}去見\twnr{世尊}{12.0}。……(中略)世尊對在一旁坐下的那些比丘說這個:

  「比丘們!而什麼是受?什麼是受\twnr{集}{67.0}?什麼是受\twnr{滅}{68.0}?什麼是導向受\twnr{滅道跡}{69.0}?什麼是受的\twnr{樂味}{295.0}?什麼是\twnr{過患}{293.0}?什麼是\twnr{出離}{294.0}?」

  「\twnr{大德}{45.0}!我們的法以世尊為根本……。」……(中略)

  「比丘們!有這三受:樂受、苦受、不苦不樂受,比丘們!這些被稱為受。

  ……(中略)觸集……(中略)。」(應該依前經那樣使之被細說) 



\sutta{19}{19}{五支經}{https://agama.buddhason.org/SN/sn.php?keyword=36.19}
  那時,木匠五支去見\twnr{尊者}{200.0}優陀夷。抵達後,向尊者優陀夷\twnr{問訊}{46.0}後,在一旁坐下。在一旁坐下的木匠五支對尊者優陀夷說這個:

  「\twnr{大德}{45.0}優陀夷!多少受被\twnr{世尊}{12.0}說?」

  「木匠!三受被世尊說:樂受、苦受、不苦不樂受,木匠!這三受被世尊說。」

  在這麼說時,木匠五支對尊者優陀夷說這個:

  「大德優陀夷!三受不被世尊說,二受被世尊說:樂受、苦受,大德!凡這不苦不樂受者,這被世尊說這在寂靜的勝妙樂中。」

  第二次,尊者優陀夷又對木匠五支說這個:

  「木匠!二受不被世尊說,三受被世尊說:樂受、苦受、不苦不樂受,這三受被世尊說。」

  第二次,木匠五支又對尊者優陀夷說這個:

  「大德優陀夷!三受不被世尊說,二受被世尊說:樂受、苦受,大德!凡這不苦不樂受者,這被世尊說這在寂靜的勝妙樂中。」

  第三次,尊者優陀夷又對木匠五支說這個:

  「木匠!二受不被世尊說,三受被世尊說:樂受、苦受、不苦不樂受,這三受被世尊說。」

  第三次,木匠五支又對尊者優陀夷說這個:

  「大德優陀夷!三受不被世尊說,二受被世尊說:樂受、苦受,大德!凡這不苦不樂受者,這被世尊說這在寂靜的勝妙樂中。」

  既非尊者優陀夷能夠說服木匠五支,也非木匠五支能夠說服尊者優陀夷。

  尊者阿難聽到尊者優陀夷與木匠五支一起的這個交談。

  那時,尊者阿難去見世尊。抵達後,在一旁坐下。在一旁坐下的尊者阿難告訴世尊尊者優陀夷與木匠五支一起有交談之所及的那一切。

  「阿難!正存在著\twnr{法門}{562.0}木匠五支不非常隨喜優陀夷\twnr{比丘}{31.0},阿難!而且也存在著法門優陀夷比丘不非常隨喜木匠五支。

  阿難!二受以法門被我說,三受也以法門被我說,\twnr{五受}{x530}也以法門被我說,六受也以法門被我說,十八受也以法門被我說,三十六受也以法門被我說,\twnr{一百零八受}{x531}也以法門被我說。

  阿難!法被我這樣以法門教導;阿難!在法被我這樣以法門教導時,凡將不贊同、不認可、不同意彼此的善說、善講述者,他們的這個能被預期:他們將住於生起爭論的、生起爭吵的、來到爭辯的、\twnr{以舌鋒互刺的}{925.0}。

  阿難!法被我這樣以法門教導;阿難!在法被我這樣以法門教導時,凡將贊同、認可、同意彼此的善說、善講述者,他們的這個能被預期:他們將住於和合的、和好的、無爭的、水乳交融的、彼此以親切眼睛互看的。

  阿難!有這\twnr{五種欲}{187.0},哪五種?能被眼識知的、想要的、所愛的、合意的、可愛形色的、伴隨欲的、誘人的諸色……(中略)能被身識知的、想要的、所愛的、合意的、可愛形色的、伴隨欲的、誘人的諸\twnr{所觸}{220.2},阿難!這些是五種欲。阿難!凡\twnr{緣於}{252.0}這五種欲,樂、喜悅生起,這被稱為\twnr{欲樂}{x532}。

  阿難!凡如果這麼說:『{存在者}[眾生們]感受樂、喜悅,這是最高的。』這個我不認可這個(我不認可他的這個-\ccchref{MN.59}{https://agama.buddhason.org/MN/dm.php?keyword=59}),那是什麼原因?阿難!有比這種樂更勝與更妙的其它樂。阿難!而什麼是比這種樂更勝與更妙的其它樂呢?阿難!這裡,比丘就從離諸欲後,從離諸不善法後,\twnr{進入後住於}{66.0}有尋、\twnr{有伺}{175.0},\twnr{離而生喜、樂}{174.0}的初禪,阿難!這是比這種樂更勝與更妙的其它樂。

  阿難!凡如果這麼說:『眾生們感受樂、喜悅,這是最高的。』這個我不認可這個,那是什麼原因?阿難!有比那種樂更勝與更妙的其它樂。阿難!而什麼是比這種樂更勝與更妙的其它樂呢?阿難!這裡,比丘從尋與伺的平息,\twnr{自身內的明淨}{256.0},\twnr{心的專一性}{255.0},進入後住於無尋、無伺,定而生喜、樂的第二禪,阿難!這是比那種樂更勝與更妙的其它樂。

  阿難!凡如果這麼說:『眾生們感受樂、喜悅,這是最高的。』這個我不認可這個,那是什麼原因?阿難!有比那種樂更勝與更妙的其它樂。阿難!而什麼是比這種樂更勝與更妙的其它樂呢?阿難!這裡,比丘從喜的\twnr{褪去}{77.0}、住於\twnr{平靜}{228.0}、有念正知、以身體感受樂,進入後住於凡聖者們告知『他是平靜者、具念者、\twnr{安樂住者}{317.0}』的第三禪,阿難!這是比那種樂更勝與更妙的其它樂。

  阿難!凡如果這麼說:『眾生們感受樂、喜悅,這是最高的。』這個我不認可這個,那是什麼原因?阿難!有比那種樂更勝與更妙的其它樂。阿難!而什麼是比這種樂更勝與更妙的其它樂呢?阿難!這裡,比丘從樂的捨斷與從苦的捨斷,就在之前諸喜悅、憂的滅沒,進入後住於不苦不樂,\twnr{平靜、念遍純淨}{494.0}的第四禪,阿難!這是比那種樂更勝與更妙的其它樂。

  阿難!凡如果這麼說:『眾生們感受樂、喜悅,這是最高的。』這個我不認可這個,那是什麼原因?阿難!有比那種樂更勝與更妙的其它樂。阿難!而什麼是比這種樂更勝與更妙的其它樂呢?阿難!這裡,比丘\twnr{從一切色想的超越}{490.0},從\twnr{有對想}{331.0}的滅沒,從不作意種種想[而知]:『虛空是無邊的』,進入後住於虛空無邊處,阿難!這是比那種樂更勝與更妙的其它樂。

  阿難!凡如果這麼說:『眾生們感受樂、喜悅,這是最高的。』這個我不認可這個,那是什麼原因?阿難!有比那種樂更勝與更妙的其它樂。阿難!而什麼是比這種樂更勝與更妙的其它樂呢?阿難!這裡,比丘超越一切虛空無邊處後[而知]:『識是無邊的』,進入後住於識無邊處,阿難!這是比那種樂更勝與更妙的其它樂。

  阿難!凡如果這麼說:『眾生們感受樂、喜悅,這是最高的。』這個我不認可這個,那是什麼原因?阿難!有比那種樂更勝與更妙的其它樂。阿難!而什麼是比這種樂更勝與更妙的其它樂呢?阿難!這裡,比丘超越一切識無邊處後[而知]:『什麼都沒有』,進入後住於無所有處,阿難!這是比那種樂更勝與更妙的其它樂。

  阿難!凡如果這麼說:『眾生們感受樂、喜悅,這是最高的。』這個我不認可這個,那是什麼原因?阿難!有比那種樂更勝與更妙的其它樂。阿難!而什麼是比這種樂更勝與更妙的其它樂呢?阿難!這裡,比丘超越一切無所有處後,進入後住於非想非非想處,阿難!這是比那種樂更勝與更妙的其它樂。

  阿難!凡如果這麼說:『眾生們感受樂、喜悅,這是最高的。』這個我不認可這個,那是什麼原因?阿難!有比那種樂更勝與更妙的其它樂。

  阿難!而什麼是比這種樂更勝與更妙的其它樂呢?阿難!這裡,比丘超越一切非想非非想處後,進入後住於\twnr{想受滅}{416.0},阿難!這是比那種樂更勝與更妙的其它樂。

  阿難!又,這存在可能性,凡\twnr{其他外道遊行者}{79.0}們這麼說:『\twnr{沙門}{29.0}\twnr{喬達摩}{80.0}說想受滅,但\twnr{安立}{143.0}它在樂中:這是什麼?這是為什麼?』

  阿難!這麼說的其他外道遊行者們應該被這麼回答:『\twnr{道友}{201.0}!世尊不只安立與樂受有關的在樂中,道友!而是不論哪裡樂被得到,\twnr{不論什麽情況}{x533},\twnr{如來}{4.0}就安立它在樂中。』」[\ccchref{MN.59}{https://agama.buddhason.org/MN/dm.php?keyword=59}]



\sutta{20}{20}{比丘經}{https://agama.buddhason.org/SN/sn.php?keyword=36.20}
  「比丘們!二受以\twnr{法門}{562.0}被我說,三受也以法門被我說,五受也以法門被我說,六受也以法門被我說,十八受也以法門被我說,三十六受也以法門被我說,一百零八受也以法門被我說。

  比丘們!法被我這樣以法門教導;比丘們!在法被我這樣以法門教導時,凡將不贊同、不認可、不同意彼此的善說、善講述者,他們的這個能被預期:他們將住於生起爭論的、生起爭吵的、來到爭辯的、\twnr{以舌鋒互刺的}{925.0}。

  比丘們!法被我這樣以法門教導;比丘們!在法被我這樣以法門教導時,凡將贊同、認可、同意彼此的善說、善講述者,他們的這個能被預期:他們將住於和合的、和好的、無爭的、水乳交融的、彼此以親切眼睛互看的。

  比丘們!有這\twnr{五種欲}{187.0}……(中略)比丘們!又,這存在可能性,凡\twnr{其他外道遊行者}{79.0}們這麼說:『\twnr{沙門}{29.0}\twnr{喬達摩}{80.0}說\twnr{想受滅}{416.0},但\twnr{安立}{143.0}它在樂中:這是什麼?這是為什麼?』

  比丘們!這麼說的其他外道遊行者們應該被這麼回答:『\twnr{道友}{201.0}!\twnr{世尊}{12.0}不只安立與樂受有關的在樂中,道友!哪裡樂被發現,不論什麽情況,\twnr{如來}{4.0}就安立它在樂中。』」

  獨處品第二,其\twnr{攝頌}{35.0}:

  「獨處、二則虛空,屋舍與二則阿難,

   眾多二說,五支與比丘。」





\pin{一百零八法門品}{21}{31}
\sutta{21}{21}{尸婆經}{https://agama.buddhason.org/SN/sn.php?keyword=36.21}
  \twnr{有一次}{2.0},\twnr{世尊}{12.0}住在王舍城栗鼠飼養處的竹林中。

  那時,\twnr{遊行者}{79.0}髻髮尸婆去見世尊。抵達後,與世尊一起互相問候。交換應該被互相問候的友好交談後,就在一旁坐下。在一旁坐下的遊行者髻髮尸婆對世尊說這個:

  「\twnr{喬達摩}{80.0}尊師!有一些這樣說、這樣見的\twnr{沙門}{29.0}\twnr{婆羅門}{29.0}:『凡這位男子個人感受任何樂、或苦、或不苦不樂,\twnr{那全部是過去所作之因}{608.0}。』這裡,喬達摩尊師怎麼說?」

  「尸婆!這裡,某些膽汁引起的感受生起,尸婆!這也能被自己知道:如是,這裡,某些膽汁引起的感受生起。尸婆!這也是世間認定的真理:如是,這裡,某些膽汁引起的感受生起。尸婆!在那裡,凡那些這樣說、這樣見的沙門婆羅門:『凡這位男子個人感受任何樂、或苦、或不苦不樂,那全部是過去所作之因。』他們越過自己知道的,以及他們越過世間中認定的真理,因此,我說:『那些沙門婆羅門的[說法]是錯的。』

  尸婆!痰引起的……也……(中略)尸婆!風引起的……也……(中略)尸婆!\twnr{[三者]集合……也}{x534}……(中略)尸婆!時節變化生的……也……(中略)尸婆!\twnr{不正注意生的}{x535}……也……(中略)尸婆!突然來襲的……也……(中略)尸婆!業果報生的,這裡,某些感受也生起。尸婆!這也能被自己感受:關於業果報生的,這裡,某些感受也生起,尸婆!這也是世間認定的真理:關於業果報生的,這裡,某些感受也生起。尸婆!那裡,凡那些這樣說、這樣見的沙門婆羅門:『凡這位男子個人感受任何樂、或苦、或不苦不樂,那全部是過去所作之因。』他們越過自己知道的,以及他們越過世間中認定的真理,因此,我說:『那些沙門婆羅門的[說法]是錯的。』」

  在這麼說時,遊行者髻髮尸婆對世尊說這個:

  「太偉大了,喬達摩尊師!太偉大了,喬達摩尊師!……(中略)請喬達摩\twnr{尊師}{203.0}記得我為\twnr{優婆塞}{98.0},從今天起\twnr{已終生歸依}{64.0}。」

  「膽、痰與風,[三者]集合、時節,

   不正、突然來襲的、以業果報為第八。」



\sutta{22}{22}{一百零八經}{https://agama.buddhason.org/SN/sn.php?keyword=36.22}
  「\twnr{比丘}{31.0}們!我將為你們教導一百零八理趣\twnr{法的教說}{562.1},\twnr{你們要聽}{43.0}它!

  比丘們!而什麼是一百零八理趣法的教說呢?

  比丘們!二受因理趣(\twnr{法門}{562.0})被我說,三受因理趣也被我說,五受因理趣也被我說,六受因理趣也被我說,十八受因理趣也被我說,三十六受因理趣也被我說,一百零八受因理趣也被我說。

  比丘們!而什麼是二受呢?身的與心的,比丘們!這些被稱為二受。

  比丘們!而什麼是三受呢?樂受、苦受、不苦不樂受,比丘們!這些被稱為三受。

  比丘們!而什麼是五受呢?樂根、苦根、喜悅根、憂根、\twnr{平靜根}{228.1},比丘們!這些被稱為五受。

  比丘們!而什麼是六受呢?眼觸所生受……(中略)意觸所生受,比丘們!這些被稱為六受。

  比丘們!而什麼是十八受呢?六\twnr{喜悅近伺察}{670.0}、六\twnr{憂近伺察}{671.0}、六\twnr{平靜近伺察}{672.0},比丘們!這些被稱為十八受。

  比丘們!而什麼是三十六受呢?六\twnr{依存於家的}{825.0}喜悅、六依存於離欲的喜悅、六依存於家的憂、六依存於離欲的憂、六依存於家的平靜、六依存於離欲的平靜,比丘們!這些被稱為三十六受。

  比丘們!而什麼是一百零八受呢?過去的三十六受、未來的三十六受、現在的三十六受,比丘們!這些被稱為一百零八受。比丘們!這是一百零八理趣法的教說。」



\sutta{23}{23}{某位比丘經}{https://agama.buddhason.org/SN/sn.php?keyword=36.23}
  那時,\twnr{某位}{39.0}比丘去見世尊。抵達後,向世尊\twnr{問訊}{46.0}後,在一旁坐下。在一旁坐下的那位比丘對世尊說這個:

  「\twnr{大德}{45.0}!什麼是受?什麼是受\twnr{集}{67.0}?什麼是導向受集道跡?什麼是受\twnr{滅}{68.0}?什麼是導向受\twnr{滅道跡}{69.0}?什麼是受的\twnr{樂味}{295.0}?什麼是\twnr{過患}{293.0}?什麼是\twnr{出離}{294.0}?」

  「比丘!有這三受:樂受、苦受、不苦不樂受,比丘!這些被稱為受。

  以觸集而有受集;渴愛是導向受集道跡。

  以觸滅有受滅。

  這\twnr{八支聖道}{525.0}就是導向受滅道跡,即:正見……(中略)正定。

  凡\twnr{緣於}{252.0}受,樂、喜悅生起,這是受的樂味。

  凡受是無常的、苦的、變易法,這是受的過患。

  凡在受上意欲貪的調伏、意欲貪的捨斷,這是受的出離。」



\sutta{24}{24}{以前經}{https://agama.buddhason.org/SN/sn.php?keyword=36.24}
  「\twnr{比丘}{31.0}們!當就在我\twnr{正覺}{185.1}以前,還是未\twnr{現正覺}{75.0}的\twnr{菩薩}{186.0}時想這個:『什麼是受?什麼是受\twnr{集}{67.0}?什麼是導向受集道跡?什麼是受\twnr{滅}{68.0}?什麼是導向受\twnr{滅道跡}{69.0}?什麼是受的\twnr{樂味}{295.0}?什麼是\twnr{過患}{293.0}?什麼是\twnr{出離}{294.0}?』

  比丘們!那個我想這個:『有這三受:樂受、苦受、不苦不樂受,這些被稱為受。

  以觸集而有受集;渴愛是導向受集道跡。……(中略)

  凡在受上意欲貪的調伏、意欲貪的捨斷,這是受的出離。』」



\sutta{25}{25}{智經}{https://agama.buddhason.org/SN/sn.php?keyword=36.25}
  「『這些是受。』\twnr{比丘}{31.0}們!在以前不曾聽聞的諸法上,我的眼生起,智生起,慧生起,明生起,\twnr{光生起}{511.0}。『這是受\twnr{集}{67.0}。』比丘們!在以前不曾聽聞的諸法上,我的眼生起……(中略)光生起。『這是導向受集道跡。』比丘們!在以前不曾聽聞的諸法上,我的眼生起……(中略)。『這是受\twnr{滅}{68.0}。』比丘們!在以前不曾聽聞的諸法上,我的眼生起……(中略)。『這是導向受\twnr{滅道跡}{69.0}。』比丘們!在以前不曾聽聞的諸法上,我的眼生起……(中略)。『這是受的\twnr{樂味}{295.0}。』比丘們!在以前不曾聽聞的諸法上,我的眼生起……(中略)。『這是受的\twnr{過患}{293.0}。』比丘們!在以前不曾聽聞的諸法上,我的眼生起……(中略)。『這是受的\twnr{出離}{294.0}。』比丘們!在以前不曾聽聞的諸法上,我的眼生起,智生起,慧生起,明生起,光生起。」



\sutta{26}{26}{眾多比丘經}{https://agama.buddhason.org/SN/sn.php?keyword=36.26}
  那時,眾多\twnr{比丘}{31.0}去見\twnr{世尊}{12.0}。抵達後……(中略)在一旁坐下的那些比丘對世尊說這個:

  「\twnr{大德}{45.0}!什麼是受?什麼是受\twnr{集}{67.0}?什麼是導向受集道跡?什麼是受\twnr{滅}{68.0}?什麼是導向受\twnr{滅道跡}{69.0}?什麼是受的\twnr{樂味}{295.0}?什麼是\twnr{過患}{293.0}?什麼是\twnr{出離}{294.0}?」

  「比丘們!有這三受:樂受、苦受、不苦不樂受,比丘們!這些被稱為受。

  以觸集而有受集;渴愛是導向受集道跡。……(中略)凡在受上意欲貪的調伏、意欲貪的捨斷,這是受的出離。」



\sutta{27}{27}{沙門婆羅門經第一}{https://agama.buddhason.org/SN/sn.php?keyword=36.27}
  「\twnr{比丘}{31.0}們!有這些三受,哪三個?樂受、苦受、不苦不樂受。

  比丘們!凡任何\twnr{沙門}{29.0}或\twnr{婆羅門}{17.0}不如實知道這三受的\twnr{集起}{67.0}、滅沒、\twnr{樂味}{295.0}、\twnr{過患}{293.0}、\twnr{出離}{294.0}者,比丘們!那些沙門或婆羅門不被我認同為\twnr{沙門中的沙門}{560.0},或婆羅門中的婆羅門,而且,那些\twnr{尊者}{200.0}也不以證智自作證後,在當生中\twnr{進入後住於}{66.0}\twnr{沙門義}{327.0}或婆羅門義。

  比丘們!而凡任何沙門或婆羅門如實知道這三受的集起、滅沒、樂味、過患、出離者,比丘們!那些沙門或婆羅門被我認同為沙門中的沙門,或婆羅門中的婆羅門,而且,那些尊者也以證智自作證後,在當生中進入後住於沙門義或婆羅門義。」



\sutta{28}{28}{沙門婆羅門經第二}{https://agama.buddhason.org/SN/sn.php?keyword=36.28}
  「\twnr{比丘}{31.0}們!有這些三受,哪三個?樂受、苦受、不苦不樂受。

  比丘們!凡任何\twnr{沙門}{29.0}或\twnr{婆羅門}{17.0}不如實知道這三受的\twnr{集起}{67.0}、滅沒、\twnr{樂味}{295.0}、\twnr{過患}{293.0}、\twnr{出離}{294.0}者……(中略)。[如實]了知……(中略)以證智自作證後,在當生中\twnr{進入後住於}{66.0}\twnr{沙門義}{327.0}或婆羅門義。」



\sutta{29}{29}{沙門婆羅門經第三}{https://agama.buddhason.org/SN/sn.php?keyword=36.29}
  「\twnr{比丘}{31.0}們!凡任何\twnr{沙門}{29.0}或\twnr{婆羅門}{17.0}不如實知道受,不如實知道受集,不如實知道受滅,不如實知道導向受\twnr{滅道跡}{69.0}者……(中略)。[如實]知道……(中略)以證智自作證後,在當生中\twnr{進入後住於}{66.0}\twnr{沙門義}{327.0}或婆羅門義。」



\sutta{30}{30}{概要經}{https://agama.buddhason.org/SN/sn.php?keyword=36.30}
  「\twnr{比丘}{31.0}們!有這三受。哪三個?樂受、苦受、不苦不樂受。比丘們!這些是三受。」



\sutta{31}{31}{精神的經}{https://agama.buddhason.org/SN/sn.php?keyword=36.31}
  「\twnr{比丘}{31.0}們!有\twnr{肉體的喜}{x536},有\twnr{精神的喜}{x537},有\twnr{比精神更精神的喜}{x538};有肉體的樂,有精神的樂,有比精神更精神的樂;有肉體的\twnr{平靜}{228.0},有精神的平靜,有比精神更精神的平靜;有肉體的解脫,有精神的解脫,有比精神更精神的解脫。

  比丘們!而哪個是肉體的喜?比丘們!有這\twnr{五種欲}{187.0},哪五種?能被眼識知的、想要的、所愛的、合意的、可愛形色的、伴隨欲的、誘人的諸色……(中略)能被身識知的、想要的、所愛的、合意的、可愛形色的、伴隨欲的、誘人的諸\twnr{所觸}{220.2},比丘們!這些是五種欲。比丘們!凡\twnr{緣於}{252.0}這五種欲,喜生起,比丘們!這被稱為肉體的喜。

  比丘們!而哪個是精神的喜?比丘們!這裡,比丘就從離諸欲後,從離諸不善法後,\twnr{進入後住於}{66.0}有尋、\twnr{有伺}{175.0},\twnr{離而生喜、樂}{174.0}的初禪;從尋與伺的平息,\twnr{自身內的明淨}{256.0},\twnr{心的專一性}{255.0},進入後住於無尋、無伺,定而生喜、樂的第二禪,比丘們!這被稱為精神的喜。

  比丘們!而哪個是比精神更精神的喜?比丘們!這裡,凡當漏已滅盡比丘觀察從貪解脫的心,觀察從瞋解脫的心,觀察從癡解脫的心時,喜生起,比丘們!這被稱為比精神更精神的喜。

  比丘們!而哪個是肉體的樂?比丘們!有這五種欲,哪五種?能被眼識知的、想要的、所愛的、合意的、可愛形色的、伴隨欲的、誘人的諸色……(中略)能被身識知的、想要的、所愛的、合意的、可愛形色的、伴隨欲的、誘人的諸所觸,比丘們!這些是五種欲。比丘們!凡緣於這五種欲,樂、喜悅生起,比丘們!這被稱為肉體的樂。

  比丘們!而哪個是精神的樂?比丘們!這裡,比丘就從離諸欲後,從離諸不善法後,進入後住於有尋、有伺,離而生喜、樂的初禪;從尋與伺的平息,自身內的明淨,心的專一性,進入後住於無尋、無伺,定而生喜、樂的第二禪;從喜的\twnr{褪去}{77.0}、住於\twnr{平靜}{228.0}、有念正知、以身體感受樂,進入後住於凡聖者們告知『他是平靜者、具念者、\twnr{安樂住者}{317.0}』的第三禪,比丘們!這被稱為精神的樂。

  比丘們!而哪個是比精神更精神的樂?比丘們!這裡,凡當漏已滅盡比丘觀察從貪解脫的心,觀察從瞋解脫的心,觀察從癡解脫的心時,樂、喜悅生起,比丘們!這被稱為比精神更精神的樂。

  比丘們!而哪個是\twnr{肉體的平靜}{x539}?比丘們!有這五種欲,哪五種?能被眼識知的、想要的、所愛的、合意的、可愛形色的、伴隨欲的、誘人的諸色……(中略)能被身識知的、想要的、所愛的、合意的、可愛形色的、伴隨欲的、誘人的諸所觸,比丘們!這些是五種欲,比丘們!凡緣於這五種欲,平靜生起,比丘們!這被稱為肉體的平靜。

  比丘們!而哪個是精神的平靜?比丘們!這裡,比丘從樂的捨斷與從苦的捨斷,就在之前諸喜悅、憂的滅沒,進入後住於不苦不樂,\twnr{平靜、念遍純淨}{494.0}的第四禪,比丘們!這被稱為精神的平靜。

  比丘們!而哪個是比精神更精神的平靜?比丘們!這裡,凡當漏已滅盡比丘觀察從貪解脫的心,觀察從瞋解脫的心,觀察從癡解脫的心時,平靜生起,比丘們!這被稱為比精神更精神的平靜。

  比丘們!而哪個是肉體的解脫?\twnr{色關聯的}{x540}解脫為肉體的解脫。

  比丘們!而哪個是精神的解脫?無色關聯的解脫為精神的解脫。

  比丘們!而哪個是比精神更精神的解脫?比丘們!這裡,凡當漏已滅盡比丘觀察從貪解脫的心,觀察從瞋解脫的心,觀察從癡解脫的心時,解脫生起,比丘們!這被稱為比精神更精神的解脫。」

  一百零八法門品第三,其\twnr{攝頌}{35.0}:

  「尸婆、一百零八、比丘,以前、智與比丘們,

   沙門婆羅門三則,概要、精神的。」

  受相應完成。





\page

\xiangying{37}{婦女相應}
\pin{第一中略品}{1}{14}
\sutta{1}{1}{婦女經}{https://agama.buddhason.org/SN/sn.php?keyword=37.1}
  「\twnr{比丘}{31.0}們!具備五支的婦女對男子來說是\twnr{一向}{168.0}不合意者,哪五個?她是不美麗者、是不富有者、是非有德行者、是懶惰者、不為他得子孫者,比丘們!具備五支的婦女對男子來說是一向不合意的。

  比丘們!具備五支的婦女對男子來說是一向合意的,哪五個?她是美麗者、是富有者、是有德行者、是不懶惰者、為他得子孫者,比丘們!具備五支的婦女對男子來說是一向合意者。」



\sutta{2}{2}{男子經}{https://agama.buddhason.org/SN/sn.php?keyword=37.2}
  「\twnr{比丘}{31.0}們!具備五支的男子對婦女來說是\twnr{一向}{168.0}不合意者,哪五個?是不英俊者、是不富有者、是非有德行者、懶惰者、不為她得子孫者,比丘們!具備五支的男子對婦女來說是一向不合意者。

  比丘們!具備五支的男子對婦女來說是一向合意者,哪五個?是英俊者、是富有者、是有德行者、是不懶惰者、為她得子孫者,比丘們!具備五支的男子對婦女來說是一向合意者。」



\sutta{3}{3}{特有的苦經}{https://agama.buddhason.org/SN/sn.php?keyword=37.3}
  「\twnr{比丘}{31.0}們!有這五種婦女特有的苦:只男子以外的婦女經驗,哪五種?

  比丘們!這裡,當年幼時婦女就到夫家,與親族別離,比丘們!這是第一種婦女特有的苦,只男子以外的婦女經驗。

  再者,比丘們!這裡,婦女有月經,比丘們!這是第二種婦女特有的苦,只男子以外的婦女經驗。

  再者,比丘們!這裡,婦女為懷胎者,比丘們!這是第三種婦女特有的苦,只男子以外的婦女經驗。

  再者,比丘們!這裡,婦女生產,比丘們!這是第四種婦女特有的苦,只男子以外的婦女經驗。

  再者,比丘們!這裡,婦女來到對男子的服侍,比丘們!這是第五種婦女特有的苦,只男子以外的婦女經驗。

  比丘們!這五種是婦女特有的苦,只男子以外的婦女經驗。」



\sutta{4}{4}{三法經}{https://agama.buddhason.org/SN/sn.php?keyword=37.4}
  「\twnr{比丘}{31.0}們!具備三法的婦女大多數以身體的崩解,死後往生\twnr{苦界}{109.0}、\twnr{惡趣}{110.0}、\twnr{下界}{111.0}、地獄,哪三個?比丘們!這裡,婦女午前時以被慳吝垢纏縛之心住於在家,日中時以被嫉妒纏縛之心住於在家,傍晚時以被欲貪纏縛之心住於在家,比丘們!具備這三法的婦女大多數以身體的崩解,死後往生苦界、惡趣、下界、地獄。」[\ccchref{AN.3.130}{https://agama.buddhason.org/AN/an.php?keyword=3.130}]



\sutta{5}{5}{易憤怒者經}{https://agama.buddhason.org/SN/sn.php?keyword=37.5}
  那時,\twnr{尊者}{200.0}阿那律去見\twnr{世尊}{12.0}。抵達後,在一旁坐下。在一旁坐下的尊者阿那律對世尊說這個:

  「\twnr{大德}{45.0}!這裡,我以清淨、超越常人的天眼,看見婦女以身體的崩解,死後是\twnr{苦界}{109.0}、\twnr{惡趣}{110.0}、\twnr{下界}{111.0}、地獄往生者,大德!具備幾法的婦女,以身體的崩解,死後往生苦界、惡趣、下界、地獄呢?」[\ccchref{AN.3.130}{https://agama.buddhason.org/AN/an.php?keyword=3.130}]

  「阿那律!具備五法的婦女,以身體的崩解,死後往生苦界、惡趣、下界、地獄,哪五個?她是無信者、是無\twnr{慚}{250.0}者、是無\twnr{愧}{251.0}者、是易憤怒者、是劣慧者,阿那律!具備這五法的婦女,以身體的崩解,死後往生苦界、惡趣、下界、地獄。」



\sutta{6}{6}{懷怨恨者經}{https://agama.buddhason.org/SN/sn.php?keyword=37.6}
  「阿那律!具備五法的婦女,以身體的崩解,死後往生\twnr{苦界}{109.0}、\twnr{惡趣}{110.0}、\twnr{下界}{111.0}、地獄,哪五個?她是無信者、是無\twnr{慚}{250.0}者、是無\twnr{愧}{251.0}者、是懷怨恨者、是劣慧者,阿那律!具備這五法的婦女,以身體的崩解,死後往生苦界、惡趣、下界、地獄。」



\sutta{7}{7}{嫉妒者經}{https://agama.buddhason.org/SN/sn.php?keyword=37.7}
  「阿那律!具備五法的婦女,以身體的崩解,死後往生\twnr{苦界}{109.0}、\twnr{惡趣}{110.0}、\twnr{下界}{111.0}、地獄,哪五個?她是無信者、是無\twnr{慚}{250.0}者、是無\twnr{愧}{251.0}者、是嫉妒者、是劣慧者,阿那律!具備這五法的婦女,以身體的崩解,死後往生苦界、惡趣、下界、地獄。」



\sutta{8}{8}{慳吝者經}{https://agama.buddhason.org/SN/sn.php?keyword=37.8}
  「阿那律!具備五法的婦女,以身體的崩解,死後往生\twnr{苦界}{109.0}、\twnr{惡趣}{110.0}、\twnr{下界}{111.0}、地獄,哪五個?她是無信者、是無\twnr{慚}{250.0}者、是無\twnr{愧}{251.0}者、是慳吝者、是劣慧者,阿那律!具備這五法的婦女……(中略)往生苦界、惡趣、下界、地獄。」



\sutta{9}{9}{通姦者經}{https://agama.buddhason.org/SN/sn.php?keyword=37.9}
  「阿那律!具備五法的婦女……(中略)往生\twnr{苦界}{109.0}、\twnr{惡趣}{110.0}、\twnr{下界}{111.0}、地獄,哪五個?她是無信者、是無\twnr{慚}{250.0}者、是無\twnr{愧}{251.0}者、是通姦者、是劣慧者,阿那律!具備這五法的婦女……(中略)往生……。」



\sutta{10}{10}{壞德行者經}{https://agama.buddhason.org/SN/sn.php?keyword=37.10}
  「阿那律!具備五法的婦女……(中略)往生……地獄,哪五個?她是無信者、是無\twnr{慚}{250.0}者、是無\twnr{愧}{251.0}者、是壞德行者、是劣慧者,阿那律!具備這五法的婦女……(中略)往生……地獄。」



\sutta{11}{11}{少聞者經}{https://agama.buddhason.org/SN/sn.php?keyword=37.11}
  「阿那律!具備五法的婦女……(中略)往生……地獄,哪五個?她是無信者、是無\twnr{慚}{250.0}者、是無\twnr{愧}{251.0}者、是少聞者、是劣慧者,阿那律!具備這五法的婦女……(中略)往生……地獄。」



\sutta{12}{12}{懈怠者經}{https://agama.buddhason.org/SN/sn.php?keyword=37.12}
  「阿那律!具備五法的婦女……(中略)往生……地獄,哪五個?她是無信者、是無\twnr{慚}{250.0}者、是無\twnr{愧}{251.0}者、是懈怠者、是劣慧者,阿那律!具備這五法的婦女……(中略)往生\twnr{苦界}{109.0}、\twnr{惡趣}{110.0}、\twnr{下界}{111.0}、地獄。」



\sutta{13}{13}{念已忘失者經}{https://agama.buddhason.org/SN/sn.php?keyword=37.13}
  「阿那律!具備五法的婦女……(中略)往生……地獄,哪五個?她是無信者、是無\twnr{慚}{250.0}者、是無\twnr{愧}{251.0}者、是\twnr{念已忘失者}{216.0}、是劣慧者,阿那律!具備這五法的婦女……(中略)往生……地獄。」



\sutta{14}{14}{五怨經}{https://agama.buddhason.org/SN/sn.php?keyword=37.14}
  「阿那律!具備五法的婦女……(中略)往生……地獄,哪五個?她是殺生者、是\twnr{未給予而取}{104.0}者、是\twnr{邪淫}{105.0}者、是\twnr{妄語}{106.0}者、是\twnr{榖酒果酒酒放逸處}{107.0}者,阿那律!具備這五法的婦女,以身體的崩解,死後往生\twnr{苦界}{109.0}、\twnr{惡趣}{110.0}、\twnr{下界}{111.0}、地獄。」

  第一中略品,其\twnr{攝頌}{35.0}:

  「婦女、男子,以及特有、三法,

   易憤怒者、懷怨恨者,以及嫉妒者、慳吝者,

   通姦者、壞德行者,以及少聞者、懈怠者,

   念已忘失者、五怨恨,使在黑側被知道。」





\pin{第二中略品}{15}{24}
\sutta{15}{15}{不易憤怒者經}{https://agama.buddhason.org/SN/sn.php?keyword=37.15}
  那時,\twnr{尊者}{200.0}阿那律去見\twnr{世尊}{12.0}。抵達後……(中略)在一旁坐下的尊者阿那律對世尊說這個:

  「\twnr{大德}{45.0}!這裡,我以清淨、超越常人的天眼,看見以身體的崩解,死後往生\twnr{善趣}{112.0}、天界的婦女,大德!具備幾法的婦女,以身體的崩解,死後往生善趣、天界呢?」

  「阿那律!具備五法的婦女,以身體的崩解,死後往生善趣、天界,哪五個?她是有信者、是有\twnr{慚}{250.0}者、是有\twnr{愧}{251.0}者、是不易憤怒者、是有慧者,阿那律!具備這五法的婦女,以身體的崩解,死後往生善趣、天界。」



\sutta{16}{16}{不懷怨恨者經}{https://agama.buddhason.org/SN/sn.php?keyword=37.16}
  「阿那律!具備五法的婦女,以身體的崩解,死後往生\twnr{善趣}{112.0}、天界,哪五個?她是有信者、是有\twnr{慚}{250.0}者、是有\twnr{愧}{251.0}者、是不懷怨恨者、是有慧者,阿那律!具備這五法的婦女,以身體的崩解,死後往生善趣、天界。」



\sutta{17}{17}{不嫉妒者經}{https://agama.buddhason.org/SN/sn.php?keyword=37.17}
  「阿那律!具備五法的婦女,以身體的崩解,死後往生\twnr{善趣}{112.0}、天界,哪五個?她是有信者、是有\twnr{慚}{250.0}者、是有\twnr{愧}{251.0}者、是不嫉妒者、是有慧者,阿那律!具備這五法的婦女,以身體的崩解,死後往生善趣、天界。」



\sutta{18}{18}{不慳吝者經}{https://agama.buddhason.org/SN/sn.php?keyword=37.18}
  ……不慳吝者、有慧者……(中略)



\sutta{19}{19}{非通姦者經}{https://agama.buddhason.org/SN/sn.php?keyword=37.19}
  ……非通姦者、有慧者……(中略)



\sutta{20}{20}{有德行者經}{https://agama.buddhason.org/SN/sn.php?keyword=37.20}
  ……有德行者、有慧者……(中略)



\sutta{21}{21}{多聞者經}{https://agama.buddhason.org/SN/sn.php?keyword=37.21}
  ……多聞者、有慧者……(中略)



\sutta{22}{22}{活力已發動者經}{https://agama.buddhason.org/SN/sn.php?keyword=37.22}
  ……活力已發動者、有慧者……(中略)



\sutta{23}{23}{念已現起經}{https://agama.buddhason.org/SN/sn.php?keyword=37.23}
  ……是\twnr{念已現起}{341.0}者、是有慧者,阿那律!具備這五法的婦女,以身體的崩解,死後往生\twnr{善趣}{112.0}、天界。」 

  這些是省略的八個經。



\sutta{24}{24}{五戒經}{https://agama.buddhason.org/SN/sn.php?keyword=37.24}
  「阿那律!具備五法的婦女,以身體的崩解,死後往生\twnr{善趣}{112.0}、天界,哪五個?她是離殺生者、是離\twnr{未給予而取}{104.0}者、是離\twnr{邪淫}{105.0}者、是離\twnr{妄語}{106.0}者、是離\twnr{榖酒果酒酒放逸處}{106.0}者,阿那律!具備這五法的婦女,以身體的崩解,死後往生善趣、天界。」

  第二中略品,其\twnr{攝頌}{35.0}:

  「而在第二[品]中不易憤怒者,不懷怨恨者、不嫉妒者,

   不慳吝者、非通姦者,有無德者、多聞者,

   活力者、有念者與戒,使在白側被知道。」





\pin{力品}{25}{34}
\sutta{25}{25}{有自信經}{https://agama.buddhason.org/SN/sn.php?keyword=37.25}
  「\twnr{比丘}{31.0}們!有這婦女的五力,哪五種?[姿]色力、財富力、親族力、兒子力、德行力,比丘們!這些是婦女的五力,比丘們!具備這五力的婦女有自信地住於在家。」



\sutta{26}{26}{壓迫後經}{https://agama.buddhason.org/SN/sn.php?keyword=37.26}
  「\twnr{比丘}{31.0}們!有這婦女的五力,哪五種?[姿]色力、財富力、親族力、兒子力、德行力,比丘們!這些是婦女的五力,比丘們!具備這五力的婦女壓迫丈夫後住家中。」



\sutta{27}{27}{征服後經}{https://agama.buddhason.org/SN/sn.php?keyword=37.27}
  「\twnr{比丘}{31.0}們!有這婦女的五力,哪五種?[姿]色力、財富力、親族力、兒子力、德行力,比丘們!這些是婦女的五力,比丘們!具備這五力的婦女壓迫、征服丈夫後轉起(存續)。」



\sutta{28}{28}{一經}{https://agama.buddhason.org/SN/sn.php?keyword=37.28}
  「\twnr{比丘}{31.0}們!具備一力的男子征服婦女後轉起,哪一力呢?被權威力征服的婦女,[姿]色力既不庇護,財富力也不庇護,親族力也不庇護,兒子力也不庇護,德行力也不庇護。」



\sutta{29}{29}{部分經}{https://agama.buddhason.org/SN/sn.php?keyword=37.29}
  「\twnr{比丘}{31.0}們!有這婦女的五力,哪五種?[姿]色力、財富力、親族力、兒子力、德行力。比丘們!婦女具備[姿]色力而無財富力者,這樣,她以那個部分是不完全者,比丘們!但當婦女具備[姿]色力與財富力,這樣,她以那個部分是完全者。比丘們!婦女具備[姿]色力與財富力而無親族力者,這樣,她以那個部分是不完全者,比丘們!但當婦女具備[姿]色力、財富力、親族力,這樣,她以那個部分是完全者。比丘們!婦女具備[姿]色力、財富力、親族力而無兒子力者,這樣,她以那個部分是不完全者,比丘們!但當婦女具備[姿]色力、財富力、親族力、兒子力,這樣,她以那個部分是完全者。比丘們!婦女具備[姿]色力、財富力、親族力、兒子力而無有德力者,這樣,她以那個部分是不完全者,比丘們!但當婦女具備[姿]色力、財富力、親族力、兒子力、德行力,這樣,她以那個部分是完全者。比丘們!這些是婦女的五力。」



\sutta{30}{30}{驅逐經}{https://agama.buddhason.org/SN/sn.php?keyword=37.30}
  「\twnr{比丘}{31.0}們!有這婦女的五力,哪五種?[姿]色力、財富力、親族力、兒子力、德行力。

  比丘們!婦女具備[姿]色力而無德行力者,他們就驅逐她、不使之住家中。比丘們!婦女具備[姿]色力、財富力而無德行力者,他們就驅逐她、不使之住家中。比丘們!婦女具備[姿]色力、財富力、親族力而無德行力者,他們就驅逐她、不使之住家中。比丘們!婦女具備[姿]色力、財富力、親族力、兒子力而無德行力者,他們就驅逐她、不使之住家中。

  比丘們!婦女具備德行力而無[姿]色力者,他們就使她住家中、不驅逐。比丘們!婦女具備德行力而無財富力者,他們就使她住家中、不驅逐。比丘們!婦女具備德行力而無親族力者,他們就使她住家中、不驅逐。比丘們!婦女具備德行力而無兒子力者,他們就使她住家中、不驅逐。

  比丘們!這些是婦女的五力。」



\sutta{31}{31}{因經}{https://agama.buddhason.org/SN/sn.php?keyword=37.31}
  「\twnr{比丘}{31.0}們!有這婦女的五力,哪五種?[姿]色力、財富力、親族力、兒子力、德行力。比丘們!非[姿]色力之因,或財富力之因,或親族力之因,或兒子力之因婦女以身體的崩解,死後往生\twnr{善趣}{112.0}、天界,比丘們!德行力之因婦女以身體的崩解,死後往生善趣、天界。比丘們!這些是婦女的五力。」



\sutta{32}{32}{處經}{https://agama.buddhason.org/SN/sn.php?keyword=37.32}
  「\twnr{比丘}{31.0}們!有這五處難被不作福德的婦女得到,哪五個?『能被出生在適合的家中。』比丘們!這是難被不作福德的婦女得到的第一處。『出生在適合的家中後,能[嫁]到適合的家中。』比丘們!這是難被不作福德的婦女得到的第二處。『出生在適合的家中、[嫁]到適合的家中後,能無共事一夫的狀態住家中。』比丘們!這是難被不作福德的婦女得到的第三處。『出生在適合的家中、[嫁]到適合的家中後,無共事一夫的狀態住家中的,能有兒子的。』比丘們!這是難被不作福德的婦女得到的第四處。『出生在適合的家中、[嫁]到適合的家中後,無共事一夫的狀態住家中的,有兒子的,持續存在征服丈夫後能使之轉起。』比丘們!這是難被不作福德的婦女得到的第五處。比丘們!這些是難被不作福德的婦女得到的五處。

  比丘們!有這五處容易被作福德的婦女得到,哪五個?『能被出生在適合的家中。』比丘們!這是容易被作福德的婦女得到的第一處。『出生在適合的家中後,能[嫁]到適合的家中。』比丘們!這是容易被作福德的婦女得到的第二處。『出生在適合的家中、[嫁]到適合的家中後,能無共事一夫的狀態住家中。』比丘們!這是容易被作福德的婦女得到的第三處。『出生在適合的家中、[嫁]到適合的家中後,無共事一夫的狀態住家中的,能有兒子的。』比丘們!這是容易被作福德的婦女得到的第四處。『出生在適合的家中、[嫁]到適合的家中後,無共事一夫的狀態住家中的,能有兒子的,持續存在征服丈夫後能使之轉起。』比丘們!這是容易被作福德的婦女得到的第五處。比丘們!這些是容易被作福德的婦女得到的五處。」



\sutta{33}{33}{五戒有自信經}{https://agama.buddhason.org/SN/sn.php?keyword=37.33}
  「\twnr{比丘}{31.0}們!具備五法的婦女有自信地住家中,哪五個?她是離殺生者、離\twnr{未給予而取}{104.0}者、離\twnr{邪淫}{105.0}者、離\twnr{妄語}{106.0}者、離\twnr{榖酒果酒酒放逸處}{106.0}者,比丘們!具備這五法的婦女有自信地住家中。」



\sutta{34}{34}{增長經}{https://agama.buddhason.org/SN/sn.php?keyword=37.34}
  「\twnr{比丘}{31.0}們!當以五個增長增長時,女聖弟子以聖增長增長,並且她是核心的取得者、殊勝身體的取得者,以哪五個?以信增長、以戒增長、以聽聞增長、以施捨增長、以慧增長,比丘們!當以這五個增長增長時,女聖弟子以聖增長增長,並且她是核心的取得者、殊勝身體的取得者。」

  「凡在這裡以信、戒增長,與以慧、以施捨、以聽聞兩者,

   那位像那樣有戒的\twnr{優婆夷}{99.0},就在這裡得到自己的核心。」

  力品第三,其\twnr{攝頌}{35.0}:

  「有自信、壓迫後、征服後,一、以部分為第五,

   驅逐、因與處,有自信、以增長為第十。」

  婦女相應完成。





\page

\xiangying{38}{閻浮車相應}
\sutta{1}{1}{涅槃的詢問經}{https://agama.buddhason.org/SN/sn.php?keyword=38.1}
  \twnr{有一次}{2.0},\twnr{尊者}{200.0}舍利弗住在摩揭陀那拉迦村落。

  那時,\twnr{遊行者}{79.0}閻浮車去見尊者舍利弗。抵達後,與尊者舍利弗一起互相問候。交換應該被互相問候的友好交談後,在一旁坐下。在一旁坐下的遊行者閻浮車對尊者舍利弗說這個:

  「舍利弗\twnr{道友}{201.0}!被稱為『涅槃、涅槃』,道友!什麼是涅槃呢?」

  「道友!凡貪的滅盡、瞋的滅盡、癡的滅盡,這被稱為涅槃。」

  「道友!那麼,為了這涅槃的作證,\twnr{有道、有道跡}{359.0}嗎?」

  「道友!為了這涅槃的作證,有道、有道跡。」

  「道友!那麼,為了這涅槃的作證,什麼是道?什麼是道跡?」

  「道友!就是這\twnr{八支聖道}{525.0},即:正見、正志、正語、正業、正命、正方便、正念、正定。道友!為了這涅槃的作證,這是道,這是道跡。」

  「道友!為了這涅槃的作證,道是善的,道跡是善的,舍利弗道友!還有,對不放逸是足夠的。」



\sutta{2}{2}{阿羅漢狀態的詢問經}{https://agama.buddhason.org/SN/sn.php?keyword=38.2}
  「舍利弗\twnr{道友}{201.0}!被稱為『阿羅漢狀態、阿羅漢狀態』,舍利弗道友!什麼是阿羅漢狀態呢?」

  「道友!凡貪的滅盡、瞋的滅盡、癡的滅盡,這被稱為阿羅漢狀態。」

  「道友!那麼,為了這阿羅漢狀態的作證,\twnr{有道、有道跡}{359.0}嗎?」

  「道友!為了這阿羅漢狀態的作證,有道、有道跡。」

  「道友!那麼,為了這阿羅漢狀態的作證,什麼是道?什麼是道跡?」

  「道友!就是這\twnr{八支聖道}{525.0},即:正見、正志、正語、正業、正命、正方便、正念、正定。道友!為了這阿羅漢狀態的作證,這是道,這是道跡。」

  「道友!為了這阿羅漢狀態的作證,道是善的,道跡是善的,舍利弗道友!還有,對不放逸是足夠的。」



\sutta{3}{3}{如法之說者的詢問經}{https://agama.buddhason.org/SN/sn.php?keyword=38.3}
  「舍利弗\twnr{道友}{201.0}!誰是世間中\twnr{如法之說者}{648.0}呢?誰是世間中\twnr{善行者}{518.0}?誰是世間中的\twnr{善逝}{8.0}呢?」

  「道友!凡為了貪的滅盡教導法,為了瞋的滅盡教導法,為了癡的滅盡教導法,他們是世間中如法之說者。

  道友!凡為了貪的滅盡之行者,為了瞋的滅盡之行者,為了癡的滅盡之行者,他們是世間中善行者。

  道友!凡對他們來說貪已被捨斷,根已被切斷,\twnr{[如]已斷根的棕櫚樹}{147.1},\twnr{成為非有}{408.0},\twnr{為未來不生之物}{229.0};瞋已被捨斷,根已被切斷,[如]已斷根的棕櫚樹,成為非有,為未來不生之物;癡已被捨斷,根已被切斷,[如]已斷根的棕櫚樹,成為非有,為未來不生之物者,他們是世間中的善逝。」

  「道友!那麼,為了這貪、瞋、癡的捨斷,\twnr{有道、有道跡}{359.0}嗎?」

  「道友!為了這貪、瞋、癡的捨斷,有道、有道跡。」

  「道友!那麼,為了這貪、瞋、癡的捨斷,什麼是道?什麼是道跡?」

  「道友!就是這\twnr{八支聖道}{525.0},即:正見、正志、正語、正業、正命、正方便、正念、正定。道友!為了這貪、瞋、癡的捨斷,這是道,這是道跡。」

  「道友!為了這貪、瞋、癡的捨斷,道是善的,道跡是善的,舍利弗道友!還有,對不放逸是足夠的。」



\sutta{4}{4}{為了什麼目的經}{https://agama.buddhason.org/SN/sn.php?keyword=38.4}
  「舍利弗\twnr{道友}{201.0}!為了什麼目的在\twnr{沙門}{29.0}\twnr{喬達摩}{80.0}處梵行被住?」

  「道友!為了苦的\twnr{遍知}{154.0}在\twnr{世尊}{12.0}處梵行被住。」

  「道友!那麼,為了這個苦的遍知,\twnr{有道、有道跡}{359.0}嗎?」

  「道友!為了這個苦的遍知,有道、有道跡。」

  「道友!那麼,為了這個苦的遍知,什麼是道?什麼是道跡?」

  「道友!就是這\twnr{八支聖道}{525.0},即:正見、正志、正語、正業、正命、正方便、正念、正定。道友!為了這個苦的遍知,這是道,這是道跡。」[≃\suttaref{SN.45.5}]

  「道友!為了這個苦的遍知,道是善的,道跡是善的,舍利弗道友!還有,對不放逸是足夠的。」



\sutta{5}{5}{已達到穌息者經}{https://agama.buddhason.org/SN/sn.php?keyword=38.5}
  「舍利弗\twnr{道友}{201.0}!被稱為『已達到穌息者、已達到穌息者』,道友!什麼情形是已達到穌息者呢?」

  「道友!當\twnr{比丘}{31.0}如實知道\twnr{六觸處}{78.0}的\twnr{集起}{67.0}、滅沒、\twnr{樂味}{295.0}、\twnr{過患}{293.0}、\twnr{出離}{294.0},道友!這個情形是已達到穌息者。」

  「道友!那麼,為了這穌息的作證,\twnr{有道、有道跡}{359.0}嗎?」

  「道友!為了這穌息的作證,有道、有道跡。」

  「道友!那麼,為了這穌息的作證,什麼是道?什麼是道跡?」

  「道友!就是這\twnr{八支聖道}{525.0},即:正見、正志、正語、正業、正命、正方便、正念、正定。道友!為了這穌息的作證,這是道,這是道跡。」

  「道友!為了這穌息的作證,道是善的,道跡是善的,舍利弗道友!還有,對不放逸是足夠的。」



\sutta{6}{6}{已達到最高穌息者經}{https://agama.buddhason.org/SN/sn.php?keyword=38.6}
  「舍利弗\twnr{道友}{201.0}!被稱為『已達到最高穌息者、已達到最高穌息者』,道友!什麼情形是已達到最高穌息者呢?」

  「道友!當\twnr{比丘}{31.0}如實知道\twnr{六觸處}{78.0}的\twnr{集起}{67.0}、滅沒、\twnr{樂味}{295.0}、\twnr{過患}{293.0}、\twnr{出離}{294.0}後,不執取後成為解脫者,道友!這個情形是已達到最高穌息者。」

  「道友!那麼,為了這最高穌息的作證,\twnr{有道、有道跡}{359.0}嗎?」

  「道友!為了這最高穌息的作證,有道、有道跡。」

  「道友!那麼,為了這最高穌息的作證,什麼是道?什麼是道跡?」

  「道友!就是這\twnr{八支聖道}{525.0},即:正見、正志、正語、正業、正命、正方便、正念、正定。道友!為了這最高穌息的作證,這是道,這是道跡。」

  「道友!為了這最高穌息的作證,道是善的,道跡是善的,舍利弗道友!還有,對不放逸是足夠的。」



\sutta{7}{7}{受的詢問經}{https://agama.buddhason.org/SN/sn.php?keyword=38.7}
  「舍利弗\twnr{道友}{201.0}!被稱為『受、受』,道友!什麼是受呢?」

  「道友!有這些三受,哪三個?樂受、苦受、不苦不樂受,道友!這些是三受。」

  「道友!那麼,為了這些三受的\twnr{遍知}{154.0},\twnr{有道、有道跡}{359.0}嗎?」

  「道友!為了這些三受的遍知,有道、有道跡。」

  「道友!那麼,為了這些三受的遍知,什麼是道?什麼是道跡?」

  「道友!就是這\twnr{八支聖道}{525.0},即:正見、正志、正語、正業、正命、正方便、正念、正定。道友!為了這些三受的遍知,這是道,這是道跡。」

  「道友!為了這些三受的遍知,道是善的,道跡是善的,舍利弗道友!還有,對不放逸是足夠的。」



\sutta{8}{8}{漏的詢問經}{https://agama.buddhason.org/SN/sn.php?keyword=38.8}
  「舍利弗\twnr{道友}{201.0}!被稱為『\twnr{漏}{188.0}、漏』,道友!什麼是漏呢?」

  「道友!有這三種漏:欲漏、有漏、\twnr{無明漏}{397.0},道友!這些是三種漏。」

  「道友!那麼,為了這些漏的捨斷,\twnr{有道、有道跡}{359.0}嗎?」

  「道友!為了這些漏的捨斷,有道、有道跡。」

  「道友!那麼,為了這些漏的捨斷,什麼是道?什麼是道跡?」

  「道友!就是這\twnr{八支聖道}{525.0},即:正見、正志、正語、正業、正命、正方便、正念、正定。道友!為了這些漏的捨斷,這是道,這是道跡。」

  「道友!為了這些漏的捨斷,道是善的,道跡是善的,舍利弗道友!還有,對不放逸是足夠的。」



\sutta{9}{9}{無明的詢問經}{https://agama.buddhason.org/SN/sn.php?keyword=38.9}
  「舍利弗\twnr{道友}{201.0}!被稱為『\twnr{無明}{207.0}、無明』,道友!什麼是無明呢?」

  「道友!凡在苦上的無知、在苦\twnr{集}{67.0}上的無知、在苦\twnr{滅}{68.0}上的無知、在導向苦\twnr{滅道跡}{69.0}上的無知,道友!這被稱為無明。」

  「道友!那麼,為了這無明的捨斷,\twnr{有道、有道跡}{359.0}嗎?」

  「道友!為了這無明的捨斷,有道、有道跡。」

  「道友!那麼,為了這無明的捨斷,什麼是道?什麼是道跡?」

  「道友!就是這\twnr{八支聖道}{525.0},即:正見、正志、正語、正業、正命、正方便、正念、正定。道友!為了這無明的捨斷,這是道,這是道跡。」

  「道友!為了這無明的捨斷,道是善的,道跡是善的,舍利弗道友!還有,對不放逸是足夠的。」



\sutta{10}{10}{渴愛的詢問經}{https://agama.buddhason.org/SN/sn.php?keyword=38.10}
  「舍利弗\twnr{道友}{201.0}!被稱為『渴愛、渴愛』,道友!什麼是渴愛呢?」

  「道友!有這三種渴愛:欲的渴愛、有的渴愛、\twnr{虛無的渴愛}{244.0},道友!這些是三種渴愛。」

  「道友!那麼,為了這渴愛的捨斷,\twnr{有道、有道跡}{359.0}嗎?」

  「道友!為了這渴愛的捨斷,有道、有道跡。」

  「道友!那麼,為了這渴愛的捨斷,什麼是道?什麼是道跡?」

  「道友!就是這\twnr{八支聖道}{525.0},即:正見、正志、正語、正業、正命、正方便、正念、正定。道友!為了這渴愛的捨斷,這是道,這是道跡。」

  「道友!為了這渴愛的捨斷,道是善的,道跡是善的,舍利弗道友!還有,對不放逸是足夠的。」



\sutta{11}{11}{暴流的詢問經}{https://agama.buddhason.org/SN/sn.php?keyword=38.11}
  「舍利弗\twnr{道友}{201.0}!被稱為『\twnr{暴流}{369.0}、暴流』,道友!什麼是暴流呢?」

  「道友!有這四種暴流:欲的暴流、有的暴流、見的暴流、無明的暴流,道友!這是四種暴流。」

  「道友!那麼,為了這暴流的捨斷,有道、\twnr{有道跡}{359.0}嗎?」

  「道友!為了這暴流的捨斷,\twnr{有道、有道跡}{359.0}。」

  「道友!那麼,為了這暴流的捨斷,什麼是道?什麼是道跡?」

  「道友!就是這\twnr{八支聖道}{525.0},即:正見、正志、正語、正業、正命、正方便、正念、正定。道友!為了這暴流的捨斷,這是道,這是道跡。」

  「道友!為了這暴流的捨斷,道是善的,道跡是善的,舍利弗道友!還有,對不放逸是足夠的。」



\sutta{12}{12}{取的詢問經}{https://agama.buddhason.org/SN/sn.php?keyword=38.12}
  「舍利弗\twnr{道友}{201.0}!被稱為『取、取』,道友!什麼是取呢?」

  「道友!有這四種取:欲取、見取、\twnr{戒禁取}{194.0}、\twnr{[真]我論取}{444.0},道友!這是四種取。」

  「道友!那麼,為了這取的捨斷,有道、\twnr{有道跡}{359.0}嗎?」

  「道友!為了這取的捨斷,\twnr{有道、有道跡}{359.0}。」

  「道友!那麼,為了這取的捨斷,什麼是道?什麼是道跡?」

  「道友!就是這\twnr{八支聖道}{525.0},即:正見、正志、正語、正業、正命、正方便、正念、正定。道友!為了這取的捨斷,這是道,這是道跡。」

  「道友!為了這取的捨斷,道是善的,道跡是善的,舍利弗道友!還有,對不放逸是足夠的。」



\sutta{13}{13}{有的詢問經}{https://agama.buddhason.org/SN/sn.php?keyword=38.13}
  「舍利弗\twnr{道友}{201.0}!被稱為『有、有』,道友!什麼是有呢?」

  「道友!有這三種有:欲有、色有、\twnr{無色有}{261.0},道友!這些是三種有。」

  「道友!那麼,為了這有的\twnr{遍知}{154.0},\twnr{有道、有道跡}{359.0}嗎?」

  「道友!為了這有的遍知,有道、有道跡。」

  「道友!那麼,為了這有的遍知,什麼是道?什麼是道跡?」

  「道友!就是這\twnr{八支聖道}{525.0},即:正見、正志、正語、正業、正命、正方便、正念、正定。道友!為了這有的遍知,這是道,這是道跡。」

  「道友!為了這有的遍知,道是善的,道跡是善的,舍利弗道友!還有,對不放逸是足夠的。」



\sutta{14}{14}{苦的詢問經}{https://agama.buddhason.org/SN/sn.php?keyword=38.14}
  「舍利弗\twnr{道友}{201.0}!被稱為『苦、苦』,道友!什麼是苦呢?」

  「道友!有這三種苦性:苦苦性、行苦性、\twnr{變易苦性}{649.0},道友!這些是三種苦。」

  「道友!那麼,為了這苦的\twnr{遍知}{154.0},\twnr{有道、有道跡}{359.0}嗎?」

  「道友!為了這個苦的遍知,有道、有道跡。」

  「道友!那麼,為了這個苦的遍知,什麼是道?什麼是道跡?」

  「道友!就是這\twnr{八支聖道}{525.0},即:正見、正志、正語、正業、正命、正方便、正念、正定。道友!為了這個苦的遍知,這是道,這是道跡。」

  「道友!為了這個苦的遍知,道是善的,道跡是善的,舍利弗道友!還有,對不放逸是足夠的。」



\sutta{15}{15}{有身的詢問經}{https://agama.buddhason.org/SN/sn.php?keyword=38.15}
  「舍利弗\twnr{道友}{201.0}!被稱為『\twnr{有身}{93.0}、有身』,道友!什麼是有身呢?」

  「道友!這些五取蘊被\twnr{世尊}{12.0}稱為有身,即:色取蘊、受取蘊、想取蘊、行取蘊、識取蘊,道友!這些五取蘊被世尊稱為有身。」

  「道友!那麼,為了這有身的\twnr{遍知}{154.0},\twnr{有道、有道跡}{359.0}嗎?」

  「道友!為了這有身的遍知,有道、有道跡。」

  「道友!那麼,為了這有身的遍知,什麼是道?什麼是道跡?」

  「道友!就是這\twnr{八支聖道}{525.0},即:正見、正志、正語、正業、正命、正方便、正念、正定。道友!為了這有身的遍知,這是道,這是道跡。」

  「道友!為了這有身的遍知,道是善的,道跡是善的,舍利弗道友!還有,對不放逸是足夠的。」



\sutta{16}{16}{難做的詢問經}{https://agama.buddhason.org/SN/sn.php?keyword=38.16}
  「舍利弗\twnr{道友}{201.0}!什麼是在這法、律中難做的呢?」

  「道友!出家是在這法、律中難做的。」

  「道友!那麼,什麼是被出家者難做的呢?」

  「道友!歡喜是被出家者難做的。」

  「道友!那麼,什麼是被歡喜者難做的呢?」

  「道友!\twnr{法、隨法行}{58.0}是被歡喜者難做的。」

  「道友!那麼,法、隨法行的\twnr{比丘}{31.0}多久會成為\twnr{阿羅漢}{5.0}?」

  「道友!不久。」

  閻浮車相應完成,其\twnr{攝頌}{35.0}:

  「涅槃與阿羅漢狀態,如法之說者、為了什麼目的,

   穌息、最高穌息,受、漏、無明,

   渴愛、暴流、取,有、苦與有身,

   在這法、律中難做的。」





\page

\xiangying{39}{沙門達葛相應}
\sutta{1}{1}{沙門達葛經}{https://agama.buddhason.org/SN/sn.php?keyword=39.1}
  有一次,\twnr{尊者}{200.0}舍利弗住在跋耆的烏迦支羅恒河邊。

  那時,\twnr{遊行者}{79.0}\twnr{沙門}{29.0}達葛去見尊者舍利弗。抵達後,與尊者舍利弗一起互相問候。交換應該被互相問候的友好交談後,在一旁坐下。在一旁坐下的遊行者沙門達葛對尊者舍利弗說這個:

  「舍利弗\twnr{道友}{201.0}!被稱為『涅槃、涅槃』,道友!什麼是涅槃呢?」

  「道友!凡貪的滅盡、瞋的滅盡、癡的滅盡,這被稱為涅槃。」

  「道友!那麼,為了這涅槃的作證,\twnr{有道、有道跡}{359.0}嗎?」

  「道友!為了這涅槃的作證,有道、有道跡。」

  「道友!那麼,為了這涅槃的作證,什麼是道?什麼是道跡?」

  「道友!就是這\twnr{八支聖道}{525.0},即:正見、正志、正語、正業、正命、正方便、正念、正定。道友!對這涅槃的作證,這是道,這是道跡。」

  「道友!這涅槃的作證,道是善的,道跡是善的,舍利弗道友!還有,對不放逸是足夠的。」

  (應該如閻浮車相應那樣使之被細說)



\sutta{2}{2}{難做的經}{https://agama.buddhason.org/SN/sn.php?keyword=39.2}
  「舍利弗\twnr{道友}{201.0}!什麼是在這法、律中難做的呢?」

  「道友!出家是在這法、律中難做的。」

  「道友!那麼,什麼是被出家者難做的呢?」

  「道友!歡喜是被出家者難做的。」

  「道友!那麼,什麼是被歡喜者難做的呢?」

  「道友!\twnr{法、隨法行}{58.0}是被歡喜者難做的。」

  「道友!那麼,法、隨法行的\twnr{比丘}{31.0}多久會成為\twnr{阿羅漢}{5.0}?」

  「道友!不久。」

  (與前面相同的\twnr{攝頌}{35.0})

  沙門達葛相應完成。





\page

\xiangying{40}{目揵連相應}
\sutta{1}{1}{初禪的詢問經}{https://agama.buddhason.org/SN/sn.php?keyword=40.1}
  \twnr{有一次}{2.0},\twnr{尊者}{200.0}目揵連住在舍衛城祇樹林給孤獨園。

  在那裡,尊者目揵連召喚\twnr{比丘}{31.0}們:

  「比丘\twnr{學友}{201.0}們!」

  「學友!」那些比丘回答尊者目揵連。

  尊者目揵連說這個:

  「學友們!這裡,當我獨處、\twnr{獨坐}{92.0}時,這樣心的深思生起:『被稱為「初禪,初禪。」什麼是初禪呢?』

  學友們!那個我想這個:『這裡,比丘就從離諸欲後,從離諸不善法後,\twnr{進入後住於}{66.0}有尋、\twnr{有伺}{175.0},\twnr{離而生喜、樂}{174.0}的初禪,這被稱為第初禪。』

  學友們!我就從離諸欲後,從離諸不善法後,進入後住於有尋、有伺,離而生喜、樂的初禪。

  學友們!當那個我以這個住處住時,與欲俱行的想、作意被執行(想起)。

  學友們!那時,\twnr{世尊}{12.0}以\twnr{神通}{503.0}來見我後,說這個:『目揵連!目揵連!不要對初禪放逸,婆羅門!請你在初禪上使心\twnr{安頓}{888.0},請你在初禪上心作專一,請你在初禪上集中心。』

  學友們!過些時候,那個我就從離諸欲後,離諸不善法後,進入後住於有尋、\twnr{有伺}{175.0},\twnr{離而生喜、樂}{174.0}的初禪。

  學友們!凡當正確說它時,應該說『被大師資助達到\twnr{大通智}{564.0}的弟子』,那是我,當正確說時,應該說『被大師資助已達到大通智的弟子。』」



\sutta{2}{2}{第二禪的詢問經}{https://agama.buddhason.org/SN/sn.php?keyword=40.2}
  [……「\twnr{學友}{201.0}們!這裡,當我獨處、\twnr{獨坐}{92.0}時,這樣心的深思生起:]

  『被稱為「第二禪,第二禪」,什麼是第二禪呢?』

  學友們!那個我想這個:『這裡,\twnr{比丘}{31.0}以尋與伺的平息,\twnr{自身內的明淨}{256.0},\twnr{心的專一性}{255.0},\twnr{進入後住於}{66.0}無尋、無伺,定而生喜、樂的第二禪,這被稱為第二禪。』

  學友們!那個我從尋與伺的平息,自身內的明淨,心的專一性,進入後住於無尋、無伺,定而生喜、樂的第二禪。

  學友們!當那個我以這個住處住時,與尋俱行的想、作意被執行(想起)。

  學友們!那時,\twnr{世尊}{12.0}以\twnr{神通}{503.0}來見我後,說這個:『目揵連!目揵連!不要對第二禪放逸,婆羅門!請你在第二禪上使心\twnr{安頓}{888.0},請你在第二禪上心作專一,請你在第二禪上集中心。』

  學友們!過些時候,那個我從尋與伺的平息,自身內的明淨,心的專一性,進入後住於無尋、無伺,定而生喜、樂的第二禪。

  學友們!凡當正確說它時,應該說『被大師資助達到\twnr{大通智}{564.0}的弟子』,那是我,當正確說時,應該說『被大師資助已達到大通智的弟子。』」



\sutta{3}{3}{第三禪的詢問經}{https://agama.buddhason.org/SN/sn.php?keyword=40.3}
  [……「\twnr{學友}{201.0}們!這裡,當我獨處、\twnr{獨坐}{92.0}時,這樣心的深思生起:]

  『被稱為「第三禪,第三禪。」什麼是第三禪呢?』

  學友們!那個我想這個:『這裡,\twnr{比丘}{31.0}從喜的\twnr{褪去}{77.0}、住於\twnr{平靜}{228.0}、有念正知、以身體感受樂,我\twnr{進入後住於}{66.0}凡聖者們告知『他是平靜者、具念者、\twnr{安樂住者}{317.0}』的第三禪,這被稱為第三禪。』

  學友們!那個我從喜的褪去、住於平靜、有念正知、以身體感受樂,我進入後住於凡聖者們宣說它『是平靜的、具念的、安樂住的』的第三禪。

  學友們!當那個我以這個住處住時,與喜俱行的想、作意被執行(想起)。

  學友們!那時,\twnr{世尊}{12.0}以\twnr{神通}{503.0}來見我後,說這個:『目揵連!目揵連!不要對第三禪放逸,婆羅門!請你在第三禪上使心\twnr{安頓}{888.0},請你在第三禪上心作專一,請你在第三禪上集中心。』

  學友們!過些時候,那個我從喜的褪去、住於平靜、有念正知、以身體感受樂,我進入後住於凡聖者們宣說它『是平靜的、具念的、安樂住的』的第三禪。

  學友們!凡當正確說它時,應該說……(中略)[『被大師資助達到\twnr{大通智}{564.0}的弟子』,那是我,當正確說時,應該說『被大師資助已達到]大通智的弟子。』」



\sutta{4}{4}{第四禪的詢問經}{https://agama.buddhason.org/SN/sn.php?keyword=40.4}
  [……「\twnr{學友}{201.0}們!這裡,當我獨處、\twnr{獨坐}{92.0}時,這樣心的深思生起:]

  『被稱為「第四禪,第四禪。」什麼是第四禪呢?』

  學友們!那個我想這個:『這裡,\twnr{比丘}{31.0}從樂的捨斷與從苦的捨斷,就在之前諸喜悅、憂的滅沒,我\twnr{進入後住於}{66.0}不苦不樂,\twnr{平靜、念遍純淨}{494.0}的第四禪,這被稱為第四禪。』

  學友們!那個我從樂的捨斷與從苦的捨斷,就在之前諸喜悅、憂的滅沒,我進入後住於不苦不樂,平靜、念遍純淨的第四禪。

  學友們!當那個我以這個住處住時,與樂俱行的想、作意被執行(想起)。

  學友們!那時,\twnr{世尊}{12.0}以\twnr{神通}{503.0}來見我後,說這個:『目揵連!目揵連!不要對第四禪放逸,婆羅門!請你在第四禪上使心\twnr{安頓}{888.0},請你在第四禪上心作專一,請你在第四禪上集中心。』

  學友們!過些時候,那個我從樂的捨斷與從苦的捨斷,就在之前諸喜悅、憂的滅沒,我進入後住於不苦不樂,平靜、念遍純淨的第四禪。

  學友們!凡當正確說它時,應該說……(中略)[『被大師資助達到\twnr{大通智}{564.0}的弟子』,那是我,當正確說時,應該說『被大師資助已達到]大通智的弟子。』」



\sutta{5}{5}{虛空無邊處的詢問經}{https://agama.buddhason.org/SN/sn.php?keyword=40.5}
  [……「\twnr{學友}{201.0}們!這裡,當我獨處、\twnr{獨坐}{92.0}時,這樣心的深思生起:]

  『被稱為「虛空無邊處,虛空無邊處。」什麼是虛空無邊處呢?』

  學友們!那個我想這個:『這裡,\twnr{比丘}{31.0}\twnr{從一切色想的超越}{490.0},從\twnr{有對想}{331.0}的滅沒,從不作意種種想[而知]:『虛空是無邊的』,\twnr{進入後住於}{66.0}虛空無邊處,這被稱為虛空無邊處。』

  學友們!那個我從一切色想的超越,從有對想的滅沒,從不作意種種想[而知]:『虛空是無邊的』,進入後住於虛空無邊處。

  學友們!當那個我以這個住處住時,與色俱行的想、作意被執行(想起)。

  學友們!那時,\twnr{世尊}{12.0}以\twnr{神通}{503.0}來見我後,說這個:『目揵連!目揵連!不要對虛空無邊處放逸,婆羅門!請你使心\twnr{安頓}{888.0}在虛空無邊處上,請你使心統一在虛空無邊處上,請你使心集中在第虛空無邊處上。』

  學友們!過些時候,那個我從一切色想的超越,從有對想的滅沒,從不作意種種想[而知]:『虛空是無邊的』,進入後住於虛空無邊處。

  學友們!凡當正確說它時,應該說……(中略)[『被大師資助達到\twnr{大通智}{564.0}的弟子』,那是我,當正確說時,應該說『被大師資助已達到]大通智的弟子。』」





\sutta{6}{6}{識無邊處的詢問經}{https://agama.buddhason.org/SN/sn.php?keyword=40.6}
  [……「\twnr{學友}{201.0}們!這裡,當我獨處、\twnr{獨坐}{92.0}時,這樣心的深思生起:]

  『被稱為「識無邊處,識無邊處。」什麼是識無邊處呢?』

  學友們!那個我想這個:『這裡,\twnr{比丘}{31.0}超越一切虛空無邊處後[而知]:『識是無邊的』,\twnr{進入後住於}{66.0}識無邊處,這被稱為識無邊處。』

  學友們!那個我超越一切虛空無邊處後[而知]:『識是無邊的』,進入後住於識無邊處。

  學友們!當那個我以這個住處住時,與虛空無邊處俱行的想、作意被執行(想起)。

  學友們!那時,\twnr{世尊}{12.0}以\twnr{神通}{503.0}來見我後,說這個:『目揵連!目揵連!不要對識無邊處放逸,婆羅門!請你在識無邊處上使心\twnr{安頓}{888.0},請你在識無邊處上心作專一,請你在識無邊處上集中心。』

  學友們!過些時候,那個我超越一切虛空無邊處後[而知]:『識是無邊的』,進入後住於識無邊處。

  學友們!凡當正確說它時,應該說……(中略)[『被大師資助達到\twnr{大通智}{564.0}的弟子』,那是我,當正確說時,應該說『被大師資助已達到]大通智的弟子。』」



\sutta{7}{7}{無所有處的詢問經}{https://agama.buddhason.org/SN/sn.php?keyword=40.7}
  [……「\twnr{學友}{201.0}們!這裡,當我獨處、\twnr{獨坐}{92.0}時,這樣心的深思生起:]

  『被稱為「\twnr{無所有處}{533.0},無所有處。」什麼是無所有處呢?』

  學友們!那個我想這個:『這裡,\twnr{比丘}{31.0}超越一切識無邊處後[而知]:『什麼都沒有』,\twnr{進入後住於}{66.0}無所有處,這被稱為無所有處。』

  學友們!那個我超越一切識無邊處後[而知]:『什麼都沒有』,進入後住於無所有處。

  學友們!當那個我以這個住處住時,與識無邊處俱行的想、作意被執行(想起)。

  學友們!那時,\twnr{世尊}{12.0}以\twnr{神通}{503.0}來見我後,說這個:『目揵連!目揵連!不要對無所有處放逸,婆羅門!請你在無所有處上使心\twnr{安頓}{888.0},請你在無所有處上心作專一,請你在無所有處上集中心。』

  學友們!過些時候,那個我超越一切識無邊處後[而知]:『什麼都沒有』,進入後住於無所有處。

  學友們!凡當正確說它時,應該說……(中略)[『被大師資助達到\twnr{大通智}{564.0}的弟子』,那是我,當正確說時,應該說『被大師資助已達到]大通智的弟子。』」



\sutta{8}{8}{非想非非想處的詢問經}{https://agama.buddhason.org/SN/sn.php?keyword=40.8}
  [……「\twnr{學友}{201.0}們!這裡,當我獨處、\twnr{獨坐}{92.0}時,這樣心的深思生起:]

  『被稱為「\twnr{非想非非想處}{534.0},非想非非想處。」什麼是非想非非想處呢?』

  學友們!那個我想這個:『這裡,\twnr{比丘}{31.0}超越一切無所有處後,\twnr{進入後住於}{66.0}非想非非想處,這被稱為非想非非想處。』

  學友們!那個我超越一切無所有處後,進入後住於非想非非想處。

  學友們!當那個我以這個住處住時,與無所有處俱行的想、作意被執行(想起)。

  學友們!那時,\twnr{世尊}{12.0}以\twnr{神通}{503.0}來見我後,說這個:『目揵連!目揵連!不要對非想非非想處放逸,婆羅門!請你在非想非非想處上使心\twnr{安頓}{888.0},請你在非想非非想處上心作專一,請你在非想非非想處上集中心。』

  學友們!過些時候,那個我超越一切無所有處後,進入後住於非想非非想處。

  學友們!凡當正確說它時,應該說……(中略)[『被大師資助達到\twnr{大通智}{564.0}的弟子』,那是我,當正確說時,應該說『被大師資助已達到]大通智的弟子。』」



\sutta{9}{9}{無相的詢問經}{https://agama.buddhason.org/SN/sn.php?keyword=40.9}
  [……「\twnr{學友}{201.0}們!這裡,當我獨處、\twnr{獨坐}{92.0}時,這樣心的深思生起:]

  『被稱為「\twnr{無相心定}{265.0}、無相心定」,什麼是無相心定呢?』

  學友們!那個我想這個:『這裡,\twnr{比丘}{31.0}以一切相的不作意,\twnr{進入後住於}{66.0}無相心定,這被稱為無相心定。』

  學友們!那個我以一切相的不作意,進入後住於無相心定。

  學友們!當那個我以這個住處住時,識是\twnr{相的隨行者}{781.0}。

  學友們!那時,\twnr{世尊}{12.0}以\twnr{神通}{503.0}來見我後,說這個:『目揵連!目揵連!不要對無相心定放逸,婆羅門!請你在無相心定上使心\twnr{安頓}{888.0},請你在無相心定上心作專一,請你在無相心定上集中心。』

  學友們!過些時候,那個我以一切相的不作意,進入後住於無相心定。

  學友們!凡當正確說它時,應該說『被\twnr{大師}{145.0}資助已達到\twnr{大通智}{564.0}的弟子』,那是我,當正確說時,應該說『被大師資助已達到大通智的弟子。』」



\sutta{10}{10}{帝釋經}{https://agama.buddhason.org/SN/sn.php?keyword=40.10}
  那時,\twnr{尊者}{200.0}大目揵連就猶如有力氣的男子伸直彎曲的手臂,或彎曲伸直的手臂,就像這樣在祇樹園消失,出現在三十三天中。

  那時,\twnr{天帝釋}{263.0}與約五百位天神一同去見尊者大目揵連。抵達後,向尊者大目揵連\twnr{問訊}{46.0}後,在一旁站立。尊者大目揵連對在一旁站立的天帝釋說這個:

  「天帝!佛之\twnr{歸依}{284.0}是\twnr{好的}{44.0},天帝!佛歸依之因,這樣,這裡一些眾生以身體的崩解,死後往生\twnr{善趣}{112.0}、天界。

  天帝!法之歸依是好的,天帝!法歸依之因,這樣,這裡一些眾生以身體的崩解,死後往生善趣、天界。

  天帝!\twnr{僧團}{375.0}之歸依是好的,天帝!僧團歸依之因,這樣,這裡一些眾生以身體的崩解,死後往生善趣、天界。」

  「親愛的目揵連尊師!佛之歸依是好的,親愛的目揵連尊師!佛歸依之因,這樣,這裡一些眾生以身體的崩解,死後往生善趣、天界。

  親愛的目揵連尊師!法之歸依是好的,親愛的目揵連尊師!法歸依之因,這樣,這裡一些眾生以身體的崩解,死後往生善趣、天界。

  親愛的目揵連尊師!僧團之歸依是好的……(中略)往生善趣、天界。」

  那時,天帝釋與六百位天神一同……(中略)那時,天帝釋與七百位天神一同……(中略)那時,天帝釋與八百位天神一同……(中略)那時,天帝釋與八萬位天神一同去見尊者大目揵連。抵達後,向尊者大目揵連問訊後,在一旁站立。在一旁站立的尊者大目揵連對天帝釋說這個:

  「天帝!佛之歸依是好的,天帝!佛歸依之因,這樣,這裡一些眾生以身體的崩解,死後往生善趣、天界。

  天帝!法之歸依是好的,天帝!法歸依之因,這樣,這裡一些眾生以身體的崩解,死後往生善趣、天界。

  天帝!僧團之歸依是好的,天帝!歸依僧團之因,這樣,這裡一些眾生以身體的崩解,死後往生善趣、天界。」

  「親愛的目揵連尊師!佛之歸依是好的,親愛的目揵連尊師!佛歸依之因,這樣,這裡一些眾生以身體的崩解,死後往生善趣、天界。

  親愛的目揵連尊師!法之歸依是好的……(中略)親愛的目揵連尊師!僧團之歸依是好的,歸依僧團之因,這樣,這裡一些眾生以身體的崩解,死後往生善趣、天界。」

  那時,天帝釋與約五百位天神一同去見尊者大目揵連。抵達後,向尊者大目揵連問訊後,在一旁站立。在一旁站立的尊者大目揵連對天帝釋說這個:

  「天帝!在佛上以\twnr{不壞淨}{233.0}之具備是\twnr{好的}{44.0}:『像這樣,那位\twnr{世尊}{12.0}是\twnr{阿羅漢}{5.0}、\twnr{遍正覺者}{6.0}、\twnr{明行具足者}{7.0}、\twnr{善逝}{8.0}、\twnr{世間知者}{9.0}、\twnr{應該被調御人的無上調御者}{10.0}、\twnr{天-人們的大師}{11.0}、\twnr{佛陀}{3.0}、世尊。』天帝!以在佛上不壞淨具備之因,這樣,這裡一些眾生以身體的崩解,死後往生善趣、天界。

  天帝!在法上以不壞淨之具備是好的:『被世尊善說的法是直接可見的、即時的、請你來看的、能引導的、\twnr{應該被智者各自經驗的}{395.0}。』天帝!以在法上不壞淨具備之因,這樣,這裡一些眾生以身體的崩解,死後往生善趣、天界。

  天帝!在僧團上以不壞淨之具備是好的:『世尊的弟子僧團是\twnr{善行者}{518.0},世尊的弟子僧團是正直行者,世尊的弟子僧團是真理行者,世尊的弟子僧團是\twnr{方正行者}{764.0},即:四雙之人、\twnr{八輩之士}{347.0},這世尊的弟子僧團應該被奉獻、應該被供奉、應該被供養、應該被合掌,為世間的無上\twnr{福田}{101.0}。』天帝!以在僧團上不壞淨具備之因,這樣,這裡一些眾生以身體的崩解,死後往生善趣、天界。

  天帝!具備聖者喜愛的諸戒是好的:無毀壞的、無瑕疵的、無污點的、無雜色的、自由的、智者稱讚的、不取著的、轉起定的。天帝!具備聖者喜愛的諸戒之因,這樣,這裡一些眾生以身體的崩解,死後往生善趣、天界。」

  「親愛的目揵連尊師!在佛上以不壞淨之具備是好的:『這樣,那位……(中略)天-人們的大師、佛陀、世尊。』親愛的目揵連尊師!以在佛上不壞淨具備之因,這樣,這裡一些眾生以身體的崩解,死後往生善趣、天界。

  親愛的目揵連尊師!在法上以不壞淨之具備是好的:『被世尊善說的法是……(中略)應該被智者各自經驗的。』親愛的目揵連尊師!以在法上不壞淨具備之因,這樣,這裡一些眾生以身體的崩解,死後往生善趣、天界。

  親愛的目揵連尊師!在僧團上以不壞淨之具備是好的:『世尊的弟子僧團是善行者……(中略)為世間的無上福田。』親愛的目揵連尊師!以在僧團上不壞淨具備之因,這樣,這裡一些眾生以身體的崩解,死後往生善趣、天界。

  親愛的目揵連尊師!以聖者喜愛的諸戒之具備是好的:『無毀壞的……(中略)轉起定的。親愛的目揵連尊師!以聖者喜愛的諸戒具備之因,這樣,這裡一些眾生以身體的崩解,死後往生善趣、天界。」

  那時,天帝釋與六百位天神一同……(中略)那時,天帝釋與七百位天神一同……(中略)那時,天帝釋與八百位天神一同……(中略)那時,天帝釋與八萬位天神一同去見尊者大目揵連。抵達後,向尊者大目揵連問訊後,在一旁站立。在一旁站立的尊者大目揵連對天帝釋說這個:

  「天帝!在佛上以不壞淨之具備是好的:『像這樣,那位世尊是……(中略)天-人們的大師、佛陀、世尊。』天帝!以在佛上不壞淨具備之因,這樣,這裡一些眾生以身體的崩解,死後往生善趣、天界。

  天帝!在法上以不壞淨之具備是好的:『被世尊善說的法是……(中略)應該被智者各自經驗的。』天帝!以在法上不壞淨具備之因,這樣,這裡一些眾生以身體的崩解,死後往生善趣、天界。

  天帝!在僧團上以不壞淨之具備是好的:『世尊的弟子僧團是善行者……(中略)為世間的無上福田。』天帝!以在僧團上不壞淨具備之因,這樣,這裡一些眾生以身體的崩解,死後往生善趣、天界。

  天帝!以聖者喜愛的諸戒之具備是好的:『無毀壞的……(中略)轉起定的。天帝!以聖者喜愛的諸戒具備之因,這樣,這裡一些眾生以身體的崩解,死後往生善趣、天界。」

  「親愛的目揵連尊師!在佛上以不壞淨之具備是好的:『像這樣,那位世尊是……(中略)天-人們的大師、佛陀、世尊。』親愛的目揵連尊師!以在佛上不壞淨具備之因,這樣,這裡一些眾生以身體的崩解,死後往生善趣、天界。

  親愛的目揵連尊師!在法上以不壞淨之具備是好的:『被世尊善說的法是……(中略)應該被智者各自經驗的。』親愛的目揵連尊師!以在法上不壞淨具備之因,這樣,這裡一些眾生以身體的崩解,死後往生善趣、天界。

  親愛的目揵連尊師!在僧團上以不壞淨之具備是好的:『世尊的弟子僧團是善行者……(中略)為世間的無上福田。』親愛的目揵連尊師!以在僧團上不壞淨具備之因,這樣,這裡一些眾生以身體的崩解,死後往生善趣、天界。

  親愛的目揵連尊師!以聖者喜愛的諸戒之具備是好的:『無毀壞的……(中略)轉起定的。親愛的目揵連尊師!以聖者喜愛的諸戒具備之因,這樣,這裡一些眾生以身體的崩解,死後往生善趣、天界。」

  那時,天帝釋與約五百位天神一同去見尊者大目揵連。……(中略)在一旁站立的尊者大目揵連對天帝釋說這個:

  「天帝!佛之歸依是好的,天帝!佛歸依之因,這樣,這裡一些眾生以身體的崩解,死後往生善趣、天界。祂們以十處超越其他天神:以天的壽命、天的容色、天的快樂、天的名聲、天的統治權、天的諸色、天的諸聲音、天的諸氣味、天的諸味道、天的諸\twnr{所觸}{220.2}。

  天帝!法之歸依是好的,天帝!法歸依之因,這樣,這裡一些眾生以身體的崩解,死後往生善趣、天界。祂們以十處超越其他天神:以天的壽命、天的容色、天的快樂、天的名聲、天的統治權、天的諸色、天的諸聲音、天的諸氣味、天的諸味道、天的諸所觸。

  天帝!僧團之歸依是好的,天帝!歸依僧團之因,這樣,這裡一些眾生以身體的崩解,死後往生善趣、天界。祂們以十處超越其他天神:以天的壽命、天的容色、天的快樂、天的名聲、天的統治權、天的諸色、天的諸聲音、天的諸氣味、天的諸味道、天的諸所觸。」

  「親愛的目揵連尊師!佛之歸依是好的,親愛的目揵連尊師!佛歸依之因,這樣,這裡一些眾生以身體的崩解,死後往生善趣、天界。祂們以十處超越其他天神:以天的壽命……(中略)天的所觸。

  親愛的目揵連尊師!法之歸依是好的……(中略)親愛的目揵連尊師!僧團之歸依是好的,親愛的目揵連尊師!歸依僧團之因,這樣,這裡一些眾生以身體的崩解,死後往生善趣、天界。祂們以十處超越其他天神:以天的壽命、天的容色、天的快樂、天的名聲、天的統治權、天的諸色、天的諸聲音、天的諸氣味、天的諸味道、天的諸所觸。」

  那時,天帝釋與六百位天神一同……(中略)那時,天帝釋與七百位天神一同……(中略)那時,天帝釋與八百位天神一同……(中略)那時,天帝釋與八萬位天神一同去見尊者大目揵連。抵達後,向尊者大目揵連問訊後,在一旁站立。在一旁站立的尊者大目揵連對天帝釋說這個:

  「天帝!佛之歸依是好的,天帝!佛歸依之因,這樣,這裡一些眾生以身體的崩解,死後往生善趣、天界。祂們以十處超越其他天神:以天的壽命……(中略)天的所觸。

  天帝!法之歸依是好的……(中略)天帝!僧團之歸依是好的,天帝!歸依僧團之因,這樣,這裡一些眾生以身體的崩解,死後往生善趣、天界。祂們以十處超越其他天神:以天的壽命、天的容色、天的快樂、天的名聲、天的統治權、天的諸色、天的諸聲音、天的諸氣味、天的諸味道、天的諸所觸。」

  「親愛的目揵連尊師!佛之歸依是好的……(中略)親愛的目揵連尊師!法之歸依是好的……(中略)親愛的目揵連尊師!僧團之歸依是好的,歸依僧團之因,這樣,這裡一些眾生以身體的崩解,死後往生善趣、天界。祂們以十處超越其他天神:以天的壽命、天的容色、天的快樂、天的名聲、天的統治權、天的諸色、天的諸聲音、天的諸氣味、天的諸味道、天的諸所觸。」

  那時,天帝釋與約五百位天神一同去見尊者大目揵連。抵達後,向尊者大目揵連問訊後,在一旁站立。在一旁站立的尊者大目揵連對天帝釋說這個:

  「天帝!在佛上以不壞淨之具備是好的:『像這樣,那位世尊是……(中略)天-人們的大師、佛陀、世尊。』天帝!以在佛上不壞淨具備之因,這樣,這裡一些眾生以身體的崩解,死後往生善趣、天界。祂們以十處超越其他天神:以天的壽命……(中略)天的所觸。

  天帝!在法上以不壞淨之具備是好的:『被世尊善說的法是……(中略)應該被智者各自經驗的。』天帝!以在法上不壞淨具備之因,這樣,這裡一些眾生以身體的崩解,死後往生善趣、天界。……(中略)。

  天帝!在僧團上以不壞淨之具備是好的:『世尊的弟子僧團是善行者……(中略)為世間的無上福田。』天帝!以在僧團上不壞淨具備之因,這樣,這裡一些眾生以身體的崩解,死後往生善趣、天界。……(中略)。

  天帝!以聖者喜愛的諸戒之具備是好的:『無毀壞的……(中略)轉起定的。天帝!以聖者喜愛的諸戒具備之因,這樣,這裡一些眾生以身體的崩解,死後往生善趣、天界。祂們以十處超越其他天神:以天的壽命……(中略)天的所觸。」

  「親愛的目揵連尊師!在佛上以不壞淨之具備是好的:『像這樣,那位世尊是……(中略)天-人們的大師、佛陀、世尊。』親愛的目揵連尊師!以在佛上不壞淨具備之因,這樣,這裡一些眾生以身體的崩解,死後往生善趣、天界。祂們以十處超越其他天神:以天的壽命……(中略)天的所觸。

  親愛的目揵連尊師!在法上以不壞淨之具備是好的:『被世尊善說的法是……(中略)應該被智者各自經驗的。』親愛的目揵連尊師!以在法上不壞淨具備之因,這樣,這裡一些眾生以身體的崩解,死後往生善趣、天界。祂們以十處超越其他天神:以天的壽命……(中略)天的所觸。

  親愛的目揵連尊師!在僧團上以不壞淨之具備是好的:『世尊的弟子僧團是善行者……(中略)為世間的無上福田。』親愛的目揵連尊師!以在僧團上不壞淨具備之因,這樣,這裡一些眾生以身體的崩解,死後往生善趣、天界。……(中略)。

  親愛的目揵連尊師!以聖者喜愛的諸戒之具備是好的:『無毀壞的……(中略)轉起定的。親愛的目揵連尊師!以聖者喜愛的諸戒具備之因,這樣,這裡一些眾生以身體的崩解,死後往生善趣、天界。祂們以十處超越其他天神:以天的壽命……(中略)天的所觸。」

  那時,天帝釋與六百位天神一同……(中略)那時,天帝釋與七百位天神一同……(中略)那時,天帝釋與八百位天神一同……(中略)那時,天帝釋與八萬位天神一同去見尊者大目揵連。抵達後,向尊者大目揵連問訊後,在一旁站立。在一旁站立的尊者大目揵連對天帝釋說這個:

  「天帝!在佛上以不壞淨之具備是好的:『像這樣,那位世尊是阿羅漢、遍正覺者、明行具足者、善逝、世間知者、應該被調御人的無上調御者、天-人們的大師、佛陀、世尊。』天帝!以在佛上不壞淨具備之因,這樣,這裡一些眾生以身體的崩解,死後往生善趣、天界。祂們以十處超越其他天神:以天的壽命、天的容色、天的快樂、天的名聲、天的統治權、天的諸色、天的諸聲音、天的諸氣味、天的諸味道、天的諸所觸。

  天帝!在法上以不壞淨之具備是好的:『被世尊善說的法是直接可見的、即時的、請你來看的、能引導的、應該被智者各自經驗的。』天帝!以在法上不壞淨具備之因,這樣,這裡一些眾生以身體的崩解,死後往生善趣、天界。祂們以十處超越其他天神:以天的壽命……(中略)天的所觸。

  天帝!在僧團上以不壞淨之具備是好的:『世尊的弟子僧團是善行者,世尊的弟子僧團是正直行者,世尊的弟子僧團是真理行者,世尊的弟子僧團是方正行者,即:四雙之人、八輩之士,這世尊的弟子僧團應該被奉獻、應該被供奉、應該被供養、應該被合掌,為世間的無上福田。』天帝!以在僧團上不壞淨具備之因,這樣,這裡一些眾生以身體的崩解,死後往生善趣、天界。祂們以十處超越其他天神:以天的壽命……(中略)天的所觸。

  天帝!具備聖者喜愛的諸戒是好的:無毀壞的、無瑕疵的、無污點的、無雜色的、自由的、智者稱讚的、不取著的、轉起定的。天帝!具備聖者喜愛的諸戒之因,這樣,這裡一些眾生以身體的崩解,死後往生善趣、天界。祂們以十處超越其他天神:以天的壽命、天的容色、天的快樂、天的名聲、天的統治權、天的諸色、天的諸聲音、天的諸氣味、天的諸味道、天的諸所觸。」

  「親愛的目揵連尊師!在佛上以不壞淨之具備是好的:『像這樣,那位世尊是……(中略)天-人們的大師、佛陀、世尊。』親愛的目揵連尊師!以在佛上不壞淨具備之因,這樣,這裡一些眾生以身體的崩解,死後往生善趣、天界。祂們以十處超越其他天神:以天的壽命……(中略)天的所觸。

  親愛的目揵連尊師!在法上以不壞淨之具備是好的:『被世尊善說的法是……(中略)應該被智者各自經驗的。』親愛的目揵連尊師!以在法上不壞淨具備之因,這樣,這裡一些眾生以身體的崩解,死後往生善趣、天界。祂們以十處超越其他天神:以天的壽命……(中略)天的所觸。

  親愛的目揵連尊師!在僧團上以不壞淨之具備是好的:『世尊的弟子僧團是善行者……(中略)為世間的無上福田。』親愛的目揵連尊師!以在僧團上不壞淨具備之因,這樣,這裡一些眾生以身體的崩解,死後往生善趣、天界。祂們以十處超越其他天神:以天的壽命……(中略)天的所觸。

  親愛的目揵連尊師!以聖者喜愛的諸戒之具備是好的:『無毀壞的……(中略)轉起定的。親愛的目揵連尊師!以聖者喜愛的諸戒具備之因,這樣,這裡一些眾生以身體的崩解,死後往生善趣、天界。祂們以十處超越其他天神:以天的壽命、天的容色、天的快樂、天的名聲、天的統治權、天的諸色、天的諸聲音、天的諸氣味、天的諸味道、天的諸所觸。」[≃\suttaref{SN.55.18}-20]



\sutta{11}{11}{檀香經}{https://agama.buddhason.org/SN/sn.php?keyword=40.11}
  那時,檀香\twnr{天子}{282.0}……(中略)。

  那時,善耶摩天子……(中略)。

  那時,珊兜率天子……(中略)。

  那時,善化作天子……(中略)。

  那時,自在天子……(中略)。

  (這五則中略應該如帝釋經那樣使之被細說)

  目揵連相應完成,其\twnr{攝頌}{35.0}:

  「有尋、無尋,以及以樂、平靜,

   虛空、識,連同無所有、非想,

   無相與帝釋,以檀香為十一。」





\page

\xiangying{41}{質多相應}
\sutta{1}{1}{結縛經}{https://agama.buddhason.org/SN/sn.php?keyword=41.1}
  \twnr{有一次}{2.0},眾多\twnr{上座}{135.0}\twnr{比丘}{31.0}住在麻七迦三達的蓭巴德葛(檳榔青)林中。

  當時,當眾多上座比丘\twnr{餐後已從施食返回}{512.0},在圓亭棚集會共坐時,這個談論中出現:

  「\twnr{學友}{201.0}們!『結』或『\twnr{會被結縛的諸法}{666.0}』,這些法是不同義理、不同辭句?或者是一種義理,僅辭句不同?」

  在那裡,被一些上座比丘這麼解說:

  「學友!『結』或『會被結縛的諸法』,這些法是不同義理,而且也不同辭句。」

  被一些上座比丘這麼解說:

  「學友!『結』或『會被結縛的諸法』,這些法是一種義理,僅辭句不同。」

  當時,\twnr{屋主}{103.0}質多正以某些應該被作的抵達彌迦巴榻迦。

  屋主質多聽聞眾多上座比丘餐後已從施食返回,在圓亭棚集會共坐時,這個談論中出現:「學友!『結』或『會被結縛的諸法』,這些法是不同義理、不同辭句?或者是一種義理,僅辭句不同?」在那裡,被一些上座比丘這麼解說:「學友!『結』或『會被結縛的諸法』,這些法是不同義理,而且也不同辭句。」被一些上座比丘這麼解說:「學友!『結』或『會被結縛的諸法』,這些法是一種義理,僅辭句不同。」

  那時,屋主質多去見上座比丘們。抵達後,向上座比丘們\twnr{問訊}{46.0}後,在一旁坐下。在一旁坐下的屋主質多對上座比丘們說這個:

  「\twnr{大德}{45.0}!這被我聽聞:眾多上座比丘餐後已從施食返回,在圓亭棚集會共坐時,這個談論中出現:『學友!「結」或「會被結縛的諸法」,這些法是不同義理、不同辭句呢?或者為一種義理,僅辭句不同呢?』被一些上座比丘這麼解說:『學友!「結」或「會被結縛的諸法」,這些法是不同義理,而且也不同辭句。』被一些上座比丘這麼解說:『學友!「結」或「會被結縛的諸法」,這些法是一種義理,僅辭句不同。』」

  「是的,屋主!」

  「大德!『結』或『會被結縛的諸法』,這些法是不同義理,而且也不同辭句。

  大德!那樣的話,我將為你們作譬喻,這裡,一些有智的男子也以譬喻了知所說的義理。大德!猶如黑牛與白牛,被一條繩子或繫繩連結(綁在一起),凡這麼說:『黑牛是白牛的結縛;白牛是黑牛的結縛。』那位說者正確地說嗎?」

  「屋主!這確實不是。」

  「大德!黑牛不是白牛的結縛;白牛不是黑牛的結縛,凡牠們被一條繩子或繫繩連結,在那裡那是結縛。同樣的,大德!眼不是諸色的結縛;諸色也不是眼的結縛,但凡\twnr{緣於}{252.0}那兩者意欲貪生起處,在那裡那是結縛。

  耳不是諸聲音……(中略)鼻不是諸氣味……舌不是諸味道……身不是諸\twnr{所觸}{220.2}的結縛;諸所觸也不是身的結,但凡緣於那兩者意欲貪生起處,在那裡那是結縛。意不是諸法的結縛;諸法也不是意的結縛,但凡緣於那兩者意欲貪生起處,在那裡那是結縛。[\suttaref{SN.35.232}]」

  「屋主!是你的利得,屋主!\twnr{是你的善得的}{350.0}:你的慧眼走入甚深的佛語中。」



\sutta{2}{2}{梨犀達多經第一}{https://agama.buddhason.org/SN/sn.php?keyword=41.2}
  \twnr{有一次}{2.0},眾多\twnr{上座}{135.0}\twnr{比丘}{31.0}住在麻七迦三達的蓭巴德葛(檳榔青)林中。

  那時,\twnr{屋主}{103.0}質多去見上座比丘們。抵達後,向上座比丘們\twnr{問訊}{46.0}後,在一旁坐下。在一旁坐下的屋主質多對上座比丘們說這個:

  「\twnr{大德}{45.0}!請上座們同意我的明天的食事。」

  上座比丘們以沈默狀態同意。

  那時,屋主質多知道上座比丘們同意後,從座位起來、向上座比丘們問訊、\twnr{作右繞}{47.0}後,離開。

  那時,那夜過後,上座比丘們午前時穿衣、拿起衣鉢後,去屋主質多的住處。抵達後,在設置的座位坐下。

  那時,屋主質多去見上座比丘們。抵達後,向上座比丘們問訊後,在一旁坐下。在一旁坐下的屋主質多對上座\twnr{尊者}{200.0}說這個:

  「上座大德!被稱為『種種界(界種種性)、種種界』,大德!什麼情形被\twnr{世尊}{12.0}稱為種種界?」

  在這麼說時,上座尊者保持沈默。

  第二次,屋主質多又對上座尊者說這個:

  「上座大德!被稱為『種種界、種種界』,大德!什麼情形被世尊稱為種種界?」第二次,上座尊者又保持沈默。

  第三次,屋主質多又對上座尊者說這個:

  「上座大德!被稱為『種種界、種種界』,大德!什麼情形被世尊稱為種種界?」第三次,上座尊者又保持沈默。

  當時,尊者梨犀達多在那個\twnr{僧團}{375.0}中是\twnr{最新人}{135.3}。

  那時,尊者梨犀達多對上座尊者說這個:

  「上座大德!我回答屋主質多的這個問題。」

  「梨犀達多\twnr{學友}{201.0}!請你回答屋主質多的這個問題。」

  「屋主!你這麼問:『上座大德!被稱為『種種界、種種界』,大德!什麼情形被世尊稱為種種界呢?』嗎?」

  「是的!大德!」

  「屋主!這被世尊稱為種種界:眼界、色界、眼識界……(中略)\twnr{意界}{337.0}、\twnr{法界}{547.0}、\twnr{意識界}{337.1},屋主!這個情形被世尊稱為種種界。」

  那時,屋主質多歡喜、\twnr{隨喜}{85.0}尊者梨犀達多所說後,以勝妙的\twnr{硬食、軟食}{153.0}親手款待上座比丘們,使之滿足。

  那時,已食、手離鉢的上座比丘們從座位起來後離開。

  那時,上座尊者對尊者梨犀達多說這個:

  「\twnr{好}{44.0}!梨犀達多學友!這個問題在你心中出現(變得清楚),這個問題不在我心中出現。梨犀達多學友!那樣的話,如果當像這樣的其它問題又到來,\twnr{你就以這個方式應答}{x541}。」



\sutta{3}{3}{梨犀達多經第二}{https://agama.buddhason.org/SN/sn.php?keyword=41.3}
  \twnr{有一次}{2.0},眾多\twnr{上座}{135.0}\twnr{比丘}{31.0}住在麻七迦三達的蓭巴德葛(檳榔青)林中。

  那時,\twnr{屋主}{103.0}質多去見上座比丘們。抵達後,向上座比丘們\twnr{問訊}{46.0}後,在一旁坐下。在一旁坐下的屋主質多對上座比丘們說這個:

  「\twnr{大德}{45.0}!請上座們同意我的明天的食事。」

  上座比丘們以沈默狀態同意。

  那時,屋主質多知道上座比丘們同意後,從座位起來、向上座比丘們問訊、\twnr{作右繞}{47.0}後,離開。

  那時,那夜過後,上座比丘們午前時穿衣、拿起衣鉢後,去屋主質多的住處。抵達後,在設置的座位坐下。

  那時,屋主質多去見上座比丘們。抵達後,向上座比丘們問訊後,在一旁坐下。在一旁坐下的屋主質多對上座\twnr{尊者}{200.0}說這個:

  「上座大德!凡世間中這許多種見生起:『世界是常恆的』,或『\twnr{世界是非常恆的}{170.0}』,或『世界是有邊的』,或『世界是無邊的』,或『命即是身體』,或『\twnr{命是一身體是另一}{169.0}』,或『死後\twnr{如來}{4.0}存在』,或『死後如來不存在』,或『\twnr{死後如來存在且不存在}{354.0}』,或『死後如來既非存在也非不存在』,又,凡\twnr{在梵網中}{x542}所說的這些六十二\twnr{惡見}{722.0},大德!這些見在什麼存在時存在,在什麼不存在時不存在呢?」

  在這麼說時,上座尊者保持沈默。

  第二次,屋主質多又……(中略)。

  第三次,屋主質多又對上座尊者說這個:

  「上座大德!凡世間中這許多種見生起:『世界是常恆的』,或『世界是非常恆的』,或『世界是有邊的』,或『世界是無邊的』,或『命即是身體』,或『命是一身體是另一』,或『死後如來存在』,或『死後如來不存在』,或『死後如來存在且不存在』,或『死後如來既非存在也非不存在』,又,凡在梵網中所說的這些六十二惡見,大德!這些見在什麼存在時存在,在什麼不存在時不存在呢?」第三次,上座尊者又保持沈默。

  當時,尊者梨犀達多在那個\twnr{僧團}{375.0}中是\twnr{最新人}{135.3}。

  那時,尊者梨犀達多對上座尊者說這個:

  「上座大德!我回答屋主質多的這個問題。」

  「梨犀達多\twnr{學友}{201.0}!請你回答屋主質多的這個問題。」

  「屋主!你這麼問:『上座大德!凡世間中這許多種見生起:『世界是常恆的』,或……(中略)大德!這些見在什麼存在時存在,在什麼不存在時不存在呢?』嗎?」

  「是的!大德!」

  「屋主!凡世間中這許多種見生起:『世界是常恆的』,或『世界是非常恆的』,或『世界是有邊的』,或『世界是無邊的』,或『命即是身體』,或『命是一身體是另一』,或『死後如來存在』,或『死後如來不存在』,或『死後如來存在且不存在』,或『死後如來既非存在也非不存在』,又,凡在梵網中所說的這些六十二惡見,屋主!這些見在有身見存在時存在,在有身見不存在時不存在。」

  「大德!那麼,怎樣是有身見存在呢?」

  「屋主!這裡,\twnr{未聽聞的一般人}{74.0}是聖者的未看見者,聖者法的不熟知者,在聖者法上未被教導者;是善人的未看見者,\twnr{善人法}{76.0}的不熟知者,在善人法上未被教導者,他\twnr{認為}{964.0}色是我,\twnr{或我擁有色}{13.0},或色在我中,\twnr{或我在色中}{14.0};認為受是我……(中略)想……諸行……認為識是我,或我擁有識,或識在我中,或我在識中,屋主!這樣有有身見存在。」

  「大德!那麼,怎樣是有身見不存在呢?」

  「屋主!這裡,有聽聞的聖弟子是聖者的看見者,聖者法的熟知者,在聖者法上被善教導者;是善人的看見者,善人法的熟知者,在善人法上被善教導者,他認為色不是我,或我不擁有色,或色不在我中,或我不在色中;認為受不……(中略)認為想不……認為諸行不……認為識不是我,或我不擁有識,或識不在我中,或我不在識中,屋主!這樣有有身見不存在。」

  「大德!\twnr{聖}{612.1}梨犀達多從哪裡來?」

  「屋主!我從阿槃提來。」

  「大德!有一位名叫梨犀達多的阿槃提\twnr{善男子}{41.0},是我們未見面的出家朋友,他被尊者見看見嗎?」

  「是的,屋主!」

  「大德!那位尊者現在住在哪裡?」

  在這麼說時,尊者梨犀達多保持沈默。

  「大德!是不是聖梨犀達多呢?」

  「是的,屋主!」

  「大德!請聖梨犀達多在麻七迦三達能被喜樂的蓭巴德葛林極喜樂,我將為聖梨犀達多的衣服、\twnr{施食}{196.0}、臥坐處、病人需物、醫藥必需品作熱心。」

  「屋主!善的被說。」

  那時,屋主質多歡喜、隨喜尊者梨犀達多所說後,以勝妙的\twnr{硬食、軟食}{153.0}親手款待上座比丘們,使之滿足。

  那時,已食、手離鉢的上座比丘們從座位起來後離開。

  那時,上座尊者對尊者梨犀達多說這個:

  「\twnr{好}{44.0}!梨犀達多學友!這個問題在你心中出現(變得清楚),這個問題不在我心中出現。梨犀達多學友!那樣的話,如果當像這樣的其它問題又到來,你就以這個方式應答。[\suttaref{SN.41.2}]」

  那時,尊者梨犀達多收起臥坐具、拿起衣鉢後,從麻七迦三達離開。凡他從麻七迦三達離開,\twnr{像這樣就已離開}{x543},沒再返回。



\sutta{4}{4}{摩訶葛神變經}{https://agama.buddhason.org/SN/sn.php?keyword=41.4}
  \twnr{有一次}{2.0},眾多\twnr{上座}{135.0}\twnr{比丘}{31.0}住在麻七迦三達的蓭巴德葛(檳榔青)林中。

  那時,\twnr{屋主}{103.0}質多去見上座比丘們。抵達後,向上座比丘們\twnr{問訊}{46.0}後,在一旁坐下。在一旁坐下的屋主質多對上座比丘們說這個:

  「\twnr{大德}{45.0}!請上座們同意我的明天在牛棚的食事。」

  上座比丘們以沈默狀態同意。

  那時,屋主質多知道上座比丘們同意後,從座位起來、向上座比丘們問訊、\twnr{作右繞}{47.0}後,離開。

  那時,那夜過後,上座比丘們午前時穿衣、拿起衣鉢後,去屋主質多的牛棚。抵達後,在設置的座位坐下。

  那時,屋主質多以勝妙的酥油乳粥親手款待上座比丘們,使之滿足。

  那時,已食、手離鉢的上座比丘們從座位起來後離開。

  屋主質多也[對家僕]說:「請你們捨棄剩餘的(乳粥)」後,在上座比丘們後面緊跟隨。

  當時是熱的、炎熱的,而那些已食那個食物的上座比丘們看起來像如身體正被融化般地行走。

  當時,尊者摩訶葛在那\twnr{僧團}{375.0}中是最新人。

  那時,尊者摩訶葛對上座尊者說這個:

  「上座大德!如果吹涼風、有雷鳴、\twnr{天空下毛毛雨}{385.0},\twnr{那就好了}{44.0}!」

  「摩訶葛\twnr{學友}{201.0}!如果吹涼風、有雷鳴、天空下毛毛雨,那就好了!」

  那時,尊者摩訶葛\twnr{造作像那樣的神通作為}{425.0}:依之吹涼風、有雷鳴、天空下毛毛雨。

  那時,屋主質多想這個:

  「凡在這僧團中最新人比丘,他的這個神通與威力是像這樣的!」

  那時,到達僧園後,尊者摩訶葛對上座尊者說這個:「上座大德!就這個程度足夠?」

  「摩訶葛學友!就這個程度足夠了:摩訶葛學友!就這個程度所做的,摩訶葛學友!就這個程度所供養的。」

  那時,上座比丘們依[自己的]住處走,尊者摩訶葛也走到自己的住處。

  那時,屋主質多去見尊者摩訶葛。抵達後,向尊者摩訶葛問訊後,在一旁坐下。在一旁坐下的屋主質多對尊者摩訶葛說這個:

  「大德!請\twnr{聖}{612.1}摩訶葛為我展現過人法的神通神變,那就好了!」

  「屋主!那樣的話,在玄關處鋪設上衣後,請你[在衣服上]使草束現出。」

  「是的,大德!」屋主質多回答尊者摩訶葛後,在玄關處鋪設上衣後,使草束現出。

  那時,尊者摩訶葛進入住處、上(施與)門拴後,造作像那樣的神通作為:依之火焰經\twnr{鑰匙孔}{x544}與門閂中間出來後,使草燃燒,不使上衣燃燒。

  那時,屋主質多拍打上衣後,驚慌、\twnr{身毛豎立}{152.0}地在一旁站立。

  那時,尊者摩訶葛從住處出來後,對屋主質多說這個:「屋主!就這個程度足夠?」

  「大德摩訶葛!就這個程度足夠了:大德摩訶葛!就這個程度所做的,大德摩訶葛!就這個程度所供養的。

  大德!請聖摩訶葛在麻七迦三達能被喜樂的蓭巴德葛林極喜樂,我將為聖摩訶葛的衣服、\twnr{施食}{196.0}、臥坐處、病人需物、醫藥必需品作熱心。」

  「屋主!善的被說。」

  那時,尊者摩訶葛收起臥坐具、拿起衣鉢後,從麻七迦三達離開。凡他從麻七迦三達離開,像這樣就已離開,沒再返回[\suttaref{SN.41.3}]。



\sutta{5}{5}{迦摩浮經第一}{https://agama.buddhason.org/SN/sn.php?keyword=41.5}
  \twnr{有一次}{2.0},\twnr{尊者}{200.0}迦摩浮住在麻七迦三達的蓭巴德葛(檳榔青)林中。

  那時,\twnr{屋主}{103.0}質多去見尊者迦摩浮。抵達後,向尊者迦摩浮\twnr{問訊}{46.0}後,在一旁坐下。尊者迦摩浮對在一旁坐下的屋主質多說這個:

  「屋主!這被說:

  『各部分無缺點的、白色篷子的,單輪輻的二輪車轉動,

   請看來者是無惱亂者,已切斷流者、無繫縛者。』[\ccchref{Ud.65}{https://agama.buddhason.org/Ud/dm.php?keyword=65}]

  屋主!對這個簡要地說的義理,應該怎樣詳細地被看見呢?」

  「\twnr{大德}{45.0}!這被\twnr{世尊}{12.0}說嗎?」

  「是的,屋主!」

  「大德!那樣的話,請你等待片刻,直到我觀察它的義理。」

  那時,屋主質多存在片刻沈默後,對尊者迦摩浮說這個:

  「大德!『各部分無缺陷的』,這是諸戒的同義語。

  大德!『白色篷子的』,這是解脫的同義語。

  大德!『單輪輻的』,這是念的同義語。

  大德!『轉動』,這是前進返回的同義語。

  大德!『二輪車』,這是這父母生成的、米粥積聚的、無常-塗身-\twnr{按摩}{967.0}-破壞-分散法的\twnr{四大}{646.0}身的同義語。

  大德!貪是惱亂;瞋是惱亂;癡是惱亂。對漏已滅盡\twnr{比丘}{31.0},那些已被捨斷,根已被切斷,\twnr{[如]已斷根的棕櫚樹}{147.1},\twnr{成為非有}{408.0},\twnr{為未來不生之物}{229.0},因此,漏已滅盡比丘被稱為『無惱亂者』。

  大德!『來者』,這是阿羅漢的同義語。

  大德!『流』,這是渴愛的同義語。對漏已滅盡比丘,那些已被捨斷,根已被切斷,[如]已斷根的棕櫚樹,成為非有,為未來不生之物,因此,漏已滅盡比丘被稱為『已切斷流者』。

  大德!貪是繫縛;瞋是繫縛;癡是繫縛。對漏已滅盡比丘,那些已被捨斷,根已被切斷,[如]已斷根的棕櫚樹,成為非有,為未來不生之物,因此,漏已滅盡比丘被稱為『無繫縛者』。

  大德!像這樣,凡那個被世尊說:

  『各部分無缺點的、白色篷子的,單輪輻的二輪車轉動,

   請看來者是無惱亂者,已切斷流者、無繫縛者。』

  大德!我對這個被世尊簡要地說的,這樣詳細地了知義理。」

  「屋主!是你的利得,屋主!\twnr{是你的善得的}{350.0}:你的慧眼走入甚深的佛語中。」



\sutta{6}{6}{迦摩浮經第二}{https://agama.buddhason.org/SN/sn.php?keyword=41.6}
  \twnr{有一次}{2.0},\twnr{尊者}{200.0}迦摩浮住在麻七迦三達的蓭巴德葛(檳榔青)林中。

  那時,\twnr{屋主}{103.0}質多去見尊者迦摩浮。抵達後,在一旁坐下。在一旁坐下的屋主質多對尊者迦摩浮說這個:

  「\twnr{大德}{45.0}!有幾種行呢?」

  「屋主!有三種行:身行、語行、心行。」

  「\twnr{好}{44.0}!大德!」屋主質多歡喜、隨喜尊者迦摩浮所說後,更進一步問尊者迦摩浮問題:

  「大德!那麼,什麼是身行?什麼是語行?什麼是心行?」

  「屋主!入息出息是身行,尋伺是語行,想與受是心行。」

  「好!大德!」屋主質多……(中略)更進一步問尊者迦摩浮問題:

  「大德!那麼,為何入息出息是身行?為何尋伺是語行?為何想與受是心行?」

  「屋主!入息出息是屬於身體的,這些法是依靠身體(被身體束縛)的,因此入息出息是身行。屋主!先尋後、伺後,之後破開言語,因此尋伺是語行。想與受是屬於心的,這些法是依靠心的,因此想與受是心行。」

  「好!大德!」……(中略)更進一步問尊者迦摩浮問題:

  「大德!那麼,怎樣有\twnr{想受滅}{416.0}\twnr{等至}{129.0}?」

  「屋主!進入想受滅的\twnr{比丘}{31.0}不這麼想:『我將進入想受滅。』或『我進入想受滅。』或『我已進入想受滅。』那時,\twnr{如之前他的心所修習的那樣}{x545},導引到那樣的狀態。」

  「好!大德!」……(中略)更進一步問尊者迦摩浮問題:

  「大德!那麼,進入想受滅比丘的哪些法第一地被滅:身行,或者語行,或者心行?」

  「屋主!進入想受滅比丘的語行第一地被滅,之後是身行,之後是心行。」

  「好!大德!」……(中略)更進一步問尊者迦摩浮問題:

  「大德!凡這已死的死者,與凡這進入想受滅的比丘,什麼是他們的差異?」

  「屋主!凡這已死的死者,他的身行被滅、被止息,語行被滅、被止息,心行被滅、被止息,壽命被遍滅盡,\twnr{暖}{x546}被熄滅,諸根被破壞。屋主!而凡這進入想受滅的比丘,也:他的身行被滅、被止息,語行被滅、被止息,心行被滅、被止息,[但]壽命未被遍滅盡,暖未被熄滅,諸根被變得明淨。屋主!凡這已死的死者,與凡這進進入想受滅的比丘,這是他們的差異。」[\ccchref{MN.43}{https://agama.buddhason.org/MN/dm.php?keyword=43}, 457段]

  「好!大德!」……(中略)更進一步問尊者迦摩浮問題:

  「大德!那麼,怎樣從想受滅等至出定呢?」

  「屋主!從想受滅等至出定的比丘不這麼想:『我將從想受滅等至出定。』或『我從想受滅等至出定。』或『我已從想受滅等至出定。』那時,如之前他的心所修習的那樣,導引到那樣的狀態。」

  「好!大德!」……(中略)更進一步問尊者迦摩浮問題:

  「大德!那麼,從想受滅等至出定比丘的哪些法第一地生起:身行,或者語行,或者心行?」

  「屋主!從想受滅等至出定比丘的心行第一地生起,之後是身行,之後是語行。」

  「好!大德!」……(中略)更進一步問尊者迦摩浮問題:

  「大德!那麼,多少觸觸達從想受滅等至出定的比丘呢?」

  「屋主!三種觸觸達從想受滅等至出定的比丘:\twnr{空觸}{x547}、無相觸、\twnr{無願觸}{x548}。」

  「好!大德!」……(中略)更進一步問尊者迦摩浮問題:

  「大德!那麼,從想受滅等至出定比丘的心是傾向什麼的、斜向什麼的、坡斜向什麼的?」

  「屋主!從想受滅等至出定比丘的心是傾向\twnr{遠離}{x549}的、斜向遠離的、坡斜向遠離的。」[\ccchref{MN.44}{https://agama.buddhason.org/MN/dm.php?keyword=44}, 463段]

  「好!大德!」屋主質多歡喜、隨喜尊者迦摩浮所說後,更進一步問尊者迦摩浮問題:

  「大德!那麼,為了想受滅等至,幾法是多助益的呢?」

  「屋主!你確實問了應該第一個被問的問題,但我仍將為你解說。為了想受滅等至,二法是多助益的:\twnr{止與觀}{178.0}。」



\sutta{7}{7}{苟達多經}{https://agama.buddhason.org/SN/sn.php?keyword=41.7}
  \twnr{有一次}{2.0},\twnr{尊者}{200.0}苟達多住在麻七迦三達的蓭巴德葛(檳榔青)林中。

  那時,\twnr{屋主}{103.0}質多去見尊者苟達多。抵達後,向尊者苟達多\twnr{問訊}{46.0}後,在一旁坐下。尊者苟達多對在一旁坐下的屋主質多說這個:

  「屋主!凡這\twnr{無量心解脫}{805.0}、凡這\twnr{無所有心解脫}{806.0}、凡這\twnr{空心解脫}{807.0}、凡這\twnr{無相心解脫}{265.1},這些法是不同義理、不同辭句?或者是一種義理,僅辭句不同?」

  「\twnr{大德}{45.0}!有法門,由於該法門,這些法是不同義理,也不同辭句。大德!又,有法門,由於該法門,這些法是一種義理,僅辭句不同。

  大德!而什麼法門,由於該法門,這些法是不同義理,也不同辭句呢?

  大德!這裡,\twnr{比丘}{31.0}以與慈俱行之心遍滿一方後而住,像這樣第二的,像這樣第三的,像這樣第四的,像這樣上下、橫向、到處,以對一切如對自己,以與慈俱行的、廣大的、變大的、無量的、無怨恨的、無瞋害的心遍滿全部世間後而住。以與悲俱行之心……(中略)以與喜悅俱行之心……(中略)以與\twnr{平靜}{228.0}俱行之心遍滿一方後而住,像這樣第二的,像這樣第三的,像這樣第四的,像這樣上下、橫向、到處,以對一切如對自己,以與平靜俱行的、廣大的、變大的、無量的、無怨恨的、無瞋害的心遍滿全部世間後而住。大德!這被稱為無量心解脫。

  大德!而什麼是無所有心解脫呢?大德!這裡,比丘超越一切識無邊處後[而知]:『什麼都沒有』,\twnr{進入後住於}{66.0}\twnr{無所有處}{533.0},大德!這被稱為無所有心解脫。

  大德!而什麼是空心解脫呢?大德!這裡,到\twnr{林野}{142.0}的,或到樹下的,或到空屋的比丘像這樣深慮:『以我或以我所,這是空。』大德!這被稱為空心解脫。

  大德!而什麼是無相心解脫呢?大德!這裡,比丘以一切相的不作意,進入後住於無相心定,大德!這被稱為無相心解脫。

  大德!這是法門,由於該法門,這些法是不同義理,也不同辭句。

  大德!而什麼法門,由於該法門,這些法是一種義理,僅辭句不同呢?

  大德!\twnr{貪是衡量的作者}{x550}(量因);瞋是衡量的作者;癡是衡量的作者,對漏已滅盡比丘,那些已被捨斷,根已被切斷,\twnr{[如]已斷根的棕櫚樹}{147.1},\twnr{成為非有}{408.0},\twnr{為未來不生之物}{229.0}。大德!無量心解脫之所及,\twnr{不動心解脫}{808.0}被告知為它們中第一的,又,那個不動心解脫,以貪是空的,以瞋是空的,以癡是空的。

  大德!\twnr{貪是件東西}{x551}(障礙);瞋是件東西;癡是件東西,對漏已滅盡比丘,那些已被捨斷,根已被切斷,[如]已斷根的棕櫚樹,成為非有,為未來不生之物。大德!無所有心解脫之所及,不動心解脫被告知為它們中第一的,又,那個不動心解脫,以貪是空的,以瞋是空的,以癡是空的。

  大德!\twnr{貪是相的作者}{x552}(相因);瞋是相的作者;癡是相的作者,對漏已滅盡比丘,那些已被捨斷,根已被切斷,[如]已斷根的棕櫚樹,成為非有,為未來不生之物。大德!無相心解脫之所及,不動心解脫被告知為它們中第一的,又,那個不動心解脫,以貪是空的,以瞋是空的,以癡是空的。

  大德!這是法門,由於該法門,這些法是一種義理,僅辭句不同。」[\ccchref{MN.43}{https://agama.buddhason.org/MN/dm.php?keyword=43}, 459段]

  「屋主!是你的利得,屋主!\twnr{是你的善得的}{350.0}:你的慧眼走入甚深的佛語中。」



\sutta{8}{8}{尼乾陀若提子經}{https://agama.buddhason.org/SN/sn.php?keyword=41.8}
  當時,尼乾陀若提子與大尼乾陀群眾一起抵達麻七迦三達。

  \twnr{屋主}{103.0}質多聽聞:「尼乾陀若提子與大尼乾陀群眾一起抵達麻七迦三達。」

  那時,屋主質多與眾多\twnr{優婆塞}{98.0}一起去見尼乾陀若提子。抵達後,與尼乾陀若提子一起互相問候。交換應該被互相問候的友好交談後,在一旁坐下。尼乾陀若提子對在一旁坐下的屋主質多說這個:

  「屋主!你信\twnr{沙門}{29.0}\twnr{喬達摩}{80.0}[所說]的有\twnr{無尋無伺定}{x553},有尋伺之滅嗎?」

  「\twnr{大德}{45.0}!在這裡,我不以信走到\twnr{世尊}{12.0}的有無尋無伺定,有尋伺之滅。」

  在這麼說時,尼乾陀若提子\twnr{仰視}{x554}後說這個:

  「請尊者們看這位:這位屋主質多是多麼正直,這位屋主質多是多麼不狡猾的,這位屋主質多是多麼不欺騙的,凡會想尋伺能被滅者,會想那個風能被網誘捕,或凡會想尋伺能被滅者,會想那個恒河的水流能被自己的拳頭抑止。」

  「大德!你怎麼想,哪個比較勝妙呢:智或信?」

  「屋主!智正比信更勝妙。」

  「大德!只要我希望,就從離諸欲後,從離諸不善法後,我\twnr{進入後住於}{66.0}有尋、\twnr{有伺}{175.0},\twnr{離而生喜、樂}{174.0}的初禪;大德!只要我希望,從尋與伺的平息……(中略)我進入後住於[無尋、無伺,定而生喜、樂的]第二禪;大德!只要我希望,以喜的\twnr{褪去}{77.0}……(中略)我進入後住於[凡聖者們告知『他是平靜者、具念者、\twnr{安樂住者}{317.0}』]的第三禪;大德!只要我希望,從樂的捨斷……(中略)我進入後住於[不苦不樂,\twnr{平靜、念遍純淨}{494.0}的]第四禪。

  大德!那個這麼知的、這麼見的我將以信走到哪位其他沙門、婆羅門的有無尋無伺定,有尋伺之滅呢?」

  在這麼說時,尼乾陀若提子仰視自己的群眾後說這個:

  「請尊者們看這位:這位屋主質多是多麼不正直,這位屋主質多是多麼狡猾的,這位屋主質多是多麼欺騙的。」

  「大德!就現在,我們這麼了知:『請尊者們看這位:這位屋主質多是多麼正直,這位屋主質多是多麼不狡猾的,這位屋主質多是多麼不欺騙的。』被你說,大德!而且,就現在,我們這麼了知:『請尊者們看這位:這位屋主質多是多麼不正直,這位屋主質多是多麼狡猾的,這位屋主質多是多麼欺騙的。』被說。大德!如果你前面的是真實的,你後面的[就]是錯誤的;又,如果你前面的是錯誤的,你後面的[就]是真實的。

  大德!又,這十如法的問題到來,當你能了知那些義理,那時你應該與尼乾陀眾一起反擊我,一個問題,一個說示(概要),\twnr{一個解答}{x555};二個問題,二個說示,二個解答;三個問題,三個說示,三個解答;四個問題,四個說示,四個解答;五個問題,五個說示,五個解答;六個問題,六個說示,六個解答;七個問題,七個說示,七個解答;八個問題,八個說示,八個解答;九個問題,九個說示,九個解答;十個問題,十個說示,十個解答。」

  那時,屋主質多沒詢問(羅馬拼音版)尼乾陀若提子這十如法的問題後,從座位起來後離開。



\sutta{9}{9}{裸行者迦葉經}{https://agama.buddhason.org/SN/sn.php?keyword=41.9}
  當時,\twnr{屋主}{103.0}質多以前在家的朋友裸行者迦葉抵達麻七迦三達。

  屋主質多聽聞:「我們以前在家的朋友裸行者迦葉抵達麻七迦三達。」

  那時,屋主質多去見裸行者迦葉。抵達後,與裸行者迦葉一起互相問候。交換應該被互相問候的友好交談後,在一旁坐下。在一旁坐下的屋主質多對裸行者迦葉說這個:

  「\twnr{大德}{45.0}迦葉!已出家多久呢?」

  「屋主!我已出家三十年。」

  「大德!那麼,經這三十年,你有任何\twnr{足以為聖者智見特質}{473.0}的\twnr{過人法}{205.0}、\twnr{安樂住}{156.0}被到達嗎?」

  「屋主!經這三十年,沒有我的任何足以為聖者智見特質的過人法、安樂住被到達,除了裸行、剃頭、清理座位的刷子外。」

  在這麼說時,屋主質多對裸行者迦葉說這個:

  「實在不可思議啊,\twnr{先生}{202.0}!實在\twnr{未曾有}{206.0}啊,先生!法的善說情況:確實是因為三十年沒有任何足以為聖者智見特質的過人法、安樂住被到達,除了裸行、剃頭、清理座位的刷子外!」

  「屋主!那麼,你已走入\twnr{優婆塞}{98.0}多久呢?」

  「大德!那麼,我也已走入優婆塞三十年。」

  「屋主!那麼,經這三十年,你有任何足以為聖者智見特質的過人法、安樂住被到達嗎?」

  「大德!\twnr{在家者也會有}{x556}:大德!只要我希望,就從離諸欲後,從離諸不善法後,我\twnr{進入後住於}{66.0}有尋、\twnr{有伺}{175.0},\twnr{離而生喜、樂}{174.0}的初禪;大德!只要我希望,從尋與伺的平息……(中略)我進入後住於[無尋、無伺,定而生喜、樂的]第二禪;大德!只要我希望,以喜的\twnr{褪去}{77.0}……(中略)我進入後住於[凡聖者們告知『他是平靜者、具念者、\twnr{安樂住者}{317.0}』]的第三禪;大德!只要我希望,從樂的捨斷……(中略)我進入後住於[不苦不樂,\twnr{平靜、念遍純淨}{494.0}的]第四禪。

  大德!又,如果我比\twnr{世尊}{12.0}先死,這並\twnr{非不可思議}{924.0}:世尊這麼\twnr{記說}{179.0}我:『沒有那個結,以該結屋主質多被結縛,再來到這個世間。』」

  在這麼說時,裸行者迦葉對屋主質多說這個:

  「實在不可思議啊,先生!實在未曾有啊,先生!法的善說情況:確實是因為\twnr{白衣}{277.0}在家人將證得像這樣足以為聖者智見特質的過人法、安樂住。屋主!願我得到在這法、律中出家,願得到\twnr{受具足戒}{124.0}。」

  那時,屋主質多帶裸行者迦葉後去見上座\twnr{比丘}{31.0}們。抵達後,對上座比丘們說這個:

  「大德!這位是我以前在家的朋友裸行者迦葉,請上座們使他出家、使受具足戒,我將為他的衣服、\twnr{施食}{196.0}、臥坐處、病人需物、醫藥必需品作熱心。」

  那時,裸行者迦葉得到在這法律中出家,得到具足戒。

  還有,已受具足戒不久,住於單獨的、隱離的、不放逸的、熱心的、自我努力的\twnr{尊者}{200.0}裸行者迦葉不久就以證智自作證後,在當生中\twnr{進入後住於}{66.0}凡\twnr{善男子}{41.0}們為了利益正確地\twnr{從在家出家成為無家者}{48.0}的那個無上梵行結尾,他證知:「出生已盡,梵行已完成,\twnr{應該被作的已作}{20.0},不再有此處[輪迴]的狀態。」



\sutta{10}{10}{看病人經}{https://agama.buddhason.org/SN/sn.php?keyword=41.10}
  當時,\twnr{屋主}{103.0}質多是生病者、受苦者、重病者。

  那時,眾多園林天神、樹林天神、樹木天神、居住在藥草與大樹中的天神們集合、會合後,對屋主質多說這個:

  「屋主!請你祈願:『願我為\twnr{未來世}{308.0}的\twnr{轉輪王}{278.0}。』」

  在這麼說時,屋主質多對那些園林天神、樹林天神、樹木天神、居住在藥草與大樹中的天神們說這個:

  「那也是無常的,那也是不堅固的,捨棄後那也應該被行。」

  在這麼說時,屋主質多的朋友、同事、親族、血親對屋主質多說這個:

  「\twnr{聖}{612.1}男子!請你使念現起,不要胡言亂語。」

  「那個我說什麼,你們會這麼說我:『聖男子!請你使念現起,不要胡言亂語。』呢?」

  「聖男子!你這麼說:『那也是無常的,那也是不堅固的,捨棄後那也應該被行。』」

  「但像那樣是因為諸園林天神、樹林天神、樹木天神、居住在藥草與大樹中的天神們對我這麼說:『屋主!請你祈願:「願我為未來世的轉輪王。」』那個我這麼說:『那也是無常的……(中略)捨棄後那也應該被行。』」

  「聖男子!當看見什麼理由時,那些園林天神、樹林天神、樹木天神、居住在藥草與大樹中的天神們這麼說:『屋主!請你祈願:「願我為未來世的轉輪王。」』」

  「那些園林天神、樹林天神、樹木天神、居住在藥草與大樹中的天神們這麼想:『這位屋主質多是持戒者、\twnr{善法者}{601.1},如果他祈願:「願我為未來世的轉輪王。」對他,這位持戒者心的誓願以清淨狀態將成功,如法者將獲得(錫蘭版)如法之果。』當看見這個理由時,那些園林天神、樹林天神、樹木天神、居住在藥草與大樹中的天神們這麼說:『屋主!請你祈願:「願我為未來世的轉輪王。」』那個我這麼說:『那也是無常的,那也是不堅固的,捨棄後那也應該被行。』」

  「聖男子!那樣的話,請你也教誡我們。」

  「因此,應該被你們這麼學:

  『我們將成為在佛上具備\twnr{不壞淨}{233.0}者:「像這樣,那位\twnr{世尊}{12.0}是\twnr{阿羅漢}{5.0}、\twnr{遍正覺者}{6.0}、\twnr{明行具足者}{7.0}、\twnr{善逝}{8.0}、\twnr{世間知者}{9.0}、\twnr{應該被調御人的無上調御者}{10.0}、\twnr{天-人們的大師}{11.0}、\twnr{佛陀}{3.0}、世尊。」

  我們將成為在法上具備不壞淨者:「被世尊善說的法是直接可見的、即時的、請你來看的、能引導的、\twnr{應該被智者各自經驗的}{395.0}。」

  我們將成為在\twnr{僧團}{375.0}上具備不壞淨者:「世尊的弟子僧團是\twnr{善行者}{518.0},世尊的弟子僧團是正直行者,世尊的弟子僧團是真理行者,世尊的弟子僧團是\twnr{方正行者}{764.0},即:四雙之人、\twnr{八輩之士}{347.0},這世尊的弟子僧團應該被奉獻、應該被供奉、應該被供養、應該被\twnr{合掌}{377.0},為世間的無上福田。」』

  又,凡在家族中任何能施與法(物),全都將成為與持戒者、善法者無差別的,應該被你們這麼學。」

  那時,屋主質多在佛上、在法上、在僧團上、在施捨上勸導朋友、同事、親族、血親後命終。

  質多羅相應完成,其\twnr{攝頌}{35.0}:

  「結縛、二則梨犀達多,摩訶葛與迦摩浮,

   苟達多與尼乾陀,裸行者與看病人。」





\page

\xiangying{42}{村長相應}
\sutta{1}{1}{兇惡經}{https://agama.buddhason.org/SN/sn.php?keyword=42.1}
  起源於舍衛城。

  那時,村長兇惡去見世尊。抵達後,向世尊\twnr{問訊}{46.0}後,在一旁坐下。在一旁坐下的村長兇惡對世尊說這個:

  「\twnr{大德}{45.0}!什麼因、什麼\twnr{緣}{180.0},以那個,這裡某人就名為(就走到稱呼)『兇惡者、兇惡者』呢?又,什麼因、什麼緣,這裡某人就名為『柔和者、柔和者』呢?」

  「村長!這裡,某人的貪未被捨斷,以貪的未被捨斷狀態其他人使之發怒,當被其他人使之發怒時,他顯露憤怒,他就名為『兇惡者』。瞋未被捨斷,以瞋的未被捨斷狀態其他人使之發怒,當被其他人使之發怒時,他顯露憤怒,他就名為『兇惡者』。癡未被捨斷,以癡的未被捨斷狀態其他人使之發怒,當被其他人使之發怒時,他顯露憤怒,他就名為『兇惡者』。村長!這是因、這是緣,以那個,這裡某人就名為『兇惡者、兇惡者』。

  村長!又,這裡,某人的貪被捨斷,以貪的被捨斷狀態其他人不使之發怒,當被其他人使之發怒時,他不顯露憤怒,他就名為『柔和者』。瞋已被捨斷,以瞋的被捨斷狀態其他人不使之發怒,當被其他人使之發怒時,他不顯露憤怒,他就名為『柔和者』。癡已被捨斷,以癡已被捨斷的已被捨斷狀態其他人不使之發怒,當被其他人使之發怒時,他不顯露憤怒,他就名為『柔和者』。村長!這是因、這是緣,這裡某人就名為『柔和者、柔和者』。」

  在這麼說時,村長兇惡對世尊說這個:

  「大德!太偉大了,大德!太偉大了,大德!猶如扶正顛倒的,或揭開隱藏的,或告知迷路者的道路,或在黑暗中持燈火:『有眼者們看見諸色。』同樣的,法被世尊以種種\twnr{法門}{562.0}說明。大德!這個我\twnr{歸依}{284.0}世尊、法、\twnr{比丘僧團}{65.0},請世尊記得我為\twnr{優婆塞}{98.0},從今天起\twnr{已終生歸依}{64.0}。」



\sutta{2}{2}{達拉普達經}{https://agama.buddhason.org/SN/sn.php?keyword=42.2}
  \twnr{有一次}{2.0},\twnr{世尊}{12.0}住在王舍城栗鼠飼養處的竹林中。

  那時,表演者村長(團長)達拉普達去見世尊。抵達後,向世尊\twnr{問訊}{46.0}後,在一旁坐下。在一旁坐下的表演者村長達拉普達對世尊說這個:

  「\twnr{大德}{45.0}!這被我從以前的老師、老師的老師表演者的講說中聽聞:『凡那位表演者在舞台中,在慶祝會中,\twnr{以真真假假}{x557}使人笑、歡樂,他以身體的崩解,死後往生為歡笑天們的共住狀態。』這裡,世尊怎麼說?」

  「夠了!村長!別理會這個,不要問我這個。」

  第二次,表演者村長達拉普達又對世尊說這個:

  「大德!這被我從以前的老師、老師的老師表演者的講說中聽聞:『表演者在舞台中,在慶祝會中,以真真假假使人笑、歡樂,他以身體的崩解,死後往生為歡笑天們的共住狀態。』這裡,世尊怎麼說?」

  「夠了!村長!別理會這個,不要問我這個。」

  第三次,表演者村長達拉普達又對世尊說這個:

  「大德!這被我從以前的老師、老師的老師表演者的講說中聽聞:『表演者在舞台中,在慶祝會中,以真真假假使人笑、歡樂,他以身體的崩解,死後往生為歡笑天們的共住狀態。』這裡,世尊怎麼說?」

  「村長!\twnr{我確實沒得到你的[理解]}{218.1}:『夠了!村長!別理會這個,不要問我這個。』但,我仍將回答你。

  村長!眾生在過去(前世)未離貪,被貪的繫縛繫縛,那些表演者在舞台中,在慶祝會中,以凡能被染之法為他們更多量地服務(堆積);村長!眾生在過去未離瞋,被瞋的繫縛繫縛,那些表演者在舞台中,在慶祝會中,以凡能被憤怒之法為他們更多量地服務;村長!眾生在過去未離癡,被癡的繫縛繫縛,那些表演者在舞台中,在慶祝會中,以凡能被使變愚癡之法為他們更多量地服務,他以自己成為陶醉的、放逸的,使他人陶醉、放逸後,以身體的崩解,死後往生\twnr{名叫歡笑的地獄那裡}{x558}。

  又,如果他有這樣的見:『表演者在舞台中,在慶祝會中,以真真假假使人笑、歡樂,他以身體的崩解,死後往生為歡笑天們的共住狀態。』那是他的邪見。村長!又,對邪見者,我說,男子個人有兩個趣處中某個趣處:地獄或畜生界。」

  在這麼說時,表演者村長達拉普達哭泣、流淚。

  「村長!我沒得到你的這個[理解]:『夠了!村長!別理會這個,不要問我這個。』」

  「大德!我不哭泣這個:凡世尊對我這麼說。大德!而是我被以前的老師、老師的老師表演者長久欺騙、欺瞞、誘拐:『表演者在舞台中,在慶祝會中,以真真假假使人笑、歡樂,他以身體的崩解,死後往生為歡笑天們的共住狀態。』

  大德!太偉大了,大德!太偉大了,大德!猶如扶正顛倒的,或揭開隱藏的,或告知迷路者的道路,或在黑暗中持燈火:『有眼者們看見諸色。』同樣的,法被世尊以種種\twnr{法門}{562.0}說明。大德!這個我\twnr{歸依}{284.0}世尊、法、\twnr{比丘僧團}{65.0}。大德!願我得到在世尊的面前出家,願我\twnr{得到具足戒}{124.1}。」

  那時,表演者村長達拉普達得到在世尊的面前出家、受具足戒。

  還有,已受具足戒不久……(中略)然後\twnr{尊者}{200.0}達拉普達成為眾\twnr{阿羅漢}{5.0}之一。



\sutta{3}{3}{戰士經}{https://agama.buddhason.org/SN/sn.php?keyword=42.3}
  那時,戰士村長(領導者)去見\twnr{世尊}{12.0}。抵達後……(中略)戰士村長對世尊說這個:

  「\twnr{大德}{45.0}!這被我從以前老師、老師的老師戰士的講說中聽聞:『凡那位戰士在戰鬥中竭力、\twnr{努力}{x559},其他人殺、處死竭力、努力的他,他以身體的崩解,死後往生為\twnr{被其他人征服天們的共住狀態}{x560}。』這裡,世尊怎麼說?」

  「夠了!村長!別理會這個,不要問我這個。」

  第二次……(中略)第三次,戰士村長又對世尊說這個:

  「大德!這被我從以前老師、老師的老師戰士的講說中聽聞:『凡那位戰士在戰鬥中竭力、努力,其他人殺、處死竭力、努力的他,他以身體的崩解,死後往生為被其他人征服天們的共住狀態。』這裡,世尊怎麼說?」

  「村長!\twnr{我確實沒得到你的[理解]}{218.1}:『夠了!村長!別理會這個,不要問我這個。』但,我仍將回答你。

  村長!凡那位戰士在戰鬥中竭力、努力,他的那個心在之前已被惡作、\twnr{惡意向捉住}{x561}:令這些眾生被殺,或令被捕捉,或令被消滅,或令被滅亡,或他們像這樣不存在。其他人殺、處死竭力、努力的他,他以身體的崩解,死後往生\twnr{名叫被其他人征服的地獄那裡}{x562}。

  又,如果他有這樣的見:『凡那位戰士在戰鬥中竭力、努力,其他人殺、處死竭力、努力的他,他以身體的崩解,死後往生為被其他人征服天們的共住狀態。』那是他的邪見,村長!又,對邪見者,我說,男子個人有兩個趣處中某個趣處:地獄或畜生界。」

  在這麼說時,戰士村長哭泣、流淚。

  「村長!我沒得到你的這個[理解]:『夠了!村長!別理會這個,不要問我這個。』」

  「大德!我不哭泣這個:凡世尊對我這麼說。大德!而是我被以前的老師、老師的老師戰士長久欺騙、欺瞞、誘拐:『凡那位戰士在戰鬥中竭力、努力,其他人殺、處死竭力、努力的他,他以身體的崩解,死後往生為被其他人征服天們的共住狀態。』

  太偉大了,大德!……(中略)從今天起\twnr{已終生歸依}{64.0}。」



\sutta{4}{4}{騎象者經}{https://agama.buddhason.org/SN/sn.php?keyword=42.4}
  那時,騎象者村長(領導者)去見\twnr{世尊}{12.0}。抵達後……(中略)「……從今天起\twnr{已終生歸依}{64.0}。」



\sutta{5}{5}{騎馬者經}{https://agama.buddhason.org/SN/sn.php?keyword=42.5}
  那時,騎馬者村長(領導者)去見\twnr{世尊}{12.0}。抵達後,在一旁坐下。在一旁坐下的騎馬者村長對世尊說這個:

  「\twnr{大德}{45.0}!這被我從以前的老師、老師的老師騎馬者的講說中聽聞:『凡那位騎馬者\twnr{在戰鬥中竭力}{x563}、努力,其他人殺、處死竭力、努力的他,他以身體的崩解,死後往生為被其他人征服天們的共住狀態。』這裡,世尊怎麼說?」

  「夠了!村長!別理會這個,不要問我這個。」

  第二次……(中略)第三次,騎馬者村長又對世尊說這個:

  「大德!這被我從以前的老師、老師的老師騎馬者的講說中聽聞:『凡那位騎馬者在戰鬥中竭力、努力,其他人殺、處死竭力、努力的他,他以身體的崩解,死後往生為被其他人征服天們的共住狀態。』這裡,世尊怎麼說?」

  「村長!\twnr{我確實沒得到你的[理解]}{218.1}:『夠了!村長!別理會這個,不要問我這個。』但,我仍將回答你。

  村長!騎馬者在戰鬥中竭力、努力,他的那個心在之前已被惡作、惡意向捉住:令這些眾生被殺,或令被捕捉,或令被消滅,或令被滅亡,或他們像這樣不存在。其他人殺、處死竭力、努力的他,他以身體的崩解,死後往生名叫被其他人征服的地獄那裡。

  又,如果他有這樣的見:『凡那位騎馬者在戰鬥中竭力、努力,其他人殺、處死竭力、努力的他,他以身體的崩解,死後往生為被其他人征服天們的共住狀態。』那是他的邪見,村長!又,對邪見者,我說,男子個人有兩個趣處中某個趣處:地獄或畜生界。」

  在這麼說時,騎馬者村長哭泣、流淚。

  「村長!我沒得到你的這個[理解]:『夠了!村長!別理會這個,不要問我這個。』」

  「大德!我不哭泣這個:凡世尊對我這麼說。大德!而是我被以前老師、老師的老師的騎馬者長久欺騙、欺瞞、誘拐:『凡那位騎馬者在戰鬥中竭力、努力,其他人殺、處死竭力、努力的他,他以身體的崩解,死後往生為被其他人征服天們的共住狀態。』

  太偉大了,大德!……(中略)從今天起\twnr{已終生歸依}{64.0}。」



\sutta{6}{6}{阿西邦大葛之子經}{https://agama.buddhason.org/SN/sn.php?keyword=42.6}
  \twnr{有一次}{2.0},\twnr{世尊}{12.0}住在那爛陀賣衣者的芒果園中。

  那時,村長阿西邦大葛之子去見世尊。抵達後,向世尊\twnr{問訊}{46.0}後,在一旁坐下。在一旁坐下的村長阿西邦大葛之子對世尊說這個:

  「\twnr{大德}{45.0}!持長口水瓶的、戴水草花環的、入水洗浴的、拜火的(侍奉火的)西部地方\twnr{婆羅門}{17.0}們,他們\twnr{確實使已死的死者離開}{x564},\twnr{確實使知}{x565},確實使進入天界,大德!但\twnr{世尊}{12.0}、\twnr{阿羅漢}{5.0}、\twnr{遍正覺者}{6.0}能夠像那樣做,依之全世界都以身體的崩解,死後往生\twnr{善趣}{112.0}、天界嗎?」

  「村長!那樣的話,就在這件事上我將反問你,你就如對你能接受的那樣回答它。村長!你怎麼想它:這裡,如果男子是殺生者、\twnr{未給予而取}{104.0}者、邪淫者、\twnr{妄語}{106.0}者、\twnr{離間語}{234.0}者、\twnr{粗惡語}{235.0}者、\twnr{雜穢語}{236.0}者、\twnr{貪婪者}{435.0}、有瞋害心者、邪見者,一大群人集合、會合後對他祈願、稱讚,\twnr{合掌}{377.0}繞行:『令這位男子以身體的崩解,死後往生善趣、天界!』村長!你怎麼想它:是否一大群人的祈願之因,或稱讚之因,或合掌繞行之因,那位男子以身體的崩解,死後往生善趣、天界呢?」

  「大德!這確實不是。」

  「村長!猶如男子投大的大石頭入深湖沼中,一大群人集合、會合後對它祈願、稱讚,合掌繞行:『大石頭先生!浮現!大石頭先生!浮起來!大石頭先生!浮到陸地來!』村長!你怎麼想它:是否一大群人的祈願之因,或稱讚之因,或合掌繞行之因,那塊大石頭浮現,或浮起來,或浮到陸地來呢?」

  「大德!這確實不是。」

  「同樣的,村長!凡那位男子是殺生者、未給予而取者、邪淫者、妄語者、離間語者、粗惡語者、雜穢語者、貪婪者、有瞋害心者、邪見者,即使一大群人集合、會合後對他祈願、稱讚,合掌繞行:『令這位男子以身體的崩解,死後往生善趣、天界!』那時,那位男子以身體的崩解,死後往生\twnr{苦界}{109.0}、\twnr{惡趣}{110.0}、\twnr{下界}{111.0}、地獄。

  村長!你怎麼想它:這裡,如果男子是離殺生者、離未給予而取者、離邪淫者、離妄語者、離離間語者、離粗惡語者、離雜穢語者、不貪婪者、無瞋害心者、正見者,一大群人集合、會合後對他祈願、稱讚,合掌繞行:『令這位男子以身體的崩解,死後往生苦界、惡趣、下界、地獄!』村長!你怎麼想它:是否一大群人的祈願之因,或稱讚之因,或合掌繞行之因,那位男子以身體的崩解,死後會往生苦界、惡趣、下界、地獄呢?」

  「大德!這確實不是。」

  「村長!猶如男子投酥陶瓶或油陶瓶入深水池後破裂,在那裡,凡是碎片或破片,它是走到向下的,而在那裡,凡是酥或油,它是走到向上的。一大群人集合、會合後對它祈願、稱讚,合掌繞行:『酥油先生!下沈!酥油先生!沈沒!酥油先生!向下走!』村長!你怎麼想它:是否一大群人的祈願之因,或稱讚之因,或合掌繞行之因,那個酥油沈下去,或沈沒,或向下走呢?」

  「大德!這確實不是。」

  「同樣的,村長!那位男子是離殺生者、離未給予而取者、離邪淫者、離妄語者、離離間語者、離粗惡語者、離雜穢語者、不貪婪者、無瞋害心者、正見者,即使一大群人集合、會合後對他祈願、稱讚,合掌繞行:『令這位男子以身體的崩解,死後往生苦界、惡趣、下界、地獄!』那時,那位男子以身體的崩解,死後往生善趣、天界。」

  在這麼說時,村長阿西邦大葛之子對世尊說這個:

  「太偉大了,大德!……(中略)從今天起\twnr{已終生歸依}{64.0}。」



\sutta{7}{7}{如田地經}{https://agama.buddhason.org/SN/sn.php?keyword=42.7}
  \twnr{有一次}{2.0},\twnr{世尊}{12.0}住在那爛陀賣衣者的芒果園中。

  那時,村長阿西邦大葛之子去見世尊。抵達後,向世尊\twnr{問訊}{46.0}後,在一旁坐下。在一旁坐下的村長阿西邦大葛之子對世尊說這個:「\twnr{大德}{45.0}!世尊住於\twnr{對一切活的生命類有憐愍的}{470.2},不是嗎?」

  「是的,村長!如來住於對一切活的生命類有憐愍的。」

  「大德!那樣的話,為何世尊為某些人徹底地教導法,不為某些人像那樣徹底地教導法呢?」

  「村長!那樣的話,就在這件事上我將反問你,你就如對你能接受的那樣回答它。村長!你怎麼想它:這裡,如果農夫\twnr{屋主}{103.0}有三塊田,一塊是最好的田,一塊是中等的田,一塊是下劣的荒地、含鹽的惡地田,村長!你怎麼想它:那位農夫屋主想要播種(住立種子),會在哪裡第一個播:在最好的那塊田?或在中等的那塊田?或在下劣的荒地、含鹽的惡地那塊田?」

  「大德!那位農夫屋主想要播種,會播在最好的那塊田那裡,在那裡播後,會播在中等的那塊田那裡,在那裡播後,會播在下劣的荒地、含鹽的惡地那塊田那裡,或不會播,那是什麼原因?至少也將成為牛的食物。」

  「村長!猶如那塊最好的田,對我來說,\twnr{比丘}{31.0}、比丘尼正是這樣。我為他們教導法:開頭是善的、中間是善的、結尾是善的;\twnr{有意義的}{81.0}、\twnr{有文字的}{82.0}法,我說明完全圓滿、\twnr{遍純淨的梵行}{483.0}。那是什麼原因?村長!因為這些住於以我為依靠、以我為庇護、以我為救護、以我為歸依。

  村長!猶如那塊中等的田,對我來說,\twnr{優婆塞}{98.0}、\twnr{優婆夷}{99.0}正是這樣。我也為他們教導法:開頭是善的、中間是善的、結尾是善的;有意義的、有文字的法,我說明完全圓滿、遍純淨的梵行,那是什麼原因?村長!因為些住住於以我為依靠、以我為庇護、以我為救護、以我為歸依。

  村長!猶如那塊下劣的荒地、含鹽的惡地田,對我來說,其他外道\twnr{沙門}{29.0}、\twnr{婆羅門}{17.0}\twnr{遊行者}{79.0}們正是這樣。我也為他們教導法:開頭是善的、中間是善的、結尾是善的;有意義的、有文字的法,我說明完全圓滿、遍純淨的梵行,那是什麼原因?或許他們也會了知一句,那對他們會有長久的利益、安樂。

  村長!猶如男子有三支水瓶:一支是無裂縫、不滲(不拿走)、不漏的水瓶,一支是無裂縫、會滲、會漏的水瓶,一支是有裂縫、會滲、會漏的水瓶,村長!你怎麼想它:那位男子想要放置水(裝水),會在哪裡第一個放置:在無裂縫、不滲、不漏的那支水瓶?或在無裂縫、會滲、會漏的那支水瓶?或在有裂縫、會滲、會漏的那支水瓶?」

  「大德!那位男子想要放置水,會放置在無裂縫、不滲、不漏的那支水瓶那裡,在那裡放置後,會放置在無裂縫、會滲、會漏的那支水瓶那裡,在那裡放置後,會放置在有裂縫、會滲、會漏的那支水瓶那裡,或不會放置,那是什麼原因?至少也將成為物品的洗淨。」

  「村長!猶如那支無裂縫、不滲、不漏的水瓶,對我來說,比丘、比丘尼正是這樣。我為他們教導法:開頭是善的、中間是善的、結尾是善的;有意義的、有文字的法,我說明完全圓滿、遍純淨的梵行,那是什麼原因?村長!因為這些住於以我為依靠、以我為庇護、以我為救護、以我為歸依。

  村長!猶如那支無裂縫、會滲、會漏的水瓶,對我來說,優婆塞、優婆夷正是這樣。我也為他們教導法:開頭是善的、中間是善的、結尾是善的;有意義的、有文字的法,我說明完全圓滿、遍純淨的梵行,那是什麼原因?村長!因為這些住於以我為依靠、以我為庇護、以我為救護、以我為歸依。

  村長!猶如那支有裂縫、會滲、會漏的水瓶,對我來說,其他外道沙門、婆羅門遊行者們正是這樣。我也為他們教導法:開頭是善的、中間是善的、結尾是善的;有意義的、有文字的法,我說明完全圓滿、遍純淨的梵行,那是什麼原因?或許他們也會了知一句,那對他們會有長久的利益、安樂。

  在這麼說時,村長阿西邦大葛之子對世尊說這個:「太偉大了,大德!……(中略)從今天起\twnr{已終生歸依}{64.0}。」



\sutta{8}{8}{吹海螺者經}{https://agama.buddhason.org/SN/sn.php?keyword=42.8}
  \twnr{有一次}{2.0},\twnr{世尊}{12.0}住在那爛陀賣衣者的芒果園中。

  那時,尼乾陀的弟子,村長阿西邦大葛之子去見世尊。抵達後,向世尊\twnr{問訊}{46.0}後,在一旁坐下。世尊對在一旁坐下的村長阿西邦大葛之子說這個:

  「村長!尼乾陀若提子如何對弟子們說法呢?」

  「\twnr{大德}{45.0}!尼乾陀若提子這麼對弟子說法:『凡任何人殺生,那全部是墮\twnr{苦界}{109.0}者、墮地獄者;凡任何人取未給予的,那全部是墮苦界者、墮地獄者;凡任何人行\twnr{邪淫}{105.0},那全部是墮苦界者、墮地獄者;凡任何人虛妄地說,那全部是墮苦界者、墮地獄者。凡一一經常地住,一一以那個被引導。』大德!尼乾陀若提子這麼對弟子說法。」

  「村長!但,『凡一一經常地住,一一以那個被引導。』在存在這樣時,將不存在任何墮苦界者、墮地獄者,如尼乾陀若提子之言。

  村長!你怎麼想它:凡那位殺生的男子,比較日或夜的從時到時(時非時?),哪個時間比較多:凡那位殺生:或凡那位不殺生?」

  「大德!凡那位殺生的男子,比較日或夜的從時到時,那個時間比較少:那位殺生,而那個時間就比較多:凡那位不殺生。」

  「村長!『凡一一經常地住,一一以那個被引導。』在存在這樣時,將不存在任何墮苦界者、墮地獄者,如尼乾陀若提子之言。

  村長!你怎麼想它:凡那位\twnr{未給予而取}{104.0}的男子,比較日或夜的從時到時,哪個時間比較多:凡那位取未給予的,或凡那位不取未給予的?」

  「大德!凡那位未給予而取的男子,比較日或夜的從時到時,那個時間比較少:凡那位不取未給予的,而那個時間就比較多:凡那位取未給予的。」

  「村長!『凡一一經常地住,一一以那個被引導。』在存在這樣時,將不存在任何墮苦界者、墮地獄者,如尼乾陀若提子之言。

  村長!你怎麼想它:凡那位邪淫的男子,比較日或夜的從時到時,哪個時間比較多:凡那位行邪淫,或凡那位不行邪淫?」

  「大德!凡那位邪淫的男子,比較日或夜的從時到時,那個時間比較少:凡那位行邪淫,而那個時間就比較多:凡那位不行邪淫。」

  「村長!『凡一一經常地住,一一以那個被引導。』在存在這樣時,將不存在任何墮苦界者、墮地獄者,如尼乾陀若提子之言。

  村長!你怎麼想它:凡那位\twnr{妄語}{106.0}的男子,比較日或夜的從時到時,哪個時間比較多:凡那位虛妄地說,或凡那位不虛妄地說?」

  「大德!凡那位妄語的男子,比較日或夜的從時到時,那個時間比較少:凡那位虛妄地說,而那個時間就比較多:凡那位不虛妄地說。」

  「村長!『凡一一經常地住,一一以那個被引導。』在存在這樣時,將不存在任何墮苦界者、墮地獄者,如尼乾陀若提子之言。

  村長!這裡,某位大師是這樣說者、這樣見者:『凡任何人殺生,那全部是墮苦界者、墮地獄者;凡任何人取未給予的,那全部是墮苦界者、墮地獄者;凡任何行邪淫,那全部是墮苦界者、墮地獄者;凡任何人虛妄地說,那全部是墮苦界者、墮地獄者。』村長!而弟子在那位大師上是\twnr{極淨信者}{340.1},他這麼想:

  『我的大師是這樣知者、這樣見者:「凡任何人殺生,那全部是墮苦界者、墮地獄者。」而有生類被我殺了,我也是墮苦界者、墮地獄者。』他得到(邪)見。村長!他不捨斷那個言語、不捨斷那個心、不\twnr{斷念}{211.0}那個見後,像這樣這麼被帶、被置於地獄中。

  『我的大師是這樣知者、這樣見者:「凡任何人取未給予的,那全部是墮苦界者、墮地獄者。」而有未給予的被我取了,我也是墮苦界者、墮地獄者。』他得到見。村長!他不捨斷那個言語、不捨斷那個心、不斷念那個見後,像這樣這麼被帶、被置於地獄中。

  『我的大師是這樣知者、這樣見者:「凡任何人行邪淫,那全部是墮苦界者、墮地獄者。」而諸邪淫被我行,我也是墮苦界者、墮地獄者。』他得到見。村長!他不捨斷那個言語、不捨斷那個心、不斷念那個見後,像這樣這麼被帶、被置於地獄中。

  『我的大師是這樣知者、這樣見者:「凡任何人虛妄地說,那全部是墮苦界者、墮地獄者。」而被我虛妄地說,我也是墮苦界者、墮地獄者。』他得到見。村長!他不捨斷那個言語、不捨斷那個心、不斷念那個見後,像這樣這麼被帶、被置於地獄中。

  村長!然而,這裡,\twnr{如來}{4.0}、\twnr{阿羅漢}{5.0}、\twnr{遍正覺者}{6.0}、\twnr{明行具足者}{7.0}、\twnr{善逝}{8.0}、\twnr{世間知者}{9.0}、\twnr{應該被調御人的無上調御者}{10.0}、\twnr{天-人們的大師}{11.0}、\twnr{佛陀}{3.0}、\twnr{世尊}{12.0}在世間出現,他以種種法門呵責、斥責殺生,並且說:『你們要戒絕殺生!』他呵責、斥責未給予而取,並且說:『你們要戒絕未給予而取!』他呵責、斥責邪淫,並且說:『你們要戒絕邪淫!』他呵責、斥責妄語,並且說:『你們要戒絕妄語!』村長!而弟子在那位大師上是極淨信者,他像這樣深慮:

  『世尊以種種法門呵責、斥責殺生,並且說:「你們要戒絕殺生!」而或多或少有生類被我殺了,凡或多或少有生類被我殺了者,這是不善的,這是不\twnr{好}{44.0}的,但我就以那個\twnr{為緣}{180.0}成為後悔者,我的這個惡業將不成為未作的。』他像這樣省察後,就捨斷那個殺生,且未來成為離殺生者。這樣,有這個惡業的捨斷,這樣,有這個惡業的超越。

  『世尊以種種法門呵責、斥責未給予而取,並且說:「你們要戒絕未給予而取!」而或多或少有未給予的被我取了,凡或多或少有未給予的被我取了者,這是不善的,這是不好的,但我就以那個為緣成為後悔者,我的這個惡業將不成為未作的。』他像這樣省察後,就捨斷那個未給予而取,且未來成為離未給予而取者,這樣,有這個惡業的捨斷,這樣,有這個惡業的超越。

  『世尊以種種法門呵責、斥責邪淫,並且說:「你們要戒絕邪淫!」而或多或少有邪淫被我實行了,凡或多或少有邪淫被我實行了者,這是不善的,這是不好的,但我就以那個為緣成為後悔者,我的這個惡業將不成為未作的。』他像這樣省察後,就捨斷那個邪淫,且未來成為離邪淫者,這樣,有這個惡業的捨斷,這樣,有這個惡業的超越。

  『世尊以種種法門呵責、斥責妄語,並且說:「你們要戒絕妄語!」而或多或少有虛妄被我說了,凡或多或少有虛妄被我說了者,這是不善的,這是不好的,但我就以那個為緣成為後悔者,我的這個惡業將不成為未作的。』他像這樣省察後,就捨斷那個妄語,且未來成為離妄語者,這樣,有這個惡業的捨斷,這樣,有這個惡業的超越。

  捨斷殺生後,他是離殺生者;捨斷未給予而取後,他是離未給予而取者;捨斷邪淫後,他是離邪淫者;捨斷妄語後,他是離妄語者;捨斷\twnr{離間語}{234.0}後,他是離離間語者;捨斷\twnr{粗惡語}{235.0}後,他是離粗惡語者;捨斷\twnr{雜穢語}{236.0}後,他是離雜穢語者;捨斷\twnr{貪婪}{435.0}後,他是不貪婪者;捨斷惡意瞋怒後,他是無瞋害心者;捨斷邪見後,他是正見者。

  村長!那位這麼離貪婪、離惡意、不癡昧、正知、朝向念的聖弟子,他以與慈俱行之心遍滿一方後而住,像這樣第二的,像這樣第三的,像這樣第四的,像這樣上下、橫向、到處、對一切如對自己,以與慈俱行的、廣大的、變大的、無量的、無怨恨的、無瞋害的心遍滿全部世間後而住。

  村長!猶如如有力氣的吹海螺者少困難地就使四方知道。同樣的,村長!當\twnr{慈心解脫}{589.0}已這麼\twnr{修習}{94.0}、已這麼\twnr{多作}{95.0}時,凡\twnr{所作的有量業}{987.0},它在那裡不殘留,它在那裡不住立。

  村長!那位這麼離貪婪、離惡意、不癡昧、正知、朝向念的聖弟子,他以與悲俱行之心……(中略)以與喜悅俱行之心……(中略)以與\twnr{平靜}{228.0}俱行之心遍滿一方後而住,像這樣第二的,像這樣第三的,像這樣第四的,像這樣上下、橫向、到處、對一切如對自己,以與平靜俱行的、廣大的、變大的、無量的、無怨恨的、無瞋害的心遍滿全部世間後而住。

  村長!猶如有力氣的吹海螺者少困難地就使四方知道。同樣的,村長!當以平靜心解脫已這麼修習、已這麼多作時,凡所作的有量業,它在那裡不殘留,它在那裡不住立。」

  在這麼說時,村長阿西邦大葛之子對世尊說這個:

  「太偉大了,大德!太偉大了,大德!……(中略)世尊記得我為\twnr{優婆塞}{98.0},從今天起\twnr{已終生歸依}{64.0}。」



\sutta{9}{9}{家經}{https://agama.buddhason.org/SN/sn.php?keyword=42.9}
  \twnr{有一次}{2.0},\twnr{世尊}{12.0}與大\twnr{比丘}{31.0}\twnr{僧團}{375.0}一起在憍薩羅國進行著\twnr{遊行}{61.0},抵達那爛陀。

  在那裡,世尊住在那爛陀賣衣者的芒果園中。

  當時,那爛陀是飢饉的、難獲得的、稻子得白病、無穗(葉子被播種)的。

  當時,尼乾陀若提子與大群尼乾陀眾一起居住在那爛陀。

  那時,尼乾陀的弟子,村長阿西邦大葛之子去見尼乾陀若提子。抵達後,向尼乾陀若提子\twnr{問訊}{46.0}後,在一旁坐下。尼乾陀若提子對在一旁坐下的村長阿西邦大葛之子說這個:

  「來!村長!請你去論破\twnr{沙門}{29.0}\twnr{喬達摩}{80.0}(使登上沙門喬達摩的論說),這樣,你的好名聲將傳出去:『這麼大神通力、這麼大威力的沙門喬達摩被村長阿西邦大葛之子論破。』」

  「\twnr{大德}{45.0}!但,我將如何論破這麼大神通力、這麼大威力的沙門喬達摩呢?」

  「來!村長!請你去見沙門喬達摩,抵達後,請你對沙門喬達摩這麼說:『大德!世尊以種種法門稱讚對諸家的同情、保護、憐愍,不是嗎?』

  村長!如果沙門喬達摩被這麼問,他這麼回答:『是的,村長!如來以種種法門稱讚對諸家的同情、保護、憐愍。』你應該對他這麼說:『大德!那樣的話,為何世尊在飢饉、難獲得、稻子得白病、無穗時,還與大比丘僧團一起進行著遊行呢?世尊是對諸家毀滅的行者,世尊是對諸家禍害的行者,世尊是對諸家傷害的行者。』村長!被你問這個\twnr{兩難}{x566}問題的沙門喬達摩既不能吐出,也不能嚥下。」

  「是的,大德!」村長阿西邦大葛之子回答尼乾陀若提子後,從座位起來、向尼乾陀若提子問訊、\twnr{作右繞}{47.0}後,去見世尊。抵達後,向世尊問訊後,在一旁坐下。在一旁坐下的村長阿西邦大葛之子對世尊說這個:

  「大德!世尊以種種法門稱讚對諸家的同情、保護、憐愍,不是嗎?」

  「是的,村長!如來以種種法門稱讚對諸家的同情、保護、憐愍。」

  「大德!那樣的話,為何世尊在飢饉、難獲得、稻子得白病、無穗時,還與大比丘僧團一起進行著遊行呢?世尊是對諸家毀滅的行者,世尊是對諸家禍害的行者,世尊是對諸家傷害的行者。」

  「村長!從從現在起,那個九十一劫凡我回憶,不證知(記得)以前任何家以僅熟施食的給與被傷害,而凡那些富有的、大富的、大財富的、多金銀的、多財產資具的、多財穀的家,那些全部是布施生成的、真實(真理)生成的、\twnr{沙門性生成的}{x567}。

  村長!有八因、八\twnr{緣}{180.0}對諸家的傷害:由於(從)國王諸家走到被傷害,或由於盜賊諸家走到被傷害,或由於火諸家走到被傷害,或由於水諸家走到被傷害,或儲藏處消失,\twnr{或劣企畫的諸工作失敗}{x568},或家中出現敗家子:撒散、破壞、碎破他們財富,以無常性為第八。村長!這些是八因、八緣對諸家的傷害。

  村長!在當這八因、八緣存在時,凡如果這麼說我:『世尊是對家毀滅的行者;世尊是對家禍害的行者;世尊是對家傷害的行者。』村長!他不捨斷那個言語、不捨斷那個心、不斷念那個見後,像這樣這麼被帶、被置於地獄中。」

  在這麼說時,村長阿西邦大葛之子對世尊說這個:

  「太偉大了,大德!太偉大了,大德!……(中略)請世尊記得我為優婆塞,從今天起\twnr{已終生歸依}{64.0}。」



\sutta{10}{10}{摩尼朱羅葛經}{https://agama.buddhason.org/SN/sn.php?keyword=42.10}
  \twnr{有一次}{2.0},世尊住在王舍城栗鼠飼養處的竹林中。

  當時,當國王隨從在國王後宮集會共坐時,這個談論中出現:

  「對\twnr{釋迦之徒的}{262.1}\twnr{沙門}{29.0}們來說,金銀適當,釋迦之徒的沙門們受用金銀,釋迦之徒的沙門們領受金銀。」

  當時,村長\twnr{摩尼朱羅葛}{x569}坐在那個群眾中。

  那時,村長摩尼朱羅葛對那個群眾說這個:

  「\twnr{紳士}{612.1}們!你們不要這麼說,對釋迦之徒的沙門們來說,金銀不適當,釋迦之徒的沙門們不受用金銀,釋迦之徒的沙門們不領受金銀,釋迦之徒的沙門們已放下珠寶黃金、已離金銀。」

  村長摩尼朱羅葛能夠說服那個群眾。

  那時,村長摩尼朱羅葛去見世尊。抵達後,向世尊\twnr{問訊}{46.0}後,在一旁坐下。在一旁坐下的村長摩尼朱羅葛對世尊說這個:

  「\twnr{大德}{45.0}!這裡,當國王隨從在國王後宮集會共坐時,這個談論中出現:『對釋迦之徒的沙門來說,金銀適當,釋迦之徒的沙門們受用金銀,釋迦之徒的沙門們領受金銀。』大德!在這麼說時,我對那個群眾說這個:『紳士們!你們不要這麼說,對釋迦之徒的沙門們來說,金銀不適當,釋迦之徒的沙門們不受用金銀,釋迦之徒的沙門們不領受金銀,釋迦之徒的沙門們已放下珠寶黃金、已離金銀。』大德!我能夠說服那個群眾。大德!當這麼回答時,是否我\twnr{是世尊的所說之說者}{115.0},而且不會以不實的誹謗世尊,以及會\twnr{法隨法地回答}{415.0},而任何如法的種種說不會來到應該被呵責處?」

  「村長!當你這麼解說時,確實是我的所說之說者,同時也不以不實的誹謗我,法隨法地回答,而任何如法的種種說不來到應該被呵責處。村長!因為,對釋迦之徒的沙門們來說,金銀不適當,釋迦之徒的沙門們不受用金銀,釋迦之徒的沙門們不領受金銀,釋迦之徒的沙門們已放下珠寶黃金、已離金銀。

  村長!凡金銀對他適當者,\twnr{五種欲}{187.0}對他也適當;凡五種欲對他適當者,村長!你能憶持這絕對是非\twnr{沙門法}{29.1}、非\twnr{釋迦人之子}{262.1}的法。

  村長!此外,我這麼說:『草能被有需要草者遍求,木材能被有需要木材者遍求,車能被有需要車者遍求,\twnr{男子(工人)能被有需要男子者遍求}{x570}。』村長!但我就不說:『能被受用的金銀能被任何法門遍求。』」



\sutta{11}{11}{薄羅葛經}{https://agama.buddhason.org/SN/sn.php?keyword=42.11}
  \twnr{有一次}{2.0},世尊住\twnr{在末羅}{x571},名叫屋盧吠羅迦巴的末羅市鎮。

  那時,村長薄羅葛去見世尊。抵達後,向世尊\twnr{問訊}{46.0}後,在一旁坐下。在一旁坐下的村長薄羅葛對世尊說這個:

  「\twnr{大德}{45.0}!請世尊為我教導苦的\twnr{集起}{67.0}與滅沒,\twnr{那就好了}{44.0}。」

  「村長!如果我為你教導關於過去世苦的集起與滅沒:『這樣是過去世。』在那裡,你會有疑惑,會有懷疑,村長!如果我為你教導關於未來世苦的集起與滅沒:『這樣是未來世。』也在那裡,你會有疑惑,會有懷疑,村長!但,就坐在這裡的我將為就坐在那裡的你教導苦的集起與滅沒,你要聽!你要\twnr{好好作意}{43.1}!我將說。」

  「是的,大德!」村長薄羅葛回答世尊。

  世尊說這個:

  「村長!你怎麼想它:在屋盧吠羅迦巴,有你的人,凡對他們以殺害,或以捕捉,或以沒收,或以斥責,你的愁、悲、苦、憂、\twnr{絕望}{342.0}會生起嗎?」

  「大德!在屋盧吠羅迦巴,有我的人,凡對他們以殺害,或以捕捉,或以沒收,或以斥責哪個人被處死,或被捕,或被沒收,或被責難,我的愁、悲、苦、憂、絕望會生起。」

  「村長!又,在屋盧吠羅迦巴,有你的人,凡對他們以殺害,或以捕捉,或以沒收,或以斥責,你的愁、悲、苦、憂、絕望不會生起嗎?」

  「大德!在屋盧吠羅迦巴,有我的人,凡對他們以殺害,或以捕捉,或以沒收,或以斥責,我的愁、悲、苦、憂、絕望不會生起。」

  「村長!什麼因、什麼\twnr{緣}{180.0},以那個在屋盧吠羅迦巴,對某些人以殺害,或以捕捉,或以沒收,或以斥責,你的愁、悲、苦、憂、絕望會生起呢?」

  「大德!凡對在屋盧吠羅迦巴我的人以殺害,或以捕捉,或以沒收,或以斥責,愁、悲、苦、憂、絕望會生起者,我在他們上有意欲貪。大德!但凡對在屋盧吠羅迦巴人們以殺害,或以捕捉,或以沒收,或以斥責,愁、悲、苦、憂、絕望不會生起者,我在他們上沒有意欲貪。」

  「村長!請以這個已看見的、已知道的、即時達到的、已深解的法,引導在過去、未來的推論(理趣):『凡任何過去世生起的苦,那全都意欲為根源、意欲為因生起,因為意欲是苦的根源;凡任何未來世生起的苦,那也全都意欲為根源、意欲為因生起,因為意欲是苦的根源。』」

  「\twnr{不可思議}{206.0}啊,大德!未曾有啊,大德!大德!又,這被世尊多麼善說:『凡任何生起的苦,那全都意欲為根源、意欲為因生起,因為意欲是苦的根源。』

  大德!\twnr{我有個男孩}{x572}名叫智羅瓦西,他居住在外面的住處,大德!那個我就在清晨起來後派遣男子:『我說,請你去瞭解智羅瓦西男孩。』大德!只要那位男子沒回來,那個我就有焦慮(變異):『但不要任何人使智羅瓦西男孩苦惱。』」

  「村長!你怎麼想它:對智羅瓦西男孩以殺害,或以捕捉,或以沒收,或以斥責,愁、悲、苦、憂、絕望會生起嗎?』」

  「大德!對智羅瓦西男孩以殺害,或以捕捉,或以沒收,或以斥責,我的生命都會有變異,又如何我的愁、悲、苦、憂、絕望將不生起呢!」

  「村長!以這個法門,這也能被知道:『任何生起的苦,那全都意欲為根源、意欲為因生起,因為意欲是苦的根源。』

  村長!你怎麼想它:『當智羅瓦西的母親未被你看見、聽聞,你對智羅瓦西的母親有意欲,或貪,或情愛嗎?』」

  「大德!這確實不是。」

  「村長!因為你的看見或聽聞,這樣你對智羅瓦西的母親有意欲,或貪,或情愛嗎?」

  「是的,大德!」

  「村長!你怎麼想它:對智羅瓦西的母親以殺害,或以捕捉,或以沒收,或以斥責,你的愁、悲、苦、憂、絕望會生起嗎?」

  「大德!對智羅瓦西男孩的母親以殺害,或以捕捉,或以沒收,或以斥責,我的生命都會有變異,又如何我的愁、悲、苦、憂、絕望將不生起呢!」

  「村長!以這個法門,這也能被知道:『任何生起的苦,那全都意欲為根源、意欲為因生起,因為意欲是苦的根源。』」



\sutta{12}{12}{羅西亞經}{https://agama.buddhason.org/SN/sn.php?keyword=42.12}
  那時,村長羅西亞去見世尊。抵達後,向世尊\twnr{問訊}{46.0}後,在一旁坐下。在一旁坐下的村長羅西亞對世尊說這個:

  「\twnr{大德}{45.0}!這被我聽聞:『\twnr{沙門}{29.0}\twnr{喬達摩}{80.0}呵責一切苦行,\twnr{一向}{168.0}地責備、呵叱一切苦行者、粗弊生活者。』大德!凡他們這麼說:『沙門喬達摩呵責一切苦行,一向地責備、呵叱一切苦行者、粗弊生活者。』者,大德!是否他們\twnr{會是世尊的所說之說者}{115.0},而且不會以不實的誹謗世尊,以及會\twnr{法隨法地解說}{415.0},而任何如法的種種說不會來到應該被呵責處?」

  「村長!凡他們這麼說:『沙門喬達摩呵責一切苦行,一向地責備、呵叱一切苦行者、粗弊生活者。』者,他們不是我的所說之說者,而且他們以不存在的、虛偽的、不實的誹謗我。

  村長!有這兩個邊(極端),不應該被出家人實行:凡這在諸欲上欲之享樂的實踐:下劣的、粗俗的、一般人的、非聖者的、伴隨無利益的,以及凡這自我折磨的實踐:苦的、非聖者的、伴隨無利益的。村長!不走入這些那些兩個邊後,\twnr{作眼、作智}{505.0}的\twnr{中道}{399.0}被\twnr{如來}{4.0}\twnr{現正覺}{75.0},它轉起寂靜、證智、\twnr{正覺}{185.1}、涅槃。

  村長!而什麼是那個作眼、作智的中道被如來現正覺,它轉起寂靜、證智、正覺、涅槃呢?就是這\twnr{八支聖道}{525.0},即:正見……(中略)正定。村長!這是那個作眼、作智的中道被如來現正覺,它轉起寂靜、證智、正覺、涅槃。

  村長!有這三種現在的世間中存在的\twnr{受用諸欲者}{x573},哪三種?

  村長!這裡,某一類的受用諸欲者以非法[、以暴力]遍求財物,以非法、以暴力遍求財物後,不使自己快樂、喜悅,不分享,不作福德。

  村長!又,這裡,某一類的受用諸欲者以非法、以暴力遍求財物,以非法、以暴力遍求財物後,使自己快樂、喜悅,不分享,不作福德。

  村長!又,這裡,某一類的受用諸欲者以非法、以暴力遍求財物,以非法、以暴力遍求財物後,使自己快樂、喜悅,分享,作福德。

  村長!又,這裡,某一類的受用諸欲者以法、非法,以暴力,也以非暴力遍求財物,以法、非法,以暴力,也以非暴力遍求財物後,不使自己快樂、喜悅,不分享,不作福德。

  村長!又,這裡,某一類的受用諸欲者以法、非法,以暴力,也以非暴力遍求財物,以法、非法,以暴力,也以非暴力遍求財物後,使自己快樂、喜悅,不分享,不作福德。

  村長!又,這裡,某一類的受用諸欲者以法、非法,以暴力,也以非暴力遍求財物,以法、非法,以暴力,也以非暴力遍求財物後,使自己快樂、喜悅,分享,作福德。

  村長!又,這裡,某一類的受用諸欲者以法、以非暴力遍求財物,以法、以非暴力遍求財物後,不使自己快樂、喜悅,不分享,不作福德。

  村長!又,這裡,某一類的受用諸欲者以法、以非暴力遍求財物,以法、以非暴力遍求財物後,使自己快樂、喜悅,不分享,不作福德。

  村長!又,這裡,某一類的受用諸欲者以法、以非暴力遍求財物,以法、以非暴力遍求財物後,使自己快樂、喜悅,分享,作福德,但繫結地、迷昏頭地、有罪過地、不看見\twnr{過患}{293.0}、無\twnr{出離}{294.0}慧地受用那些財物。

  村長!又,這裡,某一類的受用諸欲者以法、以非暴力遍求財物,以法、以非暴力遍求財物後,使自己快樂、喜悅,分享,作福德,但不繫結地、不迷昏頭地、無罪過地、看見過患地、出離慧地受用那些財物。

  村長!在那裡,凡這位受用諸欲者以非法、以暴力遍求財物,以非法、以暴力遍求財物後,不使自己快樂、喜悅,不分享,不作福德者,這位受用諸欲者以三處應該被呵責。以哪三處應該被呵責呢?『以非法、以暴力遍求財物。』以這第一處應該被呵責,『不使自己快樂、喜悅。』以這第二處應該被呵責,『不分享,不作福德。』以這第三處應該被呵責。村長!這位受用諸欲者應該以這三處被呵責。

  村長!在那裡,凡這位受用諸欲者以非法、以暴力遍求財物,以非法、以暴力遍求財物後,使自己快樂、喜悅,不分享,不作福德者,這位受用諸欲者以二處應該被呵責,以一處應該被讚賞。以哪二處應該被呵責呢?『以非法、以暴力遍求財物。』以這第一處應該被呵責,『不分享,不作福德。』以這第二處應該被呵責。以哪一處應該被讚賞呢?『使自己快樂、喜悅。』以這一處應該被讚賞。村長!這位受用諸欲者以這二處應該被呵責,以這一處應該被讚賞。

  村長!在那裡,凡這位受用諸欲者以非法、以暴力遍求財物,以非法、以暴力遍求財物後,使自己快樂、喜悅,分享,作福德者,這位受用諸欲者以一處應該被呵責,以二處應該被讚賞。以哪一處應該被呵責呢?『以非法、以暴力遍求財物。』以這一處應該被呵責。以哪二處應該被讚賞呢?『使自己快樂、喜悅。』以這第一處應該被讚賞,『分享,作福德。』以這第二處應該被讚賞。村長!這位受用諸欲者以這一處應該被呵責,以這二處應該被讚賞。

  村長!在那裡,凡這位受用諸欲者以法、非法,以暴力,也以非暴力遍求財物,以法、非法,以暴力,也以非暴力遍求財物後,不使自己快樂、喜悅,不分享,不作福德者,這位受用諸欲者以一處應該被讚賞,以三處應該被呵責。以哪一處應該被讚賞呢?『以法、以非暴力遍求財物。』以這一處應該被讚賞。以哪三處應該被呵責呢?『以非法、以暴力遍求財物。』以這第一處應該被呵責,『不使自己快樂、喜悅。」以這第二處應該被呵責,『不分享,不作福德。』以這第三處應該被呵責。村長!這位受用諸欲者以這一處應該被讚賞,以這三處應該被呵責。

  村長!在那裡,凡這位受用諸欲者以法、非法,以暴力,也以非暴力遍求財物,以法、非法,以暴力,也以非暴力遍求財物後,使自己快樂、喜悅,不分享,不作福德者,這位受用諸欲者以二處應該被讚賞,以二處應該被呵責。以哪二處應該被讚賞呢?『以法、以非暴力遍求財物。』以這第一處應該被讚賞,『使自己快樂、喜悅。』以這第二處應該被讚賞。以哪二處應該被呵責呢?『以非法、以暴力遍求財物。』以這第一處應該被呵責,『不分享,不作福德。』以這第二處應該被呵責。村長!這位受用諸欲者以這二處應該被讚賞,以這二處應該被呵責。

  村長!在那裡,凡這位受用諸欲者以法、非法,以暴力,也以非暴力遍求財物,以法、非法,以暴力,也以非暴力遍求財物後,使自己快樂、喜悅,分享,作福德者,這位受用諸欲者以三處應該被讚賞,以一處應該被呵責。以哪三處應該被讚賞呢?『以法、以非暴力遍求財物。』以這第一處應該被讚賞,『使自己快樂、喜悅。』以這第二處應該被讚賞,『分享,作福德。』以這第三處應該被讚賞。以哪一處應該被呵責呢?『以非法、以暴力遍求財物。』以這一處應該被呵責。村長!這位受用諸欲者以這三處應該被讚賞,以這一處應該被呵責。

  村長!在那裡,凡這位受用諸欲者以法、以非暴力遍求財物,以法、以非暴力遍求財物後,不使自己快樂、喜悅,不分享,不作福德者,這位受用諸欲者以一處應該被讚賞,以二處應該被呵責。以哪一處應該被讚賞呢?『以法、以非暴力遍求財物。』以這一處應該被讚賞。以哪二處應該被呵責呢?『不使自己快樂、喜悅。』以這第一處應該被呵責,『不分享,不作福德。』以這第二處應該被呵責。村長!這位受用諸欲者以這一處應該被讚賞,以這二處應該被呵責。

  村長!在那裡,凡這位受用諸欲者以法、以非暴力遍求財物,以法、以非暴力遍求財物後,使自己快樂、喜悅,不分享,不作福德者,這位受用諸欲者以二處應該被讚賞,以一處應該被呵責。以哪二處應該被讚賞呢?『以法、以非暴力遍求財物。』以這第一處應該被讚賞,『使自己快樂、喜悅。』以這第二處應該被讚賞。以哪一處應該被呵責呢?『不分享,不作福德。』以這一處應該被呵責。村長!這位受用諸欲者以這二處應該被讚賞,以這一處應該被呵責。

  村長!在那裡,凡這位受用諸欲者以法、以非暴力遍求財物,以法、以非暴力遍求財物後,使自己快樂、喜悅,分享,作福德,但被繫縛、被迷昏頭、被染著,不看見過患地、無出離慧地受用對那些財物,這位受用諸欲者以三處應該被讚賞,以一處應該被呵責。以哪三處應該被讚賞呢?『以法、以非暴力遍求財物。』以這第一處應該被讚賞,『使自己快樂、喜悅。』以這第二處應該被讚賞,『分享,作福德。』以這第三處應該被讚賞。以哪一處應該被呵責呢?『繫結地、迷昏頭地、有罪過地、不看見過患地、無出離慧地受用那些財物。』以這一處應該被呵責。村長!這位受用諸欲者以這三處應該被讚賞,以這一處應該被呵責。

  村長!在那裡,凡這位受用諸欲者以法、以非暴力遍求財物,以法、以非暴力遍求財物後,使自己快樂、喜悅,分享,作福德,不繫結地、不迷昏頭地、無罪過地、看見過患地、出離慧地受用那些財物,這位受用諸欲者以四處應該被讚賞。以哪四處應該被讚賞呢?『以法、以非暴力遍求財物。』以這第一處應該被讚賞,『使自己快樂、喜悅。』以這第二處應該被讚賞,『分享,作福德。』以這第三處應該被讚賞,『不繫結地、不迷昏頭地、無罪過地、看見過患地、出離慧地受用那些財物。』以這第四處應該被讚賞。村長!這位受用諸欲者以這四處應該被讚賞。[\ccchref{AN.10.91}{https://agama.buddhason.org/AN/an.php?keyword=10.91}]

  村長!有這三種現在的世間中存在的苦行者、粗弊生活者,哪三種?村長!這裡,某一類的苦行者、粗弊生活者,\twnr{以信從在家出家成為無家者}{48.0}:『也許我能證得善法,也許我能作證\twnr{足以為聖者智見特質}{473.0}的\twnr{過人法}{205.0}。』他使自己苦惱、痛苦,沒證得善法,與沒作證足以為聖者智見特質的過人法。

  村長!又,這裡,某一類的苦行者、粗弊生活者,以信從在家出家成為無家者:『也許我能證得善法,也許我能作證足以為聖者智見特質的過人法。』他使自己苦惱、痛苦,證得善法,沒作證足以為聖者智見特質的過人法。

  村長!又,這裡,某一類的苦行者、粗弊生活者,以信從在家出家成為無家者:『也許我能證得善法,也許我能作證足以為聖者智見特質的過人法。』他使自己苦惱、痛苦,證得善法,與作證足以為聖者智見特質的過人法。

  村長!在那裡,凡這位苦行者、粗弊生活者使自己苦惱、痛苦,沒證得善法,與沒作證足以為聖者智見特質的過人法者,村長!這位苦行者、粗弊生活者以三處應該被呵責。以哪三處應該被呵責呢?『使自己苦惱、痛苦。』以這第一處應該被呵責,『沒證得善法。』以這第二處應該被呵責,『沒作證足以為聖者智見特質的過人法。』以這第三處應該被呵責。村長!這位苦行者、粗弊生活者以這三處應該被呵責。

  村長!在那裡,凡這位苦行者、粗弊生活者使自己苦惱、痛苦,證得善法,沒作證足以為聖者智見特質的過人法者,村長!這位苦行者、粗弊生活者以二處應該被呵責,以一處應該被讚賞。以哪二處應該被呵責呢?『使自己苦惱、痛苦。』以這第一處應該被呵責,『沒作證足以為聖者智見特質的過人法。』以這第二處應該被呵責。以哪一處應該被讚賞呢?『證得善法。』以這一處應該被讚賞。村長!這位苦行者、粗弊生活者以這二處應該被呵責,以這一處應該被讚賞。

  村長!在那裡,凡這位苦行者、粗弊生活者使自己苦惱、痛苦,證得善法,與作證足以為聖者智見特質的過人法者,村長!這位苦行者、粗弊生活者以一處應該被呵責,以二處應該被讚賞。以哪一處應該被呵責呢?『使自己苦惱、痛苦。』以這一處應該被呵責。以哪二處應該被讚賞呢?『證得善法。』以這第一處應該被讚賞,『作證足以為聖者智見特質的過人法。』以這第二處應該被讚賞。村長!這位苦行者、粗弊生活者以這一處應該被呵責,以這二處應該被讚賞。

  村長!有這三種直接可見的、即時的、請你來看的、能引導的、應該被智者各自經驗的滅盡,哪三種?

  凡貪染者為了貪,意圖對自己的傷害,也意圖對別人的傷害,也意圖對兩者的傷害,在貪被捨斷時,既不意圖對自己的傷害,也不意圖對別人的傷害,也不意圖對兩者的傷害,滅盡是直接可見的、即時的、請你來看的、能引導的、應該被智者各自經驗的。

  凡憤怒者為了瞋,意圖對自己的傷害,也意圖對別人的傷害,也意圖對兩者的傷害,在瞋被捨斷時,既不意圖對自己的傷害,也不意圖對別人的傷害,也不意圖對兩者的傷害,滅盡是直接可見的、即時的、請你來看的、能引導的、應該被智者各自經驗的。

  凡愚癡者為了癡,意圖對自己的傷害,也意圖對別人的傷害,也意圖對兩者的傷害,在癡被捨斷時,既不意圖對自己的傷害,也不意圖對別人的傷害,也不意圖對兩者的傷害,滅盡是直接可見的、即時的、請你來看的、能引導的、應該被智者各自經驗的。

  村長!這些是三種直接可見的、即時的、請你來看的、能引導的、應該被智者各自經驗的滅盡。」

  在這麼說時,村長羅西亞對世尊說這個:

  「太偉大了,大德!……(中略)請世尊記得我為\twnr{優婆塞}{98.0},從今天起\twnr{已終生歸依}{64.0}。」



\sutta{13}{13}{玻得里亞經}{https://agama.buddhason.org/SN/sn.php?keyword=42.13}
  \twnr{有一次}{2.0},\twnr{世尊}{12.0}住在拘利國名叫北方的拘利族人城鎮。

  那時,\twnr{村長}{x574}玻得里亞去見世尊。抵達後,向世尊\twnr{問訊}{46.0}後,在一旁坐下。在一旁坐下的村長玻得里亞對世尊說這個:

  「\twnr{大德}{45.0}!這被我聽聞:『\twnr{沙門}{29.0}\twnr{喬達摩}{80.0}知道\twnr{幻術}{x575}。』大德!凡他們這麼說:『沙門喬達摩知道幻術。』者,大德!是否他們\twnr{會是世尊的所說之說者}{115.0},而且不會以不實的誹謗世尊,以及會\twnr{法隨法地回答}{415.0},而任何如法的種種說不會來到應該被呵責處?大德!因為我們不想要誹謗世尊。」

  「村長!凡他們這麼說:『沙門喬達摩知道幻術。』,他們是我的所說之說者,同時也不以不實的誹謗我,他們法隨法地解說,而任何如法的種種說不來到應該被呵責處。」

  「\twnr{先生}{202.0}!我們確實不相信那些沙門、\twnr{婆羅門}{17.0}的:『沙門喬達摩知道幻術。』正是真實的,先生!沙門喬達摩的確是幻術者。」

  「村長!凡這麼說:『我知道幻術。』者,他這麼說:『我是幻術者。』嗎?」

  「世尊!那就是像那樣,\twnr{善逝}{8.0}!那就是像那樣。」

  「村長!那樣的話,就在這件事上我將反問你,你就如對你能接受的那樣回答它。村長!你怎麼想它:你知道拘利國的\twnr{髮髻下垂雇員}{x576}嗎?」

  「大德!我知道拘利國的髮髻下垂雇員。」

  「村長!你怎麼想它:拘利國的髮髻下垂雇員們是什麽使命的?」

  「大德!凡拘利國的盜賊,能遮止他們,以及凡拘利國的使節,能帶來(護送)他們,大德!拘利國的髮髻下垂雇員們是這個使命的。」

  「村長!你怎麼想它:你知道拘利國的髮髻下垂雇員是持戒者,或他們是破戒者?」

  「大德!我知道拘利國的髮髻下垂雇員們是破戒者、\twnr{惡法者}{601.0},而凡世間中的破戒者、惡法者,拘利國的髮髻下垂雇員們是他們中之一。」

  「村長!凡這麼說:『村長玻得里亞知道破戒、惡法的拘利國的髮髻下垂雇員們,村長玻得里亞也是破戒者、惡法者。』那位說者正確地說嗎?」

  「大德!這確實不是。拘利國的髮髻下垂雇員們是一,我是另一,拘利國的髮髻下垂雇員們是一種法,我是另一種法。」

  「村長!你確實能(將)得到:『村長玻得里亞知道破戒、惡法的拘利國的髮髻下垂雇員們,但村長玻得里亞不是破戒者、惡法者。』為何如來不能(將)得到:『如來知道幻術,但如來不是幻術者。』村長!我知道幻術與幻術的果報,以及如是行道的幻術者以身體的崩解,死後往生\twnr{苦界}{109.0}、\twnr{惡趣}{110.0}、\twnr{下界}{111.0}、地獄,而我知道那個。

  村長!又,我知道殺生與殺生的果報,以及如是行道的殺生者以身體的崩解,死後往生苦界、惡趣、下界、地獄,而我知道那個。村長!我知道\twnr{未給予而取}{104.0}與未給予而取的果報,以及如是行道的未給予而取者以身體的崩解,死後往生苦界、惡趣、下界、地獄,而我知道那個。村長!我知道\twnr{邪淫}{105.0}與邪淫的果報,以及如是行道的邪淫者以身體的崩解,死後往生苦界、惡趣、下界、地獄,而我知道那個。村長!我知道\twnr{妄語}{106.0}與妄語的果報,以及如是行道的妄語者以身體的崩解,死後往生苦界、惡趣、下界、地獄,而我知道那個。村長!我知道\twnr{離間語}{234.0}與離間語的果報,以及如是行道的離間語者以身體的崩解,死後往生苦界、惡趣、下界、地獄,而我知道那個。村長!我知道\twnr{粗惡語}{235.0}與粗惡語的果報,以及如是行道的粗惡語者以身體的崩解,死後往生苦界、惡趣、下界、地獄,而我知道那個。村長!我知道\twnr{雜穢語}{236.0}與雜穢語的果報,以及如是行道的雜穢語者以身體的崩解,死後往生苦界、惡趣、下界、地獄,而我知道那個。村長!我知道\twnr{貪婪}{435.0}與貪婪的果報,以及如是行道的貪婪者以身體的崩解,死後往生苦界、惡趣、下界、地獄,而我知道那個。村長!我知道惡意瞋與惡意瞋的果報,以及如是行道的瞋害心者以身體的崩解,死後往生苦界、惡趣、下界、地獄,而我知道那個。村長!我知道邪見與邪見的果報,以及如是行道的邪見者以身體的崩解,死後往生苦界、惡趣、下界、地獄,而我知道那個。

  村長!有一些這樣說、這樣見的沙門婆羅門:『凡任何人殺生,那全部在當生中感受苦憂。凡任何人未給予而取,那全部在當生中感受苦憂,而我知道那個。凡任何人行邪淫,那全部在當生中感受苦憂。凡任何人虛妄地說,那全部在當生中感受苦憂。』

  村長!但,這裡,某人被看見成為戴花環的、戴耳環的、被善洗浴的、被善塗油的、頭髮鬍子被整理的、以女子諸欲使看起來像國王般侍奉的,他們隨即這麼說:『喂!這位男子作了什麼,成為戴花環的、戴耳環的、被善洗浴的、被善塗油的、頭髮鬍子被整理的、以女子諸欲使看起來像國王般侍奉的?』他們隨即這麼說:『喂!這位男子征服國王的怨敵後奪命,悅意的國王對他給與贈與,因為那樣,那位男子成為戴花環的、戴耳環的、被善洗浴的、被善塗油的、頭髮鬍子被整理的、以女子諸欲使看起來像國王般侍奉的。』

  村長!這裡,某人被看見以堅固的繩索手在背後緊緊地捆綁後,剃光頭後,以猛烈聲的銅鼓,從街道到街道;從十字路口到十字路口遍帶領後,經南門出去後,在城南正被斬首,他們隨即這麼說:『喂!這位男子作了什麼,以堅固的繩索手在背後緊緊地捆綁後,剃光頭後,以猛烈聲的銅鼓,從街道到街道;從十字路口到十字路口遍帶領後,經南門出去後,在城南斬首?』他們隨即這麼說:『喂!這位男子是國王的怨敵,他奪取女子或男子的生命,因為那樣,國王捕捉他後,使他們對他作像這樣的刑罰。』

  村長!你怎麼想它:像這樣是否被你看見或聽聞?」

  「大德!被我們看見與聽聞,將還會被聽聞。」

  「村長!在那裡,凡那些沙門、婆羅門是這樣說者、這樣見者:『凡任何人殺生,那全部在當生中感受苦憂。』他們真實地說,或虛妄地?」

  「虛妄,大德!」

  「而凡那些虛妄地扯空虛者,他們是持戒者或破戒者?」

  「破戒者,大德!」

  「而凡那些破戒者、惡法者,他們是邪行者或正行者?」

  「邪行者,大德!」

  「而凡那些邪行者,他們是邪見者或正見者?」

  「邪見者,大德!」

  「而凡那些邪見者,適合在他們上得到淨信嗎?」

  「大德!這確實不是。」

  「村長!但,這裡,某人被看見成為戴花環的、戴耳環的……(中略)以女子諸欲使看起來像國王般侍奉的,他們隨即這麼說:『喂!這位男子作了什麼,成為戴花環的、戴耳環的……(中略)以女子諸欲使看起來像國王般侍奉的?』他們隨即這麼說:『喂!這位男子征服國王的怨敵後帶來寶物,悅意的國王對他給與贈與,因為那樣,那位男子成為戴花環的、戴耳環的……(中略)以女子諸欲使看起來像國王般侍奉的。』

  村長!這裡,某人被看見以堅固的繩索手在背後緊緊地捆綁後……(中略)在城南正被斬首,他們隨即這麼說:『喂!這位男子作了什麼,以堅固的繩索手在背後緊緊地捆綁後……(中略)在城南斬首?』他們隨即這麼說:『喂!這位男子從村落或林野拿走未給予的,被稱為偷盜,因為那樣,國王捕捉他後,使他們對他作像這樣的刑罰。』

  村長!你怎麼想它:像這樣是否被你看見或聽聞?」

  「大德!被我們看見與聽聞,將還會被聽聞。」

  「村長!在那裡,凡那些沙門、婆羅門是這樣說者、這樣見者:『凡任何人未給予而取,那全部在當生中感受苦憂。』他們真實地說,或虛妄地?」……(中略)

  「而凡那些邪見者,適合在他們上得到淨信嗎?」

  「大德!這確實不是。」

  「村長!但,這裡,某人被看見成為戴花環的、戴耳環的……(中略)以女子諸欲使看起來像國王般侍奉的,他們隨即這麼說:『喂!這位男子作了什麼,成為戴花環的、戴耳環的……(中略)以女子諸欲使看起來像國王般侍奉的?』他們隨即這麼說:『喂!這位男子在國王怨敵的妻子上性交(來到實踐),悅意的國王對他給與贈與,因為那樣,那位男子成為戴花環的、戴耳環的……(中略)以女子諸欲使看起來像國王般侍奉的。』

  村長!這裡,某人被看見以堅固的繩索手在背後緊緊地捆綁後……(中略)在城南正被斬首,他們隨即這麼說:『喂!這位男子作了什麼,以堅固的繩索手在背後緊緊地捆綁後……(中略)在城南斬首?』他們隨即這麼說:『喂!這位男子在良家的女子上、在良家的少女上性交,因為那樣,國王捕捉他後,使他們對他作像這樣的刑罰。』

  村長!你怎麼想它:像這樣是否被你看見或聽聞?」 

  「大德!被我們看見與聽聞,將還會被聽聞。」 

  「村長!在那裡,凡那些沙門、婆羅門是這樣說者、這樣見者:『凡任何人行邪淫,那全部在當生中感受苦憂。』他們真實地說,或虛妄地?」……(中略)

  「而凡那些邪見者,適合在他們上得到淨信嗎?」

  「大德!這確實不是。」

  「村長!但,這裡,某人被看見成為戴花環的、戴耳環的、被善洗浴的、被善塗油的、頭髮鬍子被整理的、以女子諸欲使看起來像國王般侍奉的,他們隨即這麼說:『喂!這位男子作了什麼,成為戴花環的、戴耳環的、被善洗浴的、被善塗油的、頭髮鬍子被整理的、以女子諸欲使看起來像國王般侍奉的?』他們隨即這麼說:『喂!這位男子以妄語使國王笑,悅意的國王對他給與贈與,因為那樣,那位男子成為戴花環的、戴耳環的、被善洗浴的、被善塗油的、頭髮鬍子被整理的、以女子諸欲使看起來像國王般侍奉的。』

  村長!這裡,某人被看見以堅固的繩索手在背後緊緊地捆綁後,剃光頭後,以猛烈聲的銅鼓,從街道到街道;從十字路口到十字路口遍帶領後,經南門出去後,在城南正被斬首,他們隨即這麼說:『喂!這位男子作了什麼,以堅固的繩索手在背後緊緊地捆綁後,剃光頭後,以猛烈聲的銅鼓,從街道到街道;從十字路口到十字路口遍帶領後,經南門出去後,在城南斬首?』他們隨即這麼說:『喂!這位男子以妄語破壞\twnr{屋主}{103.0}或屋主之子的利益,因為那樣,國王捕捉他後,使他們對他作像這樣的刑罰。』

  村長!你怎麼想它:像這樣是否被你看見或聽聞?」

  「大德!被我們看見與聽聞,將還會被聽聞。」

  「村長!在那裡,凡那些沙門、婆羅門是這樣說者、這樣見者:『凡任何人虛妄地說,那全部在當生中感受苦憂。』他們真實地說,或虛妄地?」

  「虛妄,大德!」

  「而凡那些虛妄地扯空虛者,他們是持戒者或破戒者?」

  「破戒者,大德!」

  「而凡那些破戒者、惡法者,他們是邪行者或正行者?」

  「邪行者,大德!」

  「而凡那些邪行者,他們是邪見者或正見者?」

  「邪見者,大德!」

  「而凡那些邪見者,適合在他們上得到淨信嗎?」

  「大德!這確實不是。」

  「不可思議啊,大德!\twnr{未曾有}{206.0}啊,大德!大德!我有招待所,在那裡有臥床,有坐具,有水瓶,有油燈。在那裡,凡沙門或婆羅門來住者,我以那個依能力、依力量地分享。大德!從前,有四位不同見、不同信念、不同喜好的\twnr{大師}{145.0}來住在這個招待所。一位大師是這樣說者、這樣見者:『沒有被施與的,沒有被祭祀的,沒有被供養的,沒有善作惡作業的果、果報,沒有這個世間,沒有其他世間,沒有母親,沒有父親,沒有\twnr{化生}{346.0}眾生,在世間中沒有\twnr{正行的、正行道的}{441.0}沙門、婆羅門以證智自作證後告知這個世間與其他世間。』一位大師是這樣說者、這樣見者:『有被施與的,有被祭祀的,有被供養的,有善作惡作業的果、果報,有這個世間,有其他世間,有母親,有父親,有化生眾生,在世間中有正行的、正行道的沙門、婆羅門以證智自作證後告知這個世間與其他世間。』一位大師是這樣說者、這樣見者:『作者、\twnr{使他作者}{791.0},切斷者、\twnr{使他切斷者}{792.0},折磨者、\twnr{使他折磨者}{793.0},造成悲傷者、使他造成悲傷者,造成疲勞者、使他造成疲勞者,造成悸動者、使他造成悸動者,殺生者,\twnr{未給予而取}{104.0}者、\twnr{入侵人家者}{794.0}、奪取(搬運)掠奪物者、作盜匪者、\twnr{攔路搶劫}{988.0}者,通姦(走入)他人的妻子者,虛妄地說者:無惡被作,如果以剃刀輪周邊使在這大地上的生類轉成一肉聚、一肉堆,從那個因由沒有惡的,沒有惡的傳來;如果走在恒河南岸,殺者、使他殺者,切斷者、使他切斷者,折磨者、使他折磨者,從那個因由沒有惡的,沒有惡的傳來;如果走在恒河北岸,施與者、使他施與者,祭祀者、使他祭祀者,從那個因由沒有福德,沒有福德的傳來;以布施,以調御,以抑制,以真實所言的,從那個因由沒有福德,沒有福德的傳來。』一位大師是這樣說者、這樣見者:『作者、使他作者,切斷者、使他切斷者,折磨者、使他折磨者,造成悲傷者、使他造成悲傷者,造成疲勞者、使他造成疲勞者,造成悸動者、使他造成悸動者,殺生者,未給予而取者、入侵人家者、奪取掠奪物者、作盜匪者、攔路搶劫者,通姦他人的妻子者,虛妄地說者:惡被作,如果以剃刀輪周邊使在這大地上的生類轉成一肉聚、一肉堆,從那個因由有惡的,有惡的傳來;如果走在恒河南岸,殺者、使他殺者,切斷者、使他切斷者,折磨者、使他折磨者,從那個因由有惡的,有惡的傳來;如果走在恒河北岸,施與者、使他施與者,祭祀者、使他祭祀者,從那個因由有福德,有福德的傳來;以布施,以調御,以抑制,以真實所言的,從那個因由有福德,有福德的傳來。』

  大德!那個我就有疑惑,有懷疑:『這些沙門、婆羅門\twnr{尊師}{203.0}們中,誰真實地說,誰虛妄地?』」

  「村長!你當然有疑惑,當然有懷疑:在疑惑之處上你的懷疑被生起。」

  「大德!我在世尊上有這樣的\twnr{淨信}{340.0}:世尊能夠為我教導像這樣的法,依之我能捨斷這個疑惑法。」

  「村長!有\twnr{法之定}{x577},在那裡,如果你得到\twnr{心定}{701.0},這樣,你能捨斷這個疑惑法。村長!而什麼是法之定?村長!這裡,聖弟子捨斷殺生後,是離殺生者;捨斷未給予而取後,是離未給予而取者;捨斷邪淫後,是離邪淫者;捨斷妄語後,是離妄語者;捨斷離間語後,是離離間語;捨斷粗惡語後,是離粗惡語;捨斷雜穢語後,是離雜穢語;捨斷貪婪後,是不貪婪者;捨斷惡意瞋怒後,是無瞋害心者;捨斷邪見後,是正見者。

  村長!那位這麼離貪婪、離惡意、不癡昧、正知、朝向念的聖弟子,他以與慈俱行之心遍滿一方後而住,像這樣第二的,像這樣第三的,像這樣第四的,像這樣上下、橫向、到處、對一切如對自己,以與慈俱行的、廣大的、變大的、無量的、無怨恨的、無瞋害的心遍滿全部世間後而住。他像這樣深慮:『凡這位大師是這樣知者、這樣見者:「無布施,無供養,無供物,無善作的、惡作的業之果與報,無此世,無他世,無母,無父,無化生眾生,在世間中無正行的、正行道的沙門、婆羅門以證智自作證後告知這個世間與其他世間。」如果那位大師先生的言語是真實的,有我的\twnr{無缺點狀態}{580.0}:我不傷害任何懦弱者或堅強者。在這裡,兩者都是贏家:我是以身自制者,以語自制者,以意自制者,與以身體的崩解,死後我將往生\twnr{善趣}{112.0}、天界。』對他,欣悅被生起;對喜悅者,\twnr{喜}{428.0}被生起;對\twnr{意喜}{320.0}者,身變得\twnr{寧靜}{313.0};\twnr{身已寧靜}{318.0}者感受樂;對有樂者,心入定。村長!這是法之定,在那裡,如果你得到心定,這樣,你能捨斷這個疑惑法。

  村長!那位這麼離貪婪、離惡意、不癡昧、正知、朝向念的聖弟子,他以與慈俱行之心遍滿一方後而住,像這樣第二的,像這樣第三的,像這樣第四的,像這樣上下、橫向、到處、對一切如對自己,以與慈俱行的、廣大的、變大的、無量的、無怨恨的、無瞋害的心遍滿全部世間後而住。他像這樣深慮:『凡這位大師是這樣知者、這樣見者:「有被施與的,有被祭祀的,有被供養的,有善作惡作業的果、果報,有這個世間,有其他世間,有母親,有父親,有化生眾生,在世間中有正行的、正行道的沙門、婆羅門以證智自作證後告知這個世間與其他世間。」如果那位大師先生的言語是真實的,有我的無缺點狀態:我不傷害任何懦弱者或堅強者。在這裡,兩者都是贏家:我是以身自制者,以語自制者,以意自制者,與以身體的崩解,死後我將往生善趣、天界。』他的欣悅被生,對喜悅者,喜被生,對意喜者來說身變得寧靜,身已寧靜者感受樂;對有樂者,心入定,村長!這是法之定,在那裡,如果你得到心定,這樣,你能捨斷這個疑惑法。

  村長!那位這麼離貪婪、離惡意、不癡昧、正知、朝向念的聖弟子,他以與慈俱行之心遍滿一方後而住,像這樣第二的,像這樣第三的,像這樣第四的,像這樣上下、橫向、到處、對一切如對自己,以與慈俱行的、廣大的、變大的、無量的、無怨恨的、無瞋害的心遍滿全部世間後而住。他像這樣深慮:『凡這位大師是這樣知者、這樣見者:「作者、使他作者,切斷者、使他切斷者,折磨拷打者、使他折磨拷打者,悲傷者、使他悲傷者,疲累者、使他疲累者,悸動者、使他悸動者,殺生者,未給予而取者、入侵人家者、奪取掠奪物者、作盜匪者、攔路搶劫者,通姦他人的妻子者,虛妄地說者:無惡被作,如果以剃刀輪周邊使在這大地上的生類轉成一肉聚、一肉堆,從那個因由沒有惡的,沒有惡的傳來;如果走在恒河南岸,殺者、使他殺者,切斷者、使他切斷者,折磨者、使他折磨者,從那個因由沒有惡的,沒有惡的傳來;如果走在恒河北岸,施與者、使他施與者,祭祀者、使他祭祀者,從那個因由沒有福德,沒有福德的傳來;以布施,以調御,以抑制,以真實所言的,從那個因由沒有福德,沒有福德的傳來。」如果那位大師先生的言語是真實的,有我的無缺點狀態:我不傷害任何懦弱者或堅強者。在這裡,兩者都是贏家:我是以身自制者,以語自制者,以意自制者,與以身體的崩解,死後我將往生善趣、天界。』他的欣悅被生,對喜悅者,喜被生,對意喜者來說身變得寧靜,身已寧靜者感受樂;對有樂者,心入定,村長!這是法之定,在那裡,如果你得到心定,這樣,你能捨斷這個疑惑法。

  村長!那位這麼離貪婪、離惡意、不癡昧、正知、朝向念的聖弟子,他以與慈俱行之心遍滿一方後而住,像這樣第二的,像這樣第三的,像這樣第四的,像這樣上下、橫向、到處、對一切如對自己,以與慈俱行的、廣大的、變大的、無量的、無怨恨的、無瞋害的心遍滿全部世間後而住。他像這樣深慮:『凡這位大師是這樣知者、這樣見者:「作者、使他作者,切斷者、使他切斷者,折磨者、使他折磨者,造成悲傷者、使他造成悲傷者,造成疲勞者、使他造成疲勞者,造成悸動者、使他造成悸動者,殺生者,未給予而取者、入侵人家者、奪取掠奪物者、作盜匪者、攔路搶劫者,通姦他人的妻子者,虛妄地說者:惡被作,如果以剃刀輪周邊使在這大地上的生類轉成一肉聚、一肉堆,從那個因由有惡的,有惡的傳來;如果走在恒河南岸,殺者、使他殺者,切斷者、使他切斷者,折磨者、使他折磨者,從那個因由有惡的,有惡的傳來;如果走在恒河北岸,施與者、使他施與者,祭祀者、使他祭祀者,從那個因由有福德,有福德的傳來;以布施,以調御,以抑制,以真實所言的,從那個因由有福德,有福德的傳來。」如果那位大師先生的言語是真實的,有我的無缺點狀態:我不傷害任何懦弱者或堅強者。在這裡,兩者都是贏家:我是以身自制者,以語自制者,以意自制者,與以身體的崩解,死後我將往生善趣、天界。』他的欣悅被生,對喜悅者,喜被生,對意喜者來說身變得寧靜,身已寧靜者感受樂;對有樂者,心入定,村長!這是法之定,在那裡,如果你得到心定,這樣,你能捨斷這個疑惑法。

  村長!那位這麼離貪婪、離惡意、不癡昧、正知、朝向念的聖弟子,他以與悲俱行之心遍滿一方後而住……(中略)以與喜悅俱行之心遍滿一方後而住……(中略)。

  村長!那位這麼離貪婪、離惡意、不癡昧、正知、朝向念的聖弟子,他以與\twnr{平靜}{228.0}俱行之心遍滿一方後而住,像這樣第二的,像這樣第三的,像這樣第四的,像這樣上下、橫向、到處、對一切如對自己,以與平靜俱行的、廣大的、變大的、無量的、無怨恨的、無瞋害的心遍滿全部世間後而住。他像這樣深慮:『凡這位大師是這樣知者、這樣見者:「無布施,無供養,無供物,無善作的、惡作的業之果與報,無此世,無他世,無母,無父,無化生眾生,在世間中無正行的、正行道的沙門、婆羅門以證智自作證後告知這個世間與其他世間。」如果那位大師先生的言語是真實的,有我的無缺點狀態:我不傷害任何懦弱者或堅強者。在這裡,兩者都是贏家:我是以身自制者,以語自制者,以意自制者,與以身體的崩解,死後我將往生善趣、天界。』他的欣悅被生,對喜悅者,喜被生,對意喜者來說身變得寧靜,身已寧靜者感受樂;對有樂者,心入定,村長!這是法之定,在那裡,如果你得到心定,這樣,你能捨斷這個疑惑法。

  村長!那位這麼離貪婪、離惡意、不癡昧、正知、朝向念的聖弟子,他以與平靜俱行之心遍滿一方後而住,像這樣第二的,像這樣第三的,像這樣第四的,像這樣上下、橫向、到處、對一切如對自己,以與平靜俱行的、廣大的、變大的、無量的、無怨恨的、無瞋害的心遍滿全部世間後而住。他像這樣深慮:『凡這位大師是這樣知者、這樣見者:「有被施與的,有被祭祀的,有被供養的,有善作惡作業的果、果報,有這個世間,有其他世間,有母親,有父親,有化生眾生,在世間中有正行的、正行道的沙門、婆羅門以證智自作證後告知這個世間與其他世間。」如果那位大師先生的言語是真實的,有我的無缺點狀態:我不傷害任何懦弱者或堅強者。在這裡,兩者都是贏家:我是以身自制者,以語自制者,以意自制者,與以身體的崩解,死後我將往生善趣、天界。』他的欣悅被生,對喜悅者,喜被生,對意喜者來說身變得寧靜,身已寧靜者感受樂;對有樂者,心入定,村長!這是法之定,在那裡,如果你得到心定,這樣,你能捨斷這個疑惑法。

  村長!那位這麼離貪婪、離惡意、不癡昧、正知、朝向念的聖弟子,他以與平靜俱行之心遍滿一方後而住,像這樣第二的,像這樣第三的,像這樣第四的,像這樣上下、橫向、到處、對一切如對自己,以與平靜俱行的、廣大的、變大的、無量的、無怨恨的、無瞋害的心遍滿全部世間後而住。他像這樣深慮:『凡這位大師是這樣知者、這樣見者:「作者、使他作者,切斷者、使他切斷者,折磨拷打者、使他折磨拷打者,悲傷者、使他悲傷者,疲累者、使他疲累者,悸動者、使他悸動者,殺生者,未給予而取者、入侵人家者、奪取掠奪物者、作盜匪者、攔路搶劫者,通姦他人的妻子者,虛妄地說者:無惡被作,如果以剃刀輪周邊使在這大地上的生類轉成一肉聚、一肉堆,從那個因由沒有惡的,沒有惡的傳來;如果走在恒河南岸,殺者、使他殺者,切斷者、使他切斷者,折磨者、使他折磨者,從那個因由沒有惡的,沒有惡的傳來;如果走在恒河北岸,施與者、使他施與者,祭祀者、使他祭祀者,從那個因由沒有福德,沒有福德的傳來;以布施,以調御,以抑制,以真實所言的,從那個因由沒有福德,沒有福德的傳來。」如果那位大師先生的言語是真實的,有我的無缺點狀態:我不傷害任何懦弱者或堅強者。在這裡,兩者都是贏家:我是以身自制者,以語自制者,以意自制者,與以身體的崩解,死後我將往生善趣、天界。』他的欣悅被生,對喜悅者,喜被生,對意喜者來說身變得寧靜,身已寧靜者感受樂;對有樂者,心入定,村長!這是法之定,在那裡,如果你得到心定,這樣,你能捨斷這個疑惑法。

  村長!那位這麼離貪婪、離惡意、不癡昧、正知、朝向念的聖弟子,他以與平靜俱行之心遍滿一方後而住,像這樣第二的,像這樣第三的,像這樣第四的,像這樣上下、橫向、到處、對一切如對自己,以與平靜俱行的、廣大的、變大的、無量的、無怨恨的、無瞋害的心遍滿全部世間後而住。他像這樣深慮:『凡這位大師是這樣知者、這樣見者:「作者、使他作者,切斷者、使他切斷者,折磨者、使他折磨者,造成悲傷者、使他造成悲傷者,造成疲勞者、使他造成疲勞者,造成悸動者、使他造成悸動者,殺生者,未給予而取者、入侵人家者、奪取掠奪物者、作盜匪者、攔路搶劫者,通姦他人的妻子者,虛妄地說者:惡被作,如果以剃刀輪周邊使在這大地上的生類轉成一肉聚、一肉堆,從那個因由有惡的,有惡的傳來;如果走在恒河南岸,殺者、使他殺者,切斷者、使他切斷者,折磨者、使他折磨者,從那個因由有惡的,有惡的傳來;如果走在恒河北岸,施與者、使他施與者,祭祀者、使他祭祀者,從那個因由有福德,有福德的傳來;以布施,以調御,以抑制,以真實所言的,從那個因由有福德,有福德的傳來。」如果那位大師先生的言語是真實的,有我的無缺點狀態:我不傷害任何懦弱者或堅強者。在這裡,兩者都是贏家:我是以身自制者,以語自制者,以意自制者,與以身體的崩解,死後我將往生善趣、天界。』他的欣悅被生,對喜悅者,喜被生,對意喜者來說身變得寧靜,身已寧靜者感受樂;對有樂者,心入定,村長!這是法之定,在那裡,如果你得到心定,這樣,你能捨斷這個疑惑法。」

  在這麼說時,村長玻得里亞對世尊說這個:

  「太偉大了,大德!太偉大了,大德!……(中略)從今天起\twnr{已終生歸依}{64.0}。」

  聚落主相應完成,其\twnr{攝頌}{35.0}:

  「兇惡、普達、戰士,象馬、阿西邦大葛,

   教說、海螺、家、摩尼朱羅葛,薄羅葛、羅西亞、玻得里亞。」





\page

\xiangying{43}{無為相應}
\pin{第一品}{1}{11}
\sutta{1}{1}{身至念經}{https://agama.buddhason.org/SN/sn.php?keyword=43.1}
  起源於舍衛城。

  「\twnr{比丘}{31.0}們!我將為你們教導\twnr{無為}{90.1}與導向無為之道,\twnr{你們要聽}{43.0}它!

  比丘們!而什麼是無為呢?比丘們!凡貪的滅盡、瞋的滅盡、癡的滅盡,比丘們!這被稱為無為。

  比丘們!而什麼是導向無為之道呢?是\twnr{身至念}{521.0},比丘們!這被稱為導向無為之道。

  比丘們!像這樣,我已為你們教導無為、導向無為之道。

  比丘們!凡\twnr{出自憐愍}{121.0}應該被老師、利益者、憐愍者為了弟子作的,那個被我為你們做了。比丘們!有這些樹下、這些空屋,比丘們!你們要修禪,不要放逸,不要以後成為後悔者,這是我們為你們的教誡。」



\sutta{2}{2}{止觀經}{https://agama.buddhason.org/SN/sn.php?keyword=43.2}
  「\twnr{比丘}{31.0}們!我將為你們教導\twnr{無為}{90.1}與導向無為之道,\twnr{你們要聽}{43.0}它!

  比丘們!而什麼是無為呢?比丘們!凡貪的滅盡、瞋的滅盡、癡的滅盡,比丘們!這被稱為無為。

  比丘們!而什麼是導向無為之道呢?是\twnr{止與觀}{178.0},比丘們!這被稱為導向無為之道。……(中略)。」



\sutta{3}{3}{有尋有伺經}{https://agama.buddhason.org/SN/sn.php?keyword=43.3}
  「……而,\twnr{比丘}{31.0}們!而什麼是導向\twnr{無為}{90.1}之道呢?是\twnr{有尋有伺}{175.0}定、\twnr{無尋唯伺定}{141.0}、無尋無伺定,比丘們!這被稱為導向無為之道。……(中略)。」



\sutta{4}{4}{空定經}{https://agama.buddhason.org/SN/sn.php?keyword=43.4}
  「……而,\twnr{比丘}{31.0}們!而什麼是導向\twnr{無為}{90.1}之道呢?是空定、無相定、\twnr{無願定}{571.0},比丘們!這被稱為導向無為之道。……(中略)。」



\sutta{5}{5}{念住經}{https://agama.buddhason.org/SN/sn.php?keyword=43.5}
  「……而,\twnr{比丘}{31.0}們!而什麼是導向\twnr{無為}{90.1}之道呢?是\twnr{四念住}{286.0},比丘們!這被稱為導向無為之道。……(中略)。」



\sutta{6}{6}{正勤經}{https://agama.buddhason.org/SN/sn.php?keyword=43.6}
  「……而,\twnr{比丘}{31.0}們!而什麼是導向\twnr{無為}{90.1}之道呢?是\twnr{四正勤}{292.0},比丘們!這被稱為導向無為之道。……(中略)。」



\sutta{7}{7}{神足經}{https://agama.buddhason.org/SN/sn.php?keyword=43.7}
  「……而,\twnr{比丘}{31.0}們!而什麼是導向\twnr{無為}{90.1}之道呢?是\twnr{四神足}{503.1},比丘們!這被稱為導向無為之道。……(中略)。」



\sutta{8}{8}{根經}{https://agama.buddhason.org/SN/sn.php?keyword=43.8}
  「……而,\twnr{比丘}{31.0}們!而什麼是導向\twnr{無為}{90.1}之道呢?是五根,比丘們!這被稱為導向無為之道。……(中略)。」



\sutta{9}{9}{力經}{https://agama.buddhason.org/SN/sn.php?keyword=43.9}
  「……而,\twnr{比丘}{31.0}們!而什麼是導向\twnr{無為}{90.1}之道呢?是五力,比丘們!這被稱為導向無為之道。……(中略)。」



\sutta{10}{10}{覺支經}{https://agama.buddhason.org/SN/sn.php?keyword=43.10}
  「……而,\twnr{比丘}{31.0}們!而什麼是導向\twnr{無為}{90.1}之道呢?是\twnr{七覺支}{524.0},比丘們!這被稱為導向無為之道。……(中略)。」



\sutta{11}{11}{道經}{https://agama.buddhason.org/SN/sn.php?keyword=43.11}
  「……而,\twnr{比丘}{31.0}們!而什麼是導向\twnr{無為}{90.1}之道呢?是\twnr{八支聖道}{525.0},比丘們!這被稱為導向無為之道。

  比丘們!像這樣,我已為你們教導無為、導向無為之道。

  比丘們!凡\twnr{出自憐愍}{121.0}應該被老師、利益者、憐愍者為了弟子作的,那個被我為你們做了。比丘們!有這些樹下、這些空屋,比丘們!你們要修禪,不要放逸,不要以後成為後悔者,這是我們為你們的教誡。」

  第一品,其\twnr{攝頌}{35.0}:

  「身、止、有尋,空、念住,

   正勤、神足,根、力、覺支,

   以道為第十一,被說為它的攝頌。」





\pin{第二品}{12}{44}
\sutta{12}{12}{無為經}{https://agama.buddhason.org/SN/sn.php?keyword=43.12}
  「\twnr{比丘}{31.0}們!我將為你們教導\twnr{無為}{90.1}與導向無為之道,\twnr{你們要聽}{43.0}它!

  比丘們!而什麼是無為呢?比丘們!凡貪的滅盡、瞋的滅盡、癡的滅盡,比丘們!這被稱為無為。

  比丘們!而什麼是導向無為之道呢?是\twnr{止}{178.0},比丘們!這被稱為導向無為之道。

  比丘們!像這樣,我已為你們教導無為、導向無為之道。

  比丘們!凡\twnr{出自憐愍}{121.0}應該被老師、利益者、憐愍者為了弟子作的,那個被我為你們做了。比丘們!有這些樹下、這些空屋,比丘們!你們要修禪,不要放逸,不要以後成為後悔者,這是我們為你們的教誡。」

  「比丘們!我將為你們教導無為與導向無為之道,你們要聽它!

  比丘們!而什麼是無為呢?比丘們!凡貪的滅盡、瞋的滅盡、癡的滅盡,比丘們!這被稱為無為。

  比丘們!而什麼是導向無為之道呢?是觀,比丘們!這被稱為導向無為之道。

  比丘們!像這樣,我已為你們教導無為、導向無為之道。

  ……(中略)這是我們為你們的教誡。」

  「比丘們!而什麼是導向無為之道呢?是\twnr{有尋有伺}{175.0}定,比丘們!這被稱為導向無為之道。……(中略)。」

  「比丘們!而什麼是導向無為之道呢?是\twnr{無尋唯伺定}{141.0},比丘們!這被稱為導向無為之道。……(中略)。」

  「比丘們!而什麼是導向無為之道呢?是無尋無伺定,比丘們!這被稱為導向無為之道。……(中略)。」

  「比丘們!而什麼是導向無為之道呢?是\twnr{空定}{571.0},比丘們!這被稱為導向無為之道。……(中略)。」

  「比丘們!而什麼是導向無為之道呢?是無相定,比丘們!這被稱為導向無為之道。……(中略)。」

  「比丘們!而什麼是導向無為之道呢?是無願定,比丘們!這被稱為導向無為之道。……(中略)。」

  「比丘們!而什麼是導向無為之道?比丘們!這裡,比丘住於\twnr{在身上隨看著身}{176.0}:熱心的、正知的、有念的,調伏世間中的\twnr{貪婪}{435.0}、憂後,比丘們!這被稱為導向無為之道。……(中略)。」

  「比丘們!而什麼是導向無為之道?比丘們!這裡,比丘住於在諸受上隨看著受……(中略)比丘們!這被稱為導向無為之道。……(中略)。」

  「比丘們!而什麼是導向無為之道?比丘們!這裡,比丘住於在心上隨看著心……(中略)比丘們!這被稱為導向無為之道。……(中略)。」

  「比丘們!而什麼是導向無為之道?比丘們!這裡,比丘住於在諸法上隨看著法……(中略)比丘們!這被稱為導向無為之道。……(中略)。」

  「比丘們!而什麼是導向無為之道?比丘們!這裡,比丘為了未生起的惡不善法之不生起使意欲生起、努力、發動活力、盡心、勤奮,比丘們!這被稱為導向無為之道。……(中略)。」

  「比丘們!而什麼是導向無為之道?比丘們!這裡,比丘為了已生起的惡不善法之捨斷使意欲生起、努力、發動活力、盡心、勤奮,比丘們!這被稱為導向無為之道。……(中略)。」

  「比丘們!而什麼是導向無為之道?比丘們!這裡,比丘為了未生起的善法之生起使意欲生起、努力、發動活力、盡心、勤奮,比丘們!這被稱為導向無為之道。……(中略)。」

  「比丘們!而什麼是導向無為之道?比丘們!這裡,比丘為了已生起的諸善法之存續、不忘失、增大、成滿、修習圓滿使意欲生起、努力、發動活力、盡心、勤奮,比丘們!這被稱為導向無為之道。……(中略)。」

  「比丘們!而什麼是導向無為之道?比丘們!這裡,比丘修習\twnr{具備意欲定勤奮之行的神足}{568.0},比丘們!這被稱為導向無為之道。……(中略)。」

  「比丘們!而什麼是導向無為之道?比丘們!這裡,比丘修習具備活力定勤奮之行的神足,比丘們!這被稱為導向無為之道。……(中略)。」

  「比丘們!而什麼是導向無為之道?比丘們!這裡,比丘修習具備心定勤奮之行的神足,比丘們!這被稱為導向無為之道。……(中略)。」

  「比丘們!而什麼是導向無為之道?比丘們!這裡,比丘修習\twnr{具備考察定勤奮之行的神足}{569.0},比丘們!這被稱為導向無為之道。……(中略)。」

  「比丘們!而什麼是導向無為之道?比丘們!這裡,比丘\twnr{依止遠離}{322.0}、依止離貪、依止滅、\twnr{捨棄的成熟}{221.0}修習信根,比丘們!這被稱為導向無為之道。……(中略)。」

  「比丘們!而什麼是導向無為之道?比丘們!這裡,比丘依止遠離……(中略)修習\twnr{活力根}{291.0},比丘們!這被稱為導向無為之道。……(中略)。」

  「比丘們!而什麼是導向無為之道?比丘們!這裡,比丘……(中略)修習念根,比丘們!這被稱為導向無為之道。……(中略)。」

  「比丘們!而什麼是導向無為之道?比丘們!這裡,比丘……(中略)修習定根,比丘們!這被稱為導向無為之道。……(中略)。」

  「比丘們!而什麼是導向無為之道?比丘們!這裡,比丘依止遠離、依止離貪、依止滅、捨棄的成熟修習慧根,比丘們!這被稱為導向無為之道。……(中略)。」

  「比丘們!而什麼是導向無為之道?比丘們!這裡,比丘依止遠離……(中略)修習信力,比丘們!這被稱為導向無為之道。……(中略)。」

  「比丘們!而什麼是導向無為之道?比丘們!這裡,比丘……(中略)修習\twnr{活力之力}{306.0},比丘們!這被稱為導向無為之道。……(中略)。」

  「比丘們!而什麼是導向無為之道?比丘們!這裡,比丘……(中略)修習念力,比丘們!這被稱為導向無為之道。……(中略)。」

  「比丘們!而什麼是導向無為之道?比丘們!這裡,比丘……(中略)修習定力,比丘們!這被稱為導向無為之道。……(中略)。」

  「比丘們!而什麼是導向無為之道?比丘們!這裡,比丘依止遠離、依止離貪、依止滅、捨棄的成熟修習慧力,比丘們!這被稱為導向無為之道。……(中略)。」

  「比丘們!而什麼是導向無為之道?比丘們!這裡,比丘……(中略)修習\twnr{念覺支}{315.0},比丘們!這被稱為導向無為之道。……(中略)。」

  「比丘們!而什麼是導向無為之道?比丘們!這裡,比丘……(中略)修習\twnr{擇法覺支}{311.0}……(中略)修習\twnr{活力覺支}{310.0}……(中略)修習\twnr{喜覺支}{312.0}……(中略)修習\twnr{寧靜覺支}{313.1}……(中略)修習\twnr{定}{182.0}覺支,比丘依止遠離、依止離貪、依止滅、捨棄的成熟修習\twnr{平靜覺支}{314.0},比丘們!這被稱為導向無為之道。……(中略)。」

  「比丘們!而什麼是導向無為之道?比丘們!這裡,比丘依止遠離、依止離貪、依止滅、捨棄的成熟修習正見,比丘們!這被稱為導向無為之道。……(中略)。」

  「比丘們!而什麼是導向無為之道?比丘們!這裡,比丘……(中略)修習正志……(中略)修習正語……(中略)修習正業……(中略)修習正命……(中略)修習正精進……(中略)修習正念。」

  「比丘們!我將為你們教導無為與導向無為之道,你們要聽它!

  比丘們!而什麼是無為呢?……(中略)。

  比丘們!而什麼是導向無為之道?比丘們!這裡,比丘依止遠離、依止離貪、依止滅、捨棄的成熟修習正定,比丘們!這被稱為導向無為之道。

  比丘們!像這樣,我已為你們教導無為、導向無為之道。

  比丘們!凡\twnr{出自憐愍}{121.0}應該被老師、利益者、憐愍者為了弟子作的,那個被我為你們做了。比丘們!有這些樹下、這些空屋,比丘們!你們要修禪,不要放逸,不要以後成為後悔者,這是我們為你們的教誡。」



\sutta{13}{13}{無傾斜經}{https://agama.buddhason.org/SN/sn.php?keyword=43.13}
  「\twnr{比丘}{31.0}們!我將為你們教導無傾斜與導向無傾斜之道,\twnr{你們要聽}{43.0}它!

  比丘們!而什麼是無傾斜呢?……(中略)。」(應該如無為那樣使之被細說)



\sutta{14}{43}{無漏經等}{https://agama.buddhason.org/SN/sn.php?keyword=43.14}
  「\twnr{比丘}{31.0}們!我將為你們教導無\twnr{漏}{188.0}與導向無漏之道,\twnr{你們要聽}{43.0}它!

  比丘們!而什麼是無漏呢?……(中略)。」

  「比丘們!我將為你們教導真實與導向真實之道,你們要聽它!

  比丘們!而什麼是真實呢?……(中略)。」

  「比丘們!我將為你們教導\twnr{彼岸}{226.0}與導向彼岸之道,你們要聽它!

  比丘們!而什麼是彼岸呢?……(中略)。」

  「比丘們!我將為你們教導微妙的與導向微妙的之道,你們要聽它!

  比丘們!而什麼是微妙的呢?……(中略)。」

  「比丘們!我將為你們教導極難見的與導向極難見的之道,你們要聽它!

  比丘們!而什麼是極難見的呢?……(中略)。」

  「比丘們!我將為你們教導不衰老的與導向不衰老的之道,你們要聽它!

  比丘們!而什麼是不衰老的呢?……(中略)。」

  「比丘們!我將為你們教導堅固的與導向堅固的之道,你們要聽它!

  比丘們!而什麼是堅固的呢?……(中略)。」

  「比丘們!我將為你們教導不壞散的與導向不壞散的之道,你們要聽它!

  比丘們!而什麼是不壞散的呢?……(中略)。」

  「比丘們!我將為你們教導不可見與導向不可見之道,你們要聽它!

  比丘們!而什麼是不可見呢?……(中略)。」

  「比丘們!我將為你們教導無\twnr{虛妄}{952.0}與導向無虛妄之道,你們要聽它!

  比丘們!而什麼是無虛妄呢?……(中略)。」

  「比丘們!我將為你們教導寂靜的與導向寂靜的之道,你們要聽它!

  比丘們!而什麼是寂靜的呢?……(中略)。」

  「比丘們!我將為你們教導\twnr{不死}{123.0}與導向不死之道,你們要聽它!

  比丘們!而什麼是不死的呢?……(中略)。」

  「比丘們!我將為你們教導勝妙的與導向勝妙的之道,你們要聽它!

  比丘們!而什麼是勝妙的呢?……(中略)。」

  「比丘們!我將為你們教導吉祥的與導向吉祥的之道,你們要聽它!

  比丘們!而什麼是吉祥的呢?……(中略)。」

  「比丘們!我將為你們教導安穩的與導向安穩的之道,你們要聽它!

  比丘們!而什麼是安穩的呢?……(中略)。」

  「比丘們!我將為你們教導渴愛的滅盡與導向渴愛的滅盡之道,你們要聽它!

  比丘們!而什麼是渴愛的滅盡呢?……(中略)。」

  「比丘們!我將為你們教導\twnr{不可思議的}{206.0}與導向不可思議的之道,你們要聽它!

  比丘們!而什麼是不可思議的呢?……(中略)。」

  「比丘們!我將為你們教導未曾有的與導向未曾有的之道,你們要聽它!

  比丘們!而什麼是未曾有的呢?……(中略)。」

  「比丘們!我將為你們教導無災難的與導向無災難的之道,你們要聽它!

  比丘們!而什麼是無災難的呢?……(中略)。」

  「比丘們!我將為你們教導無災難法與導向無災難法之道,你們要聽它!

  比丘們!而什麼是無災難法呢?……(中略)。」

  「比丘們!我將為你們教導涅槃與導向涅槃之道,你們要聽它!

  比丘們!而什麼是涅槃呢?……(中略)。」

  「比丘們!我將為你們教導無瞋害的與導向無瞋害的之道,你們要聽它!

  比丘們!而什麼是無瞋害的呢?……(中略)。」

  「比丘們!我將為你們教導離貪與導向離貪之道,你們要聽它!

  比丘們!而什麼是離貪呢?……(中略)。」

  「比丘們!我將為你們教導純淨與導向純淨之道,你們要聽它!

  比丘們!而什麼是純淨呢?……(中略)。」

  「比丘們!我將為你們教導解脫與導向解脫之道,你們要聽它!

  比丘們!而什麼是解脫呢?……(中略)。」

  「比丘們!我將為你們教導無\twnr{阿賴耶}{391.0}與導向無阿賴耶之道,你們要聽它!

  比丘們!而什麼是無阿賴耶呢?……(中略)。」

  「比丘們!我將為你們教導洲(島)與導向洲之道,你們要聽它!

  比丘們!而什麼是洲呢?……(中略)。」

  「比丘們!我將為你們教導庇護所與導向庇護所之道,你們要聽它!

  比丘們!而什麼是庇護所呢?……(中略)。」

  「比丘們!我將為你們教導救護所與導向救護所之道,你們要聽它!

  比丘們!而什麼是救護所呢?……(中略)。」

  「比丘們!我將為你們教導歸依處與導向歸依處之道,你們要聽它!

  比丘們!而什麼是歸依處呢?……(中略)。」



\sutta{44}{44}{彼岸經}{https://agama.buddhason.org/SN/sn.php?keyword=43.44}
  「\twnr{比丘}{31.0}們!我將為你們教導\twnr{彼岸}{226.0}與導向彼岸之道,\twnr{你們要聽}{43.0}它!

  比丘們!而什麼是彼岸呢?比丘們!凡貪的滅盡、瞋的滅盡、癡的滅盡,比丘們!這被稱為彼岸。

  比丘們!而什麼是導向彼岸之道呢?是\twnr{身至念}{521.0},比丘們!這被稱為導向彼岸之道。

  比丘們!像這樣,我已為你們教導彼岸、導向彼岸之道。

  比丘們!凡\twnr{出自憐愍}{121.0}應該被老師、利益者、憐愍者為了弟子作的,那個被我為你們做了。比丘們!有這些樹下、這些空屋,比丘們!你們要修禪,不要放逸,不要以後成為後悔者,這是我們為你們的教誡。」(應該如無為那樣使之被細說)

  第二品,其\twnr{攝頌}{35.0}:

  「無為、無傾斜、無漏,真實、彼岸、微妙的、極難見的,

   不衰老的、堅固的、不壞散的,不可見、無虛妄、寂靜的。

   不死、勝妙的、吉祥的、安穩的,渴愛的滅盡、不可思議的、未曾有的,

   無災難的、無災難法,涅槃,這被善逝教導。

   無瞋害的、離貪,純淨、解脫、無阿賴耶,

   島、庇護所、救護所,歸依處、彼岸。」

  無為相應完成。





\page

\xiangying{44}{無記相應}
\sutta{1}{1}{讖摩經}{https://agama.buddhason.org/SN/sn.php?keyword=44.1}
  \twnr{有一次}{2.0},\twnr{世尊}{12.0}住在舍衛城祇樹林給孤獨園。

  當時,讖摩\twnr{比丘尼}{31.0}在憍薩羅國進行著\twnr{遊行}{61.0},在舍衛城與娑雞多城中途的兜樂那哇堵入住。

  那時,當憍薩羅國波斯匿王從娑雞多城去舍衛城時,在娑雞多城與舍衛城中途的兜樂那哇堵入住一夜。

  那時,憍薩羅國波斯匿王召喚某位男子:

  「喂!男子!來!請你在兜樂那哇堵中找(知道)我今日能拜訪的\twnr{沙門}{29.0}或\twnr{婆羅門}{17.0}。」

  「是的,陛下!」那位男子回答憍薩羅國波斯匿王後,走遍整個兜樂那哇堵,沒看見憍薩羅國波斯匿王能拜訪的沙門或婆羅門。那位男子看見在兜樂那哇堵入住的讖摩比丘尼。看見後,去見憍薩羅國波斯匿王。抵達後,對憍薩羅國波斯匿王說這個:

  「陛下!在兜樂那哇堵中沒有像那樣陛下能拜訪的沙門或婆羅門,陛下!但有一位名叫讖摩的比丘尼,是那位世尊、\twnr{阿羅漢}{5.0}、\twnr{遍正覺者}{6.0}的女弟子。又,那位聖尼有這樣的好名聲被傳播:『是賢智者、聰明者、有智慧者、多聞者、巧說者、善辯才者。』請陛下拜訪她。」

  那時,憍薩羅國波斯匿王去見讖摩比丘尼。抵達後,向讖摩比丘尼\twnr{問訊}{46.0}後,在一旁坐下。在一旁坐下的憍薩羅國波斯匿王對讖摩比丘尼說這個:

  「聖尼!怎麼樣,死後如來存在嗎?」

  「大王!這不被世尊\twnr{記說}{179.0}:『死後如來存在。』」

  「聖尼!那麼,死後如來不存在嗎?」

  「大王!這也不被世尊記說:『死後如來不存在。』」

  「聖尼!怎麼樣,\twnr{死後如來存在且不存在}{354.0}嗎?」

  「大王!這不被世尊記說:『死後如來存在且不存在。』」

  「聖尼!那麼,死後如來既非存在也非不存在嗎?」

  「大王!這也不被世尊記說:『死後如來既非存在也非不存在。』」

  「聖尼!當被像這樣問:『怎麼樣?聖尼!死後如來存在嗎?』你說:『大王!這不被世尊記說:「死後如來存在。」』當被像這樣問:『聖尼!那麼,死後如來不存在嗎?』你說:『大王!這也不被世尊記說:「死後如來不存在。」』當被像這樣問:『怎麼樣?聖尼!死後如來存在且不存在?』你說:『大王!這不被世尊記說:「死後如來存在且不存在。」』當被像這樣問:『聖尼!那麼,死後如來既非存在也非不存在嗎?』你說:『大王!這也不被世尊記說:「死後如來既非存在也非不存在。」』聖尼!什麼因、什麼\twnr{緣}{180.0},以那個這不被沙門\twnr{喬達摩}{80.0}記說?」

  「大王!那樣的話,就在這件事上我將反問你,你就如對你能接受的那樣回答它。大王!你怎麼想它:你有任何會計師,或指算者,或計算者能夠計算在恒河中的沙:『有這麽多粒沙。』或『有這麽多百粒沙。』或『有這麽多千粒沙。』或『有這麽多十萬粒沙。』嗎?」

  「聖尼!這確實沒有。」

  「大王!又,你有任何會計師,或指算者,或計算者能夠計算在大海中的水:『有這麽多升水。』或『有這麽多百升水。』或『有這麽多千升水。』或『有這麽多十萬升水。』嗎?」

  「聖尼!這確實沒有。那是什麼原因?聖尼!大海是深的、\twnr{不能被測量的}{883.0}、難被深解的。」

  「同樣的,大王!\twnr{安立}{143.0}如來者凡以色會安立,如來的那個色已被捨斷,根已被切斷,\twnr{[如]已斷根的棕櫚樹}{147.1},\twnr{成為非有}{408.0},\twnr{為未來不生之物}{229.0},大王!\twnr{從色的名稱解脫的如來}{x578}是甚深的、不能被測量的,難被深解的,猶如大海[\ccchref{MN.72}{https://agama.buddhason.org/MN/dm.php?keyword=72}]。『死後如來存在』\twnr{不適用}{717.0},『死後如來不存在』不適用,『死後如來存在且不存在』不適用,『死後如來既非存在也非不存在』不適用。

  安立如來者凡以受會安立,如來的那個受已被捨斷,根已被切斷,[如]已斷根的棕櫚樹,成為非有,為未來不生之物,大王!從受的名稱解脫的如來是甚深的、不能被測量的,難被深解的,猶如大海。『死後如來存在』不適用,『死後如來不存在』不適用,『死後如來存在且不存在』不適用,『死後如來既非存在也非不存在』不適用。

  凡以想……(中略)安立如來者凡以諸行會安立,如來的那些諸行已被捨斷,根已被切斷,[如]已斷根的棕櫚樹,成為非有,為未來不生之物,大王!從行的名稱解脫的如來是甚深的、不能被測量的,難被深解的,猶如大海。『死後如來存在』不適用,『死後如來不存在』不適用,『死後如來存在且不存在』不適用,『死後如來既非存在也非不存在』不適用。

  安立如來者凡以識會安立,如來的那個識已被捨斷,根已被切斷,[如]已斷根的棕櫚樹,成為非有,為未來不生之物,大王!從識的名稱解脫的如來是甚深的、不能被測量的,難被深解的,猶如大海。『死後如來存在』不適用,『死後如來不存在』不適用,『死後如來存在且不存在』不適用,『死後如來既非存在也非不存在』不適用。」

  那時,憍薩羅國波斯匿王歡喜、\twnr{隨喜}{85.0}讖摩比丘尼所說後,從座位起來、向讖摩比丘尼\twnr{問訊}{46.0}、\twnr{作右繞}{47.0}後,離開。

  那時,過些時候,憍薩羅國波斯匿王去見世尊。抵達後,向世尊問訊後,在一旁站立。在一旁站立的憍薩羅國波斯匿王對世尊說這個:

  「怎麼樣,大德!死後如來存在嗎?」 

  「大王!這不被我所記說:『死後如來存在。』」 

  「大德!那麼,死後如來不存在嗎?」 

  「大王!這也不被我所記說:『死後如來不存在。』」 

  「怎麼樣,大德!死後如來存在且不存在嗎?」 

  「大王!這不被我所記說:『死後如來存在且不存在。』」 

  「大德!那麼,死後如來既非存在也非不存在嗎?」 

  「大王!這也不被我所記說:『死後如來既非存在也非不存在。』」 

  「大德!當被像這樣問:『怎麼樣?大德!死後如來存在嗎?』你說:『大王!這不被我所記說:「死後如來存在。」』當被像這樣問:『大德!那麼,死後如來不存在嗎?』你說:『大王!這也不被我所記說:「死後如來不存在。」』當被像這樣問:『怎麼樣?大德!死後如來存在且不存在?』你說:『大王!這不被我所記說:「死後如來存在且不存在。」』當被像這樣問:『大德!那麼,死後如來既非存在也非不存在嗎?』你說:『大王!這也不被我所記說:「死後如來既非存在也非不存在。」』大德!什麼因、什麼緣,以那個,這不被世尊記說?」 

  「大王!那樣的話,就在這件事上我將反問你,你就如對你能接受的那樣回答它。大王!你怎麼想它:你有任何會計師,或指算者,或計算者能夠計算在恒河中的沙:『有這麽多粒沙。』……(中略)或『有這麽多十萬粒沙。』嗎?」 

  「大德!這確實沒有。」 

  「大王!又,你有任何會計師,或指算者,或計算者能夠計算在大海中的水:『有這麽多升水。』……(中略)或『有這麽多十萬升水。』嗎?」 

  「大德!這確實沒有,那是什麼原因?大德!大海是深的、不可測量的、難被深解的。」 

  「同樣的,大王!安立如來者以凡色會安立,如來的那個色已被捨斷,根已被切斷,[如]已斷根的棕櫚樹,成為非有,為未來不生之物,大王!從色的名稱解脫的如來是甚深的、不能被測量的,難被深解的,猶如大海。『死後如來存在』不適用,『死後如來不存在』不適用,『死後如來存在且不存在』不適用,『死後如來既非存在也非不存在』不適用。 

  以凡受……(中略)以凡想……(中略)以凡諸行……(中略)安立如來者以凡識會安立,如來的那個識已被捨斷,根已被切斷,[如]已斷根的棕櫚樹,成為非有,為未來不生之物,大王!從識的名稱解脫的如來是甚深的、不能被測量的,難被深解的,猶如大海。『死後如來存在』不適用,『死後如來不存在』不適用,『死後如來存在且不存在』不適用,『死後如來既非存在也非不存在』不適用。」

  「不可思議啊,\twnr{大德}{45.0}!\twnr{未曾有}{206.0}啊,大德!確實是因為\twnr{大師}{145.0}正與弟子的道理與道理、字句與字句將能會合、將能集合、將能不違失,即:在最上句上。

  大德!有這一次,我去見讖摩比丘尼後,詢問這件事,那位聖尼也以這些句、這些字句為我解說這個義理,猶如世尊。

  不可思議啊,大德!未曾有啊,大德!確實是因為大師正與弟子的道理與道理、字句與字句將能會合、將能集合、將能不違失,即:在最上句上。

  好了,大德!而現在我們離開(走),我們們有許多工作、許多應該被做的。」

  「大王!現在是那個\twnr{你考量的時間}{84.0}」

  那時,憍薩羅國波斯匿王歡喜、\twnr{隨喜}{85.0}世尊所說後,從座位起來、向世尊問訊、\twnr{作右繞}{47.0}後,離開。



\sutta{2}{2}{阿奴羅度經}{https://agama.buddhason.org/SN/sn.php?keyword=44.2}
  \twnr{有一次}{2.0},\twnr{世尊}{12.0}住在毘舍離大林\twnr{重閣}{213.0}講堂。

  當時,\twnr{尊者}{200.0}阿奴羅度住在離世尊不遠處的\twnr{林野}{142.0}小屋中。

  那時,眾多其他外道\twnr{遊行者}{79.0}去見尊者阿奴羅度。抵達後,與尊者阿奴羅度一起互相問候。交換應該被互相問候的友好交談後,在一旁坐下。在一旁坐下的那些其他外道遊行者對尊者阿奴羅度說這個:

  「阿奴羅度\twnr{道友}{201.0}!當\twnr{安立}{143.0}那位最高的人、無上的人、已證得無上成就的\twnr{如來}{4.0}時,在這四個地方安立:『死後如來存在』,或『死後如來不存在』,或『\twnr{死後如來存在且不存在}{354.0}』,或『死後如來既非存在也非不存在』。」

  [在這麼說時,尊者阿奴羅度對那些其他外道遊行者說這個:]

  「道友!當安立那位最高的人、無上的人、已證得無上成就的如來時,\twnr{從除了這四個地方外}{x427}安立:『死後如來存在』,或『死後如來不存在』,或『死後如來存在且不存在』,或『死後如來既非存在也非不存在』。」

  在這麼說時,那些其他外道遊行者們對尊者阿奴羅度說這個:

  「這位必將是出家不久的新\twnr{比丘}{31.0},又或他是愚笨的、無能的\twnr{上座}{135.0}。」

  那時,其他外道遊行者們以「新的、愚笨的」之語貶抑尊者阿奴羅度後,從座位起來後離開。

  那時,在[那些]其他外道遊行者離開不久,尊者阿奴羅度想這個:

  「如果那些其他外道遊行者進一步問我,那麼,我怎樣對那些其他外道遊行者回答,才\twnr{會是世尊的所說之說者}{115.0},而且不會以不實的誹謗世尊,以及會\twnr{法隨法地回答}{415.0},而任何如法的種種說不會來到應該被呵責處?」

  那時,尊者阿奴羅度去見世尊。抵達後,向世尊\twnr{問訊}{46.0}後,在一旁坐下。在一旁坐下的尊者阿奴羅度對世尊說這個:

  「\twnr{大德}{45.0}!這裡,我住在離世尊不遠處的林野小屋中,大德!那時,眾多其他外道遊行者來見我,抵達後,與我一起互相問候。交換應該被互相問候的友好交談後,在一旁坐下。在一旁坐下的那些其他外道遊行者對我說這個:『阿奴羅度道友!當安立那位最高的人、無上的人、已證得無上成就的如來時,在這四個地方安立:「死後如來存在」……(中略)或「死後如來既非存在也非不存在」。』大德!在這麼說時,我對那些其他外道遊行者說這個:『道友!當安立那位最高的人、無上的人、已證得無上成就的如來時,從除了這四個地方外安立:「死後如來存在」……(中略)或「死後如來既非存在也非不存在」。』大德!在這麼說時,其他外道遊行者們對我說這個:『這位一定是新比丘,出家不久,或者是愚笨的、無能的上座。』大德!那時,其他外道遊行者們以『新的、愚笨的』之語貶抑我後,從座位起來後離開。 大德!那時,當那些其他外道遊行者離開不久,那個我想這個:『如果那些其他外道遊行者進一步問我,那麼,我怎樣對那些其他外道遊行者回答,才會是世尊的所說之說,而且不會以不實的誹謗世尊,以及能法隨法地回答,而任何如法的種種說不會來到應該被呵責處?』」 

  「阿奴羅度!你怎麼想它:色是常的,或是無常的?」

  「無常的,大德!」

  「那麼,凡為無常的,那是苦的或樂的?」

  「苦的,大德!」

  「那麼,凡為無常的、苦的、\twnr{變易法}{70.0},適合認為它:『\twnr{這是我的}{32.0},\twnr{我是這個}{33.0},這是\twnr{我的真我}{34.0}。』嗎?」

  「大德!這確實不是。」

  「受是常的,或是無常的?」……(中略)想……(中略)諸行……(中略)

  「識是常的,或是無常的?」

  「無常的,大德!」

  「那麼,凡為無常的,那是苦的或樂的?」

  「苦的,大德!」

  「那麼,凡為無常的、苦的、變易法,適合認為它:『這是我的,我是這個,這是我的真我。』嗎?」

  「大德!這確實不是。」

  「阿奴羅度!因此,在這裡,凡任何色:過去、未來、現在,或內、或外,或粗、或細,或下劣、或勝妙,或凡在遠處、在近處,所有色:『\twnr{這不是我的}{32.1},\twnr{我不是這個}{33.1},\twnr{這不是我的真我}{34.2}。』這樣,這個應該以正確之慧如實被看見。

  凡任何受:過去、未來、現在……(中略)凡任何想……(中略)凡任何諸行……(中略)凡任何識:過去、未來、現在,或內、或外,或粗、或細,或下劣、或勝妙,或凡在遠處、在近處,所有識:『這不是我的,我不是這個,這不是我的真我。』這樣,這個應該以正確之慧如實被看見。

  阿奴羅度!這麼看的\twnr{有聽聞的聖弟子}{24.0}在色上\twnr{厭}{15.0},也在受上厭,也在想上厭,也在諸行上厭,也在識上厭。厭者\twnr{離染}{558.0},從\twnr{離貪}{77.0}被解脫,在已解脫時,\twnr{有『[這是]解脫』之智}{27.0},他知道:『\twnr{出生已盡}{18.0},\twnr{梵行已完成}{19.0},\twnr{應該被作的已作}{20.0},\twnr{不再有此處[輪迴]的狀態}{21.1}。』

  阿奴羅度!你怎麼想它:『色是如來。』嗎?」

  「大德!這確實不是。」

  「你認為:『受是如來。』嗎?」

  「大德!這確實不是。」

  「你認為:『想是如來。』嗎?」

  「大德!這確實不是。」

  「你認為:『諸行是如來。』嗎?」

  「大德!這確實不是。」

  「你認為:『識是如來。』嗎?」

  「大德!這確實不是。」

  「阿奴羅度!你怎麼想它:『如來是在色中。』嗎?」

  「大德!這確實不是。」

  「你認為:『除了色外有如來。』嗎?」

  「大德!這確實不是。」

  「你認為在受中……(中略)在受以外的其它處……(中略)在想中……(中略)在想以外的其它處……(中略)在行中……(中略)在行以外的其它處……(中略)你認為:『如來是在識中。』嗎?」

  「大德!這確實不是。」

  「你認為:『除了識外有如來。』嗎?」

  「大德!這確實不是。」

  「阿奴羅度!你怎麼想它:『色受想諸行識是如來。』嗎?」

  「大德!這確實不是。」

  「阿奴羅度!你怎麼想它:『這位那個無色者……無受者……無想者……無行者……無識者是如來。』嗎?」

  「大德!這確實不是。」

  「阿奴羅度!而在這裡,當在此生中真實的、實際的如來未被你得到時,你的那個記說:『道友!當安立那位最上人、最高人、得到最高成就的如來時,從除了這四個地方外安立:「死後如來存在」,或「死後如來不存在」,或「死後如來存在且不存在」,或「死後如來既非存在也非不存在」。』是否是適當的呢?」

  「大德!這確實不是。」

  「阿奴羅度!\twnr{好}{44.0}!好!阿奴羅度!從以前到現在,\twnr{我只告知苦,連同苦的滅}{x579}。」[\suttaref{SN.22.86}]



\sutta{3}{3}{舍利弗與拘絺羅經第一}{https://agama.buddhason.org/SN/sn.php?keyword=44.3}
  \twnr{有一次}{2.0},\twnr{尊者}{200.0}舍利弗與尊者摩訶拘絺羅,住在波羅奈仙人墜落處的鹿林。

  那時,尊者摩訶拘絺羅傍晚時,從\twnr{獨坐}{92.0}出來,去見尊者舍利弗。抵達後,與尊者舍利弗一起互相問候。交換應該被互相問候的友好交談後,在一旁坐下。在一旁坐下的尊者摩訶拘絺羅對尊者舍利弗說這個:

  「舍利弗\twnr{學友}{201.0}!怎麼樣,死後如來存在嗎?」

  「學友!這不被\twnr{世尊}{12.0}所\twnr{記說}{179.0}:『死後如來存在。』」

  「學友!那麼,死後如來不存在嗎?」

  「學友!這也不被世尊記說:『死後如來不存在。』」

  「學友!怎麼樣,\twnr{死後如來存在且不存在}{354.0}嗎?」

  「學友!這不被世尊記說:『死後如來存在且不存在。』」

  「學友!那麼,死後如來既非存在也非不存在嗎?」

  「學友!這也不被世尊記說:『死後如來既非存在也非不存在。』」

  「學友!當被像這樣問:『怎麼樣?學友!死後如來存在嗎?』你說:『學友!這不被世尊記說:「死後如來存在。」』……(中略)當被像這樣問:『學友!那麼,死後如來既非存在也非不存在嗎?』你說:『學友!這也不被世尊記說:「死後如來既非存在也非不存在。」』學友!什麼因、什麼\twnr{緣}{180.0},以那個這不被世尊記說?」

  「學友!『死後如來存在。』\twnr{這是色之類的}{x580};『死後如來不存在。』這是色之類的;『死後如來存在且不存在。』這是色之類的;『死後如來既非存在也非不存在。』這是色之類的。學友!『死後如來存在。』這是受之類的;『死後如來不存在。』這是受之類的;『死後如來存在且不存在。』這是受之類的;『死後如來既非存在也非不存在。』這是受之類的。學友!『死後如來存在。』這是想之類的;『死後如來不存在。』這是想之類的;『死後如來存在且不存在。』這是想之類的;『死後如來既非存在也非不存在。』這是想之類的。學友!『死後如來存在。』這是行之類的;『死後如來不存在。』這是行之類的;『死後如來存在且不存在。』這是行之類的;『死後如來既非存在也非不存在。』這是行之類的。學友!『死後如來存在。』這是識之類的;『死後如來不存在。』這是識之類的;『死後如來存在且不存在。』這是識之類的;『死後如來既非存在也非不存在。』這是\twnr{識之類的}{923.0}。學友!這是因、這是緣,以那個這不被世尊記說。」



\sutta{4}{4}{舍利弗與拘絺羅經第二}{https://agama.buddhason.org/SN/sn.php?keyword=44.4}
  \twnr{有一次}{2.0},\twnr{尊者}{200.0}舍利弗與尊者摩訶拘絺羅,住在波羅奈仙人墜落處的鹿林。……(中略)(如前經所問)

  「學友!什麼因、什麼\twnr{緣}{180.0},以那個這不被世尊記說?」

  「學友!不如實知見色者;不如實知見色\twnr{集}{67.0}者;不如實知見色\twnr{滅}{68.0}者;不如實知見導向色\twnr{滅道跡}{69.0}者,他會想:『死後如來存在。』他也會想:『死後如來不存在。』他也會想:『\twnr{死後如來存在且不存在}{354.0}。』他也會想:『死後如來既非存在也非不存在。』受……(中略)想……(中略)諸行……(中略)不如實知見識者;不如實知見識集者;不如實知見識滅者;不如實知見導向識滅道跡者,他會想:『死後如來存在。』他也會想:『死後如來不存在。』他也會想:『死後如來存在且不存在。』他也會想:『死後如來既非存在也非不存在。』

  學友!但,如實知見色者;如實知見色集者;如實知見色滅者;如實知見導向色滅道跡者,他不會想:『死後如來存在。』……(中略)他也不會想:死後如來既非存在也非不存在。』受……(中略)想……(中略)行……(中略)如實知見識者;如實知見識集者;如實知見識滅者;如實知見導向識滅道跡者,他不會想:『死後如來存在。』他也不會想:『死後如來不存在。』他也不會想:『死後如來存在且不存在。』他也不會想:死後如來既非存在也非不存在。』

  學友!這是因、這是緣,以那個這不被世尊記說。」



\sutta{5}{5}{舍利弗與拘絺羅經第三}{https://agama.buddhason.org/SN/sn.php?keyword=44.5}
  \twnr{有一次}{2.0},\twnr{尊者}{200.0}舍利弗與尊者摩訶拘絺羅,住在波羅奈仙人墜落處的鹿林。……(中略)(如前經所問)

  「學友!什麼因、什麼\twnr{緣}{180.0},以那個這不被世尊記說?」

  「學友!在諸色上是未離貪者、未離意欲、未離情愛、未離渴望、未離熱惱、未離渴愛者,他會想:『死後如來存在。』……(中略)他也會想:『死後如來既非存在也非不存在。』受……(中略)想……(中略)諸行……(中略)對識未離貪、未離意欲、未離情愛、未離渴望、未離熱惱、未離渴愛者,他會想:『死後如來存在。』……(中略)他也會想:『死後如來既非存在也非不存在。』

  學友!但,對色已離貪……(中略)受……(中略)想……(中略)諸行……(中略)對識已離貪、已離意欲、已離情愛、已離渴望、已離熱惱、已離渴愛者,他不會想:『死後如來存在。』……(中略)他也不會想:『死後如來既非存在也非不存在。』

  學友!這是因、這是緣,以那個這不被世尊記說。」



\sutta{6}{6}{舍利弗與拘絺羅經第四}{https://agama.buddhason.org/SN/sn.php?keyword=44.6}
  \twnr{有一次}{2.0},\twnr{尊者}{200.0}舍利弗與尊者摩訶拘絺羅,住在波羅奈仙人墜落處的鹿林。

  那時,尊者舍利弗傍晚時,從\twnr{獨坐}{92.0}出來,去見尊者摩訶拘絺羅。抵達後,與尊者摩訶拘絺羅一起互相問候。交換應該被互相問候的友好交談後,在一旁坐下。在一旁坐下的對尊者摩訶拘絺羅說這個:

  「拘絺羅\twnr{學友}{201.0}!怎麼樣,死後如來存在嗎?」……(中略)

  「……當被像這樣問:『學友!那麼,死後如來既非存在也非不存在嗎?』你說:『學友!這也不被世尊記說:「死後如來既非存在也非不存在。」』學友!什麼因、什麼\twnr{緣}{180.0},以那個這不被世尊記說?」

  「學友!有色的快樂、有樂於色、有喜於色,對色\twnr{滅}{68.0}不如實知見者(如實不知者不見者)會想:『死後如來存在。』也會想:『死後如來不存在。』也會想:『死後如來存在且不存在。』也會想:『死後如來既非存在也非不存在。』學友!有受的快樂、有樂於受、有喜於受,對受滅不如實知見者會想:『死後如來存在。』……(中略)學友!有想的快樂……(中略)學友!有行的快樂……(中略)學友!有識的快樂、有樂於識、有喜於識,對識滅不如實知見者會想:『死後如來存在。』……(中略)也會想:『死後如來既非存在也非不存在。』」

  學友!沒有色的快樂、沒有樂於色、沒有喜於色,對色滅如實知見者(如實知者見者)不會想:『死後如來存在。』……(中略)也不會想:死後如來既非存在也非不存在。』學友!沒有受的快樂……(中略)學友!沒有想的快樂……(中略)學友!沒有行的快樂……(中略)學友!沒有識的快樂、沒有樂於識、沒有喜於識,對識滅如實知見者不會想:『死後如來存在。』……(中略)也不會想:死後如來既非存在也非不存在。』 

  學友!這是因、這是緣,以那個這不被世尊記說。」

  「學友!會有其他\twnr{法門}{562.0},以那個,這不被世尊記說嗎?」

  「學友!會有的。學友!有-有的快樂、有樂於有、有喜於有,對有滅不如實知見者會想:『死後如來存在。』……(中略)也會想:『死後如來既非存在也非不存在。』學友!沒有-有的快樂、沒有樂於有、沒有喜於有,對有滅如實知見者不會想:『死後如來存在。』……(中略)也不會想:死後如來既非存在也非不存在。』學友!這是因、這是緣,以那個,這不被世尊記說。」

  「學友!會有其他法門,以那個,這不被世尊記說嗎?」

  「學友!會有的。學友!有取的快樂、有樂於取、有喜於取,對取滅不如實知見者會想:『死後如來存在。』……(中略)也會想:『死後如來既非存在也非不存在。』學友!沒有取的快樂、沒有樂於取、沒有喜於取,對取滅如實知見者不會想:『死後如來存在。』……(中略)也不會想:死後如來既非存在也非不存在。』學友!這是因、這是緣,以那個這不被世尊記說。」

  「學友!會有其他法門,以那個,這不被世尊記說嗎?」

  「學友!會有的。學友!有渴愛的快樂、有樂於渴愛、有喜於渴愛,對渴愛不如實知見滅者會想:『死後如來存在。』……(中略)也會想:『死後如來既非存在也非不存在。』學友!沒有渴愛的快樂、沒有樂於渴愛、沒有喜於渴愛,對渴愛滅如實知見者不會想:『死後如來存在。』……(中略)也不會想:死後如來既非存在也非不存在。』學友!這是因、這是緣,以那個這不被世尊記說。」

  「學友!會有其他法門,以那個,這不被世尊記說嗎?」

  「舍利弗學友!現在,在這裡,從這裡之上你想要什麼呢?舍利弗學友!對\twnr{渴愛之滅盡解脫的}{834.0}\twnr{比丘}{31.0}來說,沒有輪迴的\twnr{安立}{143.0}。」



\sutta{7}{7}{目揵連經}{https://agama.buddhason.org/SN/sn.php?keyword=44.7}
  那時,\twnr{遊行者}{79.0}婆蹉氏去見\twnr{尊者}{200.0}目揵連。抵達後,與尊者目揵連一起互相問候。交換應該被互相問候的友好交談後,在一旁坐下。在一旁坐下的遊行者婆蹉氏對尊者目揵連說這個:

  「目揵連尊師!怎麼樣,世界是常恆的嗎?」

  「婆蹉!這不被世尊記說:『世界是常恆的。』」

  「目揵連尊師!那麼,\twnr{世界是非常恆的}{170.0}嗎?」

  「婆蹉!這也不被世尊記說:『世界是非常恆的。』」

  「目揵連尊師!怎麼樣,世界是有邊的嗎?」

  「婆蹉!這不被世尊記說:『世界是有邊的。』」

  「目揵連尊師!那麼,世界是無邊的嗎?」

  「婆蹉!這也不被世尊記說:『世界是無邊的。』」

  「目揵連尊師!怎麼樣,命即是身體嗎?」

  「婆蹉!這不被世尊記說:『命即是身體。』」

  「目揵連尊師!那麼,\twnr{命是一身體是另一}{169.0}嗎?」

  「婆蹉!這也不被世尊記說:『命是一身體是另一。』」

  「目揵連尊師!怎麼樣,死後如來存在嗎?」

  「婆蹉!這不被世尊記說:『死後如來存在。』」

  「目揵連尊師!那麼,死後如來不存在嗎?」

  「婆蹉!這也不被世尊記說:『死後如來不存在。』」

  「目揵連尊師!怎麼樣,\twnr{死後如來存在且不存在}{354.0}嗎?」

  「婆蹉!這不被世尊記說:『死後如來存在且不存在。』」

  「目揵連尊師!那麼,死後如來既非存在也非不存在嗎?」

  「婆蹉!這也不被世尊記說:『死後如來既非存在也非不存在。』」

  「目揵連尊師!什麼因、什麼\twnr{緣}{180.0},以那個被這樣詢問的其他外道遊行者們有這樣的回答:『世界是常恆的,或世界是非常恆的,或世界是有邊的,或世界是無邊的,或命即是身體,或命是一身體是另一,或死後如來存在,或死後如來不存在,或死後如來存在且不存在,或死後如來既非存在也非不存在。』呢?目揵連尊師!又,什麼因、什麼緣,以那個被這樣詢問的\twnr{沙門}{29.0}\twnr{喬達摩}{80.0}沒有這樣的回答:『世界是常恆的,或世界是非常恆的,或世界是有邊的,或世界是無邊的,或命即是身體,或命是一身體是另一,或死後如來存在,或死後如來不存在,或死後如來存在且不存在,或死後如來既非存在也非不存在。』呢?」

  「婆蹉!其他外道遊行者們\twnr{認為}{964.0}眼:『\twnr{這是我的}{32.0},\twnr{我是這個}{33.0},這是\twnr{我的真我}{34.0}。』……(中略)認為舌:『這是我的,我是這個,這是我的真我。』……(中略)認為意:『這是我的,我是這個,這是我的真我。』因此,被這樣詢問的其他外道遊行者們有這樣的回答:『世界是常恆的……(中略)或死後如來既非存在也非不存在。』

  婆蹉!但\twnr{如來}{4.0}、\twnr{阿羅漢}{5.0}、\twnr{遍正覺者}{6.0}認為眼:『\twnr{這不是我的}{32.1},\twnr{我不是這個}{33.1},\twnr{這不是我的真我}{34.2}。』……(中略)認為舌:『這不是我的,我不是這個,這不是我的真我。』……(中略)認為意:『這不是我的,我不是這個,這不是我的真我。』因此,被這樣詢問的如來沒有這樣的回答:『世界是常恆的……(中略)或死後如來既非存在也非不存在。』」

  那時,遊行者婆蹉氏從座位起來後去見\twnr{世尊}{12.0}。抵達後,與世尊一起互相問候。交換應該被互相問候的友好交談後,在一旁坐下。在一旁坐下的遊行者婆蹉氏對世尊說這個:

  「喬達摩尊師!怎麼樣,世界是常恆的嗎?」

  「婆蹉!這不被我所記說:『世界是常恆的。』」……(中略)

  「喬達摩尊師!那麼,死後如來既非存在也非不存在嗎?」

  「婆蹉!這也不被我所記說:『死後如來既非存在也非不存在。』」

  「喬達摩尊師!什麼因、什麼緣,以那個被這樣詢問的其他外道遊行者們有這樣的回答:『世界是常恆的……(中略)或死後如來既非存在也非不存在。』呢?喬達摩尊師!又,什麼因、什麼緣,被這樣詢問的喬達摩尊師沒有這樣的回答:『世界是常恆的……(中略)或死後如來既非存在也非不存在。』呢?」

  「婆蹉!其他外道遊行者們認為眼:『這是我的,我是這個,這是我的真我。』……(中略)認為舌:『這是我的,我是這個,這是我的真我。』……(中略)認為意:『這是我的,我是這個,這是我的真我。』因此,被這樣詢問的其他外道遊行者們有這樣的回答:『世界是常恆的……(中略)或死後如來既非存在也非不存在。』

  婆蹉!但如來、阿羅漢、遍正覺者認為眼:『這不是我的,我不是這個,這不是我的真我。』……(中略)認為舌:『這不是我的,我不是這個,這不是我的真我。』……(中略)認為意:『這不是我的,我不是這個,這不是我的真我。』因此,被這樣詢問的如來沒有這樣的回答:『世界是常恆的,或世界是非常恆的,或世界是有邊的,或世界是無邊的,或命即是身體,或命是一身體是另一,或死後如來存在,或死後如來不存在,或死後如來存在且不存在,或死後如來既非存在也非不存在。』」

  「不可思議啊,喬達摩尊師!\twnr{未曾有}{206.0}啊,喬達摩尊師!確實是因為\twnr{大師}{145.0}正與弟子的道理與道理、字句與字句將能會合、將能集合、將能不違失,即:在最上句上。

  喬達摩尊師!這裡,我去見沙門目揵連後,詢問這件事,沙門目揵連也以這些句、這些字句為我解說這個義理,猶如\twnr{喬達摩}{80.0}\twnr{尊師}{203.0}。

  不可思議啊,喬達摩尊師!未曾有啊,喬達摩尊師!確實是因為大師正與弟子的道理與道理、字句與字句將能會合、將能集合、將能不違失,即:在最上句上。」



\sutta{8}{8}{婆蹉氏經}{https://agama.buddhason.org/SN/sn.php?keyword=44.8}
  那時,\twnr{遊行者}{79.0}婆蹉氏去見\twnr{世尊}{12.0}。抵達後,與世尊一起互相問候。交換應該被互相問候的友好交談後,在一旁坐下。在一旁坐下的遊行者婆蹉氏對世尊說這個:

  「\twnr{喬達摩}{80.0}尊師!怎麼樣,世界是常恆的嗎?」

  「婆蹉!這不被我所記說:『世界是常恆的。』」……(中略)

  「喬達摩尊師!那麼,死後如來既非存在也非不存在嗎?」

  「婆蹉!這也不被我所記說:『死後如來既非存在也非不存在。』」

  「喬達摩尊師!什麼因、什麼\twnr{緣}{180.0},以那個被這樣詢問的其他外道遊行者們有這樣的回答:『世界是常恆的……(中略)或死後如來既非存在也非不存在。』呢?喬達摩尊師!又,什麼因、什麼緣,被這樣詢問的喬達摩尊師沒有這樣的回答:『世界是常恆的……(中略)或死後如來既非存在也非不存在。』呢?」

  「婆蹉!其他外道遊行者們\twnr{認為}{964.0}色是我,\twnr{或我擁有色}{13.0},或色在我中,\twnr{或我在色中}{14.0},認為受是我……(中略)想……(中略)諸行……(中略)認為識是我,或我擁有識,或識在我中,或我在識中,因此,被這樣詢問的其他外道遊行者們有這樣的回答:『世界是常恆的……(中略)或死後如來既非存在也非不存在。』

  婆蹉!但\twnr{如來}{4.0}、\twnr{阿羅漢}{5.0}、\twnr{遍正覺者}{6.0}認為色不是我,或我不擁有色,或色不在我中,或我不在色中,認為受不是我……(中略)認為想不……(中略)認為諸行不……(中略)認為識不是我,或我不擁有識,或識不在我中,或我不在識中,因此,被這樣詢問的如來沒有這樣的回答:『世界是常恆的……(中略)或死後如來既非存在也非不存在。』」

  那時,遊行者婆蹉氏從座位起來後去見\twnr{尊者}{200.0}目揵連。抵達後,與尊者目揵連一起互相問候。交換應該被互相問候的友好交談後,在一旁坐下。在一旁坐下的遊行者婆蹉氏對尊者目揵連說這個:

  「目揵連尊師!怎麼樣,世界是常恆的嗎?」

  「婆蹉!這不被世尊記說:『世界是常恆的。』」……(中略)

  「目揵連尊師!那麼,死後如來既非存在也非不存在嗎?」

  「婆蹉!這也不被世尊記說:『死後如來既非存在也非不存在。』」

  「目揵連尊師!什麼因、什麼緣,以那個被這樣詢問的其他外道遊行者們有這樣的回答:『世界是常恆的……(中略)或死後如來既非存在也非不存在。』呢?目揵連尊師!又,什麼因、什麼緣,以那個被這樣詢問的\twnr{沙門}{29.0}喬達摩這麼問時,他不這麼回答:『世界是常恆的……(中略)或死後如來既非存在也非不存在。』呢?」

  「婆蹉!其他外道遊行者們認為色是我,或我擁有色,或色在我中,或我在色中,認為受是我……(中略)想……(中略)諸行……(中略)認為識是我,或我擁有識,或識在我中,或我在識中,因此,被這樣詢問的其他外道遊行者們有這樣的回答:『世界是常恆的……(中略)或死後如來既非存在也非不存在。』

  婆蹉!但如來、阿羅漢、遍正覺者認為色不是我,或我不擁有色,或色不在我中,或我不在色中,認為受不是我……(中略)認為想不……(中略)認為諸行不……(中略)認為識不是我,或我不擁有識,或識不在我中,或我不在識中,因此,被這樣詢問的如來沒有這樣的回答:『世界是常恆的,或\twnr{世界是非常恆的}{170.0},或世界是有邊的,或世界是無邊的,或命即是身體,或\twnr{命是一身體是另一}{169.0},或死後如來存在,或死後如來不存在,或死後如來存在且不存在,或死後如來既非存在也非不存在。』」

  「不可思議啊,目揵連尊師!\twnr{未曾有}{206.0}啊,目揵連尊師!確實是因為大師正與弟子的道理與道理、字句與字句將能會合、將能集合、將能不違失,即:在最上句上。

  目揵連尊師!這裡,我去見沙門喬達摩後,詢問這件事,沙門喬達摩也以這些句、這些字句為我解說這個義理,猶如目揵連\twnr{尊師}{203.0}。

  不可思議啊,目揵連尊師!未曾有啊,目揵連尊師!確實是因為大師正與弟子的道理與道理、字句與字句將能會合、將能集合、將能不違失,即:在最上句上。」



\sutta{9}{9}{論議堂經}{https://agama.buddhason.org/SN/sn.php?keyword=44.9}
  那時,\twnr{遊行者}{79.0}婆蹉氏去見\twnr{世尊}{12.0}。抵達後,與世尊一起互相問候。交換應該被互相問候的友好交談後,在一旁坐下。在一旁坐下的遊行者婆蹉氏對世尊說這個:

  「\twnr{喬達摩}{80.0}尊師!在較早的幾天前,當眾多種種外道\twnr{沙門}{29.0}、\twnr{婆羅門}{17.0}、遊行者們在論議堂集會共坐時,這個談論中出現:『這位富蘭那迦葉是有團體者,同時也是有群眾者、群眾的老師、眾所周知有名聲的開宗祖師、被眾人認定的善者,他\twnr{記說}{179.0}已死、已過世弟子的往生處:「那位已在那裡往生,[另]那位已在那裡往生。」凡他的最上人、最高人、已證得最高成就的弟子,他也記說那位已死、已過世弟子的往生處:「那位已在那裡往生,那位已在那裡往生。」這位末迦利瞿舍羅也……(中略)這位尼乾陀若提子也……(中略)這位散惹耶毘羅梨子也……(中略)這位浮陀迦旃延也……(中略)這位阿夷多翅舍欽婆羅也是有團體者,同時也是有群眾者、群眾的老師、眾所周知有名聲的開宗祖師、被眾人認定的善者,他記說已死、已過世弟子的往生處:「那位已在那裡往生,那位已在那裡往生。」凡他的最上人、最高人、已證得最高成就的弟子,他也記說那位已死、已過世弟子的往生處:「那位已在那裡往生,那位已在那裡往生。」

  這位沙門喬達摩也是有團體者,同時也是有群眾者、群眾的老師、眾所周知有名聲的開宗祖師、被眾人認定的善者,他記說已死、已過世弟子的往生處:「那位已在那裡往生,那位已在那裡往生。」凡他的最上人、最高人、已證得最高成就的弟子,他不記說那位已死、已過世弟子的往生處:「那位已在那裡往生,那位已在那裡往生。」然而這麼記說他:「他切斷渴愛,破壞結,\twnr{從慢的完全止滅}{592.0}\twnr{作苦的終結}{54.0}。」』喬達摩尊師!那個我就有疑惑,有懷疑:沙門喬達摩的法,應該怎樣被證知呢?」

  「婆蹉!你當然有疑惑,當然有懷疑:在疑惑之處上你的懷疑被生起。

  婆蹉!我\twnr{對有取著者}{x581}\twnr{安立}{143.0}往生處,非對無取著者。婆蹉!猶如有燃料的火燃燒,非無燃料的。同樣的,婆蹉!我對有取著者安立往生處,非對無取著者。」

  「喬達摩尊師!在凡火焰被風拋出,甚至走遠時,那麼,喬達摩\twnr{尊師}{203.0}在燃料上如何對這個安立呢?」

  「婆蹉!在凡火焰被風拋出,甚至走遠時,我安立那個風為燃料,婆蹉!因為,在那時風是燃料。」

  「喬達摩尊師!還有,在凡這個身體倒下,且眾生未被往生另一個身體時,那麼,喬達摩尊師在燃料上如何對這個安立呢?」

  「婆蹉!在凡這個身體倒下,且眾生未被往生另一個身體時,我說那個渴愛為燃料,婆蹉!因為,在那時渴愛是燃料。」



\sutta{10}{10}{阿難經}{https://agama.buddhason.org/SN/sn.php?keyword=44.10}
  那時,\twnr{遊行者}{79.0}婆蹉氏去見\twnr{世尊}{12.0}。抵達後,與世尊一起互相問候。交換應該被互相問候的友好交談後,在一旁坐下。在一旁坐下的遊行者婆蹉氏對世尊說這個:

  「\twnr{喬達摩}{80.0}尊師!怎麼樣,有我(真我)嗎?」

  在這麼說時,世尊保持沈默。

  「喬達摩尊師!那麼,無我嗎?」

  第二次,世尊也保持沈默。

  那時,遊行者婆蹉氏從座位起來後離開。

  那時,\twnr{尊者}{200.0}阿難在遊行者婆蹉氏離開不久,對世尊說這個:

  「\twnr{大德}{45.0}!當被問時,為何世尊不回答遊行者婆蹉氏的問題呢?」

  「阿難!遊行者婆蹉氏的『有我』,當被問時,如果我回答『有我』,阿難!凡那些恆常論的\twnr{沙門}{29.0}、\twnr{婆羅門}{17.0},這會成為與他們的一起;阿難!遊行者婆蹉氏的『無我』,當被問時,如果我回答『無我』,阿難!凡那些斷滅論的沙門、婆羅門,這會成為與他們的一起。阿難!遊行者婆蹉氏的『有我』,當被問時,如果我回答『有我』,阿難!是否我的那個[回答]為了智的生起:『\twnr{一切法是無我}{23.1}』成為隨順的呢?」

  「大德!這確實不是。」

  「阿難!遊行者婆蹉氏的『無我』,當被問時,如果我回答『無我』,阿難!對已迷惑的遊行者婆蹉氏會成為更迷亂:『我之前確實有我(之前確實有我的真我),它現在不存在了。』」



\sutta{11}{11}{沙比雅迦旃延經}{https://agama.buddhason.org/SN/sn.php?keyword=44.11}
  \twnr{有一次}{2.0},\twnr{尊者}{200.0}沙比雅迦旃延住在那低葛的磚屋中。

  那時,\twnr{遊行者}{79.0}婆蹉氏去見尊者沙比雅迦旃延。抵達後,與尊者沙比雅迦旃延一起互相問候。交換應該被互相問候的友好交談後,在一旁坐下。在一旁坐下的遊行者婆蹉氏對尊者沙比雅迦旃延說這個:

  「迦旃延尊師!怎麼樣,死後如來存在嗎?」

  「婆蹉!這不被世尊\twnr{記說}{179.0}:『死後如來存在。』」

  「迦旃延尊師!那麼,死後如來不存在嗎?」

  「婆蹉!這也不被世尊記說:『死後如來不存在。』」

  「迦旃延尊師!怎麼樣,\twnr{死後如來存在且不存在}{354.0}嗎?」

  「婆蹉!這不被世尊記說:『死後如來存在且不存在。』」

  「迦旃延尊師!那麼,死後如來既非存在也非不存在嗎?」

  「婆蹉!這也不被世尊記說:『死後如來既非存在也非不存在。』」

  「迦旃延尊師!當被像這樣問:『怎麼樣?迦旃延尊師!死後如來存在嗎?』你說:『婆蹉!這不被世尊記說:「死後如來存在。」』當被像這樣問:『迦旃延尊師!那麼,死後如來不存在嗎?』你說:『婆蹉!這也不被世尊記說:「死後如來不存在。」』當被像這樣問:『怎麼樣?迦旃延尊師!死後如來存在且不存在嗎?』你說:『婆蹉!這不被世尊記說:「死後如來存在且不存在。」』當被像這樣問:『迦旃延尊師!那麼,死後如來既非存在也非不存在嗎?』你說:『婆蹉!這也不被世尊記說:「死後如來既非存在也非不存在。」』迦旃延尊師!什麼因、什麼\twnr{緣}{180.0},以那個這不被\twnr{沙門}{29.0}\twnr{喬達摩}{80.0}記說?」

  「婆蹉!對\twnr{安立}{143.0}『有色者』或『無色者』,或『有想者』或『無想者』或『非想非非想者』的所有因與所有緣,如果那因與緣全部完全地、每一方面完全地、無殘餘地被滅,安立者如何安立他為『有色者』或『無色者』,或『有想者』或『無想者』或『非想非非想者』呢?」

  「迦旃延尊師!你是出家多久者呢?」

  「\twnr{道友}{201.0}!不久,三年。」

  「道友!凡雖是僅以這麼久,他就能有這麼多,更不用說超過這樣者了!」

  無記相應完成,其\twnr{攝頌}{35.0}:

  「讖摩長老尼、阿奴羅度,『舍利弗』、拘絺羅,

   目揵連與婆蹉,論議堂、阿難,

   沙比雅第十一。」

  \twnr{六處篇第四}{x582},其攝頌:

  「六處、受,婦女、閻浮車,

   沙門達葛、目揵連,質多、聚落主、{有}[無?]為,『無記』十種。」

  六處篇相應經典終了。





\page

\pian{大篇}{45}{56}
\xiangying{45}{道相應}
\pin{無明品}{1}{10}
\sutta{1}{1}{無明經}{https://agama.buddhason.org/SN/sn.php?keyword=45.1}
  \twnr{被我這麼聽聞}{1.0}:

  \twnr{有一次}{2.0},\twnr{世尊}{12.0}住在舍衛城祇樹林給孤獨園。

  在那裡,世尊召喚\twnr{比丘}{31.0}們:「比丘們!」

  「\twnr{尊師}{480.0}!」那些比丘回答世尊。

  世尊說這個:

  「比丘們!對不善法的\twnr{等至}{129.0}來說,\twnr{無明}{207.0}是先導,無慚、無愧隨後。

  比丘們!對\twnr{進入無明的}{645.0}無智者,邪見發生;對邪見來說,邪志發生;對邪志來說,邪語發生;對邪語來說,邪業發生;對邪業來說,邪命發生;對邪命來說,邪精進發生;對邪精進來說,邪念發生;對邪念來說,邪定發生[;對邪定來說,邪智發生;對邪智來說,邪解脫發生]。

  比丘們!對善法的等至來說,明是先導,慚、愧隨後。

  比丘們!對進入明的智者,正見發生;對正見來說,正志發生;對正志來說,正語發生;對正語來說,正業發生;對正業來說,正命發生;對正命來說,正精進發生;對正精進來說,正念發生;對正念來說,正定發生[;對正定來說,正智發生;對正智來說,正解脫發生-\ccchref{AN.10.105}{https://agama.buddhason.org/AN/an.php?keyword=10.105}]。」



\sutta{2}{2}{一半經}{https://agama.buddhason.org/SN/sn.php?keyword=45.2}
  \twnr{被我這麼聽聞}{1.0}:

  \twnr{有一次}{2.0},\twnr{世尊}{12.0}住在釋迦國,名叫那額勒葛的釋迦族人市鎮。

  那時,\twnr{尊者}{200.0}阿難去見\twnr{世尊}{12.0}。抵達後,向世尊\twnr{問訊}{46.0}後,在一旁坐下。在一旁坐下的尊者阿難對世尊說這個:

  「\twnr{大德}{45.0}!\twnr{這是梵行的一半}{747.0},即:\twnr{善的朋友之誼}{321.0}、善的同伴之誼、善的親密朋友之誼。」

  「阿難!不要這樣[說-\ccchref{DN.15}{https://agama.buddhason.org/DN/dm.php?keyword=15}],阿難!不要這樣[說],阿難!這就是梵行的全部,即:善的朋友之誼、善的同伴之誼、善的親密朋友之誼。阿難!對有善的朋友、善的同伴、善的親密朋友比丘的這個能被預期:他必將\twnr{修習}{94.0}\twnr{八支聖道}{525.0}、必將\twnr{多作}{95.0}八支聖道。

  阿難!而有善的朋友、善的同伴、善的親密朋友的比丘,如何修習八支聖道、多作八支聖道?阿難!這裡,比丘\twnr{依止遠離}{322.0}、依止離貪、依止滅、\twnr{捨棄的成熟}{221.0}修習正見;依止遠離……(中略)修習正志……(中略)修習正語……(中略)修習正業……(中略)修習正命……(中略)修習正精進……(中略)修習正念;依止遠離、依止離貪、依止滅、捨棄的成熟修習正定。阿難!有善的朋友、善的同伴、善的親密朋友的比丘這樣修習八支聖道、多作八支聖道。

  阿難!其次,以這個法門,這也能被知道:關於這就是梵行的全部,即:善的朋友之誼、善的同伴之誼、善的親密朋友之誼。阿難!由於善友的我,\twnr{生法}{587.0}的眾生從生被釋放;老法的眾生從老被釋放;死法的眾生從死被釋放;愁悲苦憂\twnr{絕望}{342.0}法的眾生從愁悲苦憂絕望被釋放。阿難!以這個法門,這也能被知道:關於這就是梵行的全部,即:善的朋友之誼、善的同伴之誼、善的親密朋友之誼。」[\suttaref{SN.3.18}]



\sutta{3}{3}{舍利弗經}{https://agama.buddhason.org/SN/sn.php?keyword=45.3}
  起源於舍衛城。

  那時,\twnr{尊者}{200.0}舍利弗去見\twnr{世尊}{12.0}。抵達後,向世尊\twnr{問訊}{46.0}後,在一旁坐下。在一旁坐下的尊者舍利弗對世尊說這個:

  「\twnr{大德}{45.0}!\twnr{這是梵行的全部}{747.1},即:善的朋友之誼、善的同伴之誼、善的親密朋友之誼。」

  「\twnr{好}{44.0}!好!舍利弗!這就是梵行的全部,即:\twnr{善的朋友之誼}{321.0}、善的同伴之誼、善的親密朋友之誼。舍利弗!對有善的朋友、善的同伴、善的親密朋友比丘的這個能被預期:他必將\twnr{修習}{94.0}\twnr{八支聖道}{525.0}、必將\twnr{多作}{95.0}八支聖道。

  舍利弗!而有善的朋友、善的同伴、善的親密朋友的比丘,如何修習八支聖道、多作八支聖道?舍利弗!這裡,比丘\twnr{依止遠離}{322.0}、依止離貪、依止滅、\twnr{捨棄的成熟}{221.0}修習正見……(中略)依止遠離、依止離貪、依止滅、捨棄的成熟修習正定。舍利弗!有善的朋友、善的同伴、善的親密朋友的比丘這樣修習八支聖道、多作八支聖道。

  舍利弗!其次,以這個法門,這也能被知道:關於這就是梵行的全部,即:善的朋友之誼、善的同伴之誼、善的親密朋友之誼。舍利弗!由於善友的我,\twnr{生法}{587.0}的眾生從生被釋放;老法的眾生從老被釋放;死法的眾生從死被釋放;愁悲苦憂\twnr{絕望}{342.0}法的眾生從愁悲苦憂絕望被釋放。舍利弗!以這個法門,這也能被知道:關於這就是梵行的全部,即:善的朋友之誼、善的同伴之誼、善的親密朋友之誼。」



\sutta{4}{4}{若奴索尼婆羅門經}{https://agama.buddhason.org/SN/sn.php?keyword=45.4}
  起源於舍衛城。

  那時,\twnr{尊者}{200.0}阿難午前時穿衣、拿起衣鉢後,\twnr{為了托鉢}{87.0}進入舍衛城。

  尊者阿難看見以全白的馬車從舍衛城出發的\twnr{若奴索尼}{980.0}婆羅門:有套上軛的諸白馬、諸白帆、白車、白附屬物、諸白繩、白鞭棒、白傘蓋、白纏頭巾、諸白衣、白鞋,被白馬尾扇扇風,人們看見後這麼說:

  「\twnr{先生}{202.0}!實在是\twnr{梵乘}{x583}!先生!實在是梵乘的樣子。」

  那時,尊者阿難在舍衛城為了托鉢行走後,\twnr{餐後已從施食返回}{512.0},去見\twnr{世尊}{12.0}。抵達後,向世尊\twnr{問訊}{46.0}後,在一旁坐下。在一旁坐下的尊者阿難對世尊說這個:

  「\twnr{大德}{45.0}!這裡,我午前時穿衣、拿起衣鉢後,為了托鉢進入舍衛城。我看見以全白的馬車從舍衛城出發的若奴索尼婆羅門:有套上軛的諸白馬、諸白帆、白車、白附屬物、諸白繩、白鞭棒、白傘蓋、白纏頭巾、諸白衣、白鞋,被白馬尾扇扇風,人們看見後這麼說:『先生!實在是梵乘!先生!實在是梵乘的樣子。』大德!在這法、律中能夠\twnr{安立}{143.0}梵乘嗎?」

  世尊說:「阿難!能夠!阿難!這就是\twnr{八支聖道}{525.0}的同義語:『梵乘』,又『\twnr{法乘}{x584}』,又『戰鬥中無上的勝利』。

  阿難!正見已\twnr{修習}{94.0}、已\twnr{多作}{95.0},有貪之調伏的完結、有瞋之調伏的完結、有癡之調伏的完結;阿難!正志已修習、已多作,有貪之調伏的完結、有瞋之調伏的完結、有癡之調伏的完結;阿難!正語已修習、已多作,有貪之調伏的完結、有瞋之調伏的完結、有癡之調伏的完結;阿難!正業已修習、已多作,有貪之調伏的完結、有瞋之調伏的完結、有癡之調伏的完結;阿難!正命已修習、已多作,有貪之調伏的完結、有瞋之調伏的完結、有癡之調伏的完結;阿難!正精進已修習、已多作,有貪之調伏的完結、有瞋之調伏的完結、有癡之調伏的完結;阿難!正念已修習、已多作,有貪之調伏的完結、有瞋之調伏的完結、有癡之調伏的完結;阿難!正定已修習、已多作,有貪之調伏的完結、有瞋之調伏的完結、有癡之調伏的完結。

  阿難!以此法門,這能被知道:這是八支聖道的同義語:『梵乘』,又『法乘』,又『戰鬥中無上的勝利』。」

  世尊說這個,說這個後,\twnr{善逝}{8.0}、\twnr{大師}{145.0}更進一步說這個:

  「凡信與慧,\twnr{成對的法}{x585}常為責任(為先頭),

   慚為轅桿、意為\twnr{繫繩}{x586},念為守護的駕駛者。

   車有戒為資助,禪定為車軸、活力為車輪,

   \twnr{平靜為軛定}{x587},無欲為帷幕。

   無惡意、不加害,遠離為其武器,

   忍耐為\twnr{皮甲冑}{x588},為了\twnr{軛安穩}{192.0}轉起。

   這個在自己上生成的,為無上的梵乘,

   明智者們從世間被引導,\twnr{必定一一成為勝利者}{x589}。」



\sutta{5}{5}{為了什麼目的經}{https://agama.buddhason.org/SN/sn.php?keyword=45.5}
  起源於舍衛城。

  那時,眾多\twnr{比丘}{31.0}去見\twnr{世尊}{12.0}……(中略)在一旁坐下。在一旁坐下的那些比丘對世尊說這個:

  「\twnr{大德}{45.0}!這裡,其他外道\twnr{遊行者}{79.0}們這麼問我們:『\twnr{道友}{201.0}們!為了什麼目的在\twnr{沙門}{29.0}\twnr{喬達摩}{80.0}處梵行被住?』

  大德!被這麼問,我們這麼回答那些其他外道遊行者:『道友們!為了苦的\twnr{遍知}{154.0}在世尊處梵行被住。』

  大德!被這麼問,當這麼回答時,是否我們就\twnr{是世尊的所說之說者}{115.0},而且不會以不實的誹謗世尊,以及會\twnr{法隨法地回答}{415.0},而任何如法的種種說不會來到應該被呵責處?」

  「比丘們!被這麼問,當你們這麼回答時,確實就是我的所說之說者,同時也不以不實的誹謗我,以及法隨法地回答,而任何如法的種種說不會來到應該被呵責處,因為是為了苦的遍知在我處行梵行。

  比丘們!如果其他外道遊行者們這麼問你們:『道友們!那麼,為了這個苦的遍知,\twnr{有道、有道跡}{359.0}嗎?』比丘們!被這麼問,你們應該這麼回答那些其他外道遊行者:『道友們!為了這個苦的遍知,有道、有道跡。』

  比丘們!而為了這個苦的遍知,什麼是道?什麼是道跡?就是這\twnr{八支聖道}{525.0},即:正見……(中略)正定。比丘們!為了這個苦的遍知,這是道,這是道跡。比丘們!被這麼問,你們應該這麼回答那些其他外道遊行者。」[≃\suttaref{SN.38.4}]



\sutta{6}{6}{某位比丘經第一}{https://agama.buddhason.org/SN/sn.php?keyword=45.6}
  起源於舍衛城。

  那時,某位比丘去見\twnr{世尊}{12.0}……(中略)在一旁坐下的那位比丘對世尊說這個:

  「\twnr{大德}{45.0}!被稱為『\twnr{梵行}{381.0}、梵行』,大德!什麼是梵行,什麼是梵行的完結呢?」

  「比丘!這\twnr{八支聖道}{525.0}是梵行,即:正見……(中略)正定。比丘!凡貪的滅盡、瞋的滅盡、癡的滅盡,這是梵行的完結。」



\sutta{7}{7}{某位比丘經第二}{https://agama.buddhason.org/SN/sn.php?keyword=45.7}
  起源於舍衛城。

  那時,某位比丘去見\twnr{世尊}{12.0}……(中略)在一旁坐下的那位比丘對世尊說這個:

  「\twnr{大德}{45.0}!被稱為『貪的調伏、瞋的調伏、癡的調伏』,大德!『貪的調伏、瞋的調伏、癡的調伏』,這是什麼的同義語呢?」

  「比丘!『貪的調伏、瞋的調伏、癡的調伏』,這是涅槃界的同義語,以那個被稱為諸漏的滅盡。」

  在這麼說時,那位比丘對世尊說這個:

  「大德!被稱為『\twnr{不死}{123.0}、不死』,大德!什麼是不死?什麼是導向不死之道呢?」

  「比丘!凡貪的滅盡、瞋的滅盡、癡的滅盡者,這被稱為不死,這\twnr{八支聖道}{525.0}就是導向不死之道,即:正見……(中略)正定。」



\sutta{8}{8}{解析經}{https://agama.buddhason.org/SN/sn.php?keyword=45.8}
  起源於舍衛城。

  「\twnr{比丘}{31.0}們!我將為你們教導、解析\twnr{八支聖道}{525.0},你們要聽它!你們要\twnr{好好作意}{43.1}!我將說。」

  「是的,\twnr{大德}{45.0}!」那些比丘回答\twnr{世尊}{12.0}。

  世尊說這個:

  「比丘們!而什麼是八支聖道呢?即:正見……(中略)正定。

  比丘們!而什麼是正見?比丘們!凡在苦上之智,在苦集上之智,在苦滅上之智,在導向苦\twnr{滅道跡}{69.0}上之智,比丘們!這被稱為正見。

  比丘們!而什麼是正志?比丘們!凡\twnr{離欲的意向}{514.0}、\twnr{無惡意的意向}{515.0}、\twnr{無加害的意向}{516.0},比丘們!這被稱為正志。

  比丘們!而什麼是正語?比丘們!凡\twnr{妄語}{106.0}的戒絕、\twnr{離間語}{234.0}的戒絕、\twnr{粗惡語}{235.0}的戒絕、\twnr{雜穢語}{236.0}的戒絕,比丘們!這被稱為正語。

  比丘們!而什麼是正業?比丘們!凡殺生的戒絕、\twnr{未給予而取}{104.0}的戒絕、非\twnr{梵行}{381.0}的戒絕,比丘們!這被稱為正業。

  比丘們!而什麼是正命?比丘們!這裡,\twnr{聖弟子}{24.0}捨斷邪命後,以正命營生,比丘們!這被稱為正命。

  比丘們!而什麼是正精進?比丘們!這裡,比丘為了未生起的惡不善法之不生起使意欲生起、努力、發動活力、盡心、勤奮;為了已生起的惡不善法之捨斷使意欲生起……(中略)為了未生起的善法之生起使意欲生起……(中略)為了已生起的諸善法之存續、不忘失、增大、成滿、修習圓滿使意欲生起、努力、發動活力、盡心、勤奮,比丘們!這被稱為正精進。

  比丘們!而什麼是正念?比丘們!這裡,比丘住於\twnr{在身上隨看著身}{176.0}:熱心的、正知的、有念的,調伏世間中的\twnr{貪婪}{435.0}、憂後;住於在諸受上隨看著受:熱心的、正知的、有念的,調伏世間中的貪婪、憂後;住於在心上隨看著心:熱心的、正知的、有念的,調伏世間中的貪婪、憂後;住於在諸法上隨看著法:熱心的、正知的、有念的,調伏世間中的貪婪、憂後,比丘們!這被稱為正念。

  比丘們!而什麼是正定?比丘們!這裡,比丘就從離諸欲後,從離諸不善法後,\twnr{進入後住於}{66.0}有尋、\twnr{有伺}{175.0},\twnr{離而生喜、樂}{174.0}的初禪;從尋與伺的平息,\twnr{自身內的明淨}{256.0},\twnr{心的專一性}{255.0},進入後住於無尋、無伺,定而生喜、樂的第二禪;從喜的\twnr{褪去}{77.0}、住於\twnr{平靜}{228.0}、有念正知、以身體感受樂,進入後住於這聖弟子宣說:『他是平靜、具念、\twnr{住於樂者}{317.0}』的第三禪;從樂的捨斷與從苦的捨斷,就在之前諸喜悅、憂的滅沒,進入後住於不苦不樂,\twnr{平靜、念遍純淨}{494.0}的第四禪,比丘們!這被稱為正定。」



\sutta{9}{9}{穗經}{https://agama.buddhason.org/SN/sn.php?keyword=45.9}
  起源於舍衛城。

  「\twnr{比丘}{31.0}們!猶如稻穗尖或麥穗尖被手或腳錯誤朝向地壓踏,『他將破裂手或腳,或將使血生起。』\twnr{這不存在可能性}{650.0},那是什麼原因?比丘們!以穗尖的錯誤朝向狀態。同樣的,比丘們!那位比丘確實以錯誤朝向的見、以錯誤朝向的道之\twnr{修習}{94.0},『他將破壞\twnr{無明}{207.0},將使明生起,將作證涅槃。』這不存在可能性,那是什麼原因?比丘們!以見的錯誤朝向狀態。

  比丘們!猶如稻穗尖或麥穗尖被手或腳正確朝向地壓踏,『他將破裂手或腳,或將使血生起。』這存在可能性,那是什麼原因?比丘們!以穗尖的正確朝向狀態。同樣的,比丘們!那位比丘確實以正確朝向的見、以正確朝向的道之修習,『他將破壞無明,將使明生起,將作證涅槃。』這存在可能性,那是什麼原因?比丘們!以見的正確朝向狀態。

  比丘們!而怎樣比丘以正確朝向的見、以正確朝向的道之修習破壞無明,使明生起,作證涅槃?比丘們!這裡,比丘\twnr{依止遠離}{322.0}、依止離貪、依止滅、\twnr{捨棄的成熟}{221.0}修習正見……(中略)依止遠離、依止離貪、依止滅、捨棄的成熟修習正定。比丘們!這樣,比丘以正確朝向的見、以正確朝向的道之修習破壞無明,使明生起,作證涅槃。[\suttaref{SN.45.154}]」



\sutta{10}{10}{難提經}{https://agama.buddhason.org/SN/sn.php?keyword=45.10}
  起源於舍衛城。

  那時,\twnr{遊行者}{79.0}難提去見\twnr{世尊}{12.0}。抵達後,與世尊一起互相問候。交換應該被互相問候的友好交談後,在一旁坐下。在一旁坐下的遊行者難提對世尊說這個:

  「\twnr{喬達摩}{80.0}尊師!幾法已\twnr{修習}{94.0}、已\twnr{多作}{95.0},導向涅槃,有涅槃為彼岸、涅槃為完結呢?」

  「難提!有這八法,已修習、已多作,導向涅槃,有涅槃為彼岸、涅槃為完結,哪八個?即:正見……(中略)正定。難提!這些是八法,已修習、已多作,導向涅槃,有涅槃為彼岸、涅槃為完結。」

  在這麼說時,遊行者難提對世尊說這個:

  「太偉大了,喬達摩尊師!太偉大了,喬達摩尊師!……(中略)請喬達摩\twnr{尊師}{203.0}記得我為\twnr{優婆塞}{98.0},從今天起\twnr{已終生歸依}{64.0}。」

  無明品第一,其\twnr{攝頌}{35.0}:

  「無明與一半,舍利弗與婆羅門,

   為了什麼目的與二則\twnr{比丘}{31.0},解析、穗、難提。」





\pin{住處品}{11}{20}
\sutta{11}{11}{住處經第一}{https://agama.buddhason.org/SN/sn.php?keyword=45.11}
  起源於舍衛城。

  「\twnr{比丘}{31.0}們!我想要\twnr{獨坐}{92.0}半個月,我不應該被任何人來見,除了以一位送\twnr{施食}{196.0}者外。」

  「是的,\twnr{大德}{45.0}!」

  那些比丘回答\twnr{世尊}{12.0}後,在那裡,確實沒任何人去見世尊,除了以一位送施食者外。

  那時,經過那半個月,從獨坐出來的世尊召喚比丘們:

  「比丘們!凡那個我以初\twnr{現正覺}{75.0}的住處住,我以那個的一部分住。那個我這麼知道:『有以邪見\twnr{為緣}{180.0}感受的;也有以正見為緣感受的……(中略)也有以邪定為緣感受的;也有以正定為緣感受的;也有以意欲為緣感受的;也有以尋為緣感受的;也有以想為緣感受的,意欲未被平息、尋未被平息、想未被平息,也有以那個為緣感受的;意欲被平息、尋被平息、想被平息,也有以那個為緣感受的;有未到達者為了到達的\twnr{精進}{x590},也在那個處之到達時,也有以那個為緣感受的。」



\sutta{12}{12}{住處經第二}{https://agama.buddhason.org/SN/sn.php?keyword=45.12}
  起源於舍衛城。

  「\twnr{比丘}{31.0}們!我想要\twnr{獨坐}{92.0}半個月,我不應該被任何人來見,除了以一位送\twnr{施食}{196.0}者外。」

  「是的,\twnr{大德}{45.0}!」

  那些比丘回答\twnr{世尊}{12.0}後,在那裡,確實沒任何人去見世尊,除了以一位送施食者外。

  那時,經過那半個月,從獨坐出來的世尊召喚比丘們:

  「比丘們!凡那個我以初\twnr{現正覺}{75.0}的住處住,我以那個的一部分住。那個我這麼知道:『有以邪見\twnr{為緣}{180.0}感受的,也有以邪見的平息為緣感受的;有以正見為緣感受的,也有以正見的平息為緣感受的……(中略)有以邪定為緣感受的,也有以邪定的平息為緣感受的;有以正定為緣感受的,也有以正定的平息為緣感受的;有以意欲為緣感受的,\twnr{也有以意欲的平息為緣感受的}{x591};有以尋為緣感受的,也有以尋的平息為緣感受的;有以想為緣感受的,也有以想的平息為緣感受的,意欲未被平息、尋未被平息、想未被平息,有以那個為緣感受的;意欲被平息、尋被平息、想被平息,也有以那個為緣感受的;有未到達者為了到達的\twnr{精進}{x592},也在那個處之到達時,也有以那個為緣感受的。」



\sutta{13}{13}{有學經}{https://agama.buddhason.org/SN/sn.php?keyword=45.13}
  起源於舍衛城。

  那時,\twnr{某位比丘}{39.0}去見\twnr{世尊}{12.0}。……(中略)在一旁坐下的那位比丘對世尊說這個:

  「\twnr{大德}{45.0}!被稱為『\twnr{有學}{193.0}、有學』,大德!什麼情形是有學呢?」

  「比丘!這裡,他是有學正見的具備者……(中略)他是有學正定的具備者,這個情形是有學。」



\sutta{14}{14}{生起經第一}{https://agama.buddhason.org/SN/sn.php?keyword=45.14}
  起源於舍衛城。

  「\twnr{比丘}{31.0}們!有這些八法,已\twnr{修習}{94.0}、已\twnr{多作}{95.0},未生起的不離\twnr{如來}{4.0}、\twnr{阿羅漢}{5.0}、\twnr{遍正覺者}{6.0}的出現生起,哪八個?即:正見……(中略)正定。比丘們!這些是八法,已修習、已多作,未生起的不離如來、阿羅漢、遍正覺者的出現生起。」



\sutta{15}{15}{生起經第二}{https://agama.buddhason.org/SN/sn.php?keyword=45.15}
  起源於舍衛城。

  「\twnr{比丘}{31.0}們!有這些八法,已\twnr{修習}{94.0}、已\twnr{多作}{95.0},未生起的不離\twnr{善逝}{8.0}之律生起,哪八個?即:正見……(中略)正定。比丘們!這些是八法,已修習、已多作,未生起的不離善逝之律生起。」



\sutta{16}{16}{遍純淨經第一}{https://agama.buddhason.org/SN/sn.php?keyword=45.16}
  起源於舍衛城。

  「比丘們!有這遍純淨的、皎潔的、無穢的、離\twnr{隨雜染}{288.0}的八法,未生起的不離\twnr{如來}{4.0}、\twnr{阿羅漢}{5.0}、\twnr{遍正覺者}{6.0}的出現生起,哪八個?即:正見……(中略)正定。比丘們!這些是遍純淨的、皎潔的、無穢的、離隨雜染的八法,未生起的不離如來、阿羅漢、遍正覺者的出現生起。」





\sutta{17}{17}{遍純淨經第二}{https://agama.buddhason.org/SN/sn.php?keyword=45.17}
  起源於舍衛城。

  「\twnr{比丘}{31.0}們!有這遍純淨的、皎潔的、無穢的、離\twnr{隨雜染}{288.0}的八法,未生起的不離\twnr{善逝}{8.0}之律生起,哪八個?即:正見……(中略)正定。比丘們!這些是遍純淨的、皎潔的、無穢的、離隨雜染的八法,未生起的不離善逝之律生起。」



\sutta{18}{18}{雞園經第一}{https://agama.buddhason.org/SN/sn.php?keyword=45.18}
  \twnr{被我這麼聽聞}{1.0}:

  \twnr{有一次}{2.0},\twnr{尊者}{200.0}阿難與尊者跋陀住在巴連弗城雞園。

  那時,尊者跋陀傍晚時,從\twnr{獨坐}{92.0}出來,去見尊者阿難。抵達後,與尊者阿難一起互相問候。交換應該被互相問候的友好交談後,在一旁坐下。在一旁坐下的尊者跋陀對尊者阿難說這個:

  「阿難\twnr{學友}{201.0}!被稱為『非\twnr{梵行}{381.0}、非梵行』,學友!什麼是非梵行呢?」

  「跋陀學友!\twnr{好}{44.0}!好!跋陀學友!你的\twnr{想法}{563.1}是善的、辯才是善的、詢問是好的,跋陀學友!因為你這麼問:『阿難學友!被稱為「非梵行、非梵行」,學友!什麼是非梵行呢?』」

  「是的,學友!」

  「學友!這八支邪道就是非梵行,即:邪見……(中略)邪定。」



\sutta{19}{19}{雞園經第二}{https://agama.buddhason.org/SN/sn.php?keyword=45.19}
  起源於巴連弗城。

  「阿難\twnr{學友}{201.0}!被稱為『\twnr{梵行}{381.0}、梵行』,學友!什麼是梵行,什麼是梵行的完結呢?」

  「跋陀學友!\twnr{好}{44.0}!好!跋陀學友!你的\twnr{想法}{563.1}是善的、辯才是善的、詢問是好的,跋陀學友!因為你這麼問:『阿難學友!被稱為「梵行、梵行」,學友!什麼是梵行,什麼是梵行的完結呢?』」

  「是的,學友!」

  「學友!這\twnr{八支聖道}{525.0}就是梵行,即:正見……(中略)正定。學友!凡貪的滅盡、瞋的滅盡、癡的滅盡,這是梵行的完結。」



\sutta{20}{20}{雞園經第三}{https://agama.buddhason.org/SN/sn.php?keyword=45.20}
  起源於巴連弗城。

  「阿難\twnr{學友}{201.0}!被稱為『\twnr{梵行}{381.0}、梵行』,學友!什麼是梵行,什麼是梵行者,什麼是梵行的完結呢?」

  「跋陀學友!\twnr{好}{44.0}!好!跋陀學友!你的\twnr{想法}{563.1}是善的、辯才是善的、詢問是好的,跋陀學友!因為你這麼問:『阿難學友!被稱為「梵行、梵行」,學友!什麼是梵行,什麼是梵行者,什麼是梵行的完結呢?』」

  「是的,學友!」

  「學友!這\twnr{八支聖道}{525.0}就是梵行,即:正見……(中略)正定。學友!凡具備這八支聖道者,這被稱為梵行者,學友!凡貪的滅盡、瞋的滅盡、癡的滅盡,這是梵行的完結。」

  三經同一起源(因緣)。

  住處品第二,其\twnr{攝頌}{35.0}:

  「二則住處與有學,生起二則在後,

   以遍純淨二說,以雞園三則。」





\pin{邪性品}{21}{30}
\sutta{21}{21}{邪性經}{https://agama.buddhason.org/SN/sn.php?keyword=45.21}
  起源於舍衛城。

  「\twnr{比丘}{31.0}們!我將為你們教導邪性與正性,\twnr{你們要聽}{43.0}它!

  比丘們!而什麼是邪性呢?即:邪見……(中略)邪定,比丘們!這被稱為邪性。比丘們!而什麼是正性呢?即:正見……(中略)正定,比丘們!這被稱為正性。」



\sutta{22}{22}{不善法經}{https://agama.buddhason.org/SN/sn.php?keyword=45.22}
  起源於舍衛城。

  「\twnr{比丘}{31.0}們!我將為你們教導不善法與善法,\twnr{你們要聽}{43.0}它!

  比丘們!而什麼是不善法呢?即:邪見……(中略)邪定,比丘們!這被稱為不善法。比丘們!而什麼是善法呢?即:正見……(中略)正定,比丘們!這被稱為善法。」



\sutta{23}{23}{道跡經第一}{https://agama.buddhason.org/SN/sn.php?keyword=45.23}
  起源於舍衛城。

  「\twnr{比丘}{31.0}們!我將為你們教導邪道跡與正道跡,\twnr{你們要聽}{43.0}它!

  比丘們!而什麼是邪道跡呢?即:邪見……(中略)邪定,比丘們!這被稱為邪道跡。比丘們!而什麼是正道跡呢?即:正見……(中略)正定,比丘們!這被稱為正道跡。」



\sutta{24}{24}{道跡經第二}{https://agama.buddhason.org/SN/sn.php?keyword=45.24}
  起源於舍衛城。

  「\twnr{比丘}{31.0}們!我不稱讚在家人或出家人的邪道跡,比丘們!在家或出家的邪行者,不因為邪行之因成為\twnr{真理、善法}{691.0}的成功者。

  比丘們!而什麼是邪道跡呢?即:邪見……(中略)邪定,比丘們!這被稱為邪道跡。比丘們!我不稱讚在家人或出家人的邪道跡,比丘們!在家或出家的邪行者,不因為邪行之因成為真理、善法的成功者。

  比丘們!我稱讚在家人或出家人的正道跡,比丘們!在家或出家的正行者,因為正行之因成為真理、善法的成功者。

  比丘們!而什麼是正道跡呢?即:正見……(中略)正定,比丘們!這被稱為正道跡。比丘們!我稱讚在家人或出家人的正道跡,比丘們!在家或出家的正行者,因為正行之因成為真理、善法的成功者。」



\sutta{25}{25}{非善人經第一}{https://agama.buddhason.org/SN/sn.php?keyword=45.25}
  起源於舍衛城。

  「\twnr{比丘}{31.0}們!我將為你們教導非善人與善人,\twnr{你們要聽}{43.0}它!

  比丘們!而什麼是非善人?比丘們!這裡,某人是邪見者、邪志者、邪語者、邪業者、邪命者、邪精進者、邪念者、邪定者,比丘們!這被稱為非善人。

  比丘們!而什麼是善人?比丘們!這裡,某人是正見者、正志者、正語者、正業者、正命者、正精進者、正念者、正定者,比丘們!這被稱為善人。」



\sutta{26}{26}{非善人經第二}{https://agama.buddhason.org/SN/sn.php?keyword=45.26}
  起源於舍衛城。

  「\twnr{比丘}{31.0}們!我將為你們教導非善人與比非善人更非善人的;善人與比善人更善人的,\twnr{你們要聽}{30.0}它!

  比丘們!而什麼是非善人?比丘們!這裡,某人是邪見者……(中略)邪定者,比丘們!這被稱為非善人。

  比丘們!而什麼是比非善人更非善人的?比丘們!這裡,某人是邪見者……(中略)邪定者、\twnr{邪智}{325.0}者、\twnr{邪解脫}{326.0}者,比丘們!這被稱為比非善人更非善人的。

  比丘們!而什麼是善人?比丘們!這裡,某人是正見者……(中略)正定者,比丘們!這被稱為善人。

  比丘們!而什麼是比善人更善人的?比丘們!這裡,某人是正見者……(中略)正定者、\twnr{正智}{976.0}者、正解脫者,比丘們!這被稱為比善人更善人的。」



\sutta{27}{27}{瓶子經}{https://agama.buddhason.org/SN/sn.php?keyword=45.27}
  起源於舍衛城。

  「\twnr{比丘}{31.0}們!猶如無支撐的瓶子是易倒的;有支撐的是不易倒的。同樣的,比丘們!無支撐的心是易倒的(易被反轉的);有支撐的是不易倒的。比丘們!而什麼是心的支撐呢?就是這\twnr{八支聖道}{525.0},即:正見……(中略)正定。比丘們!猶如無支撐的瓶子是易倒的;有支撐的是不易倒的。同樣的,比丘們!無支撐的心是易倒的;有支撐的是不易倒的。」



\sutta{28}{28}{定經}{https://agama.buddhason.org/SN/sn.php?keyword=45.28}
  起源於舍衛城。

  「\twnr{比丘}{31.0}們!我將為你們教導\twnr{有近因}{614.1}、\twnr{有資糧}{693.0}的\twnr{聖正定}{692.0},\twnr{你們要聽}{43.0}它!

  比丘們!而什麼是有近因、有資糧的聖正定呢?即:正見……(中略)正念,比丘們!凡\twnr{心一境性}{255.0}以這七支為有資糧狀態,比丘們!這被稱為『有近因』及『有資糧』的聖正定。」[≃\ccchref{AN.7.45}{https://agama.buddhason.org/AN/an.php?keyword=7.45}]



\sutta{29}{29}{受經}{https://agama.buddhason.org/SN/sn.php?keyword=45.29}
  起源於舍衛城。

  「\twnr{比丘}{31.0}們!有這些三受,哪三個?樂受、苦受、不苦不樂受,比丘們!這些是三受。

  比丘們!為了這些三受的\twnr{遍知}{154.0},\twnr{八支聖道}{525.0}應該被\twnr{修習}{94.0}。

  什麼是八支聖道呢?即:正見……(中略)正定。

  比丘們!為了這些三受的遍知,八支聖道應該被修習。」



\sutta{30}{30}{鬱低雅經}{https://agama.buddhason.org/SN/sn.php?keyword=45.30}
  起源於舍衛城。

  那時,\twnr{尊者}{200.0}鬱低雅去見世尊……(中略)在一旁坐下的尊者鬱低雅對\twnr{世尊}{12.0}說這個:

  「\twnr{大德}{45.0}!這裡,當我獨處、\twnr{獨坐}{92.0}時,這樣心的深思生起:『\twnr{五種欲}{187.0}被世尊說,哪些是被世尊說的五種欲呢?』」

  「鬱低雅!\twnr{好}{44.0}!好!鬱低雅!這五種欲被我說,哪五個?能被眼識知的、想要的、所愛的、合意的、可愛形色的、伴隨欲的、誘人的諸色,能被耳識知的……(中略)諸聲音,能被鼻識知的……(中略)諸氣味,能被舌識知的……(中略)諸味道,能被身識知的、想要的、所愛的、合意的、可愛形色的、伴隨欲的、誘人的諸\twnr{所觸}{220.2},鬱低雅!這些是被我說的五種欲。

  鬱低雅!為了這五種欲的捨斷,\twnr{八支聖道}{525.0}應該被\twnr{修習}{94.0}。什麼是八支聖道呢?即:正見……(中略)正定,鬱低雅!為了這五種欲的捨斷,八支聖道應該被修習。」

  邪性品第三,其\twnr{攝頌}{35.0}:

  「邪性、不善法,進一步道跡二則,

   非善人二則、瓶子,與定、受、鬱低雅。」





\pin{行品}{31}{40}
\sutta{31}{31}{行經第一}{https://agama.buddhason.org/SN/sn.php?keyword=45.31}
  起源於舍衛城。

  「\twnr{比丘}{31.0}們!我將為你們教導邪行與正行,\twnr{你們要聽}{43.0}它!

  比丘們!而什麼是邪行呢?即:邪見……(中略)邪定,比丘們!這被稱為邪行。比丘們!而什麼是正行呢?即:正見……(中略)正定,比丘們!這被稱為正行。」



\sutta{32}{32}{行經第二}{https://agama.buddhason.org/SN/sn.php?keyword=45.32}
  起源於舍衛城。

  「\twnr{比丘}{31.0}們!我將為你們教導邪行者與正行者,\twnr{你們要聽}{43.0}它!

  比丘們!而什麼是邪行者?比丘們!這裡,某人是邪見者……(中略)邪定者,比丘們!這被稱為邪行者。比丘們!而什麼是正行者?比丘們!這裡,某人是正見者……(中略)正定者,比丘們!這被稱為正行者。」



\sutta{33}{33}{已錯失經}{https://agama.buddhason.org/SN/sn.php?keyword=45.33}
  起源於舍衛城。

  「\twnr{比丘}{31.0}們!凡任何已錯失\twnr{八支聖道}{525.0}者,他們導向苦的完全滅盡的{\twnr{八支}{x593}}聖道已錯失;比丘們!凡任何已發動八支聖道者,他們導向苦的完全滅盡的{八支}聖道已發動。比丘們!而什麼是八支聖道呢?即:正見……(中略)正定。比丘們!凡任何已錯失八支聖道者,他們導向苦的完全滅盡的{八支}聖道已錯失;比丘們!凡任何已發動八支聖道者,他們導向苦的完全滅盡的{八支}聖道已發動。」



\sutta{34}{34}{到彼岸經}{https://agama.buddhason.org/SN/sn.php?keyword=45.34}
  起源於舍衛城。

  「\twnr{比丘}{31.0}們!有這些八法,已\twnr{修習}{94.0}、已\twnr{多作}{95.0},轉起從此岸走到\twnr{彼岸}{226.0},哪八個?即:正見……(中略)正定,比丘們!這些是八法,已修習、已多作,轉起從此岸走到彼岸。」

  世尊說這個,說這個後,\twnr{善逝}{8.0}、\twnr{大師}{145.0}又更進一步說這個:

  「在那些人中是少的:\twnr{到彼岸的人}{x594},

   其他人還,只跟隨這個岸邊跑。

   而凡在法被正確告知時,在法上順從法者,

   那些人將去彼岸:極難越過的死神界。

   賢智者捨棄黑法後,請你們修習白的,

   從家來到無家後,在難喜樂之處遠離。

   在那裡應該尋求喜樂:無所有者捨棄諸欲後,

   賢智者應該淨化自己:從心的污染。

   凡在諸正覺支上,他們的心已正確地修習者,

   在執取的\twnr{斷念}{211.0}上,凡無取著後已樂者,

   諸漏已滅盡的光輝者,他們是世間中\twnr{證涅槃}{71.0}者。」[\suttaref{SN.46.17}, \ccchref{AN.10.117}{https://agama.buddhason.org/AN/an.php?keyword=10.117}]



\sutta{35}{35}{沙門性經第一}{https://agama.buddhason.org/SN/sn.php?keyword=45.35}
  起源於舍衛城。

  「\twnr{比丘}{31.0}們!我將為你們教導\twnr{沙門性}{328.0}與沙門果,\twnr{你們要聽}{43.0}它!

  比丘們!而什麼是沙門性?這\twnr{八支聖道}{525.0},即:正見……(中略)正定,比丘們!這被稱為沙門性。

  比丘們!而什麼是沙門果?\twnr{入流果}{165.1}、\twnr{一來果}{208.2}、\twnr{不還果}{209.1}、\twnr{阿羅漢}{5.0}果,比丘們!這被稱為沙門果。」



\sutta{36}{36}{沙門性經第二}{https://agama.buddhason.org/SN/sn.php?keyword=45.36}
  起源於舍衛城。

  「\twnr{比丘}{31.0}們!我將為你們教導\twnr{沙門性}{328.0}與\twnr{沙門義}{327.0},\twnr{你們要聽}{43.0}它!

  比丘們!而什麼是沙門性?就是這\twnr{八支聖道}{525.0},即:正見……(中略)正定,比丘們!這被稱為沙門性。

  比丘們!而什麼是沙門義?比丘們!凡貪的滅盡、瞋的滅盡、癡的滅盡,比丘們!這被稱為沙門義。」



\sutta{37}{37}{婆羅門性經第一}{https://agama.buddhason.org/SN/sn.php?keyword=45.37}
  起源於舍衛城。

  「\twnr{比丘}{31.0}們!我將為你們教導婆羅門性與婆羅門果,\twnr{你們要聽}{43.0}它!

  比丘們!而什麼是婆羅門性?就是這\twnr{八支聖道}{525.0},即:正見……(中略)正定,比丘們!這被稱為婆羅門性。

  比丘們!而什麼是婆羅門果?\twnr{入流果}{165.1}、\twnr{一來果}{208.2}、\twnr{不還果}{209.1}、\twnr{阿羅漢}{5.0}果,比丘們!這被稱為婆羅門果。」



\sutta{38}{38}{婆羅門性經第二}{https://agama.buddhason.org/SN/sn.php?keyword=45.38}
  起源於舍衛城。

  「\twnr{比丘}{31.0}們!我將為你們教導婆羅門性與婆羅門義,\twnr{你們要聽}{43.0}它!

  比丘們!而什麼是婆羅門性?就是這\twnr{八支聖道}{525.0},即:正見……(中略)正定,比丘們!這被稱為婆羅門性。

  比丘們!而什麼是婆羅門義?比丘們!凡貪的滅盡、瞋的滅盡、癡的滅盡,比丘們!這被稱為婆羅門義。」



\sutta{39}{39}{梵行經第一}{https://agama.buddhason.org/SN/sn.php?keyword=45.39}
  起源於舍衛城。

  「\twnr{比丘}{31.0}們!我將為你們教導梵行與梵行果,\twnr{你們要聽}{43.0}它!

  比丘們!而什麼是梵行?就是這\twnr{八支聖道}{525.0},即:正見……(中略)正定,比丘們!這被稱為梵行。

  比丘們!而什麼是梵行果?\twnr{入流果}{165.1}、\twnr{一來果}{208.2}、\twnr{不還果}{209.1}、\twnr{阿羅漢}{5.0}果,比丘們!這被稱為梵行果。」



\sutta{40}{40}{梵行經第二}{https://agama.buddhason.org/SN/sn.php?keyword=45.40}
  起源於舍衛城。

  「\twnr{比丘}{31.0}們!我將為你們教導梵行與梵行義,\twnr{你們要聽}{43.0}它!

  比丘們!而什麼是梵行?就是這\twnr{八支聖道}{525.0},即:正見……(中略)正定,比丘們!這被稱為梵行。

  比丘們!而什麼是梵行義?比丘們!凡貪的滅盡、瞋的滅盡、癡的滅盡,比丘們!這被稱為梵行義。」

  行品第四,其\twnr{攝頌}{35.0}:

  「行、行者,與已錯失、到彼岸,

   以及以沙門性二說,婆羅門性後二說,

   以梵行二說,以那個被稱為品。」





\pin{其他外道中略品}{41}{48}
\sutta{41}{41}{貪的褪去經}{https://agama.buddhason.org/SN/sn.php?keyword=45.41}
  起源於舍衛城。

  「\twnr{比丘}{31.0}們!如果其\twnr{他外道遊行者}{79.0}們這麼問你們:『\twnr{道友}{201.0}們!為了什麼目的在\twnr{沙門}{29.0}\twnr{喬達摩}{80.0}處梵行被住?』比丘們!被這麼問,你們應該這麼回答那些其他外道遊行者:『道友們!為了貪的\twnr{褪去}{77.0},在世尊處梵行被住。』

  比丘們!如果其他外道遊行者們再這麼問你們:『道友們!那麼,為了貪的褪去,有道、\twnr{有道跡}{359.0}嗎?』比丘們!被這麼問,你們應該這麼回答那些其他外道遊行者:『道友們!為了貪的褪去,有道、有道跡。』

  比丘們!而為了貪的褪去,什麼是道?什麼是道跡?就是這\twnr{八支聖道}{525.0},即:正見……(中略)正定。

  比丘們!為了貪的褪去,這是道,這是道跡。比丘們!被這麼問,你們應該這麼回答那些其他外道遊行者。」



\sutta{42}{47}{結的捨斷等經六則}{https://agama.buddhason.org/SN/sn.php?keyword=45.42}
  「\twnr{比丘}{31.0}們!如果其\twnr{他外道遊行者}{79.0}們這麼問你們:『\twnr{道友}{201.0}們!為了什麼目的在\twnr{沙門}{29.0}\twnr{喬達摩}{80.0}處梵行被住?』比丘們!被這麼問,你們應該這麼回答那些其他外道遊行者:『道友們!為了結的捨斷,在世尊處梵行被住。』……(中略)。」

  「……『道友們!為了\twnr{煩惱潛在趨勢}{253.1}的根除,在世尊處梵行被住。』……(中略)。」

  「……『道友們!為了\twnr{[生命]旅途的遍知}{889.0},在世尊處梵行被住。』……(中略)。」

  「……『道友們!為了諸\twnr{漏}{188.0}的滅盡,在世尊處梵行被住。』……(中略)。」

  「……『道友們!為了明、解脫果的作證,在世尊處梵行被住。』……(中略)。」

  「……『道友們!為了\twnr{智見}{433.0},在世尊處梵行被住。』……(中略)。」



\sutta{48}{48}{無執取般涅槃經}{https://agama.buddhason.org/SN/sn.php?keyword=45.48}
  起源於舍衛城。

  「\twnr{比丘}{31.0}們!如果其\twnr{他外道遊行者}{79.0}們這麼問你們:『\twnr{道友}{201.0}們!為了什麼目的在\twnr{沙門}{29.0}\twnr{喬達摩}{80.0}處梵行被住?』比丘們!被這麼問,你們應該這麼回答那些其他外道遊行者:『道友們!為了無取著\twnr{般涅槃}{72.0},在世尊處梵行被住。』

  比丘們!如果其他外道遊行者們再這麼問你們:『道友們!那麼,為了無執取般涅槃,有道、\twnr{有道跡}{359.0}嗎?』比丘們!被這麼問,你們應該這麼回答那些其他外道遊行者:『道友們!為了無執取般涅槃,有道、有道跡。』

  比丘們!而為了無執取般涅槃,什麼是道?什麼是道跡?就是這\twnr{八支聖道}{525.0},即:正見……(中略)正定。

  比丘們!為了無執取般涅槃,這是道,這是道跡。比丘們!被這麼問,你們應該這麼回答那些其他外道遊行者。」

  其他外道遊行者中略品第五,其\twnr{攝頌}{35.0}:

  「褪去、結、煩惱潛在趨勢,[人生]旅途、\twnr{漏}{188.0}的滅盡,

   明、解脫與智,無執取為第八則。」





\pin{太陽中略品}{49}{62}
\sutta{49}{49}{善友經}{https://agama.buddhason.org/SN/sn.php?keyword=45.49}
  起源於舍衛城。

  「\twnr{比丘}{31.0}們!對太陽的昇起,這是先導,這是前兆,即:黎明。同樣的,比丘們!對比丘\twnr{八支聖道}{525.0}的生起,這是先導,這是前兆,即:善友誼。

  比丘們!有善友比丘的這個能被預期:他必將\twnr{修習}{94.0}八支聖道,他必將\twnr{多作}{95.0}八支聖道。

  比丘們!而怎樣有善友的比丘修習八支聖道、多作八支聖道?比丘們!這裡,比丘\twnr{依止遠離}{322.0}、依止離貪、依止滅、\twnr{捨棄的成熟}{221.0}修習正見……(中略)依止遠離、依止離貪、依止滅、捨棄的成熟修習正定。

  比丘們!這樣,有善友的比丘修習八支聖道、多作八支聖道。」



\sutta{50}{54}{戒具足等經五則}{https://agama.buddhason.org/SN/sn.php?keyword=45.50}
  「\twnr{比丘}{31.0}們!對太陽的昇起,這是先導,這是前兆,即:黎明。同樣的,比丘們!對比丘\twnr{八支聖道}{525.0}的生起,這是先導,這是前兆,即:戒具足。

  比丘們!戒具足比丘的這個能被預期:……(中略)即:意欲具足……(中略)即:\twnr{自己具足}{690.2}……(中略)即:見具足……(中略)即:不放逸具足……(中略)。」



\sutta{55}{55}{如理作意具足經}{https://agama.buddhason.org/SN/sn.php?keyword=45.55}
  「\twnr{比丘}{31.0}們!對太陽的昇起,這是先導,這是前兆,即:黎明。同樣的,比丘們!對比丘\twnr{八支聖道}{525.0}的生起,這是先導,這是前兆,即:\twnr{如理作意}{114.0}具足。

  比丘們!如理作意具足比丘的這個能被預期:他必將\twnr{修習}{94.0}八支聖道,他必將\twnr{多作}{95.0}八支聖道。

  比丘們!而怎樣如理作意具足的比丘修習八支聖道、多作八支聖道?比丘們!這裡,比丘\twnr{依止遠離}{322.0}、依止離貪、依止滅、\twnr{捨棄的成熟}{221.0}修習正見……(中略)依止遠離、依止離貪、依止滅、捨棄的成熟修習正定。

  比丘們!這樣,如理作意具足的比丘修習八支聖道、多作八支聖道。」



\sutta{56}{56}{善友經}{https://agama.buddhason.org/SN/sn.php?keyword=45.56}
  「\twnr{比丘}{31.0}們!對太陽的昇起,這是先導,這是前兆,即:黎明。同樣的,比丘們!對比丘\twnr{八支聖道}{525.0}的生起,這是先導,這是前兆,即:善友誼。

  比丘們!有善友比丘的這個能被預期:他必將\twnr{修習}{94.0}八支聖道,他必將\twnr{多作}{95.0}八支聖道。

  比丘們!而怎樣有善友的比丘修習八支聖道、多作八支聖道?比丘們!這裡,比丘有:貪之調伏的完結、瞋之調伏的完結、癡之調伏的完結修習正見……(中略)有:貪之調伏的完結、瞋之調伏的完結、癡之調伏的完結修習正定。

  比丘們!這樣,有善友的比丘修習八支聖道、多作八支聖道。」



\sutta{57}{61}{戒具足等經五則}{https://agama.buddhason.org/SN/sn.php?keyword=45.57}
  「\twnr{比丘}{31.0}們!對太陽的昇起,這是先導,這是前兆,即:黎明。同樣的,比丘們!對比丘\twnr{八支聖道}{525.0}的生起,這是先導,這是前兆,即:戒具足。……(中略)即:意欲具足……(中略)即:\twnr{自己具足}{690.2}……(中略)即:見具足……(中略)即:不放逸具足……(中略)。」



\sutta{62}{62}{如理作意具足經}{https://agama.buddhason.org/SN/sn.php?keyword=45.62}
  「……即:\twnr{如理作意}{114.0}具足。

  \twnr{比丘}{31.0}們!如理作意具足比丘的這個能被預期:他必將\twnr{修習}{94.0}\twnr{八支聖道}{525.0},他必將\twnr{多作}{95.0}八支聖道。

  比丘們!而怎樣如理作意具足的比丘修習八支聖道、多作八支聖道?比丘們!這裡,比丘有:貪之調伏的完結、瞋之調伏的完結、癡之調伏的完結修習正見……(中略)有:貪之調伏的完結、瞋之調伏的完結、癡之調伏的完結修習正定。

  比丘們!這樣,如理作意具足的比丘修習八支聖道、多作八支聖道。」

  太陽中略品第六,其\twnr{攝頌}{35.0}:

  善友、戒,與意欲、自己具足,

  見與不放逸,如理修習第七。





\pin{一法中略品}{63}{76}
\sutta{63}{63}{善友經}{https://agama.buddhason.org/SN/sn.php?keyword=45.63}
  起源於舍衛城。

  「\twnr{比丘}{31.0}們!有一法對\twnr{八支聖道}{525.0}的生起是多助益的,哪一法?即:善友誼。

  比丘們!有善友比丘的這個能被預期:他必將\twnr{修習}{94.0}八支聖道,他必將\twnr{多作}{95.0}八支聖道。

  比丘們!而怎樣有善友的比丘修習八支聖道、多作八支聖道?比丘們!這裡,比丘\twnr{依止遠離}{322.0}、依止離貪、依止滅、\twnr{捨棄的成熟}{221.0}修習正見……(中略)依止遠離、依止離貪、依止滅、捨棄的成熟修習正定。

  比丘們!這樣,有善友的比丘修習八支聖道、多作八支聖道。」



\sutta{64}{68}{戒具足等經五則}{https://agama.buddhason.org/SN/sn.php?keyword=45.64}
  「\twnr{比丘}{31.0}們!有一法對\twnr{八支聖道}{525.0}的生起是多助益的,哪一法?即:戒具足。……(中略)即:意欲具足……(中略)即:\twnr{自己具足}{690.2}……(中略)即:見具足……(中略)即:不放逸具足……(中略)。」



\sutta{69}{69}{如理作意具足經}{https://agama.buddhason.org/SN/sn.php?keyword=45.69}
  「……即:\twnr{如理作意}{114.0}具足。

  \twnr{比丘}{31.0}們!如理作意具足比丘的這個能被預期:他必將\twnr{修習}{94.0}\twnr{八支聖道}{525.0},他必將\twnr{多作}{95.0}八支聖道。

  比丘們!而怎樣如理作意具足的比丘修習八支聖道、多作八支聖道?比丘們!這裡,比丘\twnr{依止遠離}{322.0}、依止離貪、依止滅、\twnr{捨棄的成熟}{221.0}修習正見……(中略)依止遠離、依止離貪、依止滅、捨棄的成熟修習正定。

  比丘們!這樣,如理作意具足的比丘修習八支聖道、多作八支聖道。」



\sutta{70}{70}{善友經}{https://agama.buddhason.org/SN/sn.php?keyword=45.70}
  起源於舍衛城。

  「\twnr{比丘}{31.0}們!有一法對\twnr{八支聖道}{525.0}的生起是多助益的,哪一法?即:善友誼。

  比丘們!有善友比丘的這個能被預期:他必將\twnr{修習}{94.0}八支聖道,他必將\twnr{多作}{95.0}八支聖道。

  比丘們!而怎樣有善友的比丘修習八支聖道、多作八支聖道?比丘們!這裡,比丘有:貪之調伏的完結、瞋之調伏的完結、癡之調伏的完結修習正見……(中略)有:貪之調伏的完結、瞋之調伏的完結、癡之調伏的完結修習正定。

  比丘們!這樣,有善友的比丘修習八支聖道、多作八支聖道。」



\sutta{71}{75}{戒具足等經五則}{https://agama.buddhason.org/SN/sn.php?keyword=45.71}
  起源於舍衛城。

  「\twnr{比丘}{31.0}們!有一法對\twnr{八支聖道}{525.0}的生起是多助益的,哪一法?即:戒具足。……(中略)即:意欲具足……(中略)即:\twnr{自己具足}{690.2}……(中略)即:見具足……(中略)即:不放逸具足……(中略)。」



\sutta{76}{76}{如理作意具足經}{https://agama.buddhason.org/SN/sn.php?keyword=45.76}
  「……即:\twnr{如理作意}{114.0}具足。

  \twnr{比丘}{31.0}們!如理作意具足比丘的這個能被預期:他必將\twnr{修習}{94.0}\twnr{八支聖道}{525.0},他必將\twnr{多作}{95.0}八支聖道。

  比丘們!而怎樣如理作意具足的比丘修習八支聖道、多作八支聖道?比丘們!這裡,比丘……修習正見……(中略)有:貪之調伏的完結、瞋之調伏的完結、癡之調伏的完結修習正定。比丘們!這樣,如理作意具足的比丘修習八支聖道、多作八支聖道。」

  一法中略品第七,其\twnr{攝頌}{35.0}:

  善友、戒,與意欲、自己具足,

  見與不放逸,如理修習第七。





\pin{第二個一法中略品}{77}{90}
\sutta{77}{77}{善友經}{https://agama.buddhason.org/SN/sn.php?keyword=45.77}
  起源於舍衛城。

  「\twnr{比丘}{31.0}們!我不見還有其它一法,以那個,未生起的\twnr{八支聖道}{525.0}生起,或已生起的八支聖道走到修習圓滿,比丘們!如這善友誼。比丘們!有善友比丘的這個能被預期:他必將修習八支聖道,必將\twnr{多作}{95.0}八支聖道。

  比丘們!而怎樣有善友的比丘修習八支聖道、多作八支聖道?比丘們!這裡,比丘\twnr{依止遠離}{322.0}、依止離貪、依止滅、\twnr{捨棄的成熟}{221.0}修習正見……(中略)依止遠離、依止離貪、依止滅、捨棄的成熟修習正定。

  比丘們!這樣,有善友的比丘修習八支聖道、多作八支聖道。」



\sutta{78}{82}{戒具足等經五則}{https://agama.buddhason.org/SN/sn.php?keyword=45.78}
  「\twnr{比丘}{31.0}們!我不見還有其它一法,以那個,未生起的\twnr{八支聖道}{525.0}生起,或已生起之八支聖道走到修習圓滿,比丘們!如這戒具足。……(中略)比丘們!如這意欲具足……(中略)比丘們!如這\twnr{自己具足}{690.2}……(中略)比丘們!如這見具足……(中略)比丘們!如這不放逸具足……(中略)。」



\sutta{83}{83}{如理作意具足經}{https://agama.buddhason.org/SN/sn.php?keyword=45.83}
  「……即:\twnr{如理作意}{114.0}具足。

  \twnr{比丘}{31.0}們!如理作意具足比丘的這個能被預期:他必將\twnr{修習}{94.0}\twnr{八支聖道}{525.0},他必將\twnr{多作}{95.0}八支聖道。

  比丘們!而怎樣如理作意具足的比丘修習八支聖道、多作八支聖道?比丘們!這裡,比丘\twnr{依止遠離}{322.0}、依止離貪、依止滅、\twnr{捨棄的成熟}{221.0}修習正見……(中略)依止遠離、依止離貪、依止滅、捨棄的成熟修習正定。

  比丘們!這樣,如理作意具足的比丘修習八支聖道、多作八支聖道。」



\sutta{84}{84}{善友經}{https://agama.buddhason.org/SN/sn.php?keyword=45.84}
  「\twnr{比丘}{31.0}們!我不見還有其它一法,以那個,未生起的八支聖道生起,或已生起的\twnr{八支聖道}{525.0}走到修習圓滿,比丘們!如這善友誼。比丘們!有善友比丘的這個能被預期:他必將修習八支聖道,必將\twnr{多作}{95.0}八支聖道。

  比丘們!而怎樣有善友的比丘修習八支聖道、多作八支聖道?比丘們!這裡,比丘有:貪之調伏的完結、瞋之調伏的完結、癡之調伏的完結修習正見……(中略)有:貪之調伏的完結、瞋之調伏的完結、癡之調伏的完結修習正定。

  比丘們!這樣,有善友的比丘修習八支聖道、多作八支聖道。」



\sutta{85}{89}{戒具足等經五則}{https://agama.buddhason.org/SN/sn.php?keyword=45.85}
  「\twnr{比丘}{31.0}們!我不見還有其它一法,以那個,未生起的八支聖道生起,或已生起的\twnr{八支聖道}{525.0}走到修習圓滿,比丘們!如這戒具足。……(中略)比丘們!如這意欲具足……(中略)比丘們!如這\twnr{自己具足}{690.2}……(中略)比丘們!如這見具足……(中略)比丘們!如這不放逸具足……(中略)。」



\sutta{90}{90}{如理作意具足經}{https://agama.buddhason.org/SN/sn.php?keyword=45.90}
  「……即:\twnr{如理作意}{114.0}具足。

  \twnr{比丘}{31.0}們!如理作意具足比丘的這個能被預期:他必將\twnr{修習}{94.0}\twnr{八支聖道}{525.0},他必將\twnr{多作}{95.0}八支聖道。

  比丘們!而怎樣如理作意具足的比丘修習八支聖道、多作八支聖道?比丘們!這裡,比丘有:貪之調伏的完結、瞋之調伏的完結、癡之調伏的完結修習正見……(中略)有:貪之調伏的完結、瞋之調伏的完結、癡之調伏的完結修習正定。

  比丘們!這樣,如理作意具足的比丘修習八支聖道、多作八支聖道。」

  第二個一法中略品第八,其\twnr{攝頌}{35.0}:

  善友、戒,與意欲、自己具足,

  見與不放逸,如理修習第七。





\pin{恒河中略品}{91}{102}
\sutta{91}{91}{傾向東經第一}{https://agama.buddhason.org/SN/sn.php?keyword=45.91}
  起源於舍衛城。

  「\twnr{比丘}{31.0}們!猶如恒河是傾向東的、斜向東的、坡斜向東的。同樣的,比丘們!\twnr{修習}{94.0}\twnr{八支聖道}{525.0}、\twnr{多作}{95.0}八支聖道的比丘是傾向涅槃的、斜向涅槃的、坡斜向涅槃的。

  比丘們!而怎樣修習八支聖道、多作八支聖道的比丘是傾向涅槃的、斜向涅槃的、坡斜向涅槃的?比丘們!這裡,比丘\twnr{依止遠離}{322.0}、依止離貪、依止滅、\twnr{捨棄的成熟}{221.0}修習正見……(中略)依止遠離、依止離貪、依止滅、捨棄的成熟修習正定。

  比丘們!這樣修習八支聖道、多作八支聖道的比丘是傾向涅槃的、斜向涅槃的、坡斜向涅槃的。」



\sutta{92}{95}{傾向東經第二等四則}{https://agama.buddhason.org/SN/sn.php?keyword=45.92}
  「\twnr{比丘}{31.0}們!猶如耶牟那河是傾向東的、斜向東的、坡斜向東的。同樣的,比丘們!……(中略)比丘們!猶如阿致羅筏底河是傾向東的、斜向東的、坡斜向東的。同樣的,比丘們!……(中略)比丘們!猶如薩羅浮河是傾向東的、斜向東的、坡斜向東的。同樣的,比丘們!……(中略)比丘們!猶如摩醯河是傾向東的、斜向東的、坡斜向東的。同樣的,比丘們!……(中略)。」



\sutta{96}{96}{傾向東經第六}{https://agama.buddhason.org/SN/sn.php?keyword=45.96}
  「\twnr{比丘}{31.0}們!猶如凡任何大河,即:恒河、耶牟那河、阿致羅筏底河、薩羅浮河、摩醯河,那些全部是傾向東的、斜向東的、坡斜向東的。同樣的,比丘們!\twnr{修習}{94.0}\twnr{八支聖道}{525.0}、\twnr{多作}{95.0}八支聖道的比丘是傾向涅槃的、斜向涅槃的、坡斜向涅槃的。

  比丘們!而怎樣修習八支聖道、多作八支聖道的比丘是傾向涅槃的、斜向涅槃的、坡斜向涅槃的?比丘們!這裡,比丘\twnr{依止遠離}{322.0}、依止離貪、依止滅、\twnr{捨棄的成熟}{221.0}修習正見……(中略)依止遠離、依止離貪、依止滅、捨棄的成熟修習正定。

  比丘們!這樣修習八支聖道、多作八支聖道的比丘是傾向涅槃的、斜向涅槃的、坡斜向涅槃的。」



\sutta{97}{97}{傾向大海經第一}{https://agama.buddhason.org/SN/sn.php?keyword=45.97}
  「\twnr{比丘}{31.0}們!猶如恒河是傾向大海的、斜向大海的、坡斜向大海的。同樣的,比丘們!\twnr{修習}{94.0}\twnr{八支聖道}{525.0}、\twnr{多作}{95.0}八支聖道的比丘是傾向涅槃的、斜向涅槃的、坡斜向涅槃的。

  比丘們!而怎樣修習八支聖道、多作八支聖道的比丘是傾向涅槃的、斜向涅槃的、坡斜向涅槃的?比丘們!這裡,比丘\twnr{依止遠離}{322.0}、依止離貪、依止滅、\twnr{捨棄的成熟}{221.0}修習正見……(中略)依止遠離、依止離貪、依止滅、捨棄的成熟修習正定。

  比丘們!這樣修習八支聖道、多作八支聖道的比丘是傾向涅槃的、斜向涅槃的、坡斜向涅槃的。」



\sutta{98}{102}{傾向大海經第二等五則}{https://agama.buddhason.org/SN/sn.php?keyword=45.98}
  「\twnr{比丘}{31.0}們!猶如耶牟那河是傾向大海的、斜向大海的、坡斜向大海的。同樣的,比丘們!……(中略)比丘們!猶如阿致羅筏底河是傾向大海的、斜向大海的、坡斜向大海的。同樣的,比丘們!……(中略)比丘們!猶如薩羅浮河是傾向大海的、斜向大海的、坡斜向大海的。同樣的,比丘們!……(中略)比丘們!猶如摩醯河是傾向大海的、斜向大海的、坡斜向大海的。同樣的,比丘們!……(中略)。比丘們!猶如凡任何大河,即:恒河、耶牟那河、阿致羅筏底河、薩羅浮河、摩醯河都是傾向大海的、斜向大海的、坡斜向大海的。同樣的,比丘們!\twnr{修習}{94.0}\twnr{八支聖道}{525.0}、\twnr{多作}{95.0}八支聖道的比丘是傾向涅槃的、斜向涅槃的、坡斜向涅槃的。

  比丘們!而怎樣修習八支聖道、多作八支聖道的比丘是傾向涅槃的、斜向涅槃的、坡斜向涅槃的?比丘們!這裡,比丘\twnr{依止遠離}{322.0}、依止離貪、依止滅、\twnr{捨棄的成熟}{221.0}修習正見……(中略)依止遠離、依止離貪、依止滅、捨棄的成熟修習正定。

  比丘們!這樣修習八支聖道、多作八支聖道的比丘是傾向涅槃的、斜向涅槃的、坡斜向涅槃的。」

  恒河中略品第一,其\twnr{攝頌}{35.0}:

  六則傾向東的,與六則傾向大海的,

  這兩個六則成十二則,以那個被稱為品。

  恒河中略[品]傾向東的背誦之道(尋求背誦的?),依止遠離十二則第一的。





\pin{第二個恒河中略品}{103}{138}
\sutta{103}{103}{傾向東經第一}{https://agama.buddhason.org/SN/sn.php?keyword=45.103}
  「\twnr{比丘}{31.0}們!猶如恒河是傾向東的、斜向東的、坡斜向東的。同樣的,比丘們!\twnr{修習}{94.0}\twnr{八支聖道}{525.0}、\twnr{多作}{95.0}八支聖道的比丘是傾向涅槃的、斜向涅槃的、坡斜向涅槃的。

  比丘們!而怎樣修習八支聖道、多作八支聖道的比丘是傾向涅槃的、斜向涅槃的、坡斜向涅槃的?比丘們!這裡,比丘有:貪之調伏的完結、瞋之調伏的完結、癡之調伏的完結修習正見……(中略)有:貪之調伏的完結、瞋之調伏的完結、癡之調伏的完結修習正定。 

  比丘們!這樣修習八支聖道、多作八支聖道的比丘是傾向涅槃的、斜向涅槃的、坡斜向涅槃的。」 



\sutta{104}{108}{傾向東經第二等五則}{https://agama.buddhason.org/SN/sn.php?keyword=45.104}
  「\twnr{比丘}{31.0}們!猶如耶牟那河是傾向東的、斜向東的、坡斜向東的。同樣的,比丘們!……(中略)。

  比丘們!猶如阿致羅筏底河是傾向東的、斜向東的、坡斜向東的。同樣的,比丘們!……(中略)。

  比丘們!猶如薩羅浮河是傾向東的、斜向東的、坡斜向東的。同樣的,比丘們!……(中略)。

  比丘們!猶如摩醯河是傾向東的、斜向東的、坡斜向東的。同樣的,比丘們!……(中略)。

  比丘們!猶如凡任何大河,即:恒河、耶牟那河、阿致羅筏底河、薩羅浮河、摩醯河,那些全部是傾向東的、斜向東的、坡斜向東的。同樣的,比丘們!……(中略)。」



\sutta{109}{109}{傾向大海經第一}{https://agama.buddhason.org/SN/sn.php?keyword=45.109}
  「\twnr{比丘}{31.0}們!猶如恒河是傾向大海的、斜向大海的、坡斜向大海的。同樣的,比丘們!\twnr{修習}{94.0}\twnr{八支聖道}{525.0}、\twnr{多作}{95.0}八支聖道的比丘是傾向涅槃的、斜向涅槃的、坡斜向涅槃的。

  比丘們!而怎樣修習八支聖道、多作八支聖道的比丘是傾向涅槃的、斜向涅槃的、坡斜向涅槃的?比丘們!這裡,比丘有:貪之調伏的完結、瞋之調伏的完結、癡之調伏的完結修習正見……(中略)有:貪之調伏的完結、瞋之調伏的完結、癡之調伏的完結修習正定。 

  比丘們!這樣修習八支聖道、多作八支聖道的比丘是傾向涅槃的、斜向涅槃的、坡斜向涅槃的。」 



\sutta{110}{114}{傾向大海經第二等五則}{https://agama.buddhason.org/SN/sn.php?keyword=45.110}
  「\twnr{比丘}{31.0}們!猶如耶牟那河是傾向大海的、斜向大海的、坡斜向大海的。同樣的,比丘們!……(中略)。

  比丘們!猶如阿致羅筏底河是傾向大海的、斜向大海的、坡斜向大海的。同樣的,比丘們!……(中略)。

  比丘們!猶如薩羅浮河是傾向大海的、斜向大海的、坡斜向大海的。同樣的,比丘們!……(中略)。

  比丘們!猶如摩醯河是傾向大海的、斜向大海的、坡斜向大海的。同樣的,比丘們!……(中略)。

  比丘們!猶如凡任何大河,即:恒河、耶牟那河、阿致羅筏底河、薩羅浮河、摩醯河都是傾向大海的、斜向大海的、坡斜向大海的。同樣的,比丘們!\twnr{修習}{94.0}\twnr{八支聖道}{525.0}、\twnr{多作}{95.0}八支聖道的比丘是傾向涅槃的、斜向涅槃的、坡斜向涅槃的。

  比丘們!而怎樣修習八支聖道、多作八支聖道的比丘是傾向涅槃的、斜向涅槃的、坡斜向涅槃的?比丘們!這裡,比丘有:貪之調伏的完結、瞋之調伏的完結、癡之調伏的完結修習正見……(中略)有:貪之調伏的完結、瞋之調伏的完結、癡之調伏的完結修習正定。

  比丘們!這樣修習八支聖道、多作八支聖道的比丘是傾向涅槃的、斜向涅槃的、坡斜向涅槃的。」

  (「貪之調伏十二則第二個傾向大海」)



\sutta{115}{115}{傾向東經第一}{https://agama.buddhason.org/SN/sn.php?keyword=45.115}
  「\twnr{比丘}{31.0}們!猶如恒河是傾向東的、斜向東的、坡斜向東的。同樣的,比丘們!\twnr{修習}{94.0}\twnr{八支聖道}{525.0}、\twnr{多作}{95.0}八支聖道的比丘是傾向涅槃的、斜向涅槃的、坡斜向涅槃的。

  比丘們!而怎樣修習八支聖道、多作八支聖道的比丘是傾向涅槃的、斜向涅槃的、坡斜向涅槃的?比丘們!這裡,比丘有\twnr{不死}{123.0}的立足處、不死的\twnr{彼岸}{226.0}、不死的完結修習正見……(中略)有不死的立足處、不死的彼岸、不死的完結修習正定。 

  比丘們!這樣修習八支聖道、多作八支聖道的比丘是傾向涅槃的、斜向涅槃的、坡斜向涅槃的。」 



\sutta{116}{120}{傾向東經第二等五則}{https://agama.buddhason.org/SN/sn.php?keyword=45.116}
  「\twnr{比丘}{31.0}們!猶如耶牟那河是傾向東的、斜向東的、坡斜向東的。同樣的,比丘們!……(中略)。

  比丘們!猶如阿致羅筏底河是傾向東的、斜向東的、坡斜向東的。同樣的,比丘們!……(中略)。

  比丘們!猶如薩羅浮河是傾向東的、斜向東的、坡斜向東的。同樣的,比丘們!……(中略)。

  比丘們!猶如摩醯河是傾向東的、斜向東的、坡斜向東的。同樣的,比丘們!……(中略)。

  比丘們!猶如凡任何大河,即:恒河、耶牟那河、阿致羅筏底河、薩羅浮河、摩醯河,那些全部是傾向東的、斜向東的、坡斜向東的。同樣的,比丘們!……(中略)。」



\sutta{121}{121}{傾向大海經第一}{https://agama.buddhason.org/SN/sn.php?keyword=45.121}
  「\twnr{比丘}{31.0}們!猶如恒河是傾向大海的、斜向大海的、坡斜向大海的。同樣的,比丘們!\twnr{修習}{94.0}\twnr{八支聖道}{525.0}、\twnr{多作}{95.0}八支聖道的比丘是傾向涅槃的、斜向涅槃的、坡斜向涅槃的。

  比丘們!而怎樣修習八支聖道、多作八支聖道的比丘是傾向涅槃的、斜向涅槃的、坡斜向涅槃的?比丘們!這裡,比丘有\twnr{不死}{123.0}的立足處、不死的\twnr{彼岸}{226.0}、不死的完結修習正見……(中略)有不死的立足處、不死的彼岸、不死的完結修習正定。 

  比丘們!這樣修習八支聖道、多作八支聖道的比丘是傾向涅槃的、斜向涅槃的、坡斜向涅槃的。」 



\sutta{122}{126}{傾向大海經第二等五則}{https://agama.buddhason.org/SN/sn.php?keyword=45.122}
  「\twnr{比丘}{31.0}們!猶如耶牟那河是傾向大海的、斜向大海的、坡斜向大海的。同樣的,比丘們!……(中略)。

  比丘們!猶如阿致羅筏底河是傾向大海的、斜向大海的、坡斜向大海的。同樣的,比丘們!……(中略)。

  比丘們!猶如薩羅浮河是傾向大海的、斜向大海的、坡斜向大海的。同樣的,比丘們!……(中略)。

  比丘們!猶如摩醯河是傾向大海的、斜向大海的、坡斜向大海的。同樣的,比丘們!……(中略)。

  比丘們!猶如凡任何大河,即:恒河、耶牟那河、阿致羅筏底河、薩羅浮河、摩醯河都是傾向大海的、斜向大海的、坡斜向大海的。同樣的,比丘們!\twnr{修習}{94.0}\twnr{八支聖道}{525.0}、\twnr{多作}{95.0}八支聖道的比丘是傾向涅槃的、斜向涅槃的、坡斜向涅槃的。

  比丘們!而怎樣修習八支聖道、多作八支聖道的比丘是傾向涅槃的、斜向涅槃的、坡斜向涅槃的?比丘們!這裡,比丘有\twnr{不死}{123.0}的立足處、不死的\twnr{彼岸}{226.0}、不死的完結修習正見……(中略)有不死的立足處、不死的彼岸、不死的完結修習正定。

  比丘們!這樣修習八支聖道、多作八支聖道的比丘是傾向涅槃的、斜向涅槃的、坡斜向涅槃的。」

  (不死的立足處十二則第三個)



\sutta{127}{127}{傾向東經第一}{https://agama.buddhason.org/SN/sn.php?keyword=45.127}
  「\twnr{比丘}{31.0}們!猶如恒河是傾向東的、斜向東的、坡斜向東的。同樣的,比丘們!\twnr{修習}{94.0}\twnr{八支聖道}{525.0}、\twnr{多作}{95.0}八支聖道的比丘是傾向涅槃的、斜向涅槃的、坡斜向涅槃的。

  比丘們!而怎樣修習八支聖道、多作八支聖道的比丘是傾向涅槃的、斜向涅槃的、坡斜向涅槃的?比丘們!這裡,比丘傾向涅槃地、斜向涅槃地、坡斜向涅槃地修習正見……(中略)傾向涅槃地、斜向涅槃地、坡斜向涅槃地修習正定。 

  比丘們!這樣修習八支聖道、多作八支聖道的比丘是傾向涅槃的、斜向涅槃的、坡斜向涅槃的。」 



\sutta{128}{132}{傾向東經第二等五則}{https://agama.buddhason.org/SN/sn.php?keyword=45.128}
  「\twnr{比丘}{31.0}們!猶如耶牟那河是傾向東的、斜向東的、坡斜向東的。同樣的,比丘們!……(中略)。

  比丘們!猶如阿致羅筏底河是傾向東的、斜向東的、坡斜向東的。同樣的,比丘們!……(中略)。

  比丘們!猶如薩羅浮河是傾向東的、斜向東的、坡斜向東的。同樣的,比丘們!……(中略)。

  比丘們!猶如摩醯河是傾向東的、斜向東的、坡斜向東的。同樣的,比丘們!……(中略)。

  比丘們!猶如凡任何大河,即:恒河、耶牟那河、阿致羅筏底河、薩羅浮河、摩醯河,那些全部是傾向東的、斜向東的、坡斜向東的。同樣的,比丘們!\twnr{修習}{94.0}\twnr{八支聖道}{525.0}、\twnr{多作}{95.0}八支聖道的比丘是傾向涅槃的、斜向涅槃的、坡斜向涅槃的。

  比丘們!而怎樣修習八支聖道、多作八支聖道的比丘是傾向涅槃的、斜向涅槃的、坡斜向涅槃的?比丘們!這裡,比丘傾向涅槃地、斜向涅槃地、坡斜向涅槃地修習正見……(中略)傾向涅槃地、斜向涅槃地、坡斜向涅槃地修習正定。 

  比丘們!這樣修習八支聖道、多作八支聖道的比丘是傾向涅槃的、斜向涅槃的、坡斜向涅槃的。」 



\sutta{133}{133}{傾向大海經第一}{https://agama.buddhason.org/SN/sn.php?keyword=45.133}
  「\twnr{比丘}{31.0}們!猶如恒河是傾向大海的、斜向大海的、坡斜向大海的。同樣的,比丘們!\twnr{修習}{94.0}\twnr{八支聖道}{525.0}、\twnr{多作}{95.0}八支聖道的比丘是傾向涅槃的、斜向涅槃的、坡斜向涅槃的。

  比丘們!而怎樣修習八支聖道、多作八支聖道的比丘是傾向涅槃的、斜向涅槃的、坡斜向涅槃的?比丘們!這裡,比丘傾向涅槃地、斜向涅槃地、坡斜向涅槃地修習正見……(中略)傾向涅槃地、斜向涅槃地、坡斜向涅槃地修習正定。 

  比丘們!這樣修習八支聖道、多作八支聖道的比丘是傾向涅槃的、斜向涅槃的、坡斜向涅槃的。」 



\sutta{134}{138}{傾向大海經第二等五則}{https://agama.buddhason.org/SN/sn.php?keyword=45.134}
  「\twnr{比丘}{31.0}們!猶如耶牟那河是傾向大海的、斜向大海的、坡斜向大海的。同樣的,比丘們!……(中略)。

  比丘們!猶如阿致羅筏底河是傾向大海的、斜向大海的、坡斜向大海的。同樣的,比丘們!……(中略)。

  比丘們!猶如薩羅浮河是傾向大海的、斜向大海的、坡斜向大海的。同樣的,比丘們!……(中略)。

  比丘們!猶如摩醯河是傾向大海的、斜向大海的、坡斜向大海的。同樣的,比丘們!……(中略)。

  比丘們!猶如凡任何大河,即:恒河、耶牟那河、阿致羅筏底河、薩羅浮河、摩醯河都是傾向大海的、斜向大海的、坡斜向大海的。同樣的,比丘們!\twnr{修習}{94.0}\twnr{八支聖道}{525.0}、\twnr{多作}{95.0}八支聖道的比丘是傾向涅槃的、斜向涅槃的、坡斜向涅槃的。

  比丘們!而怎樣修習八支聖道、多作八支聖道的比丘是傾向涅槃的、斜向涅槃的、坡斜向涅槃的?比丘們!這裡,比丘傾向涅槃地、斜向涅槃地、坡斜向涅槃地修習正見……(中略)傾向涅槃地、斜向涅槃地、坡斜向涅槃地修習正定。

  比丘們!這樣修習八支聖道、多作八支聖道的比丘是傾向涅槃的、斜向涅槃的、坡斜向涅槃的。」

  (恒河中略[品])

  第二個恒河中略品第二,其\twnr{攝頌}{35.0}:

  「六則傾向東的,與六則傾向大海的,

   這兩個六則成為十二則,以那個被稱為品。」

  傾向涅槃的十二則[127-138]:第四個第六個九個一組的(?)。





\pin{不放逸中略品}{139}{148}
\sutta{139}{139}{如來經}{https://agama.buddhason.org/SN/sn.php?keyword=45.139}
  起源於舍衛城。

  「\twnr{比丘}{31.0}們!眾生之所及:無足的,或二足的,或四足的,或多足的,或有色的,或無色的,或有想的,或無想的,或非想非非想的,\twnr{如來}{4.0}、\twnr{阿羅漢}{5.0}、\twnr{遍正覺者}{6.0}被告知為他們中第一的。同樣的,比丘們!凡任何善法,那些全都以不放逸為根、以不放逸為會合,不放逸被說為那些法中第一的。

  比丘們!不放逸比丘的這個能被預期:他必將\twnr{修習}{94.0}\twnr{八支聖道}{525.0}、必將\twnr{多作}{95.0}八支聖道。比丘們!而怎樣不放逸的比丘修習八支聖道、多作八支聖道?比丘們!這裡,比丘\twnr{依止遠離}{322.0}、依止離貪、依止滅、\twnr{捨棄的成熟}{221.0}修習正見……(中略)依止遠離、依止離貪、依止滅、捨棄的成熟修習正定。比丘們!這樣,不放逸的比丘修習八支聖道、多作八支聖道。

  比丘們!眾生之所及:無足的,或二足的,或四足的,或多足的,或有色的,或無色的,或有想的,或無想的,或非想非非想的來說,如來、阿羅漢、遍正覺者被告知為他們中第一的。同樣的,比丘們!凡任何善法,那些全都以不放逸為根、以不放逸為會合,不放逸被說為那些法中第一的。

  比丘們!不放逸比丘的這個能被預期:他必將修習八支聖道、必將多作八支聖道。比丘們!而怎樣不放逸的比丘修習八支聖道、多作八支聖道?比丘們!這裡,比丘有:貪之調伏的完結、瞋之調伏的完結、癡之調伏的完結修習正見……(中略)有:貪之調伏的完結、瞋之調伏的完結、癡之調伏的完結修習正定。比丘們!這樣,不放逸的比丘修習八支聖道、多作八支聖道。

  比丘們!眾生之所及:無足的,或二足的,或四足的,或多足的,或有色的,或無色的,或有想的,或無想的,或非想非非想的來說,如來、阿羅漢、遍正覺者被告知為他們中第一的。同樣的,比丘們!凡任何善法,那些全都以不放逸為根、以不放逸為會合,不放逸被說為那些法中第一的。

  比丘們!不放逸比丘的這個能被預期:他必將修習八支聖道、必將多作八支聖道。比丘們!而怎樣不放逸的比丘修習八支聖道、多作八支聖道?比丘們!這裡,比丘有\twnr{不死}{123.0}的立足處、不死的\twnr{彼岸}{226.0}、不死的完結修習正見……(中略)有不死的立足處、不死的彼岸、不死的完結修習正定。比丘們!這樣,不放逸的比丘修習八支聖道、多作八支聖道。

  比丘們!眾生之所及:無足的,或二足的,或四足的,或多足的,或有色的,或無色的,或有想的,或無想的,或非想非非想的來說,如來、阿羅漢、遍正覺者被告知為他們中第一的。同樣的,比丘們!凡任何善法,那些全都以不放逸為根、以不放逸為會合,不放逸被說為那些法中第一的。

  比丘們!不放逸比丘的這個能被預期:他必將修習八支聖道、必將多作八支聖道。比丘們!而怎樣不放逸的比丘修習八支聖道、多作八支聖道?比丘們!這裡,比丘傾向涅槃、斜向涅槃、坡斜向涅槃地修習正見……(中略)傾向涅槃、斜向涅槃、坡斜向涅槃地修習正定。比丘們!這樣,不放逸的比丘修習八支聖道、多作八支聖道。」



\sutta{140}{140}{足跡經}{https://agama.buddhason.org/SN/sn.php?keyword=45.140}
  「\twnr{比丘}{31.0}們!猶如凡任何叢林生物的足跡類,那些全都在象的足跡中走到容納,象的足跡被告知為它們中第一的,即:以大的狀態。同樣的,比丘們!凡任何善法,那些全都以不放逸為根、以不放逸為會合,不放逸被說為那些法中第一的。

  比丘們!不放逸比丘的這個能被預期:他必將\twnr{修習}{94.0}\twnr{八支聖道}{525.0}、必將\twnr{多作}{95.0}八支聖道。比丘們!而怎樣不放逸的比丘修習八支聖道、多作八支聖道?比丘們!這裡,比丘依止遠離、依止離貪、依止滅、捨棄的成熟修習正見……(中略)依止遠離、依止離貪、依止滅、捨棄的成熟修習正定。……(中略)比丘們!這樣,不放逸的比丘修習八支聖道、多作八支聖道。」



\sutta{141}{145}{屋頂經等五則}{https://agama.buddhason.org/SN/sn.php?keyword=45.141}
  「\twnr{比丘}{31.0}們!猶如凡任何\twnr{重閣}{213.0}的\twnr{椽}{663.0},那些全部是走到屋頂的、傾向屋頂的、屋頂為會合,屋頂被告知為它們中第一的。同樣的,比丘們!……(中略)。」

  「比丘們!猶如凡任何香根,黑鳶尾草被告知為它們中第一的。同樣的,比丘們!……(中略)。」

  「比丘們!猶如凡任何香樹心,紫檀被告知為它們中第一的。同樣的,比丘們!……(中略)。」

  「比丘們!猶如凡任何香花,茉莉花被告知為它們中第一的。同樣的,比丘們!……(中略)。」

  「比丘們!猶如凡任何小王,他們全部是\twnr{轉輪王}{278.0}的從屬,轉輪王被告知為他們中第一的。同樣的,比丘們!……(中略)。」



\sutta{146}{148}{月亮光輝經等三則}{https://agama.buddhason.org/SN/sn.php?keyword=45.146}
  「\twnr{比丘}{31.0}們!猶如凡任何星光的光明,那些全都不及月亮光明的十六分之一,月亮光明被告知為它們中第一的。同樣的,比丘們!……。」

  「比丘們!猶如晴朗無雲的秋天,日出時,[陽光]輝耀、照亮、照耀,驅散空中一切黑暗。同樣的,比丘們!……。」

  「比丘們!猶如凡任何編織衣服,\twnr{迦尸衣}{665.0}被告知為它們中第一的。同樣的,比丘們!凡任何善法,那些全都以不放逸為根、以不放逸為會合,不放逸被說為那些法中第一的。

  比丘們!不放逸比丘的這個能被預期:他必將\twnr{修習}{94.0}\twnr{八支聖道}{525.0}、必將\twnr{多作}{95.0}八支聖道。比丘們!而怎樣不放逸的比丘修習八支聖道、多作八支聖道?比丘們!這裡,比丘依止遠離、依止離貪、依止滅、捨棄的成熟修習正見……(中略)依止遠離、依止離貪、依止滅、捨棄的成熟修習正定。……(中略)比丘們!這樣,不放逸的比丘修習八支聖道、多作八支聖道。」

  (應該如如來[經]那樣使之被細說)

  不放逸中略品第五,其\twnr{攝頌}{35.0}:

  「如來、足跡、屋頂,根、樹心與茉莉花,

   王、月、日,以衣服為第十句。」





\pin{應該被力量作的品}{149}{160}
\sutta{149}{149}{力量經}{https://agama.buddhason.org/SN/sn.php?keyword=45.149}
  起源於舍衛城。

  「\twnr{比丘}{31.0}們!猶如凡任何應該被力量作的工作被作,它們全部依止於土地後,住立於土地後,這樣,這些應該被力量作的工作被作。同樣的,比丘們!比丘依止戒後,住立於戒後,\twnr{修習}{94.0}\twnr{八支聖道}{525.0}、\twnr{多作}{95.0}八支聖道。

  比丘們!而怎樣比丘依止戒後,住立於戒後,修習八支聖道、多作八支聖道?比丘們!這裡,比丘\twnr{依止遠離}{322.0}、依止離貪、依止滅、\twnr{捨棄的成熟}{221.0}修習正見……(中略)比丘們!這樣,比丘依止戒後,住立於戒後,修習八支聖道、多作八支聖道。

   (「其它的從恒河中略的方式各經被圓滿」為詳細之道(尋求?))

  比丘們!猶如凡任何應該被力量作的工作被作,它們全部依止於土地後,住立於土地後,這樣,這些應該被力量作的工作被作。同樣的,比丘們!比丘依止戒後,住立於戒後,修習八支聖道、多作八支聖道。 

  比丘們!而怎樣比丘依止戒後,住立於戒後,修習八支聖道、多作八支聖道?比丘們!這裡,比丘有:貪之調伏的完結、瞋之調伏的完結、癡之調伏的完結修習正見……(中略)有:貪之調伏的完結、瞋之調伏的完結、癡之調伏的完結修習正定。比丘們!這樣,比丘依止戒後,住立於戒後,修習八支聖道、多作八支聖道。

  比丘們!猶如凡任何應該被力量作的工作被作,它們全部依止於土地後,住立於土地後,這樣,這些應該被力量作的工作被作。同樣的,比丘們!比丘依止戒後,住立於戒後,修習八支聖道、多作八支聖道。 

  比丘們!而怎樣比丘依止戒後,住立於戒後,修習八支聖道、多作八支聖道?比丘們!這裡,比丘有\twnr{不死}{123.0}的立足處、不死的\twnr{彼岸}{226.0}、不死的完結修習正見……(中略)有不死的立足處、不死的彼岸、不死的完結修習正定。比丘們!這樣,比丘依止戒後,住立於戒後,修習八支聖道、多作八支聖道。

  比丘們!猶如凡任何應該被力量作的工作被作,它們全部依止於土地後,住立於土地後,這樣,這些應該被力量作的工作被作。同樣的,比丘們!比丘依止戒後,住立於戒後,修習八支聖道、多作八支聖道。 

  比丘們!而怎樣比丘依止戒後,住立於戒後,修習八支聖道、多作八支聖道?比丘們!這裡,比丘傾向涅槃地、斜向涅槃地、坡斜向涅槃地修習正見……(中略)傾向涅槃地、斜向涅槃地、坡斜向涅槃地修習正定。比丘們!這樣,比丘依止戒後,住立於戒後,修習八支聖道、多作八支聖道。」



\sutta{150}{150}{種子經}{https://agama.buddhason.org/SN/sn.php?keyword=45.150}
  「\twnr{比丘}{31.0}們!猶如凡任何這些\twnr{種子類}{638.0}、草木類來到成長、增長、成滿者,它們全部依止土地後,住立於土地後,這樣,這些種子類、草木類來到成長、增長、成滿。同樣的,比丘們!依止戒後,住立於戒後\twnr{修習}{94.0}\twnr{八支聖道}{525.0}、\twnr{多作}{95.0}八支聖道的比丘在諸法上來到成長、增長、成滿。

  比丘們!而依止戒後,住立於戒後怎樣修習八支聖道、多作八支聖道的比丘在諸法上來到成長、增長、成滿?比丘們!這裡,比丘\twnr{依止遠離}{322.0}……(中略)修習正見……(中略)依止遠離、依止離貪、依止滅、\twnr{捨棄的成熟}{221.0}修習正定。……(中略)比丘們!依止戒後,住立於戒後這樣修習八支聖道、多作八支聖道的比丘在諸法上來到成長、增長、成滿。」



\sutta{151}{151}{龍經}{https://agama.buddhason.org/SN/sn.php?keyword=45.151}
  「\twnr{比丘}{31.0}們!猶如依止喜馬拉雅山山王後,諸龍使身體生長、捉取力量,牠們在那裡使身體生長、捉取力量後,進入小池;進入小池後,進入大池;進入大池後,進入小河;進入小河後,進入大河;進入大河後,進入大海,牠們在那裡以身體來到巨大狀態、成滿狀態。同樣的,比丘們!依止戒後,住立於戒後\twnr{修習}{94.0}\twnr{八支聖道}{525.0}、\twnr{多作}{95.0}八支聖道的比丘在諸法上到達巨大狀態、成滿狀態。

  比丘們!而依止戒後,住立於戒後怎樣修習八支聖道、多作八支聖道的比丘在諸法上到達大狀態、成滿狀態?比丘們!這裡,比丘\twnr{依止遠離}{322.0}、依止離貪、依止滅、\twnr{捨棄的成熟}{221.0}修習正見……(中略)依止遠離、依止離貪、依止滅、捨棄的成熟修習正定。……(中略)比丘們!依止戒後,住立於戒後這樣修習八支聖道、多作八支聖道的比丘在諸法上到達巨大狀態、成滿狀態。」



\sutta{152}{152}{樹木經}{https://agama.buddhason.org/SN/sn.php?keyword=45.152}
  「\twnr{比丘}{31.0}們!猶如樹木是傾向東的、斜向東的、坡斜向東的,被切斷根的它往哪邊倒下?」

  「\twnr{大德}{45.0}!往傾向處,往斜向處,往坡斜向處。」

  「同樣的,比丘們!\twnr{修習}{94.0}\twnr{八支聖道}{525.0}、\twnr{多作}{95.0}八支聖道的比丘是傾向涅槃的、斜向涅槃的、坡斜向涅槃的。

  比丘們!而怎樣修習八支聖道、多作八支聖道的比丘是傾向涅槃的、斜向涅槃的、坡斜向涅槃的?比丘們!這裡,比丘\twnr{依止遠離}{322.0}、依止離貪、依止滅、\twnr{捨棄的成熟}{221.0}修習正見……(中略)依止遠離、依止離貪、依止滅、捨棄的成熟修習正定。……(中略)比丘們!這樣修習八支聖道、多作八支聖道的比丘是傾向涅槃的、斜向涅槃的、坡斜向涅槃的。」



\sutta{153}{153}{瓶子經}{https://agama.buddhason.org/SN/sn.php?keyword=45.153}
  「\twnr{比丘}{31.0}們!猶如倒下的瓶子就吐出水,不逆吞回。同樣的,比丘們!\twnr{修習}{94.0}\twnr{八支聖道}{525.0}、\twnr{多作}{95.0}八支聖道的比丘就吐出諸惡不善法,不逆吞回。

  比丘們!而怎樣修習八支聖道、多作八支聖道的比丘就吐出諸惡不善法,不逆吞回?比丘們!這裡,比丘\twnr{依止遠離}{322.0}、依止離貪、依止滅、\twnr{捨棄的成熟}{221.0}修習正見……(中略)依止遠離、依止離貪、依止滅、捨棄的成熟修習正定。……(中略)比丘們!這樣修習八支聖道、多作八支聖道的比丘就吐惡不善法,不再收回。」



\sutta{154}{154}{穗尖經}{https://agama.buddhason.org/SN/sn.php?keyword=45.154}
  「\twnr{比丘}{31.0}們!猶如稻穗尖或麥穗尖被手或腳正確朝向地壓踏,『他將破裂手或腳,或將使血生起。』這存在可能性,那是什麼原因?比丘們!以穗尖的正確朝向狀態。同樣的,比丘們!比丘以正確朝向的見、以正確朝向的道之修習,『他將破壞無明,將使明生起,將作證涅槃。』這存在可能性,那是什麼原因?比丘們!以見的正確朝向狀態。

  比丘們!而怎樣比丘以正確朝向的見、以正確朝向的道之修習破壞無明,使明生起,作證涅槃?比丘們!這裡,比丘\twnr{依止遠離}{322.0}……修習正見……(中略)依止遠離、依止離貪、依止滅、\twnr{捨棄的成熟}{221.0}修習正定。比丘們!這樣,比丘以正確朝向的見、以正確朝向的道之修習破壞無明,使明生起,作證涅槃。」[\suttaref{SN.45.9}]



\sutta{155}{155}{虛空經}{https://agama.buddhason.org/SN/sn.php?keyword=45.155}
  「\twnr{比丘}{31.0}們!猶如在虛空中種種風吹:東風吹,西風也吹,北風也吹,南風也吹,有塵風也吹,無塵風也吹,冷風也吹,熱風也吹,微風也吹,強風也吹。同樣的,比丘們!對\twnr{修習}{94.0}\twnr{八支聖道}{525.0}、\twnr{多作}{95.0}八支聖道的比丘來說,\twnr{四念住}{286.0}走到修習圓滿;\twnr{四正勤}{292.0}走到修習圓滿;\twnr{四神足}{503.1}走到修習圓滿;五根走到修習圓滿;五力走到修習圓滿;\twnr{七覺支}{524.0}走到修習圓滿。

  比丘們!而對怎樣修習八支聖道、多作八支聖道的比丘來說,四念住走到修習圓滿;四正勤走到修習圓滿;四神足走到修習圓滿;五根走到修習圓滿;五力走到修習圓滿;七覺支走到修習圓滿?比丘們!這裡,比丘……修習正見……(中略)\twnr{依止遠離}{322.0}、依止離貪、依止滅、\twnr{捨棄的成熟}{221.0}修習正定。……(中略)比丘們!這樣,當比丘修習八支聖道、多作八支聖道時,四念住走到修習圓滿;四正勤走到修習圓滿;四神足走到修習圓滿;五根走到修習圓滿;五力走到修習圓滿;七覺支走到修習圓滿。」



\sutta{156}{156}{雨雲經第一}{https://agama.buddhason.org/SN/sn.php?keyword=45.156}
  「\twnr{比丘}{31.0}們!猶如在夏季的最後一個月塵垢被揚起,大非時雨(驟雨)立即地使它消失、平息。同樣的,比丘們!\twnr{修習}{94.0}\twnr{八支聖道}{525.0}、\twnr{多作}{95.0}八支聖道的比丘立即地一一使已生起的諸惡不善法消失、平息。

  比丘們!而怎樣修習八支聖道、多作八支聖道的比丘立即地一一使已生起的諸惡不善法消失、平息?比丘們!這裡,比丘……修習正見……(中略)\twnr{依止遠離}{322.0}、依止離貪、依止滅、\twnr{捨棄的成熟}{221.0}修習正定。……(中略)比丘們!這樣修習八支聖道、多作八支聖道的比丘立即地一一使已生起的諸惡不善法消失、平息。」



\sutta{157}{157}{雨雲經第二}{https://agama.buddhason.org/SN/sn.php?keyword=45.157}
  「\twnr{比丘}{31.0}們!猶如已生起的大雨雲,大風就從內部使它消失、平息。同樣的,比丘們!\twnr{修習}{94.0}\twnr{八支聖道}{525.0}、\twnr{多作}{95.0}八支聖道的比丘就從內部使一一生起的諸惡不善法消失、平息。

  比丘們!而怎樣修習八支聖道、多作八支聖道的比丘就從內部使一一生起的諸惡不善法消失、平息?比丘們!這裡,比丘……修習正見……(中略)\twnr{依止遠離}{322.0}、依止離貪、依止滅、\twnr{捨棄的成熟}{221.0}修習正定。……(中略)比丘們!這樣修習八支聖道、多作八支聖道的比丘就從內部使一一生起的諸惡不善法消失、平息。」



\sutta{158}{158}{船經}{https://agama.buddhason.org/SN/sn.php?keyword=45.158}
  「\twnr{比丘}{31.0}們!猶如對藤索固綁的航海船來說,六個月在水中消耗後,冬天被拉上陸地,被風、陽光影響,被雨雲下大雨的藤索,它們就少困難地止息,成為腐爛。同樣的,比丘們!對\twnr{修習}{94.0}\twnr{八支聖道}{525.0}、\twnr{多作}{95.0}八支聖道的比丘來說,諸結就少困難地止息,成為腐爛。

  比丘們!而對怎樣修習八支聖道、多作八支聖道的比丘來說,諸結就少困難地止息,成為腐爛?比丘們!這裡,比丘……修習正見……(中略)依\twnr{依止遠離}{322.0}、依止離貪、依止滅、\twnr{捨棄的成熟}{221.0}修習正定。……(中略)比丘們!這樣,當比丘修習八支聖道、多作八支聖道時,諸結就少困難地止息,成為腐爛。」



\sutta{159}{159}{來客經}{https://agama.buddhason.org/SN/sn.php?keyword=45.159}
  「\twnr{比丘}{31.0}們!猶如來客之屋舍,在那裡,他們從東方到來後作住所,他們從西方到來後也作住所,他們從北方到來後也作住所,他們從南方到來後也作住所;\twnr{剎帝利}{116.0}們到來後也作住所,\twnr{婆羅門}{17.0}們到來後也作住所,\twnr{毘舍}{476.0}們到來後也作住所,\twnr{首陀羅}{472.0}們到來後也作住所。同樣的,比丘們!\twnr{修習}{94.0}\twnr{八支聖道}{525.0}、\twnr{多作}{95.0}八支聖道的比丘,凡應該被證智\twnr{遍知}{154.0}的諸法,他以證智遍知那些法;凡應該被證智捨斷的諸法,他以證智捨斷那些法;凡應該被證智作證的諸法,他以證智作證那些法;凡應該被證智修習的諸法,他以證智修習那些法。

  比丘們!而哪些是應該被證智遍知的諸法?對那個,『\twnr{五取蘊}{36.0}』應該被回答。哪五個?即:色取蘊……(中略)識取蘊,這些是應該被證智遍知的諸法。

  比丘們!而哪些是應該被證智捨斷的諸法?\twnr{無明}{207.0}與\twnr{有的渴愛}{244.0},這些是應被證智捨斷的諸法。

  比丘們!而哪些是應該被證智作證的諸法?明與解脫,這些是應該被證智作證的諸法。

  比丘們!而哪些是應該被證智修習的諸法?\twnr{止與觀}{178.0},這些是應該被證智修習的諸法。

  比丘們!而怎樣修習八支聖道、多作八支聖道的比丘,凡應該被證智遍知的諸法,他以證智遍知……(中略)凡應該被證智修習的諸法,他以證智修習?比丘們!這裡,比丘……修習正見……(中略)\twnr{依止遠離}{322.0}、依止離貪、依止滅、\twnr{捨棄的成熟}{221.0}修習正定。……(中略)比丘們!這樣修習八支聖道、多作八支聖道的比丘,凡應該被證智遍知的諸法,他以證智遍知那些法;凡應該被證智捨斷的諸法,他以證智捨斷那些法;凡應該被證智作證的諸法,他以證智作證那些法;凡應該被證智修習的諸法,他以證智修習那些法。」



\sutta{160}{160}{河經}{https://agama.buddhason.org/SN/sn.php?keyword=45.160}
  「\twnr{比丘}{31.0}們!猶如恒河是傾向東的、斜向東的、坡斜向東的,那時,大群人拿鋤頭、籃子後:『我們將轉(作)這恒河成傾向西的、斜向西的、坡斜向西的。』比丘們!你們怎麼想它:是否那個大群人會轉這恒河成傾向西的、斜向西的、坡斜向西的呢?」

  「\twnr{大德}{45.0}!這確實不是,那是什麼原因?大德!恒河是傾向東的、斜向東的、坡斜向東的,不容易轉成傾向西的、斜向西的、坡斜向西的,還有,大群人最終只會是疲勞的、苦惱的\twnr{有分者}{876.0}。」

  「同樣的,比丘們!如果國王,或國王的大臣,或朋友,或同事,或親族,或血親以財富帶來後邀請\twnr{修習}{94.0}\twnr{八支聖道}{525.0}、\twnr{多作}{95.0}八支聖道比丘:『喂!來!男子!為何讓這些袈裟耗盡你?為何你實行光頭、鉢?來!還俗後請你在財富上受用與作福德。』比丘們!確實,那位修習八支聖道、多作八支聖道的比丘,『他放棄學後將還俗。』\twnr{這不存在可能性}{650.0},那是什麼原因?比丘們!因為那顆心長久是傾向遠離的、斜向遠離的、坡斜向遠離的,『他確實將還俗。』這不存在可能性。

  比丘們!而怎樣比丘修習八支聖道、多作八支聖道?比丘們!這裡,比丘\twnr{依止遠離}{322.0}……修習正見……(中略)依止遠離、依止離貪、依止滅、\twnr{捨棄的成熟}{221.0}修習正定。……(中略)比丘們!這樣,比丘修習八支聖道、多作八支聖道。」(應該如被力量作的[品]那樣使之被細說)

  應該被力量作的品第六,其\twnr{攝頌}{35.0}:

  「力量、種子與龍,樹木、以瓶子、以穗尖,

   以虛空與二則雨雲,船、來客、河。」





\pin{尋求品}{161}{171}
\sutta{161}{161}{尋求經}{https://agama.buddhason.org/SN/sn.php?keyword=45.161}
  起源於舍衛城。

  「\twnr{比丘}{31.0}們!有這些三種尋求,哪三個?欲的尋求、有的尋求、\twnr{梵行的尋求}{381.1},比丘們!這些是三種尋求。比丘們!為了這些三種尋求的證智,\twnr{八支聖道}{525.0}應該被\twnr{修習}{94.0}。比丘們!什麼是八支聖道?比丘們!這裡,比丘\twnr{依止遠離}{322.0}……修習正見……(中略)比丘依止遠離、依止離貪、依止滅、\twnr{捨棄的成熟}{221.0}修習正定。比丘們!為了這些三種尋求的證智,這八支聖道應該被修習。

   比丘們!有這些三種尋求,哪三個?欲的尋求、有的尋求、梵行的尋求,比丘們!這些是三種尋求。比丘們!為了這些三種尋求的證智,八支聖道應該被修習。比丘們!什麼是八支聖道?比丘們!這裡,比丘……修習正見……(中略)有貪之調伏的完結、瞋之調伏的完結、癡之調伏的完結修習正定。比丘們!為了這些三種尋求的證智,這八支聖道應該被修習。

   比丘們!有這些三種尋求,哪三個?欲的尋求、有的尋求、梵行的尋求,比丘們!這些是三種尋求。比丘們!為了這些三種尋求的證智,八支聖道應該被修習。比丘們!什麼是八支聖道?比丘們!這裡,比丘有\twnr{不死}{123.0}的立足處、不死的\twnr{彼岸}{226.0}、不死的完結修習正見……(中略)有不死的立足處、不死的彼岸、不死的完結修習正定。比丘們!為了這些三種尋求的證智,這八支聖道應該被修習。

   比丘們!有這些三種尋求,哪三個?欲的尋求、有的尋求、梵行的尋求,比丘們!這些是三種尋求。比丘們!為了這些三種尋求的證智,八支聖道應該被修習。比丘們!什麼是八支聖道?比丘們!這裡,比丘……修習正見……(中略)傾向涅槃地、斜向涅槃地、坡斜向涅槃地修習正定。比丘們!為了這些三種尋求的證智,這八支聖道應該被修習。

   比丘們!有這些三種尋求,哪三個?欲的尋求、有的尋求、梵行的尋求,比丘們!這些是三種尋求。比丘們!為了這三種尋求的\twnr{遍知}{154.0}……(中略)這八支聖道應該被修習。(應該如證智那樣使遍知被細說)

   比丘們!有這些三種尋求,哪三個?欲的尋求、有的尋求、梵行的尋求,比丘們!這些是三種尋求。比丘們!為了這三種尋求的遍盡……(中略)這八支聖道應該被修習。(應該如證智那樣使遍盡被細說)

   比丘們!有這些三種尋求,哪三個?欲的尋求、有的尋求、梵行的尋求,比丘們!這些是三種尋求。比丘們!為了這三種尋求的捨斷,八支聖道應該被修習。比丘們!什麼是八支聖道?比丘們!這裡,……修習正見……(中略)比丘依止遠離、依止離貪、依止滅、捨棄的成熟修習正定。……(中略)比丘們!為了這三種尋求的捨斷,這八支聖道應該被修習。」(應該如證智那樣使捨斷被細說)



\sutta{162}{162}{慢經}{https://agama.buddhason.org/SN/sn.php?keyword=45.162}
  「\twnr{比丘}{31.0}們!有這三種\twnr{慢}{647.0},哪三個?『我是優勝者』之慢、『我是同等者』之慢、『我是下劣者』之慢,比丘們!這些是三種慢。

  比丘們!為了這三種慢的證智、\twnr{遍知}{154.0}、遍盡、捨斷,\twnr{八支聖道}{525.0}應該被\twnr{修習}{94.0},哪八支聖道?比丘們!這裡,比丘……修習正見……(中略)\twnr{依止遠離}{322.0}、依止離貪、依止滅、\twnr{捨棄的成熟}{221.0}修習正定。……(中略)比丘們!為了這三種慢的證智、遍知、遍盡、捨斷,這八支聖道應該被修習。」(應該如尋求[經]那樣使之被細說)



\sutta{163}{163}{漏經}{https://agama.buddhason.org/SN/sn.php?keyword=45.163}
  「\twnr{比丘}{31.0}們!有這三種\twnr{漏}{188.0},哪三個?欲漏、有漏、\twnr{無明漏}{397.0},比丘們!這些是三種漏。比丘們!為了這三種漏的證智、\twnr{遍知}{154.0}、遍盡、捨斷……(中略)這\twnr{八支聖道}{525.0}應該被\twnr{修習}{94.0}。」



\sutta{164}{164}{有經}{https://agama.buddhason.org/SN/sn.php?keyword=45.164}
  「\twnr{比丘}{31.0}們!有這三種有,哪三個?欲有、色有、\twnr{無色有}{261.0},比丘們!這些是三種有。比丘們!為了這三種有的證智、\twnr{遍知}{154.0}、遍盡、捨斷……(中略)這\twnr{八支聖道}{525.0}應該被\twnr{修習}{94.0}。」



\sutta{165}{165}{苦性經}{https://agama.buddhason.org/SN/sn.php?keyword=45.165}
  「\twnr{比丘}{31.0}們!有這三種苦性,哪三個?苦苦性、行苦性、\twnr{變易苦性}{649.0},比丘們!這些是三種苦性。比丘們!為了這三種苦性的證智、\twnr{遍知}{154.0}、遍盡、捨斷……(中略)這\twnr{八支聖道}{525.0}應該被\twnr{修習}{94.0}。」



\sutta{166}{166}{荒蕪經}{https://agama.buddhason.org/SN/sn.php?keyword=45.166}
  「\twnr{比丘}{31.0}們!有這三種\twnr{荒蕪}{599.0},哪三個?貪荒蕪、瞋荒蕪、癡荒蕪,比丘們!這些是三種荒蕪。比丘們!為了這三種荒蕪的證智、\twnr{遍知}{154.0}、遍盡、捨斷……(中略)這\twnr{八支聖道}{525.0}應該被\twnr{修習}{94.0}。」



\sutta{167}{167}{垢經}{https://agama.buddhason.org/SN/sn.php?keyword=45.167}
  「\twnr{比丘}{31.0}們!有這三種垢,哪三個?貪垢、瞋垢、癡垢,比丘們!這些是三種垢。比丘們!為了這三種垢的證智、\twnr{遍知}{154.0}、遍盡、捨斷……(中略)這\twnr{八支聖道}{525.0}應該被\twnr{修習}{94.0}。」



\sutta{168}{168}{惱亂經}{https://agama.buddhason.org/SN/sn.php?keyword=45.168}
  「\twnr{比丘}{31.0}們!有這三種惱亂,哪三個?貪惱亂、瞋惱亂、癡惱亂,比丘們!這些是三種惱亂。比丘們!為了這三種惱亂的證智、\twnr{遍知}{154.0}、遍盡、捨斷……(中略)這\twnr{八支聖道}{525.0}應該被\twnr{修習}{94.0}。」



\sutta{169}{169}{受經}{https://agama.buddhason.org/SN/sn.php?keyword=45.169}
  「\twnr{比丘}{31.0}們!有這三種受,哪三個?樂受、苦受、不苦不樂受,比丘們!這些是三種受。比丘們!為了這三種受的證智、\twnr{遍知}{154.0}、遍盡、捨斷……(中略)這\twnr{八支聖道}{525.0}應該被\twnr{修習}{94.0}。」



\sutta{170}{170}{渴愛經}{https://agama.buddhason.org/SN/sn.php?keyword=45.170}
  「\twnr{比丘}{31.0}們!有這三種渴愛,哪三個?欲的渴愛、有的渴愛、\twnr{虛無的渴愛}{244.0},比丘們!這些是三種渴愛。比丘們!為了這三種渴愛的證智、\twnr{遍知}{154.0}、遍盡、捨斷……(中略)這\twnr{八支聖道}{525.0}應該被\twnr{修習}{94.0}。比丘們!而什麼是八支聖道?比丘們!這裡,比丘\twnr{依止遠離}{322.0}、依止離貪、依止滅、\twnr{捨棄的成熟}{221.0}修習正見……(中略)比丘依止遠離、依止離貪、依止滅、捨棄的成熟修習正定。比丘們!為了這三種渴愛的證智、遍知、遍盡、捨斷……(中略)這八支聖道應該被修習。」



\sutta{171}{171}{渴望經}{https://agama.buddhason.org/SN/sn.php?keyword=45.171}
  「\twnr{比丘}{31.0}們!有這三種\twnr{渴望}{x595},哪三個?欲的渴望、有的渴望、虛無的渴望,比丘們!這些是三種渴望。比丘們!為了這三種渴望的證智、\twnr{遍知}{154.0}、遍盡、捨斷……(中略)這\twnr{八支聖道}{525.0}應該被\twnr{修習}{94.0}。……(中略)有:貪之調伏的完結、瞋之調伏的完結、癡之調伏的完結……(中略)有\twnr{不死}{123.0}的立足處、不死的\twnr{彼岸}{226.0}、不死的完結……(中略)是傾向涅槃的、斜向涅槃的、坡斜向涅槃的……(中略)。比丘們!為了這三種渴望的證智、遍知、遍盡、捨斷……(中略)這八支聖道應該被修習。」

   尋求品第七,其\twnr{攝頌}{35.0}:

  「尋求、慢、漏,有與苦性、荒蕪,

   垢、惱亂與受,二種渴愛與渴望。」





\pin{暴流品}{172}{181}
\sutta{172}{172}{暴流經}{https://agama.buddhason.org/SN/sn.php?keyword=45.172}
  起源於舍衛城。 

  「\twnr{比丘}{31.0}們!有這四種暴流,哪四種?欲的暴流、\twnr{有的暴流}{369.0}、見的暴流、\twnr{無明}{207.0}的暴流,比丘們!這是四種暴流。比丘們!為了這四種暴流的證智、\twnr{遍知}{154.0}、遍盡、捨斷……(中略)這\twnr{八支聖道}{525.0}應該被\twnr{修習}{94.0}。」(全部應該如尋求[經]那樣使之被細說)



\sutta{173}{173}{軛經}{https://agama.buddhason.org/SN/sn.php?keyword=45.173}
  「\twnr{比丘}{31.0}們!有這四種軛,哪四種?欲軛、有軛、見軛、\twnr{無明}{207.0}軛,比丘們!這是四種軛。比丘們!為了這四種軛的證智、\twnr{遍知}{154.0}、遍盡、捨斷……(中略)這\twnr{八支聖道}{525.0}應該被\twnr{修習}{94.0}。」



\sutta{174}{174}{取經}{https://agama.buddhason.org/SN/sn.php?keyword=45.174}
  「\twnr{比丘}{31.0}們!有這四種取,哪四種?欲取、見取、\twnr{戒禁取}{194.0}、\twnr{[真]我論取}{444.0},比丘們!這是四種取。比丘們!為了這四種取的證智、\twnr{遍知}{154.0}、遍盡、捨斷……(中略)這\twnr{八支聖道}{525.0}應該被\twnr{修習}{94.0}。」



\sutta{175}{175}{束縛經}{https://agama.buddhason.org/SN/sn.php?keyword=45.175}
  「\twnr{比丘}{31.0}們!有這四種束縛,哪四種?\twnr{貪婪}{435.0}的身束縛、惡意的身束縛、\twnr{戒禁取}{194.0}的身束縛、\twnr{[只有]這是真實之執持的身束縛}{x596},比丘們!這是四種束縛。比丘們!為了這四種束縛的證智、\twnr{遍知}{154.0}、遍盡、捨斷……(中略)這\twnr{八支聖道}{525.0}應該被\twnr{修習}{94.0}。」



\sutta{176}{176}{煩惱潛在趨勢經}{https://agama.buddhason.org/SN/sn.php?keyword=45.176}
  「\twnr{比丘}{31.0}們!有這七種\twnr{煩惱潛在趨勢}{253.1},哪七種?欲煩惱潛在趨勢、\twnr{嫌惡煩惱潛在趨勢}{453.0}、\twnr{見煩惱潛在趨勢}{162.0}、\twnr{疑煩惱潛在趨勢}{223.0}、\twnr{慢煩惱潛在趨勢}{26.0}、\twnr{有貪煩惱潛在趨勢}{429.0}、\twnr{無明煩惱潛在趨勢}{454.0},比丘們!這些是七種煩惱潛在趨勢。比丘們!為了這七種煩惱潛在趨勢的證智、\twnr{遍知}{154.0}、遍盡、捨斷……(中略)這\twnr{八支聖道}{525.0}應該被\twnr{修習}{94.0}。」



\sutta{177}{177}{欲種類經}{https://agama.buddhason.org/SN/sn.php?keyword=45.177}
  「\twnr{比丘}{31.0}們!有這\twnr{五種欲}{187.0},哪五種?能被眼識知的、想要的、所愛的、合意的、可愛形色的、伴隨欲的、誘人的諸色,能被耳識知的……(中略)諸聲音,能被鼻識知的……(中略)諸氣味,能被舌識知的……(中略)諸味道,能被身識知的、想要的、所愛的、合意的、可愛形色的、伴隨欲的、誘人的諸\twnr{所觸}{220.2},比丘們!這些是五種欲。比丘們!為了這五種欲的證智、\twnr{遍知}{154.0}、遍盡、捨斷……(中略)這\twnr{八支聖道}{525.0}應該被\twnr{修習}{94.0}。」



\sutta{178}{178}{蓋經}{https://agama.buddhason.org/SN/sn.php?keyword=45.178}
  「\twnr{比丘}{31.0}們!有這\twnr{五蓋}{287.0},哪五種?\twnr{欲的意欲}{118.0}蓋、惡意蓋、惛沈睡眠蓋、掉舉後悔蓋、疑惑蓋,比丘們!這些是五蓋。比丘們!為了這五蓋的證智、\twnr{遍知}{154.0}、遍盡、捨斷……(中略)這\twnr{八支聖道}{525.0}應該被\twnr{修習}{94.0}。」



\sutta{179}{179}{取蘊經}{https://agama.buddhason.org/SN/sn.php?keyword=45.179}
  「\twnr{比丘}{31.0}們!有這\twnr{五取蘊}{36.0},哪五種?即:色取蘊、受取蘊、想取蘊、行取蘊、識取蘊,比丘們!這些是五取蘊。比丘們!為了這些五取蘊的證智、\twnr{遍知}{154.0}、遍盡、捨斷……(中略)這\twnr{八支聖道}{525.0}應該被\twnr{修習}{94.0}。」



\sutta{180}{180}{下分經}{https://agama.buddhason.org/SN/sn.php?keyword=45.180}
  「\twnr{比丘}{31.0}們!有這\twnr{五下分結}{134.0},哪五種?\twnr{有身見}{93.1}、疑、\twnr{戒禁取}{194.0}、\twnr{欲的意欲}{118.0}、惡意,比丘們!這些是五下分結。比丘們!為了這五下分結的證智、\twnr{遍知}{154.0}、遍盡、捨斷……(中略)這\twnr{八支聖道}{525.0}應該被\twnr{修習}{94.0}。」



\sutta{181}{181}{上分經}{https://agama.buddhason.org/SN/sn.php?keyword=45.181}
  「\twnr{比丘}{31.0}們!有這些五上分結,哪五個?色貪、無色貪、慢、掉舉、\twnr{無明}{207.0},比丘們!這些是五上分結。比丘們!為了這些五上分結的證智、\twnr{遍知}{154.0}、遍盡、捨斷,\twnr{八支聖道}{525.0}應該被\twnr{修習}{94.0}。哪八支聖道?比丘們!這裡,比丘\twnr{依止遠離}{322.0}……修習正見……(中略)比丘依止遠離、依止離貪、依止滅、\twnr{捨棄的成熟}{221.0}修習正定。比丘們!為了這些五上分結的證智、遍知、遍盡、捨斷,八支聖道應該被修習。

   比丘們!有這些五上分結,哪五個?色貪、無色貪、慢、掉舉、無明,比丘們!這些是五上分結。比丘們!為了這些五上分結的證智、遍知、遍盡、捨斷,八支聖道應該被修習。比丘們!而什麼是八支聖道?比丘們!這裡,比丘有:貪之調伏的完結、瞋之調伏的完結、癡之調伏的完結修習正見……(中略)有:貪之調伏的完結、瞋之調伏的完結、癡之調伏的完結修習正定。……(中略)有\twnr{不死}{123.0}的立足處、不死的\twnr{彼岸}{226.0}、不死的完結……(中略)是傾向涅槃的、斜向涅槃的、坡斜向涅槃的修習正定。比丘們!為了這些五上分結的證智、遍知、遍盡、捨斷,八支聖道應該被修習。」

  暴流品第八,其\twnr{攝頌}{35.0}:

  「暴流、軛、取,束縛、煩惱潛在趨勢,

   欲種類、蓋,蘊、下上分。」

  品的攝頌:

  「無明品第一,第二被稱為住處,

   邪性第三品,第四就以行,

   外道第五品,與第六太陽,

   \twnr{許多被作的第七品}{x597},與以生起第八,

   日中品第九,與第十不放逸,

   第十一力量品,第十二以尋求經,

   暴流品為第十三。」

  道相應第一。





\page

\xiangying{46}{覺支相應}
\pin{山品}{1}{10}
\sutta{1}{1}{喜馬拉雅山經}{https://agama.buddhason.org/SN/sn.php?keyword=46.1}
  起源於舍衛城。

  「\twnr{比丘}{31.0}們!猶如依止喜馬拉雅山山王後,諸龍使身體生長、捉取力量,牠們在那裡使身體生長、捉取力量後,進入小池;進入小池後,進入大池;進入大池後,進入小河;進入小河後,進入大河;進入大河後,進入大海,牠們在那裡以身體來到巨大狀態、成滿狀態。同樣的,比丘們!依止戒後,住立於戒後,\twnr{修習}{94.0}\twnr{七覺支}{524.0}、\twnr{多作}{95.0}七覺支的比丘在諸法上到達巨大狀態、成滿狀態。

  比丘們!而依止戒後,住立於戒後怎樣修習七覺支、多作七覺支的比丘在諸法上到達大狀態、成滿狀態?比丘們!這裡,比丘\twnr{依止遠離}{322.0}、依止離貪、依止滅、\twnr{捨棄的成熟}{221.0}修習\twnr{念覺支}{315.0}……(中略)修習\twnr{擇法覺支}{311.0}……(中略)修習\twnr{活力覺支}{310.0}……(中略)修習\twnr{喜覺支}{312.0}……(中略)修習\twnr{寧靜覺支}{313.1}……(中略)修習定覺支……(中略)依止遠離、依止離貪、依止滅、捨棄的成熟修習\twnr{平靜覺支}{314.0}。比丘們!依止戒後,住立於戒後這樣修習七覺支、多作七覺支的比丘在諸法上到達巨大狀態、成滿狀態。」



\sutta{2}{2}{身體經}{https://agama.buddhason.org/SN/sn.php?keyword=46.2}
  起源於舍衛城。

  「\twnr{比丘}{31.0}們!猶如這依食而存續的身體\twnr{緣於}{252.0}食而住立;無食而不住立。同樣的,比丘們!依食而存續的\twnr{五蓋}{287.0}緣於食而住立;無食而不住立。

  比丘們!而對未生起之\twnr{欲的意欲}{118.0}的生起,或已生起之欲的意欲的增大、成滿來說,什麼是食呢?比丘們!有\twnr{淨相}{597.0},在那裡,不\twnr{如理作意}{114.0}的\twnr{多作}{95.0},對未生起之欲的意欲生起,或已生起之欲的意欲的增大、成滿,這是食。

  比丘們!而對未生起之惡意的生起,或已生起之惡意的增大、成滿來說,什麼是食呢?比丘們!有\twnr{嫌惡相}{687.0},在那裡,不如理作意的多作,對未生起之惡意的生起,或已生起之惡意的增大、成滿,這是食。

  比丘們!而對未生起之惛沈睡眠的生起,或已生起之惛沈睡眠的增大、成滿來說,什麼是食呢?比丘們!有不樂、倦怠、打哈欠、餐後的睡意、\twnr{心的退縮}{686.1},在那裡,不如理作意的多作,對未生起之惛沈睡眠的生起,或已生起之惛沈睡眠的增大、成滿,這是食。

  比丘們!而對未生起之掉舉後悔的生起,或已生起之掉舉後悔的增大、成滿來說,什麼是食呢?比丘們!有心的不平靜,在那裡,不如理作意的多作,對未生起之掉舉後悔的生起,或已生起之掉舉後悔的增大、成滿,這是食。

  比丘們!而對未生起之疑惑的生起,或已生起之疑惑的增大、成滿來說,什麼是食呢?比丘們!有諸疑惑處之法,在那裡,不如理作意的多作,對未生起之疑惑的生起,或已生起之疑惑的增大、成滿,這是食。

  比丘們!猶如這依食而存續的身體緣於食而住立;無食而不住立。同樣的,比丘們!依食而存續的五蓋緣於食而住立;無食而不住立。

  比丘們!猶如這依食而存續的身體緣於食而住立;無食而不住立。同樣的,比丘們!依食而存續的七覺支緣於食而住立;無食而不住立。

  比丘們!而對未生起之念覺支的生起,或已生起之念覺支的修習圓滿來說,什麼是食呢?比丘們!有\twnr{念覺支處諸法}{315.1},在那裡,如理作意的多作,對未生起之念覺支的生起,或已生起之念覺支的增大、成滿,這是食。

  比丘們!而對未生起之\twnr{擇法覺支}{311.0}的生起,或已生起之擇法覺支的修習圓滿來說,什麼是食呢?比丘們!有諸善不善法,諸有罪過無罪過法,諸下劣勝妙法,諸\twnr{黑白有對比的}{550.0}法,在那裡,如理作意的多作,對未生起之擇法覺支的生起,或已生起之擇法覺支的修習圓滿,這是食。

  比丘們!而對未生起之\twnr{活力覺支}{310.0}的生起,或已生起之活力覺支的修習圓滿來說,什麼是食呢?比丘們!有發勤界、精勤界、努力界,在那裡,如理作意的多作,對未生起之活力覺支的生起,或已生起之活力覺支的修習圓滿,這是食。

  比丘們!而對未生起之\twnr{喜覺支}{312.0}的生起,或已生起之喜覺支的修習圓滿來說,什麼是食呢?比丘們!有諸喜覺支處之法,在那裡,如理作意的多作,對未生起之喜覺支的生起,或已生起之喜覺支的修習圓滿,這是食。

  比丘們!而對未生起之\twnr{寧靜覺支}{313.1}的生起,或已生起之寧靜覺支的修習圓滿來說,什麼是食呢?比丘們!有\twnr{身體的寧靜}{318.0}、\twnr{心的寧靜}{319.0},在那裡,如理作意的多作,對未生起之寧靜覺支的生起,或已生起之寧靜覺支的修習圓滿,這是食。

  比丘們!而對未生起之定覺支的生起,或已生起之定覺支的修習圓滿來說,什麼是食呢?比丘們!有\twnr{止相}{688.0}、不混亂相,在那裡,如理作意的多作,對未生起之定覺支的生起,或已生起之定覺支的修習圓滿,這是食。

  比丘們!而對未生起之\twnr{平靜覺支}{314.0}的生起,或已生起之平靜覺支的修習圓滿來說,什麼是食呢?比丘們!有諸平靜覺支處之法,在那裡,如理作意的多作,對未生起之平靜覺支的生起,或已生起之平靜覺支的修習圓滿,這是食。[\suttaref{SN.46.51}]

  比丘們!猶如這依食而存續的身體緣於食而住立;無食而不住立。同樣的,比丘們!依食而存續的七覺支緣於食而住立;無食而不住立。」



\sutta{3}{3}{戒經}{https://agama.buddhason.org/SN/sn.php?keyword=46.3}
  「\twnr{比丘}{31.0}們!凡那些戒具足、定具足、智具足、解脫具足、\twnr{解脫智見}{27.0}具足的比丘,比丘們!我說:即使那些比丘的看見也是多助益的。比丘們!我說:即使那些比丘的聽聞也是多助益的。比丘們!我說:即使那些比丘的往見也是多助益的。比丘們!我說:即使那些比丘的侍奉也是多助益的。比丘們!我說:即使那些比丘的隨念(回憶)也是多助益的。比丘們!我說:即使跟隨那些比丘的出家也是多助益的。那是什麼原因?比丘們!聽聞像這樣比丘的法後,住於以二種遠離成為遠離者:身遠離與心遠離。那位住於像這樣遠離者隨念、\twnr{隨尋思}{743.0}那個法。

  比丘們!凡在住於像這樣遠離的比丘隨念、隨尋思那個法時,在那時,比丘的\twnr{念覺支}{315.0}被發動,在那時,比丘\twnr{修習}{94.0}念覺支,在那時,比丘的念覺支走到修習圓滿。

  那位住於像這樣念者以慧考察(簡擇)、伺察、來到審慮那個法。比丘們!凡在住於像這樣念的比丘以慧考察、伺察、來到審慮那個法時,在那時,比丘的\twnr{擇法覺支}{311.0}被發動,在那時,比丘修習擇法覺支,在那時,比丘的擇法覺支走到修習圓滿。

  那位以慧考察、伺察、來到審慮那個法者的不退縮的活力被發動。比丘們!凡在比丘以慧考察、伺察、來到審慮那個法的不退縮的活力被發動時,在那時,比丘的\twnr{活力覺支}{310.0}被發動,在那時,比丘修習活力覺支,在那時,比丘的活力覺支走到修習圓滿。

  活力被發動者精神的喜生起。比丘們!凡在活力被發動比丘的精神的喜生起時,在那時,比丘的\twnr{喜覺支}{312.0}被發動,在那時,比丘修習喜覺支,在那時,比丘的喜覺支走到修習圓滿。

  \twnr{意喜}{320.0}者的身變得寧靜(輕安),心也變得寧靜。比丘們!凡在意喜比丘的身變得寧靜、心也變得寧靜時,在那時,比丘的\twnr{寧靜覺支}{313.1}被發動,在那時,比丘修習寧靜覺支,在那時,比丘的寧靜覺支走到修習圓滿。

  身寧靜者、有樂者的心入定。比丘們!凡在身寧靜、有樂比丘的心入定時,比丘的\twnr{定}{182.0}覺支被發動,在那時,比丘修習定覺支,在那時,比丘的定覺支走到修習圓滿。

  那位心像這樣得定者成為\twnr{善旁觀者}{995.0}。比丘們!凡在比丘心像這樣得定者成為善旁觀者時,在那時,比丘的\twnr{平靜覺支}{314.0}被發動,在那時,比丘修習平靜覺支,在那時,比丘的平靜覺支走到修習圓滿。

  比丘們!在七覺支上這樣已修習;在七覺支上這樣已\twnr{多作}{95.0}時,七果、七效益能被預期,哪七果、七效益呢?

  在當生提前達成\twnr{完全智}{489.0}。

  如果在當生未提前達成完全智,那麼在死時達成完全智。

  如果在當生未提前達成完全智,如果在死時未達成完全智,那麼以\twnr{五下分結}{134.0}的滅盡,成為\twnr{中般涅槃}{297.0}者。

  如果在當生未提前達成完全智,如果在死時未達成完全智,如果以五下分結的滅盡,未成為\twnr{中般涅槃}{297.0}者,那麼以五下分結的滅盡,成為\twnr{生般涅槃}{298.0}者。

  如果在當生未提前達成完全智,如果在死時未達成完全智,如果以五下分結的滅盡,未成為中般涅槃者,如果以五下分結的滅盡,未成為生般涅槃者,那麼以五下分結的滅盡,成為\twnr{無行般涅槃}{299.0}者。

  如果在當生未提前達成完全智,如果在死時未達成完全智,如果以五下分結的滅盡,未成為中般涅槃者,如果以五下分結的滅盡,未成為生般涅槃者,如果以五下分結的滅盡,未成為無行般涅槃者,那麼以五下分結的滅盡,成為\twnr{有行般涅槃}{300.0}者。

  如果在當生未提前達成完全智,如果在死時未達成完全智,如果以五下分結的滅盡,未成為中般涅槃者,如果以五下分結的滅盡,未成為生般涅槃者,如果以五下分結的滅盡,未成為無行般涅槃者,如果以五下分結的滅盡,未成為有行般涅槃者,那麼以五下分結的滅盡,成為\twnr{上流到阿迦膩吒}{301.0}者。

  比丘們!在七覺支這樣已修習;七覺支這樣已多作時,這些七果、七效益能被預期。」



\sutta{4}{4}{衣服經}{https://agama.buddhason.org/SN/sn.php?keyword=46.4}
  \twnr{有一次}{2.0},\twnr{尊者}{200.0}舍利弗住在舍衛城祇樹林給孤獨園。

  在那裡,尊者舍利弗召喚\twnr{比丘}{31.0}們:「比丘學友們!」

  「\twnr{學友}{201.0}!」那些比丘回答尊者舍利弗。

  尊者舍利弗說這個:

  「學友們!有這七覺支,哪七個?\twnr{念覺支}{315.0}、\twnr{擇法覺支}{311.0}、\twnr{活力覺支}{310.0}、\twnr{喜覺支}{312.0}、\twnr{寧靜覺支}{313.1}、定覺支、\twnr{平靜覺支}{314.0},學友們!這些是七覺支。

  學友們!這七覺支中,午前時我希望以哪個覺支住,午前時我以那個覺支住;中午時我希望以哪個覺支住,中午時我以那個覺支住;傍晚時我希望以哪個覺支住,傍晚時我以那個覺支住。

  學友們!如果我心想『\twnr{念覺支}{315.0}』,我心想『無量的』,我心想『善努力的』,而當它存續時,我知道:『它存續。』即使如果它從我這裡消失,我知道:『它以\twnr{特定條件}{636.0}從我這裡消失。』……(中略)如果我心想『\twnr{平靜覺支}{314.0}』,我心想『無量的』,我心想『善努力的』,而當它存續時,我知道:『它存續。』即使如果它從我這裡消失,我知道:『它以特定條件從我這裡消失。』

  學友們!猶如國王或國王的大臣有充滿種種染色衣服的衣箱,不論他希望哪套衣服午前時穿,午前時他就穿那套衣服;不論他希望哪套衣服中午時穿,中午時他就穿那套衣服;不論他希望哪套衣服傍晚時穿,傍晚時他就穿那套衣服。同樣的,學友們!這七覺支中,午前時我希望以哪個覺支住,午前時我以那個覺支住;中午時我希望以哪個覺支住,中午時我以那個覺支住;傍晚時我希望以哪個覺支住,傍晚時我以那個覺支住。

  學友們!如果我心想『念覺支』,我心想『無量的』,我心想『善努力的』,而當它存續時,我知道:『它存續。』即使如果它從我這裡消失,我知道:『它以特定條件從我這裡消失。』……(中略)如果我心想『平靜覺支』,我心想『無量的』,我心想『善努力的』,而當它存續時,我知道:『它存續。』即使如果它從我這裡消失,我知道:『它以特定條件從我這裡消失。』」



\sutta{5}{5}{比丘經}{https://agama.buddhason.org/SN/sn.php?keyword=46.5}
  起源於舍衛城。

  那時,\twnr{某位比丘}{39.0}去見\twnr{世尊}{12.0}。……(中略)在一旁坐下的那位比丘對世尊說這個:

  「\twnr{大德}{45.0}!被稱為『覺支、覺支』,大德!什麼情形被稱為『覺支』呢?」

  「比丘!『導向覺』,因此被稱為『覺支』。比丘!這裡,\twnr{依止遠離}{322.0}、依止離貪、依止滅、\twnr{捨棄的成熟}{221.0}修習\twnr{念覺支}{315.0}……(中略)依止遠離、依止離貪、依止滅、捨棄的成熟修習\twnr{平靜覺支}{314.0}。那位修習這七覺支者的心從欲漏被解脫,心也從有漏被解脫,心也從\twnr{無明漏}{397.0}被解脫。在已解脫時,\twnr{有『[這是]解脫』之智}{27.0},他知道:『\twnr{出生已盡}{18.0},\twnr{梵行已完成}{19.0},\twnr{應該被作的已作}{20.0},\twnr{不再有此處[輪迴]的狀態}{21.1}。』比丘!『導向覺』,因此被稱為『覺支』。」



\sutta{6}{6}{庫達利亞經}{https://agama.buddhason.org/SN/sn.php?keyword=46.6}
  \twnr{被我這麼聽聞}{1.0}:

  \twnr{有一次}{2.0},\twnr{世尊}{12.0}住在娑雞多城漆黑林的鹿園。

  那時,\twnr{遊行者}{79.0}庫達利亞去見世尊。抵達後,與世尊一起互相問候。交換應該被互相問候的友好交談後,在一旁坐下。在一旁坐下的遊行者庫達利亞對世尊說這個:

  「\twnr{喬達摩}{80.0}尊師!我是寺院的依止者,\twnr{來往各團體者}{x598}。喬達摩尊師!這是那個我吃過早餐後的[慣]行:從園林到園林、從遊園到遊園散步、漫步。在那裡,那個我看見一些只說\twnr{像那樣自由說話效益}{905.0}與\twnr{詰難效益}{903.0}談論的\twnr{沙門}{29.0}、\twnr{婆羅門}{17.0}。那麼,喬達摩\twnr{尊師}{203.0}住於什麼效益呢?」

  「庫達利亞!如來住於明解脫果效益。」

  「喬達摩尊師!那麼,哪些法已\twnr{修習}{94.0}、已\twnr{多作}{95.0}使明解脫完成呢?」

  「庫達利亞!\twnr{七覺支}{524.0}已修習、已多作使明解脫完成。」

  「喬達摩尊師!那麼,哪些法已修習、已多作使七覺支完成呢?」

  「庫達利亞!\twnr{四念住}{286.0}已修習、已多作使七覺支完成。」

  「喬達摩尊師!那麼,哪些法已修習、已多作使四念住完成呢?」

  「庫達利亞!三善行已修習、已多作使四念住完成。」

  「喬達摩尊師!那麼,哪些法已修習、已多作使三善行完成呢?」

  「庫達利亞!根的\twnr{自制}{217.0}已修習、已多作使三善行完成。」

  「庫達利亞!而根的自制怎樣已修習、怎樣已多作使三善行完成呢?庫達利亞!這裡,\twnr{比丘}{31.0}以眼見合意色後,不貪求、不喜、不使貪產生,且他的身是住立的、心是住立的,自身內是善安頓的、\twnr{善解脫}{28.0}的。還有,以眼見不合意色後,不成為氣餒的、心不住立的、意不可憐的、心無瞋害的,且他的身是住立的、心是住立的,自身內是善安頓的、善解脫的。

  庫達利亞!再者,比丘以耳聽聲音後……(中略)以鼻聞氣味後……(中略)以舌嚐味道後……以身觸\twnr{所觸}{220.2}後……以意識知合意法後,不貪求、不喜、不使貪產生,且他的身是住立的、心是住立的,自身內是善安頓的、善解脫的。還有,以意識知不合意法後,不成為氣餒的、心不住立的、意不可憐的、心無瞋害的,且他的身是住立的、心是住立的,自身內是善安頓的、善解脫的。

  庫達利亞!且當以眼見色後,在合意與不合意色上比丘的身是住立的、心是住立的,自身內是善安頓的、善解脫的;以耳聽聲音後……(中略)以鼻聞氣味後……(中略)以舌嚐味道後……(中略)以身觸所觸後……(中略)以意識知法後,在合意與不合意法上比丘的身是住立的、心是住立的,自身內是善安頓的、善解脫的。庫達利亞!根的自制這樣已修習、這樣已多作使三善行完成。

  庫達利亞!而三善行怎樣已修習、怎樣已多作使四念住完成呢?庫達利亞!這裡,比丘捨斷身惡行後,修習身善行;捨斷語惡行後,修習語善行;捨斷意惡行後,修習意善行。庫達利亞!三善行這樣已修習、這樣已多作使四念住完成。

  庫達利亞!而四念住怎樣已修習、怎樣已多作使七覺支完成呢?庫達利亞!這裡,比丘住於\twnr{在身上隨看著身}{176.0}:熱心的、正知的、有念的,調伏世間中的\twnr{貪婪}{435.0}、憂後……(中略)住於在諸法上隨看著法:熱心的、正知的、有念的,調伏世間中的貪婪、憂後。庫達利亞!這樣,四念住已修習、這樣已多作使七覺支完成。

  庫達利亞!而七覺支怎樣已修習、怎樣已多作使明與解脫完成呢?庫達利亞!這裡,比丘\twnr{依止遠離}{322.0}、依止離貪、依止滅、\twnr{捨棄的成熟}{221.0}修習\twnr{念覺支}{315.0}……(中略)依止遠離、依止離貪、依止滅、捨棄的成熟修習\twnr{平靜覺支}{314.0}。庫達利亞!這樣,七覺支已修習、這樣已多作使明與解脫完成。」

  在這麼說時,遊行者庫達利亞對世尊說這個:

  「太偉大了,喬達摩尊師!太偉大了,喬達摩尊師!喬達摩尊師!猶如扶正顛倒的,或揭開隱藏的,或告知迷路者的道路,或在黑暗中持燈火:『有眼者們看見諸色。』同樣的,法被喬達摩尊師以種種\twnr{法門}{562.0}說明。這個我\twnr{歸依}{284.0}世尊、法、\twnr{比丘僧團}{65.0},請喬達摩尊師記得我為\twnr{優婆塞}{98.0},從今天起\twnr{已終生歸依}{64.0}。」



\sutta{7}{7}{重閣經}{https://agama.buddhason.org/SN/sn.php?keyword=46.7}
  「猶如凡任何\twnr{重閣}{213.0}的\twnr{椽}{663.0},那些全部是傾向屋頂的、斜向屋頂的、坡斜向屋頂的。同樣的,\twnr{比丘}{31.0}們!\twnr{修習}{94.0}\twnr{七覺支}{524.0}、\twnr{多作}{95.0}七覺支的比丘是傾向涅槃的、斜向涅槃的、坡斜向涅槃的。

  比丘們!而怎樣修習七覺支、多作七覺支的比丘是傾向涅槃的、斜向涅槃的、坡斜向涅槃的?比丘們!這裡,比丘\twnr{依止遠離}{322.0}、依止離貪、依止滅、\twnr{捨棄的成熟}{221.0}修習\twnr{念覺支}{315.0}……(中略)依止遠離、依止離貪、依止滅、捨棄的成熟修習\twnr{平靜覺支}{314.0}。

  比丘們!這樣修習七覺支、多作七覺支的比丘是傾向涅槃的、斜向涅槃的、坡斜向涅槃的。」



\sutta{8}{8}{優波瓦那經}{https://agama.buddhason.org/SN/sn.php?keyword=46.8}
  \twnr{有一次}{2.0},\twnr{尊者}{200.0}優波瓦那與尊者舍利弗住在\twnr{憍賞彌}{140.0}瞿師羅園。

  那時,尊者舍利弗傍晚時,從\twnr{獨坐}{92.0}出來,去見尊者優波瓦那。抵達後,與尊者優波瓦那一起互相問候。交換應該被互相問候的友好交談後,在一旁坐下。在一旁坐下的尊者舍利弗對尊者優波瓦那說這個:

  「優波瓦那\twnr{學友}{201.0}!\twnr{比丘}{31.0}各自能知道:『以\twnr{如理作意}{114.0},被我這麼\twnr{善努力}{682.0}的七覺支導向\twnr{安樂住}{156.0}。』嗎?」

  「舍利弗學友!比丘各自能知道:『以如理作意,被我這麼善努力的七覺支導向安樂住。』

  學友!\twnr{念覺支}{315.0}的比丘知道:『我的心已\twnr{善解脫}{28.0}、我的惛沈睡眠已善根除、我的掉舉後悔已善排除、我的活力已發動,\twnr{作目標後}{316.0}我作意而不退縮。』……(中略)學友!發動\twnr{平靜覺支}{314.0}的比丘知道:『我的心已善解脫、我的惛沈睡眠已善根除、我的掉舉後悔已善排除、我的活力已發動,作目標後我作意而不退縮。』

  舍利弗學友!比丘這麼各自能知道:『以如理作意,被我這麼善努力的七覺支導向安樂住。』」



\sutta{9}{9}{已生起經第一}{https://agama.buddhason.org/SN/sn.php?keyword=46.9}
  「\twnr{比丘}{31.0}們!有這些\twnr{七覺支}{524.0},已\twnr{修習}{94.0}、已\twnr{多作}{95.0},未生起的不離\twnr{如來}{4.0}、\twnr{阿羅漢}{5.0}、\twnr{遍正覺者}{6.0}的出現而生起,哪七個?\twnr{念覺支}{315.0}……(中略)\twnr{平靜覺支}{314.0}。比丘們!這些是七覺支,已修習、已多作,未生起的不離如來、阿羅漢、遍正覺者的出現而生起。」



\sutta{10}{10}{已生起經第二}{https://agama.buddhason.org/SN/sn.php?keyword=46.10}
  「\twnr{比丘}{31.0}們!有這些\twnr{七覺支}{524.0},已\twnr{修習}{94.0}、已\twnr{多作}{95.0},未生起的不離\twnr{善逝}{8.0}之律生起,哪七個?\twnr{念覺支}{315.0}……(中略)\twnr{平靜覺支}{314.0}。比丘們!這些是七覺支,已修習、已多作,未生起的不離善逝之律生起。」

  山品第一,其\twnr{攝頌}{35.0}:

  「喜馬拉雅山、身體、戒,衣服、比丘與庫達利亞,

   重閣與優波瓦那,已生起二則在後。」





\pin{病品}{11}{20}
\sutta{11}{11}{生類經}{https://agama.buddhason.org/SN/sn.php?keyword=46.11}
  「\twnr{比丘}{31.0}們!猶如凡任何生類從事四種舉止行為:有時行,有時立,有時坐,有時臥,他們全部依止於土地後,住立於土地後,這樣,從事這四種舉止行為。同樣的,比丘們!比丘依止戒後,住立於戒後,\twnr{修習}{94.0}\twnr{七覺支}{524.0}、\twnr{多作}{95.0}七覺支。

  比丘們!而怎樣比丘依止戒後,住立於戒後,修習七覺支、多作七覺支?比丘們!這裡,比丘\twnr{依止遠離}{322.0}、依止離貪、依止滅、\twnr{捨棄的成熟}{221.0}修習\twnr{念覺支}{315.0}……(中略)依止遠離、依止離貪、依止滅、捨棄的成熟修習\twnr{平靜覺支}{314.0}。比丘們!這樣,比丘依止戒後,住立於戒後,修習七覺支、多作七覺支。」



\sutta{12}{12}{如太陽經第一}{https://agama.buddhason.org/SN/sn.php?keyword=46.12}
  「\twnr{比丘}{31.0}們!對太陽的昇起,這是先導,這是前兆,即:黎明。同樣的,比丘們!對比丘\twnr{七覺支}{524.0}的生起,這是先導,這是前兆,即:善友誼。

  比丘們!有善友比丘的這個能被預期:他必將\twnr{修習}{94.0}七覺支,他必將\twnr{多作}{95.0}七覺支。

  比丘們!而怎樣有善友的比丘修習七覺支、多作七覺支?比丘們!這裡,比丘\twnr{依止遠離}{322.0}、依止離貪、依止滅、\twnr{捨棄的成熟}{221.0}修習\twnr{念覺支}{315.0}……(中略)依止遠離、依止離貪、依止滅、捨棄的成熟修習\twnr{平靜覺支}{314.0}。

  比丘們!這樣,有善友的比丘修習七覺支、多作七覺支。」[\suttaref{SN.46.48}]



\sutta{13}{13}{如太陽經第二}{https://agama.buddhason.org/SN/sn.php?keyword=46.13}
  「\twnr{比丘}{31.0}們!對太陽的昇起,這是先導,這是前兆,即:黎明。同樣的,比丘們!對比丘\twnr{七覺支}{524.0}的生起,這是先導,這是前兆,即:\twnr{如理作意}{114.0}具足。

  比丘們!如理作意具足比丘的這個能被預期:他必將\twnr{修習}{94.0}七覺支,他必將\twnr{多作}{95.0}七覺支。

  比丘們!而怎樣如理作意具足的比丘修習七覺支、多作七覺支?比丘們!這裡,比丘\twnr{依止遠離}{322.0}、依止離貪、依止滅、\twnr{捨棄的成熟}{221.0}修習\twnr{念覺支}{315.0}……(中略)依止遠離、依止離貪、依止滅、捨棄的成熟修習\twnr{平靜覺支}{314.0}。

  比丘們!這樣,如理作意具足的比丘修習七覺支、多作七覺支。」



\sutta{14}{14}{病經第一}{https://agama.buddhason.org/SN/sn.php?keyword=46.14}
  \twnr{有一次}{2.0},\twnr{世尊}{12.0}住在王舍城栗鼠飼養處的竹林中。

  當時,生病的、受苦的、重病的\twnr{尊者}{200.0}大迦葉住在畢波里洞穴。

  那時,世尊傍晚時,從\twnr{獨坐}{92.0}出來,去見尊者大迦葉。抵達後,在設置的座位坐下。坐下後,世尊對尊者摩大迦葉說這個:

  「迦葉!是否能被你忍受?\twnr{是否能被[你]維持生活}{137.0}?是否苦的感受減退、不增進,減退的結局被知道,非增進?」

  「\twnr{大德}{45.0}!不能被我忍受,不能被[我]維持,我強烈苦的感受增進、不減退,增進的結局被知道,非減退。」

  「迦葉!這些被我正確告知的\twnr{七覺支}{524.0}已\twnr{修習}{94.0}、已\twnr{多作}{95.0},轉起證智、\twnr{正覺}{185.1}、涅槃,哪七個?迦葉!被我正確告知的\twnr{念覺支}{315.0}已修習、已多作,轉起證智、正覺、涅槃……(中略)被我正確告知的\twnr{平靜覺支}{314.0}已修習、已多作,轉起證智、正覺、涅槃,迦葉!這些被我正確告知的七覺支已修習、已多作,轉起證智、正覺、涅槃。」

  「大德!確實是覺支,大德!確實是覺支。」

  世尊說這個,悅意的尊者大迦葉歡喜世尊的所說。尊者大迦葉的病因此痊癒,以及尊者大迦葉的那個病像這樣被捨斷。



\sutta{15}{15}{病經第二}{https://agama.buddhason.org/SN/sn.php?keyword=46.15}
  \twnr{有一次}{2.0},\twnr{世尊}{12.0}住在王舍城栗鼠飼養處的竹林中。

  當時,生病的、受苦的、重病的\twnr{尊者}{200.0}大目揵連住在\twnr{耆闍崛山}{258.0}。

  那時,世尊傍晚時,從\twnr{獨坐}{92.0}出來,去見尊者大目揵連。抵達後,在設置的座位坐下。坐下後,世尊對尊者摩大目揵連說這個:

  「目揵連!是否能被你忍受?\twnr{是否能被[你]維持生活}{137.0}?是否苦的感受減退、不增進,減退的結局被知道,非增進?」

  「\twnr{大德}{45.0}!不能被我忍受,不能被[我]維持,我強烈苦的感受增進、不減退,增進的結局被知道,非減退。」

  「目揵連!這些被我正確告知的\twnr{七覺支}{524.0}已\twnr{修習}{94.0}、已\twnr{多作}{95.0},轉起證智、\twnr{正覺}{185.1}、涅槃,哪七個?目揵連!被我正確告知的\twnr{念覺支}{315.0}已修習、已多作,轉起證智、正覺、涅槃……(中略)被我正確告知的\twnr{平靜覺支}{314.0}已修習、已多作,轉起證智、正覺、涅槃,目揵連!這些被我正確告知的七覺支已修習、已多作,轉起證智、正覺、涅槃。」

  「大德!確實是覺支,大德!確實是覺支。」

  世尊說這個,悅意的尊者大目揵連歡喜世尊的所說。尊者大目揵連的病因此痊癒,以及尊者大目揵連的那個病像這樣被捨斷。



\sutta{16}{16}{病經第三}{https://agama.buddhason.org/SN/sn.php?keyword=46.16}
  \twnr{有一次}{2.0},\twnr{世尊}{12.0}住在王舍城栗鼠飼養處的竹林中。

  當時,世尊是生病者、受苦者、重病者。

  那時,\twnr{尊者}{200.0}摩訶純陀去見世尊。抵達後,向世尊\twnr{問訊}{46.0}後,在一旁坐下。世尊對在一旁坐下的尊者摩訶純陀說這個:

  「純陀!請你背誦覺支。」

  「\twnr{大德}{45.0}!這些被世尊正確告知的\twnr{七覺支}{524.0}已\twnr{修習}{94.0}、已\twnr{多作}{95.0},轉起證智、\twnr{正覺}{185.1}、涅槃,哪七個?大德!被世尊正確告知的\twnr{念覺支}{315.0}已修習、已多作,轉起證智、正覺、涅槃……(中略)被世尊正確告知的\twnr{平靜覺支}{314.0}已修習、已多作,轉起證智、正覺、涅槃,大德!這些被世尊正確告知的七覺支已修習、已多作,轉起證智、正覺、涅槃。」

  「純陀!確實是覺支,純陀!確實是覺支。」

  尊者純陀說這個,大師是認可者。

  世尊的病的病因此痊癒,以及世尊的那個病像這樣被捨斷。



\sutta{17}{17}{到彼岸經}{https://agama.buddhason.org/SN/sn.php?keyword=46.17}
  「\twnr{比丘}{31.0}們!有這些\twnr{七覺支}{524.0},已\twnr{修習}{94.0}、已\twnr{多作}{95.0},轉起從此岸走到\twnr{彼岸}{226.0},哪七個?\twnr{念覺支}{315.0}……(中略)\twnr{平靜覺支}{314.0},比丘們!這些是七覺支,已修習、已多作,轉起從此岸走到彼岸。」

  世尊說這個,說這個後,\twnr{善逝}{8.0}、\twnr{大師}{145.0}又更進一步說這個:

  「在那些人中是少的:到彼岸的人,

   其他人還,只跟隨這個岸邊跑。

   而凡在法被正確告知時,在法上順從法者,

   那些人將去彼岸:極難越過的死神界。

   賢智者捨棄黑法後,請你們修習白的,

   從家來到無家後,在難喜樂之處遠離。

   在那裡應該尋求喜樂:無所有者捨棄諸欲後,

   賢智者應該淨化自己:從心的污染。

   凡在諸正覺支上,他們的心已正確地修習者,

   在執取的\twnr{斷念}{211.0}上,凡無取著後已樂者,

   諸漏已滅盡的有光輝者,他們是世間中的\twnr{證涅槃}{71.0}者。」[\suttaref{SN.45.34}, \ccchref{AN.10.117}{https://agama.buddhason.org/AN/an.php?keyword=10.117}]



\sutta{18}{18}{已錯失經}{https://agama.buddhason.org/SN/sn.php?keyword=46.18}
  「\twnr{比丘}{31.0}們!凡任何已錯失\twnr{七覺支}{524.0}者,他們導向\twnr{苦的完全滅盡}{181.0}的聖道已錯失;比丘們!凡任何已發動七覺支者,他們導向苦的完全滅盡的聖道已發動。比丘們!哪七個?\twnr{念覺支}{315.0}……(中略)\twnr{平靜覺支}{314.0}。比丘們!凡任何已錯失七覺支者,他們導向苦的完全滅盡的聖道已錯失;比丘們!凡任何已發動七覺支者,他們導向苦的完全滅盡的聖道已發動。」



\sutta{19}{19}{聖經}{https://agama.buddhason.org/SN/sn.php?keyword=46.19}
  「\twnr{比丘}{31.0}們!有這\twnr{七覺支}{524.0},已\twnr{修習}{94.0}、已\twnr{多作}{95.0},是聖的、\twnr{出離的}{294.0},引導那樣行為者\twnr{苦的完全滅盡}{181.0},哪七個?\twnr{念覺支}{315.0}……(中略)\twnr{平靜覺支}{314.0}。比丘們!這些是七覺支,已修習、已多作,是聖的、出離的,引導那樣行為者苦的完全滅盡。」



\sutta{20}{20}{厭經}{https://agama.buddhason.org/SN/sn.php?keyword=46.20}
  「\twnr{比丘}{31.0}們!有這\twnr{七覺支}{524.0},已\twnr{修習}{94.0}、已\twnr{多作}{95.0},轉起\twnr{一向}{168.0}\twnr{厭}{15.0}、\twnr{離貪}{77.0}、\twnr{滅}{68.0}、寂靜、證智、\twnr{正覺}{185.1}、涅槃,哪七個?\twnr{念覺支}{315.0}……(中略)\twnr{平靜覺支}{314.0}。比丘們!這些是七覺支,已修習、已多作,對一向的厭、對離貪、對滅、對寂靜、對證智、對正覺、對涅槃轉起。」

  病品第二,其\twnr{攝頌}{35.0}:

  「生類、如太陽二則,病隨後三則,

   到彼岸、已錯失,以及聖、厭。」





\pin{優陀夷品}{21}{30}
\sutta{21}{21}{向覺經}{https://agama.buddhason.org/SN/sn.php?keyword=46.21}
  那時,\twnr{某位比丘}{39.0}去見\twnr{世尊}{12.0}。……(中略)在一旁坐下的那位比丘對世尊說這個:

  「\twnr{大德}{45.0}!被稱為『覺支、覺支』,大德!什麼情形被稱為『覺支』呢?」

  「比丘!『導向覺』,因此被稱為『覺支』。比丘!這裡,\twnr{依止遠離}{322.0}、依止離貪、依止滅、\twnr{捨棄的成熟}{221.0}修習\twnr{念覺支}{315.0}……(中略)依止遠離、依止離貪、依止滅、捨棄的成熟修習\twnr{平靜覺支}{314.0}。比丘!『導向覺』,因此被稱為『覺支』。」



\sutta{22}{22}{覺支之教導經}{https://agama.buddhason.org/SN/sn.php?keyword=46.22}
  「\twnr{比丘}{31.0}們!我將為你們教導七覺支,\twnr{你們要聽}{43.0}它!

  比丘們!而什麼是七覺支呢?\twnr{念覺支}{315.0}……(中略)\twnr{平靜覺支}{314.0},比丘們!這些是七覺支。」



\sutta{23}{23}{處經}{https://agama.buddhason.org/SN/sn.php?keyword=46.23}
  「\twnr{比丘}{31.0}們!以\twnr{對欲貪處之法}{x599}的作意多作(經常作意),未生起之\twnr{欲的意欲}{118.0}就生起,且已生起之欲的意欲轉起增大、成滿;以對惡意處之法的作意多作,未生起的惡意就生起,且已生起的惡意轉起增大、成滿;以對惛沈睡眠處之法的作意多作,未生起的惛沈睡眠就生起,且已生起的惛沈睡眠轉起增大、成滿;以對掉舉後悔處之法的作意多作,未生起的掉舉後悔就生起,且已生起的掉舉後悔轉起增大、成滿;以對疑惑處之法的作意多作,未生起的疑惑就生起,且已生起的疑惑轉起增大、成滿。

  比丘們!以對\twnr{念覺支處諸法}{315.1}的作意多作,未生起的念覺支就生起,且已生起的念覺支走到修習圓滿……(中略)比丘們!以對\twnr{平靜覺支}{314.0}處之法的作意多作,未生起的平靜覺支就生起,且已生起的平靜覺支走到修習圓滿。」



\sutta{24}{24}{不如理作意經}{https://agama.buddhason.org/SN/sn.php?keyword=46.24}
  「\twnr{比丘}{31.0}們!對不\twnr{如理作意}{114.0}者,未生起的\twnr{欲的意欲}{118.0}生起,連同已生起之欲的意欲轉起增大、成滿;未生起的惡意生起,連同已生起的惡意轉起增大、成滿;未生起的惛沈睡眠生起,連同已生起的惛沈睡眠轉起增大、成滿;未生起的掉舉後悔生起,連同已生起的掉舉後悔轉起增大、成滿;未生起的疑惑生起,連同已生起的疑惑轉起增大、成滿;未生起的念覺支不生起,連同已生起的\twnr{念覺支}{315.0}被滅……(中略)未生起的\twnr{平靜覺支}{314.0}不生起,連同已生起的平靜覺支被滅。

  比丘們!但對如理作意者,未生起的欲的意欲不生起,連同已生起的欲的意欲被捨斷;未生起的惡意不生起,連同已生起的惡意被捨斷;未生起的惛沈睡眠不生起,連同已生起的惛沈睡眠被捨斷;未生起的掉舉後悔不生起,連同已生起的掉舉後悔被捨斷;未生起的疑惑不生起,連同已生起的疑惑被捨斷;未生起的念覺支生起,連同已生起的念覺支走到修習圓滿……(中略)未生起的平靜覺支生起,連同已生起的平靜覺支走到修習圓滿。」



\sutta{25}{25}{不衰退經}{https://agama.buddhason.org/SN/sn.php?keyword=46.25}
  「\twnr{比丘}{31.0}們!我將為你們教導七不衰退法,\twnr{你們要聽}{43.0}它!比丘們!而什麼是七不衰退法呢?即:\twnr{七覺支}{524.0},哪七個?\twnr{念覺支}{315.0}……(中略)\twnr{平靜覺支}{314.0}。比丘們!這些是七不衰退法。」



\sutta{26}{26}{渴愛之滅盡經}{https://agama.buddhason.org/SN/sn.php?keyword=46.26}
  「\twnr{比丘}{31.0}們!凡道、凡道跡轉起渴愛之滅盡者,你們要\twnr{修習}{94.0}該道、該道跡!

  比丘們!而什麼是轉起渴愛之滅盡的道與道跡呢?即:\twnr{七覺支}{524.0},哪七個?\twnr{念覺支}{315.0}……(中略)\twnr{平靜覺支}{314.0}。」

  在這麼說時,\twnr{尊者}{200.0}優陀夷對\twnr{世尊}{12.0}說這個:

  「\twnr{大德}{45.0}!七覺支怎樣已\twnr{修習}{94.0}、怎樣已\twnr{多作}{95.0},轉起渴愛之滅盡呢?」

  「優陀夷!這裡,比丘\twnr{依止遠離}{322.0}、依止離貪、依止滅、\twnr{捨棄的成熟}{221.0}、廣大的、大的、無量的、無惡意的修習\twnr{念覺支}{315.0},那位依止遠離、依止離貪、依止滅、捨棄的成熟、廣大的、大的、無量的、無惡意的修習念覺支者的渴愛被捨斷;以渴愛的捨斷,業被捨斷;以業的捨斷,苦被捨斷……(中略)依止遠離、依止離貪、依止滅、捨棄的成熟、廣大的、大的、無量的、無惡意的修習\twnr{平靜覺支}{314.0},那位依止遠離、依止離貪、依止滅、捨棄的成熟、廣大的、大的、無量的、無惡意的修習平靜覺支者的渴愛被捨斷;以渴愛的捨斷,業被捨斷;以業的捨斷,苦被捨斷。優陀夷!像這樣,以渴愛的滅盡,有業的滅盡;以業的滅盡,有苦的滅盡。」



\sutta{27}{27}{渴愛之滅經}{https://agama.buddhason.org/SN/sn.php?keyword=46.27}
  「\twnr{比丘}{31.0}們!你們要\twnr{修習}{94.0}轉起渴愛之滅的道與道跡!

  比丘們!而什麼是轉起渴愛之滅的道與道跡呢?即:\twnr{七覺支}{524.0},哪七個?\twnr{念覺支}{315.0}……(中略)\twnr{平靜覺支}{314.0}。

  比丘們!七覺支怎樣已\twnr{修習}{94.0}、怎樣已\twnr{多作}{95.0},轉起渴愛之滅?比丘們!這裡,比丘\twnr{依止遠離}{322.0}、依止離貪、依止滅、\twnr{捨棄的成熟}{221.0}修習念覺支……(中略)依止遠離、依止離貪、依止滅、捨棄的成熟修習\twnr{平靜覺支}{314.0},比丘們!這樣已修習、這樣已多作七覺支,導向渴愛之滅。」



\sutta{28}{28}{洞察分經}{https://agama.buddhason.org/SN/sn.php?keyword=46.28}
  「\twnr{比丘}{31.0}們!我將為你們教導\twnr{洞察分}{853.0}之道與道跡,\twnr{你們要聽}{43.0}它!比丘們!而什麼是洞察分之道呢?即:\twnr{七覺支}{524.0},哪七個?\twnr{念覺支}{315.0}……(中略)\twnr{平靜覺支}{314.0}。」

  在這麼說時,\twnr{尊者}{200.0}優陀夷對\twnr{世尊}{12.0}說這個:

  「\twnr{大德}{45.0}!七覺支怎樣已\twnr{修習}{94.0}、怎樣已\twnr{多作}{95.0},轉起洞察呢?」

  「優陀夷!這裡,比丘\twnr{依止遠離}{322.0}、依止離貪、依止滅、\twnr{捨棄的成熟}{221.0}、廣大的、大的、無量的、無惡意的修習\twnr{念覺支}{315.0},他以已修習念覺支之心洞察、碎破以前未洞察、以前未碎破的貪聚(貪蘊);洞察、碎破以前未洞察、以前未碎破的瞋恚聚;洞察、碎破以前未洞察、以前未碎破的愚癡聚……(中略)依止遠離、依止離貪、依止滅、捨棄的成熟、廣大的、大的、無量的、無惡意的修習\twnr{平靜覺支}{314.0},他以已修習平靜覺支之心洞察、碎破以前未洞察、以前未碎破的貪聚;洞察、碎破以前未洞察、以前未碎破的瞋恚聚;洞察、碎破以前未洞察、以前未碎破的愚癡聚。優陀夷!這樣已修習、這樣已多作七覺支,導向洞察。」 



\sutta{29}{29}{一法經}{https://agama.buddhason.org/SN/sn.php?keyword=46.29}
  「\twnr{比丘}{31.0}們!我不見還有其它\twnr{一法}{522.0},凡這樣已\twnr{修習}{94.0}、這樣已\twnr{多作}{95.0},轉起會被結縛的諸法之捨斷,比丘們!如這七覺支,哪七個?\twnr{念覺支}{315.0}……(中略)\twnr{平靜覺支}{314.0}。

  比丘們!而七覺支怎樣已修習、怎樣已多作,轉起會被結縛的諸法之捨斷?比丘們!這裡,比丘\twnr{依止遠離}{322.0}、依止離貪、依止滅、\twnr{捨棄的成熟}{221.0}修習念覺支……(中略)依止遠離、依止離貪、依止滅、捨棄的成熟修習\twnr{平靜覺支}{314.0}。

  比丘們!而什麼是會被結縛的諸法?比丘們!眼是會被結縛的諸法,在這裡,這些結、繫縛、取著生起……(中略)舌是會被結縛的諸法,在這裡,這些結、繫縛、取著生起……(中略)意是會被結縛的諸法,在這裡,這些結、繫縛、取著在起,比丘們!這些被稱為會被結縛的法。」



\sutta{30}{30}{優陀夷經}{https://agama.buddhason.org/SN/sn.php?keyword=46.30}
  \twnr{有一次}{2.0},\twnr{世尊}{12.0}住在孫玻,名叫謝達葛的孫玻人城鎮。那時,\twnr{尊者}{200.0}優陀夷去見\twnr{世尊}{12.0}……(中略)在一旁坐下的尊者優陀夷對世尊說這個:

  「不可思議啊,\twnr{大德}{45.0}!\twnr{未曾有}{206.0}啊,大德!大德!在世尊上我的情愛、尊重、\twnr{慚}{250.0}、\twnr{愧}{251.0}是多麼多所作的,大德!因為,當以前是在家人時,我是非被法、\twnr{僧團}{375.0}感動者,當那個我考慮在世尊上的情愛、尊重、慚、愧時,\twnr{從在家出家成為無家者}{48.0},世尊為那個我教導法:『像這樣是色,像這樣是色的\twnr{集}{67.0},像這樣是色的滅沒;像這樣是受……(中略)像這樣是想……像這樣是諸行……像這樣是識,像這樣是識的集,像這樣是識的滅沒。

  大德!那個到空屋的我,當考察這些五取蘊的\twnr{立起與彎下}{x600}時,如實證知:『這是苦。』如實證知:『這是苦集。』如實證知:『這是苦滅。』如實證知:『這是導向苦\twnr{滅道跡}{69.0}。』大德!法被我\twnr{現觀}{53.0},以及道被我得到:凡被我\twnr{修習}{94.0}、\twnr{多作}{95.0},當一一像這樣住時,它將導引\twnr{到那樣的狀態}{x601},如是,我將知道:『\twnr{出生已盡}{18.0},\twnr{梵行已完成}{19.0},\twnr{應該被作的已作}{20.0},\twnr{不再有此處[輪迴]的狀態}{21.1}。』

  大德!\twnr{念覺支}{315.0}被我得到:凡被我修習、多作,當一一像這樣住時,它將導引到那樣的狀態,如是,我將知道:『出生已盡,梵行已完成,應該被作的已作,不再有此處[輪迴]的狀態。』 ……(中略)大德!\twnr{平靜覺支}{314.0}被我得到:凡被我修習、多作,當一一像這樣住時,它將導引到那樣的狀態,如是,我將知道:『出生已盡,梵行已完成,應該被作的已作,不再有此處[輪迴]的狀態。』大德!這個道被我得到:凡被我修習、多作,當一一像這樣住時,它將導引到那樣的狀態,如是,我將知道:『出生已盡,梵行已完成,應該被作的已作,不再有此處[輪迴]的狀態。』」

  「\twnr{好}{44.0}!好!優陀夷!優陀夷!這個道被你得到:凡被你修習、多作,當一一像這樣住時,它將導引到那樣的狀態,如是,你將知道:『出生已盡,梵行已完成,應該被作的已作,不再有此處[輪迴]的狀態。』」

  優陀夷品第三,其\twnr{攝頌}{35.0}:

  「向覺、教導、處,不如理與不衰退,

   滅盡、滅、洞察,一法、優陀夷。」





\pin{蓋品}{31}{40}
\sutta{31}{31}{善經第一}{https://agama.buddhason.org/SN/sn.php?keyword=46.31}
  「\twnr{比丘}{31.0}們!凡任何善法、善分、\twnr{善黨}{x602},那些全都以不放逸為根、以不放逸為會合,不放逸法被告知為它們中第一的。比丘們!不放逸比丘的這個能被預期:他必將\twnr{修習}{94.0}\twnr{七覺支}{524.0}、必將\twnr{多作}{95.0}七覺支。

  比丘們!而怎樣不放逸的比丘修習七覺支、多作七覺支?比丘們!這裡,比丘\twnr{依止遠離}{322.0}……修習\twnr{念覺支}{315.0}……(中略)依止遠離、依止離貪、依止滅、\twnr{捨棄的成熟}{221.0}修習\twnr{平靜覺支}{314.0}。比丘們!這樣,不放逸的比丘修習七覺支、多作七覺支。」



\sutta{32}{32}{善經第二}{https://agama.buddhason.org/SN/sn.php?keyword=46.32}
  「\twnr{比丘}{31.0}們!凡任何善法、善分、\twnr{善黨}{x603},一切以\twnr{如理作意}{114.0}為根、以如理作意為會合,如理作意被說為那些法中第一的。比丘們!如理作意具足比丘的這個能被預期:他必將\twnr{修習}{94.0}\twnr{七覺支}{524.0}、必將\twnr{多作}{95.0}七覺支。

  比丘們!而怎樣如理作意具足的比丘修習七覺支、多作七覺支?比丘們!這裡,比丘\twnr{依止遠離}{322.0}……修習\twnr{念覺支}{315.0}……(中略)依止遠離、依止離貪、依止滅、\twnr{捨棄的成熟}{221.0}修習\twnr{平靜覺支}{314.0}。比丘們!這樣,如理作意具足的比丘修習七覺支、多作七覺支。」



\sutta{33}{33}{隨雜染經}{https://agama.buddhason.org/SN/sn.php?keyword=46.33}
  「\twnr{比丘}{31.0}們!有這五種黃金的\twnr{隨雜染}{288.0},被該隨雜染雜染的黃金是不柔軟的,同時也是不\twnr{適合作業的}{412.0}、不\twnr{極光淨的}{573.0}與易破壞的,不正確地來到作業,哪五個?比丘們!鐵是黃金的隨雜染,被該隨雜染雜染的黃金是不柔軟的,同時也是不適合作業的、不極光淨與易破壞的,不正確地來到作業;比丘們!銅是黃金的隨雜染,被該隨雜染雜染的黃金……(中略)比丘們!錫是黃金的隨雜染……(中略)比丘們!鉛是黃金的隨雜染……(中略)比丘們!銀是黃金的隨雜染,被該隨雜染雜染的黃金是不柔軟的,同時也是不適合作業的、不極光淨與易破壞的,不正確地來到作業,比丘們!這是五種黃金的隨雜染,被該隨雜染雜染的黃金是不柔軟的,同時也是不適合作業的、不極光淨與易破壞的,不正確地來到作業。

  同樣的,比丘們!有這五種心的隨雜染,被該隨雜染雜染的心是不柔軟的,同時也是不適合作業的、不極光淨與易破壞的,不正確地為了諸\twnr{漏}{188.0}的滅盡入定,哪五個?比丘們!\twnr{欲的意欲}{118.0}是心的隨雜染,被該隨雜染雜染的黃金是不柔軟的,同時也是不適合作業的、不極光淨與易破壞的,不正確地為了諸漏的滅盡入定……(中略)比丘們!這是五種心的隨雜染,被該隨雜染雜染的黃金是不柔軟的,同時也是不適合作業的、不極光淨與易破壞的,不正確地為了諸漏的滅盡入定。」[≃\ccchref{AN.5.23}{https://agama.buddhason.org/AN/an.php?keyword=5.23}]



\sutta{34}{34}{非隨雜染經}{https://agama.buddhason.org/SN/sn.php?keyword=46.34}
  「\twnr{比丘}{31.0}們!有這些七覺支是非障礙的、非蓋的、心的非隨雜染的,已\twnr{修習}{94.0}、已\twnr{多作}{95.0},轉起明解脫果的作證,哪七個?比丘們!\twnr{念覺支}{315.0}是非障礙的、非蓋的、心的非隨雜染的,已修習、已多作,轉起明解脫果的作證……(中略)比丘們!\twnr{平靜覺支}{314.0}是非障礙的、非蓋的、心的非隨雜染的,已修習、已多作,轉起明解脫果的作證。比丘們!這些七覺支是非障礙的、非蓋的、心的非隨雜染的,已修習、已多作,轉起明解脫果的作證。」



\sutta{35}{35}{不如理作意經}{https://agama.buddhason.org/SN/sn.php?keyword=46.35}
  「\twnr{比丘}{31.0}們!當不\twnr{如理作意}{114.0}時,未生起之\twnr{欲的意欲}{118.0}就生起,且已生起之欲的意欲轉起增大、成滿;未生起的惡意就生起,且已生起的惡意轉起增大、成滿;未生起的惛沈睡眠就生起,且已生起的惛沈睡眠轉起增大、成滿;未生起的掉舉後悔就生起,且已生起的掉舉後悔轉起增大、成滿;未生起的疑惑就生起,且已生起的疑惑轉起增大、成滿。」



\sutta{36}{36}{如理作意經}{https://agama.buddhason.org/SN/sn.php?keyword=46.36}
  「但,\twnr{比丘}{31.0}們!當如理作意時,未生起的\twnr{念覺支}{315.0}就生起,且已生起的念覺支走到修習圓滿……(中略)未生起的\twnr{平靜覺支}{314.0}就生起,且生起的平靜覺支走到修習圓滿。」



\sutta{37}{37}{增長經}{https://agama.buddhason.org/SN/sn.php?keyword=46.37}
  「\twnr{比丘}{31.0}們!有這些\twnr{七覺支}{524.0},已\twnr{修習}{94.0}、已\twnr{多作}{95.0},轉起增長、不衰退,哪七個?念覺支……\twnr{平靜覺支}{314.0},比丘們!這些是七覺支,已修習、已多作,轉起增長、不衰退。」



\sutta{38}{38}{障礙蓋經}{https://agama.buddhason.org/SN/sn.php?keyword=46.38}
  「\twnr{比丘}{31.0}們!有這五個障礙、蓋、心的\twnr{隨雜染}{288.0}、\twnr{慧的減弱的}{283.0},哪五個?比丘們!\twnr{欲的意欲}{118.0}是障礙、蓋、心的隨雜染、慧的減弱者;比丘們!惡意是障礙、蓋、心的隨雜染、慧的減弱者;比丘們!惛沈睡眠是障礙、蓋、心的隨雜染、慧的減弱者;比丘們!掉舉後悔是障礙、蓋、心的隨雜染、慧的減弱者;比丘們!疑惑是障礙、蓋、心的隨雜染、慧的減弱者。比丘們!這是五個障礙、蓋、心的隨雜染、慧的減弱者。

  比丘們!有這七覺支是無障礙的、無蓋的、心的無隨雜染的,已\twnr{修習}{94.0}、已\twnr{多作}{95.0},轉起明解脫果的作證,哪七個?比丘們!\twnr{念覺支}{315.0}是無障礙的、無蓋的、心的無隨雜染的,已修習、已多作,轉起明解脫果的作證……(中略)比丘們!\twnr{平靜覺支}{314.0}是無障礙的、無蓋的、心的無隨雜染的,已修習、已多作,轉起明解脫果的作證。比丘們!這七覺支是無障礙的、無蓋的、心的無隨雜染的,已修習、已多作,轉起明解脫果的作證。

  比丘們!凡在\twnr{聖弟子}{24.0}\twnr{作目標後}{316.0}、作意後、\twnr{全心注意後}{479.0}傾耳聽法時,在那時,沒有這\twnr{五蓋}{287.0};在那時,七覺支走到修習圓滿。

  在那時,沒有哪五蓋呢?在那時,沒有欲的意欲蓋;在那時,沒有惡意蓋;在那時,沒有惛沈睡眠蓋;在那時,沒有掉舉後悔蓋;在那時,沒有疑惑蓋。

  在那時,哪七覺支走到修習圓滿呢?在那時,念覺支走到修習圓滿……(中略)在那時,平靜覺支走到修習圓滿。

  比丘們!凡在弟子作目標後、作意後、全心注意後傾耳聽法時,在那時,沒有這五蓋;在那時,這七覺支走到修習圓滿。」



\sutta{39}{39}{樹木經}{https://agama.buddhason.org/SN/sn.php?keyword=46.39}
  「\twnr{比丘}{31.0}們!有大樹,種子小而身大,是諸樹的壓制者,凡被它們壓制、破壞、強破壞、弄倒的諸樹都躺下。

  比丘們!而哪些大樹種子小而身大,它們是諸樹的壓制者,凡被它們壓制、破壞、強破壞、弄倒的諸樹都躺下呢?\twnr{菩提樹}{x604}、\twnr{榕樹}{x605}、\twnr{糙葉榕}{x606}、\twnr{叢生榕}{x607}、\twnr{無花果樹}{x608}、\twnr{山蘋果樹}{x609},比丘們!這些大樹種子小而身大,它們是諸樹的壓制者,凡被它們壓制、破壞、強破壞、弄倒的諸樹都躺下。同樣的,比丘們!這裡,某位\twnr{善男子}{41.0}捨去任何諸欲後\twnr{從在家出家成為無家者}{48.0},之後,被像那樣的諸欲或更惡想要的破壞、強破壞、弄倒,他躺下。

  比丘們!有這五個障礙、蓋、心被壓制者、\twnr{慧的減弱者}{283.0},哪五個?比丘們!\twnr{欲的意欲}{118.0}是障礙、蓋、心被壓制者、慧的減弱者;比丘們!惡意是障礙、蓋、心被壓制者、慧的減弱者;比丘們!惛沈睡眠是障礙、蓋、心被壓制者、慧的減弱者;比丘們!掉舉後悔是障礙、蓋、心被壓制者、慧的減弱者;比丘們!疑惑是障礙、蓋、心被壓制者、慧的減弱者。比丘們!這是五個障礙、蓋、心被壓制者、慧的減弱者。

  比丘們!有這七覺支是無障礙的、無蓋的、心不被壓制的,已\twnr{修習}{94.0}、已\twnr{多作}{95.0},轉起明解脫果的作證,哪七個?比丘們!\twnr{念覺支}{315.0}是無障礙的、無蓋的、心不被壓制的,已修習、已多作,轉起明解脫果的作證……(中略)比丘們!\twnr{平靜覺支}{314.0}是無障礙的、無蓋的、心不被壓制的,已修習、已多作,轉起明解脫果的作證。比丘們!這七覺支是無障礙的、無蓋的、心不被壓制的,已修習、已多作,轉起明解脫果的作證。」



\sutta{40}{40}{蓋經}{https://agama.buddhason.org/SN/sn.php?keyword=46.40}
  「\twnr{比丘}{31.0}們!有這些\twnr{五蓋}{287.0}是盲目所作、\twnr{不作眼}{970.0}、不作智、\twnr{慧的滅者}{998.0}、惱害的伴黨、不導向涅槃的,哪五個?比丘們!\twnr{欲的意欲}{118.0}蓋是盲目所作、不作眼、不作智、慧的滅者、惱害的伴黨,不導向涅槃的,比丘們!惡意蓋是……(中略)比丘們!惛沈睡眠蓋是……(中略)比丘們!掉舉後悔蓋是……(中略)比丘們!疑惑蓋是盲目所作、不作眼、不作智、慧的滅者、惱害的伴黨,不導向涅槃的。比丘們!這五蓋是盲目所作、不作眼、不作智、慧的滅者、惱害的伴黨、不導向涅槃的。

  比丘們!有這些七覺支是眼所作、智所作、令慧增長、不惱害的伴黨、導向涅槃的,哪七個?比丘們!\twnr{念覺支}{315.0}是眼所作、智所作、令慧增長、不惱害的伴黨、導向涅槃的……(中略)比丘們!\twnr{平靜覺支}{314.0}是眼所作、智所作、令慧增長、不惱害的伴黨、導向涅槃的,比丘們!這些七覺支是眼所作、智所作、令慧增長、不惱害的伴黨、導向涅槃的。」

  蓋品第四,其\twnr{攝頌}{35.0}:

  「二則善與隨雜染,二則如理與增長,

   \twnr{障、蓋}{x610}、樹木,以及蓋、它們為十。」





\pin{轉輪王品}{41}{50}
\sutta{41}{41}{慢經}{https://agama.buddhason.org/SN/sn.php?keyword=46.41}
  起源於舍衛城。

  「\twnr{比丘}{31.0}們!凡任何過去世的\twnr{沙門}{29.0}或\twnr{婆羅門}{17.0}曾捨斷三種\twnr{慢}{647.0}者,他們都\twnr{已自我修習}{658.0}、已自我\twnr{多作}{95.0}\twnr{七覺支}{524.0};凡任何\twnr{未來世}{308.0}的沙門或婆羅門將捨斷三種慢者,他們都已自我修習、已自我多作七覺支;凡任何現在的沙門或婆羅門捨斷三種慢者,他們都已自我修習、已自我多作七覺支,什麼是那七覺支呢?\twnr{念覺支}{315.0}……(中略)\twnr{平靜覺支}{314.0}。

  比丘們!凡任何過去世的沙門或婆羅門曾捨斷三種慢者……(中略)將捨斷三種慢……(中略)捨斷三種慢者,他們都已自我修習、已自我多作七覺支。」[⇒\suttaref{SN.45.162}]



\sutta{42}{42}{轉輪經}{https://agama.buddhason.org/SN/sn.php?keyword=46.42}
  「\twnr{比丘}{31.0}們!由於\twnr{轉輪王}{278.0}的出現,有七寶的出現,哪七個?有輪寶的出現,有象寶的出現,有馬寶的出現,有珠寶的出現,有女寶的出現,有\twnr{屋主}{103.0}寶的出現,有主兵臣寶的出現,比丘們!由於轉輪王的出現而有這七寶的出現。

  比丘們!由於\twnr{如來}{4.0}、\twnr{阿羅漢}{5.0}、\twnr{遍正覺者}{6.0}的出現,有\twnr{七覺支}{524.0}寶的出現,哪七個?有\twnr{念覺支}{315.0}寶的出現……(中略)有\twnr{平靜覺支}{314.0}的出現,比丘們!由於如來、阿羅漢、遍正覺者的出現,有這七覺支寶的出現。」



\sutta{43}{43}{魔經}{https://agama.buddhason.org/SN/sn.php?keyword=46.43}
  「\twnr{比丘}{31.0}們!我將為你們教導碎破魔軍之道,\twnr{你們要聽}{43.0}它!比丘們!而什麼是碎破魔軍之道呢?即:\twnr{七覺支}{524.0}。哪七個?\twnr{念覺支}{315.0}……(中略)\twnr{平靜覺支}{314.0},比丘們!這是碎破魔軍之道。」



\sutta{44}{44}{劣慧經}{https://agama.buddhason.org/SN/sn.php?keyword=46.44}
  那時,\twnr{某位比丘}{39.0}去見\twnr{世尊}{12.0}……(中略)在一旁坐下的那位比丘對世尊說這個:

  「\twnr{大德}{45.0}!被稱為『劣慧的聾啞者、劣慧的聾啞者』,大德!什麼情形被稱為『劣慧的聾啞者』呢?」

  「比丘!對\twnr{七覺支}{524.0}未自我\twnr{修習}{94.0}、未自我\twnr{多作}{95.0}者被稱為『劣慧的聾啞者』,對哪七個?對\twnr{念覺支}{315.0}……(中略)對\twnr{平靜覺支}{314.0}。比丘!對這些七覺支未自我修習、未自我多作者被稱為『劣慧的聾啞者』。」



\sutta{45}{45}{有慧經}{https://agama.buddhason.org/SN/sn.php?keyword=46.45}
  「\twnr{大德}{45.0}!被稱為『有慧的非聾啞者、有慧的非聾啞者』,大德!什麼情形被稱為『有慧的非聾啞者』呢?」

  「\twnr{比丘}{31.0}!以\twnr{七覺支}{524.0}的\twnr{已自我修習}{658.0}、已自我\twnr{多作}{95.0},被稱為『有慧的非聾啞者』,哪七個?\twnr{念覺支}{315.0}的……(中略)\twnr{平靜覺支}{314.0}的……。比丘!以這些七覺支已自我修習、已自我多作,被稱為『有慧的非聾啞者』。」



\sutta{46}{46}{貧窮者經}{https://agama.buddhason.org/SN/sn.php?keyword=46.46}
  「\twnr{大德}{45.0}!被稱為『貧窮者、貧窮者』,大德!什麼情形被稱為『貧窮者』呢?」

  「\twnr{比丘}{31.0}!對\twnr{七覺支}{524.0}未自我\twnr{修習}{94.0}、未自我\twnr{多作}{95.0}者被稱為『貧窮者』,對哪七個?對\twnr{念覺支}{315.0}……(中略)對\twnr{平靜覺支}{314.0}。比丘!對七覺支未自我修習、未自我多作者被稱為『貧窮者』。」



\sutta{47}{47}{不貧窮者經}{https://agama.buddhason.org/SN/sn.php?keyword=46.47}
  「\twnr{大德}{45.0}!被稱為『不貧窮者、不貧窮者』,大德!什麼情形被稱為『不貧窮者』呢?」

  「\twnr{比丘}{31.0}!以\twnr{七覺支}{524.0}的\twnr{已自我修習}{658.0}、已自我\twnr{多作}{95.0},被稱為『不貧窮者』,哪七個?以\twnr{念覺支}{315.0}的……(中略)以\twnr{平靜覺支}{314.0}的……。比丘!以這些七覺支的已自我修習、已自我多作,被稱為『不貧窮者』。」



\sutta{48}{48}{太陽經}{https://agama.buddhason.org/SN/sn.php?keyword=46.48}
  「\twnr{比丘}{31.0}們!對太陽的昇起,這是先導,這是前兆,即:黎明。同樣的,比丘們!對比丘\twnr{七覺支}{524.0}的生起,這是先導,這是前兆,即:善友誼。

  比丘們!有善友比丘的這個能被預期:他必將\twnr{修習}{94.0}七覺支,他必將\twnr{多作}{95.0}七覺支。

  比丘們!而怎樣有善友的比丘修習七覺支、多作七覺支?比丘們!這裡,比丘\twnr{依止遠離}{322.0}、依止離貪、依止滅、\twnr{捨棄的成熟}{221.0}修習\twnr{念覺支}{315.0}……(中略)依止遠離、依止離貪、依止滅、捨棄的成熟修習\twnr{平靜覺支}{314.0}。

  比丘們!這樣,有善友的比丘修習七覺支、多作七覺支。」[\suttaref{SN.48.12}]



\sutta{49}{49}{內支經}{https://agama.buddhason.org/SN/sn.php?keyword=46.49}
  「\twnr{比丘}{31.0}們!我不見還有其它一支『\twnr{內支}{574.0}』作後對\twnr{七覺支}{524.0}的生起,比丘們!如這\twnr{如理作意}{114.0}。比丘們!如理作意具足比丘的這個能被預期:他必將\twnr{修習}{94.0}七覺支,必將\twnr{多作}{95.0}七覺支。

  比丘們!而怎樣如理作意具足的比丘修習七覺支、多作七覺支?比丘們!這裡,比丘\twnr{依止遠離}{322.0}……修習\twnr{念覺支}{315.0}……(中略)依止遠離、依止離貪、依止滅、\twnr{捨棄的成熟}{221.0}修習\twnr{平靜覺支}{314.0}。比丘們!這樣,如理作意具足的比丘修習七覺支、\twnr{多作}{95.0}七覺支。」



\sutta{50}{50}{外支經}{https://agama.buddhason.org/SN/sn.php?keyword=46.50}
  「\twnr{比丘}{31.0}們!我不見還有其它一支『\twnr{外支}{574.1}』作後對\twnr{七覺支}{524.0}的生起,比丘們!如這善友誼。比丘們!有善友比丘的這個能被預期:他必將\twnr{修習}{94.0}七覺支,必將\twnr{多作}{95.0}七覺支。

  比丘們!而怎樣有善友的比丘修習七覺支、多作七覺支?比丘們!這裡,比丘\twnr{依止遠離}{322.0}……修習\twnr{念覺支}{315.0}……(中略)依止遠離、依止離貪、依止滅、\twnr{捨棄的成熟}{221.0}修習\twnr{平靜覺支}{314.0}。比丘們!這樣,有善友的比丘修習七覺支、多作七覺支。」

  轉輪品第五,其\twnr{攝頌}{35.0}:

  「慢、轉輪、魔,劣慧與有慧,

   貧窮、不貧窮,以太陽、支它們為十。」





\pin{交談品}{51}{56}
\sutta{51}{51}{食經}{https://agama.buddhason.org/SN/sn.php?keyword=46.51}
  起源於舍衛城。

  「\twnr{比丘}{31.0}們!我將教導\twnr{五蓋}{287.0}與\twnr{七覺支}{524.0}的食與非食,\twnr{你們要聽}{43.0}它!

  比丘們!而對未生起之欲的意欲的生起,或已生起之\twnr{欲的意欲}{118.0}的增大、成滿來說,什麼是食呢?比丘們!有\twnr{淨相}{597.0},在那裡,不\twnr{如理作意}{114.0}的\twnr{多作}{95.0},對未生起之欲的意欲的生起,或已生起之欲的意欲的增大、成滿,這是食。

  比丘們!而對未生起之惡意的生起,或已生起之惡意的增大、成滿來說,什麼是食呢?比丘們!有\twnr{嫌惡相}{687.0},在那裡,不如理作意的多作,對未生起之惡意的生起,或已生起之惡意的增大、成滿,這是食。

  比丘們!而對未生起之惛沈睡眠的生起,或已生起之惛沈睡眠的增大、成滿來說,什麼是食呢?比丘們!有不樂、倦怠、打哈欠、餐後的睡意、\twnr{心的退縮}{686.1},在那裡,不如理作意的多作,對未生起之惛沈睡眠的生起,或已生起之惛沈睡眠的增大、成滿,這是食。

  比丘們!而對未生起之掉舉後悔的生起,或已生起之掉舉後悔的增大、成滿來說,什麼是食呢?比丘們!有心的不平息,在那裡,不如理作意的多作,對未生起之掉舉後悔的生起,或已生起之掉舉後悔的增大、成滿,這是食。

  比丘們!而對未生起之疑惑的生起,或已生起之疑惑的增大、成滿來說,什麼是食呢?比丘們!有疑惑處之法,在那裡,不如理作意的多作,對未生起之疑惑的生起,或已生起之疑惑的增大、成滿,這是食。

  比丘們!而對未生起之念覺支的生起,或已生起之念覺支的修習圓滿來說,什麼是食呢?比丘們!有\twnr{念覺支處諸法}{315.1},在那裡,如理作意的多作,對未生起之念覺支的生起,或已生起之念覺支的修習圓滿,這是食。

  比丘們!而對未生起之\twnr{擇法覺支}{311.0}的生起,或已生起之擇法覺支的修習圓滿來說,什麼是食呢?比丘們!有諸善不善法,諸有罪過無罪過法,諸下劣勝妙法,諸\twnr{黑白有對比的}{550.0}法,在那裡,如理作意的多作,對未生起之擇法覺支的生起,或已生起之擇法覺支的修習圓滿,這是食。

  比丘們!而對未生起之\twnr{活力覺支}{310.0}的生起,或已生起之活力覺支的修習圓滿來說,什麼是食呢?比丘們!有發勤界、精勤界、努力界,在那裡,如理作意的多作,對未生起之活力覺支的生起,或已生起之活力覺支的修習圓滿,這是食。

  比丘們!而對未生起之\twnr{喜覺支}{312.0}的生起,或已生起之喜覺支的修習圓滿來說,什麼是食呢?比丘們!有諸喜覺支處之法,在那裡,如理作意的多作,對未生起之喜覺支的生起,或已生起之喜覺支的修習圓滿,這是食。

  比丘們!而對未生起之\twnr{寧靜覺支}{313.1}的生起,或已生起之寧靜覺支的修習圓滿來說,什麼是食呢?比丘們!有\twnr{身體的寧靜}{318.0}、\twnr{心的寧靜}{319.0},在那裡,如理作意的多作,對未生起之寧靜覺支的生起,或已生起之寧靜覺支的修習圓滿,這是食。

  比丘們!而對未生起之定覺支的生起,或已生起之定覺支的修習圓滿來說,什麼是食呢?比丘們!有\twnr{止相}{688.0}、不混亂相,在那裡,如理作意的多作,對未生起之定覺支的生起,或已生起之定覺支的修習圓滿,這是食。

  比丘們!而對未生起之\twnr{平靜覺支}{314.0}的生起,或已生起之平靜覺支的修習圓滿來說,什麼是食呢?比丘們!有諸平靜覺支處之法,在那裡,如理作意的多作,對未生起之平靜覺支的生起,或已生起之平靜覺支的修習圓滿,這是食。[\suttaref{SN.46.2}]

  比丘們!而對未生起之欲的意欲的生起,或已生起之欲的意欲的增大、成滿來說,什麼是非食呢?比丘們!有不淨相,在那裡,如理作意的多作,對未生起之欲的意欲的生起,或已生起之欲的意欲的增大、成滿,這是非食。

  比丘們!而對未生起之惡意的生起,或已生起之惡意的增大、成滿來說,什麼是非食呢?比丘們!有\twnr{慈心解脫}{589.0},在那裡,如理作意的多作,對未生起之惡意的生起,或已生起之惡意的增大、成滿,這是非食。

  比丘們!而對未生起之惛沈睡眠的生起,或已生起之惛沈睡眠的增大、成滿來說,什麼是非食呢?比丘們!有發勤界、精勤界、努力界,在那裡,如理作意的多作,對未生起之惛沈睡眠的生起,或已生起之惛沈睡眠的增大、成滿,這是非食。

  比丘們!而對未生起之掉舉後悔的生起,或已生起之掉舉後悔的增大、成滿來說,什麼是非食呢?比丘們!有心的平息,在那裡,如理作意的多作,對未生起之掉舉後悔的生起,或已生起之掉舉後悔的增大、成滿,這是非食。

  比丘們!而對未生起之疑惑的生起,或已生起之疑惑的增大、成滿來說,什麼是非食呢?比丘們!有諸善不善法,諸有罪過無罪過法,諸下劣勝妙法,諸黑白有對比的法,在那裡,如理作意的多作,對未生起之疑惑的生起,或已生起之疑惑的增大、成滿,這是非食。

  比丘們!而對未生起之念覺支的生起,或已生起之念覺支的修習圓滿來說,什麼是非食呢?比丘們!有念覺支處諸法,在那裡,不如理作意的多作,對未生起之念覺支的生起,或已生起之念覺支的修習圓滿,這是非食。

  比丘們!而對未生起之擇法覺支的生起,或已生起之擇法覺支的修習圓滿來說,什麼是非食呢?比丘們!有諸善不善法,諸有罪過無罪過法,諸下劣勝妙法,諸黑白有對比的法,在那裡,不如理作意的多作,對未生起之擇法覺支的生起,或已生起之擇法覺支的修習圓滿,這是非食。

  比丘們!而對未生起之活力覺支的生起,或已生起之活力覺支的修習圓滿來說,什麼是非食呢?比丘們!有發勤界、精勤界、努力界,在那裡,不如理作意的多作,對未生起之活力覺支的生起,或已生起之活力覺支的修習圓滿,這是非食。

  比丘們!而對未生起之喜覺支的生起,或已生起之喜覺支的修習圓滿來說,什麼是非食呢?比丘們!有諸喜覺支處之法,在那裡,不如理作意的多作,對未生起之喜覺支的生起,或已生起之喜覺支的修習圓滿,這是非食。

  比丘們!而對未生起之寧靜覺支的生起,或已生起之寧靜覺支的修習圓滿來說,什麼是非食呢?比丘們!有身體的寧靜、心的寧靜,在那裡,不如理作意的多作,對未生起之寧靜覺支的生起,或已生起之寧靜覺支的修習圓滿,這是非食。

  比丘們!而對未生起之定覺支的生起,或已生起之定覺支的修習圓滿來說,什麼是非食呢?比丘們!有止相、不混亂相,在那裡,不如理作意的多作,對未生起之定覺支的生起,或已生起之定覺支的修習圓滿,這是非食。

  比丘們!而對未生起之平靜覺支的生起,或已生起之平靜覺支的修習圓滿來說,什麼是非食呢?比丘們!有諸平靜覺支處之法,在那裡,不如理作意的多作,對未生起之平靜覺支的生起,或已生起之平靜覺支的修習圓滿,這是非食。」



\sutta{52}{52}{法門經}{https://agama.buddhason.org/SN/sn.php?keyword=46.52}
  那時,眾多\twnr{比丘}{31.0}午前時穿衣、拿起衣鉢後,\twnr{為了托鉢}{87.0}進入舍衛城。

  那時,那些比丘想這個:「在舍衛城為了托鉢行走大致上還太早,讓我們前往\twnr{其他外道遊行者}{79.0}們的園林。」

  那時,那些比丘前往其他外道遊行者們的園林。抵達後,與那些其他外道遊行者一起互相問候。交換應該被互相問候的友好交談後,在一旁坐下。那些其他外道遊行者對在一旁坐下的那些比丘說這個:

  「\twnr{道友們}{201.0}!\twnr{沙門}{29.0}\twnr{喬達摩}{80.0}為弟子們教導這樣的法:『來!比丘們!你們捨斷心的\twnr{隨雜染}{288.0}、\twnr{慧的減弱的}{283.0}\twnr{五蓋}{287.0}後,請你們如實\twnr{修習}{94.0}\twnr{七覺支}{524.0}。』道友們!我們也為弟子們教導這樣的法:『來!學友們!你們捨斷心的隨雜染、慧的減弱之五蓋後,請你們如實修習七覺支。』道友們!這裡,沙門喬達摩的與我們的,什麼是差別,什麼是不同,什麼是區別?即:法的教說比法的教說,教誡比教誡。」

  那時,那些比丘們對那些其他外道遊行者的所說,既不歡喜,也不斥責。不歡喜、不斥責後,從座位起來後離開:「我們在\twnr{世尊}{12.0}的面前將了知這所說的義理。」

  那時,那些比丘在舍衛城為了托鉢行走後,\twnr{餐後已從施食返回}{512.0},去見世尊。抵達後,向世尊\twnr{問訊}{46.0}後,在一旁坐下。在一旁坐下的那些比丘對世尊說這個:

  「\twnr{大德}{45.0}!這裡,我們午前時穿衣、拿起衣鉢後,為了托鉢進入舍衛城。大德!那些我們想這個:『在舍衛城為了托鉢行走大致上還太早,讓我們前往其他外道遊行者們的園林。』大德!那時,我們前往其他外道遊行者們的園林。抵達後,與那些其他外道遊行者一起互相問候。交換應該被互相問候的友好交談後,在一旁坐下。大德!那些其他外道遊行者對在一旁坐下的我們說這個:『道友們!沙門喬達摩為弟子們教導這樣的法:「來!比丘們!你們捨斷心的隨雜染、慧的減弱的五蓋後,請你們如實修習七覺支。」我們也為弟子們教導這樣的法:「來!學友們!你們捨斷心的隨雜染、慧的減弱的五蓋後,請你們如實修習七覺支。」道友們!這裡,沙門喬達摩的與我們的,什麼是差別,什麼是不同,什麼是區別?即:法的教說比法的教說,教誡比教誡。』大德!那時,我們對那些其他外道遊行者的所說,既不歡喜,也不斥責。不歡喜、不斥責後,從座位起來後離開:『我們在世尊的面前將了知這所說的義理。』」

  「比丘們!這麼說的其他外道遊行者們應該被這麼回答:『道友們!但,有\twnr{法門}{562.0},由於該法門五蓋成為十個,七覺支成為十四個。』比丘們!被這麼問,其他外道遊行者們既將會不解答,且更會來到惱害,那是什麼原因?比丘們!那個正如\twnr{不在[感官的]境域中}{783.0}那樣。比丘們!我不見那個:在包括天,在包括魔,在包括梵的世間;在包括沙門婆羅門,在包括天-人的\twnr{世代}{38.0}中,凡以這些問題的解答能使心滿意者,除了\twnr{如來}{4.0}或如來的弟子,又或從那裡聽聞後以外。

  比丘們!什麼法門,由於該法門五蓋成為十個呢?

  比丘們!凡即使自身內的\twnr{欲的意欲}{118.0},那也是蓋;凡即使外部的欲的意欲,那也是蓋,『欲的意欲蓋』像這樣走到這個總說,那樣,以這個法門,這也成為二種的。

  比丘們!凡即使自身內的惡意,那也是蓋;凡即使外部的惡意,那也是蓋,『惡意蓋』像這樣走到這個總說,那樣,以這個法門,這也成為二種的。

  比丘們!凡即使惛沈,那也是蓋;凡即使睡眠,那也是蓋,『惛沈睡眠蓋』像這樣走到這個總說,那樣,以這個法門,這也成為二種的。

  比丘們!凡即使掉舉,那也是蓋;凡即使後悔,那也是蓋,『掉舉後悔蓋』像這樣走到這個總說,那樣,以這個法門,這也成為二種的。

  比丘們!凡即使在自身內諸法上的疑惑,那也是蓋;凡即使在外部諸法上的疑惑,那也是蓋,『疑惑蓋』像這樣走到這個總說,那樣,以這個法門,這也成為二種的。

  比丘們!這是法門,由於該法門五蓋成為十個。

  比丘們!什麼法門,由於該法門七覺支成為十四呢?

  比丘們!凡即使在自身內諸法上的念,那也是\twnr{念覺支}{315.0};凡即使在外部諸法上的念,那也是念覺支,『念覺支』像這樣走到這個總說,那樣,以這個法門,這也成為二種的。

  比丘們!凡即使在自身內諸法上以慧考察、伺察、來到審慮,那也是\twnr{擇法覺支}{311.0};凡即使在外部諸法上以慧考察、伺察、來到審慮,那也是擇法覺支,『擇法覺支』像這樣走到這個總說,那樣,以這個法門,這也成為二種的。

  比丘們!凡即使身體的活力,那也是\twnr{活力覺支}{310.0};凡即使心理的活力,那也是活力覺支,『活力覺支』像這樣走到這個總說,那樣,以這個法門,這也成為二種的。

  比丘們!凡即使有尋、\twnr{有伺}{175.0}的喜,那也是\twnr{喜覺支}{312.0};凡即使無尋、無伺的喜,那也是喜覺支,『喜覺支』像這樣走到這個總說,那樣,以這個法門,這也成為二種的。

  比丘們!凡即使\twnr{身體的寧靜}{318.0},那也是\twnr{寧靜覺支}{313.1};凡即使\twnr{心的寧靜}{319.0},那也是寧靜覺支,『寧靜覺支』像這樣走到這個總說,那樣,以這個法門,這也成為二種的。

  比丘們!凡即使有尋、有伺的定,那也是定覺支;凡無尋、無伺的定,那也是定覺支,『定覺支』像這樣走到這個總說,那樣,以這個法門,這也成為二種的。

  比丘們!凡即使在自身內諸法上的\twnr{平靜}{228.0},那也是\twnr{平靜覺支}{314.0};凡即使在外部諸法上的平靜,那也是平靜覺支,『平靜覺支』像這樣走到這個總說,那樣,以這個法門,這也成為二種的。

  比丘們!這是法門,由於該法門七覺支成為十四個。」



\sutta{53}{53}{火經}{https://agama.buddhason.org/SN/sn.php?keyword=46.53}
  那時,眾多\twnr{比丘}{31.0}午前時穿衣、拿起衣鉢後,\twnr{為了托鉢}{87.0}進入舍衛城。(同法門經[\suttaref{SN.46.52}])

  「比丘們!這麼說的\twnr{其他外道遊行者}{79.0}們應該被這麼回答:『\twnr{道友們}{201.0}!凡在心是退縮的時,那時,哪些覺支的\twnr{修習}{94.0}是不適時的,哪些覺支的修習是適時的?

  道友們!又,凡在心是掉舉的時,那時,哪些覺支的修習是不適時的,哪些覺支的修習是適時的?』

  比丘們!被這麼問,其他外道遊行者們既將會不解答,且更會來到惱害,那是什麼原因?比丘們!那個正如\twnr{不在[感官的]境域中}{783.0}那樣。比丘們!我不見那個:在包括天,在包括魔,在包括梵的世間;在包括沙門婆羅門,在包括天-人的\twnr{世代}{38.0}中,凡以這些問題的解答能使心滿意者,除了\twnr{如來}{4.0}或如來的弟子,又或從那裡聽聞後以外。

  比丘們!凡在心是退縮的時,那時,\twnr{寧靜覺支}{313.1}的修習是不適時的;定覺支的修習是不適時的;\twnr{平靜覺支}{314.0}的修習是不適時的,那是什麼原因?比丘們!那顆退縮的心以這些法是難使奮起的。比丘們!猶如男子想要使小火熾燃,在那裡他投入濕草,同時也投入濕牛糞,也投入濕木柴,也噴水(給與水風),也以塵土撒佈,那位男子能夠使小火熾燃嗎?」

  「\twnr{大德}{45.0}!這確實不是。」

  「同樣的,比丘們!凡在心是退縮的時,那時,\twnr{寧靜覺支}{313.1}的修習是不適時的;定覺支的修習是不適時的;\twnr{平靜覺支}{314.0}的修習是不適時的,那是什麼原因?比丘們!那顆退縮的心以這些法是難使奮起的。

  比丘們!凡在心是退縮的時,那時,\twnr{擇法覺支}{311.0}的修習是適時的;\twnr{活力覺支}{310.0}的修習是適時的;\twnr{喜覺支}{312.0}的修習是適時的,那是什麼原因?比丘們!那顆退縮的心以這些法是易使奮起的。比丘們!猶如男子想要使小火熾燃,在那裡他投入乾草,同時也投入乾牛糞,也投入乾木柴,也吹風(給與口風),且不以塵土撒佈,那位男子能夠使小火熾燃嗎?」

  「是的,大德!」

  「同樣的,比丘們!凡在心是退縮的時,那時,擇法覺支的修習是適時的;活力覺支的修習是適時的;喜覺支的修習是適時的,那是什麼原因?比丘們!那顆退縮的心以這些法是易使奮起的。

  比丘們!凡在心是掉舉的時,那時,擇法覺支的修習是不適時的;活力覺支的修習是不適時的;喜覺支的修習是不適時的,那是什麼原因?比丘們!那顆掉舉的心以這些法是難使平靜的。比丘們!猶如男子想要使大火聚熄滅,在那裡他投入乾草,同時也投入乾牛糞,也投入乾木柴,也吹風,且不以塵土撒佈,那位男子能夠使大火聚熄滅嗎?」

  「大德!這確實不是。」

  「同樣的,比丘們!凡在心是掉舉的時,那時,擇法覺支的修習是不適時的;活力覺支的修習是不適時的;喜覺支的修習是不適時的,那是什麼原因?比丘們!那顆掉舉的心以這些法是難使平靜的。

  比丘們!凡在心是掉舉的時,那時,寧靜覺支的修習是適時的;定覺支的修習是適時的;平靜覺支的修習是適時的,那是什麼原因?比丘們!那顆掉舉的心以這些法是易使平靜的。比丘們!猶如男子想要使大火聚熄滅,在那裡他投入濕草,同時也投入濕牛糞,也投入濕木柴,也噴水,也以塵土撒佈,那位男子能夠使大火聚熄滅嗎?」

  「是的,大德!」

  「同樣的,比丘們!凡在心是掉舉的時,那時,寧靜覺支的修習是適時的;定覺支的修習是適時的;平靜覺支的修習是適時的,那是什麼原因?比丘們!那顆掉舉的心以這些法是難使平靜的。

  比丘們!而我說,念是全都適當的。」



\sutta{54}{54}{慈俱行經}{https://agama.buddhason.org/SN/sn.php?keyword=46.54}
  \twnr{有一次}{2.0},\twnr{世尊}{12.0}住在拘利國名叫哈利達瓦沙的拘利族人城鎮。

  那時,眾多\twnr{比丘}{31.0}午前時穿衣、拿起衣鉢後,\twnr{為了托鉢}{87.0}進入哈利達瓦沙。

  那時,那些比丘想這個:「在哈利達瓦沙為了托鉢行走大致上還太早,讓我們前往\twnr{其他外道遊行者}{79.0}們的園林。」

  那時,那些比丘前往其他外道遊行者們的園林。抵達後,與那些其他外道遊行者一起互相問候。交換應該被互相問候的友好交談後,在一旁坐下。在一旁坐下的那些其他外道遊行者對那些比丘說這個:

  「\twnr{道友們}{201.0}!\twnr{沙門}{29.0}\twnr{喬達摩}{80.0}為弟子們教導這樣的法:『來!比丘們!你們捨斷心的\twnr{隨雜染}{288.0}、\twnr{慧的減弱的}{283.0}\twnr{五蓋}{287.0}後,請你們以與慈俱行之心遍滿一方後而住,像這樣第二的,像這樣第三的,像這樣第四的,像這樣上下、橫向、到處,以對一切如對自己,請你們以與慈俱行之心,以廣大、出眾、無量、無怨恨、無惡意之心遍滿全部世間後而住。

  請你們以與悲俱行之心遍滿一方後而住,像這樣第二的,像這樣第三的,像這樣第四的,像這樣上下、橫向、到處,以對一切如對自己,請你們以與悲俱行之心,以廣大、出眾、無量、無怨恨、無惡意之心遍滿全部世間後而住。

  請你們以與喜悅俱行之心遍滿一方後而住,像這樣第二的,像這樣第三的,像這樣第四的,像這樣上下、橫向、到處,以對一切如對自己,請你們以與喜悅俱行之心,以廣大、出眾、無量、無怨恨、無惡意之心遍滿全部世間後而住。

  請你們以與\twnr{平靜}{228.0}俱行之心遍滿一方後而住,像這樣第二的,像這樣第三的,像這樣第四的,像這樣上下、橫向、到處,以對一切如對自己,請你們以與平靜俱行之心,以廣大、出眾、無量、無怨恨、無惡意之心遍滿全部世間後而住。』

  道友們!我們也為弟子們教導這樣的法:『來!學友們!你們捨斷心的隨雜染、慧的減弱之五蓋後,請你們以與慈俱行之心遍滿一方後而住……(中略)請你們以與悲俱行之心……(中略)請你們以與喜悅俱行之心……(中略)請你們以與平靜俱行之心遍滿一方後而住,像這樣第二的,像這樣第三的,像這樣第四的,像這樣上下、橫向、到處,以對一切如對自己,請你們以與平靜俱行之心,以廣大、出眾、無量、無怨恨、無惡意之心遍滿全部世間後而住。』

  道友們!這裡,沙門喬達摩的與我們的,什麼是差別,什麼是不同,什麼是區別?即:法的教說比法的教說,教誡比教誡。」

  那時,那些比丘們對那些其他外道遊行者的所說,既不歡喜,也不斥責。不歡喜、不斥責後,從座位起來後離開:「我們在世尊的面前將了知這所說的義理。」

  那時,那些比丘在哈利達瓦沙為了托鉢行走後,\twnr{餐後已從施食返回}{512.0},去見世尊。抵達後,向世尊問訊後,在一旁坐下。在一旁坐下的那些比丘對世尊說這個:

  「\twnr{大德}{45.0}!這裡,我們午前時穿衣、拿起衣鉢後,為了托鉢進入哈利達瓦沙。大德!那些我們想這個:『在哈利達瓦沙為了托鉢行走大致上還太早,讓我們前往其他外道遊行者們的園林。』大德!那時,我們前往其他外道遊行者們的園林。抵達後,與那些其他外道遊行者一起互相問候。交換應該被互相問候的友好交談後,在一旁坐下。大德!在一旁坐下的那些其他外道遊行者對我們說這個:『道友們!沙門喬達摩為弟子們教導這樣的法:「來!比丘們!你們捨斷心的隨雜染、慧的減弱之五蓋後,請你們以與慈俱行之心遍滿一方後而住……(中略)請你們以與悲俱行之心……(中略)請你們以與喜悅俱行之心……(中略)請你們以與平靜俱行之心遍滿一方後而住,像這樣第二的,像這樣第三的,像這樣第四的,像這樣上下、橫向、到處,以對一切如對自己,請你們以與平靜俱行之心,以廣大、出眾、無量、無怨恨、無惡意之心遍滿全部世間後而住。」我們也為弟子們教導這樣的法:「來!學友們!你們捨斷心的隨雜染、慧的減弱之五蓋後,請你們以與慈俱行之心遍滿一方後而住……(中略)請你們以與悲俱行之心……(中略)請你們以與喜悅俱行之心……(中略)請你們以與平靜俱行之心遍滿一方後而住,像這樣第二的,像這樣第三的,像這樣第四的,像這樣上下、橫向、到處,以對一切如對自己,與平靜俱行之心,以廣大、出眾、無量、無怨恨、無惡意之心遍滿全部世間後而住。」道友們!這裡,沙門喬達摩的與我們的,什麼是差別,什麼是不同,什麼是區別?即:法的教說比法的教說,教誡比教誡。』大德!那時,我們對那些其他外道遊行者的所說,既不歡喜,也不斥責。不歡喜、不斥責後,從座位起來後離開:『我們在世尊的面前將了知這所說的義理。』」

  「比丘們!這麼說的其他外道遊行者們應該被這麼回答:『道友們!那麼,慈心解脫如何被\twnr{修習}{94.0}?什麼是趣向的?什麼是最高的?什麼是結果?什麼是完結?道友們!那麼,悲心解脫如何被修習?什麼是趣向的?什麼是最高的?什麼是結果?什麼是完結?道友們!那麼,喜悅心解脫如何被修習?什麼是趣向的?什麼是最高的?什麼是結果?什麼是完結?道友們!那麼,平靜心解脫如何被修習?什麼是趣向的?什麼是最高的?什麼是結果?什麼是完結?』比丘們!被這麼問,其他外道遊行者們既將會不解答,且更會來到惱害,那是什麼原因?比丘們!那個正如\twnr{不在[感官的]境域中}{783.0}那樣。比丘們!我不見那個:在包括天,在包括魔,在包括梵的世間;在包括沙門婆羅門,在包括天-人的\twnr{世代}{38.0}中,凡以這些問題的解答能使心滿意者,除了\twnr{如來}{4.0}或如來的弟子,又或從那裡聽聞後以外。

  比丘們!而慈心解脫如何被修習?什麼是趣向的?什麼是最高的?什麼是結果?什麼是完結?比丘們!這裡,比丘\twnr{與慈俱行}{x611}……修習\twnr{念覺支}{315.0}……(中略)與慈俱行,\twnr{依止遠離}{322.0}、依止離貪、依止滅、\twnr{捨棄的成熟}{221.0}修習\twnr{平靜覺支}{314.0},如果他希望『願我在無\twnr{厭逆}{227.0}上住於有厭逆想的』,在那裡,他住於有厭逆想的;如果他希望『願我在厭逆上住於無厭逆想的』,在那裡,他住於無厭逆想的;如果他希望『願我在不厭逆與厭逆上都住於厭逆想』,在那裡,他住於有厭逆想的;如果他希望『願我在厭逆與不厭逆上都住於不厭逆想』,在那裡,他住於無厭逆想的;如果他希望『願在無厭逆與厭逆兩者上都避免後,住於\twnr{平靜}{228.0}的、具念的、正知的』,在那裡,他住於平靜的、具念的、正知的,又或\twnr{進入後住於}{66.0}清淨解脫。比丘們!這裡,對未通達更上解脫的有慧比丘來說,我說\twnr{慈心解脫}{589.0},\twnr{清淨是最高的}{x612}。

  比丘們!而悲心解脫如何被修習?什麼是趣向的?什麼是最高的?什麼是結果?什麼是完結?比丘們!這裡,比丘與悲俱行……修習念覺支……(中略)與悲俱行,依止遠離、依止離貪、依止滅、捨棄的成熟修習平靜覺支,如果他希望『願我在無厭逆上住於有厭逆想的』,在那裡,他住於有厭逆想的……(中略)如果他希望『願在無厭逆與厭逆兩者上都避免後,住於平靜,是具念的、正知的』,在那裡,他住於平靜的、具念的、正知的,又或\twnr{從一切色想的超越}{490.0},從\twnr{有對想}{331.0}的滅沒,從不作意種種想[而知]:『虛空是無邊的』,進入後住於虛空無邊處,比丘們!這裡,對未通達更上解脫的有慧比丘來說,我說悲心解脫,虛空無邊處是最高的。

  比丘們!而喜悅心解脫如何被修習?什麼是趣向的?什麼是最高的?什麼是結果?什麼是完結?比丘們!這裡,比丘與喜悅俱行……修習念覺支……(中略)與喜悅俱行,依止遠離、依止離貪、依止滅、捨棄的成熟修習平靜覺支,如果他希望『願我在無厭逆上住於有厭逆想的』,在那裡,他住於有厭逆想的……(中略)如果他希望『願在無厭逆與厭逆兩者上都避免後,住於平靜,是具念的、正知的』,在那裡,他住於平靜的、具念的、正知的,或者,超越一切虛空無邊處後[而知]:『識是無邊的』,進入後住於識無邊處,比丘們!這裡,對未通達更上解脫的有慧比丘來說,我說喜悅心解脫,識無邊處是最高的。

  比丘們!而平靜心解脫如何被修習?什麼是趣向的?什麼是最高的?什麼是結果?什麼是完結?比丘們!這裡,比丘與平靜俱行……修習念覺支……(中略)與平靜俱行,依止遠離、依止離貪、依止滅、捨棄的成熟修習平靜覺支,如果他希望『願我在無厭逆上住於有厭逆想的』,在那裡,他住於有厭逆想的;如果他希望『願我在厭逆上住於無厭逆想的』,在那裡,他住於無厭逆想的;如果他希望『願我在不厭逆與厭逆上都住於厭逆想』,在那裡,他住於有厭逆想的;如果他希望『願我在厭逆與不厭逆上都住於不厭逆想』,在那裡,他住於無厭逆想的;如果他希望『願在無厭逆與厭逆兩者上都避免後,住於平靜,是具念的、正知的』,在那裡,他住於平靜的、具念的、正知的,或者,超越一切識無邊處後[而知]:『什麼都沒有』,進入後住於\twnr{無所有處}{533.0},比丘們!這裡,對未通達更上解脫的有慧比丘來說,我說平靜心解脫,無所有處是最高的。」



\sutta{55}{55}{傷歌邏經}{https://agama.buddhason.org/SN/sn.php?keyword=46.55}
  起源於舍衛城。

  那時,傷歌邏婆羅門去見\twnr{世尊}{12.0}。抵達後,與世尊一起互相問候。交換應該被互相問候的友好交談後,在一旁坐下。在一旁坐下的傷歌邏婆羅門對世尊說這個:

   「\twnr{喬達摩}{80.0}尊師!什麼因、什麼\twnr{緣}{180.0},以那個,有時,長時間誦讀的經文也不在心中出現,更何況沒誦讀的?喬達摩尊師!什麼因、什麼緣,以那個,有時,長時間沒誦讀的經文也在心中出現,更何況誦讀的?」

  「婆羅門!凡在以被欲貪纏縛的、以被欲貪征服的心而住時,而不如實知道已生起欲貪的\twnr{出離}{294.0},那時,不如實知見(如實不知不見)自己的利益;那時,也不如實知見他人的利益;那時,也不如實知見兩者的利益,長時間誦讀的經文也不在心中出現,更不用說沒誦讀的。

  婆羅門!猶如水鉢被胭脂或薑黃或蓼藍或紫檀[等染料]參雜,在那裡,當有眼的男子觀察自己的面相時,不如實知見。同樣的,婆羅門!凡在以被欲貪纏縛的、以被欲貪征服的心而住時,而不如實知道已生起欲貪的出離,那時,不如實知見自己的利益……(中略)那時,也不如實知見兩者的利益,長時間誦讀的經文也不在心中出現,更不用說沒誦讀的。

  再者,婆羅門!凡在以被惡意纏縛的、以被惡意征服的心而住時,而不如實知道已生起惡意的出離,那時,不如實知見自己的利益……(中略)那時,也不如實知見兩者的利益,長時間誦讀的經文也不在心中出現,更不用說沒誦讀的。

  婆羅門!猶如水鉢被火加熱、被沸騰、成為滿溢的,在那裡,當有眼的男子觀察自己的面相時,不如實知見。同樣的,婆羅門!凡在以被惡意纏縛的、以被惡意征服的心而住時,而不如實知道已生起惡意的出離,那時,不如實知見自己的利益……(中略)那時,也不如實知見兩者的利益,長時間誦讀的經文也不在心中出現,更不用說沒誦讀的。

  再者,婆羅門!凡在以被惛沈睡眠纏縛的、以被惛沈睡眠征服的心而住時,而不如實知道已生起惛沈睡眠的出離,那時,不如實知見自己的利益……(中略)那時,也不如實知見兩者的利益,長時間誦讀的經文也不在心中出現,更不用說沒誦讀的。

  婆羅門!猶如水鉢被苔蘚水藻覆蓋,在那裡,當有眼的男子觀察自己的面相時,不如實知見。同樣的,婆羅門!凡在以被惛沈睡眠纏縛的、以被惛沈睡眠征服的心而住時,而不如實知道已生起惛沈睡眠的出離,那時,不如實知見自己的利益……(中略)那時,也不如實知見兩者的利益,長時間誦讀的經文也不在心中出現,更不用說沒誦讀的。

  再者,婆羅門!凡在以被掉舉後悔纏縛的、以被掉舉後悔征服的心而住時,而不如實知道已生起掉舉後悔的出離,那時,不如實知見自己的利益……(中略)那時,也不如實知見兩者的利益,長時間誦讀的經文也不在心中出現,更不用說沒誦讀的。

  婆羅門!猶如水鉢被風吹動、被搖動、被旋轉、被生起波,在那裡,當有眼的男子觀察自己的面相時,不如實知見。同樣的,婆羅門!凡在以被掉舉後悔纏縛的、以被掉舉後悔征服的心而住時,而不如實知道已生起掉舉後悔的出離,那時,不如實知見自己的利益……(中略)那時,也不如實知見兩者的利益,長時間誦讀的經文也不在心中出現,更不用說沒誦讀的。

  再者,婆羅門!凡在以被疑惑纏縛的、以被疑惑征服的心而住時,而不如實知道已生起疑惑的出離,那時,不如實知見自己的利益……(中略)那時,也不如實知見兩者的利益,長時間誦讀的經文也不在心中出現,更不用說沒誦讀的。

  婆羅門!猶如水鉢是混濁的、擾動的、變成泥沼的、放置在黑暗處的,在那裡,當有眼的男子觀察自己的面相時,不如實知見。同樣的,婆羅門!凡在以被疑惑纏縛的、以被疑惑征服的心而住時,而不如實知道已生起疑惑的出離,那時,不如實知見自己的利益;那時,也不如實知見他人的利益;那時,也不如實知見兩者的利益,長時間誦讀的經文也不在心中出現,更不用說沒誦讀的。

  婆羅門!這是因、這是緣,以那個,有時,長時間誦讀的經文不在心中出現,更不用說沒誦讀的。

  婆羅門!凡在以不被欲貪纏縛的、以不被欲貪征服的心而住時,而如實知道已生起欲貪的出離,那時,如實知見自己的利益;那時,也如實知見他人的利益;那時,也如實知見兩者的利益,長時間沒誦讀的經文也在心中出現,更不用說誦讀的。

  婆羅門!猶如水鉢不被胭脂或薑黃或蓼藍或紫檀參雜,在那裡,當有眼的男子觀察自己的面相時,如實知見。同樣的,婆羅門!凡在以不被欲貪纏縛的、以不被欲貪征服的心而住時,而如實知道已生起欲貪的出離……(中略)。

  再者,婆羅門!凡在以不被惡意纏縛的、以不被惡意征服的心而住時,而如實知道已生起惡意的出離,那時,如實知見自己的利益……(中略)他人的利益……兩者的利益,長時間沒誦讀的經文也在心中出現,更不用說誦讀的。

    婆羅門!猶如水鉢沒被火加熱、沒被沸騰、沒成為滿溢的,在那裡,當有眼的男子觀察自己的面相時,如實知見。同樣的,婆羅門!凡在以不被惡意纏縛的、以不被惡意征服的心而住時,而如實知道已生起惡意的出離,那時,如實知見自己的利益……(中略)他人的利益……兩者的利益,長時間沒誦讀的經文也在心中出現,更不用說誦讀的。

  再者,婆羅門!凡在以不被惛沈睡眠纏縛的、以不被惛沈睡眠征服的心而住時,而如實知道已生起惛沈睡眠的出,那時,如實知見自己的利益……(中略)他人的利益……兩者的利益,長時間沒誦讀的經文也在心中出現,更不用說誦讀的。

  婆羅門!猶如水鉢沒被苔蘚水藻覆蓋,在那裡,當有眼的男子觀察自己的面相時,如實知見。同樣的,婆羅門!凡在以不被惛沈睡眠纏縛的、以不被惛沈睡眠征服的心而住時,而如實知道已生起惛沈睡眠的出離,那時,如實知見自己的利益……(中略)他人的利益……兩者的利益,長時間沒誦讀的經文也在心中出現,更不用說誦讀的。

  再者,婆羅門!凡在以不被掉舉後悔纏縛的、以不被掉舉後悔征服的心而住時,而如實知道已生起掉舉後悔的出離,那時,如實知見自己的利益……(中略)他人的利益……兩者的利益,長時間沒誦讀的經文也在心中出現,更不用說誦讀的。

  婆羅門!猶如水鉢沒被風吹動、沒被搖動、沒被旋轉、沒被生起波,在那裡,當有眼的男子觀察自己的面相時,如實知見。同樣的,婆羅門!凡在以不被掉舉後悔纏縛的、以不被掉舉後悔征服的心而住時,而如實知道已生起掉舉後悔的出離,那時,如實知見自己的利益……(中略)他人的利益……兩者的利益,長時間沒誦讀的經文也在心中出現,更不用說誦讀的。

  再者,婆羅門!凡在以不被疑惑纏縛的、以不被疑惑征服的心而住時,而如實知道已生起疑惑的出離,那時,如實知見自己的利益;那時,也如實知見他人的利益;那時,也如實知見兩者的利益,長時間沒誦讀的經文也在心中出現,更不用說誦讀的。

  婆羅門!猶如水鉢是清澈的、明淨的、不混濁的、放置在光明處的,在那裡,當有眼的男子觀察自己的面相時,如實知見。同樣的,婆羅門!凡在以不被疑惑纏縛的、以不被疑惑征服的心而住時,而如實知道已生起疑惑的出離,那時,如實知見自己的利益;那時,也如實知見他人的利益;那時,也如實知見兩者的利益,長時間沒誦讀的經文也在心中出現,更不用說誦讀的。

  婆羅門!這是因、這是緣,以那個,有時,長時間沒誦讀的經文在心中出現,更不用說誦讀的。

  婆羅門!有這些七覺支是非障礙的、非蓋的、心的非隨雜染的,已\twnr{修習}{94.0}、已\twnr{多作}{95.0},轉起明解脫果的作證,哪七個?婆羅門!\twnr{念覺支}{315.0}是非障礙的、非蓋的、心的非隨雜染的,已修習、已多作,轉起明解脫果的作證……(中略)婆羅門!\twnr{平靜覺支}{314.0}是非障礙的、非蓋的、心的非隨雜染的,已修習、已多作,轉起明解脫果的作證。婆羅門!這些七覺支是非障礙的、非蓋的、心的非隨雜染的,已修習、已多作,轉起明解脫果的作證。」

  在這麼說時,傷歌邏婆羅門對世尊說這個:

  「太偉大了,喬達摩尊師!……(中略)請喬達摩\twnr{尊師}{203.0}記得我為\twnr{優婆塞}{98.0},從今天起\twnr{已終生歸依}{64.0}。」[\ccchref{AN.5.193}{https://agama.buddhason.org/AN/an.php?keyword=5.193}]



\sutta{56}{56}{無畏經}{https://agama.buddhason.org/SN/sn.php?keyword=46.56}
  \twnr{被我這麼聽聞}{1.0}:

  \twnr{有一次}{2.0},\twnr{世尊}{12.0}住在王舍城\twnr{耆闍崛山}{258.0}。

  那時,無畏王子去見世尊。抵達後,向世尊\twnr{問訊}{46.0}後,在一旁坐下。在一旁坐下的無畏王子對世尊說這個:

  「\twnr{大德}{45.0}!富蘭那迦葉說這個:『對無智無見來說,沒有因沒有\twnr{緣}{180.0};無智無見是無因無緣的,對\twnr{智見}{433.0}來說,沒有因沒有緣;智見是無因無緣的。』這裡,世尊怎麼說?」

  「王子!對無智無見來說,有因有緣;無智無見是有因有緣的,王子!對智見來說,有因有緣;智見是有因有緣的。」

  「大德!那麼,對無智無見來說,什麽是因?什麽是緣?無智無見是怎樣有因有緣的呢?」

  「王子!凡在以心被欲貪纏縛、被欲貪征服、對已生起的欲貪不如實知見(如實不知不見)\twnr{出離}{294.0}而住時,王子!對無智無見,這是因,這是緣,無智無見是這樣有因有緣的。

  再者,王子!凡在以心被惡意纏縛、被惡意征服……(中略)被惛沈睡眠纏縛……被掉舉後悔纏縛……以心被疑惑纏縛、被疑惑征服、對已生起的疑惑不如實知見出離而住時,王子!又,對無智無見,這是因,這是緣,又,無智無見是這樣有因有緣的。」

  「大德!這\twnr{法的教說}{562.1}是什麼名字呢?」

  「王子!這些名字是蓋。」

  「世尊!確實是蓋,\twnr{善逝}{8.0}!確實是蓋。大德!即使被逐一的蓋征服就會不如實知見,更不用說五蓋。

  大德!那麼,對智見來說,什麽是因?什麽是緣?智見是怎樣有因有緣的呢?」

  「王子!這裡,\twnr{比丘}{31.0}\twnr{依止遠離}{322.0}、依止離貪、依止滅、\twnr{捨棄的成熟}{221.0}修習\twnr{念覺支}{315.0},以心已修習念覺支而如實知見,王子!對智見,這是因,這是緣,智見是這樣有因有緣的。

  再者,王子!比丘……(中略)依止遠離、依止離貪、依止滅、捨棄的成熟修習\twnr{平靜覺支}{314.0},以心已修習平靜覺支而如實知見,王子!又,對智見,這是因,這是緣,又,智見是這樣有因有緣的。」

  「大德!這法的教說是什麼名字呢?」

  「王子!這名字是覺支。」

  「世尊!確實是覺支,善逝!確實是覺支。大德!即使具備逐一覺支就會如實知見,更不用說七覺支。

  大德!凡我登攀耆闍崛山的身疲勞、心疲勞,我的那個都已止息,法被我\twnr{現觀}{53.0}了。」

  交談品第六,其\twnr{攝頌}{35.0}:

  「食、法門、火,慈與傷歌邏,

   無畏詢問的問題:在耆闍崛山。」





\pin{入出息品}{57}{66}
\sutta{57}{57}{骨之大果經}{https://agama.buddhason.org/SN/sn.php?keyword=46.57}
i.

  起源於舍衛城。

  「\twnr{比丘}{31.0}們!骨想已\twnr{修習}{94.0}、已\twnr{多作}{95.0},轉起大果、\twnr{大效益}{113.0}。

  比丘們!而骨想怎樣已修習、怎樣已多作,轉起大果、大效益?比丘們!這裡,比丘與骨想俱行,\twnr{依止遠離}{322.0}、依止離貪、依止滅、\twnr{捨棄的成熟}{221.0}修習\twnr{念覺支}{315.0}……(中略)與骨想俱行,依止遠離、依止離貪、依止滅、捨棄的成熟修習\twnr{平靜覺支}{314.0},骨想這樣已修習、這樣已多作,轉起大果、大效益。」

ii.二果其中之一經

  「比丘們!在骨想已修習、已多作時,二果其中之一果能被預期:當生\twnr{完全智}{489.0},或在存在\twnr{有餘依}{323.0}時,為\twnr{阿那含}{209.0}位。比丘們!而骨想怎樣已修習、怎樣已多作,二果其中之一果能被預期:當生完全智,或在存在有餘依時,為不還者狀態?比丘們!這裡,比丘與骨想俱行……修習念覺支……(中略)與骨想俱行,依止遠離、依止離貪、依止滅、捨棄的成熟修習平靜覺支,骨想這樣已修習、這樣已多作,二果其中之一果能被預期:當生完全智,或在存在有餘依時,為不還者狀態。」

iii.大利益經

  「比丘們!骨想已修習,已多作,轉起大利益。比丘們!而骨想怎樣已修習、怎樣已多作,轉起大利益?比丘們!這裡,比丘與骨想俱行……修習念覺支……(中略)與骨想俱行,依止遠離、依止離貪、依止滅、捨棄的成熟修習平靜覺支,骨想這樣已修習、這樣已多作,轉起大利益。」

iv.軛安穩經

  「比丘們!骨想已修習,已多作,轉起\twnr{軛安穩}{192.0}。比丘們!而骨想怎樣已修習、怎樣已多作,轉起軛安穩?比丘們!這裡,比丘與骨想俱行……修習念覺支……(中略)與骨想俱行,依止遠離、依止離貪、依止滅、捨棄的成熟修習平靜覺支,骨想這樣已修習、這樣已多作,轉起軛安穩。」

v.急迫感經

  「比丘們!骨想已修習,已多作,轉起大\twnr{急迫感}{373.0}。比丘們!而骨想怎樣已修習、怎樣已多作,轉起大急迫感?比丘們!這裡,比丘與骨想俱行……修習念覺支……(中略)與骨想俱行,依止遠離、依止離貪、依止滅、捨棄的成熟修習平靜覺支,骨想這樣已修習、這樣已多作,轉起大急迫感。」

vi.安樂住經

  「比丘們!骨想已修習,已多作,轉起大\twnr{安樂住}{156.0}。比丘們!而骨想怎樣已修習、怎樣已多作,轉起大安樂住?比丘們!這裡,比丘與骨想俱行……修習念覺支……(中略)與骨想俱行,依止遠離、依止離貪、依止滅、捨棄的成熟修習平靜覺支,骨想這樣已修習、這樣已多作,轉起大安樂住。」



\sutta{58}{58}{蟲經}{https://agama.buddhason.org/SN/sn.php?keyword=46.58}
  「\twnr{比丘}{31.0}們!蟲聚想已\twnr{修習}{94.0}……(中略)。」



\sutta{59}{59}{青瘀經}{https://agama.buddhason.org/SN/sn.php?keyword=46.59}
  「\twnr{比丘}{31.0}們!青瘀想……(中略)。」



\sutta{60}{60}{斷壞經}{https://agama.buddhason.org/SN/sn.php?keyword=46.60}
  「\twnr{比丘}{31.0}們!斷壞想……(中略)。」



\sutta{61}{61}{腫脹經}{https://agama.buddhason.org/SN/sn.php?keyword=46.61}
  「\twnr{比丘}{31.0}們!腫脹想……(中略)。」



\sutta{62}{62}{慈經}{https://agama.buddhason.org/SN/sn.php?keyword=46.62}
  「\twnr{比丘}{31.0}們!\twnr{慈}{324.0}已\twnr{修習}{94.0}……(中略)。」



\sutta{63}{63}{悲經}{https://agama.buddhason.org/SN/sn.php?keyword=46.63}
  「\twnr{比丘}{31.0}們!\twnr{悲}{324.0}已\twnr{修習}{94.0}……(中略)。」



\sutta{64}{64}{喜悅經}{https://agama.buddhason.org/SN/sn.php?keyword=46.64}
  「\twnr{比丘}{31.0}們!\twnr{喜悅}{324.0}已\twnr{修習}{94.0}……(中略)。」



\sutta{65}{65}{平靜經}{https://agama.buddhason.org/SN/sn.php?keyword=46.65}
  「\twnr{比丘}{31.0}們!\twnr{平靜}{324.0}已\twnr{修習}{94.0}……(中略)。」



\sutta{66}{66}{入出息經}{https://agama.buddhason.org/SN/sn.php?keyword=46.66}
  「\twnr{比丘}{31.0}們!\twnr{入出息念}{329.0}已\twnr{修習}{94.0}……(中略)。」

  入出息品第七,其\twnr{攝頌}{35.0}:

  「骨、蟲、青瘀,斷壞、以腫脹為第五,

   慈、悲、喜悅、平靜,與入出息它們為十。」





\pin{滅品}{67}{76}
\sutta{67}{67}{不淨經}{https://agama.buddhason.org/SN/sn.php?keyword=46.67}
  「\twnr{比丘}{31.0}們!不淨想……(中略)。」



\sutta{68}{68}{死經}{https://agama.buddhason.org/SN/sn.php?keyword=46.68}
  「\twnr{比丘}{31.0}們!\twnr{死想}{545.0}……(中略)。」



\sutta{69}{69}{在食上厭逆經}{https://agama.buddhason.org/SN/sn.php?keyword=46.69}
  「\twnr{比丘}{31.0}們!在食上\twnr{厭逆}{227.0}想……(中略)。」



\sutta{70}{70}{不樂經}{https://agama.buddhason.org/SN/sn.php?keyword=46.70}
  「\twnr{比丘}{31.0}們!在一切世間上不樂想……(中略)。」



\sutta{71}{71}{無常經}{https://agama.buddhason.org/SN/sn.php?keyword=46.71}
  「\twnr{比丘}{31.0}們!無常想……(中略)。」



\sutta{72}{72}{苦經}{https://agama.buddhason.org/SN/sn.php?keyword=46.72}
  「\twnr{比丘}{31.0}們!在無常上苦想……(中略)。」



\sutta{73}{73}{無我經}{https://agama.buddhason.org/SN/sn.php?keyword=46.73}
  「\twnr{比丘}{31.0}們!在苦上無我想……(中略)。」



\sutta{74}{74}{捨斷經}{https://agama.buddhason.org/SN/sn.php?keyword=46.74}
  「\twnr{比丘}{31.0}們!捨斷想……(中略)。」



\sutta{75}{75}{離貪經}{https://agama.buddhason.org/SN/sn.php?keyword=46.75}
  「\twnr{比丘}{31.0}們!離貪想……(中略)。」



\sutta{76}{76}{滅經}{https://agama.buddhason.org/SN/sn.php?keyword=46.76}
  「\twnr{比丘}{31.0}們!滅想已\twnr{修習}{94.0}、已\twnr{多作}{95.0},有大果、\twnr{大效益}{113.0}。

  比丘們!而滅想怎樣已修習、怎樣已多作,有大果、大效益?比丘們!這裡,比丘與滅想俱行……修習\twnr{念覺支}{315.0}……(中略)與滅想俱行,\twnr{依止遠離}{322.0}、依止離貪、依止滅、\twnr{捨棄的成熟}{221.0}修習\twnr{平靜覺支}{314.0},滅想這樣已修習、這樣已多作,有大果、大效益。

  比丘們!在滅想已修習,已多作時,二果其中之一果能被預期:\twnr{當生}{42.0}\twnr{完全智}{489.0},或在存在\twnr{有餘依}{323.0}時,為\twnr{阿那含}{209.0}位。

  比丘們!而在滅想怎樣已修習、怎樣已多作時,二果其中之一果能被預期:當生完全智,或在存在有餘依時,為阿那含位?比丘們!這裡,比丘與滅想俱行……修習念覺支……(中略)與滅想俱行,依止遠離、依止離貪、依止滅、捨棄的成熟修習平靜覺支,滅想這樣已修習、這樣已多作,二果其中之一果能被預期:當生完全智,或在存在有餘依時,為阿那含位。

  比丘們!滅想已修習,已多作,轉起大利益、轉起\twnr{軛安穩}{192.0}、轉起大\twnr{急迫感}{373.0}、轉起大\twnr{安樂住}{156.0}。

  比丘們!而滅想怎樣已修習、怎樣已多作,轉起大利益、轉起軛安穩、轉起大急迫感、轉起大安樂住?比丘們!這裡,比丘依與滅想俱行……修習念覺支……(中略)與滅想俱行,依止遠離、依止離貪、依止滅、捨棄的成熟修習平靜覺支,滅想這樣已修習、這樣已多作,轉起大利益、轉起軛安穩、轉起大急迫感、轉起大安樂住。」

  滅品第八,其\twnr{攝頌}{35.0}:

  「不淨、死、在食上厭逆,以不歡喜,

   無常、苦、無我、捨斷,與離貪、滅它們為十。」





\pin{恒河中略品}{77}{88}
\sutta{77}{88}{恒河等經}{https://agama.buddhason.org/SN/sn.php?keyword=46.77}
  「\twnr{比丘}{31.0}們!猶如恒河是傾向東的、斜向東的、坡斜向東的。同樣的,比丘們!\twnr{修習}{94.0}\twnr{七覺支}{524.0}、\twnr{多作}{95.0}七覺支的比丘是傾向涅槃的、斜向涅槃的、坡斜向涅槃的。

  比丘們!而怎樣修習七覺支、多作七覺支的比丘是傾向涅槃的、斜向涅槃的、坡斜向涅槃的?比丘們!這裡,比丘\twnr{依止遠離}{322.0}、依止離貪、依止滅、\twnr{捨棄的成熟}{221.0}修習\twnr{念覺支}{315.0}……(中略)依止遠離、依止離貪、依止滅、捨棄的成熟修習\twnr{平靜覺支}{314.0}。……(中略)比丘們!這樣修習七覺支、多作七覺支的比丘是傾向涅槃的、斜向涅槃的、坡斜向涅槃的。」(應該如尋求經\twnr{那樣使之被細說}{x613})

  恒河中略品第九,其\twnr{攝頌}{35.0}:

  「六則傾向東的,與六則傾向大海的,

   這兩個六則成十二則,以那個被稱為品。」





\pin{不放逸品}{89}{98}
\sutta{89}{98}{如來等經}{https://agama.buddhason.org/SN/sn.php?keyword=46.89}
  「\twnr{比丘}{31.0}們!眾生之所及:無足的、二足的、四足的、多足的……。[\suttaref{SN.45.139}-148]」應該使之被細說。

  不放逸品第十,其\twnr{攝頌}{35.0}:

  「如來、足跡、屋頂,根、樹心、茉莉花,

   王、月、日,以衣服為第十句。」

  (覺支相應的不放逸品應該以覺支使之被細說)





\pin{力量所作品}{99}{110}
\sutta{99}{110}{力量等經}{https://agama.buddhason.org/SN/sn.php?keyword=46.99}
  「\twnr{比丘}{31.0}們!猶如當任何應該被力量作的工作被作時……(中略)。[\suttaref{SN.45.149}-160]」

  力量所作品第十一,其\twnr{攝頌}{35.0}:

  「力量、種子與龍,樹木、瓶子及穗,

   虛空與二則雨雲,船、屋舍、河。」

  (覺支相應的力量所作品應該以覺支使之被細說)





\pin{尋求品}{111}{120}
\sutta{111}{120}{尋求等經}{https://agama.buddhason.org/SN/sn.php?keyword=46.111}
  「\twnr{比丘}{31.0}們!有這些三種尋求,哪三個?欲的尋求、有的尋求、\twnr{梵行的尋求}{381.1}……。[\suttaref{SN.45.161}-171]」應該使之被細說。

   尋求品第十二,其\twnr{攝頌}{35.0}:

  「尋求、慢、漏,有與三苦性,

   荒蕪、垢、惱亂,受、渴愛與渴望。」

  (覺支相應的尋求品應該以覺支使之被細說)





\pin{暴流品}{121}{130}
\sutta{121}{129}{暴流等經}{https://agama.buddhason.org/SN/sn.php?keyword=46.121}
  「\twnr{比丘}{31.0}們!有這四種暴流,哪四種?欲的暴流、\twnr{有的暴流}{369.0}、見的暴流、\twnr{無明}{207.0}的暴流……。[\suttaref{SN.45.172}-\suttaref{SN.45.180}]」應該使之被細說。



\sutta{130}{130}{上分經}{https://agama.buddhason.org/SN/sn.php?keyword=46.130}
  起源於舍衛城。

  「\twnr{比丘}{31.0}們!有這些五上分結,哪五個?色貪、無色貪、慢、掉舉、\twnr{無明}{207.0},比丘們!這些是五上分結。比丘們!為了這些五上分結的證智、\twnr{遍知}{154.0}、遍盡、捨斷,\twnr{七覺支}{524.0}應該被\twnr{修習}{94.0}。哪七個?比丘們!這裡,比丘\twnr{依止遠離}{322.0}、依止離貪、依止滅、\twnr{捨棄的成熟}{221.0}修習\twnr{念覺支}{315.0}……(中略)比丘依止遠離、依止離貪、依止滅、捨棄的成熟修習\twnr{平靜覺支}{314.0}。有:貪之調伏的完結、瞋之調伏的完結、癡之調伏的完結……(中略)有\twnr{不死}{123.0}的立足處、不死的\twnr{彼岸}{226.0}、不死的完結……(中略)比丘們!為了這些五上分結的證智、遍知、遍盡、捨斷,這七覺支應該被修習。」

  暴流品第十三,其\twnr{攝頌}{35.0}:

  「暴流、軛、取,繫縛、煩惱潛在趨勢,

   欲種類、蓋,蘊、下上分。」





\pin{再一個恒河中略品}{131}{142}
\sutta{131}{142}{再一個恒河等經}{https://agama.buddhason.org/SN/sn.php?keyword=46.131}
  [再一個恒河中略品]第十四,其\twnr{攝頌}{35.0}:

  「六則傾向東的,與六則傾向大海的,

   這兩個六則成十二則,以那個被稱為品。」

   (覺支相應的恒河中略品應該以貪的影響使之被細說) 





\pin{再一個不放逸品}{143}{152}
\sutta{143}{152}{如來等經}{https://agama.buddhason.org/SN/sn.php?keyword=46.143}
  [再一個不放逸品]第十五,其\twnr{攝頌}{35.0}:

  「如來、足跡、屋頂,根、樹心、茉莉花,

   王、月、日,以衣服為第十句。」

  (不放逸品應該以貪的影響使之被細說)





\pin{再一個力量所作品}{153}{164}
\sutta{153}{164}{再一個力量等經}{https://agama.buddhason.org/SN/sn.php?keyword=46.153}
  [再一個力量所作品]第十六,其\twnr{攝頌}{35.0}:

  「力量、種子與龍,樹木、瓶子及穗,

   虛空與二則雨雲,船、屋舍、河。」

  (覺支相應的力量所作品應該以貪的影響使之被細說)





\pin{再一個尋求品}{165}{174}
\sutta{165}{174}{再一個尋求等經}{https://agama.buddhason.org/SN/sn.php?keyword=46.165}
   再一個尋求品第十七,其\twnr{攝頌}{35.0}:

  「尋求、慢、漏,有與三苦性,

   荒蕪、垢、惱亂,受、渴愛與渴望。」





\pin{再一個暴流品}{175}{184}
\sutta{175}{184}{再一個暴流等經}{https://agama.buddhason.org/SN/sn.php?keyword=46.175}
  覺支相應的暴流品第十八,其\twnr{攝頌}{35.0}:

  「暴流、軛、取,繫縛、煩惱潛在趨勢,

   欲種類、蓋,蘊、下上分。」

  (應該以有貪之調伏的完結、瞋之調伏的完結、癡之調伏的完結使之被細說)(凡道相應應該使之被細說的,都應該如覺支相使之被細說)

  覺支相應第二。





\page

\xiangying{47}{念住相應}
\pin{蓭婆巴利品}{1}{10}
\sutta{1}{1}{蓭婆巴利經}{https://agama.buddhason.org/SN/sn.php?keyword=47.1}
  \twnr{被我這麼聽聞}{1.0}:

  \twnr{有一次}{2.0},\twnr{世尊}{12.0}住在毘舍離蓭婆巴利的園林。

  在那裡,世尊召喚\twnr{比丘}{31.0}們:「比丘們!」

  「\twnr{尊師}{480.0}!」那些比丘回答世尊。

  世尊說這個:

  「比丘們!為了眾生的清淨、為了愁悲的超越、為了苦憂的滅沒、為了方法的獲得、為了涅槃的作證,這是\twnr{無岔路之道}{565.0},即:\twnr{四念住}{286.0},哪四個?比丘們!這裡,比丘住於\twnr{在身上隨看著身}{176.0}:熱心的、正知的、有念的,調伏世間中的\twnr{貪婪}{435.0}、憂後;住於在諸受上隨看著受:熱心的、正知的、有念的,調伏世間中的貪婪、憂後;住於在心上隨看著心:熱心的、正知的、有念的,調伏世間中的貪婪、憂後;住於在諸法上隨看著法:熱心的、正知的、有念的,調伏世間中的貪婪、憂後。比丘們!為了眾生的清淨、為了愁悲的超越、為了苦憂的滅沒、為了方法的獲得、為了涅槃的作證,這是無岔路之道,即:四念住。」

  世尊說這個,那些悅意的比丘歡喜世尊的所說。



\sutta{2}{2}{念經}{https://agama.buddhason.org/SN/sn.php?keyword=47.2}
  \twnr{有一次}{2.0},\twnr{世尊}{12.0}住在毘舍離蓭婆巴利的園林。

  在那裡,世尊召喚\twnr{比丘}{31.0}們:「比丘們!」

  「\twnr{尊師}{480.0}!」那些比丘回答世尊。

  世尊說這個:

  「比丘們!比丘應該住於具念的、正知的,這是我們為你們的教誡。

  比丘們!而怎樣比丘是具念的?比丘們!這裡,比丘住於\twnr{在身上隨看著身}{176.0}:熱心的、正知的、有念的,調伏世間中的\twnr{貪婪}{435.0}、憂後;在諸受上……(中略)在心上……(中略)住於在諸法上隨看著法:熱心的、正知的、有念的,調伏世間中的貪婪、憂後。比丘們!這樣,比丘是具念的。

  比丘們!而怎樣比丘是正知的?比丘們!這裡,比丘在前進後退時是\twnr{正知的行為者}{544.0};在前視環視時是正知的行為者;在[肢體]屈伸時是正知的行為者;在\twnr{大衣}{270.0}、鉢、衣服的受持時是正知的行為者;在飲、食、嚼、嚐時是正知的行為者;在大小便動作時是正知的行為者;在行、住、坐、臥、清醒、語、默狀態時是正知的行為者。比丘們!這樣,比丘是正知的。

  比丘們!比丘應該住於具念的、正知的,這是我們為你們的教誡。」[\ccchref{DN.16}{https://agama.buddhason.org/DN/dm.php?keyword=16}, 160段]



\sutta{3}{3}{比丘經}{https://agama.buddhason.org/SN/sn.php?keyword=47.3}
  \twnr{有一次}{2.0},\twnr{世尊}{12.0}住在舍衛城祇樹林給孤獨園。

  那時,某位比丘去見世尊。抵達後,向世尊\twnr{問訊}{46.0}後,在一旁坐下。在一旁坐下的那位比丘對世尊說這個:

  「\twnr{大德}{45.0}!請世尊為我簡要地教導法,凡我聽聞世尊的法後,會住於單獨的、隱離的、不放逸的、熱心的、自我努力的,\twnr{那就好了}{44.0}!」

  「可是,這裡,一些無用的男子只這樣請求我,在法被說時,他們只想我應該被跟隨。」

  「大德!請世尊為我簡要地教導法,請\twnr{善逝}{8.0}簡要地教導法,也許我會了知世尊所說的義理,也許我會成為世尊所說的繼承人。」  「比丘!因此,在這裡,請你就在最初的諸善法上淨化。而什麼是最初的諸善法呢?善清淨的戒與正直的見。

  比丘!當如果你有善清淨的戒與正直的見,比丘!之後,依止戒、在戒上住立後,你應該以三種\twnr{修習}{94.0}\twnr{四念住}{286.0},哪四個?

  比丘!這裡,請你住於對內\twnr{在身上隨看著身}{176.0}:熱心的、正知的、有念的,調伏世間中的\twnr{貪婪}{435.0}、憂後;或住於在外部的身上隨看著身:熱心的、正知的、有念的,調伏世間中的貪婪、憂後;或住於\twnr{在內外的身上隨看著身}{271.0}:熱心的、正知的、有念的,調伏世間中的貪婪、憂後。

  對內在諸受上……(中略)或對外在諸受上……(中略)或住於對內外在諸受上隨看著受:熱心的、正知的、有念的,調伏世間中的貪婪、憂後。

  對內在心上……(中略)或對外在心上……(中略)或住於對內外在心上隨看著心:熱心的、正知的、有念的,調伏世間中的貪婪、憂後。

  對內在諸法上……(中略)或對外在諸法上……(中略)或住於對內外在諸法上隨看著法:熱心的、正知的、有念的,調伏世間中的貪婪、憂後。

  比丘!當依止戒、在戒上住立後,如果你以三種這樣修習這四念住時,比丘!之後,對你來說,不論日或夜到來,在善法上僅增長能被預期,非減損。」

  那時,那位比丘歡喜、\twnr{隨喜}{85.0}世尊所說後,從座位起來、向世尊問訊、\twnr{作右繞}{47.0}後,離開。

  那時,住於單獨的、隱離的、不放逸的、熱心的、自我努力的那位比丘不久就以證智自作證後,在當生中\twnr{進入後住於}{66.0}凡\twnr{善男子}{41.0}們為了利益正確地\twnr{從在家出家成為無家者}{48.0}的那個無上梵行結尾,他證知:「\twnr{出生已盡}{18.0},\twnr{梵行已完成}{19.0},\twnr{應該被作的已作}{20.0},\twnr{不再有此處[輪迴]的狀態}{21.1}。」然後那位比丘成為眾\twnr{阿羅漢}{5.0}之一。[\suttaref{SN.47.16}]



\sutta{4}{4}{薩羅經}{https://agama.buddhason.org/SN/sn.php?keyword=47.4}
  \twnr{有一次}{2.0},\twnr{世尊}{12.0}住在憍薩羅國的薩羅\twnr{婆羅門}{17.0}村落。

  在那裡,世尊召喚\twnr{比丘}{31.0}們……(中略)說這個:

  「比丘們!凡那些出家不久、最近來此法律的新比丘,比丘們!那些比丘應該被你們為了\twnr{四念住}{286.0}的\twnr{修習}{94.0}勸導、應該被使確立、應該被使建立,對哪四個?

  來!學友們!請你們為了身的如實智住於\twnr{在身上隨看著身}{176.0},熱心的、正知的、\twnr{成為專一的}{x614}、\twnr{心明淨的}{x615}、入定的、\twnr{心一境}{255.0}的;請你們為了諸受的如實智住於在諸受上隨看著受,熱心的、正知的、成為專一的、心明淨的、入定的、心一境的;請你們為了心的如實智住於在心上隨看著心,熱心的、正知的、成為專一的、心明淨的、入定的、心一境的;請你們為了諸法的如實智住於在諸法上隨看著法,熱心的、正知的、成為專一的、心明淨的、入定的、心一境的。

  比丘們!凡那些心意未達成、住於希求著無上\twnr{軛安穩}{192.0}的\twnr{有學}{193.0}比丘,他們也為了身的\twnr{遍知}{154.0}住於在身上隨看著身,熱心的、正知的、成為專一的、心明淨的、入定的、心一境的;為了諸受的遍知住於在諸受上隨看著受,熱心的、正知的、成為專一的、心明淨的、入定的、心一境的;為了心的遍知住於在心上隨看著心,熱心的、正知的、成為專一的、心明淨的、入定的、心一境的;為了諸法的遍知住於在諸法上隨看著法,熱心的、正知的、成為專一的、心明淨的、入定的、心一境的。

  比丘們!凡那些漏已滅盡的、已完成的、\twnr{應該被作的已作的}{20.0}、負擔已卸的、\twnr{自己的利益已達成的}{189.0}、\twnr{有之結已滅盡的}{190.0}、以\twnr{究竟智}{191.0}解脫的阿羅漢比丘,他們也身離結縛地住於在身上隨看著身,熱心的、正知的、成為專一的、心明淨的、入定的、心一境的;諸受離結縛地住於在諸受上隨看著受,熱心的、正知的、成為專一的、心明淨的、入定的、心一境的;心離結縛地住於在心上隨看著心,熱心的、正知的、成為專一的、心明淨的、入定的、心一境的;諸法離結縛地住於在諸法上隨看著法,熱心的、正知的、成為專一的、心明淨的、入定的、心一境的。

  比丘們!凡那些出家不久、最近來此法律的新比丘,比丘們!那些比丘應該被你們對四念住的修習勸導、應該被使確立、應該被使建立。」



\sutta{5}{5}{不善聚經}{https://agama.buddhason.org/SN/sn.php?keyword=47.5}
  起源於舍衛城。

  在那裡,\twnr{世尊}{12.0}說這個:

  「\twnr{比丘}{31.0}們!當說『不善聚』時,當正確說時,應該說\twnr{五蓋}{287.0}。比丘們!因為這是完全的不善聚,即:五蓋,哪五個?\twnr{欲的意欲}{118.0}蓋、惡意蓋、惛沈睡眠蓋、掉舉後悔蓋、疑惑蓋。比丘們!當說『不善聚』時,當正確說時,應該說五蓋。比丘們!因為這是完全的不善聚,即:五蓋。

  比丘們!當說『善聚』時,當正確說時,應該說\twnr{四念住}{286.0}。比丘們!因為這是完全的善聚,即:四念住,哪四個?比丘們!這裡,比丘住於\twnr{在身上隨看著身}{176.0}:熱心的、正知的、有念的,調伏世間中的\twnr{貪婪}{435.0}、憂後;在諸受上……(中略)在心上……(中略)住於在諸法上隨看著法:熱心的、正知的、有念的,調伏世間中的貪婪、憂後。比丘們!當說『善聚』時,當正確說時,應該說四念住。比丘們!因為這是完全的善聚,即:四念住。」



\sutta{6}{6}{鷹經}{https://agama.buddhason.org/SN/sn.php?keyword=47.6}
  「\twnr{比丘}{31.0}們!從前,鷹突然俯衝捉住鵪鶉鳥。比丘們!那時,當鵪鶉鳥被鷹帶走時,這麼悲泣:『我們就是不幸運,我們少福德:凡我們走\twnr{在非行境}{951.0}、他人的領域。如果現在我們走在行境、自己父親的領域,這隻鷹是不足以對我的,即:在戰鬥上。』『鵪鶉!那麼,什麼是你的行境、自己父親的領域呢?』『即:犁耕作過的土塊處。』

  比丘們!那時,在自己力量下不傲慢的、在自己力量下不同意的鷹釋放鵪鶉鳥:『鵪鶉!你走!即使到那裡後,你也將不脫離我。』比丘們!那時,鵪鶉鳥到犁耕作過的土塊處、登上大土塊後,站立說著:『現在對我來啊!鷹!現在對我來啊!鷹!』比丘們!那時,在自己力量下不傲慢的、在自己力量下不同意的鷹縮緊兩翼後,突然對鵪鶉鳥俯衝。比丘們!當鵪鶉鳥知道:『這隻鷹已很接近我。』時,就進入那個土塊的內側。比丘們!那時,鷹就在那裡撞擊胸部。比丘們!那確實是這樣:凡走在非行境、他人的領域者。

  比丘們!因此,在這裡,你們不要走在非行境、他人的領域。比丘們!在非行境、他人的領域行走者,魔將得到機會,魔將得到對象。比丘們!而什麼是比丘的非行境、他人的領域呢?即:\twnr{五種欲}{187.0},哪五個?能被眼識知的、想要的、所愛的、合意的、可愛形色的、伴隨欲的、誘人的諸色,能被耳識知的……(中略)諸聲音,能被鼻識知的……(中略)諸氣味,能被舌識知的……(中略)諸味道,能被身識知的、想要的、所愛的、合意的、可愛形色的、伴隨欲的、誘人的諸\twnr{所觸}{220.2}。比丘們!這是比丘的非行境、他人的領域。

  比丘們!你們要走在行境、自己父親的領域。比丘們!在行境、自己父親的領域行走者,魔將不得到機會,魔將不得到對象。比丘們!而什麼是比丘的行境、自己父親的領域呢?即:\twnr{四念住}{286.0},哪四個?比丘們!這裡,比丘住於\twnr{在身上隨看著身}{176.0}:熱心的、正知的、有念的,調伏世間中的\twnr{貪婪}{435.0}、憂後;在諸受上……(中略)在心上……(中略)住於在諸法上隨看著法:熱心的、正知的、有念的,調伏世間中的貪婪、憂後。比丘們!這是比丘的行境、自己父親的領域。」



\sutta{7}{7}{猴子經}{https://agama.buddhason.org/SN/sn.php?keyword=47.7}
  「\twnr{比丘}{31.0}們!喜馬拉雅山山王有難行、不平的區域,該處既非猴子也非人類的行走處。

  比丘們!喜馬拉雅山山王有難行、不平的區域,該處是猴子的行走處,非人類的。

  比丘們!喜馬拉雅山山王有平坦、能被喜樂的土地,該處是猴子的行走處,同時也是人類的。比丘們!在那裡,獵人們在猴子的路徑上佈置捕捉猴子的黏膠陷阱。

  比丘們!在那裡,凡那些不愚之類的、無動貪之類的猴子,看見那個黏膠後,牠們遠遠地避開,但凡牠是愚之類的、動貪之類的猴子,牠走近那個黏膠後,以手抓取,牠在那裡被捕捉。

  『我將使手脫離』:以第二隻手抓取,牠在那裡被捕捉。

  『我將使雙手脫離』:以腳抓取,牠在那裡被捕捉。

  『我將使雙手與腳脫離』:以第二隻腳抓取,牠在那裡被捕捉。

  『我將使雙手與雙腳脫離』:以嘴抓取,牠在那裡被捕捉。

  比丘們!這樣,那隻五處中陷阱呻吟的猴子躺下,來到不幸的、來到災厄的、獵人的為所欲為的:比丘們!獵人射擊牠後,{就在那塊木頭所作的炭火上放棄後}[就在那個上抓起、不釋放猴子後],往想去的地方出發。

  比丘們!那確實是這樣:凡走\twnr{在非行境}{951.0}、他人的領域者。 

  比丘們!因此,在這裡,你們不要走在非行境、他人的領域。比丘們!在非行境、他人的領域行走者,魔將得到機會,魔將得到對象。比丘們!而什麼是比丘的非行境、他人的領域呢?即:\twnr{五種欲}{187.0},哪五個?能被眼識知的、想要的、所愛的、合意的、可愛形色的、伴隨欲的、誘人的諸色,能被耳識知的……(中略)諸聲音,能被鼻識知的……諸氣味,能被舌識知的……諸味道,能被身識知的、想要的、所愛的、合意的、可愛形色的、伴隨欲的、誘人的諸\twnr{所觸}{220.2}。比丘們!這是比丘的非行境、他人的領域。

  比丘們!你們要走在行境、自己父親的領域。比丘們!在行境、自己父親的領域行走者,魔將不得到機會,魔將不得到對象。比丘們!而什麼是比丘的行境、自己父親的領域呢?即:\twnr{四念住}{286.0},哪四個?比丘們!這裡,比丘住於\twnr{在身上隨看著身}{176.0}:熱心的、正知的、有念的,調伏世間中的\twnr{貪婪}{435.0}、憂後;在諸受上……(中略)在心上……(中略)住於在諸法上隨看著法:熱心的、正知的、有念的,調伏世間中的貪婪、憂後。比丘們!這是比丘的行境、自己父親的領域。」



\sutta{8}{8}{廚師經}{https://agama.buddhason.org/SN/sn.php?keyword=47.8}
  「\twnr{比丘}{31.0}們!猶如愚笨、無能、\twnr{不善巧}{x616}的廚師以極種種的咖哩(湯汁)侍奉國王或國王的大臣:以酸為首的(酸為主的),也以苦為首的,也以辣為首的,也以甜為首的,\twnr{也以鹼的}{x617},也以無鹼的,也以鹹的,也以不鹹的。

  比丘們!那位愚笨、無能、不善巧的廚師他不掌握自己主人的相:『我的主人今天喜歡這種咖哩:\twnr{吃這種}{x618},或拿很多這種,或稱讚這種;我的主人今天喜歡酸為首的咖哩:吃酸為首的咖哩,或拿很多酸為首的咖哩,或稱讚酸為首的咖哩;我的主人今天喜歡苦為首的咖哩……(中略)我的主人今天喜歡辣為首的咖哩……(中略)我的主人今天喜歡甜為首的咖哩……(中略)我的主人今天喜歡鹼的咖哩……(中略)我的主人今天喜歡不鹼的咖哩……(中略)我的主人今天喜歡鹹的咖哩……(中略)我的主人今天喜歡不鹹的咖哩:吃不鹹的咖哩,或拿很多不鹹的咖哩,或稱讚不鹹的咖哩。』

  比丘們!那位愚笨、無能、不善巧的廚師他既不是衣服的利得者,也非工資的利得者,也非諸贈與的利得者,那是什麼原因?比丘們!因為,像這樣,那位愚笨、無能、不善巧的廚師不掌握自己主人的相。同樣的,比丘們!這裡,某一類愚笨、無能、不善巧的比丘住於\twnr{在身上隨看著身}{176.0}:熱心的、正知的、有念的,調伏世間中的\twnr{貪婪}{435.0}、憂後,當他住於在身上隨看著身時,心不入定,諸\twnr{隨雜染}{288.0}不被捨斷,他不掌握那個相;住於在諸受上隨看著受……(中略)住於在心上隨看著心……(中略)住於在諸法上隨看著法:熱心的、正知的、有念的,調伏世間中的貪婪、憂後,當他住於在諸法上隨看著法時,心不入定,諸隨雜染不被捨斷,他不掌握那個相。

  比丘們!那位愚笨、無能、不善巧的比丘他既不是當生\twnr{安樂住}{317.0}的利得者,也非念、正知的利得者,那是什麼原因?比丘們!因為,像這樣,那位愚笨、無能、不善巧的比丘不\twnr{掌握自己心的相}{x619}。

  比丘們!猶如賢智的、聰明的、善巧的廚師以極種種咖哩侍奉國王或國王的大臣:以酸為首的,也以苦為首的,也以辣為首的,也以甜為首的,也以鹼的,也以無鹼的,也以鹹的,也以不鹹的。

  比丘們!那位賢智的、聰明的、善巧的廚師他掌握自己主人的相:『我的主人今天喜歡這種咖哩:吃這種,或拿很多這種,或稱讚這種;我的主人今天喜歡酸為首的咖哩:吃酸為首的咖哩,或拿很多酸為首的咖哩,或稱讚酸為首的咖哩;我的主人今天喜歡苦為首的咖哩……(中略)我的主人今天喜歡辣為首的咖哩……(中略)我的主人今天喜歡甜為首的咖哩……(中略)我的主人今天喜歡鹼的咖哩……(中略)我的主人今天喜歡不鹼的咖哩……(中略)我的主人今天喜歡鹹的咖哩……(中略)我的主人今天喜歡不鹹的咖哩:吃不鹹的咖哩,或拿很多不鹹的咖哩,或稱讚不鹹的咖哩。』

  比丘們!那位賢智的、聰明的、善巧的廚師他既是衣服的利得者,也是工資的利得者,也是諸贈與的利得者,那是什麼原因?比丘們!因為那樣賢智的、聰明的、善巧的廚師掌握自己主人的相。同樣的,比丘們!這裡,某一類賢智的、聰明的、善巧的比丘住於在身上隨看著身:熱心的、正知的、有念的,調伏世間中的貪婪、憂後,當他住於在身上隨看著身時,心入定,諸隨雜染被捨斷,他掌握那個相;住於在諸受上隨看著受……(中略)住於在心上隨看著心……(中略)住於在諸法上隨看著法:熱心的、正知的、有念的,調伏世間中的貪婪、憂後,當他住於在諸法上隨看著法時,心入定,諸隨雜染被捨斷,他掌握那個相。

  比丘們!那位賢智的、聰明的、善巧的比丘他既是當生安樂住的利得者,也是念、正知的利得者,那是什麼原因?比丘們!因為,像這樣,那位賢智的、聰明的、善巧的比丘掌握自己心的相。」



\sutta{9}{9}{病經}{https://agama.buddhason.org/SN/sn.php?keyword=47.9}
  \twnr{被我這麼聽聞}{1.0}:

  \twnr{有一次}{2.0},\twnr{世尊}{12.0}住在毘舍離木瓜樹小村中。在那裡,世尊召喚\twnr{比丘}{31.0}們:

  「來!比丘們!請你們全部在毘舍離依朋友、依熟人、依友人進入雨季安居\twnr{雨季安居}{231.0},我就在木瓜樹小村這裡進入雨季安居。」 

  「是的,大德!」那些比丘回答世尊後,全部在毘舍離依朋友、依熟人、依友人進入雨季安居雨季安居,世尊就在木瓜樹小村那裡進入雨季安居。 

  那時,當世尊已進入雨季安居時,重病生起,激烈的、瀕臨死亡的諸感受轉起。在那裡,世尊具念、正知、不被惱害地忍受。那時,世尊想這個:

  「那對我是不適當的:凡我沒召喚隨侍們、沒通知比丘\twnr{僧團}{375.0}後應該\twnr{般涅槃}{72.0}。讓我以活力擋開這個病後,決意\twnr{壽行}{766.0}後住。」

  那時,世尊以活力擋開那個病後,決意壽行後住。(那時,世尊止息那個病。)

  那時,從病痊癒的世尊從病痊癒不久,從住處出去後,在住處陰影處設置的座位坐下。那時,\twnr{尊者}{200.0}阿難去見世尊。抵達後,向世尊\twnr{問訊}{46.0}後,在一旁坐下。在一旁坐下的尊者阿難對世尊說這個:

  「\twnr{大德}{45.0}!世尊的安樂被我看見;大德!能被世尊忍受被我看見;大德!能被世尊存活被我看見。大德!此外,以世尊的生病,我的身體如變成酒醉的,我的諸方向也不清晰,諸法也不在我心中出現。大德!但我就有某種程度的安心:『世尊將不會就只那樣般涅槃:只要世尊未說出關於僧團的就任何事。』」

  「阿難!但比丘僧團於我期待什麼呢?阿難!不作被內外後我教導法,阿難!如來在法上沒有\twnr{師傅留一手}{x620},阿難!凡確實如果這麼想:『我將照顧比丘僧團。』或『比丘僧團是指定我的。』阿難!他確實應該說出關於僧團的就任何事。阿難!如來不這麼想:『我將照顧比丘僧團。』或『比丘僧團是指定我的。』阿難!那個如來為何將會說出關於僧團的就任何事?阿難!又,我現在是衰老的、年老的、高齡的、老年的、到達老年的:我的八十歲轉起,阿難!猶如老舊貨車\twnr{以包纏物複合的}{x621}使之存續。同樣的,阿難!如來的身體看起來像以包纏物複合的使之存續。

  阿難!凡在如來以一切相的不作意、以一些受的滅、\twnr{進入後住於}{66.0}\twnr{無相心定}{265.0}時,阿難!那時,如來的身體成為較安樂的。阿難!因此,在這裡,你們要住於\twnr{以自己為島}{537.0},\twnr{以自己為歸依}{482.0},不以其他為歸依;以法為島,以法為歸依,不以其他為歸依。

  阿難!而怎樣比丘以自己為島,以自己為歸依,不以其他為歸依;以法為島,以法為歸依,不以其他為歸依呢?阿難!這裡,比丘住於\twnr{在身上隨看著身}{176.0}:熱心的、正知的、有念的,調伏世間中的\twnr{貪婪}{435.0}、憂後;在諸受上……(中略)在心上……(中略)住於在諸法上隨看著法:熱心的、正知的、有念的,調伏世間中的貪婪、憂後。阿難!比丘這樣住於以自己為島,以自己為歸依,不以其他為歸依;以法為島,以法為歸依,不以其他為歸依。阿難!現在或我死後,凡任何人將住於以自己為島,以自己為歸依,不以其他為歸依;以法為島,以法為歸依,不以其他為歸依,阿難!那些比丘將是我的凡任何欲學者最第一的。」[\ccchref{DN.16}{https://agama.buddhason.org/DN/dm.php?keyword=16}, 163-165段]



\sutta{10}{10}{比丘尼住所經}{https://agama.buddhason.org/SN/sn.php?keyword=47.10}
  那時,\twnr{尊者}{200.0}阿難午前時穿衣、拿起衣鉢後,去某個比丘尼的住所。抵達後,在設置的座位坐下。

  那時,眾多比丘尼去見尊者阿難。抵達後,向尊者阿難\twnr{問訊}{46.0}後,在一旁坐下。在一旁坐下的那些比丘尼對尊者阿難說這個:

  「阿難\twnr{大德}{45.0}!這裡,眾多在\twnr{四念住}{286.0}上住於心善建立的比丘尼,依次地\twnr{認知}{583.0}卓越的\twnr{特質}{438.0}。」

  「這是這樣,姊妹們!這是這樣,姊妹們!

  姊妹們!凡任何比丘或比丘尼在四念住上住於心善建立者,他的這個能被預期:『他將依次地認知卓越的特質。』」

  那時,尊者阿難對那些比丘尼以法說開示、勸導、鼓勵、\twnr{使歡喜}{86.0}後,從座位起來後離開。

  那時,尊者阿難在舍衛城\twnr{為了托鉢}{87.0}行走後,\twnr{餐後已從施食返回}{512.0},去見世尊。抵達後,向世尊問訊後,在一旁坐下。在一旁坐下的尊者阿難對世尊說這個:

  「大德!這裡,我午前時穿衣、拿起衣鉢後,去某個比丘尼的住所。抵達後,在設置的座位坐下。那時,眾多比丘尼去見我。抵達後,向我問訊後,在一旁坐下。在一旁坐下的大德!那些比丘尼對我說這個:『阿難大德!這裡,眾多在四念住上住於心善建立的比丘尼,依次地認知卓越的特質。』『這是這樣,姊妹們!這是這樣,姊妹們!

  姊妹們!凡任何比丘或比丘尼在四念住上住於心善建立者,他的這個能被預期:「他將依次地認知卓越的特質。」』」

  「這是這樣,阿難!這是這樣,阿難!阿難!凡任何比丘或比丘尼在四念住上住於心善建立者,他的這個能被預期:『他將依次地認知卓越的特質。』哪四個?

  阿難!這裡,比丘住於\twnr{在身上隨看著身}{176.0}:熱心的、正知的、有念的,調伏世間中的\twnr{貪婪}{435.0}、憂後,當他住於在身上隨看著身時,在身上生起身所緣的熱惱,或\twnr{心的退縮}{686.1},或心向外散亂,阿難!因為那樣,心就應該被比丘定置在某個能被明淨的相。對那位就定置心在某個能被明淨的相者,欣悅被生起;對喜悅者,\twnr{喜}{428.0}被生起;對\twnr{意喜}{320.0}者,身變得\twnr{寧靜}{313.0};\twnr{身已寧靜}{318.0}者感受樂;對有樂者,心入定。他像這樣深慮:『我為了利益定置心,那個利益被我獲得,好了,現在我要撤回。』他就撤回,也不尋思,也不伺察,他知道:『我存在無尋、\twnr{無伺}{175.1}、自身內有念的,我是樂的。』

  再者,阿難!比丘在諸受上……(中略)在心上……(中略)住於在諸法上隨看著法:熱心的、正知的、有念的,調伏世間中的貪婪、憂後,當他住於在諸法上隨看著法時,在諸法上生起法所緣的熱惱,或心的退縮,或心向外散亂,阿難!因為那樣,心就應該被比丘定置在某個能被明淨的相。對那位就定置心在某個能被明淨的相者,欣悅被生起;對喜悅者喜被生,對意喜者來說身變得寧靜,身已寧靜者感受樂;對有樂者,心入定。他像這樣深慮:『我為了利益定置心,那個利益被我獲得,好了,現在我要撤回。』他就撤回,也不尋思,也不伺察,他知道:『我存在無尋、無伺、自身內有念的,我是樂的。』

  阿難!這樣是定置後的\twnr{修習}{94.0}。

  阿難!而怎樣是無定置後的修習?

  阿難!比丘不定置心在外後,知道:『我的心不定置在外。』那時,知道:『\twnr{前後無昏昧(昧略)的}{x622}、解脫的、無定置的。』還有,知道:『我住於在身上隨看著身:熱心的、正知的、有念的,我是樂的。』

  阿難!比丘不定置心在外後,知道:『我的心不定置在外。』那時,知道:『於前後無昏昧的、解脫的、無定置的。』還有,知道:『我住於在諸受上隨看著受:熱心的、正知的、有念的,我是樂的。』

  阿難!比丘不定置心在外後,知道:『我的心不定置在外。』那時,知道:『於前後無昏昧的、解脫的、無定置的。』還有,知道:『我住於在心上隨看著心:熱心的、正知的、有念的,我是樂的。』

  阿難!比丘不定置心在外後,知道:『我的心不定置在外。』那時,知道:『於前後無昏昧的、解脫的、無定置的。』還有,知道:『我住於在諸法上隨看著法:熱心的、正知的、有念的,我是樂的。』

  阿難!這樣是無定置後的修習。

  像這樣,阿難!定置後的修習被我教導,無定置後的修習被我教導。阿難!凡\twnr{出自憐愍}{121.0}應該被老師、利益者、憐愍者為了弟子作的,那個被我為你們做了。阿難!有這些樹下、這些空屋,阿難!你們要修禪,不要放逸,不要以後成為後悔者,這是我們為你們的教誡。」

  世尊說這個,悅意的尊者阿難歡喜世尊的所說。

  蓭婆巴利品第一,其\twnr{攝頌}{35.0}:

  「蓭婆巴利、念、比丘,薩羅與[不]善聚,

   鷹、猴子、廚師,病、比丘尼住所。」





\pin{那難陀品}{11}{20}
\sutta{11}{11}{大丈夫經}{https://agama.buddhason.org/SN/sn.php?keyword=47.11}
  起源於舍衛城。

  那時,\twnr{尊者}{200.0}舍利弗去見世尊。抵達後,向世尊\twnr{問訊}{46.0}後,在一旁坐下。在一旁坐下的尊者舍利弗對世尊說這個:

  「\twnr{大德}{45.0}!被稱為『\twnr{大丈夫}{553.0}、大丈夫』,大德!什麼情形是大丈夫呢?」

  「舍利弗!我說心解脫者是『大丈夫』;我說心不解脫者是『非大丈夫』。

  舍利弗!而怎樣是心解脫者?舍利弗!這裡,\twnr{比丘}{31.0}住於\twnr{在身上隨看著身}{176.0}:熱心的、正知的、有念的,調伏世間中的\twnr{貪婪}{435.0}、憂後,當他住於在身上隨看著身時,心離染,不執取後從諸\twnr{漏}{188.0}被解脫;在諸受上……(中略)在心上……(中略)住於在諸法上隨看著法:熱心的、正知的、有念的,調伏世間中的貪婪、憂後,當他住於在諸法上隨看著法時,心離染,不執取後從諸\twnr{漏}{188.0}被解脫。舍利弗!這樣是心解脫者。

  舍利弗!我說心解脫者是『大丈夫』;我說心不解脫者是『非大丈夫』。」



\sutta{12}{12}{那難陀經}{https://agama.buddhason.org/SN/sn.php?keyword=47.12}
  \twnr{有一次}{2.0},\twnr{世尊}{12.0}住在那難陀賣衣者的芒果園中。

  那時,\twnr{尊者}{200.0}舍利弗去見世尊。抵達後,向世尊\twnr{問訊}{46.0}後,在一旁坐下。在一旁坐下的尊者舍利弗對世尊說這個:

  「\twnr{大德}{45.0}!我在世尊上有這樣的\twnr{淨信}{340.0}:不曾有與將沒有,以及現在不存在其他\twnr{沙門}{29.0}或\twnr{婆羅門}{17.0}比世尊更高證智的,即:\twnr{正覺}{185.1}。」

  「舍利弗!這偉大的、取\twnr{一向的}{168.0}、吼獅子吼的\twnr{如牛王之語}{675.0}被你說:『大德!我在世尊上有這樣的淨信:不曾有與將沒有,以及現在不存在其他沙門或婆羅門比世尊更高的證智,即:正覺。』

  舍利弗!凡那些存在於過去世的\twnr{阿羅漢}{5.0}、\twnr{遍正覺者}{6.0};那一切世尊被你\twnr{以心熟知心後}{393.0}知道:『那些世尊是這樣戒者。』或『那些世尊是\twnr{這樣法者}{981.0}。』或『那些世尊是這樣慧者。』或『那些世尊是\twnr{這樣住處者}{982.0}。』或『那些世尊是\twnr{這樣解脫者}{983.0}。』嗎?」

  「大德!這確實不是。」

  「舍利弗!又,凡那些存在於未來世的阿羅漢、遍正覺者,那一切世尊被你以心熟知心後知道:『那些世尊將有這樣的戒。』或『那些世尊將有這樣的法。』或『那些世尊將有這樣的慧。』或『那些世尊將是這樣的住處者。』或『那些世尊將是這樣的解脫者。』嗎?」

  「大德!這確實不是。」

  「舍利弗!又,我現在阿羅漢、遍正覺者被你以心熟知心後知道:『世尊有這樣的戒。』或『世尊有這樣的法。』或『世尊有這樣的慧。』或『世尊是這樣的住處者。』或『世尊是這樣的解脫者。』嗎?」

  「大德!這確實不是。」

  「舍利弗!而在這裡,你在過去、未來、現在阿羅漢、遍正覺者們上沒有\twnr{他心智}{260.0},舍利弗!那麼,那樣的話,為何偉大的、取一向的、吼獅子吼的如牛王之語被你說:『大德!我在世尊上有這樣的淨信:不曾有與將沒有,以及現在不存在其他沙門或婆羅門比世尊更高的證智,即:正覺。』呢?」 

  「大德!我在過去、未來、現在阿羅漢、遍正覺者們上沒有他心智,但\twnr{法的類比}{801.0}被我知道。大德!猶如國王邊境的城市,有堅固的壁壘,堅固的城牆與城門,只有一個門,在那裡,有賢智的、聰明的、有智慧的、對不認識的制止的、對認識的使進入的守門人。當他沿那個城市全部環繞的道路走時,沒看見甚至連貓出去大小的城牆間隙或裂口,他這麼想:『凡任何粗大的生物進入或出去這個城市,他們全部僅經由門進入或出去。』同樣的,大德!法的類比被我知道:『大德!凡那些存在於過去世的阿羅漢、遍正覺者;那些世尊全部捨斷心的\twnr{隨雜染}{288.0}、\twnr{慧的減弱的}{283.0}\twnr{五蓋}{287.0}後,在\twnr{四念住}{286.0}上心善建立,如實\twnr{修習}{94.0}七覺支後,\twnr{現正覺}{75.0}\twnr{無上遍正覺}{37.0}。大德!凡那些將存在於未來世的阿羅漢、遍正覺者;那些世尊全部也捨斷心的隨雜染、慧的減弱之五蓋後,在四念住上心善建立,如實修習七覺支後,將現正覺無上遍正覺。大德!現在的世尊、阿羅漢、遍正覺者也捨斷心的隨雜染、慧的減弱之五蓋後,在四念住上心善建立,如實修習七覺支後,現正覺無上遍正覺。』」[\ccchref{DN.16}{https://agama.buddhason.org/DN/dm.php?keyword=16}, 145-146段, \ccchref{DN.28}{https://agama.buddhason.org/DN/dm.php?keyword=28}, 141-143段]

  「舍利弗!\twnr{好}{44.0}!好!舍利弗!因此,在這裡,你應該經常對\twnr{比丘}{31.0}、比丘尼、\twnr{優婆塞}{98.0}、\twnr{優婆夷}{99.0}說這個\twnr{法的教說}{562.1},舍利弗!對凡無用的男子們來說,也將會有在如來上的疑惑或懷疑,又,對他們來說,聽聞這個法的教說後,凡在如來上的疑惑或懷疑將被捨斷。」



\sutta{13}{13}{純陀經}{https://agama.buddhason.org/SN/sn.php?keyword=47.13}
  \twnr{有一次}{2.0},\twnr{世尊}{12.0}住在舍衛城祇樹林給孤獨園。

  當時,生病的、受苦的、重病的\twnr{尊者}{200.0}舍利弗住在摩揭陀國的那羅迦村,而純陀沙彌為尊者舍利弗的隨侍者。

  那時,尊者舍利弗就因那個病般涅槃了。

  那時,純陀沙彌取尊者舍利弗的衣鉢後,到舍衛城祇樹林給孤獨園,去見尊者阿難。抵達後,向尊者阿難\twnr{問訊}{46.0}後,在一旁坐下。在一旁坐下的純陀沙彌對尊者阿難說這個:

  「\twnr{大德}{45.0}!尊者舍利弗已般涅槃,這是他的衣鉢。」

  「純陀\twnr{學友}{201.0}!這是為了見世尊的談論主題,純陀學友!我們走,我們將去見世尊。抵達後,我們將告訴世尊這件事。」

  「是的,大德!」純陀沙彌回答尊者阿難。

  那時,尊者阿難與純陀沙彌去見世尊。抵達後,向世尊問訊後,在一旁坐下。在一旁坐下的尊者阿難對世尊說這個:

  「大德!這位純陀沙彌對我這麼說:『大德!尊者舍利弗已般涅槃,這是他的衣鉢。』大德!此外,聽到『尊者舍利弗已般涅槃』後,我的身體如變成酒醉的,我的諸方向也不清晰,諸法也不在我心中出現。」

  「阿難!舍利弗拿走你的\twnr{戒蘊}{374.0}後般涅槃,或拿走定蘊後般涅槃,或拿走慧蘊後般涅槃,或拿走解脫蘊後般涅槃,或拿走解脫智見蘊後般涅槃嗎?」

  「大德!非尊者舍利弗拿走我的戒蘊後般涅槃,或定蘊……(中略)或慧蘊……或解脫蘊……或拿走解脫智見蘊後般涅槃,大德!但尊者舍利弗對我是教誡者、受其情愛影響者、教授者、開示者、勸導者、鼓勵者、使我歡喜者、對說法不疲倦者、\twnr{同梵行者}{138.0}的資助者,我們回憶那個尊者舍利弗的法之滋養、法之受用、法之助益。」

  「阿難!這被我就預先告知,不是嗎:就與一切所愛的、合意的分離、別離、異離。阿難!在這裡,\twnr{那如何可得}{847.0}:『凡那個被生的、\twnr{存在的}{x623}、\twnr{有為}{90.0}的、壞散之法,那個不要被破壞。』\twnr{這不存在可能性}{650.0}。阿難!猶如有\twnr{心材}{356.0}的、住立的大樹中,凡較大的枝幹它被折斷。同樣的,阿難!有心材的、住立的大\twnr{比丘}{31.0}\twnr{僧團}{375.0}中,舍利弗已般涅槃。阿難!在這裡,那如何可得:『凡那個被生的、存在的、有為的、壞散之法,那個不要被破壞。』這不存在可能性!

  阿難!因此,在這裡,你們要住於\twnr{以自己為島}{537.0},\twnr{以自己為歸依}{482.0},不以其他為歸依;以法為島,以法為歸依,不以其他為歸依。阿難!而怎樣比丘以自己為島,以自己為歸依,不以其他為歸依;以法為島,以法為歸依,不以其他為歸依呢?阿難!這裡,比丘住於\twnr{在身上隨看著身}{176.0}:熱心的、正知的、有念的,調伏世間中的\twnr{貪婪}{435.0}、憂後;在諸受上……(中略)在心上……(中略)住於在諸法上隨看著法:熱心的、正知的、有念的,調伏世間中的貪婪、憂後。阿難!比丘這樣住於以自己為島,以自己為歸依,不以其他為歸依;以法為島,以法為歸依,不以其他為歸依。

  阿難!現在或我死後,凡任何人將住於以自己為島,以自己為歸依,不以其他為歸依;以法為島,以法為歸依,不以其他為歸依,阿難!那些比丘將是我的凡任何欲學者最第一的。」



\sutta{14}{14}{烏迦支羅經}{https://agama.buddhason.org/SN/sn.php?keyword=47.14}
  \twnr{有一次}{2.0},在舍利弗、目揵連已\twnr{般涅槃}{72.0}不久,\twnr{世尊}{12.0}與大\twnr{比丘}{31.0}\twnr{僧團}{375.0}共住在跋耆的烏迦支羅恒河邊。

  當時,世尊被比丘僧團圍繞,坐\twnr{在屋外}{385.0}

  那時,世尊環視沈默的比丘僧團後,召喚比丘們:

  「比丘們!對我來說,在舍利弗、目揵連已般涅槃時,這個團體(眾)看起來像是空的。

  比丘們!對我來說,團體是不空的,在那些方向是不關注的:凡在舍利弗、目揵連住的方向。

  比丘們!凡那些存在於過去世的\twnr{阿羅漢}{5.0}、\twnr{遍正覺者}{6.0},那些世尊也都有這最勝的一對弟子,猶如我的舍利弗、目揵連;凡那些存在於\twnr{未來世}{308.0}的阿羅漢、遍正覺者,那些世尊也都有這最勝的一對弟子,猶如我的舍利弗、目揵連。

  比丘們!對[這一對]弟子們來說,\twnr{不可思議}{206.0}啊!比丘們!對弟子們來說,未曾有啊!他們將是遵循老師教誡者、教誡糾正者,以及他們將是\twnr{四眾的}{x624}所愛者、合意者、應該被尊重尊敬者。

  比丘們!如來的不可思議啊!比丘們!如來的未曾有啊!在像這樣一對弟子般涅槃時,如來沒有愁或悲。

  比丘們!在這裡,\twnr{那如何可得}{847.0}:『凡那個被生的、\twnr{存在的}{x623}、\twnr{有為}{90.0}的、壞散之法,那個不要被破壞!』\twnr{這不存在可能性}{650.0}。比丘們!猶如有\twnr{心材}{356.0}的、住立的大樹中,凡較大的枝幹它被折斷。同樣的,比丘們!有心材的、住立的大比丘僧團中,舍利弗已般涅槃。比丘們!在這裡,那如何可得:『凡那個被生的、存在的、有為的、壞散之法,那個不要被破壞。』這不存在可能性!

  比丘們!因此,在這裡,你們要住於\twnr{以自己為島}{537.0},\twnr{以自己為歸依}{482.0},不以其他為歸依;以法為島,以法為歸依,不以其他為歸依。比丘們!而怎樣比丘以自己為島,以自己為歸依,不以其他為歸依;以法為島,以法為歸依,不以其他為歸依?比丘們!這裡,比丘住於\twnr{在身上隨看著身}{176.0}:熱心的、正知的、有念的,調伏世間中的\twnr{貪婪}{435.0}、憂後;在諸受上……(中略)在心上……(中略)住於在諸法上隨看著法:熱心的、正知的、有念的,調伏世間中的貪婪、憂後。比丘們!比丘這樣住於以自己為島,以自己為歸依,不以其他為歸依;以法為島,以法為歸依,不以其他為歸依。

  比丘們!現在或我死後,凡任何將住於以自己為島,以自己為歸依,不以其他為歸依;以法為島,以法為歸依,不以其他為歸依者,比丘們!對我,這些比丘將是凡任何欲學者中最第一的。」



\sutta{15}{15}{婆醯雅經}{https://agama.buddhason.org/SN/sn.php?keyword=47.15}
  起源於舍衛城。

  那時,\twnr{尊者}{200.0}婆醯雅去見世尊。抵達後,向世尊\twnr{問訊}{46.0}後,在一旁坐下。在一旁坐下的尊者婆醯雅對\twnr{世尊}{12.0}說這個:

  「\twnr{大德}{45.0}!請世尊為我簡要地教導法,凡我聽聞世尊的法後,會住於單獨的、隱離的、不放逸的、熱心的、自我努力的,\twnr{那就好了}{44.0}!」

  「那麼,婆醯雅!在這裡,請你就在最初的諸善法上淨化。而什麼是最初的諸善法呢?善清淨的戒與正直的見。

  婆醯雅!當如果你有善清淨的戒與正直的見時,婆醯雅!之後,依止戒、在戒上住立後,你應該在\twnr{四念住}{286.0}上\twnr{修習}{94.0},哪四個?這裡,婆醯雅!請你住於\twnr{在身上隨看著身}{176.0}:熱心的、正知的、有念的,調伏世間中的\twnr{貪婪}{435.0}、憂後;在諸受上……(中略)在心上……(中略)請你住於在諸法上隨看著法:熱心的、正知的、有念的,調伏世間中的貪婪、憂後。

  婆醯雅!當你依止戒、在戒上住立後在這四念住上這樣修習時,婆醯雅!對你來說,不論日或夜到來,在善法上僅增長能被預期,非減損。」

  那時,尊者婆醯雅歡喜、隨喜世尊所說後,從座位起來、向世尊問訊、\twnr{作右繞}{47.0}後,離開。

  那時,住於單獨的、隱離的、不放逸的、熱心的、自我努力的尊者婆醯雅不久就以證智自作證後,在當生中\twnr{進入後住於}{66.0}凡\twnr{善男子}{41.0}們為了利益正確地\twnr{從在家出家成為無家者}{48.0}的那個無上梵行結尾,他證知:「出生已盡,梵行已完成,\twnr{應該被作的已作}{20.0},\twnr{不再有此處[輪迴]的狀態}{21.1}。」然後\twnr{尊者婆醯雅}{x625}成為眾\twnr{阿羅漢}{5.0}之一。



\sutta{16}{16}{鬱低雅經}{https://agama.buddhason.org/SN/sn.php?keyword=47.16}
  起源於舍衛城。

  那時,\twnr{尊者}{200.0}鬱低雅去見世尊……(中略)在一旁坐下的尊者鬱低雅對\twnr{世尊}{12.0}說這個:

  「\twnr{大德}{45.0}!請世尊為我簡要地教導法,凡我聽聞世尊的法後,會住於單獨的、隱離的、不放逸的、熱心的、自我努力的,\twnr{那就好了}{44.0}!」

  「那麼,鬱低雅!在這裡,請你就在最初的諸善法上淨化。而什麼是最初的諸善法呢?善清淨的戒與正直的見。

  鬱低雅!當如果你有善清淨的戒與正直的見時,鬱低雅!之後,依止戒、在戒上住立後,你應該在\twnr{四念住}{286.0}上\twnr{修習}{94.0},哪四個?這裡,鬱低雅!請你住於\twnr{在身上隨看著身}{176.0}:熱心的、正知的、有念的,調伏世間中的\twnr{貪婪}{435.0}、憂後;在諸受上……(中略)在心上……(中略)請你住於在諸法上隨看著法:熱心的、正知的、有念的,調伏世間中的貪婪、憂後。

  鬱低雅!當你依止戒、在戒上住立後在這四念住上這樣修習時,鬱低雅!你將走到死亡(死神)領域的\twnr{彼岸}{226.0}。」

  那時,尊者鬱低雅歡喜、隨喜世尊所說後,從座位起來、向世尊\twnr{問訊}{46.0}、\twnr{作右繞}{47.0}後,離開。

  那時,住於單獨的、隱離的、不放逸的、熱心的、自我努力的尊者鬱低雅不久就以證智自作證後,在當生中\twnr{進入後住於}{66.0}凡\twnr{善男子}{41.0}們為了利益正確地\twnr{從在家出家成為無家者}{48.0}的那個無上梵行結尾,他證知:「出生已盡,梵行已完成,\twnr{應該被作的已作}{20.0},不再有此處[輪迴]的狀態。」然後尊者鬱低雅成為眾\twnr{阿羅漢}{5.0}之一。[\suttaref{SN.47.3}]



\sutta{17}{17}{聖經}{https://agama.buddhason.org/SN/sn.php?keyword=47.17}
  「\twnr{比丘}{31.0}們!有這些\twnr{四念住}{286.0},已\twnr{修習}{94.0}、已\twnr{多作}{95.0},是聖的、\twnr{出離的}{294.0},引導那樣行為者\twnr{苦的完全滅盡}{181.0},哪四個?比丘們!這裡,比丘住於\twnr{在身上隨看著身}{176.0}:熱心的、正知的、有念的,調伏世間中的\twnr{貪婪}{435.0}、憂後;在諸受上……(中略)在心上……(中略)住於在諸法上隨看著法:熱心的、正知的、有念的,調伏世間中的貪婪、憂後。比丘們!這些是四念住,已修習、已多作,是聖的、出離的,引導那樣行為者苦的完全滅盡。」



\sutta{18}{18}{梵王經}{https://agama.buddhason.org/SN/sn.php?keyword=47.18}
  \twnr{有一次}{2.0},初\twnr{現正覺}{75.0}的\twnr{世尊}{12.0}住在優樓頻螺,尼連禪河邊牧羊人的榕樹處。

  那時,當世尊獨處、\twnr{獨坐}{92.0}時,這樣心的深思生起:

  「為了眾生的清淨、為了愁悲的超越、為了苦憂的滅沒、為了方法的獲得、為了涅槃的作證,這是\twnr{無岔路之道}{565.0},即:\twnr{四念住}{286.0},哪四個?\twnr{比丘}{31.0}能住於\twnr{在身上隨看著身}{176.0}:熱心的、正知的、有念的,調伏世間中的\twnr{貪婪}{435.0}、憂後;或比丘能住於在諸受上……(中略)或比丘能住於在心上……(中略)或比丘能住於在諸法上隨看著法:熱心的、正知的、有念的,調伏世間中的貪婪、憂後。為了眾生的清淨、為了愁悲的超越、為了苦憂的滅沒、為了方法的獲得、為了涅槃的作證,這是無岔路之道,即:四念住。」

  那時,\twnr{梵王娑婆主}{215.0}以心了知世尊心中的深思後,就猶如有力氣的男子伸直彎曲的手臂,或彎曲伸直的手臂,就像這樣在梵天世界消失,出現在世尊的面前。

  那時,梵王娑婆主置(作)上衣到一邊肩膀,向世尊\twnr{合掌}{377.0}鞠躬後,對世尊說這個:

  「這是這樣,世尊!這是這樣,善逝!\twnr{大德}{45.0}!為了眾生的清淨、為了愁悲的超越、為了苦憂的滅沒、為了方法的獲得、為了涅槃的作證,這是無岔路之道,即:四念住,哪四個?比丘能住於在身上隨看著身:熱心的、正知的、有念的,調伏世間中的貪婪、憂後;大德!或比丘能住於在諸受上……(中略)大德!或比丘能住於在心上……(中略)大德!或比丘能住於在諸法上隨看著法:熱心的、正知的、有念的,調伏世間中的貪婪、憂後。為了眾生的清淨、為了愁悲的超越、為了苦憂的滅沒、為了方法的獲得、為了涅槃的作證,這是無岔路之道,即:四念住。」

  梵王娑婆主說這個,說這個後,又更進一步說這個:

  「生的滅盡之看見者、有益的憐愍者,知道無岔路之道,

   在以前他們曾以此道渡過暴流,且凡他們將渡過、現在渡過(將來與現在都是)。」[\suttaref{SN.47.43}]



\sutta{19}{19}{私達迦經}{https://agama.buddhason.org/SN/sn.php?keyword=47.19}
  \twnr{有一次}{2.0},\twnr{世尊}{12.0}住在孫巴,名叫私達迦的孫巴城鎮。

  在那裡,世尊召喚\twnr{比丘}{31.0}們:

  「比丘們!從前,旃陀羅竹竿特技表演者舉起旃陀羅竹竿後,召喚徒弟昧達迦大林迦:『來!親愛的昧達迦大林迦!登上旃陀羅竹竿後,請你站在我的肩膀上。』

  『是的,老師!』

  比丘們!旃陀羅竹竿特技表演者的徒弟昧達迦大林迦回答後,登上旃陀羅竹竿後,站在老師的肩膀上。

  比丘們!那時,旃陀羅竹竿特技表演者對徒弟昧達迦大林迦說這個:『親愛的昧達迦大林迦!請你守護我,我將守護你,這樣,我們相互被保護,相互被守護,我們將展現技術,同時也將得到利得,並且將平安地從旃陀羅竹竿下來。』

  比丘們!在這麼說時,徒弟昧達迦大林迦對旃陀羅竹竿特技表演者說這個:『老師!但這將不是這樣。老師!請你守護自己,我將守護自己,這樣,我們被自己保護,被自己守護,我們將展現技術,同時也將得到利得,並且將平安地從旃陀羅竹竿下來。』」

  「在那裡,那是\twnr{正確方式}{x626}。」世尊說這個。

  「如徒弟昧達迦大林迦對老師說,比丘們!『我將守護自己』,念住應該被實踐;比丘們!『我將守護他人』,念住應該被實踐。

  比丘們!守護自己者守護他人;守護他人者守護自己。

  比丘們!而怎樣守護自己者守護他人呢?比丘們!以練習、\twnr{修習}{94.0}、多作(多行為),這樣守護自己者守護他人。

  比丘們!而怎樣守護他人者守護自己呢?比丘們!以忍辱、不害、慈心的狀態、憐憫的狀態,這樣守護他人者守護自己。

  比丘們!『我將守護自己』,念住應該被實踐;比丘們!『我將守護他人』,念住應該被實踐。

  比丘們!守護自己者守護他人;守護他人者守護自己。」



\sutta{20}{20}{地方上的美女經}{https://agama.buddhason.org/SN/sn.php?keyword=47.20}
  \twnr{被我這麼聽聞}{1.0}:

  \twnr{有一次}{2.0},\twnr{世尊}{12.0}住在孫巴,名叫私達迦的孫巴城鎮。

  在那裡,世尊召喚\twnr{比丘}{31.0}們:「比丘們!」

  「\twnr{尊師}{480.0}!」那些比丘回答世尊。

  世尊說這個:

  「比丘們!猶如『有地方上美女,有地方上美女。』比丘們!大群人聚集。又,那位地方上美女是在舞蹈上最勝產出者、在歌唱上最勝產出者,比丘們!『地方上的美女跳舞、唱歌。』更多量的大群人聚集。

  那時,想要活命、不想要死,想要樂、厭逆苦的男子到來,他(有人)對他這麼說:『喂!男子!這滿到邊緣的油鉢應該被你經這大慶祝會與這地方上的美女中間搬運,且劍已拔起的男子將在你後面緊隨,就在它如果只溢出一點點之處,就在那裡,使你的頭落下。』

  比丘們!你們怎麼想它:是否那位男子不作意那個油鉢後在外部放逸地搬運呢?」

  「大德!這確實不是。」

  「比丘們!為了義理的使知這個譬喻被我作:比丘們!在這裡,這就是義理:『滿到邊緣的油鉢。』這是\twnr{身至念}{521.0}的同義語。

  比丘們!因此,在這裡,應該被這麼學:『\twnr{身至念}{521.0}將被我們\twnr{修習}{94.0}、被\twnr{多作}{95.0}、被作為車輛、被作為基礎、被實行、被累積、\twnr{被善努力}{682.0}。』比丘們!應該被這麼學。」

  那難陀品第二,其\twnr{攝頌}{35.0}:

  「大丈夫、那難陀,純陀、支羅與婆醯雅,

   鬱低雅、聖、梵王,私達迦與以地方。」





\pin{戒持續品}{21}{30}
\sutta{21}{21}{戒經}{https://agama.buddhason.org/SN/sn.php?keyword=47.21}
  \twnr{被我這麼聽聞}{1.0}:

  \twnr{有一次}{2.0},\twnr{尊者}{200.0}阿難與尊者跋陀住在巴連弗城雞園。

  那時,尊者跋陀傍晚時,從\twnr{獨坐}{92.0}出來,去見尊者阿難。抵達後,與尊者阿難一起互相問候。交換應該被互相問候的友好交談後,在一旁坐下。在一旁坐下的尊者跋陀對尊者阿難說這個:

  「阿難\twnr{學友}{201.0}!凡這些被\twnr{世尊}{12.0}說的善戒,這些善戒是什麼目的被世尊說呢?」

  「跋陀學友!\twnr{好}{44.0}!好!跋陀學友!你的\twnr{想法}{563.1}是善的、辯才是善的、詢問是善的,跋陀學友!因為你這麼問:『阿難學友!凡這些被世尊說的善戒,這些善戒是什麼目的被世尊說呢?』」

  「是的,學友!」

  「跋陀學友!凡這些被世尊說的善戒,這些善戒就是為了\twnr{四念住}{286.0}的\twnr{修習}{94.0}被世尊說,哪四個?學友!這裡,\twnr{比丘}{31.0}住於\twnr{在身上隨看著身}{176.0}:熱心的、正知的、有念的,調伏世間中的\twnr{貪婪}{435.0}、憂後;在諸受上……(中略)在心上……(中略)住於在諸法上隨看著法:熱心的、正知的、有念的,調伏世間中的貪婪、憂後,跋陀學友!凡這些被世尊說的善戒,這些善戒就是為了四念住的修習被世尊說。」



\sutta{22}{22}{久住經}{https://agama.buddhason.org/SN/sn.php?keyword=47.22}
  都是那個起源。

  在一旁坐下的\twnr{尊者}{200.0}跋陀對尊者阿難說這個:

  「阿難\twnr{學友}{201.0}!什麼因、什麼\twnr{緣}{180.0},以那個在如來已\twnr{般涅槃}{72.0}時正法是不久住的呢?什麼因、什麼緣,以那個在如來已般涅槃時正法是久住的呢?」

  「跋陀學友!\twnr{好}{44.0}!好!跋陀學友!你的\twnr{想法}{563.1}是善的、辯才是善的、詢問是善的,跋陀學友!因為你這麼問:『什麼因、什麼緣,以那個在如來已般涅槃時正法是不久住的呢?什麼因、什麼緣,以那個在如來已般涅槃時正法是久住的呢?』」

  「是的,學友!」

  「學友!以\twnr{四念住}{286.0}的未自我\twnr{修習}{94.0}、未自我\twnr{多作}{95.0},在如來已般涅槃時正法是不久住的,以四念住的\twnr{已自我修習}{658.0}、已自我多作,在如來已般涅槃時正法是久住的,哪四個?學友!這裡,\twnr{比丘}{31.0}住於\twnr{在身上隨看著身}{176.0}:熱心的、正知的、有念的,調伏世間中的\twnr{貪婪}{435.0}、憂後;在諸受上……(中略)在心上……(中略)住於在諸法上隨看著法:熱心的、正知的、有念的,調伏世間中的貪婪、憂後,學友!以這些四念住的未自我修習、未自我多作,在如來已般涅槃時正法是不久住的;學友!以這些四念住的已自我修習、已自我多作,在如來已般涅槃時正法是久住的。」



\sutta{23}{23}{衰退經}{https://agama.buddhason.org/SN/sn.php?keyword=47.23}
  \twnr{有一次}{2.0},\twnr{尊者}{200.0}阿難與尊者跋陀住在巴連弗城雞園。

  那時,尊者跋陀傍晚時,從\twnr{獨坐}{92.0}出來,去見尊者阿難。抵達後,與尊者阿難一起互相問候。交換應該被互相問候的友好交談後,在一旁坐下。在一旁坐下的尊者跋陀對尊者阿難說這個:

  「阿難\twnr{學友}{201.0}!什麼因、什麼\twnr{緣}{180.0},以那個正法衰退呢?阿難學友!什麼因、什麼緣,以那個正法不衰退呢?」

  「跋陀學友!\twnr{好}{44.0}!好!跋陀學友!你的\twnr{想法}{563.1}是善的、辯才是善的、詢問是善的,跋陀學友!因為你這麼問:『阿難學友!什麼因、什麼緣,以那個正法衰退呢?阿難學友!什麼因、什麼緣,以那個正法不衰退呢?』」

  「是的,學友!」

  「學友!以\twnr{四念住}{286.0}的未自我\twnr{修習}{94.0}、未自我\twnr{多作}{95.0},正法衰退,以四念住的\twnr{已自我修習}{658.0}、已自我多作,正法不衰退,哪四個?學友!這裡,\twnr{比丘}{31.0}住於\twnr{在身上隨看著身}{176.0}:熱心的、正知的、有念的,調伏世間中的\twnr{貪婪}{435.0}、憂後;在諸受上……(中略)在心上……(中略)住於在諸法上隨看著法:熱心的、正知的、有念的,調伏世間中的貪婪、憂後,學友!以四念住的未自我修習、未自我多作,正法衰退,以四念住的已自我修習、已自我多作,正法不衰退。」



\sutta{24}{24}{概要經}{https://agama.buddhason.org/SN/sn.php?keyword=47.24}
  起源於舍衛城。

  「\twnr{比丘}{31.0}們!有這些\twnr{四念住}{286.0},哪四個?比丘們!這裡,比丘住於\twnr{在身上隨看著身}{176.0}:熱心的、正知的、有念的,調伏世間中的\twnr{貪婪}{435.0}、憂後;在諸受上……(中略)在心上……(中略)住於在諸法上隨看著法:熱心的、正知的、有念的,調伏世間中的貪婪、憂後。比丘們!這些是四念住。」



\sutta{25}{25}{某位婆羅門經}{https://agama.buddhason.org/SN/sn.php?keyword=47.25}
  \twnr{被我這麼聽聞}{1.0}:

  \twnr{有一次}{2.0},\twnr{世尊}{12.0}住在舍衛城祇樹林給孤獨園。

  那時,\twnr{某位}{39.0}婆羅門去見世尊。抵達後,與世尊一起互相問候。交換應該被互相問候的友好交談後,在一旁坐下。在一旁坐下的那位\twnr{婆羅門}{17.0}對世尊說這個:

  「\twnr{喬達摩}{80.0}尊師!什麼因、什麼\twnr{緣}{180.0},以那個在如來已\twnr{般涅槃}{72.0}時正法是不久住的呢?什麼因、什麼緣,以那個在如來已般涅槃時正法是久住的呢?」

  「婆羅門!以\twnr{四念住}{286.0}的未自我\twnr{修習}{94.0}、未自我\twnr{多作}{95.0},在如來已般涅槃時正法是不久住的,以四念住的\twnr{已自我修習}{658.0}、已自我多作,在如來已般涅槃時正法是久住的,哪四個?婆羅門!這裡,\twnr{比丘}{31.0}住於\twnr{在身上隨看著身}{176.0}:熱心的、正知的、有念的,調伏世間中的\twnr{貪婪}{435.0}、憂後;在諸受上……(中略)在心上……(中略)住於在諸法上隨看著法:熱心的、正知的、有念的,調伏世間中的貪婪、憂後,婆羅門!以這些四念住的未自我修習、未自我多作,在如來已般涅槃時正法是不久住的;婆羅門!以這些四念住的已自我修習、已自我多作,在如來已般涅槃時正法是久住的。」

  在這麼說時,那位婆羅門對世尊說這個:

  「太偉大了,喬達摩尊師!……(中略)請喬達摩\twnr{尊師}{203.0}記得我為\twnr{優婆塞}{98.0},從今天起\twnr{已終生歸依}{64.0}。」



\sutta{26}{26}{部分經}{https://agama.buddhason.org/SN/sn.php?keyword=47.26}
  \twnr{有一次}{2.0},\twnr{尊者}{200.0}舍利弗、尊者大目揵連、尊者阿那律住在娑雞多城荊棘林。

  那時,尊者舍利弗、尊者大目揵連傍晚時,從\twnr{獨坐}{92.0}出來,去見尊者阿那律。抵達後,與尊者阿那律一起互相問候。交換應該被互相問候的友好交談後,在一旁坐下。在一旁坐下的尊者舍利弗對尊者阿那律說這個:

  「阿那律\twnr{學友}{201.0}!被稱為『\twnr{有學}{193.0}、有學』,學友!什麼情形是有學呢?」

  「學友!對\twnr{四念住}{286.0}部分\twnr{已自我修習}{658.0}者是有學,哪四個?學友!這裡,\twnr{比丘}{31.0}住於\twnr{在身上隨看著身}{176.0}:熱心的、正知的、有念的,調伏世間中的\twnr{貪婪}{435.0}、憂後;在諸受上……(中略)在心上……(中略)住於在諸法上隨看著法:熱心的、正知的、有念的,調伏世間中的貪婪、憂後,學友!對這些四念住部分已自我修習者是有學。」



\sutta{27}{27}{完全經}{https://agama.buddhason.org/SN/sn.php?keyword=47.27}
  都是那個起源。 

  在一旁坐下的\twnr{尊者}{200.0}舍利弗對尊者阿那律說這個:

  「阿那律\twnr{學友}{201.0}!被稱為『\twnr{無學}{193.1}、無學』,學友!什麼情形是無學呢?」

  「學友!對\twnr{四念住}{286.0}完全\twnr{已自我修習}{658.0}者是無學,哪四個?學友!這裡,\twnr{比丘}{31.0}住於\twnr{在身上隨看著身}{176.0}:熱心的、正知的、有念的,調伏世間中的\twnr{貪婪}{435.0}、憂後;在諸受上……(中略)在心上……(中略)住於在諸法上隨看著法:熱心的、正知的、有念的,調伏世間中的貪婪、憂後,學友!對這些四念住完全已自我修習者是無學。」



\sutta{28}{28}{世界經}{https://agama.buddhason.org/SN/sn.php?keyword=47.28}
  都是那個起源。 

  在一旁坐下的\twnr{尊者}{200.0}舍利弗對尊者阿那律說這個:

  「阿那律\twnr{學友}{201.0}!以哪些法的\twnr{已自我修習}{658.0}、已自我\twnr{多作}{95.0},\twnr{大通智}{564.0}被到達呢?」

  「學友!以\twnr{四念住}{286.0}的已自我修習、已自我多作,大通智被到達,哪四個?學友!這裡,我住於\twnr{在身上隨看著身}{176.0}:熱心的、正知的、有念的,調伏世間中的\twnr{貪婪}{435.0}、憂後;在諸受上……(中略)在心上……(中略)我住於在諸法上隨看著法:熱心的、正知的、有念的,調伏世間中的貪婪、憂後,學友!我以這些四念住的已自我修習、已自我多作,大通智被到達,學友!而且,我以這些四念住的已自我修習、已自我多作,\twnr{證知}{242.0}千個世界。」



\sutta{29}{29}{富吉經}{https://agama.buddhason.org/SN/sn.php?keyword=47.29}
  \twnr{有一次}{2.0},\twnr{尊者}{200.0}阿難住在王舍城栗鼠飼養處的竹林中。

  當時,\twnr{屋主}{103.0}富吉是生病者、受苦者、重病者。那時,屋主富吉喚某位男子: 

  「喂!男子!來!請你去見尊者阿難。抵達後,請你以我的名義以頭禮拜尊者阿難的足:『\twnr{大德}{45.0}!屋主富吉是生病者、受苦者、重病者,他以頭禮拜尊者阿難的足。』且請你這麼說:『大德!請尊者阿難\twnr{出自憐愍}{121.0},去屋主富吉的住處,\twnr{那就好了}{44.0}!』」

  「是的。」那位男子回答屋主富吉後,去見尊者阿難。抵達後,向尊者阿難\twnr{問訊}{46.0}後,在一旁坐下。在一旁坐下的那位男子對尊者阿難說這個:

  「大德!屋主富吉是生病者、受苦者、重病者,他以頭禮拜尊者阿難的足,且這麼說:『大德!請尊者阿難出自憐愍,去屋主富吉的住處,\twnr{那就好了}{44.0}!』」

  尊者阿難以沈默狀態同意。

  那時,尊者阿難午前時穿衣、拿起衣鉢後,去屋主富吉的住處。抵達後,在設置的座位坐下。坐下後,尊者阿難對屋主富吉說這個:

  「屋主!是否能被你忍受?\twnr{是否能被[你]維持生活}{137.0}?是否苦的感受減退、不增進,減退的結局被知道,非增進?」

  「大德!不能被我忍受,不能被[我]維持,我強烈苦的感受增進、不減退,增進的結局被知道,非減退。」

   「屋主!因此,在這裡,應該被你這麼學:『我要住於\twnr{在身上隨看著身}{176.0}:熱心的、正知的、有念的,調伏世間中的\twnr{貪婪}{435.0}、憂後;在諸受上……(中略)在心上……(中略)我要住於在諸法上隨看著法:熱心的、正知的、有念的,調伏世間中的貪婪、憂後。』屋主!應該被你這麼學。」

   「大德!凡這些被世尊教導的\twnr{四念住}{286.0},那些法在我之中被發現,且我在那些法中被看見,大德!我住於在身上隨看著身:熱心的、正知的、有念的,調伏世間中的貪婪、憂後;在諸受上……(中略)在心上……(中略)我住於在諸法上隨看著法:熱心的、正知的、有念的,調伏世間中的貪婪、憂後。大德!且凡這些被世尊教導的\twnr{五下分結}{134.0},大德!我不認為那些中的任何者在自己之中未被捨斷。」

   「屋主!是你的利得,屋主!\twnr{是你的善得的}{350.0},屋主!\twnr{不還果}{209.1}被你\twnr{記說}{179.0}。」 



\sutta{30}{30}{摩那提那經}{https://agama.buddhason.org/SN/sn.php?keyword=47.30}
  都是那個起源。

  當時,\twnr{屋主}{103.0}摩那提那是生病者、受苦者、重病者。

  那時,屋主摩那提那召喚某位男子:

  「喂!男子!來!……(中略)。」

  「\twnr{大德}{45.0}!不能被我忍受,不能被[我]維持,我強烈苦的感受增進、不減退,增進的結局被知道,非減退,大德!但當我被這樣苦受接觸時,我住於\twnr{在身上隨看著身}{176.0}:熱心的、正知的、有念的,調伏世間中的\twnr{貪婪}{435.0}、憂後;在諸受上……(中略)在心上……(中略)我住於在諸法上隨看著法:熱心的、正知的、有念的,調伏世間中的貪婪、憂後,大德!且凡這些被世尊教導的\twnr{五下分結}{134.0},大德!我不認為那些中的任何者在自己之中未被捨斷。」

   「屋主!是你的利得,屋主!\twnr{是你的善得的}{350.0},屋主!\twnr{不還果}{209.1}被你\twnr{記說}{179.0}。」

  戒與住品第三,其\twnr{攝頌}{35.0}:

  「戒、住、衰退,概要、婆羅門、部分,

   完全、世界、富吉,以摩那提那它們為十。」





\pin{不曾聽聞品}{31}{40}
\sutta{31}{31}{不曾聽聞經}{https://agama.buddhason.org/SN/sn.php?keyword=47.31}
  起源於舍衛城。

  「『這是在身上\twnr{隨看}{59.0}身。』\twnr{比丘}{31.0}們!在以前不曾聽聞的諸法上,我的眼生起,智生起,慧生起,明生起,\twnr{光生起}{511.0}。『又,這個在身上隨看身應該被他\twnr{修習}{94.0}。』比丘們!……(中略)『……已修習。』比丘們!在以前不曾聽聞的諸法上,我的眼生起,智生起,慧生起,明生起,光生起。

  『這是在諸受上隨看受。』比丘們!在以前不曾聽聞的諸法上,我的眼生起,智生起,慧生起,明生起,光生起。『又,這個在諸受上隨看受應該被他修習。』比丘們!……(中略)『……已修習。』比丘們!在以前不曾聽聞的諸法上,我的眼生起,智生起,慧生起,明生起,光生起。

  『這是在心上隨看心。』比丘們!在以前不曾聽聞的諸法上,我的眼生起,智生起,慧生起,明生起,光生起。『又,這個在心上隨看心應該被他修習。』比丘們!……(中略)『……已修習。』比丘們!在以前不曾聽聞的諸法上,我的眼生起,智生起,慧生起,明生起,光生起。

  『這是在諸法上隨看法。』比丘們!在以前不曾聽聞的諸法上,我的眼生起,智生起,慧生起,明生起,光生起。『又,這個在諸法上隨看法應該被他修習。』比丘們!……(中略)『……已修習。』比丘們!在以前不曾聽聞的諸法上,我的眼生起,智生起,慧生起,明生起,光生起。」



\sutta{32}{32}{離貪經}{https://agama.buddhason.org/SN/sn.php?keyword=47.32}
  「\twnr{比丘}{31.0}們!有這些\twnr{四念住}{286.0},已\twnr{修習}{94.0}、已\twnr{多作}{95.0},轉起\twnr{一向}{168.0}\twnr{厭}{15.0}、\twnr{離貪}{77.0}、\twnr{滅}{68.0}、寂靜、證智、\twnr{正覺}{185.1}、涅槃,哪四個?比丘們!這裡,比丘住於\twnr{在身上隨看著身}{176.0}:熱心的、正知的、有念的,調伏世間中的\twnr{貪婪}{435.0}、憂後;在諸受上……(中略)在心上……(中略)比丘住於在諸法上隨看著法:熱心的、正知的、有念的,調伏世間中的貪婪、憂後。比丘們!這些是四念住,已修習、已多作,對一向的厭、對離貪、對滅、對寂靜、對證智、對正覺、對涅槃轉起。」



\sutta{33}{33}{已錯失經}{https://agama.buddhason.org/SN/sn.php?keyword=47.33}
  「\twnr{比丘}{31.0}們!凡任何已錯失\twnr{四念住}{286.0}者,他們導向\twnr{苦的完全滅盡}{181.0}的聖道已錯失;比丘們!凡任何已發動四念住者,他們導向苦的完全滅盡的聖道已發動。哪四個?比丘們!這裡,比丘住於\twnr{在身上隨看著身}{176.0}:熱心的、正知的、有念的,調伏世間中的\twnr{貪婪}{435.0}、憂後;在諸受上……(中略)在心上……(中略)比丘住於在諸法上隨看著法:熱心的、正知的、有念的,調伏世間中的貪婪、憂後。比丘們!凡任何已錯失四念住者,他們導向苦的完全滅盡的聖道已錯失;比丘們!凡任何已發動四念住者,他們導向苦的完全滅盡的聖道已發動。」



\sutta{34}{34}{已修習經}{https://agama.buddhason.org/SN/sn.php?keyword=47.34}
  「\twnr{比丘}{31.0}們!有這些\twnr{四念住}{286.0},已\twnr{修習}{94.0}、已\twnr{多作}{95.0},轉起從此岸走到\twnr{彼岸}{226.0},哪四個?比丘們!這裡,比丘住於\twnr{在身上隨看著身}{176.0}:熱心的、正知的、有念的,調伏世間中的\twnr{貪婪}{435.0}、憂後;在諸受上……(中略)在心上……(中略)住於在諸法上隨看著法:熱心的、正知的、有念的,調伏世間中的貪婪、憂後。比丘們!這些是四念住,已修習、已多作,轉起從此岸走到彼岸。」



\sutta{35}{35}{念經}{https://agama.buddhason.org/SN/sn.php?keyword=47.35}
  起源於舍衛城。

  「\twnr{比丘}{31.0}們!比丘應該住於具念的、正知的,這是我們為你們的教誡。

  比丘們!而怎樣比丘是具念的?比丘們!這裡,比丘住於\twnr{在身上隨看著身}{176.0}:熱心的、正知的、有念的,調伏世間中的\twnr{貪婪}{435.0}、憂後;住於在諸受上隨看著受……(中略)住於在心上隨看著心……(中略)住於在諸法上隨看著法:熱心的、正知的、有念的,調伏世間中的貪婪、憂後,比丘們!這樣,比丘是具念的。

  比丘們!而怎樣比丘是正知的?比丘們!這裡,比丘的知道的諸受生起、知道的\twnr{現起}{x627}、知道的滅沒;知道的諸\twnr{尋}{175.0}生起、知道的現起、知道的走到滅沒;知道的諸想生起、知道的現起、知道的走到滅沒[\ccchref{Ps.3}{https://agama.buddhason.org/Ps/Ps3.htm}, 167段]。比丘們!這樣,比丘是正知的。

  比丘們!比丘應該住於具念的、正知的,這是我們為你們的教誡。」



\sutta{36}{36}{完全智經}{https://agama.buddhason.org/SN/sn.php?keyword=47.36}
  「\twnr{比丘}{31.0}們!有這些\twnr{四念住}{286.0},哪四個?比丘們!這裡,比丘住於\twnr{在身上隨看著身}{176.0}:熱心的、正知的、有念的,調伏世間中的\twnr{貪婪}{435.0}、憂後;在諸受上……(中略)在心上……(中略)住於在諸法上隨看著法:熱心的、正知的、有念的,調伏世間中的貪婪、憂後。比丘們!這些是四念住。比丘們!以這些四念住的\twnr{已自我修習}{658.0}、已自我\twnr{多作}{95.0},二果其中之一果能被預期:當生\twnr{完全智}{489.0},或在存在\twnr{有餘依}{323.0}時,為\twnr{阿那含}{209.0}位。」



\sutta{37}{37}{意欲經}{https://agama.buddhason.org/SN/sn.php?keyword=47.37}
  「\twnr{比丘}{31.0}們!有這些\twnr{四念住}{286.0},哪四個?比丘們!這裡,比丘住於\twnr{在身上隨看著身}{176.0}:熱心的、正知的、有念的,調伏世間中的\twnr{貪婪}{435.0}、憂後。對那位住於在身上隨看著身者,凡在身上的意欲,那個被捨斷,以意欲的捨斷,\twnr{不死}{123.0}被作證。

  住於在諸受上隨看著受:熱心的、正知的、有念的,調伏世間中的貪婪、憂後。對那位住於在諸受上隨看著受者,凡在諸受上的意欲,那個被捨斷,以意欲的捨斷,不死被作證。

  住於在心上隨看著心:熱心的、正知的、有念的,調伏世間中的貪婪、憂後。對那位住於在心上隨看著心者,凡在心上的意欲,那個被捨斷,以意欲的捨斷,不死被作證。

  住於在諸法上隨看著法:熱心的、正知的、有念的,調伏世間中的貪婪、憂後。對那位住於在諸法上隨看著法者,凡在諸法上的意欲,那個被捨斷,以意欲的捨斷,不死被作證。」



\sutta{38}{38}{被遍知經}{https://agama.buddhason.org/SN/sn.php?keyword=47.38}
  「\twnr{比丘}{31.0}們!有這些\twnr{四念住}{286.0},哪四個?比丘們!這裡,比丘住於\twnr{在身上隨看著身}{176.0}:熱心的、正知的、有念的,調伏世間中的\twnr{貪婪}{435.0}、憂後。對那位住於在身上隨看著身者,身被\twnr{遍知}{154.0},以身的遍知狀態,\twnr{不死}{123.0}被作證。

  住於在諸受上隨看著受:熱心的、正知的、有念的,調伏世間中的貪婪、憂後。對那位住於在諸受上隨看著受來說,諸受被遍知,以諸受的遍知狀態,不死被作證。

  住於在心上隨看著心:熱心的、正知的、有念的,調伏世間中的貪婪、憂後。對那位住於在心上隨看著心來說,心被遍知,以心的遍知狀態,不死被作證。

  住於在諸法上隨看著法:熱心的、正知的、有念的,調伏世間中的貪婪、憂後。對那位住於在諸法上隨看著法來說,諸法被遍知,以諸法的遍知狀態,不死被作證。」



\sutta{39}{39}{修習經}{https://agama.buddhason.org/SN/sn.php?keyword=47.39}
  「\twnr{比丘}{31.0}們!我將教導\twnr{四念住}{286.0}的\twnr{修習}{94.0},\twnr{你們要聽}{43.0}它!比丘們!而什麼是四念住的修習?比丘們!這裡,比丘住於\twnr{在身上隨看著身}{176.0}:熱心的、正知的、有念的,調伏世間中的\twnr{貪婪}{435.0}、憂後;在諸受上……(中略)在心上……(中略)住於在諸法上隨看著法:熱心的、正知的、有念的,調伏世間中的貪婪、憂後。比丘們!這些是四念住的修習。」



\sutta{40}{40}{解析經}{https://agama.buddhason.org/SN/sn.php?keyword=47.40}
  「\twnr{比丘}{31.0}們!我將為你們教導念住,與念住的\twnr{修習}{94.0},以及導向念住的修習道跡,\twnr{你們要聽}{43.0}它!

  比丘們!而什麼是念住?比丘們!這裡,比丘住於\twnr{在身上隨看著身}{176.0}:熱心的、正知的、有念的,調伏世間中的\twnr{貪婪}{435.0}、憂後;住於在諸受上隨看著受……(中略)住於在心上隨看著心……(中略)住於在諸法上隨看著法:熱心的、正知的、有念的,調伏世間中的貪婪、憂後。比丘們!這被稱為念住。

  比丘們!而什麼是念住的修習?比丘們!這裡,比丘住於在身上\twnr{隨看著}{59.0}\twnr{集法}{67.1},住於在身上隨看著\twnr{消散法}{155.0},在身上隨看著集、消散法:熱心的、正知的、有念的,調伏世間中的貪婪、憂後;住於在諸受上隨看著集法……(中略)住於在心上隨看著集法……(中略)住於在諸法上隨看著集法,在諸法上隨看著消散法,在諸法上隨看著集、消散法:熱心的、正知的、有念的,調伏世間中的貪婪、憂後。比丘們!這被稱為念住的修習。

  比丘們!而什麼是導向念住的修習道跡呢?就是\twnr{八支聖道}{525.0},即:正見、正志、正語、正業、正命、正精進、正念、正定。比丘們!這被稱為導向念住的修習道跡。」

  不曾聽聞品第四,其\twnr{攝頌}{35.0}:

  「不曾聽聞、離貪,已錯失、已修習、念,

   完全智、意欲、遍知,修習與解析。」





\pin{不死品}{41}{50}
\sutta{41}{41}{不死經}{https://agama.buddhason.org/SN/sn.php?keyword=47.41}
  起源於舍衛城。

  「\twnr{比丘}{31.0}們!你們要住於在\twnr{四念住}{286.0}上心善建立,不要你們的\twnr{不死}{123.0}消失,哪四個?比丘們!這裡,比丘住於\twnr{在身上隨看著身}{176.0}:熱心的、正知的、有念的,調伏世間中的\twnr{貪婪}{435.0}、憂後;在諸受上……(中略)在心上……(中略)住於在諸法上隨看著法:熱心的、正知的、有念的,調伏世間中的貪婪、憂後。你們要住於在四念住上心善建立,不要你們的不死消失。」



\sutta{42}{42}{集起經}{https://agama.buddhason.org/SN/sn.php?keyword=47.42}
  「\twnr{比丘}{31.0}們!我將教導\twnr{四念住}{286.0}的\twnr{集起}{67.0}與滅沒,\twnr{你們要聽}{43.0}它!

  比丘們!而什麼是身的集起?從食\twnr{集}{67.0}而有身的集起;從食\twnr{滅}{68.0}有身的滅沒,從觸集而有受的集起;從觸滅有受的滅沒,從名色集而有心的集起;從名色滅有心的滅沒,從\twnr{作意}{43.1}集而有法的集起;從作意滅有法的滅沒。」



\sutta{43}{43}{道經}{https://agama.buddhason.org/SN/sn.php?keyword=47.43}
  起源於舍衛城。 

  在那裡,世尊召喚\twnr{比丘}{31.0}們:

  「比丘們!有這一次,我住在優樓頻螺,尼連禪河邊牧羊人的榕樹下,初\twnr{現正覺}{75.0}。比丘們!獨處、\twnr{獨坐}{92.0}的那個我這樣心的深思生起:『為了眾生的清淨、為了愁悲的超越、為了苦憂的滅沒、為了方法的獲得、為了涅槃的作證,這是\twnr{無岔路之道}{565.0},即:\twnr{四念住}{286.0},哪四個?比丘能住於\twnr{在身上隨看著身}{176.0}:熱心的、正知的、有念的,調伏世間中的\twnr{貪婪}{435.0}、憂後;或比丘能住於在諸受上……(中略)或比丘能住於在心上……(中略)或比丘能住於在諸法上隨看著法:熱心的、正知的、有念的,調伏世間中的貪婪、憂後。為了眾生的清淨、為了愁悲的超越、為了苦憂的滅沒、為了方法的獲得、為了涅槃的作證,這是無岔路之道,即:四念住。』

  比丘們!那時,\twnr{梵王娑婆主}{215.0}以心了知我心中的深思後,就猶如有力氣的男子伸直彎曲的手臂,或彎曲伸直的手臂,就像這樣在梵天世界消失,出現在我的面前。比丘們!那時,梵王娑婆主置(作)上衣到一邊肩膀,向我\twnr{合掌}{377.0}鞠躬後,對我說這個:『這是這樣,世尊!這是這樣,善逝!\twnr{大德}{45.0}!為了眾生的清淨、為了愁悲的超越、為了苦憂的滅沒、為了方法的獲得、為了涅槃的作證,這是無岔路之道,即:四念住,哪四個?比丘能住於在身上隨看著身:熱心的、正知的、有念的,調伏世間中的貪婪、憂後;大德!或比丘能住於在諸受上……(中略)大德!或比丘能住於在心上……(中略)大德!或比丘能住於在諸法上隨看著法:熱心的、正知的、有念的,調伏世間中的貪婪、憂後。為了眾生的清淨、為了愁悲的超越、為了苦憂的滅沒、為了方法的獲得、為了涅槃的作證,這是無岔路之道,即:四念住。』

  比丘們!梵王娑婆主說這個,說這個後,又更進一步說這個:

  『生的滅盡之看見者、有益的憐愍者,知道無岔路之道,

   在以前他們曾以此道渡過暴流,且凡他們將渡過、現在渡過(將來與現在都是)。』」[\suttaref{SN.47.18}]



\sutta{44}{44}{念經}{https://agama.buddhason.org/SN/sn.php?keyword=47.44}
  「\twnr{比丘}{31.0}們!比丘應該住於具念的,這是我們為你們的教誡。比丘們!而怎樣比丘是具念的?比丘們!這裡,比丘住於\twnr{在身上隨看著身}{176.0}:熱心的、正知的、有念的,調伏世間中的\twnr{貪婪}{435.0}、憂後;住於在諸受上隨看著受……(中略)住於在心上隨看著心……(中略)住於在諸法上隨看著法:熱心的、正知的、有念的,調伏世間中的貪婪、憂後,比丘們!這樣,比丘是具念的。比丘們!比丘應該住於具念的,這是我們為你們的教誡。」



\sutta{45}{45}{善聚經}{https://agama.buddhason.org/SN/sn.php?keyword=47.45}
  「\twnr{比丘}{31.0}們!當說『善聚』時,當正確說時,應該說\twnr{四念住}{286.0}。比丘們!因為這是完全的善聚,即:四念住,哪四個?比丘們!這裡,比丘住於\twnr{在身上隨看著身}{176.0}:熱心的、正知的、有念的,調伏世間中的\twnr{貪婪}{435.0}、憂後;在諸受上……(中略)在心上……(中略)住於在諸法上隨看著法:熱心的、正知的、有念的,調伏世間中的貪婪、憂後。比丘們!當說『善聚』時,當正確說時,應該說四念住。比丘們!因為這是完全的善聚,即:四念住。」



\sutta{46}{46}{波羅提木叉的自制經}{https://agama.buddhason.org/SN/sn.php?keyword=47.46}
  那時,\twnr{某位比丘}{39.0}去見\twnr{世尊}{12.0}。……(中略)在一旁坐下的那位比丘對世尊說這個:

  「\twnr{大德}{45.0}!請世尊為我簡要地教導法,凡我聽聞世尊的法後,會住於單獨的、隱離的、不放逸的、熱心的、自我努力的,\twnr{那就好了}{44.0}!」

  「比丘!因此,在這裡,請你就在最初的諸善法上淨化。而什麼是最初的諸善法呢?比丘!這裡,請你住於被\twnr{波羅提木叉}{289.0}的\twnr{自制}{217.0}自制,\twnr{具足正行行境}{418.0},在諸微罪中看見可怕的,請你\twnr{在諸學處上受持後學習}{382.0},比丘!當你住於被波羅提木叉的自制自制,具足正行行境,在諸微罪中看見可怕的,在諸學處上受持後學習,比丘!之後,依止戒、在戒上住立後,你應該在\twnr{四念住}{286.0}上\twnr{修習}{94.0},哪四個?比丘!這裡,請你住於\twnr{在身上隨看著身}{176.0}:熱心的、正知的、有念的,調伏世間中的\twnr{貪婪}{435.0}、憂後;在諸受上……(中略)在心上……(中略)請你住於在諸法上隨看著法:熱心的、正知的、有念的,調伏世間中的貪婪、憂後。

  比丘!當你依止戒、在戒上住立後在這四念住上這樣修習時,比丘!對你來說,不論日或夜到來,在善法上僅增長能被預期,非減損。」

  那時,那位比丘歡喜、\twnr{隨喜}{85.0}世尊所說後,從座位起來、向世尊\twnr{問訊}{46.0}、\twnr{作右繞}{47.0}後,離開。

  那時,住於單獨的、隱離的、不放逸的、熱心的、自我努力的那位比丘不久就以證智自作證後,在當生中\twnr{進入後住於}{66.0}凡\twnr{善男子}{41.0}們為了利益正確地\twnr{從在家出家成為無家者}{48.0}的那個無上梵行結尾,他證知:「\twnr{出生已盡}{18.0},\twnr{梵行已完成}{19.0},\twnr{應該被作的已作}{20.0},\twnr{不再有此處[輪迴]的狀態}{21.1}。」然後那位比丘成為眾\twnr{阿羅漢}{5.0}之一。



\sutta{47}{47}{惡行經}{https://agama.buddhason.org/SN/sn.php?keyword=47.47}
  那時,\twnr{某位比丘}{39.0}去見\twnr{世尊}{12.0}。……(中略)

  「\twnr{大德}{45.0}!請世尊為我簡要地教導法,凡我聽聞世尊的法後,會住於單獨的、隱離的、不放逸的、熱心的、自我努力的,\twnr{那就好了}{44.0}!」

  「比丘!因此,在這裡,請你就在最初的諸善法上淨化。而什麼是最初的諸善法呢?比丘!這裡,捨斷身惡行後,你將修習身善行;捨斷語惡行後,你將修習語善行;捨斷意惡行後,你將修習意善行,比丘!當捨斷身惡行後,你將修習身善行;捨斷語惡行後,你將修習語善行;捨斷意惡行後,你將修習意善行,比丘!之後,依止戒、在戒上住立後,你應該在\twnr{四念住}{286.0}上\twnr{修習}{94.0},哪四個?比丘!這裡,請你住於\twnr{在身上隨看著身}{176.0}:熱心的、正知的、有念的,調伏世間中的\twnr{貪婪}{435.0}、憂後;在諸受上……(中略)在心上……(中略)請你住於在諸法上隨看著法:熱心的、正知的、有念的,調伏世間中的貪婪、憂後。

  比丘!當你依止戒、在戒上住立後在這四念住上這樣修習時,比丘!對你來說,不論日或夜到來,在善法上僅增長能被預期,非減損。」

  ……(中略)然後那位比丘成為眾\twnr{阿羅漢}{5.0}之一。



\sutta{48}{48}{朋友經}{https://agama.buddhason.org/SN/sn.php?keyword=47.48}
  「\twnr{比丘}{31.0}們!凡你們會憐愍,以及凡他們會想應該被聽聞的朋友,或同事,或親族,或有血緣者,比丘們!他們應該被你們為了\twnr{四念住}{286.0}的\twnr{修習}{94.0}勸導、應該被使確立、應該被使建立,哪四個?比丘們!這裡,比丘住於\twnr{在身上隨看著身}{176.0}:熱心的、正知的、有念的,調伏世間中的\twnr{貪婪}{435.0}、憂後;在諸受上……(中略)在心上……(中略)住於在諸法上隨看著法:熱心的、正知的、有念的,調伏世間中的貪婪、憂後。比丘們!凡你們會憐愍,以及凡他們會想應該被聽聞的朋友,或同事,或親族,或有血緣者,比丘們!他們應該被你們為了四念住的修習勸導、應該被使確立、應該被使建立。」[≃\suttaref{SN.55.17}, \suttaref{SN.56.26}]



\sutta{49}{49}{受經}{https://agama.buddhason.org/SN/sn.php?keyword=47.49}
  「\twnr{比丘}{31.0}們!有這些三受,哪三個?樂受、苦受、不苦不樂受,比丘們!這些是三受。

  比丘們!為了這些三受的\twnr{遍知}{154.0},\twnr{四念住}{286.0}應該被\twnr{修習}{94.0},哪四個?比丘們!這裡,比丘住於\twnr{在身上隨看著身}{176.0}:熱心的、正知的、有念的,調伏世間中的\twnr{貪婪}{435.0}、憂後;在諸受上……(中略)在心上……(中略)住於在諸法上隨看著法:熱心的、正知的、有念的,調伏世間中的貪婪、憂後。比丘們!為了這些三受的遍知,這四念住應該被修習。」



\sutta{50}{50}{漏經}{https://agama.buddhason.org/SN/sn.php?keyword=47.50}
  「\twnr{比丘}{31.0}們!有這三種\twnr{漏}{188.0},哪三個?欲漏、有漏、\twnr{無明漏}{397.0},比丘們!這些是三種漏。比丘們!為了這三種漏的捨斷,\twnr{四念住}{286.0}應該被\twnr{修習}{94.0},哪四個?比丘們!這裡,比丘住於\twnr{在身上隨看著身}{176.0}:熱心的、正知的、有念的,調伏世間中的\twnr{貪婪}{435.0}、憂後;在諸受上……(中略)在心上……(中略)住於在諸法上隨看著法:熱心的、正知的、有念的,調伏世間中的貪婪、憂後。比丘們!為了這三種漏的捨斷,這四念住應該被修習。」

  不死品第五,其\twnr{攝頌}{35.0}:

  「不死、集起、道,念與善聚,

   波羅提木叉、惡行,朋友、受與以漏。」





\pin{恒河中略品}{51}{62}
\sutta{51}{62}{恒河等經十二則}{https://agama.buddhason.org/SN/sn.php?keyword=47.51}
  「\twnr{比丘}{31.0}們!猶如恒河是傾向東的、斜向東的、坡斜向東的。同樣的,比丘們!\twnr{修習}{94.0}\twnr{四念住}{286.0}、\twnr{多作}{95.0}四念住的比丘是傾向涅槃的、斜向涅槃的、坡斜向涅槃的。

  比丘們!而怎樣修習四念住、多作四念住的比丘是傾向涅槃的、斜向涅槃的、坡斜向涅槃的?比丘們!這裡,比丘住於\twnr{在身上隨看著身}{176.0}:熱心的、正知的、有念的,調伏世間中的\twnr{貪婪}{435.0}、憂後;在諸受上……(中略)在心上……(中略)住於在諸法上隨看著法:熱心的、正知的、有念的,調伏世間中的貪婪、憂後。

  比丘們!這樣修習四念住、多作四念住的比丘是傾向涅槃的、斜向涅槃的、坡斜向涅槃的。」應該使之被細說。[按:全品應如\suttaref{SN.45.91}-102那樣]

  恒河中略品第六,其\twnr{攝頌}{35.0}:

  「六則傾向東的,與六則傾向大海的,

   這兩個六則成十二則,以那個被稱為品。」





\pin{不放逸品}{63}{72}
\sutta{63}{72}{如來等經十則}{https://agama.buddhason.org/SN/sn.php?keyword=47.63}
  「\twnr{比丘}{31.0}們!眾生之所及:無足的、二足的、四足的、多足的……。[\suttaref{SN.45.139}-148]」應該使之被細說。

  不放逸品第七,其\twnr{攝頌}{35.0}:

  「如來、足跡、屋頂,根、樹心、茉莉花,

   王、月、日,以衣服為第十句。」





\pin{力量所作品}{73}{84}
\sutta{73}{84}{力量等經十二則}{https://agama.buddhason.org/SN/sn.php?keyword=47.73}
  「\twnr{比丘}{31.0}們!猶如當任何應該被力量作的工作被作時……(中略)。」應該使之被細說。[按:全品應如\suttaref{SN.45.149}-160那樣]

  力量所作品第八,其\twnr{攝頌}{35.0}:

  「力量、種子與龍,樹木、瓶子及穗,

   虛空與二則雨雲,船、屋舍、河。」





\pin{尋求品}{85}{94}
\sutta{85}{94}{尋求等經十則}{https://agama.buddhason.org/SN/sn.php?keyword=47.85}
  「\twnr{比丘}{31.0}們!有這些三種尋求,哪三個?欲的尋求、有的尋求、\twnr{梵行的尋求}{381.1}……。[\suttaref{SN.45.161}-171]」應該使之被細說。

   尋求品第九,其\twnr{攝頌}{35.0}:

  「尋求、慢、漏,有與三苦性,

   荒蕪、垢、惱亂,受、渴愛與渴望。」





\pin{暴流品}{95}{104}
\sutta{95}{104}{上分等經十則}{https://agama.buddhason.org/SN/sn.php?keyword=47.95}
  「\twnr{比丘}{31.0}們!有這些五上分結,哪五個?色貪、無色貪、慢、掉舉、\twnr{無明}{207.0},比丘們!這些是五上分結。比丘們!為了這五上分結的\twnr{證知}{242.0}、\twnr{遍知}{154.0}、遍盡、捨斷,這\twnr{四念住}{286.0}應該被\twnr{修習}{94.0}。比丘們!哪四個?比丘們!這裡,比丘住於\twnr{在身上隨看著身}{176.0}:熱心的、正知的、有念的,調伏世間中的\twnr{貪婪}{435.0}、憂後;在諸受上……(中略)在心上……(中略);住於在諸法上隨看著法:熱心的、正知的、有念的,調伏世間中的貪婪、憂後。比丘們!為了這五上分結的證知、遍知、遍盡、捨斷,這四念住應該被修習。」

   (念住相應應該如\twnr{道相應}{x628}那樣使之被細說).

  暴流品第十,其\twnr{攝頌}{35.0}:

  「暴流、軛、取,繫縛、煩惱潛在趨勢,

   欲種類、蓋,蘊、下上分。」

  念住相應第三。





\page

\xiangying{48}{根相應}
\pin{概要品}{1}{10}
\sutta{1}{1}{概要經}{https://agama.buddhason.org/SN/sn.php?keyword=48.1}
  起源於舍衛城。

  「\twnr{比丘}{31.0}們!有這些五根,哪五個?信根、\twnr{活力根}{291.0}、念根、定根、慧根。比丘們!這些是五根。」



\sutta{2}{2}{入流者經第一}{https://agama.buddhason.org/SN/sn.php?keyword=48.2}
  「\twnr{比丘}{31.0}們!有這些五根,哪五個?信根、\twnr{活力根}{291.0}、念根、定根、慧根。

  比丘們!當\twnr{聖弟子}{24.0}如實知道這些五根的\twnr{樂味}{295.0}、\twnr{過患}{293.0}、\twnr{出離}{294.0},比丘們!這位聖弟子被稱為\twnr{入流者}{165.0}、不墮惡趣法者、\twnr{決定者}{159.0}、\twnr{正覺為彼岸者}{160.0}。」



\sutta{3}{3}{入流者經第二}{https://agama.buddhason.org/SN/sn.php?keyword=48.3}
  「\twnr{比丘}{31.0}們!有這些五根,哪五個?信根、\twnr{活力根}{291.0}、念根、定根、慧根。

  比丘們!當聖弟子如實知道這些五根的\twnr{集起}{67.0}、滅沒、\twnr{樂味}{295.0}、\twnr{過患}{293.0}、\twnr{出離}{294.0},比丘們!這位被稱為入流者、不墮惡趣法者、\twnr{決定者}{159.0}、\twnr{正覺為彼岸者}{160.0}聖弟子。」



\sutta{4}{4}{阿羅漢經第一}{https://agama.buddhason.org/SN/sn.php?keyword=48.4}
  「\twnr{比丘}{31.0}們!有這些五根,哪五個?信根、\twnr{活力根}{291.0}、念根、定根、慧根。

  比丘們!當\twnr{聖弟子}{24.0}如實知道這些五根的\twnr{樂味}{295.0}、\twnr{過患}{293.0}、\twnr{出離}{294.0}後,不執取後成為解脫者,比丘們!這被稱為漏已滅盡的、已完成的、\twnr{應該被作的已作的}{20.0}、負擔已卸的、\twnr{自己的利益已達成的}{189.0}、\twnr{有之結已滅盡的}{190.0}、以\twnr{究竟智}{191.0}解脫的\twnr{阿羅漢}{5.0}比丘。」



\sutta{5}{5}{阿羅漢經第二}{https://agama.buddhason.org/SN/sn.php?keyword=48.5}
  「\twnr{比丘}{31.0}們!有這些五根,哪五個?信根、\twnr{活力根}{291.0}、念根、定根、慧根。

  比丘們!當聖弟子如實知道這些五根的\twnr{集起}{67.0}、滅沒、\twnr{樂味}{295.0}、\twnr{過患}{293.0}、\twnr{出離}{294.0}後,不執取後成為解脫者,比丘們!這被稱為漏已滅盡的、已完成的、應該被作的已作的、負擔已卸的、\twnr{自己的利益已達成的}{189.0}、\twnr{有之結已滅盡的}{190.0}、以\twnr{究竟智}{191.0}解脫的\twnr{阿羅漢}{5.0}比丘。」



\sutta{6}{6}{沙門婆羅門經第一}{https://agama.buddhason.org/SN/sn.php?keyword=48.6}
  「\twnr{比丘}{31.0}們!有這些五根,哪五個?信根……(中略)慧根。比丘們!凡任何\twnr{沙門}{29.0}或\twnr{婆羅門}{17.0}不如實知道這些五根的\twnr{樂味}{295.0}、\twnr{過患}{293.0}、\twnr{出離}{294.0}者,比丘們!那些沙門或婆羅門不被我認同為\twnr{沙門中的沙門}{560.0},或婆羅門中的婆羅門,而且,那些\twnr{尊者}{200.0}也不以證智自作證後,在當生中\twnr{進入後住於}{66.0}\twnr{沙門義}{327.0}或婆羅門義。

  比丘們!而凡任何沙門或婆羅門如實知道這些五根的\twnr{集起}{67.0}、滅沒、\twnr{樂味}{295.0}、\twnr{過患}{293.0}、\twnr{出離}{294.0}者,比丘們!那些沙門或婆羅門被我認同為沙門中的沙門,或婆羅門中的婆羅門,而且,那些尊者也以證智自作證後,在當生中進入後住於沙門義或婆羅門義。」



\sutta{7}{7}{沙門婆羅門經第二}{https://agama.buddhason.org/SN/sn.php?keyword=48.7}
  「\twnr{比丘}{31.0}們!凡任何\twnr{沙門}{29.0}或\twnr{婆羅門}{17.0}不知道信根、不知道信根集、不知道信根滅、不知道導向信根\twnr{滅道跡}{69.0};不知道\twnr{活力根}{291.0}……(中略)不知道念根……(中略)不知道定根……(中略)不知道慧根、不知道慧根集、不知道慧根滅、不知道導向慧根滅道跡者,比丘們!那些沙門或婆羅門不被我認同為\twnr{沙門中的沙門}{560.0},或婆羅門中的婆羅門,而且,那些\twnr{尊者}{200.0}也不以證智自作證後,在當生中\twnr{進入後住於}{66.0}\twnr{沙門義}{327.0}或婆羅門義。

  比丘們!而凡任何沙門或婆羅門知道信根、知道信根集、知道信根滅、知道導向信根滅道跡;知道活力根、知道活力根集、知道活力根滅、知道導向活力根滅道跡;知道念根……(中略)知道定根……(中略)知道慧根、知道慧根集、知道慧根滅、知道導向慧根滅道跡者,比丘們!那些沙門或婆羅門被我認同為沙門中的沙門,或婆羅門中的婆羅門,而且,那些尊者也以證智自作證後,在當生中進入後住於沙門義或婆羅門義。」



\sutta{8}{8}{應該被看見經}{https://agama.buddhason.org/SN/sn.php?keyword=48.8}
  「\twnr{比丘}{31.0}們!有這些五根,哪五個?信根……(中略)慧根。

  比丘們!而信根應該在哪裡被看見?在四\twnr{入流支}{370.0}上:在這裡信根應該被看見。

  比丘們!而\twnr{活力根}{291.0}應該在哪裡被看見?在\twnr{四正勤}{292.0}上:在這裡活力根應該被看見。

  比丘們!而念根應該在哪裡被看見?在\twnr{四念住}{286.0}上:在這裡念根應該被看見。

  比丘們!而定根應該在哪裡被看見?在四禪上:在這裡定根應該被看見。

  比丘們!而慧根應該在哪裡被看見?在四聖諦上:在這裡慧根應該被看見。

  比丘們!這些是五根。」



\sutta{9}{9}{解析經第一}{https://agama.buddhason.org/SN/sn.php?keyword=48.9}
  「\twnr{比丘}{31.0}們!有這些五根,哪五個?信根……(中略)慧根。

  比丘們!而什麼是信根?比丘們!這裡,\twnr{聖弟子}{24.0}是有信者,相信如來的\twnr{覺}{185.0}:『像這樣,那位世尊是\twnr{阿羅漢}{5.0}、遍正覺者、\twnr{明行具足者}{7.0}、\twnr{善逝}{8.0}、\twnr{世間知者}{9.0}、\twnr{應該被調御人的無上調御者}{10.0}、\twnr{天-人們的大師}{11.0}、佛陀、世尊。』比丘們!這被稱為信根。

  比丘們!而什麼是\twnr{活力根}{291.0}?比丘們!這裡,聖弟子為了不善法的捨斷、為了善法的具足,住於活力已被發動的、強力的、堅固努力的、在諸善法上不放下負擔的。比丘們!這被稱為活力根。

  比丘們!而什麼是念根?比丘們!這裡,聖弟子是有念者,具備最高的\twnr{念與聰敏}{590.0},是很久以前做過的及很久以前說過的記得者、回憶者。比丘們!這被稱為念根。

  比丘們!而什麼是定根?比丘們!這裡,聖弟子作\twnr{捨棄為所緣}{495.0}後,得到定、得到\twnr{心一境性}{255.0}。比丘們!這被稱為定根。

  比丘們!而什麼是慧根?比丘們!這裡,聖弟子是有慧者,具備\twnr{導向生起與滅沒}{498.0}、聖、洞察、導向\twnr{苦的完全滅盡}{181.0}之慧。比丘們!這被稱為慧根。

  比丘們!這些被稱為五根。」



\sutta{10}{10}{解析經第二}{https://agama.buddhason.org/SN/sn.php?keyword=48.10}
  「\twnr{比丘}{31.0}們!有這些五根,哪五個?信根……(中略)慧根。

  比丘們!而什麼是信根?比丘們!這裡,聖弟子是有信者,相信如來的\twnr{覺}{185.0}:『像這樣,那位世尊是阿羅漢、遍正覺者、明行具足者、善逝、\twnr{世間知者}{9.0}、\twnr{應該被調御人的無上調御者}{10.0}、\twnr{天-人們的大師}{11.0}、佛陀、世尊。』比丘們!這被稱為信根。

  比丘們!而什麼是\twnr{活力根}{291.0}?比丘們!這裡,聖弟子為了不善法的捨斷、為了善法的具足,住於活力已被發動的、強力的、堅固努力的、在諸善法上不放下負擔的。

  他為了未生起的惡不善法之不生起使意欲生起、努力、發動活力、盡心、勤奮;為了已生起的惡不善法之捨斷使意欲生起、努力、發動活力、盡心、勤奮;為了未生起的善法之生起使意欲生起、努力、發動活力、盡心、勤奮;為了已生起的諸善法之存續、不忘失、增大、成滿、修習圓滿使意欲生起、努力、發動活力、盡心、勤奮。比丘們!這被稱為活力根。

  比丘們!而什麼是念根?比丘們!這裡,聖弟子是有念者,具備最高的\twnr{念與聰敏}{590.0},是很久以前做過的及很久以前說過的記得者、回憶者。

  他住於\twnr{在身上隨看著身}{176.0}:熱心的、正知的、有念的,調伏世間中的\twnr{貪婪}{435.0}、憂後;在諸受上……(中略)在心上……(中略)住於在諸法上隨看著法:熱心的、正知的、有念的,調伏世間中的貪婪、憂後。比丘們!這被稱為念根。

  比丘們!而什麼是定根?比丘們!這裡,聖弟子作\twnr{捨棄為所緣}{495.0}後,得到定、得到心一境性。

  他就從離諸欲後,從離諸不善法後,\twnr{進入後住於}{66.0}有尋、\twnr{有伺}{175.0},\twnr{離而生喜、樂}{174.0}的初禪;從尋與伺的平息,\twnr{自身內的明淨}{256.0},\twnr{心的專一性}{255.0},進入後住於無尋、無伺,定而生喜、樂的第二禪;從喜的\twnr{褪去}{77.0}、住於\twnr{平靜}{228.0}、有念正知、以身體感受樂,進入後住於這聖弟子宣說:『他是平靜、具念、\twnr{住於樂者}{317.0}』的第三禪;從樂的捨斷與從苦的捨斷,就在之前諸喜悅、憂的滅沒,進入後住於不苦不樂,\twnr{平靜、念遍純淨}{494.0}的第四禪。比丘們!這被稱為定根。

  比丘們!而什麼是慧根?比丘們!這裡,聖弟子是有慧者,具備導向生起與滅沒之慧;聖、洞察,導向苦的完全滅盡[之慧]。

  他如實知道『這是苦』,如實知道『這是苦集』,如實知道『這是苦滅』,如實知道『這是導向苦滅道跡』。比丘們!這被稱為慧根。

  比丘們!這些被稱為五根。」

  概要品第一,其\twnr{攝頌}{35.0}:

  「概要、二則流,阿羅漢二則在後,

   沙門婆羅門、應該被看見,解析二則在後。」





\pin{較弱品}{11}{20}
\sutta{11}{11}{獲得經}{https://agama.buddhason.org/SN/sn.php?keyword=48.11}
  「\twnr{比丘}{31.0}們!有這些五根,哪五個?信根……(中略)慧根。……(中略)。

  比丘們!而什麼是信根?比丘們!這裡,\twnr{聖弟子}{24.0}是有信者,相信如來的\twnr{覺}{185.0}:『像這樣,那位世尊是\twnr{阿羅漢}{5.0}、\twnr{遍正覺者}{6.0}、\twnr{明行具足者}{7.0}、\twnr{善逝}{8.0}、\twnr{世間知者}{9.0}、\twnr{應該被調御人的無上調御者}{10.0}、\twnr{天-人們的大師}{11.0}、佛陀、世尊。』比丘們!這被稱為信根。

  比丘們!而什麼是\twnr{活力根}{291.0}?比丘們!凡獲得關於\twnr{四正勤}{292.0}的活力者(凡發動四正勤後獲得活力者),比丘們!這被稱為活力根。

  比丘們!而什麼是念根?凡獲得關於\twnr{四念住}{286.0}的念者,比丘們!這被稱為念根。

  比丘們!而什麼是定根?比丘們!這裡,聖弟子作\twnr{捨棄為所緣}{495.0}後,得到定、得到\twnr{心一境性}{255.0}。比丘們!這被稱為定根。

  比丘們!而什麼是慧根?比丘們!這裡,聖弟子是有慧者,具備\twnr{導向生起與滅沒}{498.0}、聖、洞察、導向\twnr{苦的完全滅盡}{181.0}之慧。比丘們!這被稱為慧根。

  比丘們!這些被稱為五根。」



\sutta{12}{12}{簡要經第一}{https://agama.buddhason.org/SN/sn.php?keyword=48.12}
  「\twnr{比丘}{31.0}們!有這些五根,哪五個?信根……(中略)慧根。比丘們!這些是五根。

  比丘們!這些五根的達成者、完成者是\twnr{阿羅漢}{5.0};較之弱者是\twnr{不還者}{209.0};較之弱者是\twnr{一來者}{208.0};較之弱者是\twnr{入流者}{165.0};較之弱者是\twnr{隨法行}{167.0}者;較之弱者是隨\twnr{信行}{166.0}者。」



\sutta{13}{13}{簡要經第二}{https://agama.buddhason.org/SN/sn.php?keyword=48.13}
  「\twnr{比丘}{31.0}們!有這些五根,哪五個?信根……(中略)慧根。比丘們!這些是五根。

  比丘們!這些五根的達成者、完成者是\twnr{阿羅漢}{5.0};較之弱者是\twnr{不還者}{209.0};較之弱者是\twnr{一來者}{208.0};較之弱者是\twnr{入流者}{165.0};較之弱者是\twnr{隨法行}{167.0}者;較之弱者是隨\twnr{信行}{166.0}者。

  比丘們!像這樣,\twnr{根的不同有果的不同}{x629},果的不同有人的不同。」



\sutta{14}{14}{簡要經第三}{https://agama.buddhason.org/SN/sn.php?keyword=48.14}
  「\twnr{比丘}{31.0}們!有這些五根,哪五個?信根……(中略)慧根。比丘們!這些是五根。

  比丘們!這些五根的達成者、完成者是\twnr{阿羅漢}{5.0};較之弱者是\twnr{不還者}{209.0};較之弱者是\twnr{一來者}{208.0};較之弱者是\twnr{入流者}{165.0};較之弱者是\twnr{隨法行}{167.0}者;較之弱者是隨\twnr{信行}{166.0}者。

  比丘們!像這樣,\twnr{完全的實行者}{333.0}到達完全的,部分行者到達部分。我說:『比丘們!就像這樣,五根是\twnr{功不唐捐的}{334.1}。』」



\sutta{15}{15}{詳細經第一}{https://agama.buddhason.org/SN/sn.php?keyword=48.15}
  「\twnr{比丘}{31.0}們!有這些五根,哪五個?信根……(中略)慧根。比丘們!這些是五根。

  比丘們!這些五根的達成者、完成者是\twnr{阿羅漢}{5.0};較之弱者是\twnr{中般涅槃}{297.0}者;較之弱者是\twnr{生般涅槃}{298.0}者;較之弱者是\twnr{無行般涅槃}{299.0}者;較之弱者是\twnr{有行般涅槃}{300.0}者;較之弱者是\twnr{上流到阿迦膩吒}{301.0}者;較之弱者是\twnr{一來}{208.0}者;較之弱者是\twnr{入流者}{165.0};較之弱者是\twnr{隨法行}{167.0}者;較之弱者是\twnr{隨信行}{166.0}者。」



\sutta{16}{16}{詳細經第二}{https://agama.buddhason.org/SN/sn.php?keyword=48.16}
  「\twnr{比丘}{31.0}們!有這些五根,哪五個?信根……(中略)慧根。比丘們!這些是五根。

  比丘們!這些五根的達成者、完成者是\twnr{阿羅漢}{5.0};較之弱者是\twnr{中般涅槃}{297.0}者;較之弱者是\twnr{生般涅槃}{298.0}者;較之弱者是\twnr{無行般涅槃}{299.0}者;較之弱者是\twnr{有行般涅槃}{300.0}者;較之弱者是\twnr{上流到阿迦膩吒}{301.0}者;較之弱者是\twnr{一來}{208.0}者;較之弱者是\twnr{入流者}{165.0};較之弱者是\twnr{隨法行}{167.0}者;較之弱者是\twnr{隨信行}{166.0}者。

  比丘們!像這樣,\twnr{根的不同有果的不同}{x630},果的不同有人的不同。」



\sutta{17}{17}{詳細經第三}{https://agama.buddhason.org/SN/sn.php?keyword=48.17}
  「\twnr{比丘}{31.0}們!有這些五根,哪五個?信根……(中略)慧根。比丘們!這些是五根。

  比丘們!這些五根的達成者、完成者是\twnr{阿羅漢}{5.0};較之弱者是\twnr{中般涅槃}{297.0}者;較之弱者是\twnr{生般涅槃}{298.0}者;較之弱者是\twnr{無行般涅槃}{299.0}者;較之弱者是\twnr{有行般涅槃}{300.0}者;較之弱者是\twnr{上流到阿迦膩吒}{301.0}者;較之弱者是\twnr{一來}{208.0}者;較之弱者是\twnr{入流者}{165.0};較之弱者是\twnr{隨法行}{167.0}者;較之弱者是\twnr{隨信行}{166.0}者。

  比丘們!像這樣,\twnr{完全的實行者}{333.0}到達完全的,部分行者到達部分。我說:『比丘們!就像這樣,五根是\twnr{功不唐捐的}{334.1}。』」



\sutta{18}{18}{行者經}{https://agama.buddhason.org/SN/sn.php?keyword=48.18}
  「\twnr{比丘}{31.0}們!有這些五根,哪五個?信根……(中略)慧根。比丘們!這些是五根。

  比丘們!這些五根的達成者、完成者是\twnr{阿羅漢}{5.0};較之弱者是為了阿羅漢果的作證之行者;較之弱者是\twnr{不還者}{209.0};較之弱者是為了不還果的作證之行者;較之弱者是\twnr{一來}{208.0}者;較之弱者是為了\twnr{一來}{208.0}果的作證之行者;較之弱者是\twnr{入流者}{165.0};較之弱者\twnr{是為了入流果的作證之行者}{296.0}。

  比丘們!凡這些五根全部完全地、每一方面完全地缺乏者,我說他是『在外者;站在凡夫側者』。」



\sutta{19}{19}{具足經}{https://agama.buddhason.org/SN/sn.php?keyword=48.19}
  那時,\twnr{某位比丘}{39.0}去見世尊。抵達後,向世尊\twnr{問訊}{46.0}後,在一旁坐下。在一旁坐下的那位比丘對世尊說這個:

  「\twnr{大德}{45.0}!被稱為『根具足者,根具足者』,大德!什麼情形是根具足者呢?」

  「比丘!這裡,比丘導向寂靜、導向\twnr{正覺}{6.0}地\twnr{修習}{94.0}信根;導向寂靜、導向正覺地修習\twnr{活力根}{291.0};導向寂靜、導向正覺地修習念根;導向寂靜、導向正覺地修習定根;導向寂靜、導向正覺地修習慧根;導向寂靜、導向正覺,比丘!這個情形,比丘是根具足者。」



\sutta{20}{20}{漏的滅盡經}{https://agama.buddhason.org/SN/sn.php?keyword=48.20}
  「\twnr{比丘}{31.0}們!有這些五根,哪五個?信根……(中略)慧根。比丘們!這些是五根。

  比丘們!以這些五根的\twnr{已自我修習}{658.0}、已自我\twnr{多作}{95.0},比丘以諸\twnr{漏}{188.0}的滅盡,以證智自作證後,在當生中\twnr{進入後住於}{66.0}無漏\twnr{心解脫}{16.0}、\twnr{慧解脫}{539.0}。」

  較弱品第二,其\twnr{攝頌}{35.0}:

  「獲得、三則簡要,詳細三則在後,

   行者與具足者,第十則漏的滅盡。」





\pin{六根品}{21}{30}
\sutta{21}{21}{再有經}{https://agama.buddhason.org/SN/sn.php?keyword=48.21}
  「\twnr{比丘}{31.0}們!有這些五根,哪五個?信根……(中略)慧根。

  比丘們!只要我不如實證知這五根的\twnr{集起}{67.0}、滅沒、\twnr{樂味}{295.0}、\twnr{過患}{293.0}、\twnr{出離}{294.0},比丘們!我在包括天,在包括魔,在包括梵的世間;在包括沙門婆羅門,在包括天-人的\twnr{世代}{38.0}中,就不自稱『已\twnr{現正覺}{75.0}\twnr{無上遍正覺}{37.0}』。

  比丘們!但當我如實證知這五根的集起、滅沒、樂味、過患、出離,比丘們!那時,我在包括天,在包括魔,在包括梵的世間;在包括沙門婆羅門,在包括天-人的世代中,才自稱『已現正覺無上遍正覺』。而且,我的\twnr{智與見}{433.0}生起:『我的解脫是不動搖的,這是最後的出生,現在,沒有\twnr{再有}{21.0}。』」



\sutta{22}{22}{命根經}{https://agama.buddhason.org/SN/sn.php?keyword=48.22}
  「\twnr{比丘}{31.0}們!有這三根,哪三個?女根、男根、命根。比丘們!這些是三根。」



\sutta{23}{23}{完全智根經}{https://agama.buddhason.org/SN/sn.php?keyword=48.23}
  「\twnr{比丘}{31.0}們!有這\twnr{三根}{x631},哪三個?\twnr{『我將知未知的』根}{x632}、\twnr{完全智根}{x633}、\twnr{具知根}{x634}。比丘們!這些是三根。」



\sutta{24}{24}{一種子者經}{https://agama.buddhason.org/SN/sn.php?keyword=48.24}
  「\twnr{比丘}{31.0}們!有這些五根,哪五個?信根……(中略)、慧根。比丘們!這些是五根。

  比丘們!這些五根的達成者、完成者是\twnr{阿羅漢}{5.0};較之弱者是\twnr{中般涅槃}{297.0}者;較之弱者是\twnr{生般涅槃}{298.0}者;較之弱者是\twnr{無行般涅槃}{299.0}者;較之弱者是\twnr{有行般涅槃}{300.0}者;較之弱者是\twnr{上流到阿迦膩吒}{301.0}者;較之弱者是\twnr{一來}{208.0}者;較之弱者是\twnr{一種子者}{303.0};較之弱者是\twnr{良家到良家者}{302.0};較之弱者是\twnr{最多七次}{161.0}者;較之弱者是\twnr{隨法行}{167.0}者;較之弱者是\twnr{隨信行}{166.0}者。」



\sutta{25}{25}{概要經}{https://agama.buddhason.org/SN/sn.php?keyword=48.25}
  「\twnr{比丘}{31.0}們!有這些六根,哪六個?眼根、耳根、鼻根、舌根、身根、意根,比丘們!這些是六根。」



\sutta{26}{26}{入流者經}{https://agama.buddhason.org/SN/sn.php?keyword=48.26}
  「\twnr{比丘}{31.0}們!有這些六根,哪六個?眼根……(中略)意根,比丘們!當聖弟子如實知道這些六根的\twnr{集起}{67.0}、滅沒、\twnr{樂味}{295.0}、\twnr{過患}{293.0}、\twnr{出離}{294.0},比丘們!這位聖弟子被稱為\twnr{入流者}{165.0}、不墮惡趣法者、\twnr{決定者}{159.0}、\twnr{正覺為彼岸者}{160.0}。」 



\sutta{27}{27}{阿羅漢經}{https://agama.buddhason.org/SN/sn.php?keyword=48.27}
  「\twnr{比丘}{31.0}們!有這些六根,哪六個?眼根、耳根、鼻根、舌根、身根、意根,比丘們!當比丘如實知道這些六根的\twnr{集起}{67.0}、滅沒、\twnr{樂味}{295.0}、\twnr{過患}{293.0}、\twnr{出離}{294.0}後,不執取後成為解脫者,比丘們!這被稱為漏已滅盡的、已完成的、\twnr{應該被作的已作的}{20.0}、負擔已卸的、\twnr{自己的利益已達成的}{189.0}、\twnr{有之結已滅盡的}{190.0}、以\twnr{究竟智}{191.0}解脫的\twnr{阿羅漢}{5.0}比丘。」 



\sutta{28}{28}{正覺經}{https://agama.buddhason.org/SN/sn.php?keyword=48.28}
  「\twnr{比丘}{31.0}們!有這些六根,哪六個?眼根、耳根、鼻根、舌根、身根、意根。

  比丘們!只要我不如實證知這六根的\twnr{集起}{67.0}、滅沒、\twnr{樂味}{295.0}、\twnr{過患}{293.0}、\twnr{出離}{294.0},比丘們!我在包括天,在包括魔,在包括梵的世間;在包括沙門婆羅門,在包括天-人的\twnr{世代}{38.0}中,就不自稱『已\twnr{現正覺}{75.0}\twnr{無上遍正覺}{37.0}』。

  比丘們!但當我如實證知這六根的集起、滅沒、樂味、過患、出離,比丘們!那時,我在包括天,在包括魔,在包括梵的世間;在包括沙門婆羅門,在包括天-人的世代中,才自稱『已現正覺無上遍正覺』。而且,我的\twnr{智與見}{433.0}生起:『我的解脫是不動搖的,這是最後的出生,現在,沒有\twnr{再有}{21.0}。』」



\sutta{29}{29}{沙門婆羅門經第一}{https://agama.buddhason.org/SN/sn.php?keyword=48.29}
  「\twnr{比丘}{31.0}們!有這些六根,哪六個?眼根、耳根、鼻根、舌根、身根、意根。比丘們!凡任何\twnr{沙門}{29.0}或\twnr{婆羅門}{17.0}不如實知道這些六根的\twnr{集起}{67.0}、滅沒、\twnr{樂味}{295.0}、\twnr{過患}{293.0}、\twnr{出離}{294.0}者,比丘們!那些沙門或婆羅門不被我認同為\twnr{沙門中的沙門}{560.0},或婆羅門中的婆羅門,而且,那些\twnr{尊者}{200.0}也不以證智自作證後,在當生中\twnr{進入後住於}{66.0}\twnr{沙門義}{327.0}或婆羅門義。

  比丘們!而凡任何沙門或婆羅門如實知道這些六根的集起、滅沒、樂味、過患、出離者,比丘們!那些沙門或婆羅門被我認同為沙門中的沙門,或婆羅門中的婆羅門,而且,那些尊者也以證智自作證後,在當生中進入後住於沙門義或婆羅門義。」



\sutta{30}{30}{沙門婆羅門經第二}{https://agama.buddhason.org/SN/sn.php?keyword=48.30}
  「\twnr{比丘}{31.0}們!凡任何\twnr{沙門}{29.0}或\twnr{婆羅門}{17.0}不知道眼根、不知道眼根集、不知道眼根滅、不知道導向眼根\twnr{滅道跡}{69.0};不知道耳根……(中略)不知道鼻根……(中略)不知道舌根……(中略)不知道身根……(中略)不知道意根、不知道意根集、不知道意根滅、不知道導向意根滅道跡者,比丘們!那些沙門或婆羅門不被我認同為\twnr{沙門中的沙門}{560.0},或婆羅門中的婆羅門,而且,那些\twnr{尊者}{200.0}也不以證智自作證後,在當生中\twnr{進入後住於}{66.0}\twnr{沙門義}{327.0}或婆羅門義。

  比丘們!而凡任何沙門或婆羅門知道眼根、知道眼根集、知道眼根滅、知道導向眼根滅道跡;耳根……(中略)鼻根……(中略)舌根……(中略)身根……(中略)知道意根、知道意根集、知道意根滅、知道導向意根滅道跡者,比丘們!那些沙門或婆羅門被我認同為沙門中的沙門,或婆羅門中的婆羅門,而且,那些尊者也以證智自作證後,在當生中進入後住於沙門義或婆羅門義。」

  六根品第三,其\twnr{攝頌}{35.0}:

  「再有、命、完全智,一種子者與概要,

   流、阿羅漢、正覺,以及二則沙門婆羅門。」





\pin{樂根品}{31}{40}
\sutta{31}{31}{概要經}{https://agama.buddhason.org/SN/sn.php?keyword=48.31}
  「\twnr{比丘}{31.0}們!有這些五根,哪五個?樂根、苦根、喜悅根、憂根、\twnr{平靜根}{228.1},比丘們!這些是五根。」



\sutta{32}{32}{入流者經}{https://agama.buddhason.org/SN/sn.php?keyword=48.32}
  「\twnr{比丘}{31.0}們!有這些五根,哪五個?樂根、苦根、喜悅根、憂根、\twnr{平靜根}{228.1},比丘們!當聖弟子如實知道這些五根的\twnr{集起}{67.0}、滅沒、\twnr{樂味}{295.0}、\twnr{過患}{293.0}、\twnr{出離}{294.0},比丘們!這位聖弟子被稱為\twnr{入流者}{165.0}、不墮惡趣法者、\twnr{決定者}{159.0}、\twnr{正覺為彼岸者}{160.0}。」 



\sutta{33}{33}{阿羅漢經}{https://agama.buddhason.org/SN/sn.php?keyword=48.33}
  「\twnr{比丘}{31.0}們!有這些五根,哪五個?樂根、苦根、喜悅根、憂根、\twnr{平靜根}{228.1}。比丘們!當比丘如實知道這些五根的\twnr{集起}{67.0}、滅沒、\twnr{樂味}{295.0}、\twnr{過患}{293.0}、\twnr{出離}{294.0}後,不執取後成為解脫者,比丘們!這被稱為漏已滅盡的、已完成的、\twnr{應該被作的已作的}{20.0}、負擔已卸的、\twnr{自己的利益已達成的}{189.0}、\twnr{有之結已滅盡的}{190.0}、以\twnr{究竟智}{191.0}解脫的\twnr{阿羅漢}{5.0}比丘。」 



\sutta{34}{34}{沙門婆羅門經第一經}{https://agama.buddhason.org/SN/sn.php?keyword=48.34}
  「\twnr{比丘}{31.0}們!有這些五根,哪五個?樂根、苦根、喜悅根、憂根、\twnr{平靜根}{228.1},比丘們!凡任何\twnr{沙門}{29.0}或\twnr{婆羅門}{17.0}不如實知道這些五根的\twnr{集起}{67.0}、滅沒、\twnr{樂味}{295.0}、\twnr{過患}{293.0}、\twnr{出離}{294.0}者,比丘們!那些沙門或婆羅門不被我認同為\twnr{沙門中的沙門}{560.0},或婆羅門中的婆羅門,而且,那些\twnr{尊者}{200.0}也不以證智自作證後,在當生中\twnr{進入後住於}{66.0}\twnr{沙門義}{327.0}或婆羅門義。

  比丘們!而凡任何沙門或婆羅門如實知道這些五根的集起、滅沒、樂味、過患、出離者,比丘們!那些沙門或婆羅門被我認同為沙門中的沙門,或婆羅門中的婆羅門,而且,那些尊者也以證智自作證後,在當生中進入後住於沙門義或婆羅門義。」



\sutta{35}{35}{沙門婆羅門經第二}{https://agama.buddhason.org/SN/sn.php?keyword=48.35}
  「\twnr{比丘}{31.0}們!有這些五根,哪五個?樂根、苦根、喜悅根、憂根、\twnr{平靜根}{228.1}。比丘們!凡任何\twnr{沙門}{29.0}或\twnr{婆羅門}{17.0}不知道樂根、不知道樂根集、不知道樂根滅、不知道導向樂根\twnr{滅道跡}{69.0};不知道苦根……(中略)不知道喜悅根……(中略)不知道憂根……(中略)不知道平靜根、不知道平靜根集、不知道平靜根滅、不知道導向平靜根滅道跡者,比丘們!那些沙門或婆羅門不被我認同為\twnr{沙門中的沙門}{560.0},或婆羅門中的婆羅門,而且,那些\twnr{尊者}{200.0}也不以證智自作證後,在當生中\twnr{進入後住於}{66.0}\twnr{沙門義}{327.0}或婆羅門義。

  比丘們!而凡任何沙門或婆羅門知道樂根、知道樂根集、知道樂根滅、知道導向樂根滅道跡;知道苦根……(中略)知道喜悅根……(中略)知道憂根……(中略)知道平靜根、知道平靜根集、知道平靜根滅、知道導向平靜根滅道跡者,比丘們!那些沙門或婆羅門被我認同為沙門中的沙門,或婆羅門中的婆羅門,而且,那些尊者也以證智自作證後,在當生中進入後住於沙門義或婆羅門義。」



\sutta{36}{36}{解析經第一}{https://agama.buddhason.org/SN/sn.php?keyword=48.36}
  「\twnr{比丘}{31.0}們!有這些五根,哪五個?樂根、苦根、喜悅根、憂根、\twnr{平靜根}{228.1}。

  比丘們!而什麼是樂根呢?比丘們!凡身的樂,身的愉快,身觸所生樂、愉快的被感受,比丘們!這被稱為樂根。

  比丘們!而什麼是苦根呢?比丘們!凡身的苦,身的不愉快,身觸所生苦、不愉快的被感受,比丘們!這被稱為苦根。

  比丘們!而什麼是喜悅根呢?比丘們!凡心的樂,心的愉快,意觸所生樂、愉快的被感受,比丘們!這被稱為喜悅根。

  比丘們!而什麼是憂根呢?比丘們!凡心的苦,心的不愉快,意觸所生苦、不愉快的被感受,比丘們!這被稱為憂根。

  比丘們!而什麼是平靜根呢?比丘們!凡身的或心的既非愉快也非不愉快被感受,比丘們!這被稱為平靜根。

  比丘們!這些是五根。」



\sutta{37}{37}{解析經第二}{https://agama.buddhason.org/SN/sn.php?keyword=48.37}
  「\twnr{比丘}{31.0}們!有這些五根,哪五個?樂根、苦根、喜悅根、憂根、\twnr{平靜根}{228.1}。

  比丘們!而什麼是樂根呢?比丘們!凡身的樂,身的愉快,身觸所生樂、愉快的被感受,比丘們!這被稱為樂根。

  比丘們!而什麼是苦根呢?比丘們!凡身的苦,身的不愉快,身觸所生苦、不愉快的被感受,比丘們!這被稱為苦根。

  比丘們!而什麼是喜悅根呢?比丘們!凡心的樂,心的愉快,意觸所生樂、愉快的被感受,比丘們!這被稱為喜悅根。

  比丘們!而什麼是憂根呢?比丘們!凡心的苦,心的不愉快,意觸所生苦、不愉快的被感受,比丘們!這被稱為憂根。

  比丘們!而什麼是平靜根呢?比丘們!凡身的或心的既非愉快也非不愉快被感受,比丘們!這被稱為平靜根。

  比丘們!在那裡,凡樂根與凡喜悅根,應該被看作那是樂受。

  比丘們!在那裡,凡苦根與凡憂根,應該被看作那是苦受。

  比丘們!在那裡,凡這個平靜根,應該被看作那是不苦不樂受。

  比丘們!這些是五根。」



\sutta{38}{38}{解析經第三}{https://agama.buddhason.org/SN/sn.php?keyword=48.38}
  「\twnr{比丘}{31.0}們!有這些五根,哪五個?樂根、苦根、喜悅根、憂根、\twnr{平靜根}{228.1}。

  比丘們!而什麼是樂根呢?比丘們!凡身的樂,身的愉快,身觸所生樂、愉快的被感受,比丘們!這被稱為樂根。

  比丘們!而什麼是苦根呢?比丘們!凡身的苦,身的不愉快,身觸所生苦、不愉快的被感受,比丘們!這被稱為苦根。

  比丘們!而什麼是喜悅根呢?比丘們!凡心的樂,心的愉快,意觸所生樂、愉快的被感受,比丘們!這被稱為喜悅根。

  比丘們!而什麼是憂根呢?比丘們!凡心的苦,心的不愉快,意觸所生苦、不愉快的被感受,比丘們!這被稱為憂根。

  比丘們!而什麼是平靜根呢?比丘們!凡身的或心的既非愉快也非不愉快被感受,比丘們!這被稱為平靜根。

  比丘們!在那裡,凡樂根與凡喜悅根,應該被看作那是樂受。

  比丘們!在那裡,凡苦根與凡憂根,應該被看作那是苦受。

  比丘們!在那裡,凡這個平靜根,應該被看作那是不苦不樂受。

  比丘們!像這樣,這些五根依\twnr{法門}{562.0}有五個後成為三個,有三個後成為五個。」



\sutta{39}{39}{如木柴經}{https://agama.buddhason.org/SN/sn.php?keyword=48.39}
  「\twnr{比丘}{31.0}們!有這些五根,哪五個?樂根、苦根、喜悅根、憂根、\twnr{平靜根}{228.1}。

  比丘們!\twnr{緣於}{252.0}能被感受為樂之觸,樂根生起,正當是樂的時,知道:『我是樂的。』就以那個能被感受為樂之觸的\twnr{滅}{68.0},他知道:『凡對應那個所感受的:緣於能被感受為樂之觸所生起的樂根,它被滅,它被平息。』

  比丘們!緣於能被感受為苦之觸,苦根生起,正當是苦的時,知道:『我是苦的。』就以那個能被感受為苦之觸的滅,他知道:『凡對應那個所感受的:緣於能被感受為苦之觸所生起的苦根,它被滅,它被平息。』

  比丘們!緣於能感受喜悅之觸,喜悅根生起,正當是快樂的時,知道:『我是快樂的。』就以那個能感受喜悅之觸的滅,他知道:『凡對應那個所感受的:緣於能感受喜悅之觸所生起的喜悅根,它被滅,它被平息。』

  比丘們!緣於能感受憂之觸,憂根生起,正當是不快樂的時,知道:『我是不快樂的。』就以那能感受憂之觸的滅,他知道:『凡對應那個所感受的:緣於能感受憂之觸所生起的憂根,它被滅,它被平息。』

  比丘們!緣於能感受平靜之觸,平靜根生起,正當是平靜的時,知道:『我是平靜的。』就以那能感受平靜之觸的滅,他知道:『凡對應那個所感受的:緣於能感受平靜之觸所生起的平靜根,它被滅,它被平息。』

  比丘們!猶如從兩塊柴的磨擦、結合,熱被產生,火生起。就從那兩塊柴的分離分置,凡對應那個的熱,它被滅,它被平息。同樣的,比丘們!緣於能被感受為樂之觸,樂根生起,正當是樂的時,知道:『我是樂的。』就以那個能被感受為樂之觸的滅,他知道:『凡對應那個所感受的:緣於能被感受為樂之觸所生起的樂根,它被滅,它被平息。

  比丘們!緣於能被感受為苦之觸……(中略)比丘們!緣於能感受喜悅之觸……(中略)比丘們!緣於能感受憂之觸……(中略)比丘們!緣於能感受平靜之觸,平靜根生起,正當是平靜的時,知道:『我是平靜的。』就以那能感受平靜之觸的滅,他知道:『凡對應那個所感受的:緣於能感受平靜之觸所生起的平靜根,它被滅,它被平息。』」[\suttaref{SN.12.62}, \suttaref{SN.36.10}]



\sutta{40}{40}{非慣常順序的經}{https://agama.buddhason.org/SN/sn.php?keyword=48.40}
  「\twnr{比丘}{31.0}們!有這些五根,哪五個?苦根、憂根、樂根、喜悅根、\twnr{平靜根}{228.1}。

  比丘們!這裡,住於不放逸的、熱心的、自我努力的比丘的苦根生起,他這麼知道:『我的這苦根已生起,那是有相的、有因的、\twnr{有行的}{x635}、有\twnr{緣}{180.0}的,「而那個無相的、無因的、無行的、無緣的苦根將生起。」\twnr{這不存在可能性}{650.0}。』他知道苦根、知道苦根集、知道苦根滅、知道那個已生起的苦根無殘餘地被滅之處。已生起的苦根在哪裡無殘餘地被滅?比丘們!這裡,比丘就從離諸欲後,從離諸不善法後,\twnr{進入後住於}{66.0}有尋、\twnr{有伺}{175.0},\twnr{離而生喜、樂}{174.0}的初禪,在那裡,已生起的苦根無殘餘地被滅。比丘們!這被稱為比丘知道苦根的\twnr{滅}{68.0},他\twnr{為了那個目的}{x636}集中心。

  比丘們!又,這裡,住於不放逸的、熱心的、自我努力的比丘的憂根生起,他這麼知道:『我的這憂根已生起,那是有相的、有因的、有行的、有緣的,「而那個無相的、無因的、無行的、無緣的憂根將生起。」這不存在可能性。』他知道憂根、知道憂根集、知道憂根滅、知道那個已生起的憂根無殘餘地被滅之處。已生起的憂根在哪裡無殘餘地被滅?比丘們!這裡,比丘從尋與伺的平息,\twnr{自身內的明淨}{256.0},\twnr{心的專一性}{255.0},進入後住於無尋、無伺,定而生喜、樂的第二禪,在那裡,已生起的憂根無殘餘地被滅。比丘們!這被稱為比丘知道憂根的滅,他為了那個目的集中心。

  比丘們!又,這裡,住於不放逸的、熱心的、自我努力的比丘的樂根生起,他這麼知道:『我的這樂根已生起,那是有相的、有因的、有行的、有緣的,「而那個無相的、無因的、無行的、無緣的樂根將生起。」這不存在可能性。』他知道樂根、知道樂根集、知道樂根滅、知道那個已生起的樂根無殘餘地被滅之處。已生起的樂根在哪裡無殘餘地被滅?比丘們!這裡,比丘從喜的\twnr{褪去}{77.0}、住於\twnr{平靜}{228.0}、有念正知、以身體感受樂,進入後住於凡聖者們告知『他是平靜者、具念者、\twnr{安樂住者}{317.0}』的第三禪,在那裡,已生起的樂根無殘餘地被滅。比丘們!這被稱為比丘知道樂根的滅,他為了那個目的集中心。

  比丘們!又,這裡,住於不放逸的、熱心的、自我努力的比丘的喜悅根生起,他這麼知道:『我的這喜悅根已生起,那是有相的、有因的、有行的、有緣的,「而那個無相的、無因的、無行的、無緣的喜悅根將生起。」這不存在可能性。』他知道喜悅根、知道喜悅根集、知道喜悅根滅、知道那個已生起的喜悅根無殘餘地被滅之處。已生起的喜悅根在哪裡無殘餘地被滅?比丘們!這裡,比丘從樂的捨斷與從苦的捨斷,就在之前諸喜悅、憂的滅沒,進入後住於不苦不樂,\twnr{平靜、念遍純淨}{494.0}的第四禪,在那裡,已生起的喜悅根無殘餘地被滅。比丘們!這被稱為比丘知道喜悅根的滅,他為了那個目的集中心。

  比丘們!又,這裡,住於不放逸的、熱心的、自我努力的比丘的平靜根生起,他這麼知道:『我的這平靜根已生起,那是有相的、有因的、有行的、有緣的,「而那個無相的、無因的、無行的、無緣的平靜根將生起。」這不存在可能性。』他知道平靜根、知道平靜根集、知道平靜根滅、知道那個已生起的平靜根無殘餘地被滅之處。已生起的平靜根在哪裡無殘餘地被滅?比丘們!這裡,比丘超越一切非想非非想處後,進入後住於\twnr{想受滅}{416.0},在那裡,已生起的平靜根無殘餘地被滅。比丘們!這被稱為比丘知道平靜根的滅,他為了那個目的集中心。」

  樂根品第四,其\twnr{攝頌}{35.0}:

  「概要、流、阿羅漢,二則沙門婆羅門,

   解析三說,木柴、非慣常順序的。」





\pin{老品}{41}{50}
\sutta{41}{41}{老法經}{https://agama.buddhason.org/SN/sn.php?keyword=48.41}
  \twnr{被我這麼聽聞}{1.0}:

  \twnr{有一次}{2.0},\twnr{世尊}{12.0}住在舍衛城東園鹿母講堂。

  當時,世尊傍晚時,從\twnr{獨坐}{92.0}出來,坐在西方的陽光中曬著背。

  那時,\twnr{尊者}{200.0}阿難去見世尊。抵達後,向世尊\twnr{問訊}{46.0},接著以手逐一地按摩世尊的肢體,說這個:

  「不可思議啊,\twnr{大德}{45.0}!\twnr{未曾有}{206.0}啊,大德!大德!而像這樣,現在世尊的膚色不是那麼遍純淨的、皎潔的,而肢體全部是鬆弛的、起皺的、身體前傾的,諸根的變異被看見:眼根的、耳根的、鼻根的、舌根的、身根的。」

  「這確實是這樣,阿難!在年輕時有老法;在無病時有病法;在活命時有\twnr{死法}{587.3},而膚色正不是那麼遍純淨的、皎潔的,而肢體全部是鬆弛的、起皺的、身體前傾的,諸根的變異被看見:眼根的、耳根的、鼻根的、舌根的、身根的。」

  世尊說這個,說這個後,\twnr{善逝}{8.0}、\twnr{大師}{145.0}又更進一步說這個:

  「討厭你卑微的老,醜之產生的老,

   那麼適意的身體,被老壓碎。

   凡即使能活百歲者,他還是死亡作終的,

   不回避任何東西,只壓碎一切。」



\sutta{42}{42}{巫男巴婆羅門經}{https://agama.buddhason.org/SN/sn.php?keyword=48.42}
  起源於舍衛城。

  那時,巫男巴婆羅門去見\twnr{世尊}{12.0}。抵達後,與世尊一起互相問候。交換應該被互相問候的友好交談後,在一旁坐下。在一旁坐下的巫男巴婆羅門對世尊說這個:

  「\twnr{喬達摩}{80.0}尊師!有這些不同境域、\twnr{不同行境}{897.0}的五根,不經驗彼此的行境、境域,哪五個?眼根、耳根、鼻根、舌根、身根,喬達摩尊師!這些不同境域、不同行境、不經驗彼此行境、境域之五根的依處是什麼?什麼經驗他們的行境、境域呢?」

  「婆羅門!有這不同行境、不同境域的五根,他們不經驗彼此的行境、境域,哪五個?眼根、耳根、鼻根、舌根、身根,婆羅門!這些不同境域、不同行境、不經驗彼此行境、境域之五根的依處是意,意就經驗他們的行境、境域。」

  「喬達摩尊師!那麼,意的依處是什麼?」 

  「婆羅門!念是意的依處。」

  「喬達摩尊師!那麼,念的依處是什麼?」 

  「婆羅門!解脫是念的依處。」

  「喬達摩尊師!那麼,解脫的依處是什麼?」 

  「婆羅門!涅槃是解脫的依處。」

  「喬達摩尊師!那麼,涅槃的依處是什麼?」 

  「婆羅門!你超越問題,你不能把握問題的極限,婆羅門!因為\twnr{梵行}{381.0}被住於涅槃為立足處、\twnr{涅槃為彼岸}{226.2}、涅槃為完結。」

  那時,巫男巴婆羅門歡喜、\twnr{隨喜}{85.0}世尊所說後,從座位起來、向世尊\twnr{問訊}{46.0}、\twnr{作右繞}{47.0}後,離開。

  那時,在巫男巴婆羅門離開不久,世尊召喚\twnr{比丘}{31.0}們:

  「比丘們!猶如\twnr{重閣}{213.0}或重閣會堂,在太陽昇起時,從東邊窗戶,光線經窗戶進入後,會被住立在何處?」

  「\twnr{大德}{45.0}!在西邊的牆壁。」

  「同樣的,比丘們!巫男巴婆羅門在如來上的信已建立、已生根、已住立,是堅固的,不能被沙門,或被婆羅門,或被天,或被魔,或被梵,或被世間中任何人動搖,比丘們!在這時,如果巫男巴婆羅門死了,沒有結,以該結巫男巴婆羅門被結縛再回來這個世間。」



\sutta{43}{43}{娑雞多經}{https://agama.buddhason.org/SN/sn.php?keyword=48.43}
  \twnr{被我這麼聽聞}{1.0}:

  \twnr{有一次}{2.0},\twnr{世尊}{12.0}住在娑雞多城漆黑林的鹿園。

  在那裡,世尊召喚\twnr{比丘}{31.0}們:

  「比丘們!有\twnr{法門}{562.0},由於該法門,五根成為五力;五力成為五根嗎?」

  「大德!我們的法是世尊為根本的、\twnr{世尊為導引的}{56.0}、世尊為依歸的,大德!就請世尊說明這個所說的義理,\twnr{那就好了}{44.0}!聽聞世尊的[教說]後,比丘們將會\twnr{憶持}{57.0}。」

  「比丘們!有法門,由於該法門,五根成為五力;五力成為五根。比丘們!什麼法門,由於該法門,五根成為五力;五力成為五根呢?比丘們!凡信根者,那是信力;凡信力者,那是信根,凡活力根者,那是\twnr{活力之力}{306.0};凡活力之力者,那是活力根,凡念根者,那是念力;凡念力者,那是念根,凡定根者,那是定力;凡定力者,那是定根,凡慧根者,那是慧力;凡慧力者,那是慧根。

  比丘們!猶如傾向東的、斜向東的、坡斜向東的河,在它的中央有洲島,比丘們!有法門,由於該法門,就名為(就走到稱呼)『那條河有單一條水流』。比丘們!又,有法門,由於該法門,就名為『那條河有二條水流』。

  比丘們!什麼法門,由於該法門,就名為『那條河有單一條水流』呢?比丘們!凡在那個島東邊的水與凡在那個島西邊的水,比丘們!這是法門,由於該法門,就名為『那條河有單一條水流』。

  比丘們!什麼法門,由於該法門,就名為『那條河有二條水流』呢?比丘們!凡在那個島北邊的水與凡在那個島南邊的水,比丘們!這是法門,由於該法門,就名為『那條河有二條水流』。

  同樣的,比丘們!凡信根者,那是信力;凡信力者,那是信根,凡活力根者,那是活力之力;凡活力之力者,那是活力根,凡念根者,那是念力;凡念力者,那是念根,凡定根者,那是定力;凡定力者,那是定根,凡慧根者,那是慧力;凡慧力者,那是慧根。

  比丘們!以這些五根的\twnr{已自我修習}{658.0}、已自我\twnr{多作}{95.0},比丘以諸\twnr{漏}{188.0}的滅盡,以證智自作證後,在當生中\twnr{進入後住於}{66.0}無漏\twnr{心解脫}{16.0}、\twnr{慧解脫}{539.0}。」



\sutta{44}{44}{東門屋經}{https://agama.buddhason.org/SN/sn.php?keyword=48.44}
  \twnr{被我這麼聽聞}{1.0}:

  \twnr{有一次}{2.0},\twnr{世尊}{12.0}住在舍衛城的東門屋。

  在那裡,世尊召喚\twnr{尊者}{200.0}舍利弗:

  「舍利弗!你相信:信根已\twnr{修習}{94.0}、已\twnr{多作}{95.0},有\twnr{不死}{123.0}的立足處、不死的\twnr{彼岸}{226.0}、不死的完結……(中略)慧根已修習、已多作,有不死的立足處、不死的彼岸、不死的完結嗎?」

  「\twnr{大德}{45.0}!在這裡,我對世尊不以信走到:信根……(中略)慧根已修習、已多作,有不死的立足處、不死的彼岸、不死的完結。大德!凡如果這個未被知道、未被看見、未被發現、未被作證、未被以慧觸達者,在那裡,他們對其他人以信走到:信根……(中略)慧根已修習、已多作,有不死的立足處、不死的彼岸、不死的完結。大德!但,凡這個已被知道、已被看見、已被發現、已被作證、已被以慧觸達者,在那裡,他們是自信的、無疑的:信根……(中略)慧根已修習、已多作,有不死的立足處、不死的彼岸、不死的完結。大德!而對我,這個已被知道、已被看見、已被發現、已被作證、已被以慧觸達,在那裡,我是自信的、無疑的:信根……(中略)慧根已修習、已多作,有不死的立足處、不死的彼岸、不死的完結。」

  「\twnr{好}{44.0}!好!舍利弗!舍利弗!凡如果這個未被知道、未被看見、未被發現、未被作證、未被以慧觸達者,在那裡,他們對其他人以信走到:信根已修習、已多作,有不死的立足處、不死的彼岸、不死的完結……(中略)慧根已修習、已多作,有不死的立足處、不死的彼岸、不死的完結。舍利弗!但,凡這個已被知道、已被看見、已被發現、已被作證、已被以慧觸達者,在那裡,他們是自信的、無疑的:信根……(中略)慧根已修習、已多作,有不死的立足處、不死的彼岸、不死的完結。」



\sutta{45}{45}{東園經第一}{https://agama.buddhason.org/SN/sn.php?keyword=48.45}
  \twnr{被我這麼聽聞}{1.0}:

  \twnr{有一次}{2.0},\twnr{世尊}{12.0}住在舍衛城東園鹿母講堂。

  在那裡,世尊召喚\twnr{比丘}{31.0}們:

  「比丘們!以幾根的\twnr{已自我修習}{658.0}、已自我\twnr{多作}{95.0},漏盡比丘\twnr{記說}{179.0}\twnr{完全智}{489.0}:『我了知:「\twnr{出生已盡}{18.0},\twnr{梵行已完成}{19.0},\twnr{應該被作的已作}{20.0},\twnr{不再有此處[輪迴]的狀態}{21.1}。」』呢?」

  「大德!我們的法以世尊為根本……(中略)。」

  「比丘們!以一根已自我修習、已自我多作,漏盡比丘記說完全智:『我了知:「出生已盡,梵行已完成,應該被作的已作,不再有此處[輪迴]的狀態。」』哪一個?慧根。比丘們!對有慧的聖弟子來說,隨行那個的信確立;隨行那個的活力確立;隨行那個的念確立;隨行那個的定確立,比丘們!以這一根已自我修習、已自我多作,漏盡比丘記說完全智:『我了知:「出生已盡,梵行已完成,應該被作的已作,不再有此處[輪迴]的狀態。」』」



\sutta{46}{46}{東園經第二}{https://agama.buddhason.org/SN/sn.php?keyword=48.46}
  就那個起源(與前一經相同的序)。 

  「\twnr{比丘}{31.0}們!以幾根的\twnr{已自我修習}{658.0}、已自我\twnr{多作}{95.0},漏盡比丘\twnr{記說}{179.0}\twnr{完全智}{489.0}:『我了知:「\twnr{出生已盡}{18.0},\twnr{梵行已完成}{19.0},\twnr{應該被作的已作}{20.0},\twnr{不再有此處[輪迴]的狀態}{21.1}。」』呢?」

  「大德!我們的法以世尊為根本……(中略)。」

  「比丘們!以二根的已自我修習、已自我多作,漏盡比丘記說完全智:『我了知:「出生已盡,梵行已完成,應該被作的已作,不再有此處[輪迴]的狀態。」』哪二個?聖慧與聖解脫。比丘們!因為凡如果有聖慧,那是慧根,比丘們!因為凡如果有聖解脫,那是定根。比丘們!以這些二根的已自我修習、已自我多作,漏盡比丘記說完全智:『我了知:「出生已盡,梵行已完成,應該被作的已作,不再有此處[輪迴]的狀態。」』」



\sutta{47}{47}{東園經第三}{https://agama.buddhason.org/SN/sn.php?keyword=48.47}
  就那個起源(與前一經相同的序)。 

  「\twnr{比丘}{31.0}們!以幾根的\twnr{已自我修習}{658.0}、已自我\twnr{多作}{95.0},漏盡比丘\twnr{記說}{179.0}\twnr{完全智}{489.0}:『我了知:「\twnr{出生已盡}{18.0},\twnr{梵行已完成}{19.0},\twnr{應該被作的已作}{20.0},\twnr{不再有此處[輪迴]的狀態}{21.1}。」』呢?」

  「大德!我們的法以世尊為根本……(中略)。」

  「比丘們!以四根已自我修習、已自我多作,漏盡比丘記說完全智:『我了知:「出生已盡,梵行已完成,應該被作的已作,不再有此處[輪迴]的狀態。」』哪四個?\twnr{活力根}{291.0}、念根、定根、慧根。比丘們!以這些四根的已自我修習、已自我多作,漏盡比丘記說完全智:『我了知:「出生已盡,梵行已完成,應該被作的已作,不再有此處[輪迴]的狀態。」』」



\sutta{48}{48}{東園經第四}{https://agama.buddhason.org/SN/sn.php?keyword=48.48}
  就那個起源(與前一經相同的序)。 

  「\twnr{比丘}{31.0}們!以幾根的\twnr{已自我修習}{658.0}、已自我\twnr{多作}{95.0},漏盡比丘\twnr{記說}{179.0}\twnr{完全智}{489.0}:『我了知:「\twnr{出生已盡}{18.0},\twnr{梵行已完成}{19.0},\twnr{應該被作的已作}{20.0},\twnr{不再有此處[輪迴]的狀態}{21.1}。」』呢?」

  「大德!我們的法以世尊為根本……(中略)。」

  「比丘們!以五根的已自我修習、已自我多作,漏盡比丘記說完全智:『我了知:「出生已盡,梵行已完成,應該被作的已作,不再有此處[輪迴]的狀態。」』哪五個?信根、\twnr{活力根}{291.0}、念根、定根、慧根。比丘們!以這些五根的已自我修習、已自我多作,漏盡比丘記說完全智:『我了知:「出生已盡,梵行已完成,應該被作的已作,不再有此處[輪迴]的狀態。」』」



\sutta{49}{49}{賓頭盧婆羅墮若經}{https://agama.buddhason.org/SN/sn.php?keyword=48.49}
  \twnr{被我這麼聽聞}{1.0}:

  \twnr{有一次}{2.0},\twnr{世尊}{12.0}住在\twnr{憍賞彌}{140.0}瞿師羅園。當時,\twnr{完全智}{489.0}被\twnr{尊者}{200.0}賓頭盧婆羅墮若\twnr{記說}{179.0}:「我了知:『\twnr{出生已盡}{18.0},\twnr{梵行已完成}{19.0},\twnr{應該被作的已作}{20.0},\twnr{不再有此處[輪迴]的狀態}{21.1}。』」那時,眾多\twnr{比丘}{31.0}去見世尊。抵達後,向世尊\twnr{問訊}{46.0}後,在一旁坐下。在一旁坐下的那些比丘對世尊說這個:

  「\twnr{大德}{45.0}!完全智被尊者賓頭盧婆羅墮若記說:『我了知:「出生已盡,梵行已完成,應該被作的已作,不再有此處[輪迴]的狀態。」』大德!什麼因由,完全智被看見的尊者賓頭盧婆羅墮若記說:『我了知:「出生已盡,梵行已完成,應該被作的已作,不再有此處[輪迴]的狀態。」』呢?」

  「比丘們!以三根的\twnr{已自我修習}{658.0}、已自我\twnr{多作}{95.0},完全智被賓頭盧婆羅墮若比丘記說:『我了知:「出生已盡,梵行已完成,應該被作的已作,不再有此處[輪迴]的狀態。」』哪三個?念根、定根、慧根。比丘們!以這三根的已自我修習、已自我多作,完全智被賓頭盧婆羅墮若比丘記說:『我了知:「出生已盡,梵行已完成,應該被作的已作,不再有此處[輪迴]的狀態。」』比丘們!而這三根有什麼為終極(目標)?滅盡為終極。什麼的滅盡為終極?生、老、死的。比丘們!『生、老、死的滅盡』,完全智被看見的賓頭盧婆羅墮若記說:『我了知:「出生已盡,梵行已完成,應該被作的已作,不再有此處[輪迴]的狀態。」』」



\sutta{50}{50}{阿巴那經}{https://agama.buddhason.org/SN/sn.php?keyword=48.50}
  \twnr{被我這麼聽聞}{1.0}:

  \twnr{有一次}{2.0},\twnr{世尊}{12.0}住在鴦伽,名叫阿巴那的鴦伽市鎮。

  在那裡,世尊召喚\twnr{尊者}{200.0}舍利弗:

  「舍利弗!凡那位在如來上到達\twnr{一向}{168.0}\twnr{極淨信的}{340.1}聖弟子,他在如來或如來的教說上是否會疑惑,或會懷疑呢?」

  「\twnr{大德}{45.0}!凡那位在如來上到達一向極淨信的聖弟子,他在如來或如來的教說上不會疑惑,或會懷疑。

  大德!對有信的聖弟子來說,能這麼被預期:為了不善法的捨斷、為了善法的具足,他必將住於活力已被發動的、強力的、堅固努力的、在諸善法上不放下負擔的。大德!凡他的活力,那是他的\twnr{活力根}{291.0}。

  大德!對有信的、活力已被發動的聖弟子來說,能這麼被預期:他必將是有念的,具備最高的\twnr{念與聰敏}{590.0},是很久以前做過的及很久以前說過的記得者、回憶者。大德!凡他的念,那是他的念根。

  大德!對有信的、活力已被發動的、\twnr{念已現起}{341.0}的聖弟子來說,能這麼被預期:作\twnr{捨棄為所緣}{495.0}後,他必將得到定;他必將得到\twnr{心一境性}{255.0}。大德!凡他的定,那是他的定根。

  大德!對有信的、活力已被發動的、念已現起的、心已得定的聖弟子來說,能這麼被預期:他必將這麼知道:輪迴是無始的,\twnr{無明蓋}{158.0}、渴愛結眾生的流轉的、輪迴的\twnr{起始點}{639.0}不被知道,但就那個無明黑闇聚集的\twnr{無餘褪去與滅}{491.0},這是寂靜之境,這是勝妙之境,即:\twnr{一切行的止}{55.0}、一切\twnr{依著}{198.0}的\twnr{斷念}{211.0}、渴愛的滅盡、\twnr{離貪}{77.0}、\twnr{滅}{68.0}、涅槃。大德!凡他的慧,那是他的慧根。

  大德!那位有信的聖弟子這麼一再努力後;這麼一再憶念後;這麼一再得定後;這麼一再知道後,他這麼相信:『這些、那些法凡在以前被我聽聞者,而我現在以那個身觸後住,且以慧\twnr{貫通後看見}{219.0}。』大德!凡他的信,那是他的信根。」

  「\twnr{好}{44.0}!好!舍利弗!舍利弗!凡那位在如來上到達一向極淨信的聖弟子,他在如來或如來的教說上不會疑惑,或會懷疑。

  舍利弗!有信的聖弟子的這個能被預期:為了不善法的捨斷、為了善法的具足,他必將住於活力已被發動的、強力的、堅固努力的、在諸善法上不放下負擔的。舍利弗!凡他的活力,那是他的活力根。

  舍利弗!有信的、活力已被發動的聖弟子的這個能被預期:他必將是有念的,具備最高的念與聰敏,是很久以前做過的及很久以前說過的記得者、回憶者。舍利弗!凡他的念,那是他的念根。

  舍利弗!有信的、活力已被發動的、念已現起的聖弟子的這個能被預期:作捨棄為所緣後,他必將得到定;他必將得到心一境性。舍利弗!凡他的定,那是他的定根。

  舍利弗!有信的、活力已被發動的、念已現起的、心已得定的聖弟子的這個能被預期:他必將這麼知道:輪迴是無始的,無明蓋、渴愛結眾生的流轉的、輪迴的起始點不被知道,但就那個無明黑闇聚集的無餘褪去與滅,這是寂靜之境,這是勝妙之境,即:一切行的止、一切依著的斷念、渴愛的滅盡、離貪、滅、涅槃。舍利弗!凡他的慧,那是他的慧根。

  舍利弗!那位有信的聖弟子這麼一再努力後;這麼一再憶念後;這麼一再得定後;這麼一再知道後,他這麼相信:『這些、那些法凡在以前被我聽聞者,而我現在以那個身觸後住,且以慧貫通後看見。』舍利弗!凡他的信,那是他的信根。」

  老品第五,其\twnr{攝頌}{35.0}:

  「老、巫男巴婆羅門,娑雞多、東門屋,

   東園四則,以及賓頭盧與阿巴那。」





\pin{野豬洞品}{51}{60}
\sutta{51}{51}{薩羅經}{https://agama.buddhason.org/SN/sn.php?keyword=48.51}
  \twnr{被我這麼聽聞}{1.0}:

  \twnr{有一次}{2.0},\twnr{世尊}{12.0}住在憍薩羅國的薩羅\twnr{婆羅門}{17.0}村落。

  在那裡,世尊召喚\twnr{比丘}{31.0}們:

  「比丘們!猶如凡任何畜生類,獸王獅子被說為牠們中第一的,即:以力量、以速度、以勇氣。同樣的,比丘們!凡任何\twnr{覺分法}{567.0},慧根被告知為它們中第一的,即:以覺。

  比丘們!而哪些是覺分法呢?比丘們!信根是覺分法,它轉起(導向)覺;\twnr{活力根}{291.0}是覺分法,它轉起覺;念根是覺分法,它轉起覺;定根是覺分法,它轉起覺;慧根是覺分法,它轉起覺。

  比丘們!猶如凡任何畜生類,獸王獅子被說為牠們中第一的,即:以力量、以速度、以勇氣。同樣的,比丘們!凡任何覺分法,慧根被告知為它們中第一的,即:以覺。」



\sutta{52}{52}{末羅經}{https://agama.buddhason.org/SN/sn.php?keyword=48.52}
  \twnr{被我這麼聽聞}{1.0}:

  \twnr{有一次}{2.0},\twnr{世尊}{12.0}住\twnr{在末羅}{x637},名叫屋盧吠羅迦巴的末羅市鎮。

  在那裡,世尊召喚\twnr{比丘}{31.0}們:

  「比丘們!只要\twnr{聖弟子}{24.0}的聖智未被生起,就還未有[其它]四根的定立;就還未有四根的確立,比丘們!但當聖弟子的聖智已被生起,那時有四根的定立;那時有四根的確立。

  比丘們!猶如只要\twnr{重閣}{213.0}的尖頂未被立起,就還未有\twnr{椽}{663.0}的定立;就還未有椽的確立,比丘們!但當重閣的尖頂已被立起,那時有椽的定立;那時有椽的確立。同樣的,比丘們!只要聖弟子的聖智未被生起,就還未有四根的定立;就還未有四根的確立,比丘們!但當聖弟子的聖智已被生起,那時有四根的定立;那時有四根的確立。

  哪四個的呢?信根的、\twnr{活力根}{291.0}的、念根的、定根的。

  比丘們!對有慧的聖弟子來說,隨行那個的信確立;隨行那個的活力確立;隨行那個的念確立;隨行那個的定確立。」



\sutta{53}{53}{有學經}{https://agama.buddhason.org/SN/sn.php?keyword=48.53}
  \twnr{被我這麼聽聞}{1.0}:

  \twnr{有一次}{2.0},\twnr{世尊}{12.0}住在\twnr{憍賞彌}{140.0}瞿師羅園。

  在那裡,世尊召喚\twnr{比丘}{31.0}們:

  「比丘們!有\twnr{法門}{562.0},由於該法門,住立於有學階位的有學比丘會知道:『我是\twnr{有學}{193.0}。』住立於無學階位的無學比丘會知道:『我是\twnr{無學}{193.1}。』嗎?」

  「\twnr{大德}{45.0}!我們的法以世尊為根本……(中略)。」 

  「比丘們!有法門,由於該法門,住立於有學階位的有學比丘會知道:『我是有學。』住立於無學階位的無學比丘會知道:『我是無學。』

  比丘們!而什麼法門,由於該法門,住立於有學階位的有學比丘知道:『我是有學。』?比丘們!這裡,有學比丘如實知道:『這是苦。』如實知道:『這是苦集。』如實知道:『這是苦滅。』如實知道:『這是導向苦\twnr{滅道跡}{69.0}。』比丘們!這是法門,由於該法門,住立於有學階位的有學比丘知道:『我是有學。』

   再者,有學比丘像這樣深慮:『在這裡之外有其他\twnr{沙門}{29.0}或\twnr{婆羅門}{17.0}教導這樣真實的、真正的、如實的法,如世尊嗎?』他這麼知道:『在這裡之外沒有其他沙門或婆羅門教導這樣真實的、真正的、如實的法,如世尊。』比丘們!這也是法門,由於該法門,住立於有學階位的有學比丘知道:『我是有學。』

   再者,有學比丘知道五根:信根、\twnr{活力根}{291.0}、念根、定根、慧根,其趣向的、其最高的、其果、其完結,他仍未以身觸達後住,但以慧\twnr{貫通後看見}{219.0},比丘們!這也是法門,由於該法門,住立於有學階位的有學比丘知道:『我是有學。』

  比丘們!什麼法門,由於該法門,住立於無學階位的無學比丘知道:『我是無學。』?比丘們!這裡,無學比丘知道五根:信根、活力根、念根、定根、慧根,其趣向的、其最高的、其果、其完結,他以身觸達後住,以及以慧貫通後看見,這也是法門,由於該法門,住立於無學階位的無學比丘知道:『我是無學。』

  再者,比丘們!這裡,無學比丘知道六根:眼根、耳根、鼻根、舌根、身根、意根,他知道:『這些六根將全部完全地、每一方面完全地、無殘餘地被滅,無論在哪裡,在不論哪種上將無其它六根生起。』這也是法門,由於該法門,住立於無學階位的無學比丘知道:『我是無學。』」



\sutta{54}{54}{足跡經}{https://agama.buddhason.org/SN/sn.php?keyword=48.54}
  「\twnr{比丘}{31.0}們!猶如凡任何叢林生物的足跡類者,那些全都在象的足跡中走到容納,象的足跡被告知為它們中第一的,即:以大的狀態。同樣的,比丘們!凡任何轉起覺的足跡者,慧根被告知為它們中第一的,即:以覺。

  比丘們!而哪些轉起覺的足跡呢?比丘們!信根是足跡,它轉起覺;\twnr{活力根}{291.0}是足跡,它轉起覺;念根是足跡,它轉起覺;定根是足跡,它轉起覺;慧根是足跡,它轉起覺。

  比丘們!猶如凡任何叢林生物的足跡類者,那些全都在象的足跡中走到容納,象的足跡被告知為它們中第一的,即:以大的狀態。同樣的,比丘們!凡任何轉起覺的足跡者,慧根被告知為它們中第一的,即:以覺。」



\sutta{55}{55}{樹心經}{https://agama.buddhason.org/SN/sn.php?keyword=48.55}
  「\twnr{比丘}{31.0}們!猶如凡任何香樹心,紫檀被告知為它們中第一的。同樣的,比丘們!凡任何\twnr{覺分法}{567.0},慧根被告知為它們中第一的,即:以覺。

  比丘們!而哪些是覺分法呢?比丘們!信根是覺分法,它轉起覺;\twnr{活力根}{291.0}……(中略)念根……(中略)定根……(中略)慧根是覺分法,它轉起覺。

  比丘們!猶如凡任何香樹心,紫檀被告知為它們中第一的。同樣的,比丘們!凡任何覺分法,慧根被告知為它們中第一的,即:以覺。」



\sutta{56}{56}{已住立經}{https://agama.buddhason.org/SN/sn.php?keyword=48.56}
  「\twnr{比丘}{31.0}們!在一法上住立的比丘有被\twnr{修習}{94.0}、被善修習的五根,在哪一法上呢?在不放逸上。比丘們!而怎樣是不放逸?比丘們!這裡,比丘在漏上與在有漏法上守護心,當在漏上與在有漏法上守護他的心時,信根走到修習圓滿;活力根也走到修習圓滿;念根也走到修習圓滿;定根也走到修習圓滿;慧根也走到修習圓滿,比丘們!這樣,在一法上住立的比丘有被修習、被善修習的五根。」



\sutta{57}{57}{梵王娑婆主經}{https://agama.buddhason.org/SN/sn.php?keyword=48.57}
  \twnr{有一次}{2.0},初\twnr{現正覺}{75.0}的\twnr{世尊}{12.0}住在優樓頻螺,尼連禪河邊牧羊人的榕樹處。

  那時,當世尊獨處、\twnr{獨坐}{92.0}時,這樣心的深思生起:

  「五根已\twnr{修習}{94.0}、已\twnr{多作}{95.0},有\twnr{不死}{123.0}的立足處、不死的\twnr{彼岸}{226.0}、不死的完結,哪五個?信根已修習、已多作,有不死的立足處、不死的彼岸、不死的完結;活力根……(中略)念根……(中略)定根……(中略)慧根已修習、已多作,有不死的立足處、不死的彼岸、不死的完結,當這些五根修習、多作時,有不死的立足處、不死的彼岸、不死的完結。」

  那時,\twnr{梵王娑婆主}{215.0}以心了知世尊心中的深思後,就猶如有力氣的男子伸直彎曲的手臂,或彎曲伸直的手臂,就像這樣在梵天世界消失,出現在世尊的面前。

  那時,梵王娑婆主置(作)上衣到一邊肩膀,向世尊\twnr{合掌}{377.0}鞠躬後,對世尊說這個:

  「這是這樣,世尊!這是這樣,\twnr{善逝}{8.0}!五根已修習、已多作,有不死的立足處、不死的彼岸、不死的完結,哪五個?信根已修習、已多作,有不死的立足處、不死的彼岸、不死的完結……(中略)慧根已修習、已多作,有不死的立足處、不死的彼岸、不死的完結,當這些五根已修習、已多作,有不死的立足處、不死的彼岸、不死的完結。

  大德!從前,我在迦葉\twnr{遍正覺者}{6.0}處行梵行,在那裡,他們都這麼知道我:『娑婆葛\twnr{比丘}{31.0}、娑婆葛比丘。』大德!那個我就以這些五根的\twnr{已自我修習}{658.0}、已自我多作,在欲上使\twnr{欲的意欲}{118.0}離染後,以身體的崩解,死後往生\twnr{善趣}{112.0}梵天世界,在那裡,他們都這麼知道我:『梵王娑婆主、梵王娑婆主。』這是這樣,世尊!這是這樣,善逝!我知道這個、我看見這個:依五些根已修習、已多作,有不死的立足處、不死的彼岸、不死的完結。」



\sutta{58}{58}{野豬洞經}{https://agama.buddhason.org/SN/sn.php?keyword=48.58}
  \twnr{有一次}{2.0},\twnr{世尊}{12.0}住在王舍城\twnr{耆闍崛山}{258.0}的野豬洞中。

  在那裡,世尊召喚\twnr{尊者}{200.0}舍利弗:

  「舍利弗!當看見什麼利益(因由)時,諸漏已滅盡的\twnr{比丘}{31.0}在如來或如來的教說上,當轉起(禮敬)時轉起最高的\twnr{身體伏在地上之禮}{40.0}呢?」
 

  「\twnr{大德}{45.0}!當看見無上\twnr{軛安穩}{192.0}時,漏盡比丘在如來或如來的教說上,當轉起時轉起最高的身體伏在地上之禮。」

  「\twnr{好}{44.0}!好!舍利弗!舍利弗!當看見無上軛安穩時,漏盡比丘在如來或如來的教說上,當轉起時轉起最高的身體伏在地上之禮。舍利弗!而什麼是無上軛安穩:當看見時,漏盡比丘在如來或如來的教說上,當轉起時轉起最高的身體伏在地上之禮呢?」
 

  「大德!這裡,漏盡比丘\twnr{修習}{94.0}信根,導向寂靜、導向正覺;修習\twnr{活力根}{291.0}……(中略)修習念根……(中略)修習定根……(中略)修習慧根,導向寂靜、導向正覺,大德!這是無上軛安穩:當看見時,漏盡比丘在如來或如來的教說上,當轉起時轉起最高的身體伏在地上之禮。」

  「好!好!舍利弗!舍利弗!這是無上軛安穩:當看見時,漏盡比丘在如來或如來的教說上,當轉起時轉起最高的身體伏在地上之禮。舍利弗!而什麼是最高的身體伏在地上之禮:漏盡比丘在如來或如來的教說上,當轉起時轉起最高的身體伏在地上之禮呢?」
 

  「大德!這裡,漏盡比丘在大師上住於尊敬的、順從的,在法上住於尊敬的、順從的,在\twnr{僧團}{375.0}住於上尊敬的、順從的,在學上住於尊敬的、順從的,在正定上住於尊敬的、順從的,大德!這是最高的身體伏在地上之禮:漏盡比丘在如來或如來的教說上,當轉起時轉起最高的身體伏在地上之禮。」

  「好!好!舍利弗!舍利弗!這是最高的身體伏在地上之禮:漏盡比丘在如來或如來的教說上,當轉起時轉起最高的身體伏在地上之禮。」



\sutta{59}{59}{生起經第一}{https://agama.buddhason.org/SN/sn.php?keyword=48.59}
  起源於舍衛城。

  「\twnr{比丘}{31.0}們!有這些五根,已\twnr{修習}{94.0}、已\twnr{多作}{95.0},未生起的不離\twnr{如來}{4.0}、\twnr{阿羅漢}{5.0}、\twnr{遍正覺者}{6.0}的出現而生起,哪五個?信根、\twnr{活力根}{291.0}、念根、定根、慧根。比丘們!這些是五根,已修習、已多作,未生起的不離如來、阿羅漢、遍正覺者的出現而生起。」



\sutta{60}{60}{生起經第二}{https://agama.buddhason.org/SN/sn.php?keyword=48.60}
  起源於舍衛城。

  「\twnr{比丘}{31.0}們!有這些五根,已\twnr{修習}{94.0}、已\twnr{多作}{95.0},未生起的不離\twnr{善逝}{8.0}之律生起,哪五個?信根、\twnr{活力根}{291.0}、念根、定根、慧根。比丘們!這些是五根,已修習、已多作,未生起的不離善逝之律生起。」

  野豬洞品第一,其\twnr{攝頌}{35.0}:

  「薩羅、末羅與有學,足跡、樹心、已住立,

   梵天、野豬洞,生起二則在後。」





\pin{覺分品}{61}{70}
\sutta{61}{61}{結經}{https://agama.buddhason.org/SN/sn.php?keyword=48.61}
  起源於舍衛城。

  「\twnr{比丘}{31.0}們!有這些五根,已\twnr{修習}{94.0}、已\twnr{多作}{95.0},轉起結的捨斷,哪五個?信根……(中略)慧根。比丘們!這些是五根,已修習、已多作,轉起結的捨斷。」



\sutta{62}{62}{煩惱潛在趨勢經}{https://agama.buddhason.org/SN/sn.php?keyword=48.62}
  「\twnr{比丘}{31.0}們!有這些五根,已\twnr{修習}{94.0}、已\twnr{多作}{95.0},轉起\twnr{煩惱潛在趨勢}{253.1}的根除,哪五個?信根……(中略)慧根。比丘們!這些是五根,已修習、已多作,轉起煩惱潛在趨勢的根除。」



\sutta{63}{63}{遍知經}{https://agama.buddhason.org/SN/sn.php?keyword=48.63}
  「\twnr{比丘}{31.0}們!有這些五根,已\twnr{修習}{94.0}、已\twnr{多作}{95.0},轉起\twnr{[生命]旅途的遍知}{889.0},哪五個?信根……(中略)慧根。比丘們!這些是五根,已修習、已多作,轉起[生命]旅途的遍知。」



\sutta{64}{64}{漏的滅盡經}{https://agama.buddhason.org/SN/sn.php?keyword=48.64}
  「\twnr{比丘}{31.0}們!有這些五根,已\twnr{修習}{94.0}、已\twnr{多作}{95.0},轉起諸\twnr{漏}{188.0}的滅盡,哪五個?信根……(中略)慧根。比丘們!這些是五根,已修習、已多作,轉起諸漏的滅盡。

  比丘們!這些五根,已修習、已多作,轉起結的捨斷;轉起\twnr{煩惱潛在趨勢}{253.1}的根除;\twnr{[生命]旅途的遍知}{889.0};轉起諸漏的滅盡,哪五個?信根……(中略)慧根。比丘們!這些五根,已修習、已多作,轉起諸漏的滅盡。比丘們!這些五根,已修習、已多作,轉起結的捨斷;轉起煩惱潛在趨勢的根除;[生命]旅途的遍知;轉起諸漏的滅盡。」



\sutta{65}{65}{果經第一}{https://agama.buddhason.org/SN/sn.php?keyword=48.65}
  「\twnr{比丘}{31.0}們!有這些五根,哪五根呢?信根……(中略)慧根,比丘們!這些是五根。

  比丘們!以這些五根的\twnr{已自我修習}{658.0}、已自我\twnr{多作}{95.0},二果其中之一果能被預期:\twnr{當生}{42.0}\twnr{完全智}{489.0},或在存在\twnr{有餘依}{323.0}時,為\twnr{阿那含}{209.0}位。」



\sutta{66}{66}{果經第二}{https://agama.buddhason.org/SN/sn.php?keyword=48.66}
  「\twnr{比丘}{31.0}們!有這些五根,哪五根呢?信根……(中略)慧根,比丘們!這些是五根。

  比丘們!以這些五根的\twnr{已自我修習}{658.0}、已自我\twnr{多作}{95.0},七果、七效益能被預期,哪七果、七效益呢?

  在當生提前達成\twnr{完全智}{489.0}。

  如果在當生未提前達成完全智,那麼在死時達成完全智。

  如果在當生未到達完全智,如果在死時未達成完全智,則以\twnr{五下分結}{134.0}的滅盡,成為\twnr{中般涅槃}{297.0}者、為\twnr{生般涅槃}{298.0}者、為\twnr{無行般涅槃}{299.0}者、為\twnr{有行般涅槃}{300.0}者、為\twnr{上流到阿迦膩吒}{301.0}者。

  比丘們!以這些五根的已自我修習、已自我多作,這些七果、七效益能被預期。」



\sutta{67}{67}{樹經第一}{https://agama.buddhason.org/SN/sn.php?keyword=48.67}
  「\twnr{比丘}{31.0}們!猶如凡任何閻浮提的樹,閻浮樹被告知為它們中第一的。同樣的,比丘們!凡任何\twnr{覺分法}{567.0},慧根被告知為它們中第一的,即:以覺。

  比丘們!而哪些是覺分法呢?比丘們!信根是覺分法,它轉起覺;\twnr{活力根}{291.0}……(中略)念根……(中略)定根……(中略)慧根是覺分法,它轉起覺。

  比丘們!猶如凡任何閻浮提樹,閻浮樹被告知為它們中第一的。同樣的,比丘們!凡任何覺分法,慧根被告知為它們中第一的,即:以覺。」



\sutta{68}{68}{樹經第二}{https://agama.buddhason.org/SN/sn.php?keyword=48.68}
  「\twnr{比丘}{31.0}們!猶如凡任何三十三天的樹,晝度樹被告知為它們中第一的。同樣的,比丘們!凡任何\twnr{覺分法}{567.0},慧根被告知為它們中第一的,即:以覺。

  比丘們!而哪些是覺分法呢?比丘們!信根是覺分法,它轉起覺;\twnr{活力根}{291.0}……(中略)念根……(中略)定根……(中略)慧根是覺分法,它轉起覺。

  比丘們!猶如凡任何三十三天的樹,晝度樹被告知為它們中第一的。同樣的,比丘們!凡任何覺分法,慧根被告知為它們中第一的,即:以覺。」



\sutta{69}{69}{樹經第三}{https://agama.buddhason.org/SN/sn.php?keyword=48.69}
  「\twnr{比丘}{31.0}們!猶如凡任何阿修羅的樹,質多玻達哩樹被告知為它們中第一的。同樣的,比丘們!凡任何\twnr{覺分法}{567.0},慧根被告知為它們中第一的,即:以覺。

  比丘們!而哪些是覺分法呢?比丘們!信根是覺分法,它轉起覺;\twnr{活力根}{291.0}……(中略)慧根是覺分法,它轉起覺。

  比丘們!猶如凡任何阿修羅的樹,質多玻達哩樹被告知為它們中第一的。同樣的,比丘們!凡任何覺分法,慧根被告知為它們中第一的,即:以覺。」



\sutta{70}{70}{樹經第四}{https://agama.buddhason.org/SN/sn.php?keyword=48.70}
  「\twnr{比丘}{31.0}們!猶如凡任何金翅鳥的樹,尖頂絹綿樹被告知為它們中第一的。同樣的,比丘們!凡任何\twnr{覺分法}{567.0},慧根被告知為它們中第一的,即:以覺。

  比丘們!而哪些是覺分法呢?比丘們!信根是覺分法,它轉起覺;\twnr{活力根}{291.0}……(中略)慧根是覺分法,它轉起覺。

  比丘們!猶如凡任何金翅鳥的樹,尖頂絹綿樹被告知為它們中第一的。同樣的,比丘們!凡任何覺分法,慧根被告知為它們中第一的,即:以覺。」

  覺分品第七,其\twnr{攝頌}{35.0}:

  「結、煩惱潛在趨勢,遍知、\twnr{漏}{188.0}的滅盡,

   二則果、四則樹,以那個被稱為品。」





\pin{恒河中略品}{71}{114}
\sutta{71}{82}{向東低斜等經十二則}{https://agama.buddhason.org/SN/sn.php?keyword=48.71}
  「\twnr{比丘}{31.0}們!猶如恒河是傾向東的、斜向東的、坡斜向東的。同樣的,比丘們!\twnr{修習}{94.0}五根、\twnr{多作}{95.0}五根的比丘是傾向涅槃的、斜向涅槃的、坡斜向涅槃的。

  比丘們!而怎樣修習五根、多作五根的比丘是傾向涅槃的、斜向涅槃的、坡斜向涅槃的?比丘們!這裡,比丘\twnr{依止遠離}{322.0}、依止離貪、依止滅、\twnr{捨棄的成熟}{221.0}修習信根……(中略)\twnr{活力根}{291.0}……念根……定根;依止遠離、依止離貪、依止滅、捨棄的成熟修習慧根。

  比丘們!這樣修習五根、多作五根的比丘是傾向涅槃的、斜向涅槃的、坡斜向涅槃的。」[按:全品應如\suttaref{SN.45.139}-148那樣]

  恒河中略品第八,其\twnr{攝頌}{35.0}:

  「六則傾向東的,與六則傾向大海的,

   這兩個六則成十二則,以那個被稱為品。」

  不放逸品應該使之被細說[按:如\suttaref{SN.45.139}-148那樣], 其攝頌:

  「如來、足跡、屋頂,根、樹心、茉莉花, 

   王、月、日,以衣服為第十句。」 

  力量所作品應該使之被細說[按:如\suttaref{SN.45.149}-160那樣],其攝頌: 

  「力量、種子與龍,樹木、瓶子及穗, 

   虛空與二則雨雲,船、屋舍、河。」

  尋求品應該使之被細說[按:如\suttaref{SN.45.161}-171那樣],其攝頌: 

  「尋求、慢、漏,有與三苦性, 

   荒蕪、垢、惱亂,受、渴愛、渴望。」



\sutta{83}{114}{}{https://agama.buddhason.org/SN/sn.php?keyword=48.83}
  (略去)  





\pin{暴流品}{115}{124}
\sutta{115}{124}{暴流經十則}{https://agama.buddhason.org/SN/sn.php?keyword=48.115}
  「\twnr{比丘}{31.0}們!有這些五上分結,哪五個?色貪、無色貪、慢、掉舉、\twnr{無明}{207.0},比丘們!這些是五上分結。比丘們!為了這些五上分結的證智、\twnr{遍知}{154.0}、遍盡、捨斷,這些五根應該被\twnr{修習}{94.0}。比丘們!哪五個?比丘們!這裡,比丘\twnr{依止遠離}{322.0}、依止離貪、依止滅、\twnr{捨棄的成熟}{221.0}修習信根……(中略)比丘依止遠離、依止離貪、依止滅、捨棄的成熟修習慧根。比丘們!為了這些五上分結的證智、遍知、遍盡、捨斷,這些五根應該被修習。」(應該如\twnr{道相應}{x628}那樣使之被細說)

  暴流品第十二,其\twnr{攝頌}{35.0}:

  「暴流、軛、取,繫縛、煩惱潛在趨勢,

   欲種類、蓋,蘊、下上分。」





\pin{恒河中略品}{125}{168}
\sutta{125}{136}{向東低斜等經十二則}{https://agama.buddhason.org/SN/sn.php?keyword=48.125}
  「\twnr{比丘}{31.0}們!猶如恒河是傾向東的、斜向東的、坡斜向東的。同樣的,比丘們!\twnr{修習}{94.0}五根、\twnr{多作}{95.0}五根的比丘是傾向涅槃的、斜向涅槃的、坡斜向涅槃的。

  比丘們!而怎樣修習五根、多作五根的比丘是傾向涅槃的、斜向涅槃的、坡斜向涅槃的?比丘們!這裡,比丘修習信根,有貪之調伏的完結、有瞋之調伏的完結、有癡之調伏的完結……(中略)修習慧根,有貪之調伏的完結、有瞋之調伏的完結、有癡之調伏的完結。

  比丘們!這樣修習五根、多作五根的比丘是傾向涅槃的、斜向涅槃的、坡斜向涅槃的。」[按:全品應如\suttaref{SN.45.91}-102那樣]

  恒河中略品第十三,其\twnr{攝頌}{35.0}:

  「六則傾向東的,與六則傾向大海的,

   這兩個六則成十二則,以那個被稱為品。」

  不放逸品[按:全品應如SN.139-148那樣]、力量所作品[按:全品應如\suttaref{SN.45.149}-160那樣]、尋求品[按:全品應如\suttaref{SN.45.161}-171那樣],應該使之被細說。



\sutta{137}{168}{}{https://agama.buddhason.org/SN/sn.php?keyword=48.137}
  (略去)  





\pin{暴流品}{169}{178}
\sutta{169}{178}{暴流經十則}{https://agama.buddhason.org/SN/sn.php?keyword=48.169}
  「\twnr{比丘}{31.0}們!有這些五上分結,哪五個?色貪、無色貪、慢、掉舉、\twnr{無明}{207.0},比丘們!這些是五上分結。比丘們!為了這些五上分結的證智、\twnr{遍知}{154.0}、遍盡、捨斷,這些五根應該被\twnr{修習}{94.0}。比丘們!哪五個?比丘們!這裡,比丘修習信根,有貪之調伏的完結、有瞋之調伏的完結、有癡之調伏的完結;\twnr{活力根}{291.0}……(中略)念根……定根……修習慧根,有貪之調伏的完結、有瞋之調伏的完結、有癡之調伏的完結。比丘們!為了這些五上分結的證智、遍知、遍盡、捨斷,這些五根應該被修習。」[按:全品應如\suttaref{SN.45.172}-181那樣] 

  暴流品第十七,其\twnr{攝頌}{35.0}:

  「暴流、軛、取,繫縛、煩惱潛在趨勢,

   欲種類、蓋,蘊、下上分。」

  根相應第四。





\page

\xiangying{49}{正勤相應}
\pin{恒河中略品}{1}{12}
\sutta{1}{12}{東的等經十二則}{https://agama.buddhason.org/SN/sn.php?keyword=49.1}
  起源於舍衛城。

  在那裡,\twnr{世尊}{12.0}說這個:

  「\twnr{比丘}{31.0}們!有這些\twnr{四正勤}{292.0},哪四個?比丘們!這裡,比丘為了未生起的惡不善法之不生起使意欲生起、努力、發動活力、盡心、勤奮;為了已生起的惡不善法之捨斷使意欲生起、努力、發動活力、盡心、勤奮;為了未生起的善法之生起使意欲生起、努力、發動活力、盡心、勤奮;為了已生起的諸善法之存續、不忘失、增大、成滿、修習圓滿使意欲生起、努力、發動活力、盡心、勤奮,比丘們!這些是四正勤。

  比丘們!猶如恒河是傾向東的、斜向東的、坡斜向東的。同樣的,比丘們!修習四正勤、多作四正勤的比丘是傾向涅槃的、斜向涅槃的、坡斜向涅槃的。

  比丘們!而怎樣\twnr{修習}{94.0}四正勤、\twnr{多作}{95.0}四正勤的比丘是傾向涅槃的、斜向涅槃的、坡斜向涅槃的?比丘們!這裡,比丘為了未生起的惡不善法之不生起使意欲生起、努力、發動活力、盡心、勤奮;為了已生起的惡不善法之捨斷使意欲生起、努力、發動活力、盡心、勤奮;為了未生起的善法之生起使意欲生起、努力、發動活力、盡心、勤奮;為了已生起的諸善法之存續、不忘失、增大、成滿、修習圓滿使意欲生起、努力、發動活力、盡心、勤奮。比丘們!這樣修習四正勤、多作四正勤的比丘是傾向涅槃的、斜向涅槃的、坡斜向涅槃的。」

  (正勤相應的恒河中略[品],應該以正勤使之被細說[按:如\suttaref{SN.45.91}-102那樣,但以正勤取代])

  恒河中略品第一,其\twnr{攝頌}{35.0}:

  「六則傾向東的,與六則傾向大海的,

   這兩個六則成十二則,以那個被稱為品。」





\pin{不放逸品}{13}{22}
\sutta{13}{22}{}{https://agama.buddhason.org/SN/sn.php?keyword=49.13}
  (不放逸品應該以正勤[取代]那樣使之被細說)

  [不放逸第二,]其攝頌:

  「如來、足跡、屋頂,根、樹心、茉莉花,

   王、月、日,以衣服為第十句。」





\pin{力量所作品}{23}{34}
\sutta{23}{34}{力量所作等經十二則}{https://agama.buddhason.org/SN/sn.php?keyword=49.23}
  「\twnr{比丘}{31.0}們!猶如凡任何應該被力量作的工作被作,它們全部依止土地後,住立於土地後,這樣,這些應該被力量作的工作被作。同樣的,比丘們!比丘依止戒後,住立於戒後,\twnr{修習}{94.0}\twnr{四正勤}{292.0}、\twnr{多作}{95.0}四正勤。

  比丘們!而怎樣比丘依止戒後,住立於戒後,修習四正勤、多作四正勤?比丘們!這裡,比丘為了未生起的惡不善法之不生起使意欲生起、努力、發動活力、盡心、勤奮……(中略)為了已生起的諸善法之存續、不忘失、增大、成滿、修習圓滿使意欲生起、努力、發動活力、盡心、勤奮。比丘們!這樣,比丘依止戒後,住立於戒後,修習四正勤、多作四正勤。」(力量所作品應該以正勤[取代]那樣使之被細說)[按:全品應如\suttaref{SN.45.149}-160那樣]

  力量所作品第三,其\twnr{攝頌}{35.0}:

  「力量、種子與龍,樹木、瓶子及穗,

   虛空與二則雨雲,船、屋舍、河。」





\pin{尋求品}{35}{44}
\sutta{35}{44}{尋求等經十則}{https://agama.buddhason.org/SN/sn.php?keyword=49.35}
  「\twnr{比丘}{31.0}們!有這些三種尋求,哪三個?欲的尋求、有的尋求、\twnr{梵行的尋求}{381.1},比丘們!這些是三種尋求。比丘們!為了這些三種尋求的證智、\twnr{遍知}{154.0}、遍盡、捨斷,\twnr{四正勤}{292.0}應該被\twnr{修習}{94.0}。比丘們!哪四個?比丘們!這裡,比丘為了未生起的……(中略)為了已生起的諸善法之存續、不忘失、增大、成滿、修習圓滿使意欲生起、努力、發動活力、盡心、勤奮。比丘們!為了這些三種尋求的證智、遍知、遍盡、捨斷,這些四正勤應該被修習。」(應該使之被細說[按:如\suttaref{SN.45.161}-171])

   尋求品第四,其\twnr{攝頌}{35.0}:

  「尋求、慢、漏,有與三苦性,

   荒蕪、垢、惱亂,受、渴愛、渴望。」





\pin{暴流品}{45}{54}
\sutta{45}{54}{暴流等經十則}{https://agama.buddhason.org/SN/sn.php?keyword=49.45}
  「\twnr{比丘}{31.0}們!有這些五上分結,哪五個?色貪、無色貪、慢、掉舉、\twnr{無明}{207.0},比丘們!這些是五上分結。比丘們!為了這些五上分結的證智、\twnr{遍知}{154.0}、遍盡、捨斷,這\twnr{四正勤}{292.0}應該被\twnr{修習}{94.0}。比丘們!哪四個?比丘們!這裡,比丘為了未生起的……(中略)為了已生起的諸善法之存續、不忘失、增大、成滿、修習圓滿使意欲生起、努力、發動活力、盡心、勤奮。比丘們!為了這些五上分結的證智、遍知、遍盡、捨斷,這些四正勤應該被修習。」(應該使之[如\suttaref{SN.45.172}-181那樣]詳細)

  暴流品第五,其\twnr{攝頌}{35.0}:

  「暴流、軛、取,繫縛、煩惱潛在趨勢,

   欲種類、蓋,蘊、下上分。」

  正勤相應第五。





\page

\xiangying{50}{力相應}
\pin{恒河中略品}{1}{12}
\sutta{1}{12}{力經十二則}{https://agama.buddhason.org/SN/sn.php?keyword=50.1}
  「\twnr{比丘}{31.0}們!有這些五力,哪五力呢?信力、\twnr{活力之力}{306.0}、念力、定力、慧力,比丘們!這是五力。

  比丘們!猶如恒河是傾向東的、斜向東的、坡斜向東的。同樣的,比丘們!\twnr{修習}{94.0}五力、\twnr{多作}{95.0}五力的比丘是傾向涅槃的、斜向涅槃的、坡斜向涅槃的。

  比丘們!而怎樣修習五力、多作五力的比丘是傾向涅槃的、斜向涅槃的、坡斜向涅槃的?比丘們!這裡,比丘\twnr{依止遠離}{322.0}、依止離貪、依止滅、\twnr{捨棄的成熟}{221.0}修習信力……(中略)活力之力……念力……定力;依止遠離、依止離貪、依止滅、捨棄的成熟修習慧力。

  比丘們!這樣修習五力、多作五力的比丘是傾向涅槃的、斜向涅槃的、坡斜向涅槃的。」[按:全品應如\suttaref{SN.45.91}-102那樣]

  恒河中略品第一,其\twnr{攝頌}{35.0}:

  「六則傾向東的,與六則傾向大海的,

   這兩個六則成十二則,以那個被稱為品。」





\pin{不放逸品}{13}{44}
\sutta{13}{24}{}{https://agama.buddhason.org/SN/sn.php?keyword=50.13}
  不放逸品應該使之[如\suttaref{SN.45.149}-160那樣]詳細, 其\twnr{攝頌}{35.0}:

  「如來、足跡、屋頂,根、樹心、茉莉花, 

   王、月、日,以衣服為第十句。」 

  力量所作品應該使之被細說,其攝頌: 

  「力量、種子與龍,樹木、瓶子及穗, 

   虛空與二則雨雲,船、屋舍、河。」

  尋求品應該使之被細說[按:如\suttaref{SN.45.161}-171],其攝頌: 

  「尋求、慢、漏,有與三苦性, 

   荒蕪、垢、惱亂,受、渴愛、渴望。」



\sutta{25}{44}{}{https://agama.buddhason.org/SN/sn.php?keyword=50.25}
  (略去)  





\pin{暴流品}{45}{54}
\sutta{45}{54}{暴流等經十則}{https://agama.buddhason.org/SN/sn.php?keyword=50.45}
  「\twnr{比丘}{31.0}們!有這些五上分結,哪五個?色貪、無色貪、慢、掉舉、\twnr{無明}{207.0},比丘們!這些是五上分結。比丘們!為了這些五上分結的證智、\twnr{遍知}{154.0}、遍盡、捨斷,這些五力應該被\twnr{修習}{94.0}。比丘們!哪五個?比丘們!這裡,比丘\twnr{依止遠離}{322.0}、依止離貪、依止滅、\twnr{捨棄的成熟}{221.0}修習信力……(中略)\twnr{活力之力}{306.0}……念力……定力;依止遠離、依止離貪、依止滅、捨棄的成熟修習慧力。比丘們!為了這些五上分結的證智、遍知、遍盡、捨斷,這些五力應該被修習。」(應該這樣使之[如\suttaref{SN.45.172}-181那樣]詳細)

  暴流品第五,其\twnr{攝頌}{35.0}:

  「暴流、軛、取,繫縛、煩惱潛在趨勢,

   欲種類、蓋,蘊、下上分。」





\pin{恒河中略品}{55}{87}
\sutta{55}{66}{向東低斜等經十二則}{https://agama.buddhason.org/SN/sn.php?keyword=50.55}
  「\twnr{比丘}{31.0}們!猶如恒河是傾向東的、斜向東的、坡斜向東的。同樣的,比丘們!\twnr{修習}{94.0}五力、\twnr{多作}{95.0}五力的比丘是傾向涅槃的、斜向涅槃的、坡斜向涅槃的。

  比丘們!而怎樣修習五力、多作五力的比丘是傾向涅槃的、斜向涅槃的、坡斜向涅槃的?比丘們!這裡,比丘修習信力,有貪之調伏的完結、有瞋之調伏的完結、有癡之調伏的完結……(中略)。比丘們!這樣修習五力、多作五力的比丘是傾向涅槃的、斜向涅槃的、坡斜向涅槃的。」[按:全品應如\suttaref{SN.45.91}-102那樣] 

  恒河中略品第六,其\twnr{攝頌}{35.0}:

  「六則傾向東的,與六則傾向大海的,

   這兩個六則成十二則,以那個被稱為品。」

  不放逸品[按:全品應如SN.139-148那樣]、力量所作品[按:全品應如\suttaref{SN.45.149}-160那樣],應該使之被細說。



\sutta{67}{87}{}{https://agama.buddhason.org/SN/sn.php?keyword=50.67}
  (略去)  





\pin{尋求品}{88}{98}
\sutta{88}{98}{尋求等經十則}{https://agama.buddhason.org/SN/sn.php?keyword=50.88}
  尋求經應該這樣使之被細說[按:如\suttaref{SN.45.161}-171]:有:貪之調伏的完結、瞋之調伏的完結、癡之調伏的完結。

   尋求品第九,其\twnr{攝頌}{35.0}:

  「尋求、慢、漏,有與三苦性,

   荒蕪、垢、惱亂,受、渴愛、渴望。」





\pin{暴流品}{99}{108}
\sutta{99}{108}{暴流等經十則}{https://agama.buddhason.org/SN/sn.php?keyword=50.99}
  「\twnr{比丘}{31.0}們!有這些五上分結,哪五個?色貪、無色貪、慢、掉舉、\twnr{無明}{207.0},比丘們!這些是五上分結。比丘們!為了這些五上分結的證智、\twnr{遍知}{154.0}、遍盡、捨斷,這些五力應該被\twnr{修習}{94.0}。比丘們!哪五個?比丘們!這裡,比丘修習信力……(中略)修習慧力,有貪之調伏的完結、有瞋之調伏的完結、有癡之調伏的完結。比丘們!為了這些五上分結的證智、遍知、遍盡、捨斷,這些五力應該被修習。」[按:全品應如\suttaref{SN.45.172}-181那樣]

  暴流品第十,其\twnr{攝頌}{35.0}:

  「暴流、軛、取,繫縛、煩惱潛在趨勢,

   欲種類、蓋,蘊、下上分。」

  力相應第六。





\page

\xiangying{51}{神足相應}
\pin{價玻勒品}{1}{10}
\sutta{1}{1}{此岸經}{https://agama.buddhason.org/SN/sn.php?keyword=51.1}
  「\twnr{比丘}{31.0}們!有這些\twnr{四神足}{503.1},已\twnr{修習}{94.0}、已\twnr{多作}{95.0},轉起從此岸到\twnr{彼岸}{226.0},哪四個?比丘們!這裡,比丘修習\twnr{具備意欲定勤奮之行的神足}{568.0};修習具備活力定勤奮之行的神足;修習具備心定勤奮之行的神足;修習\twnr{具備考察定勤奮之行的神足}{569.0}。比丘們!這些是四神足,已修習、已多作,轉起從此岸到彼岸。」



\sutta{2}{2}{已錯失經}{https://agama.buddhason.org/SN/sn.php?keyword=51.2}
  「\twnr{比丘}{31.0}們!凡任何已錯失\twnr{四神足}{503.1}者,他們導向苦的完全滅盡的聖道已錯失;比丘們!凡任何已發動四神足者,他們導向苦的完全滅盡的聖道已發動。哪四個?比丘們!這裡,比丘\twnr{修習}{94.0}\twnr{具備意欲定勤奮之行的神足}{568.0};活力定……(中略)心定……(中略)修習\twnr{具備考察定勤奮之行的神足}{569.0}。比丘們!凡任何已錯失四神足者,他們導向苦的完全滅盡的聖道已錯失;比丘們!凡任何已發動四神足者,他們導向苦的完全滅盡的聖道已發動。」



\sutta{3}{3}{聖經}{https://agama.buddhason.org/SN/sn.php?keyword=51.3}
  「\twnr{比丘}{31.0}們!有這些\twnr{四神足}{503.1},已\twnr{修習}{94.0}、已\twnr{多作}{95.0},是聖的、\twnr{出離的}{294.0},引導那樣行為者\twnr{苦的完全滅盡}{181.0},哪四個?比丘們!這裡,比丘修習\twnr{具備意欲定勤奮之行的神足}{568.0};活力定……(中略)心定……(中略)修習\twnr{具備考察定勤奮之行的神足}{569.0}。比丘們!這些是四神足,已修習、已多作,是聖的、出離的,引導那樣行為者苦的完全滅盡。」



\sutta{4}{4}{厭經}{https://agama.buddhason.org/SN/sn.php?keyword=51.4}
  「\twnr{比丘}{31.0}們!有這些\twnr{四神足}{503.1},已\twnr{修習}{94.0}、已\twnr{多作}{95.0},轉起\twnr{一向}{168.0}\twnr{厭}{15.0}、\twnr{離貪}{77.0}、\twnr{滅}{68.0}、寂靜、證智、\twnr{正覺}{185.1}、涅槃,哪四個?比丘們!這裡,比丘修習\twnr{具備意欲定勤奮之行的神足}{568.0};活力定……(中略)心定……(中略)修習\twnr{具備考察定勤奮之行的神足}{569.0}。比丘們!這些是四神足,已修習、已多作,對一向的厭、對離貪、對滅、對寂靜、對證智、對正覺、對涅槃轉起。」



\sutta{5}{5}{部分神通經}{https://agama.buddhason.org/SN/sn.php?keyword=51.5}
  「\twnr{比丘}{31.0}們!凡任何過去世的\twnr{沙門}{29.0}或\twnr{婆羅門}{17.0}曾獲得部分\twnr{神通}{503.0}者,他們全部\twnr{已自我修習}{658.0}、已自我\twnr{多作}{95.0}\twnr{四神足}{503.1};凡任何\twnr{未來世}{308.0}的沙門或婆羅門將獲得部分神通者,他們全部已自我修習、已自我多作四神足;凡任何現在的沙門或婆羅門獲得部分神通者,他們全部已自我修習、已自我多作四神足,哪四個?比丘們!這裡,比丘修習\twnr{具備意欲定勤奮之行的神足}{568.0};活力定……(中略)心定……(中略)修習\twnr{具備考察定勤奮之行的神足}{569.0}。

  比丘們!凡任何過去世的沙門或婆羅門曾獲得部分神通者,他們全部就已自我修習、已自我多作這些四神足;凡任何未來世的沙門、婆羅門將獲得部分神通者,他們全部就已自我修習、已自我多作這些四神足;凡任何現在的沙門、婆羅門獲得部分神通者,他們全部就已自我修習、已自我多作這些四神足。」



\sutta{6}{6}{完全經}{https://agama.buddhason.org/SN/sn.php?keyword=51.6}
  「\twnr{比丘}{31.0}們!凡任何過去世的\twnr{沙門}{29.0}或\twnr{婆羅門}{17.0}曾獲得完全\twnr{神通}{503.0}者,他們全部是以\twnr{四神足}{503.1}的\twnr{已自我修習}{658.0}、已自我\twnr{多作}{95.0};凡任何\twnr{未來世}{308.0}的沙門或婆羅門將獲得完全神通者,他們全部是以四神足的已自我修習、已自我多作;凡任何現在的沙門或婆羅門獲得完全神通者,他們全部是以四神足的已自我修習、已自我多作,哪四個?比丘們!這裡,比丘修習\twnr{具備意欲定勤奮之行的神足}{568.0};活力定……(中略)心定……(中略)修習\twnr{具備考察定勤奮之行的神足}{569.0}。

  比丘們!凡任何過去世的沙門或婆羅門曾獲得完全神通者,他們全部是以就這四神足的已自我修習、已自我多作;凡任何未來世的沙門或婆羅門將獲得完全神通者,他們全部是以就這四神足的已自我修習、已自我多作;凡任何現在的沙門或婆羅門獲得完全神通者,他們全部是以就這四神足的已自我修習、已自我多作。」



\sutta{7}{7}{比丘經}{https://agama.buddhason.org/SN/sn.php?keyword=51.7}
  「比丘們!凡任何過去世諸漏已滅盡,以證智自作證後,\twnr{在當生中}{42.0}\twnr{進入後住於}{66.0}無漏\twnr{心解脫}{16.0}、\twnr{慧解脫}{539.0}的比丘,他們全部是以\twnr{四神足}{503.1}的\twnr{已自我修習}{658.0}、已自我\twnr{多作}{95.0};凡任何\twnr{未來世}{308.0}諸漏已滅盡,以證智自作證後,在當生中進入後住於無漏心解脫、慧解脫的比丘,他們全部是以四神足的已自我修習、已自我多作;凡任何現在諸漏已滅盡,以證智自作證後,在當生中進入後住於無漏心解脫、慧解脫的比丘,他們全部是以四神足的已自我修習、已自我多作,哪四個?比丘們!這裡,比丘修習\twnr{具備意欲定勤奮之行的神足}{568.0};活力定……(中略)心定……(中略)修習\twnr{具備考察定勤奮之行的神足}{569.0}。

  比丘們!凡任何過去世諸漏已滅盡,以證智自作證後,在當生中進入後住於無漏心解脫、慧解脫的比丘,他們全部是以就這四神足的已自我修習、已自我多作;凡任何未來世諸漏已滅盡,以證智自作證後,在當生中進入後住於無漏心解脫、慧解脫的比丘,他們全部是以就這四神足的已自我修習、已自我多作;凡任何現在諸漏已滅盡,以證智自作證後,在當生中進入後住於無漏心解脫、慧解脫的比丘,他們全部是以就這四神足的已自我修習、已自我多作。」



\sutta{8}{8}{佛陀經}{https://agama.buddhason.org/SN/sn.php?keyword=51.8}
  「\twnr{比丘}{31.0}們!有這些\twnr{四神足}{503.1},哪四個?比丘們!這裡,比丘\twnr{修習}{94.0}\twnr{具備意欲定勤奮之行的神足}{568.0};活力定……(中略)心定……(中略)修習\twnr{具備考察定勤奮之行的神足}{569.0},比丘們!這些是四神足。

  比丘們!以這些四神足的\twnr{已自我修習}{658.0}、已自我\twnr{多作}{95.0},\twnr{如來}{4.0}被稱為『\twnr{阿羅漢}{5.0}、\twnr{遍正覺者}{6.0}』。」



\sutta{9}{9}{智經}{https://agama.buddhason.org/SN/sn.php?keyword=51.9}
  「『這是\twnr{具備意欲定勤奮之行的神足}{568.0}。』\twnr{比丘}{31.0}們!在以前不曾聽聞的諸法上,我的眼生起,智生起,慧生起,明生起,\twnr{光生起}{511.0}。『又,這個具備意欲定勤奮之行的神足應該被他\twnr{修習}{94.0}。』比丘們!……(中略)『……已修習。』比丘們!在以前不曾聽聞的諸法上,我的眼生起,智生起,慧生起,明生起,光生起。

  『這是具備活力定勤奮之行的神足。』比丘們!在以前不曾聽聞的諸法上,我的眼生起,智生起,慧生起,明生起,光生起。『又,這個具備活力定勤奮之行的神足應該被他修習。』比丘們!……(中略)『……已修習。』比丘們!在以前不曾聽聞的諸法上,我的眼生起,智生起,慧生起,明生起,光生起。

  『這是具備心定勤奮之行的神足。』比丘們!在以前不曾聽聞的諸法上,我的眼生起,智生起,慧生起,明生起,光生起。『又,這個具備心定勤奮之行的神足應該被他修習。』比丘們!……(中略)『……已修習。』比丘們!在以前不曾聽聞的諸法上,我的眼生起,智生起,慧生起,明生起,光生起。

  『這是\twnr{具備考察定勤奮之行的神足}{569.0}。』比丘們!在以前不曾聽聞的諸法上,我的眼生起,智生起,慧生起,明生起,光生起。『又,具備考察定勤奮之行的神足應該被他修習。』比丘們!……(中略)『……已修習。』比丘們!在以前不曾聽聞的諸法上,我的眼生起,智生起,慧生起,明生起,光生起。」



\sutta{10}{10}{塔廟經}{https://agama.buddhason.org/SN/sn.php?keyword=51.10}
  \twnr{被我這麼聽聞}{1.0}:

  \twnr{有一次}{2.0},\twnr{世尊}{12.0}住在毘舍離大林重閣講堂。

  那時,世尊午前時穿衣、拿起衣鉢後,\twnr{為了托鉢}{87.0}進入毘舍離。在毘舍離為了托鉢行走後,\twnr{餐後已從施食返回}{512.0},召喚\twnr{尊者}{200.0}阿難:

  「阿難!請你取\twnr{坐墊布}{183.0},\twnr{為了白天的住處}{128.0}我們將去價玻勒\twnr{塔廟}{366.0}。」

  「是的,\twnr{大德}{45.0}!」尊者阿難回答世尊後,取坐墊布,在世尊後面緊跟隨。

  那時,世尊去價玻勒塔廟。抵達後,在設置的座位坐下。尊者阿難向世尊問訊後,也在一旁坐下。世尊對在一旁坐下的尊者阿難說這個:

  「阿難!毘舍離是能被喜樂的,屋跌塔廟是能被喜樂的,喬答摩葛塔廟是能被喜樂的,七芒果樹塔廟是能被喜樂的,多子塔廟是能被喜樂的,沙楞達達塔廟是能被喜樂的,價玻勒塔廟是能被喜樂的。阿難!凡任何人的\twnr{四神足}{503.1}被\twnr{修習}{94.0}、被\twnr{多作}{95.0}、被作為車輛、被作為基礎、被實行、被累積、\twnr{被善努力}{682.0}者,阿難!當他希望時,會住留一劫或\twnr{一劫的剩餘}{840.0}。阿難!如來的四神足被修習、被多作、被作為車輛、被作為基礎、被實行、被累積、被善努力,阿難!當如來希望時,會住留一劫或一劫的剩餘。」

  尊者阿難即使這麼在被世尊作粗大的徵相、作粗大的暗示時,不能夠通達,沒要求世尊:

  「大德!為了眾人的利益,為了眾人的安樂,為了世間的憐愍,為了天-人們的需要、利益、安樂,請世尊住留一劫,請\twnr{善逝}{8.0}住留一劫。」如有那個被魔纏縛的心。

  第二次,世尊又……(中略)。

  第三次,世尊又召喚尊者阿難:

  「阿難!毘舍離是能被喜樂的,屋跌塔廟是能被喜樂的,喬答摩葛塔廟是能被喜樂的,七芒果樹塔廟是能被喜樂的,多子塔廟是能被喜樂的,沙楞達達塔廟是能被喜樂的,價玻勒塔廟是能被喜樂的。阿難!凡任何人的四神足被修習、被多作、被作為車輛、被作為基礎、被實行、被累積、被善努力,阿難!當他希望時,會住留一劫或一劫的剩餘。阿難!如來的四神足被修習、被多作、被作為車輛、被作為基礎、被實行、被累積、被善努力,阿難!當如來希望時,會住留一劫或一劫的剩餘。」

  尊者阿難即使這麼在被世尊作粗大的徵相、作粗大的暗示時,不能夠通達,沒要求世尊:

  「大德!為了眾人的利益,為了眾人的安樂,為了世間的憐愍,為了天-人們的需要、利益、安樂,請世尊住留一劫,請善逝住留一劫。」如有那個被魔纏縛的心。

  那時,世尊召喚尊者阿難:

  「阿難!請你走,現在是那個\twnr{你考量的時間}{84.0}。」

  「是的,大德!」尊者阿難回答世尊後,從座位起來、向世尊\twnr{問訊}{46.0}、\twnr{作右繞}{47.0}後,坐在不遠處的某棵樹下。

  那時,魔\twnr{波旬}{49.0}在尊者阿難離開不久,去見世尊。抵達後,對世尊說這個:

  「大德!現在,請世尊般涅槃,現在,請善逝般涅槃,大德!現在是世尊般涅槃的時機。大德!又,這個言語被世尊說:『波旬!我將不般涅槃,除非直到我的\twnr{比丘}{31.0}弟子們將成為聰明的、已被教導的、有自信的、多聞的、\twnr{持法的}{763.0}、\twnr{法隨法行}{58.0}的、\twnr{方正行的}{764.0}、\twnr{隨法行的}{765.0},把握自己\twnr{老師}{100.0}的後將告知、教導、使知、建立、開顯、解析、闡明,對已生起的異論以如法善折伏,折伏後將教導\twnr{有神變的法}{620.0}。』大德!又,現在,世尊的比丘弟子們是聰明的、已被教導的、有自信的、多聞的、持法的、法隨法行的、如法而行的、隨法行的,把握自己老師的後,告知、教導、使知、建立、開顯、解析、闡明,對已生起的異論以如法善折伏,折伏後教導有神變的法。大德!現在,請世尊般涅槃,現在,請善逝般涅槃,大德!現在是世尊般涅槃的時機。

  大德!又,這個言語被世尊說:『波旬!我將不般涅槃,除非直到我的比丘尼弟子們將成為聰明的、已被教導的、有自信的、多聞的、持法的、法隨法行的、如法而行的、隨法行的,把握自己老師的後將告知、教導、使知、建立、開顯、解析、闡明,對已生起的異論以如法善折伏,折伏後將教導有神變的法。』大德!又,現在,世尊的比丘尼弟子們是聰明的、已被教導的、有自信的、多聞的、持法的、法隨法行的、如法而行的、隨法行的,把握自己老師的後,告知、教導、使知、建立、開顯、解析、闡明,對已生起的異論以如法善折伏,折伏後教導有神變的法。大德!現在,請世尊般涅槃,現在,請善逝般涅槃,大德!現在是世尊般涅槃的時機。

  大德!又,這個言語被世尊說:『波旬!我將不般涅槃,除非直到我的\twnr{優婆塞}{98.0}弟子們成為……(中略)除非我的\twnr{優婆夷}{99.0}弟子們將成為聰明的、已被教導的、有自信的、多聞的、持法的、法隨法行的、如法而行的、隨法行的,把握自己老師的後將告知、教導、使知、建立、開顯、解析、闡明,對已生起的異論以如法善折伏,折伏後將教導有神變的法。』大德!又,現在,世尊的優婆夷弟子們是聰明的、已被教導的、有自信的、多聞的、持法的、法隨法行的、如法而行的、隨法行的,把握自己老師的後,告知、教導、使知、建立、開顯、解析、闡明,對已生起的異論以如法善折伏,折伏後教導有神變的法。大德!現在,請世尊般涅槃,現在,請善逝般涅槃,大德!現在是世尊般涅槃的時機

  大德!又,這個言語被世尊說:『波旬!我將不般涅槃,除非直到我的這個梵行成為成功的,同時也繁榮的、有名的、人多的、廣的、直到被天-人們善知道的。』大德!{那}[又,現在,]世尊的這個梵行是成功的,同時也繁榮的、有名的、人多的、廣的、直到被天-人們善知道的。大德!現在,請世尊般涅槃,請善逝般涅槃,大德!現在是世尊般涅槃的時機。」

  在這麼說時,世尊對魔波旬說這個:

  「波旬!請你成為\twnr{放心的}{906.0},不久,將有如來的般涅槃,從現在起三個月後,如來將般涅槃。」

  那時,世尊在價玻勒塔廟具念地、正知地放棄\twnr{壽行}{766.0}。而在世尊放棄壽行時,有恐怖的、共\twnr{身毛豎立}{152.0}的大地震,\twnr{且天鼓破裂}{767.1}。

  那時,世尊知道這件事後,在那時候吟出這個\twnr{優陀那}{184.0}:

  「\twnr{權衡不可比的與存在}{768.0},\twnr{牟尼}{125.0}\twnr{放棄有行}{769.1},

   自身內樂者入定者,破裂\twnr{如鎧甲般自己的存在}{770.0}。」[\ccchref{AN.8.70}{https://agama.buddhason.org/AN/an.php?keyword=8.70}, \ccchref{DN.16}{https://agama.buddhason.org/DN/dm.php?keyword=16} 166-167段, \ccchref{Ud.51}{https://agama.buddhason.org/Ud/dm.php?keyword=51}]

  價玻勒品第一,其\twnr{攝頌}{35.0}:

  「此岸與已錯失,聖與厭,

   部分、完全、比丘,佛陀、智與塔廟。」





\pin{講堂震動品}{11}{20}
\sutta{11}{11}{以前經}{https://agama.buddhason.org/SN/sn.php?keyword=51.11}
  起源於舍衛城。

  「\twnr{比丘}{31.0}們!當就在我\twnr{正覺}{185.1}以前,還是未\twnr{現正覺}{75.0}的\twnr{菩薩}{186.0}時想這個:『什麼因、什麼\twnr{緣}{180.0},有神足的修習呢?』比丘們!那個我想這個:『這裡,比丘\twnr{修習}{94.0}\twnr{具備意欲定勤奮之行的神足}{568.0}:像這樣,我的意欲將不是過鬆的、將不是過緊的、將不是向內收斂的、將不是向外散亂的,以及住於\twnr{前後有感知的}{570.0}:後如前那樣地,前如後那樣地;上如下那樣地,下如上那樣地;在夜間如在白天那樣地,在白天如在夜間那樣地。像這樣,以打開的、無覆蓋的心,修習有光輝的心。

  修習具備活力定勤奮之行的神足:像這樣,我的活力將不是過鬆的、將不是過緊的、將不是向內收斂的、將不是向外散亂的,以及住於前後有感知的:後如前那樣地,前如後那樣地;上如下那樣地,下如上那樣地;在夜間如在白天那樣地,在白天如在夜間那樣地。像這樣,以打開的、無覆蓋的心,修習有光輝的心。

  修習具備心定勤奮之行的神足:像這樣,我的心將不是過鬆的、將不是過緊的、將不是向內收斂的、將不是向外散亂的,以及住於前後有感知的:後如前那樣地,前如後那樣地;上如下那樣地,下如上那樣地;在夜間如在白天那樣地,在白天如在夜間那樣地。像這樣,以打開的、無覆蓋的心,修習有光輝的心。

  修習\twnr{具備考察定勤奮之行的神足}{569.0}:像這樣,我的考察將不是過鬆的、將不是過緊的、將不是向內收斂的、將不是向外散亂的,以及住於前後有感知的:後如前那樣地,前如後那樣地;上如下那樣地,下如上那樣地;在夜間如在白天那樣地,在白天如在夜間那樣地。像這樣,以打開的、無覆蓋的心,修習有光輝的心。

  比丘們!比丘在\twnr{四神足}{503.1}已這樣修習、已這樣\twnr{多作}{95.0}時,體驗各種神通種類:是一個後變成多個,又,是多個後變成一個;現身、隱身、穿牆、穿壘、穿山無阻礙地行走猶如在虛空中;在地中作浮沈猶如在水中,又,在不被破裂的水上行走猶如在地上;在空中以盤腿來去猶如有翅膀的鳥,又,以手碰觸、撫摸這些這麼大神通力、這麼大威力的日月;以身體行使自在直到梵天世界。

  比丘們!比丘在四神足這樣已修習、這樣已多作時,以清淨、超越常人的天耳界聽到二者的聲音:「天與人,以及在遠處、近處。」

  比丘們!比丘在四神足這樣已修習、這樣已多作時,對其他眾生、其他個人\twnr{以心熟知心後}{393.0}知道:有貪的心為「有貪的心」,知道離貪的心為「離貪的心」;知道有瞋的心為「有瞋的心」,知道離瞋的心為「離瞋的心」;知道有癡的心為「有癡的心」,知道離癡的心為「離癡的心」;知道\twnr{收斂的心}{674.0}為「收斂的心」,知道散亂的心為「散亂的心」;知道廣大的心為「廣大的心」,知道非廣大的心為「非廣大的心」;知道有更上的心為「有更上的心」,知道無更上的心為「無更上的心」;知道得定的心為「得定的心」,知道未得定的心為「未得定的心」;知道已解脫的心為「已解脫的心」,知道未解脫的心為「未解脫的心」。

  比丘們!比丘在四神足這樣已修習、這樣已多作時,回憶(隨念)許多前世住處,即:一生、二生、三生、四生、五生、十生、二十生、三十生、四十生、五十生、百生、千生、十萬生、許多壞劫、許多成劫、許多\twnr{壞成劫}{403.0}:「在那裡我是這樣的名、這樣的姓氏、這樣的容貌、這樣的食物、這樣的苦樂感受、這樣的壽長,那位從那裡死後我出生在那裡,在那裡我又是這樣的名、這樣的姓氏、這樣的容貌、這樣的食物、這樣的苦樂感受、這樣的壽長,那位從那裡死後被再生在這裡。」像這樣,回憶許多\twnr{有行相的、有境遇的}{500.0}前世住處。

  比丘們!比丘在四神足這樣已修習、這樣已多作時,以清淨、超越常人的天眼看見死沒往生的眾生:下劣的、勝妙的,美的、醜的,善去的、惡去的,知道依業到達的眾生:「確實,這些尊師眾生具備身惡行、具備語惡行、具備意惡行,是對聖者斥責者、邪見者、邪見行為的受持者,他們以身體的崩解,死後已往生\twnr{苦界}{109.0}、\twnr{惡趣}{110.0}、\twnr{下界}{111.0}、地獄,又或這些尊師眾生具備身善行、具備語善行、具備意善行,是對聖者不斥責者、正見者、正見行為的受持者,他們以身體的崩解,死後已往生\twnr{善趣}{112.0}、天界。」像這樣,以清淨、超越常人的天眼看見死沒往生的眾生:下劣的、勝妙的,美的、醜的,善去的、惡去的,知道依業到達的眾生。

  比丘們!比丘在四神足這樣已修習、這樣已多作時,以諸\twnr{漏}{188.0}的滅盡,以證智自作證後,在當生中以\twnr{進入後住於}{66.0}無漏\twnr{心解脫}{16.0}、\twnr{慧解脫}{539.0}。』」



\sutta{12}{12}{大果經}{https://agama.buddhason.org/SN/sn.php?keyword=51.12}
  「\twnr{比丘}{31.0}們!有這些\twnr{四神足}{503.1},已\twnr{修習}{94.0}、已\twnr{多作}{95.0},有大果、\twnr{大效益}{113.0}。

  比丘們!而怎樣四神足已修習、怎樣已多作,有大果、大效益?比丘們!這裡,比丘修習\twnr{具備意欲定勤奮之行的神足}{568.0}:像這樣,我的意欲將不是過鬆的、將不是過緊的、將不是向內收斂的、將不是向外散亂的,以及住於\twnr{前後有感知的}{570.0}:後如前那樣地,前如後那樣地;上如下那樣地,下如上那樣地;在夜間如在白天那樣地,在白天如在夜間那樣地。像這樣,以打開的、無覆蓋的心,修習有光輝的心。

  活力定……(中略)心定……(中略)修習\twnr{具備考察定勤奮之行的神足}{569.0}:像這樣,我的考察將不是過鬆的、將不是過緊的、將不是向內收斂的、將不是向外散亂的,以及住於前後有感知的:後如前那樣地,前如後那樣地;上如下那樣地,下如上那樣地;在夜間如在白天那樣地,在白天如在夜間那樣地。像這樣,以打開的、無覆蓋的心,修習有光輝的心。當四神足這麼已修習、這麼已多作,有大果、大效益。

  比丘們!這樣,在四神足已修習、這樣已多作時,比丘體驗各種神通種類:是一個後變成多個,又,是多個後變成一個……(中略)以身體行使自在直到梵天世界。……(中略)

  比丘們!這樣,在四神足已修習、這樣已多作時,比丘以諸\twnr{漏}{188.0}的滅盡,\twnr{當生}{42.0}以證智自作證後,\twnr{進入後住於}{66.0}無漏\twnr{心解脫}{16.0}、\twnr{慧解脫}{539.0}。」



\sutta{13}{13}{意欲定經}{https://agama.buddhason.org/SN/sn.php?keyword=51.13}
  「\twnr{比丘}{31.0}們!即使比丘依止意欲而得到定、得到\twnr{心一境性}{255.0},這被稱為『意欲定』。

  他為了未生起的惡不善法之不生起使意欲生起、努力、發動活力、盡心、勤奮;為了已生起的惡不善法之捨斷使意欲生起、努力、發動活力、盡心、勤奮;為了未生起的善法之生起使意欲生起、努力、發動活力、盡心、勤奮;為了已生起的諸善法之存續、不忘失、增大、成滿、修習圓滿使意欲生起、努力、發動活力、盡心、勤奮,這被稱為『勤奮之行』。

  像這樣,這是意欲、這是意欲定、這些是勤奮之行:比丘們!這被稱為『\twnr{具備意欲定勤奮之行的神足}{568.0}』。

  比丘們!如果比丘依止活力而得到定、得到心一境性,這被稱為『\twnr{活力定}{x638}』。

  他為了未生起的……(中略)為了已生起的諸善法之存續、不忘失、增大、成滿、修習圓滿使意欲生起、努力、發動活力、盡心、勤奮,這被稱為『勤奮之行』。

  像這樣,這是活力、這是活力定、這些是勤奮之行:比丘們!這被稱為『具備活力定勤奮之行的神足』。

  比丘們!如果比丘依止心而得到定、得到心一境性,這被稱為『\twnr{心定}{x639}』。

  他為了未生起的惡……(中略)為了已生起的諸善法之存續、不忘失、增大、成滿、修習圓滿使意欲生起、努力、發動活力、盡心、勤奮,這被稱為『勤奮之行』。

  像這樣,這是\twnr{心}{x640}、這是心定、這些是勤奮之行:比丘們!這被稱為『具備心定勤奮之行的神足』。

  比丘們!如果比丘依止\twnr{考察}{x641}而得到定、得到心一境性,這被稱為『考察定』。

  他為了未生起的惡不善法之不生起使意欲生起、努力、發動活力、盡心、勤奮……(中略)為了已生起的諸善法之存續、不忘失、增大、成滿、修習圓滿使意欲生起、努力、發動活力、盡心、勤奮,這被稱為『勤奮之行』。

  像這樣,這是考察、這是考察定、這些是勤奮之行:比丘們!這被稱為『\twnr{具備考察定勤奮之行的神足}{569.0}』。」



\sutta{14}{14}{目揵連經}{https://agama.buddhason.org/SN/sn.php?keyword=51.14}
  \twnr{被我這麼聽聞}{1.0}:

  \twnr{有一次}{2.0},\twnr{世尊}{12.0}住在舍衛城東園鹿母講堂。

  當時,眾多\twnr{比丘}{31.0}住在鹿母講堂下,是掉舉的、傲慢的、浮躁的、饒舌的、言語散亂的、\twnr{念已忘失的}{216.0}、不正知的、不得定的、散亂心的、根不控制的。

  那時,世尊召喚\twnr{尊者}{200.0}大目揵連:

  「目揵連!這些\twnr{同梵行者}{138.0}們住在鹿母講堂下,是掉舉的、傲慢的、浮躁的、饒舌的、言語散亂的、念已忘失的、不正知的、不得定的、散亂心的、根不控制的,目揵連!請你去使那些比丘\twnr{激起急迫感}{373.0}。」

  「是的,\twnr{大德}{45.0}!」尊者目揵連回答世尊後,依以腳拇指\twnr{造作像那樣的神通作為}{425.0}方式,使鹿母講堂震動、大震動、大搖動。

  那時,那些比丘驚怖、生起\twnr{身毛豎立}{152.0}地在一旁站立:「實在不可思議啊,\twnr{先生}{202.0}!實在\twnr{未曾有}{206.0}啊,先生!確實無風,且這深基礎的、善埋的、不動的、沒有大震動的鹿母講堂卻被震動、被大震動、被大搖動。」

  那時,世尊去見那些比丘。抵達後,對那些比丘說這個:

  「比丘們!你們為何驚怖、身毛豎立地在一旁站立呢?」

  「不可思議啊,大德!未曾有啊,大德!確實無風,且這深基礎的、善埋的、不動的、沒有大震動的鹿母講堂卻被震動、被大震動、被大搖動。」 

  「比丘們!以想要使你們就激起急迫感,鹿母講堂被目揵連比丘以腳拇指震動、大震動、大搖動。比丘們!你們怎麼想它:以什麼法的\twnr{已自我修習}{658.0}、已自我\twnr{多作}{95.0},目揵連比丘有這麼\twnr{大神通力}{405.0}、這麼大威力呢?」

  「\twnr{大德}{45.0}!我們的法是世尊為根本的、\twnr{世尊為導引的}{56.0}、世尊為依歸的,大德!就請世尊說明這個所說的義理,\twnr{那就好了}{44.0}!聽聞世尊的[教說]後,比丘們將會\twnr{憶持}{57.0}。」

  「比丘們!那樣的話,\twnr{你們要聽}{43.0}!比丘們!以\twnr{四神足}{503.1}的已自我修習、已自我多作,目揵連比丘有這麼大神通力、這麼大威力,哪四個?比丘們!這裡,目揵連比丘修習\twnr{具備意欲定勤奮之行的神足}{568.0};活力定……(中略)心定……(中略)修習\twnr{具備考察定勤奮之行的神足}{569.0}:『像這樣,我的考察將不過鬆,也不過緊;不內斂,也不外散。』他住於前後有感知的:『後如前那樣地,前如後那樣地;上如下那樣地,下如上那樣地;在夜間如在白天那樣地,在白天如在夜間那樣地。』像這樣,以打開的、無覆蓋的心,修習有光輝的心。比丘們!以這些四神足的已自我修習、已自我多作,目揵連比丘有這麼大神通力、這麼大威力。比丘們!又,以這些四神足的已自我修習、已自我多作,目揵連比丘體驗各種神通種類:是一個後變成多個,又,是多個後變成一個……(中略)以身體行使自在直到梵天世界。……(中略)比丘們!又,以這些四神足的已自我修習、已自我多作,目揵連比丘以諸\twnr{漏}{188.0}的滅盡,\twnr{當生}{42.0}以證智自作證後,\twnr{進入後住於}{66.0}無漏\twnr{心解脫}{16.0}、\twnr{慧解脫}{539.0}。」



\sutta{15}{15}{巫男巴婆羅門經}{https://agama.buddhason.org/SN/sn.php?keyword=51.15}
  \twnr{被我這麼聽聞}{1.0}:

  \twnr{有一次}{2.0},\twnr{尊者}{200.0}阿難住在\twnr{憍賞彌}{140.0}瞿師羅園。

  那時,巫男巴婆羅門去見尊者阿難。抵達後,與尊者阿難一起互相問候。交換應該被互相問候的友好交談後,在一旁坐下。在一旁坐下的巫男巴婆羅門對尊者阿難說這個:

  「阿難尊師!為了什麼目的在\twnr{沙門}{29.0}\twnr{喬達摩}{80.0}處梵行被住?」

  「婆羅門!為了意欲的捨斷在沙門喬達摩處梵行被住。」

  「阿難尊師!那麼,為了這個意欲的捨斷,有道、\twnr{有道跡}{359.0}嗎?」

  「婆羅門!為了這個意欲的捨斷,有道、有道跡。」

  「阿難尊師!那麼,為了這個意欲的捨斷,什麼是道、什麼是道跡呢?」

  「婆羅門!這裡,\twnr{比丘}{31.0}\twnr{修習}{94.0}\twnr{具備意欲定勤奮之行的神足}{568.0},\twnr{活力定}{x638}……(中略)\twnr{心定}{x639}……(中略)\twnr{具備考察定勤奮之行的神足}{569.0},婆羅門!為了這個意欲的捨斷,這是道,這是道跡。」

  「阿難尊師!在存在這樣時,{\twnr{是有邊的,不是無邊的}{x642}}[是無邊的,不是有邊的],『就以意欲將捨斷意欲』,\twnr{這不存在可能性}{650.0}。」

  「婆羅門!那樣的話,就在這件事上我將反問你,你就如對你能接受的那樣回答它。婆羅門!你怎麼想它:你先前有『我將到園裡』的意欲,當那個你已到園裡時,凡對應那個的意欲它被止息嗎?」

  「是的,先生!」

  「你先前有『我將到園裡』的活力,當那個你已到園裡時,凡對應那個的活力被止息嗎?」

  「是的,先生!」

  「你先前有『我將到園裡』的\twnr{心}{x640},當那個你已到園裡時,凡對應那個的心被止息嗎?」

  「是的,先生!」

  「你先前有『我將到園裡』的\twnr{考察}{x641},當那個你已到園裡時,凡對應那個的考察被止息嗎?」

  「是的,先生!」

  「同樣的,婆羅門!凡那位漏已滅盡的、已完成的、應該被作的已作的、負擔已卸的、自己的利益已達成的、有之結已滅盡的、以\twnr{究竟智}{191.0}解脫的\twnr{阿羅漢}{5.0}比丘,他凡先前有為了達到阿羅漢境界的意欲,在達到阿羅漢境界時,凡對應那個的意欲被止息;凡先前有為了達到阿羅漢境界的活力,在達到阿羅漢境界時,凡對應那個的活力被止息;凡先前有為了達到阿羅漢境界的心,在達到阿羅漢境界時,凡對應那個的心被止息;凡先前有為了達到阿羅漢境界的考察,在達到阿羅漢境界時,凡對應那個的考察被止息。

  婆羅門!你怎麼想它:這樣是有邊的,或是無邊的呢?」

  「阿難尊師!確實,這樣是有邊的,不是無邊的。

  太偉大了,阿難尊師!太偉大了,阿難尊師!阿難尊師!猶如扶正顛倒的,或揭開隱藏的,或告知迷路者的道路,或在黑暗中持燈火:『有眼者們看見諸色。』同樣的,法被阿難\twnr{尊師}{203.0}以種種法門說明。阿難尊師!這個我\twnr{歸依}{284.0}那喬達摩世尊、法、\twnr{比丘僧團}{65.0},請阿難尊師記得我為\twnr{優婆塞}{98.0},從今天起\twnr{已終生歸依}{64.0}。」



\sutta{16}{16}{沙門婆羅門經第一}{https://agama.buddhason.org/SN/sn.php?keyword=51.16}
  「\twnr{比丘}{31.0}們!凡任何過去世\twnr{沙門}{29.0}或\twnr{婆羅門}{17.0}曾有\twnr{大神通力}{405.0}、大威力者,他們全部是以四神足的\twnr{已自我修習}{658.0}、已自我\twnr{多作}{95.0};凡任何\twnr{未來世}{308.0}沙門或婆羅門將有大神通力、大威力者,他們全部是以四神足的已自我修習、已自我多作;凡任何現在沙門或婆羅門將有大神通力、大威力者,他們全部是以四神足的已自我修習、已自我多作,哪四個?比丘們!這裡,比丘修習\twnr{具備意欲定勤奮之行的神足}{568.0};活力定……(中略)心定……(中略)修習\twnr{具備考察定勤奮之行的神足}{569.0}。

  比丘們!凡任何過去世沙門或婆羅門曾有大神通力、大威力者,他們全部是以就這四神足的已自我修習、已自我多作;凡任何未來世沙門或婆羅門將有大神通力、大威力者,他們全部是以就這四神足的已自我修習、已自我多作;凡任何現在沙門或婆羅門將有大神通力、大威力者,他們全部是以就這四神足的已自我修習、已自我多作。」



\sutta{17}{17}{沙門婆羅門經第二}{https://agama.buddhason.org/SN/sn.php?keyword=51.17}
  「\twnr{比丘}{31.0}們!凡任何過去世的\twnr{沙門}{29.0}或\twnr{婆羅門}{17.0}體驗各種神通種類:是一個後變成多個,又,是多個後變成一個;現身、隱身、穿牆、穿壘、穿山無阻礙地行走猶如在虛空中;在地中作浮沈猶如在水中,又,在不被破裂的水上行走猶如在地上;在空中以盤腿來去猶如有翅膀的鳥,又,以手碰觸、撫摸這些這麼大神通力、這麼大威力的日月;以身體行使自在直到梵天世界者,他們全部是以\twnr{四神足}{503.1}的\twnr{已自我修習}{658.0}、已自我\twnr{多作}{95.0}。

  凡任何\twnr{未來世}{308.0}的沙門或婆羅門將體驗各種神通種類:是一個後他將變成多個,又,是多個後他將變成一個;又,他將現身、隱身、穿牆、穿壘、穿山無阻礙地行走猶如在虛空中;又,他將在地中作浮沈猶如在水中;又,他將在不被破裂的水上行走猶如在地上;又,他將在空中以盤腿來去猶如有翅膀的鳥;又,他將以手碰觸、撫摸這些這麼大神通力、這麼大威力的日月;又,他將以身體行使自在直到梵天世界者,他們全部是以四神足的已自我修習、已自我多作。

  凡任何現在的沙門或婆羅門體驗各種神通種類:是一個後變成多個,又,是多個後變成一個;現身、隱身、穿牆、穿壘、穿山無阻礙地行走猶如在虛空中;在地中作浮沈猶如在水中,又,在不被破裂的水上行走猶如在地上;在空中以盤腿來去猶如有翅膀的鳥,又,以手碰觸、撫摸這些這麼大神通力、這麼大威力的日月;以身體行使自在直到梵天世界者,他們全部是以四神足的已自我修習、已自我多作。

  哪四個?比丘們!這裡,比丘修習\twnr{具備意欲定勤奮之行的神足}{568.0};活力定……(中略)心定……(中略)修習\twnr{具備考察定勤奮之行的神足}{569.0}。比丘們!凡任何過去世的沙門或婆羅門體驗各種神通種類:是一個後他變成多個……(中略)以身體行使自在直到梵天世界者,他們全部是以就這四神足的已自我修習、已自我多作。

  凡任何未來世的沙門或婆羅門將體驗各種神通種類:有了一個後他將變成多個……(中略)又,他將以身體行使自在直到梵天世界者,他們全部是以就這四神足的已自我修習、已自我多作。

  凡任何現在的沙門或婆羅門體驗各種神通種類:是一個後他變成多個……(中略)以身體行使自在直到梵天世界者,他們全部是以就這四神足的已自我修習、已自我多作。」



\sutta{18}{18}{比丘經}{https://agama.buddhason.org/SN/sn.php?keyword=51.18}
  「比丘們!以\twnr{四神足}{503.1}的\twnr{已自我修習}{658.0}、已自我\twnr{多作}{95.0},比丘以諸\twnr{漏}{188.0}的滅盡,以證智自作證後,在當生中\twnr{進入後住於}{66.0}無漏\twnr{心解脫}{16.0}、\twnr{慧解脫}{539.0}。哪四個?比丘們!這裡,比丘修習\twnr{具備意欲定勤奮之行的神足}{568.0};活力定……(中略)心定……(中略)修習\twnr{具備考察定勤奮之行的神足}{569.0}。比丘們!以這些四神足的已自我修習、已自我多作,比丘以諸漏的滅盡,以證智自作證後,在當生中進入後住於無漏心解脫、慧解脫。」



\sutta{19}{19}{神通等之教導經}{https://agama.buddhason.org/SN/sn.php?keyword=51.19}
  「\twnr{比丘}{31.0}們!我將為你們教導神通、神足、神足的\twnr{修習}{94.0}、導向神足的修習道跡,\twnr{你們要聽}{43.0}它!

  比丘們!而什麼是神通?比丘們!這裡,比丘體驗各種神通種類:是一個後他變成多個……(中略)以身體行使自在直到梵天世界。比丘們!這被稱為神通。

  比丘們!而什麼是神足呢?比丘們!凡那個道、凡道跡轉起神通之得到、神通之獲得者,比丘們!這被稱為神足。

  比丘們!而什麼是神足的修習?比丘們!這裡,比丘修習\twnr{具備意欲定勤奮之行的神足}{568.0};活力定……(中略)心定……(中略)修習\twnr{具備考察定勤奮之行的神足}{569.0}。比丘們!這被稱為神足的修習。

  比丘們!而什麼是導向神足的修習道跡呢?就是\twnr{八支聖道}{525.0},即:正見、正志、正語、正業、正命、正精進、正念、正定。比丘們!這被稱為導向神足的修習道跡。」[\suttaref{SN.51.27}~ \suttaref{SN.51.30}]



\sutta{20}{20}{解析經}{https://agama.buddhason.org/SN/sn.php?keyword=51.20}
  「\twnr{比丘}{31.0}們!有這些\twnr{四神足}{503.1},已\twnr{修習}{94.0}、已\twnr{多作}{95.0},有大果、\twnr{大效益}{113.0}。

  比丘們!而怎樣四神足已修習、怎樣已多作,有大果、大效益?比丘們!這裡,比丘修習\twnr{具備意欲定勤奮之行的神足}{568.0}:像這樣,我的意欲將不是過鬆的、將不是過緊的、將不是向內收斂的、將不是向外散亂的,以及住於\twnr{前後有感知的}{570.0}:後如前那樣地,前如後那樣地;上如下那樣地,下如上那樣地;在夜間如在白天那樣地,在白天如在夜間那樣地。像這樣,以打開的、無覆蓋的心,修習有光輝的心。

  活力定……(中略)心定……(中略)修習\twnr{具備考察定勤奮之行的神足}{569.0}:像這樣,我的考察將不是過鬆的、將不是過緊的、將不是向內收斂的、將不是向外散亂的,以及住於前後有感知的:後如前那樣地,前如後那樣地;上如下那樣地,下如上那樣地;在夜間如在白天那樣地,在白天如在夜間那樣地。像這樣,以打開的、無覆蓋的心,修習有光輝的心。

  比丘們!而什麼是過鬆的意欲呢?比丘們!凡與倦怠俱行、與倦怠相應的意欲,比丘們!這被稱為過鬆的意欲。

  比丘們!而什麼是過緊的意欲呢?比丘們!凡與掉舉俱行、與掉舉相應的意欲,比丘們!這被稱為過緊的意欲。

  比丘們!而什麼是向內收斂的意欲呢?比丘們!凡與惛沈睡眠俱行、與惛沈睡眠相應的意欲,比丘們!這被稱為向內收斂的意欲。

  比丘們!而什麼是向外散亂的意欲呢?比丘們!凡隨關於外部的\twnr{五種欲}{187.0}散亂、散開的意欲,比丘們!這被稱為向外散亂的意欲。

  比丘們!而怎樣比丘住於前後有感知的:後如前那樣地,前如後那樣地?比丘們!這裡,前後想被比丘以慧善把握、善\twnr{作意}{43.1}、善理解、善貫通,比丘們!比丘這樣住於前後有感知的:後如前那樣地,前如後那樣地。

  比丘們!而怎樣比丘住於上如下那樣地,下如上那樣地?比丘們!這裡,比丘就這個身體從腳掌之上,從髮梢之下,皮膚為邊界,有種種種類不淨充滿的,省察:『在這個身體中有頭髮、體毛、指甲、牙齒、皮膚、肌肉、筋腱、骨骼、骨髓、腎臟、心臟、肝臟、肋膜、脾臟、肺臟、腸子、腸間膜、胃、糞便、膽汁、痰、膿、血、汗、脂肪、眼淚、油脂、唾液、鼻涕、關節液、尿。』比丘們!比丘這樣住於上如下那樣地,下如上那樣地。

  比丘們!而怎樣比丘住於在夜間如在白天那樣地,在白天如在夜間那樣地?比丘們!這裡,比丘白天凡以那些行相、相標、特相修習具備意欲定勤奮之行的神足者,那個夜晚[也]以那些行相、相標、特相相修習具備意欲定勤奮之行的神足;又或夜晚凡以那些行相、相標、特相習具備意欲定勤奮之行的神足者,那個白天[也]以那些行相、相標、特相相修習具備意欲定勤奮之行的神足,比丘們!比丘這樣住於在夜間如在白天那樣地,在白天如在夜間那樣地。

  比丘們!而怎樣比丘以打開的、無覆蓋的心修習有光輝的心?比丘們!這裡,\twnr{光明想}{504.0}被比丘善把握,白天想被\twnr{善決意}{x643},比丘們!比丘這樣以打開的、無覆蓋的心修習有光輝的心。

  比丘們!而什麼是過鬆的活力呢?比丘們!凡與倦怠俱行、與倦怠相應的活力,比丘們!這被稱為過鬆的活力。 

  比丘們!而什麼是過緊的活力呢?比丘們!凡與掉舉俱行、與掉舉相應活力,比丘們!這被稱為過緊的活力。 

  比丘們!而什麼是向內收斂的活力呢?比丘們!凡與惛沈睡眠俱行、與惛沈睡眠相應的活力,比丘們!這被稱為向內收斂的活力。

  比丘們!而什麼是向外散亂的活力呢?比丘們!凡隨關於外部的五種欲散亂、散開的活力,比丘們!這被稱為向外散亂的活力。……(中略)

  比丘們!而怎樣比丘以打開的、無覆蓋的心修習有光輝的心?比丘們!這裡,比丘善把握光明想,善確立白天想,比丘們!比丘這樣以打開的、無覆蓋的心修習有光輝的心。

  比丘們!而什麼是過鬆的心呢?比丘們!凡與倦怠俱行、與倦怠相應的心,比丘們!這被稱為過鬆的心。 

  比丘們!而什麼是過緊的心呢?比丘們!凡與掉舉俱行、與掉舉相應的心,比丘們!這被稱為過緊的心。 

  比丘們!而什麼是向內收斂的心呢?比丘們!凡與惛沈睡眠俱行、與惛沈睡眠相應的心,比丘們!這被稱為向內收斂的心。

  比丘們!而什麼是向外散亂的心呢?比丘們!凡隨關於外部的五種欲散亂、散開的心,比丘們!這被稱為向外散亂的心。……(中略)比丘們!比丘這樣以打開的、無覆蓋的心修習有光輝的心。

  比丘們!而什麼是過鬆的考察呢?比丘們!凡與倦怠俱行、與倦怠相應的考察,比丘們!這被稱為過鬆的考察。 

  比丘們!而什麼是過緊的考察呢?比丘們!凡與掉舉俱行、與掉舉相應考察,比丘們!這被稱為過緊的考察。 

  比丘們!而什麼是向內收斂的考察呢?比丘們!凡與惛沈睡眠俱行、與惛沈睡眠相應的考察,比丘們!這被稱為向內收斂的考察。

  比丘們!而什麼是向外散亂的考察呢?比丘們!凡隨關於外部的五種欲散亂、散開的考察,比丘們!這被稱為向外散亂的考察。……(中略)比丘們!比丘這樣以打開的、無覆蓋的心修習有光輝的心。

  比丘們!四神足這樣已修習、這樣已多作,有大果、大效益。

  比丘們!比丘這樣,在四神足已修習、這樣已多作時,比丘體驗各種神通種類:是一個後變成多個,又,是多個後變成一個……(中略)以身體行使自在直到梵天世界。……比丘們!比丘這樣,在四神足已修習、這樣已多作時,比丘以諸\twnr{漏}{188.0}的滅盡,\twnr{當生}{42.0}以證智自作證後,\twnr{進入後住於}{66.0}無漏\twnr{心解脫}{16.0}、\twnr{慧解脫}{539.0}。」

  講堂震動品第二,其\twnr{攝頌}{35.0}:

  「以前、大果、意欲,目揵連與巫男巴,

   二則沙門婆羅門、比丘,教導與以解析。」





\pin{鐵球品}{21}{32}
\sutta{21}{21}{道經}{https://agama.buddhason.org/SN/sn.php?keyword=51.21}
  起源於舍衛城。

  「\twnr{比丘}{31.0}們!當就在我\twnr{正覺}{185.1}以前,還是未\twnr{現正覺}{75.0}的\twnr{菩薩}{186.0}時想這個:『對神足的修習來說,什麼是道、什麼是道跡呢?』比丘們!那個我想這個:『這裡,比丘\twnr{修習}{94.0}\twnr{具備意欲定勤奮之行的神足}{568.0}:像這樣,我的意欲將不是過鬆的、將不是過緊的、將不是向內收斂的、將不是向外散亂的,以及住於\twnr{前後有感知的}{570.0}:後如前那樣地,前如後那樣地;上如下那樣地,下如上那樣地;在夜間如在白天那樣地,在白天如在夜間那樣地。像這樣,以打開的、無覆蓋的心,修習有光輝的心。

  修習具備活力定……(中略)心定……(中略)修習\twnr{具備考察定勤奮之行的神足}{569.0}:像這樣,我的考察將不是過鬆的、將不是過緊的、將不是向內收斂的、將不是向外散亂的,以及住於前後有感知的:後如前那樣地,前如後那樣地;上如下那樣地,下如上那樣地;在夜間如在白天那樣地,在白天如在夜間那樣地。像這樣,以打開的、無覆蓋的心,修習有光輝的心。

  比丘們!比丘在\twnr{四神足}{503.1}這樣已修習、這樣已多作時,他體驗各種神通種類:是一個後變成多個,又,是多個後變成一個……(中略)以身體行使自在直到梵天世界。……

  比丘們!比丘這樣,在四神足已修習、這樣已多作時,以諸\twnr{漏}{188.0}的滅盡,以證智自作證後,在當生中以\twnr{進入後住於}{66.0}無漏\twnr{心解脫}{16.0}、\twnr{慧解脫}{539.0}。』」

   (六證智都應該使之被細說)



\sutta{22}{22}{鐵球經}{https://agama.buddhason.org/SN/sn.php?keyword=51.22}
  起源於舍衛城。

  那時,\twnr{尊者}{200.0}阿難去見\twnr{世尊}{12.0}。抵達後,向世尊\twnr{問訊}{46.0}後,在一旁坐下。在一旁坐下的尊者阿難對世尊說這個:

  「\twnr{大德}{45.0}!世尊記得(證知)以\twnr{神通}{503.0}、以\twnr{意生身}{755.1}到梵天世界嗎?」

  「阿難!我記得以神通、以意生身到梵天世界。」

  「大德!世尊記得以這\twnr{四大}{646.0}身,以神通到梵天世界嗎?」  

  「阿難!我記得以這四大身,以神通到梵天世界。」  

  「大德!凡世尊\twnr{能夠}{x644}以神通、以意生身到梵天世界,大德!以及凡世尊記得以這四大之身,以神通到梵天世界者,大德!這正是世尊的不可思議的與\twnr{未曾有}{206.0}的。」  

  「阿難!如來正是不可思議的與具備不可思議法的,阿難!如來正是未曾有的與具備未曾有法的。阿難!凡在如來將身\twnr{收存}{x645}於心中,也將心收存於身中時,在身上進入幸福(樂)想與輕想後而住,阿難!那時,如來的身體就成為較輕的、較柔軟的、較\twnr{適合作業的}{412.0}、較輝耀的。

  阿難!猶如在白天被加熱的鐵球就成為較輕的、較柔軟的、較適合作業的、較輝耀的。同樣的,阿難!凡在如來將身收存於心中,也將心收存於身中時,在身上進入幸福想與輕想後而住,阿難!那時,如來的身體成為較輕的、較柔軟的、較適合作業的、較輝耀的。

  阿難!凡在如來將身收存於心中,也將心收存於身中時,在身上進入幸福想與輕想後而住,阿難!那時,如來的身體就少困難地從地面上升到天空,他體驗各種神通種類:是一個後變成多個,又,是多個後變成一個……(中略)以身體行使自在直到梵天世界。

  阿難!猶如被風支撐的輕棉花絨或木棉花絨,就少困難地從地面上升到虛空。同樣的,凡在如來將身收存於心中,也將心收存於身中時,在身上進入幸福想與輕想後而住,阿難!那時,如來的身體就少困難地從地面上升到天空,他體驗各種神通種類:是一個後變成多個,又,是多個後變成一個……(中略)以身體行使自在直到梵天世界。」



\sutta{23}{23}{比丘經}{https://agama.buddhason.org/SN/sn.php?keyword=51.23}
  「比丘們!有這些\twnr{四神足}{503.1},哪四個?比丘們!這裡,比丘\twnr{修習}{94.0}\twnr{具備意欲定勤奮之行的神足}{568.0};活力定……(中略)心定……(中略)修習\twnr{具備考察定勤奮之行的神足}{569.0},比丘們!這些是四神足。比丘們!以這些四神足的\twnr{已自我修習}{658.0}、已自我\twnr{多作}{95.0},比丘以諸\twnr{漏}{188.0}的滅盡,以證智自作證後,在當生中\twnr{進入後住於}{66.0}無漏\twnr{心解脫}{16.0}、\twnr{慧解脫}{539.0}。」



\sutta{24}{24}{概要經}{https://agama.buddhason.org/SN/sn.php?keyword=51.24}
  「\twnr{比丘}{31.0}們!有這些\twnr{四神足}{503.1},哪四個?比丘們!這裡,比丘\twnr{修習}{94.0}\twnr{具備意欲定勤奮之行的神足}{568.0};活力定……(中略)心定……(中略)修習\twnr{具備考察定勤奮之行的神足}{569.0},比丘們!這些是四神足。」



\sutta{25}{25}{大果經第一}{https://agama.buddhason.org/SN/sn.php?keyword=51.25}
  「\twnr{比丘}{31.0}們!有這些\twnr{四神足}{503.1},哪四個?比丘們!這裡,比丘\twnr{修習}{94.0}\twnr{具備意欲定勤奮之行的神足}{568.0};活力定……(中略)心定……(中略)修習\twnr{具備考察定勤奮之行的神足}{569.0},比丘們!這些是四神足。

  比丘們!以這些四神足的\twnr{已自我修習}{658.0}、已自我\twnr{多作}{95.0},二果其中之一果能被比丘預期:\twnr{當生}{42.0}\twnr{完全智}{489.0},或在存在\twnr{有餘依}{323.0}時,為\twnr{阿那含}{209.0}位。」



\sutta{26}{26}{大果經第二}{https://agama.buddhason.org/SN/sn.php?keyword=51.26}
  「\twnr{比丘}{31.0}們!有這些\twnr{四神足}{503.1},哪四個?比丘們!這裡,比丘\twnr{修習}{94.0}\twnr{具備意欲定勤奮之行的神足}{568.0};活力定……(中略)心定……(中略)修習\twnr{具備考察定勤奮之行的神足}{569.0},比丘們!這些是四神足。

  比丘們!以這些四神足的\twnr{已自我修習}{658.0}、已自我\twnr{多作}{95.0},七果、七效益能被預期,哪七果、七效益呢?

  在當生提前達成\twnr{完全智}{489.0}。

  如果在當生未提前達成完全智,那麼在死時達成完全智。

  如果在當生未到達完全智,如果在死時未達成完全智,則以\twnr{五下分結}{134.0}的滅盡,成為\twnr{中般涅槃}{297.0}者、為\twnr{生般涅槃}{298.0}者、為\twnr{無行般涅槃}{299.0}者、為\twnr{有行般涅槃}{300.0}者、為\twnr{上流到阿迦膩吒}{301.0}者。

  比丘們!以這些四神足已自我修習、已自我多作,這些七果、七效益能被預期。」



\sutta{27}{27}{阿難經第一}{https://agama.buddhason.org/SN/sn.php?keyword=51.27}
  起源於舍衛城。

  那時,\twnr{尊者}{200.0}阿難去見\twnr{世尊}{12.0}。抵達後,向世尊\twnr{問訊}{46.0}後,在一旁坐下。在一旁坐下的尊者阿難對世尊說這個:

  「\twnr{大德}{45.0}!什麼是神通?什麼是\twnr{神足}{503.1}?什麼是神足的\twnr{修習}{94.0}?以及什麼是導向神足的修習道跡呢?」

  「阿難!這裡,\twnr{比丘}{31.0}體驗各種神通種類:是一個後他變成多個……(中略)以身體行使自在直到梵天世界,阿難!這被稱為神通。

  阿難!而什麼是神足呢?阿難!凡道、凡道跡轉起神通之得到、神通之獲得者,阿難!這被稱為神足。

  阿難!而什麼是神足的修習呢?阿難!這裡,比丘修習\twnr{具備意欲定勤奮之行的神足}{568.0};活力定……(中略)心定……(中略)修習\twnr{具備考察定勤奮之行的神足}{569.0}。阿難!這被稱為神足的修習。

  阿難!而什麼是導向神足的修習道跡呢?就是\twnr{八支聖道}{525.0},即:正見……(中略)正定。阿難!這被稱為導向神足的修習道跡。」[\suttaref{SN.51.19}]



\sutta{28}{28}{阿難經第二}{https://agama.buddhason.org/SN/sn.php?keyword=51.28}
  \twnr{世尊}{12.0}對在一旁坐下的\twnr{尊者}{200.0}阿難說這個:

  「阿難!什麼是神通?什麼是\twnr{神足}{503.1}?什麼是神足的\twnr{修習}{94.0}?以及什麼是導向神足的修習道跡呢?」

  「\twnr{大德}{45.0}!我們的法以世尊為根本……(中略)。」 

  「阿難!這裡,\twnr{比丘}{31.0}體驗各種神通種類:是一個後他變成多個……(中略)以身體行使自在直到梵天世界,阿難!這被稱為神通。

  阿難!而什麼是神足呢?阿難!凡道、凡道跡轉起神通之得到、神通之獲得者,阿難!這被稱為神足。

  阿難!而什麼是神足的修習呢?阿難!這裡,比丘修習\twnr{具備意欲定勤奮之行的神足}{568.0};活力定……(中略)心定……(中略)修習\twnr{具備考察定勤奮之行的神足}{569.0}。阿難!這被稱為神足的修習。

  阿難!而什麼是導向神足的修習道跡呢?就是\twnr{八支聖道}{525.0},即:正見……(中略)正定。阿難!這被稱為導向神足的修習道跡。」[\suttaref{SN.51.19}]



\sutta{29}{29}{比丘經第一}{https://agama.buddhason.org/SN/sn.php?keyword=51.29}
  那時,眾多比丘去見\twnr{世尊}{12.0}。抵達後,向世尊\twnr{問訊}{46.0}後,在一旁坐下。在一旁坐下的那些比丘對世尊說這個:

  「\twnr{大德}{45.0}!什麼是神通?什麼是\twnr{神足}{503.1}?什麼是神足的\twnr{修習}{94.0}?以及什麼是導向神足的修習道跡呢?」

  「比丘們!這裡,比丘體驗各種神通種類:是一個後他變成多個……(中略)以身體行使自在直到梵天世界,比丘們!這被稱為神通。

  比丘們!而什麼是神足呢?比丘們!凡道、凡道跡轉起神通之得到、神通之獲得者,比丘們!這被稱為神足。

  比丘們!而什麼是神足的修習?比丘們!這裡,比丘修習\twnr{具備意欲定勤奮之行的神足}{568.0};活力定……(中略)心定……(中略)修習\twnr{具備考察定勤奮之行的神足}{569.0}。比丘們!這被稱為神足的修習。

  比丘們!而什麼是導向神足的修習道跡呢?就是\twnr{八支聖道}{525.0},即:正見……(中略)正定。比丘們!這被稱為導向神足的修習道跡。」[\suttaref{SN.51.19}]



\sutta{30}{30}{比丘經第二}{https://agama.buddhason.org/SN/sn.php?keyword=51.30}
  那時,眾多比丘去見\twnr{世尊}{12.0}。……(中略)世尊對在一旁坐下的那些比丘說這個:

  「比丘們!而什麼是神通?什麼是\twnr{神足}{503.1}?什麼是神足的\twnr{修習}{94.0}?以及什麼是導向神足的修習道跡呢?」

  「\twnr{大德}{45.0}!我們的法以世尊為根本……(中略)。」 

  「比丘們!而什麼是神通?比丘們!這裡,比丘體驗各種神通種類:是一個後他變成多個……(中略)以身體行使自在直到梵天世界,比丘們!這被稱為神通。

  比丘們!而什麼是神足呢?比丘們!凡道、凡道跡轉起神通之得到、神通之獲得者,比丘們!這被稱為神足。

  比丘們!而什麼是神足的修習?比丘們!這裡,比丘修習\twnr{具備意欲定勤奮之行的神足}{568.0};活力定……(中略)心定……(中略)修習\twnr{具備考察定勤奮之行的神足}{569.0}。比丘們!這被稱為神足的修習。

  比丘們!而什麼是導向神足的修習道跡呢?就是\twnr{八支聖道}{525.0},即:正見……(中略)正定。比丘們!這被稱為導向神足的修習道跡。」[\suttaref{SN.51.19}]



\sutta{31}{31}{目揵連經}{https://agama.buddhason.org/SN/sn.php?keyword=51.31}
  在那裡,世尊召喚\twnr{比丘}{31.0}們:

  「比丘們!你們怎麼想它:以什麼法的\twnr{已自我修習}{658.0}、已自我多作,目揵連比丘有這麼\twnr{大神通力}{405.0}、這麼大威力呢?」

  「\twnr{大德}{45.0}!我們的法以世尊為根本……(中略)。」

  「比丘們!以\twnr{四神足}{503.1}的已自我\twnr{修習}{94.0}、已自我\twnr{多作}{95.0},目揵連比丘有這麼大神通力、這麼大威力,哪四個?比丘們!這裡,目揵連比丘修習\twnr{具備意欲定勤奮之行的神足}{568.0}:像這樣,我的意欲將不是過鬆的、將不是過緊的、將不是向內收斂的、將不是向外散亂的,以及住於前後有感知的:後如前那樣地,前如後那樣地;上如下那樣地,下如上那樣地;在夜間如在白天那樣地,在白天如在夜間那樣地。像這樣,以打開的、無覆蓋的心,修習有光輝的心;活力定……(中略)心定……(中略)修習\twnr{具備考察定勤奮之行的神足}{569.0}:像這樣,我的考察將不過鬆,也不過緊;不內斂,也不外散,以及住於前後有感知的:後如前那樣地,前如後那樣地;上如下那樣地,下如上那樣地;在夜間如在白天那樣地,在白天如在夜間那樣地。像這樣,以打開的、無覆蓋的心,修習有光輝的心。比丘們!以這四神足的已自我修習、已自我多作,目揵連比丘有這麼大神通力、這麼大威力。

  比丘們!而且,以這些四神足已自我修習、已自我多作,目揵連比丘體驗各種神通種類:是一個後變成多個,又,是多個後變成一個……(中略)以身體行使自在直到梵天世界。……(中略)比丘們!而且,以這些四神足已自我修習、已自我多作,目揵連比丘以諸\twnr{漏}{188.0}的滅盡,\twnr{當生}{42.0}以證智自作證後,\twnr{進入後住於}{66.0}無漏\twnr{心解脫}{16.0}、\twnr{慧解脫}{539.0}。」



\sutta{32}{32}{如來經}{https://agama.buddhason.org/SN/sn.php?keyword=51.32}
  在那裡,世尊召喚\twnr{比丘}{31.0}們:

  「比丘們!你們怎麼想它:以什麼法的\twnr{已自我修習}{658.0}、已自我\twnr{多作}{95.0},如來有這麼\twnr{大神通力}{405.0}、這麼大威力呢?」

  「\twnr{大德}{45.0}!我們的法以世尊為根本……(中略)。」

  「比丘們!以\twnr{四神足}{503.1}的已自我修習、已自我多作,如來有這麼大神通力、這麼大威力,哪四個?比丘們!這裡,如來修習\twnr{具備意欲定勤奮之行的神足}{568.0}:像這樣,我的意欲將不是過鬆的、將不是過緊的、將不是向內收斂的、將不是向外散亂的,以及住於前後有感知的:後如前那樣地,前如後那樣地;上如下那樣地,下如上那樣地;在夜間如在白天那樣地,在白天如在夜間那樣地。像這樣,以打開的、無覆蓋的心,修習有光輝的心;活力定……(中略)心定……(中略)修習\twnr{具備考察定勤奮之行的神足}{569.0}:像這樣,我的考察將不過鬆,也不過緊;不內斂,也不外散,以及住於前後有感知的:後如前那樣地,前如後那樣地;上如下那樣地,下如上那樣地;在夜間如在白天那樣地,在白天如在夜間那樣地。像這樣,以打開的、無覆蓋的心,修習有光輝的心。比丘們!以這些四神足的已自我修習、已自我多作,如來有這麼大神通力、這麼大威力。

  比丘們!而且,以這些四神足已自我修習、已自我多作,如來體驗各種神通種類:是一個後變成多個,又,是多個後變成一個……(中略)以身體行使自在直到梵天世界。……(中略)比丘們!而且,以這些四神足已自我修習、已自我多作,如來以諸\twnr{漏}{188.0}的滅盡,\twnr{當生}{42.0}以證智自作證後,\twnr{進入後住於}{66.0}無漏\twnr{心解脫}{16.0}、\twnr{慧解脫}{539.0}。」

   (六證智都應該使之被細說)

  鐵球品第三,其\twnr{攝頌}{35.0}:

  「道、鐵球、比丘,概要與二則果,

   二則阿難、二則比丘,目揵連、如來。」





\pin{恒河中略品}{33}{76}
\sutta{33}{44}{恒河等經十二則}{https://agama.buddhason.org/SN/sn.php?keyword=51.33}
  「\twnr{比丘}{31.0}們!猶如恒河是傾向東的、斜向東的、坡斜向東的。同樣的,比丘們!\twnr{修習}{94.0}\twnr{四神足}{503.1}、\twnr{多作}{95.0}四神足的比丘是傾向涅槃的、斜向涅槃的、坡斜向涅槃的。

  比丘們!而怎樣修習四神足、多作四神足的比丘是傾向涅槃的、斜向涅槃的、坡斜向涅槃的?比丘們!這裡,比丘修習\twnr{具備意欲定勤奮之行的神足}{568.0};活力定……(中略)心定……(中略)修習\twnr{具備考察定勤奮之行的神足}{569.0}。比丘們!這樣修習四神足、多作四神足的比丘是傾向涅槃的、斜向涅槃的、坡斜向涅槃的。」[按:全品應如\suttaref{SN.45.91}-102那樣]

  恒河中略品第四,其\twnr{攝頌}{35.0}:

  「六則傾向東的,與六則傾向大海的,

   這兩個六則成十二則,以那個被稱為品。」

  不放逸品應該使之被細說[按:如\suttaref{SN.45.139}-148那樣], 其攝頌:

  「如來、足跡、屋頂,根、樹心、茉莉花, 

   王、月、日,以衣服為第十句。」 

  力量所作品應該使之被細說[按:如\suttaref{SN.45.149}-160那樣],其攝頌: 

  「力量、種子與龍,樹木、瓶子及穗, 

   虛空與二則雨雲,船、屋舍、河。」

  尋求品應該使之被細說[按:如\suttaref{SN.45.161}-171],其攝頌: 

  「尋求、慢、漏,有與三苦性, 

   荒蕪、垢、惱亂,受、渴愛、渴望。」



\sutta{45}{76}{}{https://agama.buddhason.org/SN/sn.php?keyword=51.45}
  (略去)  





\pin{暴流品}{77}{86}
\sutta{77}{86}{暴流經十則}{https://agama.buddhason.org/SN/sn.php?keyword=51.77}
  「\twnr{比丘}{31.0}們!有這些五上分結,哪五個?色貪、無色貪、慢、掉舉、\twnr{無明}{207.0},比丘們!這些是五上分結。比丘們!為了這些五上分結的證智、\twnr{遍知}{154.0}、遍盡、捨斷,這\twnr{四神足}{503.1}應該被\twnr{修習}{94.0}。比丘們!哪四個?比丘們!這裡,比丘修習\twnr{具備意欲定勤奮之行的神足}{568.0};活力定……(中略)心定……(中略)修習\twnr{具備考察定勤奮之行的神足}{569.0}。比丘們!為了這些五上分結的證智、遍知、遍盡、捨斷,這些四神足應該被修習。」(應該如\twnr{道相應}{x628}那樣使之被細說)

  暴流品第八,其\twnr{攝頌}{35.0}:

  「暴流、軛、取,繫縛、煩惱潛在趨勢,

   欲種類、蓋,蘊、下上分。」

  神足相應第七。





\page

\xiangying{52}{阿那律相應}
\pin{獨處品}{1}{10}
\sutta{1}{1}{獨處經第一}{https://agama.buddhason.org/SN/sn.php?keyword=52.1}
  \twnr{被我這麼聽聞}{1.0}:

  \twnr{有一次}{2.0},\twnr{尊者}{200.0}阿那律住在舍衛城祇樹林給孤獨園。

  那時,當\twnr{尊者}{200.0}阿那律獨處、\twnr{獨坐}{92.0}時,這樣心的深思生起:

  「凡任何已錯失\twnr{四念住}{286.0}者,他們導向\twnr{苦的完全滅盡}{181.0}的聖道已錯失;凡任何已發動四念住者,他們導向苦的完全滅盡的聖道已發動。」

  那時,尊者大目揵連以心了知尊者阿那律心中的深思後,就猶如有力氣的男子伸直彎曲的手臂,或彎曲伸直的手臂,就像這樣在尊者阿那律的面前出現。

  那時,尊者大目揵連對尊者阿那律說這個:

  「阿那律\twnr{學友}{201.0}!什麼情形\twnr{比丘}{31.0}的四念住已被發動呢?」

  「學友!這裡,比丘住於在自身內的身上\twnr{隨看著}{59.0}\twnr{集法}{67.1},住於在自己的身上隨看著\twnr{消散法}{155.0},住於在自身內的身上隨看著集與消散法:熱心的、正知的、有念的,調伏世間中的\twnr{貪婪}{435.0}、憂後;住於在外部的身上隨看著集法,住於在外部的身上隨看著消散法,住於在外部的身上隨看著集與消散法:熱心的、正知的、有念的,調伏世間中的貪婪、憂後;住於在內外的身上隨看著集法,住於在內外的身上隨看著消散法,住於在內外的身上隨看著集與消散法:熱心的、正知的、有念的,調伏世間中的貪婪、憂後。

  如果他希望『願我在無\twnr{厭逆}{227.0}上住於有厭逆想的』,在那裡,他住於有厭逆想的。

  如果他希望『願我在厭逆上住於無厭逆想的』,在那裡,他住於無厭逆想的。

  如果他希望『願我在不厭逆與厭逆上都住於厭逆想』,在那裡,他住於有厭逆想的。

  如果他希望『願我在厭逆與不厭逆上都住於不厭逆想』,在那裡,他住於無厭逆想的。

  如果他希望『願我在無厭逆與厭逆兩者上都避免後,住於\twnr{平靜}{228.0}的、具念的、正知的』,在那裡,他住於平靜的、具念的、正知的。

  住於在自身內的諸受上隨看著集法,住於在自身內的諸受上隨看著消散法,住於在自身內的諸受上隨看著集與消散法:熱心的、正知的、有念的,調伏世間中的貪婪、憂後;住於在外部的諸受上隨看著集法,住於在外部的諸受上隨看著消散法,住於在外部的諸受上隨看著集與消散法:熱心的、正知的、有念的,調伏世間中的貪婪、憂後;住於在內外的諸受上隨看著集法,住於在內外的諸受上隨看著消散法,住於在內外的諸受上隨看著集與消散法:熱心的、正知的、有念的,調伏世間中的貪婪、憂後。

  如果他希望『願我在無厭逆上住於有厭逆想的』,在那裡,他住於有厭逆想的。

  如果他希望『願我在厭逆上住於無厭逆想的』,在那裡,他住於無厭逆想的。

  如果他希望『願我在不厭逆與厭逆上都住於厭逆想』,在那裡,他住於有厭逆想的。

  如果他希望『願我在厭逆與不厭逆上都住於不厭逆想』,在那裡,他住於無厭逆想的。

  如果他希望『願我在無厭逆與厭逆兩者上都避免後,住於平靜,是具念的、正知的』,在那裡,他住於平靜的、具念的、正知的。

  在自己的心上……(中略)在外部的心上……(中略)住於在內外的心上隨看著集法,住於在內外的心上隨看著消散法,住於在內外的心上隨看著集與消散法,熱心……(中略)貪婪、憂。

  如果他希望『願我在無厭逆上住於有厭逆想的』,在那裡,他住於有厭逆想的。……(中略)在那裡,他住於平靜的、具念的、正知的。

  在自己的諸法上……(中略)在外部的諸法上……(中略)住於在內外的諸法上隨看著集法,住於在內外的諸法上隨看著消散法,住於在內外的諸法上隨看著集與消散法:熱心的、正知的、有念的,調伏世間中的貪婪、憂後。

  如果他希望『願我在無厭逆上住於有厭逆想的』,在那裡,他住於有厭逆想的。……(中略)在那裡,他住於平靜的、具念的、正知的。

  學友!這個情形比丘的四念住已被發動。」



\sutta{2}{2}{獨處經第二}{https://agama.buddhason.org/SN/sn.php?keyword=52.2}
  起源於舍衛城。

  那時,當\twnr{尊者}{200.0}阿那律獨處、\twnr{獨坐}{92.0}時,這樣心的深思生起:

  「凡任何已錯失\twnr{四念住}{286.0}者,他們導向\twnr{苦的完全滅盡}{181.0}的聖道已錯失;凡任何已發動四念住者,他們導向苦的完全滅盡的聖道已發動。」

  那時,尊者大目揵連以心了知尊者阿那律心中的深思後,就猶如有力氣的男子伸直彎曲的手臂,或彎曲伸直的手臂,就像這樣在尊者阿那律的面前出現。

  那時,尊者大目揵連對尊者阿那律說這個:

  「阿那律\twnr{學友}{201.0}!什麼情形\twnr{比丘}{31.0}的四念住已被發動呢?」

  「學友!這裡,比丘住於在自身內的身上隨看著身:熱心的、正知的、有念的,調伏世間中的\twnr{貪婪}{435.0}、憂後;住於在外部的身上隨看著身……(中略)住於\twnr{在內外的身上隨看著身}{271.0}:熱心的、正知的、有念的,調伏世間中的貪婪、憂後。

  住於在自身內的諸受上隨看著受:熱心的、正知的、有念的,調伏世間中的貪婪、憂後;住於在外部的諸受上隨看著受:熱心的、正知的、有念的,調伏世間中的貪婪、憂後;住於在內外的諸受上隨看著受:熱心的、正知的、有念的,調伏世間中的貪婪、憂後。

  在自己的心上……(中略)在外部的心上……(中略)住於在內外的心上隨看著心:熱心的、正知的、有念的,調伏世間中的貪婪、憂後。

  在自己的諸法上……(中略)在外部的諸法上……(中略)住於在內外的諸法上隨看著法:熱心的、正知的、有念的,調伏世間中的貪婪、憂後。

  學友!這個情形比丘的四念住已被發動。」



\sutta{3}{3}{殊達奴經}{https://agama.buddhason.org/SN/sn.php?keyword=52.3}
  \twnr{有一次}{2.0},\twnr{尊者}{200.0}阿那律住在舍衛城殊達奴河邊。

  那時,眾多\twnr{比丘}{31.0}去見尊者阿那律。抵達後,與尊者阿那律一起互相問候。交換應該被互相問候的友好交談後,在一旁坐下。在一旁坐下的那些比丘對尊者阿那律說這個: 

  「尊者阿那律以什麼法的\twnr{已自我修習}{658.0}、已自我\twnr{多作}{95.0},已得到\twnr{大通智}{564.0}呢?」

  「\twnr{學友}{201.0}們!我以\twnr{四念住}{286.0}的已自我修習、已自我多作,已得到大通智,哪四個?學友們!這裡,我住於\twnr{在身上隨看著身}{176.0}:熱心的、正知的、有念的,調伏世間中的\twnr{貪婪}{435.0}、憂後;在諸受上……(中略)在心上……(中略)我住於在諸法上隨看著法:熱心的、正知的、有念的,調伏世間中的貪婪、憂後。學友們!以這些四念住的已自我修習、已自我多作,我已得到大通智。

  學友!而且,我以這些四念住的已自我修習、已自我多作,證知下劣法為下劣的;證知中等法為中等的;證知勝妙法為勝妙的。」



\sutta{4}{4}{荊棘(林)經第一}{https://agama.buddhason.org/SN/sn.php?keyword=52.4}
  \twnr{有一次}{2.0},\twnr{尊者}{200.0}阿那律、尊者舍利弗、尊者大目揵連住在娑雞多城荊棘林。

  那時,尊者舍利弗、尊者大目揵連傍晚時,從\twnr{獨坐}{92.0}出來,去見尊者阿那律。抵達後,與尊者阿那律一起互相問候。交換應該被互相問候的友好交談後,在一旁坐下。在一旁坐下的尊者舍利弗對尊者阿那律說這個:

  「阿那律\twnr{學友}{201.0}!哪些法應該被\twnr{有學}{193.0}\twnr{比丘}{31.0}進入後而住呢?」

  「舍利弗學友!\twnr{四念住}{286.0}應該被有學比丘進入後而住,哪四個?學友!這裡,比丘住於\twnr{在身上隨看著身}{176.0}:熱心的、正知的、有念的,調伏世間中的\twnr{貪婪}{435.0}、憂後;在諸受上……(中略)在心上……(中略)住於在諸法上隨看著法:熱心的、正知的、有念的,調伏世間中的貪婪、憂後,舍利弗學友!這些四念住應該被有學比丘進入後而住。」



\sutta{5}{5}{荊棘(林)經第二}{https://agama.buddhason.org/SN/sn.php?keyword=52.5}
  起源於娑雞多城。

  在一旁坐下的\twnr{尊者}{200.0}舍利弗對尊者阿那律說這個:

  「阿那律\twnr{學友}{201.0}!哪些法應該被\twnr{無學}{193.1}\twnr{比丘}{31.0}進入後而住呢?」

  「舍利弗學友!\twnr{四念住}{286.0}應該被無學比丘進入後而住呢,哪四個?學友!這裡,比丘住於\twnr{在身上隨看著身}{176.0}:熱心的、正知的、有念的,調伏世間中的\twnr{貪婪}{435.0}、憂後;在諸受上……(中略)在心上……(中略)住於在諸法上隨看著法:熱心的、正知的、有念的,調伏世間中的貪婪、憂後,舍利弗學友!這些四念住應該被無學比丘進入後而住。」



\sutta{6}{6}{荊棘(林)經第三}{https://agama.buddhason.org/SN/sn.php?keyword=52.6}
  起源於娑雞多城。

  在一旁坐下的\twnr{尊者}{200.0}舍利弗對尊者阿那律說這個:

  「尊者阿那律以什麼法的\twnr{已自我修習}{658.0}、已自我\twnr{多作}{95.0},已得到\twnr{大通智}{564.0}呢?」

  「\twnr{學友}{201.0}!我以\twnr{四念住}{286.0}的已自我修習、已自我多作,已得到大通智,哪四個?學友!這裡,我住於\twnr{在身上隨看著身}{176.0}:熱心的、正知的、有念的,調伏世間中的\twnr{貪婪}{435.0}、憂後;在諸受上……(中略)在心上……(中略)我住於在諸法上隨看著法:熱心的、正知的、有念的,調伏世間中的貪婪、憂後。學友!我以這些四念住的已自我修習、已自我多作,已得到大通智。

  學友!而且,我以這些四念住的已自我修習、已自我多作,\twnr{證知}{242.0}千世界。」



\sutta{7}{7}{渴愛之滅盡經}{https://agama.buddhason.org/SN/sn.php?keyword=52.7}
  起源於舍衛城。

  在那裡,\twnr{尊者}{200.0}阿那律召喚\twnr{比丘}{31.0}們:

  「比丘\twnr{學友}{201.0}們!」

  「學友!」那些比丘回答尊者阿那律。

  尊者阿那律說這個:

  「學友們!有這些\twnr{四念住}{286.0},已\twnr{修習}{94.0}、已\twnr{多作}{95.0},轉起渴愛之滅盡,哪四個?學友們!這裡,比丘住於\twnr{在身上隨看著身}{176.0}:熱心的、正知的、有念的,調伏世間中的\twnr{貪婪}{435.0}、憂後;在諸受上……(中略)在心上……(中略)住於在諸法上隨看著法:熱心的、正知的、有念的,調伏世間中的貪婪、憂後。學友們!這些是四念住,已修習、已多作,轉起渴愛之滅盡。」



\sutta{8}{8}{撒拉拉之小屋經}{https://agama.buddhason.org/SN/sn.php?keyword=52.8}
  \twnr{有一次}{2.0},\twnr{尊者}{200.0}阿那律住在舍衛城撒拉拉之小屋。

  在那裡,尊者阿那律召喚\twnr{比丘}{31.0}們:……(中略)說這個:

  「學友們!猶如恒河是傾向東的、斜向東的、坡斜向東的,那時,大群人拿鋤頭、籃子後:『我們將轉(作)這恒河成傾向西的、斜向西的、坡斜向西的。』學友們!你們怎麼想它:是否那個大群人會轉這恒河成傾向西的、斜向西的、坡斜向西的呢?」

  「\twnr{學友}{201.0}!這確實不是,那是什麼原因?學友!恒河是傾向東的、斜向東的、坡斜向東的,不容易轉成傾向西的、斜向西的、坡斜向西的,還有,大群人最終只會是疲勞的、苦惱的\twnr{有分者}{876.0}。」

  「同樣的,學友們!如果國王,或國王的大臣,或朋友,或同事,或親族,或血親以財富帶來後邀請\twnr{修習}{94.0}\twnr{四念住}{286.0}、\twnr{多作}{95.0}四念住的比丘:『喂!來!男子!為何讓這些袈裟耗盡你?為何你實行光頭、鉢?來!還俗後請你在財富上受用與作福德。』學友們!確實,那位修習四念住、多作四念住的比丘,『他放棄學後將還俗。』\twnr{這不存在可能性}{650.0},那是什麼原因?學友們!因為那顆心長久是傾向遠離的、斜向遠離的、坡斜向遠離的,『他確實將還俗。』這不存在可能性。

  學友們!而怎樣比丘修習四念住、多作四念住呢?學友!這裡,比丘住於\twnr{在身上隨看著身}{176.0},在受上……(中略)在心上……(中略)住於在諸法上隨看著法:熱心的、正知的、有念的,調伏世間中的\twnr{貪婪}{435.0}、憂後,學友!比丘這樣修習四念住、多作四念住。」



\sutta{9}{9}{蓭婆巴利的園林經}{https://agama.buddhason.org/SN/sn.php?keyword=52.9}
  \twnr{有一次}{2.0},\twnr{尊者}{200.0}阿那律與尊者舍利弗住在毘舍離蓭婆巴利的園林。

  那時,尊者舍利弗傍晚時,從\twnr{獨坐}{92.0}出來……(中略)在一旁坐下的尊者舍利弗對尊者阿那律說這個:

  「阿那律\twnr{學友}{201.0}!你的諸根明淨,臉色清淨、皎潔,現在尊者阿那律多以什麼住處住呢?」

  「學友!我現在多住於在\twnr{四念住}{286.0}上心善建立,哪四個?學友!這裡,我住於\twnr{在身上隨看著身}{176.0}:熱心的、正知的、有念的,調伏世間中的\twnr{貪婪}{435.0}、憂後;在諸受上……(中略)在心上……(中略)住於在諸法上隨看著法:熱心的、正知的、有念的,調伏世間中的貪婪、憂後,學友!我現在多住於在這些四念住上心善建立。

  學友!凡那位漏已滅盡的、已完成的、\twnr{應該被作的已作的}{20.0}、負擔已卸的、\twnr{自己的利益已達成的}{189.0}、\twnr{有之結已滅盡的}{190.0}、以\twnr{究竟智}{191.0}解脫的阿羅漢\twnr{比丘}{31.0},他多住於在這四念住中心善建立。」

  「學友!確實是我們的利得,確實是我們的善得的:當他說\twnr{如牛王之語}{675.0}時,我們就在尊者阿那律的面前聽聞了。」



\sutta{10}{10}{重病經}{https://agama.buddhason.org/SN/sn.php?keyword=52.10}
  \twnr{有一次}{2.0},生病的、受苦的、重病的\twnr{尊者}{200.0}阿那律住在舍衛城\twnr{盲者的樹林}{88.0}。

  那時,眾多\twnr{比丘}{31.0}去見尊者阿那律。抵達後,對尊者阿那律說這個:

  「當以什麼住處住時,尊者阿那律的已生起身體苦受不\twnr{持續遍取}{530.0}心呢?」

  「學友們!當住於在\twnr{四念住}{286.0}上心善建立時,我的已生起身體苦受不持續遍取心,哪四個?學友們!這裡,我住於\twnr{在身上隨看著身}{176.0}……(中略)在受上……(中略)在心上……(中略)我住於在諸法上隨看著法:熱心的、正知的、有念的,調伏世間中的\twnr{貪婪}{435.0}、憂後,學友們!當住於在四念住上心善建立時,我的已生起身體苦受不持續遍取心。」

  獨處品第一,其\twnr{攝頌}{35.0}:

  「以獨處二說,殊達奴、荊棘(林)三則,

   渴愛的滅盡、撒拉拉之小屋,蓭婆巴利與病。」





\pin{第二品}{11}{24}
\sutta{11}{11}{千劫經}{https://agama.buddhason.org/SN/sn.php?keyword=52.11}
  \twnr{有一次}{2.0},\twnr{尊者}{200.0}阿那律住在舍衛城祇樹林給孤獨園。

  那時,眾多\twnr{比丘}{31.0}去見尊者阿那律。抵達後,與尊者阿那律互相歡迎。……(中略)在一旁坐下的那些比丘對尊者阿那律說這個:

  「尊者阿那律以什麼法的\twnr{已自我修習}{658.0}、已自我\twnr{多作}{95.0},已得到\twnr{大通智}{564.0}呢?」

  「\twnr{學友}{201.0}們!我以\twnr{四念住}{286.0}的已自我修習、已自我多作,已得到大通智,哪四個?學友們!這裡,我住於\twnr{在身上隨看著身}{176.0}:熱心的、正知的、有念的,調伏世間中的\twnr{貪婪}{435.0}、憂後;在諸受上……(中略)在心上……(中略)我住於在諸法上隨看著法:熱心的、正知的、有念的,調伏世間中的貪婪、憂後。學友們!我以這些四念住的已自我修習、已自我多作,已得到大通智。

  學友們!而且,以這些四念住的已自我修習、已自我多作,我回憶千劫。」



\sutta{12}{12}{各種神通經}{https://agama.buddhason.org/SN/sn.php?keyword=52.12}
  「\twnr{學友}{201.0}們!我以這些\twnr{四念住}{286.0}的\twnr{已自我修習}{658.0}、已自我\twnr{多作}{95.0},體驗各種神通種類:是一個後變成多個,是多個後變成一個……(中略)以身體行使自在直到梵天世界。」



\sutta{13}{13}{天耳經}{https://agama.buddhason.org/SN/sn.php?keyword=52.13}
  「\twnr{學友}{201.0}們!我以這些\twnr{四念住}{286.0}的\twnr{已自我修習}{658.0}、已自我\twnr{多作}{95.0},以清淨、超越常人的天耳界聽到二者的聲音:『天與人,以及在遠處、近處。』」



\sutta{14}{14}{他心經}{https://agama.buddhason.org/SN/sn.php?keyword=52.14}
  「\twnr{學友}{201.0}們!我以這些\twnr{四念住}{286.0}的\twnr{已自我修習}{658.0}、已自我\twnr{多作}{95.0},對其他眾生、其他個人\twnr{以心熟知心後}{393.0}知道:有貪的心為『有貪的心』……(中略)知道未解脫的心為『未解脫的心』。」



\sutta{15}{15}{可能經}{https://agama.buddhason.org/SN/sn.php?keyword=52.15}
  「\twnr{學友}{201.0}們!我以這些\twnr{四念住}{286.0}的\twnr{已自我修習}{658.0}、已自我\twnr{多作}{95.0},如實知道可能為可能、\twnr{不可能為不可能}{683.0}。」



\sutta{16}{16}{業之受持經}{https://agama.buddhason.org/SN/sn.php?keyword=52.16}
  「\twnr{學友}{201.0}們!我以這些\twnr{四念住}{286.0}的\twnr{已自我修習}{658.0}、已自我\twnr{多作}{95.0},從道理從原因如實知道過去、未來、現在業之\twnr{受持}{57.0}果報。」



\sutta{17}{17}{導向一切處經}{https://agama.buddhason.org/SN/sn.php?keyword=52.17}
  「\twnr{學友}{201.0}們!我以這些\twnr{四念住}{286.0}的\twnr{已自我修習}{658.0}、已自我\twnr{多作}{95.0},如實知道導向一切處之行道(路徑)。」



\sutta{18}{18}{不同界經}{https://agama.buddhason.org/SN/sn.php?keyword=52.18}
  「\twnr{學友}{201.0}們!我以這些\twnr{四念住}{286.0}的\twnr{已自我修習}{658.0}、已自我\twnr{多作}{95.0},如實知道種種界、不同界的世間。」



\sutta{19}{19}{不同志向經}{https://agama.buddhason.org/SN/sn.php?keyword=52.19}
  「\twnr{學友}{201.0}們!我以這些\twnr{四念住}{286.0}的\twnr{已自我修習}{658.0}、已自我\twnr{多作}{95.0},如實知道眾生的不同\twnr{志向}{257.0}之狀態。」



\sutta{20}{20}{根之優劣經}{https://agama.buddhason.org/SN/sn.php?keyword=52.20}
  「\twnr{學友}{201.0}們!我以這些\twnr{四念住}{286.0}的\twnr{已自我修習}{658.0}、已自我\twnr{多作}{95.0},如實知道其他眾生、其他個人的根之優劣。」



\sutta{21}{21}{禪等經}{https://agama.buddhason.org/SN/sn.php?keyword=52.21}
  「\twnr{學友}{201.0}們!我以這些\twnr{四念住}{286.0}的\twnr{已自我修習}{658.0}、已自我\twnr{多作}{95.0},如實知道禪、解脫、定、\twnr{等至}{129.0}的污染、明淨、出定。」



\sutta{22}{22}{前世住處經}{https://agama.buddhason.org/SN/sn.php?keyword=52.22}
  「\twnr{學友}{201.0}們!我以這些\twnr{四念住}{286.0}的\twnr{已自我修習}{658.0}、已自我\twnr{多作}{95.0},回憶(隨念)許多前世住處,即:一生、二生……(中略)像這樣,我回憶許多\twnr{有行相的、有境遇的}{500.0}前世住處。」



\sutta{23}{23}{天眼經}{https://agama.buddhason.org/SN/sn.php?keyword=52.23}
  「\twnr{學友}{201.0}!又,我以這些\twnr{四念住}{286.0}的\twnr{已自我修習}{658.0}、已自我\twnr{多作}{95.0},以清淨、超越常人的天眼,看見當眾生死時、往生時……(中略)這樣,我以清淨、超越常人的天眼看見死沒往生的眾生:下劣的、勝妙的,美的、醜的,善去的、惡去的,知道依業到達的眾生。」



\sutta{24}{24}{漏的滅盡經}{https://agama.buddhason.org/SN/sn.php?keyword=52.24}
  「\twnr{學友}{201.0}們!而且,我以這些\twnr{四念住}{286.0}的\twnr{已自我修習}{658.0}、已自我\twnr{多作}{95.0},以諸\twnr{漏}{188.0}的滅盡,\twnr{當生}{42.0}以證智自作證後,\twnr{進入後住於}{66.0}無漏\twnr{心解脫}{16.0}、\twnr{慧解脫}{539.0}。」

  第二品,其\twnr{攝頌}{35.0}:

  「大通智、神通、天耳,他心、可能、業,

   一切處、界、志向,根、禪、三則明。」

  阿那律相應第八。





\page

\xiangying{53}{禪相應}
\pin{恒河中略品}{1}{44}
\sutta{1}{12}{禪等經十二則}{https://agama.buddhason.org/SN/sn.php?keyword=53.1}
  起源於舍衛城。

  「\twnr{比丘}{31.0}們!有這些四禪,哪四個?比丘們!這裡,比丘就從離諸欲後,從離諸不善法後,\twnr{進入後住於}{66.0}有尋、\twnr{有伺}{175.0},\twnr{離而生喜、樂}{174.0}的初禪;從尋與伺的平息,\twnr{自身內的明淨}{256.0},\twnr{心的專一性}{255.0},進入後住於無尋、無伺,定而生喜、樂的第二禪;從喜的\twnr{褪去}{77.0}、住於\twnr{平靜}{228.0}、有念正知、以身體感受樂,進入後住於凡聖者們告知『他是平靜者、具念者、\twnr{安樂住者}{317.0}』的第三禪;從樂的捨斷與從苦的捨斷,就在之前諸喜悅、憂的滅沒,進入後住於不苦不樂,\twnr{平靜、念遍純淨}{494.0}的第四禪,比丘們!這些是四禪。

  比丘們!猶如恒河是傾向東的、斜向東的、坡斜向東的。同樣的,比丘們!\twnr{修習}{94.0}四禪、\twnr{多作}{95.0}四禪的比丘是傾向涅槃的、斜向涅槃的、坡斜向涅槃的。

  比丘們!而怎樣修習四禪、多作四禪的比丘是傾向涅槃的、斜向涅槃的、坡斜向涅槃的?比丘們!這裡,比丘就從離諸欲後,從離諸不善法後,進入後住於有尋、有伺,離而生喜、樂的初禪;從尋與伺的平息……(中略)第二禪……(中略)的第三禪……(中略)第四禪。比丘們!這樣修習四禪、多作四禪的比丘是傾向涅槃的、斜向涅槃的、坡斜向涅槃的。」[按:全品應如\suttaref{SN.45.139}-148那樣]

  恒河中略品第一,其\twnr{攝頌}{35.0}:

  「六則傾向東的,與六則傾向大海的,

   這兩個六則成十二則,以那個被稱為品。」

  不放逸品應該使之被細說[按:如\suttaref{SN.45.139}-148那樣], 其攝頌:

  「如來、足跡、屋頂,根、樹心、茉莉花, 

   王、月、日,以衣服為第十句。」 

  力量所作品應該使之被細說[按:如\suttaref{SN.45.149}-160那樣],其攝頌: 

  「力量、種子與龍,樹木、瓶子及穗, 

   虛空與二則雨雲,船、屋舍、河。」

  尋求品應該使之被細說[按:如\suttaref{SN.45.161}-171],其攝頌: 

  「尋求、慢、漏,有與三苦性, 

   荒蕪、垢、惱亂,受、渴愛、渴望。」



\sutta{13}{44}{}{https://agama.buddhason.org/SN/sn.php?keyword=53.13}
  (略去)  





\pin{暴流品}{45}{54}
\sutta{45}{54}{暴流等經}{https://agama.buddhason.org/SN/sn.php?keyword=53.45}
  「\twnr{比丘}{31.0}們!有這些五上分結,哪五個?色貪、無色貪、慢、掉舉、\twnr{無明}{207.0},比丘們!這些是五上分結。比丘們!為了這些五上分結的證智、\twnr{遍知}{154.0}、遍盡、捨斷,這些四禪應該被\twnr{修習}{94.0}。比丘們!哪四個?比丘們!這裡,比丘就從離諸欲後,從離諸不善法後,\twnr{進入後住於}{66.0}有尋、\twnr{有伺}{175.0},\twnr{離而生喜、樂}{174.0}的初禪;從尋與伺的平息,\twnr{自身內的明淨}{256.0},\twnr{心的專一性}{255.0},進入後住於無尋、無伺,定而生喜、樂的第二禪……(中略)的第三禪……(中略)第四禪。比丘們!為了這些五上分結的證智、遍知、遍盡、捨斷,這些四禪應該被修習。」(應該如\twnr{道相應}{x628}那樣使之被細說)

  暴流品第五,其\twnr{攝頌}{35.0}:

  「暴流、軛、取,繫縛、煩惱潛在趨勢,

   欲種類、蓋,蘊、下上分。」

  禪相應第九。





\page

\xiangying{54}{入出息相應}
\pin{一法品}{1}{10}
\sutta{1}{1}{一法經}{https://agama.buddhason.org/SN/sn.php?keyword=54.1}
  起源於舍衛城。

  在那裡……(中略)說這個:

  「\twnr{比丘}{31.0}們!一法已\twnr{修習}{94.0}、已\twnr{多作}{95.0},有大果、\twnr{大效益}{113.0},哪一法?\twnr{入出息念}{329.0}。

  比丘們!而怎樣入出息念已修習、怎樣已多作,有大果、大效益?比丘們!這裡,到\twnr{林野}{142.0}的,或到樹下的,或到空屋的比丘坐下,\twnr{盤腿}{240.0}、定置端直的身體、\twnr{建立面前的念後}{529.0},\twnr{他只具念地吸氣}{526.0}、只具念地呼氣:

  當吸氣長時,知道:『我吸氣長。』或當呼氣長時,知道:『我呼氣長。』

  當吸氣短時,知道:『我吸氣短。』或當呼氣短時,知道:『我呼氣短。』

  學習:『\twnr{經驗著一切身}{527.0},我將吸氣。』學習:『經驗著一切身,我將呼氣。』

  學習:『\twnr{使身行寧靜著}{528.0},我將吸氣。』學習:『使身行寧靜著,我將呼氣。』

  學習:『\twnr{經驗著喜}{754.1},我將吸氣。』學習:『經驗著喜,我將呼氣。』

  學習:『經驗著樂,我將吸氣。』學習:『經驗著樂,我將呼氣。』

  學習:『經驗著\twnr{心行}{230.0},我將吸氣。』學習:『經驗著心行,我將呼氣。』

  學習:『使心行寧靜著,我將吸氣。』學習:『使心行寧靜著,我將呼氣。』

  學習:『\twnr{經驗著心}{968.0},我將吸氣。』學習:『經驗著心,我將呼氣。』

  學習:『\twnr{使心喜悅著}{700.0},我將吸氣。』學習:『使心喜悅著,我將呼氣。』

  學習:『\twnr{集中著心}{701.0},我將吸氣。』學習:『集中著心,我將呼氣。』

  學習:『\twnr{使心解脫著}{702.0},我將吸氣。』學習:『使心解脫著,我將呼氣。』

  學習:『隨看著無常,我將吸氣。』學習:『隨看著無常,我將呼氣。』

  學習:『隨看著\twnr{離貪}{77.0},我將吸氣。』學習:『隨看著離貪,我將呼氣。』

  學習:『\twnr{隨看著滅}{703.0},我將吸氣。』學習:『隨看著滅,我將呼氣。』

  學習:『\twnr{隨看著斷念}{211.1},我將吸氣。』學習:『隨看著斷念,我將呼氣。』

  比丘們!這樣,入出息念已修習、這樣已多作,有大果、大效益。」



\sutta{2}{2}{覺支經}{https://agama.buddhason.org/SN/sn.php?keyword=54.2}
  「\twnr{比丘}{31.0}們!\twnr{入出息念}{329.0}已\twnr{修習}{94.0}、已\twnr{多作}{95.0},有大果、\twnr{大效益}{113.0}。比丘們!而怎樣入出息念已修習、怎樣已多作,有大果、大效益?比丘們!這裡,比丘與入出息念俱行,\twnr{依止遠離}{322.0}、依止離貪、依止滅、\twnr{捨棄的成熟}{221.0}修習\twnr{念覺支}{315.0}……(中略)與入出息念俱行,依止遠離、依止離貪、依止滅、捨棄的成熟修習\twnr{平靜覺支}{314.0}。比丘們!入出息念已修習、已多作,有大果、大效益。」



\sutta{3}{3}{概要經}{https://agama.buddhason.org/SN/sn.php?keyword=54.3}
  「\twnr{比丘}{31.0}們!\twnr{入出息念}{329.0}已\twnr{修習}{94.0}、已\twnr{多作}{95.0},有大果、\twnr{大效益}{113.0}。

  比丘們!而怎樣入出息念已修習、怎樣已多作,有大果、大效益呢?

  比丘們!這裡,到\twnr{林野}{142.0}的,或到樹下的,或到空屋的比丘坐下,\twnr{盤腿}{240.0}、定置端直的身體、\twnr{建立面前的念後}{529.0},\twnr{他只具念地吸氣}{526.0}、只具念地呼氣:

  ……(中略)

  學習:『\twnr{隨看著斷念}{211.1},我將吸氣。』學習:『隨看著斷念,我將呼氣。』

  比丘們!這樣,入出息念已修習、這樣已多作,有大果、大效益。」



\sutta{4}{4}{大果經第一}{https://agama.buddhason.org/SN/sn.php?keyword=54.4}
  「\twnr{比丘}{31.0}們!\twnr{入出息念}{329.0}\twnr{修習}{94.0}、已\twnr{多作}{95.0},有大果、\twnr{大效益}{113.0}。

  比丘們!而怎樣入出息念已修習、怎樣已多作,有大果、大效益呢?

  比丘們!這裡,到\twnr{林野}{142.0}的,或到樹下的,或到空屋的比丘坐下,\twnr{盤腿}{240.0}、定置端直的身體、\twnr{建立面前的念後}{529.0},\twnr{他只具念地吸氣}{526.0}、只具念地呼氣:

  ……(中略)

  學習:『\twnr{隨看著斷念}{211.1},我將吸氣。』學習:『隨看著斷念,我將呼氣。』

  比丘們!這樣,入出息念已修習、這樣已多作,有大果、大效益。

  比丘們!在這樣,入出息念已修習、這樣已多作時,二果其中之一果能被預期:當生\twnr{完全智}{489.0},或在存在\twnr{有餘依}{323.0}時,為\twnr{阿那含}{209.0}位。」



\sutta{5}{5}{大果經第二}{https://agama.buddhason.org/SN/sn.php?keyword=54.5}
  「\twnr{比丘}{31.0}們!\twnr{入出息念}{329.0}已\twnr{修習}{94.0}、已\twnr{多作}{95.0},有大果、\twnr{大效益}{113.0}。

  比丘們!而怎樣入出息念已修習、怎樣已多作,有大果、大效益呢?

  比丘們!這裡,到\twnr{林野}{142.0}的,或到樹下的,或到空屋的比丘坐下,\twnr{盤腿}{240.0}、定置端直的身體、\twnr{建立面前的念後}{529.0},\twnr{他只具念地吸氣}{526.0}、只具念地呼氣:

  ……(中略)

  學習:『\twnr{隨看著斷念}{211.1},我將吸氣。』學習:『隨看著斷念,我將呼氣。』

  比丘們!這樣,入出息念已修習、這樣已多作,有大果、大效益。

  比丘們!在入出息念這麼已修習、這麼已多作時,七果、七利益能被預期,哪七果、七利益呢?

  在當生提前達成\twnr{完全智}{489.0}。

  如果在當生未提前達成完全智,那麼在死時達成完全智。

  如果在當生未提前達成完全智,如果在死時未達成完全智,那麼以\twnr{五下分結}{134.0}的滅盡,成為\twnr{中般涅槃}{297.0}者。

  ……(中略)為\twnr{生般涅槃}{298.0}者。

  ……(中略)為\twnr{無行般涅槃}{299.0}者。

  ……(中略)為\twnr{有行般涅槃}{300.0}者。

  ……(中略)為\twnr{上流到阿迦膩吒}{301.0}者。

  比丘們!在這樣,入出息念已修習、這樣已多作時,這些七果、七利益能被預期。」



\sutta{6}{6}{阿梨瑟吒經}{https://agama.buddhason.org/SN/sn.php?keyword=54.6}
  起源於舍衛城。

  在那裡,世尊……(中略)說這個:

  「比丘們!你們修習入出息念嗎?」

  在這麼說時,\twnr{尊者}{200.0}阿梨瑟吒對世尊說這個:

  「\twnr{大德}{45.0}!我修習入出息念。」

  「阿梨瑟吒!那麼,如怎樣你修習入出息念的呢?」

  「大德!在過去諸欲上,欲的意欲被我捨斷;在未來諸欲上,欲的意欲被我驅離;在自身內外諸法上,\twnr{有對想}{331.0}被我排除,那個我只具念地吸氣、只具念地呼氣,大德!我這樣修習入出息念。」

  「阿梨瑟吒!我說:『這是入出息念,這非不是。』阿梨瑟吒!然而,有關入出息念以細說被圓滿,你要聽!你要\twnr{好好作意}{43.1}!我將說。」

  「是的,大德!」尊者阿梨瑟吒回答世尊。

  世尊說這個:

  「阿梨瑟吒!而入出息念怎樣被詳細地圓滿呢?阿梨瑟吒!這裡,到\twnr{林野}{142.0}的,或到樹下的,或到空屋的比丘坐下,盤腿、定置端直的身體、\twnr{建立面前的念後}{529.0},\twnr{他只具念地吸氣}{526.0}、只具念地呼氣:

  當吸氣長時,知道:『我吸氣長。』或當呼氣長時,知道:『我呼氣長。』

  ……(中略)

  學習:『\twnr{隨看著斷念}{211.1},我將吸氣。』學習:『隨看著斷念,我將呼氣。』

  阿梨瑟吒!入出息念這樣以細說被圓滿。」



\sutta{7}{7}{摩訶迦賓經}{https://agama.buddhason.org/SN/sn.php?keyword=54.7}
  起源於舍衛城。

  當時,\twnr{尊者}{200.0}摩訶迦賓坐在世尊的不遠處,\twnr{盤腿}{240.0}、定置端直的身體、\twnr{建立面前的念後}{529.0}。\twnr{世尊}{12.0}看見坐在不遠處盤腿、定置端直的身體、建立面前的念後的尊者摩訶迦賓。看見後,召喚\twnr{比丘}{31.0}們:

  「比丘們!你們看見這位比丘身體的動搖狀態或搖動狀態嗎?」

  「\twnr{大德}{45.0}!每當我們看見那位尊者在\twnr{僧團}{375.0}中坐,或獨自獨處坐,我們都沒看見那位比丘身體的動搖狀態或搖動狀態。」

  「比丘們!凡以那種定的\twnr{已自我修習}{658.0}、已自我\twnr{多作}{95.0},既沒有身體的動搖狀態或搖動狀態,也沒有心的動搖狀態或搖動狀態,比丘們!那位比丘是定的隨欲得到者、不困難得到者、無困難得到者。

  比丘們!而以什麼定的已自我修習、已自我多作,既沒有身體的動搖狀態或搖動狀態,也沒有心的動搖狀態或搖動狀態呢?比丘們!以\twnr{入出息念}{329.0}之定的已自我修習、已自我多作,既沒有身體的動搖狀態或搖動狀態,也沒有心的動搖狀態或搖動狀態。

  比丘們!而怎樣入出息念之定已修習、怎樣已多作,就沒有身體的動搖狀態或搖動狀態,沒有心的動搖狀態或搖動狀態呢?

  比丘們!這裡,到\twnr{林野}{142.0}的,或到樹下的,或到空屋的比丘坐下,盤腿、定置端直的身體、\twnr{建立面前的念後}{529.0},\twnr{他只具念地吸氣}{526.0}、只具念地呼氣:

  ……(中略)

  學習:『\twnr{隨看著斷念}{211.1},我將吸氣。』學習:『隨看著斷念,我將呼氣。』

  比丘們!在這樣,入出息念之定已修習、這樣已多作時,就沒有身體的動搖狀態或搖動狀態,沒有心的動搖狀態或搖動狀態。」



\sutta{8}{8}{如燈經}{https://agama.buddhason.org/SN/sn.php?keyword=54.8}
  「\twnr{比丘}{31.0}們!\twnr{入出息念}{329.0}之定已\twnr{修習}{94.0}、已\twnr{多作}{95.0},有大果、\twnr{大效益}{113.0}。

  比丘們!而怎樣入出息念之定已修習、怎樣已多作,有大果、大效益?比丘們!這裡,到\twnr{林野}{142.0}的,或到樹下的,或到空屋的比丘坐下,\twnr{盤腿}{240.0}、定置端直的身體、\twnr{建立面前的念後}{529.0},\twnr{他只具念地吸氣}{526.0}、只具念地呼氣:

  當吸氣長時,知道:『我吸氣長。』……(中略)

  學習:『\twnr{隨看著斷念}{211.1},我將吸氣。』學習:『隨看著斷念,我將呼氣。』

  比丘們!這樣,入出息念之定已修習、這樣已多作,有大果、大效益。

  比丘們!當就在我\twnr{正覺}{185.1}以前,還是未\twnr{現正覺}{75.0}的\twnr{菩薩}{186.0}時,我也就以這個相同的住處多住。比丘們!當以這個住處多住時,那個我的身體既不疲倦,兩眼也不,且不執取後我的心從諸\twnr{漏}{188.0}被解脫。

  比丘們!因此,在這裡,如果比丘也希望:『我的身體既不疲倦,兩眼也不,且不執取後我的心從諸\twnr{漏}{188.0}被解脫。』這個入出息念之定就應該被\twnr{好好作意}{43.1}。

  比丘們!因此,在這裡,如果比丘也希望:『凡我掛慮家的憶念與意向,那些被捨斷。』這個入出息念之定就應該被好好作意。

  比丘們!因此,在這裡,如果比丘也希望:『在無\twnr{厭逆}{227.0}上住於有厭逆想的。』這個入出息念之定就應該被好好作意。

  比丘們!因此,在這裡,如果比丘也希望:『在厭逆上住於無厭逆想的。』這個入出息念之定就應該被好好作意。

  比丘們!因此,在這裡,如果比丘也希望:『在厭逆與不厭逆上住於有厭逆想的。』這個入出息念之定就應該被好好作意。

  比丘們!因此,在這裡,如果比丘也希望:『在厭逆與無厭逆上住於無厭逆想的。』這個入出息念之定就應該被好好作意。

  比丘們!因此,在這裡,如果比丘也希望:『在無厭逆與厭逆兩者上都避免後住於\twnr{平靜}{228.0}的、具念的、正知的。』這個入出息念之定就應該被好好作意。

  比丘們!因此,在這裡,如果比丘也希望:『就從離諸欲後,從離諸不善法後,\twnr{進入後住於}{66.0}有尋、\twnr{有伺}{175.0},\twnr{離而生喜、樂}{174.0}的初禪。』這個入出息念之定就應該被好好作意。

  比丘們!因此,在這裡,如果比丘也希望:『從尋與伺的平息,\twnr{自身內的明淨}{256.0},\twnr{心的專一性}{255.0},進入後住於無尋、無伺,定而生喜、樂的第二禪。』這個入出息念之定就應該被好好作意。

  比丘們!因此,在這裡,如果比丘也希望:『從喜的\twnr{褪去}{77.0}、住於\twnr{平靜}{228.0}、有念正知、以身體感受樂,進入後住於這\twnr{聖弟子}{24.0}宣說:

  「他是平靜、具念、\twnr{住於樂者}{317.0}」的第三禪。』這個入出息念之定就應該被好好作意。

  比丘們!因此,在這裡,如果比丘也希望:『從樂的捨斷與從苦的捨斷,就在之前諸喜悅、憂的滅沒,進入後住於不苦不樂,\twnr{平靜、念遍純淨}{494.0}的第四禪。』這個入出息念之定就應該被好好作意。

  比丘們!因此,在這裡,如果比丘也希望:『\twnr{從一切色想的超越}{490.0},從\twnr{有對想}{331.0}的滅沒,從不作意種種想[而知]:『虛空是無邊的』,進入後住於虛空無邊處。』這個入出息念之定就應該被好好作意。

  比丘們!因此,在這裡,如果比丘也希望:『超越一切虛空無邊處後[而知]:『識是無邊的』,進入後住於識無邊處。』這個入出息念之定就應該被好好作意。

  比丘們!因此,在這裡,如果比丘也希望:『超越一切識無邊處後[而知]:『什麼都沒有』,進入後住於無所有處。』這個入出息念之定就應該被好好作意。

  比丘們!因此,在這裡,如果比丘也希望:『超越一切無所有處後,進入後住於非想非非想處。』這個入出息念之定就應該被好好作意。

  比丘們!因此,在這裡,如果比丘也希望:『超越一切非想非非想處後,進入後住於想受滅。』這個入出息念之定就應該被好好作意。

  比丘們!在這樣,入出息念之定已修習、這樣已多作時,他如果感受樂受,知道:『那是無常的。』知道:『是不被固執的。』知道:『是不被歡喜的。』

  他如果感受苦受,知道:『那是無常的。』知道:『是不被固執的。』知道:『是不被歡喜的。』

  他如果感受不苦不樂受,知道:『那是無常的。』知道:『是不被固執的。』知道:『是不被歡喜的。』

  他如果感受樂受,離結縛地感受它;他如果感受苦受,離結縛地感受它;他如果感受不苦不樂受,離結縛地感受它。

  他\twnr{當感受身體終了的感受時}{720.0},知道:『我感受身體終了的感受。』\twnr{當感受生命終了的感受時}{721.0},知道:『我感受生命終了的感受。』他知道:『以身體的崩解,隨後生命耗盡,就在這裡,一切所感受的、不被歡喜的將成為清涼[,遺骸被留下(剩下)-\suttaref{SN.12.51}]。』

  比丘們!猶如\twnr{緣於}{252.0}油與緣於燈芯,油燈燃燒。就那個油燈的油與燈芯之耗盡,無食物者被熄滅。同樣的,比丘們!比丘當感受身體終了的感受時,知道:『我感受身體終了的感受。』或當感受生命終了的感受時,知道:『我感受生命終了的感受。』他知道:『以身體的崩解,隨後生命耗盡,就在這裡,一切所感受的、不被歡喜的將成為清涼[,遺骸被留下(剩下)]。』」



\sutta{9}{9}{毘舍離經}{https://agama.buddhason.org/SN/sn.php?keyword=54.9}
  \twnr{被我這麼聽聞}{1.0}:

  \twnr{有一次}{2.0},\twnr{世尊}{12.0}住在毘舍離國大林\twnr{重閣}{213.0}講堂。

  當時,世尊以種種法門為\twnr{比丘}{31.0}們講述不淨說,稱讚不淨,稱讚不淨之\twnr{修習}{94.0}。

  那時,世尊召喚比丘們:

  「比丘們!我想要\twnr{獨坐}{92.0}半個月,我不應該被任何人來見,除了以一位送\twnr{施食}{196.0}者外。」

  「是的,\twnr{大德}{45.0}!」

  那些比丘回答世尊後,在那裡,確實沒任何人去見世尊,除了以一位送施食者外。

  那時,那些比丘:「世尊以種種法門講述不淨說,稱讚不淨,稱讚不淨之修習。」住於\twnr{以種種行相與差別}{x646},致力於不淨之修習實踐。

  以這個身體為厭惡、羞恥、嫌惡的他們遍求殺手,在一天中,或十位比丘取刀,或二十位……或三十位比丘取刀。

  那時,經過那半個月,世尊從獨坐出來,召喚\twnr{尊者}{200.0}阿難:

  「阿難!為什麼比丘眾(僧團)像是變少的呢?」

  「大德!因為像這樣:『世尊以種種法門為比丘們講述不淨說,稱讚不淨,稱讚不淨之修習。』他們住於以種種行相與差別,致力於不淨之修習實踐,以這個身體為厭惡、羞恥、嫌惡的他們遍求殺手,在一天中,或十位比丘取刀,或二十位……或三十位比丘取刀。大德!請世尊告知其他法門,依之比丘眾能建立\twnr{完全智}{489.0},\twnr{那就好了}{44.0}!」

  「阿難!那樣的話,比丘依止毘舍離住之所及,請你使他們全部集合到講堂。」

  「是的,大德!」尊者阿難回答世尊後,所有比丘依毘舍離居住者,使他們全集合到講堂中後,去見世尊。抵達後,對世尊說這個:

  「大德!比丘眾已經集合,大德!現在是那個世尊\twnr{考量的時間}{84.0}。」

  那時,世尊去講堂。抵達後,在已設置的座位上坐下。坐下後,世尊召喚比丘們:

  「比丘們!這\twnr{入出息念}{329.0}之定已\twnr{修習}{94.0}、已\twnr{多作}{95.0},就是寂靜的、勝妙的、無混濁的、安樂的住處,且使已生起的惡不善法立即地消失、平息。

  比丘們!猶如在夏季的最後一個月揚起的塵垢,大驟雨使它立即地消失、平息。同樣的,比丘們!入出息念之定已修習、已多作,就是寂靜的、勝妙的、無混濁的、安樂的住處,且使已生起的惡不善法立即地消失、平息。

  比丘們!而怎樣入出息念之定已修習、怎樣已多作,就是寂靜的、勝妙的、無混濁的、安樂的住處,且使已生起的惡不善法立即地消失、平息呢?

  比丘們!這裡,到\twnr{林野}{142.0}的,或到樹下的,或到空屋的比丘坐下,\twnr{盤腿}{240.0}、定置端直的身體、\twnr{建立面前的念後}{529.0},\twnr{他只具念地吸氣}{526.0}、只具念地呼氣:

  ……(中略)

  學習:『\twnr{隨看著斷念}{211.1},我將吸氣。』學習:『隨看著斷念,我將呼氣。』

  比丘們!這樣,入出息念之定已修習、這樣已多作,就是寂靜的、勝妙的、無混濁的、安樂的住處,且使已生起的惡不善法立即地消失、平息。」



\sutta{10}{10}{金毘羅經}{https://agama.buddhason.org/SN/sn.php?keyword=54.10}
  \twnr{被我這麼聽聞}{1.0}:

  \twnr{有一次}{2.0},\twnr{世尊}{12.0}住在金毘羅竹林。

  在那裡,世尊召喚\twnr{尊者}{200.0}金毘羅:

  「金毘羅!\twnr{入出息念}{329.0}之定怎樣已\twnr{修習}{94.0}、怎樣已\twnr{多作}{95.0},有大果、\twnr{大效益}{113.0}呢?」

  在這麼說時,尊者金毘羅保持沈默。

  第二次,世尊又……(中略)。

  第三次,世尊又召喚尊者金毘羅:

  「金毘羅!怎樣入出息念之定已修習、怎樣已多作,有大果、大效益呢?」

  第三次,尊者金毘羅又保持沈默。

  在這麼說時,尊者阿難對世尊說這個:

  「世尊!是為了這個的適當時機,\twnr{善逝}{8.0}!是為了這個的適當時機:凡如果世尊說入出息念之定。聽聞世尊的[教說]後,\twnr{比丘}{31.0}們將會憶持。」

  「阿難!那樣的話,你要聽!你要\twnr{好好作意}{43.1}!我將說。」

  「是的,世尊!」尊者阿難回答世尊。

  世尊說這個:

  「阿難!怎樣入出息念之定已修習、怎樣已多作,有大果、大效益呢?

  阿難!這裡,到\twnr{林野}{142.0}的,或到樹下的,或到空屋的比丘坐下,\twnr{盤腿}{240.0}、定置端直的身體、\twnr{建立面前的念後}{529.0},\twnr{他只具念地吸氣}{526.0}、只具念地呼氣:

  ……(中略)

  學習:『\twnr{隨看著斷念}{211.1},我將吸氣。』學習:『隨看著斷念,我將呼氣。』

  阿難!這樣,入出息念之定已修習、這樣已多作,有大果、大效益。

  阿難!比丘凡在當吸氣長時,知道:『我吸氣長。』或當呼氣長時,知道:『我呼氣長。』或當吸氣短時,知道:『我吸氣短。』或當呼氣短時,知道:『我呼氣短。』學習:『\twnr{經驗著一切身}{527.0},我將吸氣。』學習:『經驗著一切身,我將呼氣。』學習:『\twnr{使身行寧靜著}{528.0},我將吸氣。』學習:『使身行寧靜著,我將呼氣。』阿難!在那時,比丘住於\twnr{在身上隨看著身}{176.0}:熱心的、正知的、有念的,調伏世間中的\twnr{貪婪}{435.0}、憂後。那是什麼原因?阿難!我說這是身的一種,即:吸氣、呼氣。阿難!因此,在這裡,在那時,比丘住於在身上隨看著身:熱心的、正知的、有念的,調伏世間中的貪婪、憂後。

  阿難!凡在比丘學習:『\twnr{經驗著喜}{754.1},我將吸氣。』學習:『經驗著喜,我將呼氣。』學習:『經驗著樂,我將吸氣。』學習:『經驗著樂,我將呼氣。』學習:『經驗著\twnr{心行}{230.0},我將吸氣。』學習:『經驗著心行,我將呼氣。』學習:『使心行寧靜著,我將吸氣。』學習:『使心行寧靜著,我將呼氣。』時,阿難!在那時,比丘住於在諸受上隨看著受:熱心的、正知的、有念的,調伏世間中的貪婪、憂後。那是什麼原因?阿難!我說這是受的一種,即:吸氣、呼氣的好好作意。阿難!因此,在這裡,在那時,比丘住於在諸受上隨看著受:熱心的、正知的、有念的,調伏世間中的貪婪、憂後。

  阿難!凡在比丘學習:『\twnr{經驗著心}{968.0},我將吸氣。』學習:『經驗著心,我將呼氣。』學習:『\twnr{使心喜悅著}{700.0},我將吸氣。』使心喜悅著……(中略)\twnr{集中著心}{701.0}……(中略)學習:『\twnr{使心解脫著}{702.0},我將吸氣。』學習:『使心解脫著,我將呼氣。』時,阿難!在那時,比丘住於在心上隨看著心:熱心的、正知的、有念的,調伏世間中的貪婪、憂後。那是什麼原因?阿難!我不說\twnr{念已忘失}{216.0}、不正知者有入出息念之定的修習。阿難!因此,在這裡,在那時,比丘住於在心上隨看著心:熱心的、正知的、有念的,調伏世間中的貪婪、憂後。

  阿難!凡在比丘學習:『\twnr{隨看著無常}{59.1},我將吸氣。』……(中略)隨看著\twnr{離貪}{77.0}……(中略)\twnr{隨看著滅}{703.0}……(中略)學習:『\twnr{隨看著斷念}{211.1},我將吸氣。』學習:『隨看著斷念,我將呼氣。』時,阿難!在那時,比丘住於在諸法上隨看著法:熱心的、正知的、有念的,調伏世間中的貪婪、憂後。凡他以慧看見後有貪婪、憂的捨斷,\twnr{他是善旁觀者}{995.0}。阿難!因此,在這裡,在那時,比丘住於在諸法上隨看著法:熱心的、正知的、有念的,調伏世間中的貪婪、憂後。

  阿難!猶如在十字路口處有大土堆,如果貨車或馬車從東方到來,就破壞那個土堆,如果也從西方到來……(中略)如果也從北方到來……(中略)如果貨車或馬車也從南方到來,就破壞那個土堆。同樣的,阿難!當比丘住於在身上隨看著身時,就破壞諸惡不善法,在諸受上……(中略)在心上……(中略)當也住於在諸法上隨看著法時,就破壞諸惡不善法。」

  一法品第一,其\twnr{攝頌}{35.0}:

  「一法與覺支,概要與二則大果,

   阿梨瑟吒、迦賓、燈,毘舍離與以金毘羅。」





\pin{第二品}{11}{20}
\sutta{11}{11}{一奢能伽羅經}{https://agama.buddhason.org/SN/sn.php?keyword=54.11}
  \twnr{有一次}{2.0},\twnr{世尊}{12.0}住在一奢能伽羅的一奢能伽羅叢林中。

  在那裡,世尊召喚\twnr{比丘}{31.0}們:

  「比丘們!我想要\twnr{獨坐}{92.0}三個月,我不應該被任何人來見,除了以一位送\twnr{施食}{196.0}者外。」

  「是的,\twnr{大德}{45.0}!」

  那些比丘回答世尊後,在那裡,確實沒任何人去見世尊,除了以一位送施食者外。

  那時,經過那三個月,世尊從獨坐出來,召喚比丘們:

  「比丘們!如果\twnr{其他外道遊行者}{79.0}們這麼問:『\twnr{道友們}{201.0}!\twnr{沙門}{29.0}\twnr{喬達摩}{80.0}多以什麼住處在雨季安居中住呢?』比丘們!被這麼問,你們應該這麼回答那些其他外道遊行者:『道友們!世尊在雨季安居中多住於\twnr{入出息念}{329.0}之定。』比丘們!這裡,我具念地吸氣、具念地呼氣:

  當吸氣長時,知道:『我吸氣長。』當呼氣長時,知道:『我呼氣長。』

  當吸氣短時,知道:『我吸氣短。』當呼氣短時,知道:『我呼氣短。』

  知道:『\twnr{經驗著一切身}{527.0},我將吸氣。』……(中略)

  知道:『\twnr{隨看著斷念}{211.1},我將吸氣。』知道:『隨看著斷念,我將呼氣。』

  比丘們!凡當正確說它時,應該說『聖住』,及『梵住』,及『\twnr{如來住}{699.0}』,當正確說時,是入出息念之定,會說『聖住』,及『梵住』,及『如來住』。

  比丘們!凡那些心意未達成、住於希求著無上\twnr{軛安穩}{192.0}的\twnr{有學}{193.0},他們的入出息念之定已\twnr{修習}{94.0}、已\twnr{多作}{95.0},轉起諸漏的滅盡。

  比丘們!而凡那些漏已滅盡的、已完成的、應該被作的已作的、負擔已卸的、自己的利益已達成的、有之結已滅盡的、以\twnr{究竟智}{191.0}解脫的\twnr{阿羅漢}{5.0}比丘,他們的入出息念之定已修習、已多作,就轉起當生樂的住處,以及念、正知。

  比丘們!凡當正確說它時,應該說『聖住』,及『梵住』,及『如來住』,當正確說時,是入出息念之定,會說『聖住』,及『梵住』,及『如來住』。」



\sutta{12}{12}{會疑惑的經}{https://agama.buddhason.org/SN/sn.php?keyword=54.12}
  \twnr{有一次}{2.0},\twnr{尊者}{200.0}羅瑪沙迦賓亞住在釋迦族人的迦毘羅衛城尼拘律園。

  那時,釋迦族人摩訶男去見尊者羅瑪沙迦賓亞。抵達後,向尊者羅瑪沙迦賓亞\twnr{問訊}{46.0}後,在一旁坐下。在一旁坐下的釋迦族人摩訶男對尊者羅瑪沙迦賓亞說這個:

  「\twnr{大德}{45.0}!那個有學住即是那個\twnr{如來住}{699.0}呢?還是有學住是一,如來住是另一呢?」

  「摩訶男\twnr{學友}{201.0}!那個有學住非即是那個如來住,有學住是一,如來住是另一。摩訶男學友!凡那些心意未達成、住於希求著無上\twnr{軛安穩}{192.0}的\twnr{有學}{193.0}\twnr{比丘}{31.0},他們\twnr{捨斷五蓋後}{x647}而住,哪五個?捨斷\twnr{欲的意欲}{118.0}蓋後而住、惡意蓋……惛沈睡眠蓋……掉舉後悔蓋……捨斷疑惑蓋後而住。摩訶男學友!凡那些心意未達成、住於希求著無上軛安穩的有學比丘,他們捨斷這些五蓋後而住。

  摩訶男學友!而凡那些漏已滅盡的、已完成的、\twnr{應該被作的已作的}{20.0}、負擔已卸的、\twnr{自己的利益已達成的}{189.0}、\twnr{有之結已滅盡的}{190.0}、以\twnr{究竟智}{191.0}解脫的\twnr{阿羅漢}{5.0}比丘,他們的五蓋已被捨斷,根已被切斷,\twnr{[如]已斷根的棕櫚樹}{147.1},\twnr{成為非有}{408.0},\twnr{為未來不生之物}{229.0},哪五個?欲的意欲蓋已被捨斷,根已被切斷,[如]已斷根的棕櫚樹,成為非有,為未來不生之物、惡意蓋已被捨斷……惛沈睡眠蓋……掉舉後悔蓋……疑惑蓋已被捨斷,根已被切斷,[如]已斷根的棕櫚樹,成為非有,為未來不生之物。摩訶男學友!凡那些漏已滅盡的、已完成的、應該被作的已作的、負擔已卸的、自己的利益已達成的、有之結已滅盡的、以究竟智解脫的阿羅漢比丘,他們的這些五蓋已被捨斷,根已被切斷,[如]已斷根的棕櫚樹,成為非有,為未來不生之物。

  摩訶男學友!那樣,以這個法門,這也能被知道:關於有學住是一,如來住是另一。

  摩訶男學友!有這一次,\twnr{世尊}{12.0}住在一奢能伽羅的一奢能伽羅叢林中。摩訶男學友!在那裡,世尊召喚比丘們:『比丘們!我想要\twnr{獨坐}{92.0}三個月,我不應該被任何人來見,除了以一位送\twnr{施食}{196.0}者外。』

  『是的,\twnr{大德}{45.0}!』

  摩訶男學友!那些比丘回答世尊後,在那裡,確實沒任何人去見世尊,除了以一位送施食者外。

  那時,經過那三個月,世尊從獨坐出來,召喚比丘們:『比丘們!如果\twnr{其他外道遊行者}{79.0}們這麼問:「\twnr{道友們}{201.0}!\twnr{沙門}{29.0}\twnr{喬達摩}{80.0}多以什麼住處在\twnr{雨季安居}{231.0}中住呢?」比丘們!被這麼問,你們應該這麼回答那些其他外道遊行者:「道友們!世尊在雨季安居中多住於\twnr{入出息念}{329.0}之定。」

  比丘們!這裡,我具念地吸氣、具念地呼氣:

  當吸氣長時,知道:「我吸氣長。」

  當呼氣長時,知道:「我呼氣長。」

  ……(中略)

  知道:「\twnr{隨看著斷念}{211.1},我將吸氣。」

  知道:「隨看著斷念,我將呼氣。」

  比丘們!凡當正確說它時,應該說「聖住」,及「梵住」,及「\twnr{如來住}{699.0}」,當正確說時,是入出息念之定,會說「聖住』,及「梵住」,及「如來住」。

  比丘們!凡那些有心意未達成、住於希求著無上軛安穩的學比丘,他們的入出息念之定已\twnr{修習}{94.0}、已\twnr{多作}{95.0},轉起諸漏的滅盡。

  比丘們!而凡那些漏已滅盡的、已完成的、應該被作的已作的、負擔已卸的、自己的利益已達成的、有之結已滅盡的、以究竟智解脫的阿羅漢比丘,他們的入出息念之定已修習、已多作,就轉起當生樂的住處,以及念、正知。

  比丘們!凡當正確說它時,應該說「聖住」,及「梵住」,及「如來住」,當正確說時,是入出息念之定,會說「聖住』,及「梵住」,及「如來住」。』

  摩訶男學友!那樣,以這個法門,這也能被知道:關於有學住是一,如來住是另一。」



\sutta{13}{13}{阿難經第一}{https://agama.buddhason.org/SN/sn.php?keyword=54.13}
  起源於舍衛城。

  那時,\twnr{尊者}{200.0}阿難去見\twnr{世尊}{12.0}。抵達後,向世尊\twnr{問訊}{46.0}後,在一旁坐下。在一旁坐下的尊者阿難對世尊說這個:

  「\twnr{大德}{45.0}!有一法已\twnr{修習}{94.0}、已\twnr{多作}{95.0},使四法完成;四法已修習、已多作,使七法完成;七法已修習、已多作,使二法完成嗎?」

  「阿難!有一法已修習、已多作,使四法完成;四法已修習、已多作,使七法完成;七法已修習、已多作,使二法完成。」

  「大德!那麼,哪一法已修習、已多作,使四法完成;四法已修習、已多作,使七法完成;七法已修習、已多作,使二法完成?」

  「阿難!\twnr{入出息念}{329.0}之定是一法,已修習、已多作,使\twnr{四念住}{286.0}完成;四念住已修習、已多作,使\twnr{七覺支}{524.0}完成;七覺支已修習、已多作,使明與解脫完成。

  阿難!怎樣入出息念之定已修習、怎樣已多作,使四念住完成呢?阿難!這裡,到\twnr{林野}{142.0}的,或到樹下的,或到空屋的\twnr{比丘}{31.0}坐下,\twnr{盤腿}{240.0}、定置端直的身體、\twnr{建立面前的念後}{529.0},\twnr{他只具念地吸氣}{526.0}、只具念地呼氣:

  當吸氣長時,知道:『我吸氣長。』或當呼氣長時,知道:『我呼氣長。』

  ……(中略)

  學習:『\twnr{隨看著斷念}{211.1},我將吸氣。』學習:『隨看著斷念,我將呼氣。』

  阿難!比丘凡在當吸氣長時,知道:『我吸氣長。』或當呼氣長時,知道:『我呼氣長』……短……(中略)學習:『\twnr{使身行寧靜著}{528.0},我將吸氣。』學習:『使身行寧靜著,我將呼氣。』時,阿難!在那時,比丘住於\twnr{在身上隨看著身}{176.0}:熱心的、正知的、有念的,調伏世間中的\twnr{貪婪}{435.0}、憂後。那是什麼原因?阿難!我說這是身的一種,即:吸氣、呼氣。阿難!因此,在這裡,在那時,比丘住於在身上隨看著身:熱心的、正知的、有念的,調伏世間中的貪婪、憂後。

  阿難!凡在比丘學習:『\twnr{經驗著喜}{754.1},我將吸氣。』……(中略)經驗著樂……(中略)經驗著\twnr{心行}{230.0}……(中略)學習:『使心行寧靜著,我將吸氣。』學習:『使心行寧靜著,我將呼氣。』時,阿難!在那時,比丘住於在諸受上隨看著受:熱心的、正知的、有念的,調伏世間中的貪婪、憂後。那是什麼原因?阿難!我說這是受的一種,即:吸氣、呼氣的\twnr{好好作意}{43.1}。阿難!因此,在這裡,在那時,比丘住於在諸受上隨看著受:熱心的、正知的、有念的,調伏世間中的貪婪、憂後。

  阿難!凡在比丘學習:『\twnr{經驗著心}{968.0},我將吸氣。』學習:『經驗著心,我將呼氣。』……\twnr{使心喜悅著}{700.0}……(中略)\twnr{集中著心}{701.0}……(中略)學習:『\twnr{使心解脫著}{702.0},我將吸氣。』學習:『使心解脫著,我將呼氣。』時,阿難!在那時,比丘住於在心上隨看著心:熱心的、正知的、有念的,調伏世間中的貪婪、憂後。那是什麼原因?阿難!我不說\twnr{念已忘失}{216.0}、不正知者有入出息念之定的修習。阿難!因此,在這裡,在那時,比丘住於在心上隨看著心:熱心的、正知的、有念的,調伏世間中的貪婪、憂後。

  阿難!比丘凡在……\twnr{隨看著無常}{59.1}……(中略)隨看著\twnr{離貪}{77.0}……(中略)\twnr{隨看著滅}{703.0}……(中略)學習:『\twnr{隨看著斷念}{211.1},我將吸氣。』學習:『隨看著斷念,我將呼氣。』時,阿難!在那時,比丘住於在諸法上隨看著法:熱心的、正知的、有念的,調伏世間中的貪婪、憂後。凡他以慧看見後有貪婪、憂的捨斷,\twnr{他是善旁觀者}{995.0}。阿難!因此,在這裡,在那時,比丘住於在諸法上隨看著法:熱心的、正知的、有念的,調伏世間中的貪婪、憂後。

  阿難!這樣,入出息念之定已修習、這樣已多作,使四念住完成。

  阿難!四念住怎樣已修習、怎樣已多作,使七覺支完成呢?

  阿難!凡在比丘住於在身上看到身時,在那時,那位比丘的\twnr{念被現起}{341.0},不被忘失。阿難!凡在比丘的念被現起,不被忘失時,在那時,比丘的\twnr{念覺支}{315.0}被發動,在那時,比丘修習念覺支,在那時,比丘的念覺支走到修習圓滿。

  那位住於像這樣念者以慧考察(簡擇)、伺察、來到審慮那個法。阿難!凡在住於像這樣念的比丘以慧考察、伺察、來到審慮那個法時,在那時,比丘的\twnr{擇法覺支}{311.0}被發動,在那時,比丘修習擇法覺支,在那時,比丘的擇法覺支走到修習圓滿。

  那位以慧考察、伺察、來到審慮那個法者的不退縮的活力被發動。阿難!凡在比丘以慧考察、伺察、來到審慮那個法的不退縮的活力被發動時,在那時,比丘的\twnr{活力覺支}{310.0}被發動,在那時,比丘修習活力覺支,在那時,比丘的活力覺支走到修習圓滿。

  活力被發動者精神的喜生起。阿難!凡在活力被發動比丘的精神的喜生起時,比丘的\twnr{喜覺支}{312.0}被發動,在那時,比丘修習喜覺支,在那時,比丘的喜覺支走到修習圓滿。

  \twnr{意喜}{320.0}者的身變得寧靜(輕安),心也變得寧靜。阿難!凡在意喜比丘的身變得寧靜、心也變得寧靜時,在那時,比丘的\twnr{寧靜覺支}{313.1}被發動,在那時,比丘修習寧靜覺支,在那時,比丘的寧靜覺支走到修習圓滿。

  身寧靜者、有樂者的心入定。阿難!凡在身寧靜、有樂比丘的心入定時,比丘的定覺支就被發動,在那時,比丘修習定覺支,在那時,比丘的定覺支走到修習圓滿。

  那位心像這樣得定者成為\twnr{善旁觀者}{995.0}。阿難!凡在比丘心像這樣得定者成為善旁觀者時,在那時,比丘的\twnr{平靜覺支}{314.0}被發動,在那時,比丘修習\twnr{平靜覺支}{314.0},在那時,比丘的平靜覺支走到修習圓滿。

  阿難!凡在比丘住於在受上……(中略)在心上……(中略)住於在諸法上隨看著法時,在那時,那位比丘的念被現起,不被忘失。阿難!凡在比丘的念被現起,不被忘失時,在那時,比丘的念覺支被發動,在那時,比丘修習念覺支,在那時,比丘的念覺支走到修習圓滿。(……應該如同最初念住那樣使之被細說)

  那位心像這樣得定者成為善旁觀者。阿難!凡在比丘心像這樣得定者成為善旁觀者時,在那時,比丘的平靜覺支被發動,在那時,比丘修習平靜覺支,在那時,比丘的平靜覺支走到修習圓滿。

  阿難!四念住這麼已修習、這麼已多作,使七覺支完成。

  阿難!七覺支怎樣已修習、怎樣已多作,使明與解脫完成呢?阿難!這裡,比丘\twnr{依止遠離}{322.0}、依止離貪、依止滅、\twnr{捨棄的成熟}{221.0}修習念覺支……修習擇法覺支……(中略)依止遠離、依止離貪、依止滅、捨棄的成熟修習平靜覺支。阿難!這樣,七覺支已修習、這樣已多作,使明與解脫完成。」



\sutta{14}{14}{阿難經第二}{https://agama.buddhason.org/SN/sn.php?keyword=54.14}
  那時,\twnr{尊者}{200.0}阿難去見\twnr{世尊}{12.0}。抵達後,向世尊\twnr{問訊}{46.0}後,在一旁坐下。世尊對在一旁坐下的尊者阿難說這個:

  「阿難!有一法已\twnr{修習}{94.0}、已\twnr{多作}{95.0},使四法完成;四法已修習、已多作,使七法完成;七法已修習、已多作,使二法完成嗎?」

  「大德!我們的法以世尊為根本……(中略)。」……

  「阿難!有一法修習、已多作,使四法完成;四法已修習、已多作,使七法完成;七法已修習、已多作,使二法完成。

  阿難!而哪一法,已修習、已多作,使四法完成;四法已修習、已多作,使七法完成;七法已修習、已多作,使二法完成?阿難!\twnr{入出息念}{329.0}之定是一法,已修習、已多作,使\twnr{四念住}{286.0}完成;四念住已修習、已多作,使\twnr{七覺支}{524.0}完成;七覺支已修習、已多作,使明與解脫完成。

  阿難!而怎樣入出息念之定已修習、怎樣已多作,使四念住完成呢?阿難!這裡,\twnr{比丘}{31.0}到林野……(中略)。

  阿難!這樣,七覺支已修習、這樣已多作,使明與解脫完成。」



\sutta{15}{15}{比丘經第一}{https://agama.buddhason.org/SN/sn.php?keyword=54.15}
  那時,眾多比丘去見世尊。抵達後,向世尊\twnr{問訊}{46.0}後,在一旁坐下。在一旁坐下的那些比丘對世尊說這個:

  「\twnr{大德}{45.0}!有一法已\twnr{修習}{94.0}、已\twnr{多作}{95.0},使四法完成;四法已修習、已多作,使七法完成;七法已修習、已多作,使二法完成嗎?」

  「比丘們!有一法修習、已多作,使四法完成;四法已修習、已多作,使七法完成;七法已修習、已多作,使二法完成。」

  「大德!那麼,哪一法已修習、已多作,使四法完成;四法已修習、已多作,使七法完成;七法已修習、已多作,使二法完成?」

  「比丘們!\twnr{入出息念}{329.0}之定是一法,已修習、已多作,使\twnr{四念住}{286.0}完成;四念住已修習、已多作,使\twnr{七覺支}{524.0}完成;七覺支已修習、已多作,使明與解脫完成。

  比丘們!而怎樣入出息念之定已修習、怎樣已多作,使四念住完成?比丘們!這裡,比丘到\twnr{林野}{142.0}……(中略)。

  比丘們!這樣,七覺支已修習、這樣已多作,使明與解脫完成。」



\sutta{16}{16}{比丘經第二}{https://agama.buddhason.org/SN/sn.php?keyword=54.16}
  那時,眾多\twnr{比丘}{31.0}去見\twnr{世尊}{12.0}。抵達後,向世尊\twnr{問訊}{46.0}後,在一旁坐下。世尊對在一旁坐下的那些比丘說這個:

  「比丘們!有一法已\twnr{修習}{94.0}、已\twnr{多作}{95.0},使四法完成;四法已修習、已多作,使七法完成;七法已修習、已多作,使二法完成嗎?」

  「大德!我們的法以世尊為根本……(中略)聽聞世尊的[教說]後,比丘們將會憶持。」

  「比丘們!有一法修習、已多作,使四法完成;四法已修習、已多作,使七法完成;七法已修習、已多作,使二法完成。

  比丘們!而什麼是一法,已修習、已多作,使四法完成;四法已修習、已多作,使七法完成;七法已修習、已多作,使二法完成?比丘們!\twnr{入出息念}{329.0}之定是一法,已修習、已多作,使\twnr{四念住}{286.0}完成;四念住已修習、已多作,使\twnr{七覺支}{524.0}完成;七覺支已修習、已多作,使明與解脫完成。

  比丘們!而怎樣入出息念之定已修習、怎樣已多作,使四念住完成呢?阿難!這裡,到林野的,或到樹下的,或到空屋的比丘坐下,\twnr{盤腿}{240.0}、定置端直的身體、\twnr{建立面前的念後}{529.0},\twnr{他只具念地吸氣}{526.0}、只具念地呼氣:

  ……(中略)

  學習:『\twnr{隨看著斷念}{211.1},我將吸氣。』學習:『隨看著斷念,我將呼氣。』

  比丘們!比丘凡在當吸氣長時,知道:『我吸氣長。』或當呼氣長時,知道:『我呼氣長。』或當吸氣短時,知道:『我吸氣短。』……(中略)\twnr{經驗著一切身}{527.0}……(中略)學習:『\twnr{使身行寧靜著}{528.0},我將吸氣。』學習:『使身行寧靜著,我將呼氣。』時,比丘們!在那時,比丘住於\twnr{在身上隨看著身}{176.0}:熱心的、正知的、有念的,調伏世間中的\twnr{貪婪}{435.0}、憂後。那是什麼原因?比丘們!我說這是身的一種,即:吸氣、呼氣。比丘們!因此,在這裡,在那時,比丘住於在身上隨看著身:熱心的、正知的、有念的,調伏世間中的貪婪、憂後。

  比丘們!比丘凡在……\twnr{經驗著喜}{754.1}……(中略)經驗著樂……(中略)經驗著\twnr{心行}{230.0}……(中略)學習:『使心行寧靜著,我將吸氣。』學習:『使心行寧靜著,我將呼氣。』時,比丘們!在那時,比丘住於在諸受上隨看著受:熱心的、正知的、有念的,調伏世間中的貪婪、憂後。那是什麼原因?比丘們!我說這是受的一種,即:吸氣、呼氣的\twnr{好好作意}{43.1}。比丘們!因此,在這裡,在那時,比丘住於在諸受上隨看著受:熱心的、正知的、有念的,調伏世間中的貪婪、憂後。

  比丘們!比丘凡在……\twnr{經驗著心}{968.0}……(中略)\twnr{使心喜悅著}{700.0}……(中略)他學習:『\twnr{集中著心}{701.0},我將吸氣。』學習:『集中著心,我將呼氣。』學習:『\twnr{使心解脫著}{702.0},我將吸氣。』學習:『使心解脫著,我將呼氣。』時,比丘們!在那時,比丘住於在心上隨看著心:熱心的、正知的、有念的,調伏世間中的貪婪、憂後。那是什麼原因?比丘們!我不說\twnr{念已忘失}{216.0}、不正知者有入出息念之定的修習。比丘們!因此,在這裡,在那時,比丘住於在心上隨看著心:熱心的、正知的、有念的,調伏世間中的貪婪、憂後。

  比丘們!比丘凡在……\twnr{隨看著無常}{59.1}……(中略)隨看著\twnr{離貪}{77.0}……(中略)\twnr{隨看著滅}{703.0}……(中略)他學習:『\twnr{隨看著斷念}{211.1},我將吸氣。』學習:『隨看著斷念,我將呼氣。』時,比丘們!在那時,比丘住於在諸法上隨看著法:熱心的、正知的、有念的,調伏世間中的貪婪、憂後。凡他以慧看見後有貪婪、憂的捨斷,\twnr{他是善旁觀者}{995.0}。比丘們!因此,在這裡,在那時,比丘住於在諸法上隨看著法:熱心的、正知的、有念的,調伏世間中的貪婪、憂後。

  比丘們!這樣,入出息念之定已修習、這樣已多作,使四念住完成。

  比丘們!而四念住怎樣已修習、怎樣已多作,使七覺支完成呢?

  比丘們!凡在比丘住於在身上看到身時,在那時,比丘的\twnr{念被現起}{341.0},不被忘失。比丘們!凡在比丘的念被現起,不被忘失時,在那時,比丘的\twnr{念覺支}{315.0}被發動,在那時,比丘修習念覺支,在那時,比丘的念覺支走到修習圓滿。

  那位住於像這樣念者以慧考察(簡擇)、伺察、來到審慮那個法。比丘們!凡在住於像這樣念的比丘以慧考察、伺察、來到審慮那個法時,在那時,比丘的\twnr{擇法覺支}{311.0}被發動,在那時,比丘修習擇法覺支,在那時,比丘的擇法覺支走到修習圓滿。

  那位以慧考察、伺察、來到審慮那個法者的不退縮的活力被發動。比丘們!凡在比丘以慧考察、伺察、來到審慮那個法的不退縮的活力被發動時,在那時,比丘的\twnr{活力覺支}{310.0}被發動,在那時,比丘修習活力覺支,在那時,比丘的活力覺支走到修習圓滿。

  活力被發動者精神的喜生起。比丘們!凡在活力被發動比丘的精神的喜生起時,比丘的\twnr{喜覺支}{312.0}被發動,在那時,比丘修習喜覺支,在那時,比丘的喜覺支走到修習圓滿。

  \twnr{意喜}{320.0}者的身變得寧靜(輕安),心也變得寧靜。比丘們!凡在意喜比丘的身變得寧靜、心也變得寧靜時,在那時,比丘的\twnr{寧靜覺支}{313.1}被發動,在那時,比丘修習寧靜覺支,在那時,比丘的寧靜覺支走到修習圓滿。

  身寧靜者、有樂者的心入定。比丘們!凡在身寧靜、有樂比丘的心入定時,比丘的\twnr{定}{182.0}覺支就被發動,在那時,比丘修習定覺支,在那時,比丘的定覺支走到修習圓滿。

  那位心像這樣得定者成為\twnr{善旁觀者}{995.0}。比丘們!凡在比丘心像這樣得定者成為善旁觀者時,在那時,比丘的\twnr{平靜覺支}{314.0}被發動,在那時,比丘修習\twnr{平靜覺支}{314.0},那時,比丘的平靜覺支走到修習圓滿。

  比丘們!凡在比丘住於在諸受上……(中略)在心上……(中略)住於在諸法上隨看著法時,在那時,那位比丘的念被現起,不被忘失。比丘們!凡在比丘的念被現起,不被忘失時,在那時,比丘的念覺支被發動,在那時,比丘修習念覺支,在那時,比丘的念覺支走到修習圓滿。……(中略)

  那位心像這樣得定者成為\twnr{善旁觀者}{995.0}。比丘們!凡在比丘心像這樣得定者成為善旁觀者時,在那時,比丘的平靜覺支被發動,在那時,比丘修習平靜覺支,在那時,比丘的平靜覺支走到修習圓滿。

  比丘們!這樣,四念住已修習、這樣已多作,使七覺支完成。

  比丘們!七覺支怎樣已修習、怎樣已多作,使明與解脫完成?比丘們!這裡,比丘\twnr{依止遠離}{322.0}、依止離貪、依止滅、\twnr{捨棄的成熟}{221.0}修習念覺支,依止遠離、依止離貪、依止滅、捨棄的成熟修習擇法覺支……(中略)依止遠離、依止離貪、依止滅、捨棄的成熟修習平靜覺支。

  比丘們!這樣,七覺支已修習、這樣已多作,使明與解脫完成。」



\sutta{17}{17}{結的捨斷經}{https://agama.buddhason.org/SN/sn.php?keyword=54.17}
  「\twnr{比丘}{31.0}們!\twnr{入出息念}{329.0}之定已\twnr{修習}{94.0}、已\twnr{多作}{95.0},轉起結的捨斷……(中略)。」



\sutta{18}{18}{煩惱潛在趨勢的根除經}{https://agama.buddhason.org/SN/sn.php?keyword=54.18}
  「……轉起\twnr{煩惱潛在趨勢}{253.1}的根除……。」



\sutta{19}{19}{[生命]旅途的遍知經}{https://agama.buddhason.org/SN/sn.php?keyword=54.19}
  「……轉起\twnr{[生命]旅途的遍知}{889.0}……。」



\sutta{20}{20}{漏的滅盡經}{https://agama.buddhason.org/SN/sn.php?keyword=54.20}
  「……轉起諸\twnr{漏}{188.0}的滅盡,\twnr{比丘}{31.0}們!\twnr{入出息念}{329.0}之定怎樣已\twnr{修習}{94.0}、已\twnr{多作}{95.0},轉起結的捨斷……轉起\twnr{煩惱潛在趨勢}{253.1}的根除……轉起\twnr{[生命]旅途的遍知}{889.0}……轉起諸漏的滅盡?比丘們!這裡,比丘到\twnr{林野}{142.0},或到樹下……(中略)他學習:『\twnr{隨看著斷念}{211.1},我將吸氣。』學習:『隨看著斷念,我將呼氣。』

  比丘們!這樣,入出息念之定已修習、這麼已多作,轉起結的捨斷……轉起煩惱潛在趨勢的根除……轉起[生命]旅途的遍知……轉起諸漏的滅盡。」

  第二品,其\twnr{攝頌}{35.0}:

  「一奢能伽羅、會疑惑的,阿難二則在後,

   比丘、結、煩惱潛在趨勢,[生命]旅途、漏的滅盡。」

  入出息相應第十。





\page

\xiangying{55}{入流相應}
\pin{竹門品}{1}{10}
\sutta{1}{1}{轉輪王經}{https://agama.buddhason.org/SN/sn.php?keyword=55.1}
  起源於舍衛城。

  在那裡,\twnr{世尊}{12.0}……(中略)說這個:

  「\twnr{比丘}{31.0}們!即使\twnr{轉輪王}{278.0}行使(使作)四洲的自在支配王權後,以身體的崩解,死後往生\twnr{善趣}{112.0}、天界,為三十三天們的共住狀態,在那裡,他在\twnr{歡喜園}{704.0}中被仙女眾圍繞,且賦有、擁有天的\twnr{五種欲}{187.0}自娛,他未具備四法,那時,他仍未從地獄被解脫,未從畜生界被解脫,未從餓鬼界被解脫,未從\twnr{苦界}{109.0}、惡趣、\twnr{下界}{111.0}被解脫。

  比丘們!即使聖弟子以團食維持生活與穿弊壞衣,他具備四法,那時,他從地獄被解脫,從畜生界被解脫,從餓鬼界被解脫,從苦界、惡趣、下界被解脫。哪四個?

  比丘們!這裡,聖弟子在佛上具備不壞淨:『像這樣,那位世尊是\twnr{阿羅漢}{5.0}、\twnr{遍正覺者}{6.0}、\twnr{明行具足者}{7.0}、\twnr{善逝}{8.0}、\twnr{世間知者}{9.0}、\twnr{應該被調御人的無上調御者}{10.0}、\twnr{天-人們的大師}{11.0}、\twnr{佛陀}{3.0}、\twnr{世尊}{12.0}。』

  在法上具備不壞淨:『被世尊善說的法是直接可見的、即時的、請你來看的、能引導的、\twnr{應該被智者各自經驗的}{395.0}。』

  在\twnr{僧團}{375.0}上具備不壞淨:『世尊的弟子僧團是\twnr{善行者}{518.0},世尊的弟子僧團是正直行者,世尊的弟子僧團是真理行者,世尊的弟子僧團是\twnr{方正行者}{764.0},即:四雙之人、\twnr{八輩之士}{347.0},這世尊的弟子僧團應該被奉獻、應該被供奉、應該被供養、應該被\twnr{合掌}{377.0},為世間的無上\twnr{福田}{101.0}。』

  具備聖者喜愛的諸戒:無毀壞的、無瑕疵的、無污點的、無雜色的、自由的、智者稱讚的、不取著的、轉起定的。

  具備這四法。

  比丘們!凡四洲的獲得與凡四法的獲得,四洲的獲得比四法的獲得,不值得十六分之一。」



\sutta{2}{2}{梵行立足處經}{https://agama.buddhason.org/SN/sn.php?keyword=55.2}
  「\twnr{比丘}{31.0}們!具備四法的\twnr{聖弟子}{24.0}是\twnr{入流者}{165.0}、不墮惡趣法者、\twnr{決定者}{159.0}、\twnr{正覺為彼岸者}{160.0},哪四個?比丘們!這裡,聖弟子在佛上具備\twnr{不壞淨}{233.0}:『像這樣,那位世尊是\twnr{阿羅漢}{5.0}、\twnr{遍正覺者}{6.0}、\twnr{明行具足者}{7.0}、\twnr{善逝}{8.0}、\twnr{世間知者}{9.0}、\twnr{應該被調御人的無上調御者}{10.0}、\twnr{天-人們的大師}{11.0}、\twnr{佛陀}{3.0}、\twnr{世尊}{12.0}。』在法上……(中略)在\twnr{僧團}{375.0}上……(中略)具備聖者喜愛的諸戒:無毀壞的……(中略)轉起定的。比丘們!具備這四法的聖弟子是入流者、不墮惡趣法者、決定者、正覺為彼岸者。」

  世尊說這個,說這個後,\twnr{善逝}{8.0}、\twnr{大師}{145.0}又更進一步說這個:

  「對凡信與戒,\twnr{淨信}{507.0}、法的看見者,

   他們確實在適當時間到達(回到),\twnr{梵行立足處}{x648}的安樂。」 



\sutta{3}{3}{長壽優婆塞經}{https://agama.buddhason.org/SN/sn.php?keyword=55.3}
  \twnr{有一次}{2.0},\twnr{世尊}{12.0}住在王舍城栗鼠飼養處的竹林中。

  當時,長壽\twnr{優婆塞}{98.0}是生病者、受苦者、重病者。

  那時,長壽優婆塞召喚父親\twnr{屋主}{103.0}樹提:

  「來!屋主!請你去見世尊。抵達後,請你以我的名義\twnr{以頭禮拜世尊的足}{40.0}:『\twnr{大德}{45.0}!長壽優婆塞是生病者、受苦者、重病者,他以頭禮拜世尊的足。』以及請你這麼說:『大德!請世尊\twnr{出自憐愍}{121.0},去長壽優婆塞的住處,\twnr{那就好了}{44.0}!』」

  「是的,兒子!」屋主樹提回答長壽優婆塞後,去見世尊。抵達後,向世尊\twnr{問訊}{46.0}後,在一旁坐下。在一旁坐下的屋主樹提對世尊說這個:

  「大德!長壽優婆塞是生病者、受苦者、重病者,他以頭禮拜世尊的足,以及他這麼說:『大德!請世尊出自憐愍,去長壽優婆塞的住處,那就好了!』」

  世尊以沈默狀態同意。

  那時,世尊穿衣、拿起衣鉢後,去長壽優婆塞的住處。抵達後,在設置的座位坐下。坐下後,世尊對長壽優婆塞說這個:

  「長壽!是否能被你忍受?\twnr{是否能被[你]維持生活}{137.0}?是否苦的感受減退、不增進,減退的結局被知道,非增進?」

  「大德!不能被我忍受,不能被[我]維持,我強烈苦的感受增進、不減退,增進的結局被知道,非減退。」

  「長壽!因此,在這裡,應該被你這麼學:『我將在佛上具備\twnr{不壞淨}{233.0}:「像這樣,那位世尊是\twnr{阿羅漢}{5.0}、\twnr{遍正覺者}{6.0}、\twnr{明行具足者}{7.0}、\twnr{善逝}{8.0}、\twnr{世間知者}{9.0}、\twnr{應該被調御人的無上調御者}{10.0}、\twnr{天-人們的大師}{11.0}、\twnr{佛陀}{3.0}、\twnr{世尊}{12.0}。」在法上……(中略)在\twnr{僧團}{375.0}上……(中略)我將具備聖者喜愛的諸戒:「無毀壞的……(中略)轉起定的。」』長壽!應該被你這麼學。」

  「大德!凡被世尊教導的這些四\twnr{入流支}{370.0},那些法在我中被發現,且我在那些法中被發現。

  大德!因為我在佛上具備不壞淨:『像這樣,那位世尊是……(中略)天-人們的大師、佛陀、世尊。』在法上……(中略)在僧團上……(中略)我具備聖者喜愛的諸戒:無毀壞的……(中略)轉起定的。」

  「長壽!因此,在這裡,在這些四入流支上住立後,你應該在六個\twnr{明的一部分}{576.0}之法上更上地\twnr{修習}{94.0}。長壽!這裡,請你住於在一切諸行上\twnr{隨看著}{59.0}無常,在無常上苦想,在苦上無我想、捨斷想、離貪想、滅想。』長壽!應該被你這麼學。」

  「大德!凡被世尊教導的這六個明的一部分之法,那些法在我中被發現,且我在那些法中被發現。大德!因為我住於在一切諸行上隨看著無常,在無常上苦想,在苦上無我想、捨斷想、離貪想、滅想。大德!但是,我這麼想:『這位屋主樹提不要就因為我死後來到惱害!』」

  「兒子長壽!你不要這麼作意,來吧!兒子長壽!凡任何世尊對你說的,你要就那個\twnr{好好作意}{43.1}。」

  那時,世尊以這個教誡教誡長壽優婆塞後,從座位起來後離開。

  那時,當世尊離開不久,長壽優婆塞命終。

  那時,眾多\twnr{比丘}{31.0}去見世尊。抵達後,向世尊問訊後,在一旁坐下。在一旁坐下的那些比丘對世尊說這個:

  「大德!那位被世尊簡要教誡教誡,名叫長壽的優婆塞,他已命終,他的趣處是什麼?來世是什麼?」

  「比丘們!長壽優婆塞是賢智者,實行法的隨法,且不因為法困擾我。比丘們!長壽優婆塞以\twnr{五下分結}{134.0}的滅盡,成為\twnr{化生}{346.0}者、在那裡般涅槃者、不從那個世間返還者。」



\sutta{4}{4}{舍利弗經第一}{https://agama.buddhason.org/SN/sn.php?keyword=55.4}
  \twnr{有一次}{2.0},\twnr{尊者}{200.0}舍利弗與尊者阿難住在舍衛城祇樹林給孤獨園。

  那時,尊者阿難傍晚時,從\twnr{獨坐}{92.0}出來……(中略)在一旁坐下的尊者阿難對尊者舍利弗說這個:

  「舍利弗\twnr{學友}{201.0}!幾法的具備之因,這樣,這個人被世尊\twnr{記說}{179.0}為入流者、不墮\twnr{惡趣}{110.0}法者、\twnr{決定者}{159.0}、\twnr{正覺為彼岸者}{160.0}呢?」

  「學友!四法的具備之因,這樣,這個人被世尊記說為入流者、不墮惡趣法者、決定者、正覺為彼岸者,哪四個?學友!這裡,\twnr{聖弟子}{24.0}在佛上具備不壞淨:『像這樣,那位\twnr{世尊}{12.0}……(中略)\twnr{天-人們的大師}{11.0}、\twnr{佛陀}{3.0}、\twnr{世尊}{12.0}。』在法上……(中略)在\twnr{僧團}{375.0}上……(中略)具備聖者喜愛的諸戒:無毀壞的……(中略)轉起定的。學友!這四法的具備之因,這樣,這個人被世尊記說為入流者、不墮惡趣法者、決定者、正覺為彼岸者。」



\sutta{5}{5}{舍利弗經第二}{https://agama.buddhason.org/SN/sn.php?keyword=55.5}
  那時,\twnr{尊者}{200.0}舍利弗去見世尊。抵達後,向世尊\twnr{問訊}{46.0}後,在一旁坐下。世尊對在一旁坐下的尊者舍利弗說這個:

  「舍利弗!這被稱為『\twnr{入流支}{370.0}、入流支』,舍利弗!什麼是『入流支』呢?」

  「\twnr{大德}{45.0}!善人的親近是入流支,正法的聽聞是入流支,\twnr{如理作意}{114.0}是入流支,法、隨法的實踐是入流支。」

  「舍利弗!\twnr{好}{44.0}!好!舍利弗!善人的親近是入流支,正法的聽聞是入流支,如理作意是入流支,法、隨法的實踐是入流支。

  舍利弗!這被稱為『流、流』,舍利弗!什麼是『流』呢?」

  「大德!這\twnr{八支聖道}{525.0}就是流,即:正見、正志、正語、正業、正命、正精進、正念、正定。」

  「舍利弗!好!好!舍利弗!這八支聖道就是流,即:正見……(中略)正定。

  舍利弗!這被稱為『入流者、入流者』,舍利弗!什麼是『入流者』呢?」

  「大德!凡具備這八支聖道者,這位被稱為入流者:他是這樣名、這樣姓的這位尊者。」

  「舍利弗!好!好!舍利弗!凡具備這八支聖道者,這位被稱為入流者:他是這樣名、這樣姓的這位尊者。」



\sutta{6}{6}{侍從官經}{https://agama.buddhason.org/SN/sn.php?keyword=55.6}
  起源於舍衛城。

  當時,眾多\twnr{比丘}{31.0}為\twnr{世尊}{12.0}作衣服的工作:「經過三個月,完成衣服的世尊將出發\twnr{遊行}{61.0}。」

  那時,侍從官梨師達多、富蘭那正以某些應該被作的居住在沙督迦。侍從官梨師達多、富蘭那聽聞:

  「聽說眾多比丘為世尊作衣服的工作:『經過三個月,完成衣服的世尊將出發遊行。』」

  那時,侍從官梨師達多、富蘭那在路上留置男子:「喂!男子!當你如果看見那位到來的世尊、\twnr{阿羅漢}{5.0}、\twnr{遍正覺者}{6.0},那時,你通知我們。」

  站立二、三天的那位男子看見正從遠處到來的世尊。看見後,去見侍從官梨師達多、富蘭那。抵達後,對侍從官梨師達多、富蘭那說這個:

  「\twnr{大德}{45.0}!這位世尊、阿羅漢、遍正覺者他到來,現在是那個你們\twnr{考量的時間}{84.0}。」

  那時,侍從官梨師達多、富蘭那去見世尊。抵達後,向世尊\twnr{問訊}{46.0}後,在後面緊跟隨。

  那時,世尊離開道路後,去某棵樹下。抵達後,在設置的座位坐下。

  侍從官梨師達多、富蘭那向世尊問訊後,在一旁坐下。在一旁坐下的侍從官梨師達多、富蘭那對世尊說這個:

  「大德!當我們聽聞世尊『將從舍衛城出發到憍薩羅國遊行』,那時,我們有不悅意的狀態、憂:『世尊將遠離我們。』

  大德!又,當我們聽聞世尊『已從舍衛城出發到憍薩羅國遊行』,那時,我們有不悅意的狀態、憂:『世尊遠離我們。』

  大德!又,當我們聽聞世尊『將從憍薩羅國出發到末羅遊行』,那時,我們有不悅意的狀態、憂:『世尊將遠離我們。』

  大德!又,當我們聽聞世尊『已從憍薩羅國出發到末羅遊行』,那時,我們有不悅意的狀態、憂:『世尊遠離我們。』

  大德!又,當我們聽聞世尊『將從末羅出發到跋耆遊行』,那時,我們有不悅意的狀態、憂:『世尊將遠離我們。』

  大德!又,當我們聽聞世尊『已從末羅出發到跋耆遊行』,那時,我們有不悅意的狀態、憂:『世尊遠離我們。』

  大德!又,當我們聽聞世尊『將從跋耆出發到迦尸遊行』,那時,我們有不悅意的狀態、憂:『世尊將遠離我們。』

  大德!又,當我們聽聞世尊『已從跋耆出發到迦尸遊行』,那時,我們有不悅意的狀態、憂:『世尊遠離我們。』

  大德!又,當我們聽聞世尊『將從迦尸出發到摩揭陀遊行』,那時,我們有不悅意的狀態、憂:『世尊將遠離我們。』

  大德!又,當我們聽聞世尊『已從迦尸出發到摩揭陀遊行』,那時,我們有不少的不悅意狀態、不少的憂:『世尊遠離我們。』

  大德!又,當我們聽聞世尊『將從摩揭陀出發到迦尸遊行』,那時,我們有悅意的狀態、喜悅:『世尊將接近我們。』

  大德!又,當我們聽聞世尊『已從摩揭陀出發到迦尸遊行』,那時,我們有悅意的狀態、喜悅:『世尊接近我們。』

  大德!又,當我們聽聞世尊『將從迦尸出發到跋耆遊行』,那時,我們有悅意的狀態、喜悅:『世尊將接近我們。』

  大德!又,當我們聽聞世尊『已從迦尸出發到跋耆遊行』,那時,我們有悅意的狀態、喜悅:『世尊接近我們。』

  大德!又,當我們聽聞世尊『將從跋耆出發到末羅遊行』,那時,我們有悅意的狀態、喜悅:『世尊將接近我們。』

  大德!又,當我們聽聞世尊『已從跋耆出發到末羅遊行』,那時,我們有悅意的狀態、喜悅:『世尊接近我們。』

  大德!又,當我們聽聞世尊『將從末羅出發到憍薩羅國遊行』,那時,我們有悅意的狀態、喜悅:『世尊將接近我們。』

  大德!又,當我們聽聞世尊『已從末羅出發到憍薩羅國遊行』,那時,我們有悅意的狀態、喜悅:『世尊接近我們。』

  大德!又,當我們聽聞世尊『將從憍薩羅國出發到舍衛城遊行』,那時,我們有悅意的狀態、喜悅:『世尊將接近我們。』

  大德!又,當我們聽聞世尊『住在舍衛城祇樹林給孤獨園』,那時,我們有不少的悅意的狀態、不少的喜悅:『世尊接近我們。』」

  「侍從官!因此,在這裡,居家生活是障礙,是塵垢之路;出家是\twnr{露地}{385.0},侍從官!還有,足以有你們的不放逸(對你們的不放逸來說是足夠的)。」

  「大德!我們有比這個障礙更障礙的另一個障礙,且更被稱為障礙的。」

  「侍從官!那麼,你們有哪個比這個障礙更障礙的另一個障礙,且更被稱為障礙的呢?」

  「大德!這裡,當憍薩羅國波斯匿王想要出發到遊樂園,我們準備適合當交通工具的那些憍薩羅國波斯匿王的象後,我們使那憍薩羅國波斯匿王所愛的、合意的夫人們坐下,一個在前,一個在後。大德!她們姊妹有像這樣的香氣:猶如香料盒被打開那一刻的;如那位被芳香裝飾之公主的。大德!又,她們姊妹有像這樣的身觸:猶如棉花絨的或木綿花絨的;如那位被安樂養大之公主的。大德!又,在那時,象應該被守護,那姊妹們也應該被守護,自己也應該被守護。大德!又,我們記得(證知)在那姊妹們上是無使惡心生起者。大德!這是我們有比這個障礙更障礙的另一個障礙,且更被稱為障礙的。」

  「侍從官!因此,在這裡,居家生活是障礙,是塵垢之路;出家是露地,侍從官!還有,足以有你們的不放逸。

  侍從官!具備四法的\twnr{聖弟子}{24.0}是\twnr{入流者}{165.0}、不墮\twnr{惡趣}{110.0}法者、\twnr{決定者}{159.0}、\twnr{正覺為彼岸者}{160.0},哪四個?侍從官!這裡,聖弟子在佛上具備\twnr{不壞淨}{233.0}:『像這樣,那位世尊是……(中略)\twnr{天-人們的大師}{11.0}、\twnr{佛陀}{3.0}、\twnr{世尊}{12.0}。』在法上……(中略)在\twnr{僧團}{375.0}上……(中略)以離慳垢之心住於在家,是\twnr{自由施捨者}{348.0}、親手施與者、樂於棄捨者、回應乞求者、\twnr{樂於布施物均分者}{349.0}。侍從官!具備這四法的聖弟子是入流者、不墮惡趣法者、決定者、正覺為彼岸者。

  侍從官!你們在佛上具備不壞淨:『像這樣,那位世尊是……(中略)天-人們的大師、佛陀、世尊。』在法上……(中略)在僧團上……(中略)又,凡在家中任何能施之物,那一切都無差別地被[施與]持戒者、\twnr{善法者}{601.1}。

  侍從官!你怎麼想它:在憍薩羅國中,有多少人等同你們的,即:在布施物均分上呢?」

  「大德!是我們的利得,大德!是我們的善得的:世尊這麼知道我們。」



\sutta{7}{7}{竹門人經}{https://agama.buddhason.org/SN/sn.php?keyword=55.7}
  \twnr{被我這麼聽聞}{1.0}:

  \twnr{有一次}{2.0},\twnr{世尊}{12.0}與大\twnr{比丘}{31.0}\twnr{僧團}{375.0}一起在憍薩羅國進行著遊行,抵達名叫竹門的憍薩羅婆羅門村落。

  那些竹門村的婆羅門\twnr{屋主}{103.0}們聽聞:

  「\twnr{先生}{202.0}!釋迦人之子、從釋迦族出家的\twnr{沙門}{29.0}\twnr{喬達摩}{80.0},與大比丘僧團一起在憍薩羅國進行著遊行,已抵達竹門,又,對那位喬達摩\twnr{尊師}{203.0},有這樣的好名聲被傳播:『像這樣,那位世尊是\twnr{阿羅漢}{5.0}、\twnr{遍正覺者}{6.0}、\twnr{明行具足者}{7.0}、\twnr{善逝}{8.0}、\twnr{世間知者}{9.0}、\twnr{應該被調御人的無上調御者}{10.0}、\twnr{天-人們的大師}{11.0}、\twnr{佛陀}{3.0}、\twnr{世尊}{12.0}。』他以證智自作證後,告知這個包括天、魔、梵的世間;包括沙門婆羅門,包括天-人的\twnr{世代}{38.0},他教導開頭是善的、中間是善的、結尾是善的;\twnr{有意義的}{81.0}、\twnr{有文字的}{82.0}法,他說明完全圓滿、\twnr{遍純淨的梵行}{483.0}。又,有像那樣阿羅漢的看見,\twnr{那就好了}{44.0}!」

  那時,那些竹門村的婆羅門\twnr{屋主}{103.0}們去見世尊。抵達後,一些向世尊\twnr{問訊}{46.0}後,在一旁坐下;一些與世尊一起互相問候。交換應該被互相問候的友好交談後,在一旁坐下;一些向世尊\twnr{合掌}{377.0}鞠躬後,在一旁坐下;一些在世尊的面前告知姓名後,在一旁坐下;一些沈默地在一旁坐下。在一旁坐下的那些竹門村的婆羅門屋主們對世尊說這個:

  「喬達摩尊師!我們有這樣的欲,有這樣的意欲,有這樣的欲求:『願我們居住兒子擁擠的床,願我們享用迦尸的檀香,願使我們持有花環、香料、塗油,願我們受用金銀,願我們以身體的崩解,死後往生\twnr{善趣}{112.0}、天界。』請喬達摩尊師為那些這樣的欲,這樣的意欲,這樣的欲求的我們教導法,如是我們能住於多子之家……(中略)能往生善趣天界。」

  「屋主們!我將為你們教導\twnr{關係自己之法的法門}{x649},你們要聽它!你們\twnr{要好好作意}{43.1}!我將說。」

  「是的,先生!」那些竹門村的婆羅門屋主們回答世尊。

  世尊說這個:

  「屋主們!什麼是關係自己之法的法門?

  屋主們!這裡,聖弟子像這樣深慮:『我是想要活命、不想要死;想要樂、厭逆苦者,如果奪取想要活命、不想要死;想要樂、厭逆苦的我之生命,這對我來說不是所愛的、合意的。那麼,如果我就奪取想要活命、不想要死;想要樂、厭逆苦的他人之生命,對他人來說,那也是非所愛的、不合意的。凡對我來說這個非所愛的、不合意的法,對他人來說這也是非所愛的、不合意的法。凡對我來說這個非所愛的、不合意的法,我如何能給與他人呢!』他像這樣省察後,以自己為離殺生者、以殺生的戒絕勸導他人、稱讚殺生的戒絕,這樣,這是這個的\twnr{身行為}{460.0}三邊被清淨。

  再者,屋主們!聖弟子這樣深慮:『如果對我\twnr{未給予而取}{104.0}被稱為偷盜的,這對我來說不是所愛的、合意的。那麼,如果我對他人未給予而取被稱為偷盜的,對他人來說,那也是非所愛的、不合意的。凡對我來說這個非所愛的、不合意的法,對他人來說這也是非所愛的、不合意的法。凡對我來說這個非所愛的、不合意的法,我如何能給與他人呢!』他像這樣省察後,以自己為離未給予而取者、以未給予而取的戒絕勸導他人、稱讚未給予而取的戒絕,這樣,這是這個的身行為三邊被清淨。

  再者,屋主們!聖弟子這樣深慮:『如果在我的妻子們上性交,這對我來說不是所愛的、合意的。那麼,如果我在他人的妻子們上性交,對他人來說,那也是非所愛的、不合意的。凡對我來說這個非所愛的、不合意的法,對他人來說這也是非所愛的、不合意的法。凡對我來說這個非所愛的、不合意的法,我如何能給與他人呢!』他像這樣省察後,以自己為離\twnr{邪淫}{105.0}者、以邪淫的戒絕勸導他人、稱讚邪淫的戒絕,這樣,這是這個的身行為三邊被清淨。

  再者,屋主們!聖弟子這樣深慮:『如果以\twnr{妄語}{106.0}破壞我的利益,這對我來說不是所愛的、合意的。那麼,如果我以妄語破壞他人的利益,對他人來說,那也是非所愛的、不合意的。凡對我來說這個非所愛的、不合意的法,對他人來說這也是非所愛的、不合意的法。凡對我來說這個非所愛的、不合意的法,我如何能給與他人呢!』他像這樣省察後,以自己為離妄語者、以妄語的戒絕勸導他人、稱讚妄語的戒絕,這樣,這是這個的語行為三邊被清淨。

  再者,屋主們!聖弟子這樣深慮:『如果以\twnr{離間語}{234.0}使我與朋友破裂,這對我來說不是所愛的、合意的。那麼,如果我以離間語使他人與朋友破裂,對他人來說,那也是非所愛的、不合意的……(中略)。』這樣,這是這個的語行為三邊被清淨。

  再者,屋主們!聖弟子這樣深慮:『如果以\twnr{粗惡語}{235.0}對我說話,這對我來說不是所愛的、合意的。那麼,如果我以粗惡語對他人說話,對他人來說,那也是非所愛的、不合意的。凡我不愛、不合意之法……(中略)。』這樣,這是這個的語行為三邊被清淨。

  再者,屋主們!聖弟子這樣深慮:『如果以\twnr{雜穢語}{236.0}對我說話,這對我來說不是所愛的、合意的。那麼,如果我以雜穢語對他人說話,對他人來說,那也是非所愛的、不合意的。凡對我來說這個非所愛的、不合意的法,對他人來說這也是非所愛的、不合意的法。凡對我來說這個非所愛的、不合意的法,我如何能給與他人呢!』他像這樣省察後,以自己為離雜穢語者、以雜穢語的戒絕勸導他人、稱讚雜穢語的戒絕,這樣,這是這個的語行為三邊被清淨。

  他在佛上具備不壞淨:『像這樣,那位世尊是……(中略)天-人們的大師、佛陀、世尊。』在法上……(中略)在僧團上具備不壞淨:『世尊的弟子僧團是\twnr{善行者}{518.0}……(中略)為世間的無上\twnr{福田}{101.0}。』具備聖者喜愛的諸戒:無毀壞的……(中略)轉起定的。

  屋主們!當聖弟子具備這些七善法、這些四希望處,當他希望時,他就能由自己\twnr{記說}{179.0}自己:『我是地獄已盡者,畜生界已盡者,\twnr{餓鬼界}{362.0}已盡者,\twnr{苦界}{109.0}、\twnr{惡趣}{110.0}、\twnr{下界}{111.0}已盡者,我是\twnr{入流者}{165.0}、不墮惡趣法者、\twnr{決定者}{159.0}、\twnr{正覺為彼岸者}{160.0}。』」

  在這麼說時,竹門人的婆羅門屋主們對世尊說這個:

  「太偉大了,喬達摩尊師!……(中略)這些我們\twnr{歸依}{284.0}喬達摩尊師、法、比丘僧團,請喬達摩尊師記得我們為\twnr{優婆塞}{98.0},從今天起\twnr{已終生歸依}{64.0}。」



\sutta{8}{8}{磚屋經第一}{https://agama.buddhason.org/SN/sn.php?keyword=55.8}
  \twnr{被我這麼聽聞}{1.0}:

  \twnr{有一次}{2.0},世尊住在\twnr{親戚村}{709.0}的磚屋中。

  那時,\twnr{尊者}{200.0}阿難去見世尊。抵達後,向世尊\twnr{問訊}{46.0}後,在一旁坐下。在一旁坐下的尊者阿難對世尊說這個:

  「\twnr{大德}{45.0}!名叫薩哈的\twnr{比丘}{31.0}死了,他的趣處是什麼?來世是什麼?

   大德!名叫難陀的比丘尼死了,她的趣處是什麼?來世是什麼?

   大德!名叫善施的優婆塞死了,他的趣處是什麼?來世是什麼?

   大德!名叫善生的優婆夷死了,她的趣處是什麼?來世是什麼?」

  「阿難!已命終的薩哈比丘以諸漏的滅盡,以證智自作證後,在當生中\twnr{進入後住於}{66.0}無漏心解脫、慧解脫。

  阿難!已命終的難陀比丘尼以\twnr{五下分結}{134.0}的滅盡,成為\twnr{化生}{346.0}者、在那裡般涅槃者、不從那個世間返還者。

  阿難!已命終的善施優婆塞以三結的遍盡,以貪、瞋、癡薄的狀態,為\twnr{一來}{208.0}者,只回來這個世間一次後,將作苦的終結。

  阿難!已命終的善生優婆夷以三結的遍盡,為\twnr{入流者}{165.0}、不墮\twnr{惡趣}{110.0}法者、\twnr{決定者}{159.0}、\twnr{正覺為彼岸者}{160.0}。

  阿難!又,這非\twnr{不可思議}{206.0}:凡生為人的會命終。如果在每一位已命終時,來見我後你們詢問這件事,阿難!這對如來也會是傷害。阿難!因此,在這裡,我將教導名叫\twnr{法鏡}{556.0}之\twnr{法的教說}{562.1},具備那個的\twnr{聖弟子}{24.0},當希望時,就能由自己記說自己:『我是地獄已盡者,畜生界已盡者,\twnr{餓鬼界}{362.0}已盡者,\twnr{苦界}{109.0}、\twnr{惡趣}{110.0}、\twnr{下界}{111.0}已盡者,我是入流者、不墮惡趣法者、決定者、正覺為彼岸者。』

  阿難!而什麼是法鏡之法的教說,凡具備的聖弟子,當希望時,就以自己對自己記說:『我是地獄已盡者,畜生界已盡者,餓鬼界已盡者,苦界、惡趣、下界已盡者,我是入流者、不墮惡趣法者、決定者、正覺為彼岸者。』呢?

  阿難!這裡,聖弟子在佛上具備不壞淨:『像這樣,那位世尊是……(中略)\twnr{天-人們的大師}{11.0}、\twnr{佛陀}{3.0}、\twnr{世尊}{12.0}。』在法上……(中略)在\twnr{僧團}{375.0}上……(中略)具備聖者喜愛的諸戒:無毀壞的……(中略)轉起定的。

  阿難!這是那個法鏡之法的教說,凡具備的聖弟子,當希望時,就以自己對自己記說:『我是地獄已盡者,畜生界已盡者,餓鬼界已盡者,苦界、惡趣、下界已盡者,我是入流者、不墮惡趣法者、決定者、正覺為彼岸者。』」

  (三經都同一因緣)



\sutta{9}{9}{磚屋經第二}{https://agama.buddhason.org/SN/sn.php?keyword=55.9}
  在一旁坐下的\twnr{尊者}{200.0}阿難對\twnr{世尊}{12.0}說這個:

  「\twnr{大德}{45.0}!名叫無憂的\twnr{比丘}{31.0}死了,他的趣處是什麼?來世是什麼?大德!名叫無憂的比丘尼死了……(中略)大德!名叫無憂的\twnr{優婆塞}{98.0}死了……(中略)大德!名叫無憂的\twnr{優婆夷}{99.0}死了,她的趣處是什麼?來世是什麼?」

  「阿難!已命終的無憂比丘以諸漏的滅盡,以證智自作證後,在當生中\twnr{進入後住於}{66.0}無漏\twnr{心解脫}{16.0}、\twnr{慧解脫}{539.0}。……(中略)(與先前的解說同一因緣)

  阿難!這是那個\twnr{法鏡}{556.0}之\twnr{法的教說}{562.1},凡具備的聖弟子,當希望時,就以自己對自己記說:『我是地獄已盡者,畜生界已盡者,\twnr{餓鬼界}{362.0}已盡者,\twnr{苦界}{109.0}、\twnr{惡趣}{110.0}、\twnr{下界}{111.0}已盡者,我是\twnr{入流者}{165.0}、不墮惡趣法者、\twnr{決定者}{159.0}、\twnr{正覺為彼岸者}{160.0}。』」



\sutta{10}{10}{磚屋經第三}{https://agama.buddhason.org/SN/sn.php?keyword=55.10}
  在一旁坐下的\twnr{尊者}{200.0}阿難對\twnr{世尊}{12.0}說這個:

  「\twnr{大德}{45.0}!名叫大鹿的優婆塞在\twnr{親戚村}{709.0}死了,他的趣處是什麼?來世是什麼?大德!名叫迦哩巴的優婆塞在親戚村已命終……(中略)大德!名叫尼迦達的優婆塞在親戚村已命終……(中略)大德!名叫迦低沙哈的優婆塞在親戚村已命終……(中略)大德!名叫滿足的優婆塞在親戚村已命終……(中略)大德!名叫善滿足的優婆塞在親戚村已命終……(中略)大德!名叫吉祥的優婆塞在親戚村已命終……(中略)大德!名叫善吉祥的優婆塞在親戚村已命終,他的趣處是什麼?來世是什麼?」

  「阿難!已命終的大鹿優婆塞以\twnr{五下分結}{134.0}的滅盡,成為\twnr{化生}{346.0}者、在那裡般涅槃者、不從那個世間返還者。阿難!已命終的迦哩巴優婆塞……(中略)阿難!已命終的尼迦達優婆塞……(中略)阿難!已命終的迦低沙哈優婆塞……(中略)阿難!已命終的滿足優婆塞……(中略)阿難!已命終的善滿足優婆塞……(中略)阿難!已命終的吉祥優婆塞……(中略)阿難!已命終的善吉祥優婆塞以五下分結的滅盡,成為化生者、在那裡般涅槃者、不從那個世間返還者。(全都應作同一去處)

  阿難!超過五十位在親戚村死去的優婆塞以五下分結的滅盡,成為化生者、在那裡般涅槃者、不從那個世間返還者。

  阿難!九十多位在親戚村死去的優婆塞以三結的遍盡,以貪、瞋、癡薄的狀態,為\twnr{一來}{208.0}者,只回來這個世間一次後,將作苦的終結。

  阿難!五百零六位在親戚村死去的優婆塞以三結的遍盡,為\twnr{入流者}{165.0}、不墮\twnr{惡趣}{110.0}法者、\twnr{決定者}{159.0}、\twnr{正覺為彼岸者}{160.0}。

  阿難!又,這非\twnr{不可思議}{206.0}:凡生為人的會命終。如果在每一位已命終時,來見我後你們詢問這件事,阿難!這對如來也會是傷害。阿難!因此,在這裡,我將教導名叫\twnr{法鏡}{556.0}之\twnr{法的教說}{562.1},凡具備的聖弟子,當希望時,就以自己對自己記說:『我是地獄已盡者,畜生界已盡者,\twnr{餓鬼界}{362.0}已盡者,\twnr{苦界}{109.0}、\twnr{惡趣}{110.0}、\twnr{下界}{111.0}已盡者,我是入流者、不墮惡趣法者、決定者、正覺為彼岸者。』

  阿難!而什麼是法鏡之法的教說,凡具備的聖弟子,當希望時,就以自己對自己記說:『我是地獄已盡者,畜生界已盡者,餓鬼界已盡者,苦界、惡趣、下界已盡者,我是入流者、不墮惡趣法者、決定者、正覺為彼岸者。』呢?阿難!這裡,聖弟子在佛上具備不壞淨:『像這樣,那位世尊是……(中略)\twnr{天-人們的大師}{11.0}、\twnr{佛陀}{3.0}、\twnr{世尊}{12.0}。』在法上……(中略)在\twnr{僧團}{375.0}上……(中略)具備聖者喜愛的諸戒:無毀壞的……(中略)轉起定的。阿難!這是那個法鏡之法的教說,凡具備的聖弟子,當希望時,就以自己對自己記說:『我是地獄已盡者,畜生界已盡者,餓鬼界已盡者,苦界、惡趣、下界已盡者,我是入流者、不墮惡趣法者、決定者、正覺為彼岸者。』」[\ccchref{DN.16}{https://agama.buddhason.org/DN/dm.php?keyword=16}, 156段]

  竹門品第一,其\twnr{攝頌}{35.0}:

  「王、立足處、長壽,舍利弗二則,

   侍從官、竹門人,磚屋三則。」





\pin{國王的園林品}{11}{20}
\sutta{11}{11}{千位比丘尼的僧團經}{https://agama.buddhason.org/SN/sn.php?keyword=55.11}
  \twnr{有一次}{2.0},\twnr{世尊}{12.0}住在舍衛城國王的園林。

  那時,千位比丘尼的僧團去見世尊。抵達後,向世尊\twnr{問訊}{46.0}後,在一旁站立。世尊對在一旁站立的那些比丘尼說這個:

  「比丘尼們!具備四法的\twnr{聖弟子}{24.0}是\twnr{入流者}{165.0}、不墮\twnr{惡趣}{110.0}法者、\twnr{決定者}{159.0}、\twnr{正覺為彼岸者}{160.0},哪四個?比丘尼們!這裡,聖弟子在佛上具備\twnr{不壞淨}{233.0}:『像這樣,那位世尊是……(中略)\twnr{天-人們的大師}{11.0}、\twnr{佛陀}{3.0}、\twnr{世尊}{12.0}。』在法上……(中略)在\twnr{僧團}{375.0}上……(中略)具備聖者喜愛的諸戒:無毀壞的……(中略)轉起定的。比丘尼們!具備這四法的聖弟子是入流者、不墮惡趣法者、決定者、正覺為彼岸者。」



\sutta{12}{12}{婆羅門經}{https://agama.buddhason.org/SN/sn.php?keyword=55.12}
  起源於舍衛城。

  「\twnr{比丘}{31.0}們!婆羅門們\twnr{安立}{143.0}名叫向上(導向上升)道跡,他們這麼勸導弟子:『喂!男子!來!在清晨起來後,請你面向東走,那個你請不要避開坑洞、斷崖、殘株、荊棘處、溝池、污水坑,你會跌落之處,就在那裡,你應該期待死亡。喂!男子!這樣,以身體的崩解,死後你將往生\twnr{善趣}{112.0}、天界。』

  比丘們!但這是婆羅門們的,這是愚蠢的行走,這是愚昧的行走,它不對\twnr{厭}{15.0}、不對\twnr{離貪}{77.0}、不對\twnr{滅}{68.0}、不對寂靜、不對證智、不對\twnr{正覺}{185.1}、不對涅槃轉起。

  比丘們!而在聖者之律中,我安立向上道跡:轉起\twnr{一向}{168.0}\twnr{厭}{15.0}、離貪、滅、寂靜、證智、正覺、涅槃。比丘們!而哪個是那個向上道跡:對一向的厭?……(中略)對涅槃轉起?比丘們!這裡,\twnr{聖弟子}{24.0}在佛上具備\twnr{不壞淨}{233.0}:『像這樣,那位\twnr{世尊}{12.0}是\twnr{阿羅漢}{5.0}、\twnr{遍正覺者}{6.0}……(中略)\twnr{天-人們的大師}{11.0}、\twnr{佛陀}{3.0}、\twnr{世尊}{12.0}。』在法上……(中略)在\twnr{僧團}{375.0}上……(中略)具備聖者喜愛的諸戒:無毀壞的……(中略)轉起定的。

  比丘們!這是那個向上道跡:轉起一向的厭……(中略)涅槃。」



\sutta{13}{13}{上座阿難經}{https://agama.buddhason.org/SN/sn.php?keyword=55.13}
  \twnr{有一次}{2.0},\twnr{尊者}{200.0}阿難與尊者舍利弗住在舍衛城祇樹林給孤獨園。

  那時,尊者舍利弗傍晚時,從\twnr{獨坐}{92.0}出來,去見尊者阿難。抵達後,與尊者阿難一起互相問候。交換應該被互相問候的友好交談後,在一旁坐下。在一旁坐下的尊者舍利弗對尊者阿難說這個:

  「阿難\twnr{學友}{201.0}!以幾法的捨斷,幾法的具備之因,這樣,這個人被\twnr{世尊}{12.0}記說為\twnr{入流者}{165.0}、不墮\twnr{惡趣}{110.0}法者、\twnr{決定者}{159.0}、\twnr{正覺為彼岸者}{160.0}呢?」

  「學友!以四法的捨斷,四法的具備之因,這樣,這個人被世尊記說為入流者、不墮惡趣法者、決定者、正覺為彼岸者。」

  「哪四個?學友!在佛上具備像這樣無\twnr{淨信}{340.0}的\twnr{未聽聞的一般人}{74.0}以身體的崩解,死後往生\twnr{苦界}{109.0}、惡趣、\twnr{下界}{111.0}、地獄,在佛上像那樣的無淨信不存在。學友!而在佛上具備像這樣\twnr{不壞淨}{233.0}的\twnr{有聽聞的聖弟子}{24.0}以身體的崩解,死後往生\twnr{善趣}{112.0}、天界,在佛上像那樣的不壞淨存在:『像這樣,那位世尊……(中略)\twnr{天-人們的大師}{11.0}、\twnr{佛陀}{3.0}、世尊。』

  學友!而在法上具備像這樣無淨信的未聽聞的一般人以身體的崩解,死後往生苦界、惡趣、下界、地獄,在法上像那樣的無淨信不存在。學友!而在法上具備像這樣不壞淨的有聽聞的聖弟子以身體的崩解,死後往生善趣、天界,在法上像那樣的不壞淨存在:『被世尊善說的法是……(中略)應該被智者各自經驗的。』

  學友!而在\twnr{僧團}{375.0}上具備像這樣無淨信的未聽聞的一般人以身體的崩解,死後往生苦界、惡趣、下界、地獄,在僧團上像那樣的無淨信不存在。學友!而在僧團上具備像這樣不壞淨的有聽聞的聖弟子以身體的崩解,死後往生善趣、天界,在僧團上像那樣的不壞淨存在:『世尊的弟子僧團是\twnr{善行者}{518.0}……(中略)為世間的無上\twnr{福田}{101.0}。』

  學友!而具備像這樣\twnr{破戒}{708.0}的未聽聞的一般人以身體的崩解,死後往生苦界、惡趣、下界、地獄,像那樣的破戒不存在。學友!而具備像這樣聖者喜愛的諸戒的有聽聞的聖弟子以身體的崩解,死後往生善趣、天界,像那樣的聖者喜愛的諸戒存在:『無毀壞的……(中略)轉起定的。

  學友!以這四法的捨斷,這四法的具備之因,這樣,這個被世尊記說為入流者、不墮惡趣法者、決定者、正覺為彼岸者。」



\sutta{14}{14}{害怕惡趣經}{https://agama.buddhason.org/SN/sn.php?keyword=55.14}
  「\twnr{比丘}{31.0}們!具備四法的\twnr{聖弟子}{24.0}都是已超越害怕\twnr{惡趣}{110.0}者,哪四個?比丘們!這裡,聖弟子在佛上具備\twnr{不壞淨}{233.0}:『像這樣,那位世尊是……(中略)\twnr{天-人們的大師}{11.0}、\twnr{佛陀}{3.0}、\twnr{世尊}{12.0}。』在法上……(中略)在\twnr{僧團}{375.0}上……(中略)具備聖者喜愛的諸戒:無毀壞的……(中略)轉起定的。比丘們!具備這四法的聖弟子都是已超越害怕惡趣者。」



\sutta{15}{15}{害怕惡趣下界經}{https://agama.buddhason.org/SN/sn.php?keyword=55.15}
  「\twnr{比丘}{31.0}們!具備四法的\twnr{聖弟子}{24.0}都是已超越害怕\twnr{惡趣}{110.0}、\twnr{下界}{111.0}者,哪四個?比丘們!這裡,聖弟子在佛上具備\twnr{不壞淨}{233.0}:『像這樣,那位世尊是……(中略)\twnr{天-人們的大師}{11.0}、\twnr{佛陀}{3.0}、\twnr{世尊}{12.0}。』在法上……(中略)在\twnr{僧團}{375.0}上……(中略)具備聖者喜愛的諸戒:無毀壞的……(中略)轉起定的。比丘們!具備這四法的聖弟子都是已超越害怕惡趣、下界者。」



\sutta{16}{16}{朋友同事經第一}{https://agama.buddhason.org/SN/sn.php?keyword=55.16}
  「\twnr{比丘}{31.0}們!凡你們會憐愍,以及凡他們會想應該被聽聞的朋友,或同事,或親族,或有血緣者,比丘們!他們應該在四\twnr{入流支}{370.0}上被[你們]勸導、應該被使確立、應該被使建立,哪四個?應該在佛上不壞淨被勸導、應該被使確立、應該被使建立:『像這樣,那位\twnr{世尊}{12.0}……(中略)\twnr{天-人們的大師}{11.0}、\twnr{佛陀}{3.0}、世尊。』在法上……(中略)在\twnr{僧團}{375.0}上……(中略)應該在聖者喜愛的諸戒上被勸導、應該被使確立、應該被使建立:『無毀壞的……(中略)轉起定的。

  比丘們!凡你們會憐愍,以及凡他們會想應該被聽聞的朋友,或同事,或親族,或有血緣者,比丘們!他們應該在這四入流支上被勸導、應該被使確立、應該被使建立。」



\sutta{17}{17}{朋友同事經第二}{https://agama.buddhason.org/SN/sn.php?keyword=55.17}
  「\twnr{比丘}{31.0}們!凡你們會憐愍,以及凡他們會想應該被聽聞的朋友,或同事,或親族,或有血緣者,比丘們!他們應該在四\twnr{入流支}{370.0}上被[你們]勸導、應該被使確立、應該被使建立,哪四個?在佛上不壞淨應該被勸導、應該被使確立、應該被使建立:『像這樣,那位\twnr{世尊}{12.0}……(中略)\twnr{天-人們的大師}{11.0}、\twnr{佛陀}{3.0}、世尊。』

  比丘們!會有\twnr{四大}{646.0}的變異:地界的、水界的、火界的、風界的,然而不會有在佛上具備不壞淨聖弟子的變異。在這裡,這個變異是:那位確實在佛上具備不壞淨的聖弟子將往生地獄、畜生胎、\twnr{餓鬼界}{362.0}中,\twnr{這不存在可能性}{650.0}。

  在法上……(中略)在\twnr{僧團}{375.0}上……(中略)應該在聖者喜愛的諸戒上被勸導、應該被使確立、應該被使建立:『無毀壞的……(中略)轉起定的。

  比丘們!會有四大的變異:地界的、水界的、火界的、風界的,然而不會有具備聖者喜愛的諸戒聖弟子的變異。在這裡,這個變異是:那位確實具備聖者喜愛的諸戒的聖弟子將往生地獄、畜生界、餓鬼界,這不存在可能性。

  比丘們!凡你們會憐愍,以及凡他們會想應該被聽聞的朋友,或同事,或親族,或有血緣者,比丘們!他們應該在這些四入流支上被勸導、應該被使確立、應該被使建立。」[≃\suttaref{SN.47.48}, \suttaref{SN.56.26}, \ccchref{AN.3.76}{https://agama.buddhason.org/AN/an.php?keyword=3.76}]



\sutta{18}{18}{天行經第一}{https://agama.buddhason.org/SN/sn.php?keyword=55.18}
  起源於舍衛城。

  那時,\twnr{尊者}{200.0}大目揵連就猶如有力氣的男子伸直彎曲的手臂,或彎曲伸直的手臂,就像這樣在祇樹園消失,出現在三十三天中。

  那時,眾多三十三天眾的天神去見尊者大目揵連。抵達後,向尊者大目揵連\twnr{問訊}{46.0}後,在一旁站立。尊者大目揵連對在一旁站立的那些天神說這個:

  「\twnr{朋友}{201.0}!在佛上以\twnr{不壞淨}{233.0}之具備是\twnr{好的}{44.0}:『像這樣,那位\twnr{世尊}{12.0}是……(\twnr{中略}{x650})\twnr{天-人們的大師}{11.0}、\twnr{佛陀}{3.0}、世尊。』朋友!以在佛上不壞淨具備之因,這樣,這裡一些眾生以身體的崩解,死後往生\twnr{善趣}{112.0}、天界。

  朋友!在法上……(中略)在\twnr{僧團}{375.0}上……(中略)朋友!以聖者喜愛的諸戒……(中略)轉起定的。朋友!以聖者喜愛的諸戒具備之因,這樣,這裡一些眾生以身體的崩解,死後往生善趣、天界。」

  「親愛的目揵連尊師!在佛上以不壞淨之具備是好的:『像這樣,那位世尊是……(中略)天-人們的大師、佛陀、世尊。』親愛的目揵連尊師!以在佛上不壞淨具備之因,這樣,這裡一些眾生以身體的崩解,死後往生善趣、天界。

  親愛的目揵連尊師!在法上……(中略)在\twnr{僧團}{375.0}上……(中略)以聖者喜愛的諸戒……(中略)轉起定的。親愛的目揵連尊師!以聖者喜愛的諸戒具備之因,這樣,這裡一些眾生以身體的崩解,死後往生善趣、天界。」[≃\suttaref{SN.40.10}]



\sutta{19}{19}{天行經第二}{https://agama.buddhason.org/SN/sn.php?keyword=55.19}
  起源於舍衛城。

  那時,\twnr{尊者}{200.0}大目揵連就猶如有力氣的男子伸直彎曲的手臂,或彎曲伸直的手臂,就像這樣在祇樹園消失,出現在三十三天中。

  那時,眾多三十三天眾的天神去見尊者大目揵連。抵達後,向尊者大目揵連\twnr{問訊}{46.0}後,在一旁站立。尊者大目揵連對在一旁站立的那些天神說這個:

  「\twnr{朋友}{201.0}!在佛上以不壞淨之具備是好的:『像這樣,那位\twnr{世尊}{12.0}是……(中略)\twnr{天-人們的大師}{11.0}、\twnr{佛陀}{3.0}、世尊。』朋友!以在佛上不壞淨具備之因,這樣,這裡一些眾生以身體的崩解,死後已往生善趣、天界。

  朋友!在法上……(中略)在\twnr{僧團}{375.0}上……(中略)以聖者喜愛的諸戒……(中略)轉起定的。朋友!以聖者喜愛的諸戒具備之因,這樣,這裡一些眾生以身體的崩解,死後已往生善趣、天界。」

  「親愛的目揵連尊師!在佛上以不壞淨之具備是好的:『像這樣,那位世尊是……(中略)天-人們的大師、佛陀、世尊。』親愛的目揵連尊師!以在佛上不壞淨具備之因,這樣,這裡一些眾生以身體的崩解,死後已往生善趣、天界。

  親愛的目揵連尊師!在法上……(中略)在\twnr{僧團}{375.0}上……(中略)以聖者喜愛的諸戒……(中略)轉起定的。親愛的目揵連尊師!以聖者喜愛的諸戒具備之因,這樣,這裡一些眾生以身體的崩解,死後已往生\twnr{善趣}{112.0}、天界。」[≃\suttaref{SN.40.10}]



\sutta{20}{20}{天行經第三}{https://agama.buddhason.org/SN/sn.php?keyword=55.20}
  那時,\twnr{世尊}{12.0}就猶如有力氣的男子伸直彎曲的手臂,或彎曲伸直的手臂,就像這樣在祇樹園消失,出現在三十三天中。

  那時,眾多三十三天眾的天神去見世尊。抵達後,向世尊\twnr{問訊}{46.0}後,在一旁站立。世尊對在一旁站立的那些天神說這個:

  「\twnr{朋友}{201.0}!在佛上以不壞淨之具備是好的:『像這樣,那位世尊是……(中略)\twnr{天-人們的大師}{11.0}、\twnr{佛陀}{3.0}、世尊。』朋友!以在佛上不壞淨具備之因,這樣,這裡一些眾生為\twnr{入流者}{165.0}、不墮\twnr{惡趣}{110.0}法者、\twnr{決定者}{159.0}、\twnr{正覺為彼岸者}{160.0}。

  朋友!在法上……(中略)在\twnr{僧團}{375.0}上……(中略)以聖者喜愛的諸戒……(中略)轉起定的。朋友!以聖者喜愛的諸戒具備之因,這樣,這裡一些眾生為入流者、不墮惡趣法者、決定者、正覺為彼岸者。」

  「\twnr{親愛的先生}{204.0}!在佛上以不壞淨之具備是好的:『像這樣,那位世尊是……(中略)天-人們的大師、佛陀、世尊。』親愛的先生!以在佛上不壞淨具備之因,這樣,人們為入流者、不墮惡趣法者、決定者、正覺為彼岸者。

  親愛的先生!在法上……(中略)在\twnr{僧團}{375.0}上……(中略)以聖者喜愛的諸戒……(中略)親愛的先生!以聖者喜愛的諸戒具備之因,這樣,人們為入流者、不墮惡趣法者、決定者、正覺為彼岸者。」

  國王的園林品第二,其\twnr{攝頌}{35.0}:

  「千、婆羅門、阿難,惡趣二則在後,

   朋友同事兩說,三則天行。」





\pin{色勒那尼品}{21}{30}
\sutta{21}{21}{摩訶男經第一}{https://agama.buddhason.org/SN/sn.php?keyword=55.21}
  \twnr{被我這麼聽聞}{1.0}:

  \twnr{有一次}{2.0},世尊住在釋迦族人的迦毘羅衛城尼拘律園。

  那時,釋迦族人摩訶男去見世尊。抵達後,向世尊\twnr{問訊}{46.0}後,在一旁坐下。在一旁坐下的釋迦族人摩訶男對世尊說這個:

  「\twnr{大德}{45.0}!這裡,迦毘羅衛城是繁榮的,同時也富裕的,以及人多的、人雜亂的。大德!那個我拜訪世尊或值得尊敬的\twnr{比丘}{31.0}們後,在傍晚當進入迦毘羅衛城時,遇到遊走的象,也遇到遊走的馬,也遇到遊走的車,也遇到遊走的貨車,也遇到遊走的人。大德!在那時,對那個我來說就忘失關於佛之念,忘失關於法之念,忘失關於僧伽之念。大德!那個我這麼想:『如果在這時我死去,我的趣處是什麼?來世是什麼?』」

  「摩訶男!你不要害怕,摩訶男!你不要害怕,你將有無惡的死、無惡的死亡(你的無惡的死、無惡的死亡將存在)。摩訶男!凡任何長久心已遍\twnr{修習}{94.0}信、心已遍修習戒、心已遍修習所聞的、心已遍修習施捨、心已遍修習慧者,他的這身體:有色的、\twnr{四大}{646.0}的、父母生成的、米粥積聚的、無常-塗身-\twnr{按摩}{967.0}-破壞-分散法,就在這裡,烏鴉吃,或鷲吃,或鷹吃,或狗吃,或狐狼吃,或許多種生出的蟲吃,而凡長久他的心已遍修習信……(中略)心已遍修習慧者,他是走到向上的,走到殊勝的。

  摩訶男!猶如男子投酥陶瓶或油陶瓶入深水池後破裂,在那裡,凡是碎片或破片,它是走到向下的,而在那裡,凡是酥或油,它是走到殊勝的。同樣的,摩訶男!凡長久心已遍修習信……(中略)心已遍修習慧者,他的這身體:有色的、四大的、父母生成的、米粥積聚的、無常-塗身-按摩-破壞-分散法,就在這裡,烏鴉吃,或鷲吃,或鷹吃,或狗吃,或狐狼吃,或許多種生出的蟲吃,而凡長久他的心已遍修習信……心已遍修習慧者,他是走到向上的,走到殊勝的。

  摩訶男!又,長久你的心已遍修習信……心已遍修習慧,摩訶男!你不要害怕,摩訶男!你不要害怕,你將有無惡的死、無惡的死亡。」



\sutta{22}{22}{摩訶男經第二}{https://agama.buddhason.org/SN/sn.php?keyword=55.22}
  \twnr{被我這麼聽聞}{1.0}:

  \twnr{有一次}{2.0},\twnr{世尊}{12.0}住在釋迦族人的迦毘羅衛城尼拘律園。

  那時,釋迦族人摩訶男去見世尊。抵達後,向世尊\twnr{問訊}{46.0}後,在一旁坐下。在一旁坐下的釋迦族人摩訶男對世尊說這個:

  「大德!這裡,迦毘羅衛城是繁榮的,同時也富裕的,以及人多的、人雜亂的。大德!那個我拜訪世尊或值得尊敬的\twnr{比丘}{31.0}們後,在傍晚當進入迦毘羅衛城時,遇到遊走的象,也遇到遊走的馬,也遇到遊走的車,也遇到遊走的貨車,也遇到遊走的人。大德!在那時,對那個我來說就忘失關於佛之念,忘失關於法之念,忘失關於僧伽之念。大德!那個我這麼想:『如果在這時我死去,我的趣處是什麼?來世是什麼?』」

  「摩訶男!你不要害怕,摩訶男!你不要害怕,你將有無惡的死、無惡的死亡(你的無惡的死、無惡的死亡將存在)。

  摩訶男!具備四法的聖弟子是傾向涅槃的、斜向涅槃的、坡斜向涅槃的,哪四個?摩訶男!這裡,聖弟子在佛上具備\twnr{不壞淨}{233.0}:『像這樣,那位世尊是……(中略)\twnr{天-人們的大師}{11.0}、\twnr{佛陀}{3.0}、世尊。』在法上……(中略)在\twnr{僧團}{375.0}上……(中略)具備聖者喜愛的諸戒:無毀壞的……(中略)轉起定的。

  摩訶男!猶如樹木是傾向東的、斜向東的、坡斜向東的,被切斷根的它往哪邊倒下?」

  「大德!往傾向處,往斜向處,往坡斜向處。」

  「同樣的,摩訶男!具備這四法的聖弟子是傾向涅槃的、斜向涅槃的、坡斜向涅槃的。」



\sutta{23}{23}{釋迦族人喬塔經}{https://agama.buddhason.org/SN/sn.php?keyword=55.23}
  起源於迦毘羅衛城。

  那時,釋迦族人摩訶男去見釋迦族人喬塔。抵達後,對釋迦族人喬塔說這個:

  「喬塔!你了知具備幾法的個人為\twnr{入流者}{165.0}、不墮\twnr{惡趣}{110.0}法者、\twnr{決定者}{159.0}、\twnr{以正覺為彼岸者}{160.0}呢?」

  「摩訶男!我了知具備三法的個人為入流者、不墮惡趣法者、決定者、以正覺為彼岸者,哪三個?摩訶男!這裡,\twnr{聖弟子}{24.0}在佛上具備不壞淨:『像這樣,那位世尊……(中略)\twnr{天-人們的大師}{11.0}、\twnr{佛陀}{3.0}、世尊。』在法上……(中略)在\twnr{僧團}{375.0}上具備不壞淨:『世尊的弟子僧團是\twnr{善行者}{518.0}……(中略)為世間的無上\twnr{福田}{101.0}。』摩訶男!我了知具備這三法的個人為入流者、不墮惡趣法者、決定者、以正覺為彼岸者。摩訶男!那麼,你了知具備幾法的個人為入流者、不墮惡趣法者、決定者、以正覺為彼岸者呢?」

  「喬塔!我了知具備四法的個人為入流者、不墮惡趣法者、決定者、以正覺為彼岸者,哪四個?喬塔!這裡,聖弟子在佛上具備不壞淨:『像這樣,那位世尊……(中略)天-人們的大師、佛陀、世尊。』在法上……(中略)在僧團上……(中略)具備聖者喜愛的諸戒:無毀壞的……(中略)轉起定的。喬塔!我了知具備這四法的個人為入流者、不墮惡趣法者、決定者、以正覺為彼岸者。」

  「請你等一下,摩訶男!請等一下,摩訶男!僅世尊能知道具備或不具備這些法。」

  「喬塔!我們走,讓我們去見世尊。抵達後,告知世尊這件事。」

  那時,釋迦族人摩訶男與釋迦族人喬塔去見世尊。抵達後,向世尊\twnr{問訊}{46.0}後,在一旁坐下。在一旁坐下的釋迦族人摩訶男對世尊說這個:

  「\twnr{大德}{45.0}!這裡,我去見釋迦族人喬塔。抵達後,對釋迦族人喬塔說這個:『喬塔!你了知具備幾法的個人為入流者、不墮惡趣法者、決定者、以正覺為彼岸者嗎?』大德!在這麼說時,釋迦族人喬塔對我說這個:『摩訶男!我了知具備三法的個人為入流者、不墮惡趣法者、決定者、以正覺為彼岸者,哪三個?摩訶男!這裡,聖弟子在佛上具備不壞淨:「像這樣,那位世尊……(中略)天-人們的大師、佛陀、世尊。」對法……(中略)在僧團上具備不壞淨:「世尊的弟子僧團是善行者……(中略)為世間的無上福田。」摩訶男!我了知具備這三法的個人為入流者、不墮惡趣法者、決定者、以正覺為彼岸者。摩訶男!那麼,你了知具備幾法的個人為入流者、不墮惡趣法者、決定者、以正覺為彼岸者呢?』大德!在這麼說時,我對釋迦族人喬塔說這個:『喬塔!我了知具備四法的個人為入流者、不墮惡趣法者、決定者、以正覺為彼岸者,哪四個?喬塔!這裡,聖弟子在佛上具備不壞淨:「像這樣,那位世尊……(中略)天-人們的大師、佛陀、世尊。」在法上……(中略)在\twnr{僧團}{375.0}上……(中略)具備聖者喜愛的諸戒:「無毀壞的……(中略)轉起定的。」喬塔!我了知具備這四法的個人為入流者、不墮惡趣法者、決定者、以正覺為彼岸者。』

  大德!在這麼說時,釋迦族人喬塔對我說這個:『請你等一下,摩訶男!請等一下,摩訶男!僅世尊能知道具備或不具備這些法。』大德!這裡,如果就某種法的爭論生起:如果世尊是一邊,\twnr{比丘}{31.0}僧團是一邊,我就是往世尊那一邊者,大德!請世尊記得這麼\twnr{淨信的}{340.0}我。大德!這裡,如果就某種法的爭論生起:如果世尊是一邊,比丘僧團與比丘尼僧團是一邊,我就是往世尊那一邊者,大德!請世尊記得這麼淨信的我。大德!這裡,如果就某種法的爭論生起:如果世尊是一邊,比丘僧團、比丘尼僧團與優婆塞是一邊,我就是往世尊那一邊者,大德!請世尊記得這麼淨信的我。大德!這裡,如果就某種法的爭論生起:如果世尊是一邊,比丘僧團、比丘尼僧團、優婆塞與優婆夷是一邊,我就是往世尊那一邊者,大德!請世尊記得這麼淨信的我。大德!這裡,如果就某種法的爭論生起:如果世尊是一邊,比丘僧團、比丘尼僧團、優婆塞、優婆夷與包括天、魔、梵的世間;包括沙門婆羅門,包括天-人的\twnr{世代}{38.0}是一邊,我就是往世尊那一邊者,大德!請世尊記得這麼淨信的我。」

  「喬塔!對這麼說的釋迦族人摩訶男,你怎麼說?」

  「大德!對這麼說的釋迦族人摩訶男,除了好,除了善巧外,我不說什麼。」



\sutta{24}{24}{釋迦族人色勒那尼經第一}{https://agama.buddhason.org/SN/sn.php?keyword=55.24}
  起源於迦毘羅衛城。

  當時,釋迦族人色勒那尼死了,他被\twnr{世尊}{12.0}\twnr{記說}{179.0}為\twnr{入流者}{165.0}、不墮\twnr{惡趣}{110.0}法者、\twnr{決定者}{159.0}、\twnr{正覺為彼岸者}{160.0}。

  在那裡,眾多釋迦族人集合、會合後,譏嫌、失望、誹謗:

  「實在不可思議啊,\twnr{先生}{202.0}!實在\twnr{未曾有}{206.0}啊,先生!在這種情況下(在這裡),現在誰將不是入流者!確實是因為釋迦族人色勒那尼死了,他被世尊記說為入流者、不墮惡趣法者、決定者、正覺為彼岸者,釋迦族人色勒那尼來到學的薄弱:他喝酒。」

  那時,釋迦族人摩訶男去見世尊。抵達後,向世尊\twnr{問訊}{46.0}後,在一旁坐下。在一旁坐下的釋迦族人摩訶男對世尊說這個:

  「\twnr{大德}{45.0}!這裡,釋迦族人色勒那尼死了,他被世尊記說為入流者、不墮惡趣法者、決定者、正覺為彼岸者。大德!在那裡,眾多釋迦族人集合、會合後,譏嫌、失望、誹謗:『實在不可思議啊,先生!實在未曾有啊,先生!在這種情況下,現在誰將不是入流者!確實是因為釋迦族人色勒那尼死了,他被世尊記說為入流者、不墮惡趣法者、決定者、正覺為彼岸者,釋迦族人色勒那尼來到學的薄弱:他喝酒。』」

  「摩訶男!凡長久\twnr{歸依}{284.0}佛、歸依法、\twnr{歸依僧團}{65.0}的那位\twnr{優婆塞}{98.0},他如何會走入\twnr{下界}{111.0}呢?摩訶男!凡當正確說它時,應該說『長久歸依佛、歸依法、歸依\twnr{僧團}{375.0}的優婆塞』,是釋迦族人色勒那尼,當正確說時,應該說。摩訶男!釋迦族人色勒那尼是長久歸依佛、歸依法、歸依僧團的優婆塞,他如何會走入下界呢?

  摩訶男!這裡,某一類人在佛上具備\twnr{不壞淨}{233.0}:『像這樣,那位世尊……(中略)\twnr{天-人們的大師}{11.0}、\twnr{佛陀}{3.0}、世尊。』在法上……(中略)在僧團上……(中略)是捷慧者,是速慧者,且是解脫的具備者,他以諸漏的滅盡,以證智自作證後,在當生中\twnr{進入後住於}{66.0}無漏\twnr{心解脫}{16.0}、\twnr{慧解脫}{539.0},摩訶男!這位個人從地獄被解脫;從畜生界被解脫;從餓鬼界被解脫;從\twnr{苦界}{109.0}、惡趣、下界被解脫。

  摩訶男!又,這裡,某一類人在佛上具備不壞淨:『像這樣,那位世尊……(中略)天-人們的大師、佛陀、世尊。』在法上……(中略)在僧團上……(中略)是捷慧者,是速慧者,但不是解脫的具備者,他以\twnr{五下分結}{134.0}的滅盡,成為\twnr{化生}{346.0}者、在那裡般涅槃者、不從那個世間返還者,摩訶男!這位個人也從地獄被解脫;從畜生界被解脫;從餓鬼界被解脫;從苦界、惡趣、下界被解脫。

  摩訶男!又,這裡,某一類人在佛上具備不壞淨:『像這樣,那位世尊……(中略)天-人們的大師、佛陀、世尊。』在法上……(中略)在僧團上……(中略)不是捷慧者,不是速慧者,也不是解脫的具備者,他以三結的遍盡,以貪、瞋、癡薄的狀態,為\twnr{一來}{208.0}者,只回來這個世間一次後,作苦的終結,摩訶男!這位個人也從地獄被解脫;從畜生界被解脫;從餓鬼界被解脫;從苦界、惡趣、下界被解脫。

  摩訶男!又,這裡,某一類人在佛上具備不壞淨:『像這樣,那位世尊……(中略)天-人們的大師、佛陀、世尊。』在法上……(中略)在僧團上……(中略)不是捷慧者,不是速慧者,也不是解脫的具備者,他以三結的遍盡,為入流者、不墮惡趣法者、決定者、正覺為彼岸者,摩訶男!這位個人也從地獄被解脫;從畜生界被解脫;從餓鬼界被解脫;從苦界、惡趣、下界被解脫。

  摩訶男!又,這裡,某一類人就不在佛上具備不壞淨;不在法上……(中略)不在僧團上……(中略)不是捷慧者,不是速慧者,也不是解脫的具備者,但他有這些法(他的這些法存在):信根、\twnr{活力根}{291.0}、念根、定根、慧根,且\twnr{那個如來宣說的諸法以慧足夠沉思地接受}{x651},摩訶男!這位個人也是不到地獄者;不到畜生界者;不到餓鬼界者;不到苦界、惡趣、下界者。

  摩訶男!又,這裡,某一類人不在佛上具備不壞淨;不在法上……(中略)不在僧團上……(中略)不是捷慧者,不是速慧者,也不是解脫的具備者,但他有這些法:信根……(中略)慧根,且他在如來上有足夠信者、\twnr{足夠情愛者}{977.0},摩訶男!這位個人也是不到地獄者;不到畜生界者;不到餓鬼界者;不到苦界、惡趣、下界者。

  摩訶男!如果這些大娑羅[樹]也了知善說、惡說,我也記說這些大娑羅為入流者、不墮惡趣法者、決定者、正覺為彼岸者,又更何況是釋迦族人色勒那尼!摩訶男!釋迦族人色勒那尼在死時,\twnr{受持}{57.0}了學。」



\sutta{25}{25}{釋迦族人色勒那尼經第二}{https://agama.buddhason.org/SN/sn.php?keyword=55.25}
  起源於迦毘羅衛城。

  當時,釋迦族人色勒那尼死了,他被\twnr{世尊}{12.0}\twnr{記說}{179.0}為\twnr{入流者}{165.0}、不墮\twnr{惡趣}{110.0}法者、\twnr{決定者}{159.0}、\twnr{以正覺為彼岸}{160.0}。

  在那裡,眾多釋迦族人集合、會合後,譏嫌、失望、誹謗:

  「實在不可思議啊,\twnr{先生}{202.0}!實在\twnr{未曾有}{206.0}啊,先生!在這種情況下(在這裡),現在誰將不是入流者!確實是因為釋迦族人色勒那尼死了,他被世尊記說為入流者、不墮惡趣法者、決定者、正覺為彼岸者,釋迦族人色勒那尼在學上是非完全的實行者。」

  那時,釋迦族人摩訶男去見世尊。抵達後,向世尊\twnr{問訊}{46.0}後,在一旁坐下。在一旁坐下的釋迦族人摩訶男對世尊說這個:

  「\twnr{大德}{45.0}!這裡,釋迦族人色勒那尼死了,他被世尊記說為入流者、不墮惡趣法者、決定者、正覺為彼岸者。大德!在那裡,眾多釋迦族人集合、會合後,譏嫌、失望、誹謗:『實在不可思議啊,先生!實在未曾有啊,先生!在這種情況下,現在誰將不是入流者!確實是因為釋迦族人色勒那尼死了,他被世尊記說為入流者、不墮惡趣法者、決定者、正覺為彼岸者,釋迦族人色勒那尼在學上是非完全的實行者。』」

  「摩訶男!凡長久\twnr{歸依}{284.0}佛、歸依法、\twnr{歸依僧團}{65.0}的\twnr{優婆塞}{98.0},他如何會走入\twnr{下界}{111.0}呢?摩訶男!凡當正確說它時,應該說『長久歸依佛、歸依法、歸依\twnr{僧團}{375.0}的優婆塞』,是釋迦族人色勒那尼,當正確說時,應該說。摩訶男!釋迦族人色勒那尼是長久歸依佛、歸依法、歸依僧團的優婆塞,他如何會走入下界呢?

  摩訶男!這裡,某一類人是在佛上到達\twnr{一向}{168.0}\twnr{極淨信者}{340.1}:『像這樣,那位世尊……(中略)\twnr{天-人們的大師}{11.0}、\twnr{佛陀}{3.0}、世尊。』在法上……(中略)在僧團上……(中略)是捷慧者,是速慧者,且是解脫的具備者,他以諸漏的滅盡,以證智自作證後,在當生中\twnr{進入後住於}{66.0}無漏\twnr{心解脫}{16.0}、\twnr{慧解脫}{539.0},摩訶男!這位個人從地獄被解脫;從畜生界被解脫;從餓鬼界被解脫;從\twnr{苦界}{109.0}、惡趣、下界被解脫。

  摩訶男!又,這裡,某一類人是在佛上到達一向極淨信者:『像這樣,那位世尊……(中略)天-人們的大師、佛陀、世尊。』在法上……(中略)在僧團上……(中略)是捷慧者,是速慧者,但不是解脫的具備者,他以\twnr{五下分結}{134.0}的滅盡,成為\twnr{中般涅槃}{297.0}者、\twnr{生般涅槃}{298.0}者、\twnr{無行般涅槃}{299.0}者、\twnr{有行般涅槃}{300.0}者、\twnr{上流到阿迦膩吒}{301.0}者,摩訶男!這位個人也從地獄被解脫;從畜生界被解脫;從餓鬼界被解脫;從苦界、惡趣、下界被解脫。

  摩訶男!又,這裡,某一類人是在佛上到達一向極淨信者:『像這樣,那位世尊……(中略)天-人們的大師、佛陀、世尊。』在法上……(中略)在僧團上……(中略)不是捷慧者,不是速慧者,也不是解脫的具備者,他以三結的遍盡,以貪、瞋、癡薄的狀態,為\twnr{一來}{208.0}者,只回來這個世間一次後,作苦的終結,摩訶男!這位個人也從地獄被解脫;從畜生界被解脫;從餓鬼界被解脫;從苦界、惡趣、下界被解脫。

  摩訶男!又,這裡,某一類人是在佛上到達一向極淨信者:『像這樣,那位世尊……(中略)天-人們的大師、佛陀、世尊。』在法上……(中略)在僧團上……(中略)不是捷慧者,不是速慧者,也不是解脫的具備者,他以三結的遍盡,為入流者、不墮惡趣法者、決定者、正覺為彼岸者,摩訶男!這位個人也從地獄被解脫;從畜生界被解脫;從餓鬼界被解脫;從苦界、惡趣、下界被解脫。

  摩訶男!又,這裡,某一類人不是在佛上到達一向極淨信者;不在法上……(中略)不在僧團上……(中略)不是捷慧者,不是速慧者,也不是解脫的具備者,但他有這些法(他的這些法存在):信根……(中略)慧根,且那個如來宣說的諸法以慧足夠沉思地接受[\suttaref{SN.55.24}],摩訶男!這位個人也是不到地獄者;不到畜生界者;不到餓鬼界者;不到苦界、惡趣、下界者。

  摩訶男!又,這裡,某一類人不是在佛上到達一向極淨信者;不在法上……(中略)不在僧團上……(中略)不是捷慧者,不是速慧者,也不是解脫的具備者,但他有這些法:信根……(中略)慧根,且他在如來上有足夠信者、\twnr{足夠情愛者}{977.0},摩訶男!這位個人也是不到地獄者;不到畜生界者;不到餓鬼界者;不到苦界、惡趣、下界者。

  摩訶男!猶如有殘株未除去的惡田、惡地,且是破碎的、腐爛的、被風吹日曬破壞的、非新成熟的、非安全播下的種子,天又不正確地隨給與水流,是否那些種子來到成長、增長、成滿呢?」

  「大德!這確實不是。」 

  「同樣的,摩訶男!這裡,被惡說的、被惡宣說的、不\twnr{出離的}{294.0}、導向不寂靜的、非遍正覺者宣說的法,我說這是關於惡田。而弟子住於在該法上為\twnr{法隨法行者}{58.0}、\twnr{方正行者}{764.0}、\twnr{隨法行者}{765.0},我說這是關於惡種子。

  摩訶男!猶如有殘株已善除的善田、善地,且無毀壞的、無腐爛的、無風吹日曬破壞的、新成熟的、安全播下的種子,天又隨給與正確的水流,那些種子是否來到成長、增長、成滿呢?」

  「是的,大德!」 

  「同樣的,摩訶男!這裡,被善說的、被善宣說的、出離的、導向寂靜的、遍正覺者宣說的法,我說這是關於善田。而弟子住於在該法上為法隨法行者、方正行者、隨法行者,我說這是關於善種子,又更何況是釋迦族人色勒那尼!摩訶男!釋迦族人色勒那尼死時在學上是完全的實行者。」 



\sutta{26}{26}{給孤獨經第一}{https://agama.buddhason.org/SN/sn.php?keyword=55.26}
  起緣於舍衛城。

  當時,\twnr{屋主}{103.0}給孤獨是生病者、受苦者、重病者。

  那時,屋主給孤獨召喚某位男子:

  「喂!男子!來!請你去見\twnr{尊者}{200.0}舍利弗。抵達後,請你以我的名義以頭禮拜尊者舍利弗的足:『\twnr{大德}{45.0}!屋主給孤獨是生病者、受苦者、重病者,他以頭禮拜尊者舍利弗的足。』以及請你這麼說:『大德!請尊者舍利弗\twnr{出自憐愍}{121.0},去屋主給孤獨的住處,\twnr{那就好了}{44.0}!』」

  「是的。」那位男子回答屋主給孤獨後,去見尊者舍利弗。抵達後,向尊者舍利弗\twnr{問訊}{46.0}後,在一旁坐下。在一旁坐下的那位男子對尊者舍利弗說這個:「大德!屋主給孤獨是生病者、受苦者、重病者,他以頭禮拜尊者舍利弗的足,以及這麼說:『大德!請尊者舍利弗出自憐愍,去屋主給孤獨的住處,那就好了!』」

  尊者舍利弗以沈默狀態同意。

  那時,尊者舍利弗午前時穿衣、拿起衣鉢後,以尊者阿難為隨從\twnr{沙門}{29.0},去屋主給孤獨的住處。抵達後,在設置的座位坐下。坐下後,尊者舍利弗對屋主給孤獨說這個:

  「屋主!是否能被你忍受?\twnr{是否能被[你]維持生活}{137.0}?是否苦的感受減退、不增進,減退的結局被知道,非增進?」

  「大德!不能被我忍受,不能被[我]維持,我強烈苦的感受增進、不減退,增進的結局被知道,非減退。」

  「屋主!在佛上具備像這樣無\twnr{淨信}{340.0}的\twnr{未聽聞的一般人}{74.0}以身體的崩解,死後往生\twnr{苦界}{109.0}、惡趣、\twnr{下界}{111.0}、地獄,你沒有在佛上像那樣的無淨信。屋主!而你有在佛上\twnr{不壞淨}{233.0}:『像這樣,那位世尊……(中略)\twnr{天-人們的大師}{11.0}、\twnr{佛陀}{3.0}、\twnr{世尊}{12.0}。』而且,當看見在自己之中那個你的在佛上不壞淨時,痛苦(受)會立即止息。

  屋主!在法上具備像這樣無淨信的未聽聞的一般人以身體的崩解,死後往生苦界、惡趣、下界、地獄,你沒有在法上像那樣的無淨信。屋主!而你有在法上不壞淨:『被世尊善說的法是……(中略)應該被智者各自經驗的。』而且,當看見在自己之中那個你的在法上不壞淨時,痛苦會立即止息。

  屋主!在\twnr{僧團}{375.0}上具備像這樣無淨信的未聽聞的一般人以身體的崩解,死後往生苦界、惡趣、下界、地獄,你沒有在僧團上像那樣的無淨信。屋主!而你有在僧團上不壞淨:『世尊的弟子僧團是\twnr{善行者}{518.0}……(中略)為世間的無上\twnr{福田}{101.0}。』而且,當看見在自己之中那個你的在僧團上不壞淨時,痛苦會立即止息。

  屋主!具備像這樣\twnr{破戒}{708.0}的未聽聞的一般人以身體的崩解,死後往生苦界、惡趣、下界、地獄,你沒有像那樣的破戒。屋主!而你有像這樣聖者喜愛的諸戒:『無毀壞的……(中略)轉起定的。而且,當看見在自己之中那個你的聖者喜愛的諸戒時,痛苦會立即止息。

  屋主!具備像這樣邪見的未聽聞的一般人以身體的崩解,死後往生苦界、惡趣、下界、地獄,你沒有像那樣的邪見。屋主!而你有正見。而且,當看見在自己之中那個你的正見時,痛苦會立即止息。

  屋主!具備像這樣邪志的未聽聞的一般人以身體的崩解,死後往生苦界、惡趣、下界、地獄,你沒有像那樣的邪志。屋主!而你有正志。而且,當看見在自己之中那個你的正志時,痛苦會立即止息。

  屋主!具備像這樣邪語的未聽聞的一般人以身體的崩解,死後往生苦界、惡趣、下界、地獄,你沒有像那樣的邪語。屋主!而你有正語。而且,當看見在自己之中那個你的正語時,痛苦會立即止息。

  屋主!具備像這樣邪業的未聽聞的一般人以身體的崩解,死後往生苦界、惡趣、下界、地獄,你沒有像那樣的邪業。屋主!而你有正業。而且,當看見在自己之中那個你的正業時,痛苦會立即止息。

  屋主!具備像這樣邪命的未聽聞的一般人以身體的崩解,死後往生苦界、惡趣、下界、地獄,你沒有像那樣的邪命。屋主!而你有正命。而且,當看見在自己之中那個你的正命時,痛苦會立即止息。

  屋主!具備像這樣邪精進的未聽聞的一般人以身體的崩解,死後往生苦界、惡趣、下界、地獄,你沒有像那樣的邪精進。屋主!而你有正精進。而且,當看見在自己之中那個你的正精進時,痛苦會立即止息。

  屋主!具備像這樣邪念的未聽聞的一般人以身體的崩解,死後往生苦界、惡趣、下界、地獄,你沒有像那樣的邪念。屋主!而你有正念。而且,當看見在自己之中那個你的正念時,痛苦會立即止息。

  屋主!具備像這樣邪定的未聽聞的一般人以身體的崩解,死後往生苦界、惡趣、下界、地獄,你沒有像那樣的邪定。屋主!而你有正定。而且,當看見在自己之中那個你的正定時,痛苦會立即止息。

  屋主!具備像這樣邪智的未聽聞的一般人以身體的崩解,死後往生苦界、惡趣、下界、地獄,你沒有像那樣的邪智。屋主!而你有\twnr{正智}{976.0}。而且,當看見在自己之中那個你的正智時,痛苦會立即止息。

  屋主!具備像這樣邪解脫的未聽聞的一般人以身體的崩解,死後往生苦界、惡趣、下界、地獄,你沒有像那樣的邪解脫。屋主!而你有正解脫。而且,當看見在自己之中那個你的正解脫時,痛苦會立即止息。」

  那時,屋主給孤獨的痛苦立即止息。那時,屋主給孤獨就以自己盤子的食物招待尊者舍利弗與尊者阿難。那時,對已食、手離鉢的尊者舍利弗,屋主給孤獨取某個低的坐具後,在一旁坐下。尊者舍利弗以這些偈頌感謝在一旁坐下的屋主給孤獨: 

  「凡在如來上有信者:不動的、已善住立的,

   以及凡有善戒者:聖者喜愛的、所稱讚的。

   凡在\twnr{僧團}{375.0}上有\twnr{淨信}{507.0}者,且見已成為正直的,

   他們說他是『不貧窮的』,他的生命是不空虛的。

   因此對信與戒,對淨信、法的看見,

   有智慧者應該實踐:憶念諸佛的教說者。」

  那時,尊者舍利弗以這些偈頌感謝屋主給孤獨後,從座位起來後離開。 

  那時,尊者阿難去見世尊。抵達後,向世尊問訊後,在一旁坐下。世尊對在一旁坐下的尊者阿難說這個: 

  「阿難!那麼,你中午從哪裡來呢?」

  「大德!屋主給孤獨被尊者舍利弗以這個與這個教誡教誡。」

  「阿難!尊者舍利弗是賢智者;阿難!尊者舍利弗是大慧者,確實是因為對四\twnr{入流支}{370.0}他將以十種方式(行相)解析。」



\sutta{27}{27}{給孤獨經第二}{https://agama.buddhason.org/SN/sn.php?keyword=55.27}
  起緣於舍衛城。

  當時,\twnr{屋主}{103.0}給孤獨是生病者、受苦者、重病者。

  那時,屋主給孤獨召喚某位男子:

  「喂!男子!來!請你去見\twnr{尊者}{200.0}阿難。抵達後,請你以我的名義以頭禮拜尊者阿難的足:『\twnr{大德}{45.0}!屋主給孤獨是生病者、受苦者、重病者,他以頭禮拜尊者阿難的足。』且請你這麼說:『大德!請尊者阿難\twnr{出自憐愍}{121.0},去屋主給孤獨的住處,\twnr{那就好了}{44.0}!』」

  「是的。」那位男子回答屋主給孤獨後,去見尊者阿難。抵達後,向尊者阿難\twnr{問訊}{46.0}後,在一旁坐下。在一旁坐下的那位男子對尊者阿難說這個:「大德!屋主給孤獨是生病者、受苦者、重病者,他以頭禮拜尊者阿難的足,且這麼說:『大德!請尊者阿難出自憐愍,去屋主給孤獨的住處,那就好了!』」

  尊者阿難以沈默狀態同意。

  那時,尊者阿難午前時穿衣、拿起衣鉢後,去屋主給孤獨的住處。抵達後,在設置的座位坐下。坐下後,尊者阿難對屋主給孤獨說這個:

  「屋主!是否能被你忍受?\twnr{是否能被[你]維持生活}{137.0}?是否苦的感受減退、不增進,減退的結局被知道,非增進?」

  「大德!不能被我忍受,不能被[我]維持,我強烈苦的感受增進、不減退,增進的結局被知道,非減退。」

  「屋主!具備四法之\twnr{未聽聞的一般人}{74.0}有恐懼、有僵硬狀態(恐怖)、\twnr{有來生死亡的害怕}{x652},哪四個?

  屋主!這裡,未聽聞的一般人在佛上具備無淨信,而且,當看見在自己之中那個他的在佛上無淨信時,有恐懼、有僵硬狀態、有來生死亡的害怕。

  再者,屋主!未聽聞的一般人在法上具備無淨信,而且,當看見在自己之中那個他的在法上無淨信時,有恐懼、有僵硬狀態、有來生死亡的害怕。

  再者,屋主!未聽聞的一般人在\twnr{僧團}{375.0}上具備無淨信,而且,當看見在自己之中那個他的在僧團上無淨信時,有恐懼、有僵硬狀態、有來生死亡的害怕。

  再者,屋主!未聽聞的一般人具備破戒,而且,當看見在自己之中那個他的破戒時,有恐懼、有僵硬狀態、有來生死亡的害怕。

  屋主!具備這四法之未聽聞的一般人有恐懼、有僵硬狀態、有來生死亡的害怕。

  屋主!具備四法之\twnr{有聽聞的聖弟子}{24.0}沒有恐懼、沒有僵硬狀態、沒有來生死亡的害怕,哪四個?

  屋主!這裡,有聽聞的聖弟子在佛上具備\twnr{不壞淨}{233.0}:『像這樣,那位\twnr{世尊}{12.0}是……(中略)\twnr{天-人們的大師}{11.0}、\twnr{佛陀}{3.0}、世尊。』而且,當看見在自己之中那個他的在佛上不壞淨時,沒有恐懼、沒有僵硬狀態、沒有來生死亡的害怕。

  再者,屋主!有聽聞的聖弟子在法上具備不壞淨:『被世尊善說的法是……(中略)應該被智者各自經驗的。』而且,當看見在自己之中那個他的在法上不壞淨時,沒有恐懼、沒有僵硬狀態、沒有來生死亡的害怕。

  再者,屋主!有聽聞的聖弟子在僧團上具備不壞淨:『世尊的弟子僧團是\twnr{善行者}{518.0}……(中略)為世間的無上\twnr{福田}{101.0}。』而且,當看見在自己之中那個他的在僧團上不壞淨時,沒有恐懼、沒有僵硬狀態、沒有來生死亡的害怕。

  再者,屋主!有聽聞的聖弟子具備聖者喜愛的諸戒:無毀壞的……(中略)轉起定的。而且,當看見在自己之中那個他的聖者喜愛的諸戒時,沒有恐懼、沒有僵硬狀態、沒有來生死亡的害怕。

  屋主!具備這四法之有聽聞的聖弟子沒有恐懼、沒有僵硬狀態、沒有來生死亡的害怕。」

  「阿難大德!我不害怕,為何我要(將)害怕?大德!因為我在佛上具備不壞淨:『像這樣,那位世尊是……(中略)天-人們的大師、佛陀、世尊。』在法上……(中略)我在僧團上具備不壞淨:『世尊的弟子僧團是善行者……(中略)為世間的無上福田。』大德!凡這些被世尊教導的在家方正\twnr{學處}{392.0},我沒看見在自己之中那些的任何毀壞。」

  「屋主!是你的利得,屋主!\twnr{是你的善得的}{350.0}:屋主!\twnr{入流果}{165.1}被你\twnr{記說}{179.0}。」



\sutta{28}{28}{恐怖怨恨被平息經第一}{https://agama.buddhason.org/SN/sn.php?keyword=55.28}
  起源於舍衛城。

  \twnr{世尊}{12.0}對在一旁坐下的\twnr{屋主}{103.0}給孤獨說這個:

  「屋主!當\twnr{聖弟子}{24.0}的五恐怖、怨恨被平息,與具備四\twnr{入流支}{370.0},以及他的聖方法(理趣)被慧善見、善洞察,當他希望時,就能以自己\twnr{記說}{179.0}自己:『我是地獄已盡者,畜生界已盡者,\twnr{餓鬼界}{362.0}已盡者,\twnr{苦界}{109.0}、\twnr{惡趣}{110.0}、\twnr{下界}{111.0}已盡者,我是\twnr{入流者}{165.0}、不墮惡趣法者、\twnr{決定者}{159.0}、\twnr{正覺為彼岸者}{160.0}。』

  哪五個恐怖、怨恨被平息?屋主!凡殺生者,以殺生\twnr{為緣}{180.0}產生當生的恐怖、怨恨,也產生來生的恐怖、怨恨,也感受心的憂苦,這樣,離殺生者的那個恐怖、怨恨被平息。屋主!凡未給予而取者……(中略)凡邪淫者……(中略)凡妄語者……(中略)凡榖酒、果酒、酒放逸處者,以榖酒、果酒、\twnr{酒放逸處}{107.0}為緣產生當生的恐怖、怨恨,也產生來生的恐怖、怨恨,也感受心的憂苦,這樣,離榖酒、果酒、酒放逸處者的那個恐怖、怨恨被平息。這是五個恐怖、怨恨被平息。

  具備哪四入流支?屋主!這裡,聖弟子在佛上具備\twnr{不壞淨}{233.0}:『像這樣,那位世尊是……(中略)\twnr{天-人們的大師}{11.0}、\twnr{佛陀}{3.0}、世尊。』在法上……(中略)在\twnr{僧團}{375.0}上……(中略)具備聖者喜愛的諸戒:無毀壞的、無瑕疵的、無污點的、無雜色的、自由的、智者稱讚的、不取著的、轉起定的。具備這四入流支。

  而什麼是以及他的聖方法被慧善見、善洞察?屋主!這裡,聖弟子就徹底地\twnr{如理作意}{114.0}\twnr{緣起}{225.0}:『像這樣,在這個存在時那個存在,以這個的生起那個生起。像這樣,在這個不存在時那個不存在,以這個的滅\twnr{那個被滅}{394.0},即:以\twnr{無明}{207.0}\twnr{為緣}{180.0}有諸行(而諸行存在);以行為緣有識……(中略)這樣是這整個\twnr{苦蘊}{83.0}的\twnr{集}{67.0}。但以無明的\twnr{無餘褪去與滅}{491.0}有行滅(而行滅存在)……(中略)以觸滅有受滅;以受滅有渴愛滅……(中略)這樣是這整個苦蘊的滅。』這是他的聖方法(理趣)被慧善見、善洞察。

  屋主!當聖弟子的這些五恐怖、怨恨被平息,與具備這些四入流支,以及他的聖方法(理趣)被慧善見、善洞察,當他希望時,就能以自己記說自己:『我是地獄已盡者,畜生界已盡者,餓鬼界已盡者,苦界、惡趣、下界已盡者,我是入流者、不墮惡趣法者、決定者、以正覺為彼岸。』」[\suttaref{SN.12.41}]



\sutta{29}{29}{恐怖怨恨被平息經第二}{https://agama.buddhason.org/SN/sn.php?keyword=55.29}
  起源於舍衛城。……(中略)

  「\twnr{比丘}{31.0}們!當\twnr{聖弟子}{24.0}的五恐怖、怨恨被平息,與具備四\twnr{入流支}{370.0},以及他的聖方法(理趣)被慧善見、善洞察,當他希望時,就能以自己\twnr{記說}{179.0}自己:『我是地獄已盡者,畜生界已盡者,\twnr{餓鬼界}{362.0}已盡者,\twnr{苦界}{109.0}、\twnr{惡趣}{110.0}、\twnr{下界}{111.0}已盡者,我是\twnr{入流者}{165.0}、不墮惡趣法者、\twnr{決定者}{159.0}、\twnr{正覺為彼岸者}{160.0}。』[\suttaref{SN.12.42}]





\sutta{30}{30}{離車人難達葛經}{https://agama.buddhason.org/SN/sn.php?keyword=55.30}
  \twnr{有一次}{2.0},\twnr{世尊}{12.0}住在毘舍離大林重閣講堂。

  那時,離車人大臣難達葛去見世尊。抵達後,向世尊\twnr{問訊}{46.0}後,在一旁坐下。世尊對在一旁坐下的離車人大臣難達葛說這個:

  「難達葛!具備四法的\twnr{聖弟子}{24.0}是\twnr{入流者}{165.0}、不墮\twnr{惡趣}{110.0}法者、\twnr{決定者}{159.0}、\twnr{正覺為彼岸者}{160.0},哪四個?難達葛!這裡,聖弟子在佛上具備\twnr{不壞淨}{233.0}:『像這樣,那位世尊是……(中略)\twnr{天-人們的大師}{11.0}、\twnr{佛陀}{3.0}、世尊。』在法上……(中略)在\twnr{僧團}{375.0}上……(中略)具備聖者喜愛的諸戒:無毀壞的……(中略)轉起定的。

  難達葛!具備這四法的聖弟子是入流者、不墮惡趣法者、決定者、正覺為彼岸者。

  難達葛!還有,具備這四法的聖弟子,不論天或人,是長壽的(加入壽命的);不論天或人,是美貌的;不論天或人,是快樂的;不論天或人,是有名聲的;不論天或人,是\twnr{有統治權}{x653}的。難達葛!又,我非聽聞其他\twnr{沙門}{29.0}或\twnr{婆羅門}{17.0}後說它,而是,凡正以我自己知道的、自己看見的、自己發現的,我就說它。」

  在這麼說時,某位男子對離車人大臣難達葛說這個:

  「\twnr{大德}{45.0}!是沐浴的時候。」

  「我說,現在夠了,以這個外沐浴。這個內沐浴將是足夠的,即:在世尊上有\twnr{淨信}{507.0}。」

  色勒那尼品第三,其\twnr{攝頌}{35.0}:

  「以摩訶男二說,喬塔與色勒那二則,

   二則給孤獨,與二則恐怖怨恨,

   離車人為第十說,以那個被稱為品。」





\pin{福德的流出品}{31}{40}
\sutta{31}{31}{福德的流出經第一}{https://agama.buddhason.org/SN/sn.php?keyword=55.31}
  起源於舍衛城。

  「\twnr{比丘}{31.0}們!有這四個\twnr{福德的流出}{705.0}、善的流出、安樂的食物,哪四個?

  比丘們!這裡,\twnr{聖弟子}{24.0}在佛上具備\twnr{不壞淨}{233.0}:『像這樣,那位\twnr{世尊}{12.0}……(中略)\twnr{天-人們的大師}{11.0}、\twnr{佛陀}{3.0}、世尊。』這是第一個福德的流出、善的流出、安樂的食物。

  再者,比丘們!聖弟子在法上具備不壞淨:『被世尊善說的法是……(中略)應該被智者各自經驗的。』這是第二個福德的流出、善的流出、安樂的食物。

  再者,比丘們!聖弟子在\twnr{僧團}{375.0}上具備不壞淨:『世尊的弟子僧團是\twnr{善行者}{518.0}……(中略)為世間的無上\twnr{福田}{101.0}。』這是第三個福德的流出、善的流出、安樂的食物。

  再者,比丘們!聖弟子具備聖者喜愛的諸戒:無毀壞的……(中略)轉起定的。這是第四個福德的流出、善的流出、安樂的食物。

  比丘們!這是四個福德的流出、善的流出、安樂的食物。」



\sutta{32}{32}{福德的流出經第二}{https://agama.buddhason.org/SN/sn.php?keyword=55.32}
  「\twnr{比丘}{31.0}們!有這四個\twnr{福德的流出}{705.0}、善的流出、安樂的食物,哪四個?

  比丘們!這裡,\twnr{聖弟子}{24.0}在佛上具備\twnr{不壞淨}{233.0}:『像這樣,那位\twnr{世尊}{12.0}……(中略)\twnr{天-人們的大師}{11.0}、\twnr{佛陀}{3.0}、世尊。』這是第一個福德的流出、善的流出、安樂的食物。

  再者,比丘們!聖弟子具備對法……(中略)對\twnr{僧團}{375.0}……(中略)。

  再者,比丘們!聖弟子以離慳垢之心住於在家,是\twnr{自由施捨者}{348.0}、親手施與者、樂於棄捨者、回應乞求者、\twnr{樂於布施物均分者}{349.0}。這是第四個福德的流出、善的流出、安樂的食物。

  比丘們!這是四個福德的流出、善的流出、安樂的食物。」



\sutta{33}{33}{福德的流出經第三}{https://agama.buddhason.org/SN/sn.php?keyword=55.33}
  「\twnr{比丘}{31.0}們!有這四個\twnr{福德的流出}{705.0}、善的流出、安樂的食物,哪四個?

  比丘們!這裡,\twnr{聖弟子}{24.0}在佛上具備\twnr{不壞淨}{233.0}:『像這樣,那位\twnr{世尊}{12.0}……(中略)\twnr{天-人們的大師}{11.0}、\twnr{佛陀}{3.0}、世尊。』這是第一個福德的流出、善的流出、安樂的食物。

  再者,比丘們!聖弟子在法上……(中略)在\twnr{僧團}{375.0}上……(中略)。

  再者,比丘們!聖弟子是有慧者,具備\twnr{導向生起與滅沒}{498.0}、聖、洞察、導向\twnr{苦的完全滅盡}{181.0}之慧。這是第四個福德的流出、善的流出、安樂的食物。

  比丘們!這是四個福德的流出、善的流出、安樂的食物。」



\sutta{34}{34}{天路經第一}{https://agama.buddhason.org/SN/sn.php?keyword=55.34}
  起源於舍衛城。

  「\twnr{比丘}{31.0}們!為了未清淨眾生之清淨、為了未淨化眾生之淨化,有這四個諸天的天路,哪四個?

  比丘們!這裡,\twnr{聖弟子}{24.0}在佛上具備\twnr{不壞淨}{233.0}:『像這樣,那位\twnr{世尊}{12.0}是……(中略)\twnr{天-人們的大師}{11.0}、\twnr{佛陀}{3.0}、世尊。』為了未清淨眾生之清淨、為了未淨化眾生之淨化,這是第一個諸天的天路。

  再者,比丘們!聖弟子具備對法……(中略)對\twnr{僧團}{375.0}……(中略)。

  再者,比丘們!聖弟子具備聖者喜愛的諸戒:無毀壞的……(中略)轉起定的。為了未清淨眾生之清淨、為了未淨化眾生之淨化,這是第四個諸天的天路。

  比丘們!為了未清淨眾生之清淨、為了未淨化眾生之淨化,這是四個諸天的天路。」



\sutta{35}{35}{天路經第二}{https://agama.buddhason.org/SN/sn.php?keyword=55.35}
  「\twnr{比丘}{31.0}們!為了未清淨眾生之清淨、為了未淨化眾生之淨化,有這四個諸天的天路,哪四個?

  比丘們!這裡,\twnr{聖弟子}{24.0}在佛上具備\twnr{不壞淨}{233.0}:『像這樣,那位\twnr{世尊}{12.0}是……(中略)\twnr{天-人們的大師}{11.0}、\twnr{佛陀}{3.0}、世尊。』他像這樣深慮:『什麼是諸天的天路呢?』他這麼知道:『我聽聞現在無瞋害的諸天是最上的,而我不傷害任何懦弱者或堅強者,我確實住於具備諸天路之法。』為了未清淨眾生之清淨、為了未淨化眾生之淨化,這是第一個諸天的天路。

  再者,比丘們!聖弟子具備對法……(中略)對\twnr{僧團}{375.0}……(中略)。

  再者,比丘們!聖弟子具備聖者喜愛的諸戒:無毀壞的……(中略)轉起定的。他像這樣深慮:『什麼是諸天的天路呢?』他這麼知道:『我聽聞現在無瞋害的諸天是最上的,而我不傷害任何懦弱者或堅強者,我確實住於具備諸天路之法。』為了未清淨眾生之清淨、為了未淨化眾生之淨化,這是第四個諸天的天路。

  比丘們!為了未清淨眾生之清淨、為了未淨化眾生之淨化,這是四個諸天的天路。」



\sutta{36}{36}{天同類經}{https://agama.buddhason.org/SN/sn.php?keyword=55.36}
  「\twnr{比丘}{31.0}們!具備四法者,悅意的諸天說同類,哪四個?

  比丘們!這裡,\twnr{聖弟子}{24.0}在佛上具備\twnr{不壞淨}{233.0}:『像這樣,那位\twnr{世尊}{12.0}是……(中略)\twnr{天-人們的大師}{11.0}、\twnr{佛陀}{3.0}、世尊。』凡那些在佛上具備不壞淨、從這裡死沒在那裡往生的諸天,他們這樣想:『我們在佛上具備像這樣的不壞淨,為從那裡死沒在這裡往生者,聖弟子也在佛上具備像那樣的不壞淨,他將來到諸天的面前。』

  再者,比丘們!聖弟子具備在諸法上……(中略)在\twnr{僧團}{375.0}上……(中略)具備聖者喜愛的諸戒:無毀壞的……(中略)轉起定的。凡那些在佛上具備不壞淨、從這裡死沒在那裡往生的諸天,他們這樣想:『我們在佛上具備像這樣的不壞淨,為從那裡死沒在這裡往生者,聖弟子也在佛上具備像那樣的不壞淨,他將來到諸天的面前。』

  比丘們!具備這四法者,悅意的諸天說同類。」



\sutta{37}{37}{摩訶男經}{https://agama.buddhason.org/SN/sn.php?keyword=55.37}
  \twnr{有一次}{2.0},\twnr{世尊}{12.0}住在釋迦族人的迦毘羅衛城尼拘律園。

  那時,釋迦族人摩訶男去見世尊。抵達後,向世尊\twnr{問訊}{46.0}後,在一旁坐下。在一旁坐下的釋迦族人摩訶男對世尊說這個:

  「\twnr{大德}{45.0}!什麼情形是\twnr{優婆塞}{98.0}?」

  「摩訶男!當已歸依佛、已歸依法、已\twnr{歸依僧}{65.0}團,摩訶男!這個情形是優婆塞。」

  「大德!又,什麼情形是具足戒的優婆塞?」

  「摩訶男!當優婆塞是離殺生者,是離\twnr{未給予而取}{104.0}者,是離\twnr{邪淫}{105.0}者,是離\twnr{妄語}{106.0}者,是離榖酒、果酒、\twnr{酒放逸處}{107.0}者時,摩訶男!這個情形是具足戒的優婆塞。」

  「大德!又,什麼情形是具足信的優婆塞?」

  「摩訶男!這裡,優婆塞是有信者,相信如來的\twnr{覺}{185.0}:『像這樣,那位世尊……(中略)\twnr{天-人們的大師}{11.0}、\twnr{佛陀}{3.0}、世尊。』摩訶男!這個情形是具足信的優婆塞。」

  「大德!又,什麼情形是具足施捨的優婆塞?」

  「摩訶男!這裡,優婆塞以離慳垢之心住於在家,是\twnr{自由施捨者}{348.0}、親手施與者、樂於棄捨者、回應乞求者、\twnr{樂於布施物均分者}{349.0},摩訶男!這個情形,優婆塞具足施捨。」

  「大德!又,什麼情形是具足慧的優婆塞?」

  「摩訶男!這裡,優婆塞是有慧者,具備\twnr{導向生起與滅沒}{498.0}、聖、洞察、導向\twnr{苦的完全滅盡}{181.0}之慧,摩訶男!這個情形,優婆塞具足慧。」



\sutta{38}{38}{雨經}{https://agama.buddhason.org/SN/sn.php?keyword=55.38}
  「\twnr{比丘}{31.0}們!猶如在天下大雨時,在有大雨的山上,那個依向下流動的水使山洞、裂縫、支流充滿,填滿(充滿)的山洞、裂縫、支流使小水池充滿,填滿的小水池使大水池充滿,填滿的大水池使小河充滿,填滿的小河使大河充滿,填滿的大河使大海洋充滿。同樣的,比丘們!對\twnr{聖弟子}{24.0}來說,凡在佛上\twnr{不壞淨}{233.0}、凡在法上不壞淨、在\twnr{僧團}{375.0}上不壞淨、凡聖者喜愛的諸戒,當這些諸法流動,走到\twnr{彼岸}{226.0}後,轉起諸\twnr{漏}{188.0}的滅盡。」



\sutta{39}{39}{葛利鉤達經}{https://agama.buddhason.org/SN/sn.php?keyword=55.39}
  \twnr{有一次}{2.0},\twnr{世尊}{12.0}住在釋迦族人的迦毘羅衛城尼拘律園。

  那時,世尊午前時穿衣、拿起衣鉢後,\twnr{為了托鉢}{87.0}去釋迦族女葛利鉤達的住處。抵達後,在設置的座位坐下。

  那時,釋迦族女葛利鉤達去見世尊。抵達後,向世尊\twnr{問訊}{46.0}後,在一旁坐下。世尊對在一旁坐下的釋迦族女葛利鉤達說這個:

  「葛利鉤達!具備四法的\twnr{聖弟子}{24.0}是\twnr{入流者}{165.0}、不墮惡趣法者、\twnr{決定者}{159.0}、\twnr{正覺為彼岸者}{160.0},哪四個?葛利鉤達!這裡,聖弟子在佛上具備\twnr{不壞淨}{233.0}:『像這樣,那位世尊是……(中略)\twnr{天-人們的大師}{11.0}、\twnr{佛陀}{3.0}、\twnr{世尊}{12.0}。』在法上……(中略)在\twnr{僧團}{375.0}上……(中略)以離慳垢之心住於在家,是\twnr{自由施捨者}{348.0}、親手施與者、樂於棄捨者、回應乞求者、\twnr{樂於布施物均分者}{349.0},葛利鉤達!具備這四法的聖弟子是入流者、不墮惡趣法者、決定者、正覺為彼岸者。」

  「大德!凡被世尊教導的這些四\twnr{入流支}{370.0},那些法在我中被發現,且我在那些法中被發現。 

  大德!因為我在佛上具備不壞淨:『像這樣,那位世尊是……(中略)天-人們的大師、佛陀、世尊。』在法上……(中略)在僧團上……(中略)又,凡在家中任何能施之物,那一切都無差別地被[施與]持戒者、\twnr{善法者}{601.1}。」

  「葛利鉤達!是你的利得,葛利鉤達!\twnr{是你的善得的}{350.0},葛利鉤達!\twnr{入流果}{165.1}被你\twnr{記說}{179.0}。」



\sutta{40}{40}{釋迦族人難提經}{https://agama.buddhason.org/SN/sn.php?keyword=55.40}
  \twnr{有一次}{2.0},\twnr{世尊}{12.0}住在釋迦族人的迦毘羅衛城尼拘律園。

  那時,釋迦族人難提去見世尊。抵達後,向世尊\twnr{問訊}{46.0}後,在一旁坐下。在一旁坐下的釋迦族人難提對世尊說這個:

  「\twnr{大德}{45.0}!凡\twnr{聖弟子}{24.0}的四\twnr{入流支}{370.0}都全部完全地、每一方面完全地不存在者,大德!那位聖弟子是放逸住者嗎?」

  「難提!凡四入流支全部完全地、每一方面完全地缺乏者,我說他是『在外者;站在凡夫側者』。

  難提!此外,關於聖弟子是放逸住者與不放逸住者,你要聽!你要\twnr{好好作意}{43.1}!我將說。」

  「是的,大德!」釋迦族人難提回答世尊。

  「難提!而怎樣是放逸住的聖弟子呢?難提!這裡,聖弟子在佛上具備\twnr{不壞淨}{233.0}:『像這樣,那位世尊是……(中略)\twnr{天-人們的大師}{11.0}、\twnr{佛陀}{3.0}、世尊。』他被那個在佛上不壞淨滿足,不更努力在白天獨居、在夜間\twnr{獨坐}{92.0}。對那位這樣住於放逸者,欣悅不存在;在欣悅不存在時,喜不存在;在喜不存在時,\twnr{寧靜}{313.0}不存在;在寧靜不存在時,住於苦;對心苦者來說不入定;在心不得定時,諸法不變成明顯;以諸法的不明顯,就名為(就走到稱呼)『住放逸者』。

  再者,聖弟子具備在諸法上……(中略)在\twnr{僧團}{375.0}上……(中略)具備聖者喜愛的諸戒:無毀壞的……(中略)轉起定的。他被那些聖者喜愛的諸戒滿足,不更努力在白天獨居、在夜間獨坐。對那位這樣住於放逸者,欣悅不存在;在欣悅不存在時,喜不存在;在喜不存在時,寧靜不存在;在寧靜不存在時,住於苦;對心苦者來說不入定;在心不得定時,諸法不變成明顯;以諸法的不明顯,就名為『住放逸者』。難提!這樣是放逸住的聖弟子。

  難提!而怎樣是不放逸住的聖弟子呢?難提!這裡,聖弟子在佛上具備不壞淨:『像這樣,那位世尊是……(中略)天-人們的大師、佛陀、世尊。』他不被那個在佛上不壞淨滿足,更努力在白天獨居、在夜間獨坐。對那位這樣住於不放逸者,欣悅被生起;對喜悅者,喜被生起;對\twnr{意喜}{320.0}者,身變得寧靜;\twnr{身已寧靜}{318.0}者感受樂;對有樂者,心入定;在心得定時,諸法變成明顯[\suttaref{SN.35.97}];以諸法的明顯,就名為『住不放逸者』。

  再者,聖弟子在法上……(中略)在\twnr{僧團}{375.0}上……(中略)具備聖者喜愛的諸戒:無毀壞的……(中略)轉起定的。他不被那些聖者喜愛的諸戒滿足,更努力在白天獨居、在夜間獨坐。對那位這樣住於不放逸者,欣悅被生起;對喜悅者,喜被生起;對意喜者,身變得寧靜;身已寧靜者感受樂;對有樂者,心入定;在心得定時,諸法變成明顯[\suttaref{SN.35.97}];以諸法的明顯,就名為『住不放逸者』。」

  福德的流出品第四,其\twnr{攝頌}{35.0}:

  「流出三說,天路二則,

   同類、摩訶男,雨、葛利與難提。」





\pin{有偈的福德的流出品}{41}{50}
\sutta{41}{41}{流出經第一}{https://agama.buddhason.org/SN/sn.php?keyword=55.41}
  「\twnr{比丘}{31.0}們!有這四個\twnr{福德的流出}{705.0}、善的流出、安樂的食物,哪四個?

  比丘們!這裡,\twnr{聖弟子}{24.0}在佛上具備\twnr{不壞淨}{233.0}:『像這樣,那位\twnr{世尊}{12.0}……(中略)\twnr{天-人們的大師}{11.0}、\twnr{佛陀}{3.0}、世尊。』這是第一個福德的流出、善的流出、安樂的食物。

  再者,比丘們!聖弟子在法上……(中略)在\twnr{僧團}{375.0}上……(中略)。

  再者,比丘們!聖弟子具備聖者喜愛的諸戒:無毀壞的……(中略)轉起定的。這是第四個福德的流出、善的流出、安樂的食物。

  比丘們!這是四個福德的流出、善的流出、安樂的食物。

  比丘們!不容易計算具備這四個福德的流出、善的流出聖弟子福德的量:『有這麼多福德的流出、善的流出、安樂的食物。』那時,就名為(就走到稱呼)『不能被計算的、\twnr{不能被測量的}{883.0}大福德蘊』。

  比丘們!猶如不容易計算在大海中水的量:『有這麼多升水。』或『有這麼多百升水。』或『有這麼多千升水。』或『有這麼多十萬升水。』那時,就名為『不能被計算的、不能被測量的大水蘊』。同樣的,比丘們!不容易計算具備這四個福德的流出、善的流出聖弟子福德的量:『有這麼多福德的流出、善的流出、安樂的食物。』那時,就名為『不能被計算的、不能被測量的大福德蘊』。」

  世尊說這個,說這個後,\twnr{善逝}{8.0}、\twnr{大師}{145.0}又更進一步說這個:

  「大海無量的大流,極恐怖的\twnr{寶物聚集之阿賴耶}{707.0},

   正如\twnr{被人群群眾使用的}{706.0}諸河,個個流動、進入海洋。

   像這樣施與人食物飲料衣服者,床床單的施與者,

   福德的水流進入賢智者,就如對海洋運送水的諸河。」



\sutta{42}{42}{流出經第二}{https://agama.buddhason.org/SN/sn.php?keyword=55.42}
  「\twnr{比丘}{31.0}們!有這四個\twnr{福德的流出}{705.0}、善的流出、安樂的食物,哪四個?

  比丘們!這裡,\twnr{聖弟子}{24.0}在佛上具備\twnr{不壞淨}{233.0}:『像這樣,那位\twnr{世尊}{12.0}……(中略)\twnr{天-人們的大師}{11.0}、\twnr{佛陀}{3.0}、世尊。』這是第一個福德的流出、善的流出、安樂的食物。

  再者,比丘們!聖弟子具備對法……(中略)對\twnr{僧團}{375.0}……(中略)。

  再者,比丘們!聖弟子具備聖者喜愛的諸戒:無毀壞的……(中略)轉起定的。這是第四個福德的流出、善的流出、安樂的食物。

  比丘們!這是四個福德的流出、善的流出、安樂的食物。

  比丘們!不容易計算具備這四個福德的流出、善的流出聖弟子福德的量:『有這麼多福德的流出、善的流出、安樂的食物。』那時,就名為(就走到稱呼)『不能被計算的、\twnr{不能被測量的}{883.0}大福德蘊』。

  比丘們!猶如這些大河會合、集合之處,即:恒河、耶牟那、阿致羅筏底、薩羅浮、摩醯,在那裡不容易計算水的量:『有這麼多升水。』或『有這麼多百升水。』或『有這麼多千升水。』或『有這麼多十萬升水。』那時,就名為『不能被計算的、不能被測量的大水蘊』。同樣的,比丘們!不容易計算具備這四個福德的流出、善的流出聖弟子福德的量:『有這麼多福德的流出、善的流出、安樂的食物。』那時,就名為『不能被計算的、不能被測量的大福德蘊』。」

  世尊說這個,說這個後,\twnr{善逝}{8.0}、\twnr{大師}{145.0}又更進一步說這個:

  世尊說這個,說這個後,\twnr{善逝}{8.0}、\twnr{大師}{145.0}又更進一步說這個:

  「大海無量的大流,極恐怖的\twnr{寶物聚集之阿賴耶}{707.0},

   正如\twnr{被人群群眾使用的}{706.0}諸河,個個流動、進入海洋。

   像這樣施與人食物飲料衣服者,床床單的施與者,

   福德的水流進入賢智者,就如對海洋運送水的諸河。」



\sutta{43}{43}{流出經第三}{https://agama.buddhason.org/SN/sn.php?keyword=55.43}
  「\twnr{比丘}{31.0}們!有這四個\twnr{福德的流出}{705.0}、善的流出、安樂的食物,哪四個?

  比丘們!這裡,\twnr{聖弟子}{24.0}在佛上具備\twnr{不壞淨}{233.0}:『像這樣,那位\twnr{世尊}{12.0}……(中略)\twnr{天-人們的大師}{11.0}、\twnr{佛陀}{3.0}、世尊。』這是第一個福德的流出、善的流出、安樂的食物。

  再者,比丘們!聖弟子具備對法……(中略)對\twnr{僧團}{375.0}……(中略)。

  再者,比丘們!聖弟子是有慧者,具備\twnr{導向生起與滅沒}{498.0}、聖、洞察、導向\twnr{苦的完全滅盡}{181.0}之慧,這是第四個福德的流出、善的流出、安樂的食物。

  比丘們!這是四個福德的流出、善的流出、安樂的食物。

  比丘們!不容易計算具備這四個福德的流出、善的流出聖弟子福德的量:『有這麼多福德的流出、善的流出、安樂的食物。』那時,就名為(就走到稱呼)『不能被計算的、\twnr{不能被測量的}{883.0}大福德蘊』。

  世尊說這個,說這個後,\twnr{善逝}{8.0}、\twnr{大師}{145.0}又更進一步說這個:

  「凡想要福德、在善的之上住立者,為了\twnr{不死}{123.0}的獲得\twnr{修習}{94.0}道。

   那位到達法的核心者樂於滅盡上,在死王之到來時他不顫抖。」



\sutta{44}{44}{大富者經第一}{https://agama.buddhason.org/SN/sn.php?keyword=55.44}
  「\twnr{比丘}{31.0}們!具備四法的\twnr{聖弟子}{24.0}被稱為『富裕者、大富者、大財富者』,哪四個?比丘們!這裡,聖弟子在佛上具備\twnr{不壞淨}{233.0}:『像這樣,那位\twnr{世尊}{12.0}是……(中略)\twnr{天-人們的大師}{11.0}、\twnr{佛陀}{3.0}、世尊。』在法上……(中略)在\twnr{僧團}{375.0}上……(中略)具備聖者喜愛的諸戒:無毀壞的……(中略)轉起定的。

  比丘們!具備這四法的聖弟子被稱為『富裕者、大富者、大財富者』。」



\sutta{45}{45}{大富者經第二}{https://agama.buddhason.org/SN/sn.php?keyword=55.45}
  「\twnr{比丘}{31.0}們!具備四法的\twnr{聖弟子}{24.0}被稱為『富裕者、大富者、大財富者、大名聲者』,哪四個?比丘們!這裡,聖弟子在佛上具備\twnr{不壞淨}{233.0}:『像這樣,那位\twnr{世尊}{12.0}是……(中略)\twnr{天-人們的大師}{11.0}、\twnr{佛陀}{3.0}、世尊。』在法上……(中略)在僧團上……(中略)具備聖者喜愛的諸戒:無毀壞的……(中略)轉起定的。

  比丘們!具備這四法的聖弟子被稱為『富裕者、大富者、大財富者、大名聲者』。」



\sutta{46}{46}{概要經}{https://agama.buddhason.org/SN/sn.php?keyword=55.46}
  「\twnr{比丘}{31.0}們!具備四法的\twnr{聖弟子}{24.0}是\twnr{入流者}{165.0}、不墮惡趣法者、\twnr{決定者}{159.0}、\twnr{正覺為彼岸者}{160.0},哪四個?比丘們!這裡,聖弟子在佛上具備\twnr{不壞淨}{233.0}:『像這樣,那位\twnr{世尊}{12.0}是……(中略)\twnr{天-人們的大師}{11.0}、\twnr{佛陀}{3.0}、世尊。』在法上……(中略)在\twnr{僧團}{375.0}上……(中略)具備聖者喜愛的諸戒:無毀壞的……(中略)轉起定的。比丘們!具備這四法的聖弟子是入流者、不墮惡趣法者、決定者、正覺為彼岸者。」



\sutta{47}{47}{難提經}{https://agama.buddhason.org/SN/sn.php?keyword=55.47}
  起源於迦毘羅衛城。

  \twnr{世尊}{12.0}對在一旁坐下的釋迦族人難提說這個:

  「難提!具備四法的\twnr{聖弟子}{24.0}是\twnr{入流者}{165.0}、不墮惡趣法者、\twnr{決定者}{159.0}、\twnr{正覺為彼岸者}{160.0},哪四個?難提!這裡,聖弟子在佛上具備\twnr{不壞淨}{233.0}:『像這樣,那位\twnr{世尊}{12.0}是……(中略)\twnr{天-人們的大師}{11.0}、\twnr{佛陀}{3.0}、世尊。』在法上……(中略)在\twnr{僧團}{375.0}上……(中略)具備聖者喜愛的諸戒:無毀壞的……(中略)轉起定的。難提!具備這四法的聖弟子是入流者、不墮惡趣法者、決定者、正覺為彼岸者。」



\sutta{48}{48}{拔提亞經}{https://agama.buddhason.org/SN/sn.php?keyword=55.48}
  起源於迦毘羅衛城。

  \twnr{世尊}{12.0}對在一旁坐下的釋迦族人拔提亞說這個:

  「拔提亞!具備四法的\twnr{聖弟子}{24.0}是\twnr{入流者}{165.0}、不墮惡趣法者、\twnr{決定者}{159.0}、\twnr{正覺為彼岸者}{160.0},哪四個?

  拔提亞!這裡,聖弟子具備對佛……(中略)在諸法上……(中略)在\twnr{僧團}{375.0}上……(中略)具備聖者喜愛的諸戒:無毀壞的……(中略)轉起定的。

  拔提亞!具備這四法的聖弟子是入流者、不墮惡趣法者、決定者、正覺為彼岸者。」



\sutta{49}{49}{摩訶男經}{https://agama.buddhason.org/SN/sn.php?keyword=55.49}
  起源於迦毘羅衛城。

  \twnr{世尊}{12.0}對在一旁坐下的釋迦族人摩訶男說這個:

  「摩訶男!具備四法的\twnr{聖弟子}{24.0}是\twnr{入流者}{165.0}……(中略)、\twnr{以正覺為彼岸}{160.0},哪四個?

  摩訶男!這裡,聖弟子具備對佛……(中略)在諸法上……(中略)在\twnr{僧團}{375.0}上……(中略)具備聖者喜愛的諸戒:無毀壞的……(中略)轉起定的。 

  摩訶男!具備這四法的聖弟子是入流者、不墮惡趣法者、\twnr{決定者}{159.0}、\twnr{正覺為彼岸者}{160.0}。」



\sutta{50}{50}{支經}{https://agama.buddhason.org/SN/sn.php?keyword=55.50}
  「\twnr{比丘}{31.0}們!有這四\twnr{入流支}{370.0},哪四個?善人的親近,聽聞正法,\twnr{如理作意}{114.0},\twnr{法、隨法行}{58.0},比丘們!這些是四入流支。」

  有偈的福德潤澤品第五,其\twnr{攝頌}{35.0}:

  「流出三說,與以二則大富者,

   概要、難提、拔提亞,以摩訶男、支它們為十。」





\pin{有慧品}{51}{61}
\sutta{51}{51}{有偈經}{https://agama.buddhason.org/SN/sn.php?keyword=55.51}
  「\twnr{比丘}{31.0}們!具備四法的\twnr{聖弟子}{24.0}是\twnr{入流者}{165.0}、不墮惡趣法者、\twnr{決定者}{159.0}、\twnr{正覺為彼岸者}{160.0},哪四個?比丘們!這裡,聖弟子在佛上具備\twnr{不壞淨}{233.0}:『像這樣,那位\twnr{世尊}{12.0}是……(中略)\twnr{天-人們的大師}{11.0}、\twnr{佛陀}{3.0}、世尊。』在法上……(中略)在\twnr{僧團}{375.0}上……(中略)具備聖者喜愛的諸戒:無毀壞的……(中略)轉起定的。比丘們!具備這四法的聖弟子是入流者、不墮惡趣法者、決定者、正覺為彼岸者。」

  世尊說這個,說這個後,\twnr{善逝}{8.0}、\twnr{大師}{145.0}又更進一步說這個:

  「凡其在如來上的信,是不動的、已善住立的,

   以及凡其善戒,是聖者喜愛的、所稱讚的。

   凡其在僧團上有\twnr{淨信}{507.0},且見已成為正直的,

   他們說他『是不貧窮的』,他的生命是不空虛的。

   因此信與戒,淨信、法的看見,

   憶念佛陀教誡的有智慧者,應該實踐。」



\sutta{52}{52}{已住過雨季安居經}{https://agama.buddhason.org/SN/sn.php?keyword=55.52}
  \twnr{有一次}{2.0},\twnr{世尊}{12.0}住在舍衛城祇樹林給孤獨園。

  那時,某位在舍衛城已住過\twnr{雨季安居}{231.0}的\twnr{比丘}{31.0},正以某些應該被作的已抵達迦毘羅衛城。

  迦毘羅衛城的釋迦族人聽聞:

  「聽說某位在舍衛城已住過雨季安居的比丘,已抵達迦毘羅衛城。」

  那時,迦毘羅衛城的釋迦族人去見那位比丘。抵達後,向那位比丘\twnr{問訊}{46.0}後,在一旁坐下。在一旁坐下的迦毘羅衛城的釋迦族人對那位比丘說這個:

  「\twnr{大德}{45.0}!世尊是否是無病的,同時也是有氣力的呢?」

  「\twnr{朋友}{201.0}們!世尊是無病的,同時也是有氣力的。」

  「大德!又,舍利弗、目揵連是否也是無病的,同時也是有氣力的呢?」

  「朋友們!舍利弗、目揵連也是無病的,同時也是有氣力的。」

  「大德!又,是否比丘\twnr{僧團}{375.0}也是無病的,同時也是有氣力的呢?」

  「朋友們!比丘僧團也是無病的,同時也是有氣力的。」

  「大德!又,在這雨季安居中間,有什麼被你從世尊的面前聽聞,從面前領受?」

  「朋友們!這被我從世尊的面前聽聞,從面前領受:『比丘們!那些是少的:凡比丘們以諸\twnr{漏}{188.0}的滅盡,以證智自作證後,在當生中\twnr{進入後住於}{66.0}無漏\twnr{心解脫}{16.0}、\twnr{慧解脫}{539.0},而這些正是更多的:凡比丘們以\twnr{五下分結}{134.0}的滅盡,成為\twnr{化生}{346.0}者、在那裡般涅槃者、不從那個世間返還者。』

  朋友們!其次,這也被我從世尊的面前聽聞,從面前領受:『比丘們!那些是少的:凡比丘們以五下分結的滅盡,成為化生者、在那裡般涅槃者、不從那個世間返還者,而這些正是更多的:凡比丘們以三結的遍盡,以貪、瞋、癡薄的狀態,為\twnr{一來}{208.0}者,只回來這個世間一次後,將作苦的終結。』

  朋友們!其次,這也被我從世尊的面前聽聞,從面前領受:『比丘們!那些是少的:凡比丘們以三結的遍盡,以貪、瞋、癡薄的狀態,為一來者,只回來這個世間一次後,將作苦的終結,而這些正是更多的:凡比丘們以三結的遍盡,為\twnr{入流者}{165.0}、不墮惡趣法者、\twnr{決定者}{159.0}、\twnr{正覺為彼岸者}{160.0}。』」



\sutta{53}{53}{法施經}{https://agama.buddhason.org/SN/sn.php?keyword=55.53}
  \twnr{有一次}{2.0},\twnr{世尊}{12.0}住在波羅奈仙人墜落處的鹿林。

  那時,法施\twnr{優婆塞}{98.0}與約五百位優婆塞去見世尊。抵達後,向世尊\twnr{問訊}{46.0}後,在一旁坐下。在一旁坐下的法施優婆塞對世尊說這個:

  「\twnr{大德}{45.0}!請世尊教誡我們,大德!請世尊訓誡我們:凡對我們會有長久的利益、安樂。」 

  「法施!因此,在這裡,應該被你們這麼學:『我們將經常\twnr{進入後住於}{66.0}那些被如來說的甚深、義之甚深、出世間、\twnr{空關聯的}{637.0}經典。』法施!應該被你們這麼學。」

  「大德!以居住兒子擁擠的床的,享用迦尸的檀香的,持有花環、香料、塗油的,受用金銀的我們,這是不容易的:『我們將經常進入後住於那些被如來說的甚深、義之甚深、出世間、空關聯的經典。』大德!對那些在五\twnr{學處}{392.0}上住立的我們,請世尊為我們教導更上的法。」

  「法施!因此,在這裡,應該被你們這麼學:『我們將在佛上具備\twnr{不壞淨}{233.0}:『像這樣,那位世尊是……(中略)\twnr{天-人們的大師}{11.0}、\twnr{佛陀}{3.0}、\twnr{世尊}{12.0}。』在法上……(中略)在\twnr{僧團}{375.0}上……(中略)具備聖者喜愛的諸戒:無毀壞的……(中略)轉起定的。法施!應該被你們這麼學。」

  「大德!凡被世尊教導的這些四\twnr{入流支}{370.0},那些法在我們中被發現,且我們在那些法中被發現。大德!因為我們在佛上具備不壞淨:『像這樣,那位世尊是……(中略)天-人們的大師、佛陀、世尊。』在法上……(中略)在僧團上……(中略)我具備聖者喜愛的諸戒:無毀壞的……(中略)轉起定的。」

  「法施!是你的利得,法施!\twnr{是你們的善得的}{350.0},法施!\twnr{入流果}{165.1}被你們\twnr{記說}{179.0}。」



\sutta{54}{54}{病經}{https://agama.buddhason.org/SN/sn.php?keyword=55.54}
  \twnr{有一次}{2.0},\twnr{世尊}{12.0}住在釋迦族人的迦毘羅衛城尼拘律園。

  當時,眾多\twnr{比丘}{31.0}為世尊作衣服的工作:「經過三個月,完成衣服的世尊將出發\twnr{遊行}{61.0}。」

  釋迦族人摩訶男聽聞:

  「聽說眾多比丘為世尊作衣服的工作:『經過三個月,完成衣服的世尊將出發遊行。』」

  那時,釋迦族人摩訶男去見世尊。抵達後,向世尊\twnr{問訊}{46.0}後,在一旁坐下。在一旁坐下的釋迦族人摩訶男對世尊說這個:

  「\twnr{大德}{45.0}!這被聽聞:『聽說眾多比丘為世尊作衣服的工作:「經過三個月,完成衣服的世尊將出發遊行。」』

  大德!這沒被我們從世尊的面前聽聞,從面前領受:生病的、受苦的、重病的有慧\twnr{優婆塞}{98.0}應該[如何]被有慧的優婆塞教誡?」

  「摩訶男!生病的、受苦的、重病的有慧優婆塞應該被有慧的優婆塞以四個能被安心的法使之安心(蘇息):

  請\twnr{尊者}{200.0}安心:尊者有在佛上\twnr{不壞淨}{233.0}:『像這樣,那位世尊……(中略)\twnr{天-人們的大師}{11.0}、\twnr{佛陀}{3.0}、世尊。』

  請尊者請安心:尊者有在法上……(中略)在\twnr{僧團}{375.0}上……(中略)聖者喜愛的諸戒:『無毀壞的……(中略)轉起定的。

  摩訶男!生病的、受苦的、重病的有慧優婆塞應該被有慧的優婆塞以這四個能被安心的法使之安心後,應該被這回答:『尊者有在父母上的關注(掛慮)嗎?』

  如果他這麼說:『有我在父母上的關注。』他應該被這回答:『\twnr{親愛的先生}{204.0}!尊者為\twnr{死法}{587.3},如果尊者在父母上作關注,也仍將死;如果尊者在父母上不作關注,也仍將死,請尊者捨斷你在父母上的關注,\twnr{那就好了}{44.0}!』

  如果他這麼說:『凡我在父母上的關注,那個已捨斷。』他應該被這回答:『又,尊者有在妻兒上的關注嗎?』

  如果他這麼說:『有我在妻兒上的關注。』他應該被這回答:『親愛的先生!尊者為死法,如果尊者在妻兒上作關注,也仍將死;如果尊者在妻兒上不作關注,也仍將死,請尊者捨斷你在妻兒上的關注,那就好了!』

  如果他這麼說:『凡我在妻兒上的關注,那個已捨斷。』他應該被這回答:『又,尊者有在人的五種欲上的關注嗎?』

  如果他這麼說:『有我在人的五種欲上的關注。』他應該被這回答:『\twnr{朋友}{201.0}!天的諸欲比人的諸欲是更優越的與更勝妙的,尊者使心從人的諸欲出來後,請你使心志向四大王天,那就好了!』

  如果他這麼說:『我的心已從人的諸欲出來,已使心志向四大王天。』他應該被這回答:『朋友!三十三天比四大王天是更優越的與更勝妙的,尊者使心從四大王天出來後,請你使心志向三十三天,那就好了!』

  如果他這麼說:『我的心已從四大王天出來,已使心志向三十三天。』他應該被這回答:『朋友!夜摩天比三十三天……(中略)兜率天……(中略)\twnr{化樂}{371.0}天……(中略)\twnr{他化自在天}{372.0}……(中略)朋友!梵天世界比他化自在天是更優越的與更勝妙的,尊者使心從他化自在天出來後,請你使心志向梵天世界,那就好了!』

  如果他這麼說:『我的心已從他化自在天出來,已使心志向梵天世界。』他應該被這回答:『朋友!梵天世界也是無常的、不堅固的、\twnr{有身}{93.0}所包含的,尊者使心從梵天出來後,請你集中心於有身的滅,那就好了!』

  如果他這麼說:『我的心已從梵天世界出來,我集中心於有身的滅。』摩訶男!我說:『對這樣解脫心的優婆塞來說,\twnr{與心從諸}{x654}\twnr{漏}{188.0}解脫的比丘沒任何差異,即:解脫者與解脫者。』」



\sutta{55}{55}{入流果經}{https://agama.buddhason.org/SN/sn.php?keyword=55.55}
  「\twnr{比丘}{31.0}們!有這四法,已\twnr{修習}{94.0}、已\twnr{多作}{95.0},轉起\twnr{入流果}{165.1}的作證,哪四個?善人的親近,聽聞正法,如理作意,法、\twnr{隨法行}{58.0},比丘們!這些是四法,已修習、已多作,轉起入流果的作證。」



\sutta{56}{56}{一來果經}{https://agama.buddhason.org/SN/sn.php?keyword=55.56}
  「\twnr{比丘}{31.0}們!有這四法,已\twnr{修習}{94.0}、已\twnr{多作}{95.0},轉起\twnr{一來果}{208.2}的作證,哪四個?……(中略)轉起一來果的作證。」



\sutta{57}{57}{不還果經}{https://agama.buddhason.org/SN/sn.php?keyword=55.57}
  「……(中略)轉起\twnr{不還果}{209.1}的作證……(中略)。」



\sutta{58}{58}{阿羅漢果經}{https://agama.buddhason.org/SN/sn.php?keyword=55.58}
  「……(中略)轉起\twnr{阿羅漢}{5.0}果的作證……(中略)。」



\sutta{59}{59}{慧的獲得經}{https://agama.buddhason.org/SN/sn.php?keyword=55.59}
  「……(中略)轉起慧的獲得……(中略)。」



\sutta{60}{60}{慧的增長經}{https://agama.buddhason.org/SN/sn.php?keyword=55.60}
  「……(中略)轉起慧的增長……(中略)。」



\sutta{61}{61}{慧的廣大經}{https://agama.buddhason.org/SN/sn.php?keyword=55.61}
  「……(中略)轉起慧的擴展……(中略)。」

  有慧品第六,其\twnr{攝頌}{35.0}:

  「有偈、已住過雨季安居,法施與病,

   四則果、獲得,增長、廣大。」





\pin{大慧品}{62}{74}
\sutta{62}{62}{大慧經}{https://agama.buddhason.org/SN/sn.php?keyword=55.62}
  「\twnr{比丘}{31.0}們!有這四法,已\twnr{修習}{94.0}、已\twnr{多作}{95.0},轉起大慧的狀態,哪四個?善人的親近,聽聞正法,如理作意,法、\twnr{隨法行}{58.0},比丘們!這些是四法,已修習、已多作,轉起大慧的狀態。」



\sutta{63}{63}{博慧經}{https://agama.buddhason.org/SN/sn.php?keyword=55.63}
  「……(中略)轉起博慧的狀態。」



\sutta{64}{64}{廣大慧經}{https://agama.buddhason.org/SN/sn.php?keyword=55.64}
  「……(中略)轉起廣大慧的狀態。」



\sutta{65}{65}{深慧經}{https://agama.buddhason.org/SN/sn.php?keyword=55.65}
  「……(中略)轉起深慧的狀態。」



\sutta{66}{66}{不放逸慧經}{https://agama.buddhason.org/SN/sn.php?keyword=55.66}
  「……(中略)轉起\twnr{不放逸慧的狀態}{x655}。」



\sutta{67}{67}{廣慧經}{https://agama.buddhason.org/SN/sn.php?keyword=55.67}
  「……(中略)轉起廣慧的狀態。」



\sutta{68}{68}{豐富慧經}{https://agama.buddhason.org/SN/sn.php?keyword=55.68}
  「……(中略)轉起豐富慧的狀態。」



\sutta{69}{69}{急速慧經}{https://agama.buddhason.org/SN/sn.php?keyword=55.69}
  「……(中略)轉起急速慧的狀態。」



\sutta{70}{70}{輕快慧經}{https://agama.buddhason.org/SN/sn.php?keyword=55.70}
  「……(中略)轉起輕快慧的狀態。」



\sutta{71}{71}{捷慧經}{https://agama.buddhason.org/SN/sn.php?keyword=55.71}
  「……(中略)轉起捷慧的狀態。」



\sutta{72}{72}{速慧經}{https://agama.buddhason.org/SN/sn.php?keyword=55.72}
  「……(中略)轉起速慧的狀態。」



\sutta{73}{73}{利慧經}{https://agama.buddhason.org/SN/sn.php?keyword=55.73}
  「……(中略)轉起\twnr{利慧}{978.0}的狀態。」



\sutta{74}{74}{洞察慧經}{https://agama.buddhason.org/SN/sn.php?keyword=55.74}
  「……(中略)轉起\twnr{洞察慧}{566.0}的狀態,哪四個?善人的親近,聽聞正法,如理作意,法、\twnr{隨法行}{58.0},\twnr{比丘}{31.0}們!有這四法,已\twnr{修習}{94.0}、已\twnr{多作}{95.0},轉起洞察慧的狀態。」

  大慧品第七,其\twnr{攝頌}{35.0}:

  「大、廣、廣大、深,不放逸、寬廣、豐富,

   速、輕、疾、速行,以及銳利、洞察。」

  入流相應第十一。





\page

\xiangying{56}{諦相應}
\pin{定品}{1}{10}
\sutta{1}{1}{定經}{https://agama.buddhason.org/SN/sn.php?keyword=56.1}
  起源於舍衛城。

  「\twnr{比丘}{31.0}們!你們要\twnr{修習}{94.0}定。比丘們!得定的比丘如實知道。而如實知道什麼?如實知道『這是苦』;如實知道『這是苦集』;如實知道『這是苦滅』;如實知道『這是導向苦\twnr{滅道跡}{69.0}』。

  比丘們!你們要修習定。比丘們!得定的比丘如實知道。

  比丘們!因此,在這裡,『這是苦。』努力應該被作,『這是苦集。』努力應該被作,『這是苦滅。』努力應該被作,『這是導向苦滅道跡。』努力應該被作。」



\sutta{2}{2}{獨坐經}{https://agama.buddhason.org/SN/sn.php?keyword=56.2}
  「\twnr{比丘}{31.0}們!你們在\twnr{獨坐}{92.0}上來到努力,比丘們!獨坐的比丘如實知道。而如實知道什麼?如實知道『這是苦』;如實知道『這是苦集』;如實知道『這是苦滅』;如實知道『這是導向苦\twnr{滅道跡}{69.0}』。

  比丘們!你們要著手努力於獨坐,比丘們!獨坐的比丘如實知道。

  比丘們!因此,在這裡,『這是苦。』努力應該被作,『這是苦集。』努力應該被作,『這是苦滅。』努力應該被作,『這是導向苦滅道跡。』努力應該被作。」



\sutta{3}{3}{善男子經第一}{https://agama.buddhason.org/SN/sn.php?keyword=56.3}
  「\twnr{比丘}{31.0}們!凡過去世任何\twnr{善男子}{41.0}曾從在家正確地出家成為無家者,他們全部是為了四聖諦的如實\twnr{現觀}{53.0}。

  比丘們!凡\twnr{未來世}{308.0}任何善男子將從在家正確地出家成為無家者,他們全部是為了四聖諦的如實現觀。

  比丘們!凡現在任何善男子從在家正確地出家成為無家者,他們全部是為了四聖諦的如實現觀。

  哪四個?苦聖諦、苦集聖諦、苦滅聖諦、導向苦\twnr{滅道跡}{69.0}聖諦。

  比丘們!凡過去世任何善男子從在家正確地出家成為無家者……(中略)將出家……(中略)出家,他們全部是為了就這些四聖諦的如實現觀。

  比丘們!因此,在這裡,『這是苦。』努力應該被作,『這是苦集。』努力應該被作,『這是苦滅。』努力應該被作,『這是導向苦滅道跡。』努力應該被作。」



\sutta{4}{4}{善男子經第二}{https://agama.buddhason.org/SN/sn.php?keyword=56.4}
  「\twnr{比丘}{31.0}們!凡過去世任何從在家正確地出家\twnr{成為無家者}{48.0}的\twnr{善男子}{41.0}曾如實\twnr{現觀}{53.0},他們全部曾如實現觀四聖諦。

  比丘們!凡未來世任何將從在家正確地出家成為無家者的善男子將如實現觀,他們全部將如實現觀四聖諦。

  比丘們!凡現在任何從在家正確地出家成為無家者的善男子如實現觀,他們全部如實現觀四聖諦。

  哪四個?苦聖諦、苦集聖諦、苦滅聖諦、導向苦\twnr{滅道跡}{69.0}聖諦。

  比丘們!凡過去世任何從在家正確地出家成為無家者的善男子曾如實現觀……(中略)將現觀……(中略)現觀,他們全部如實現觀這些四聖諦。

  比丘們!因此,在這裡,『這是苦。』努力應該被作……(中略)『這是導向苦滅道跡。』努力應該被作。」



\sutta{5}{5}{沙門婆羅門經第一}{https://agama.buddhason.org/SN/sn.php?keyword=56.5}
  「\twnr{比丘}{31.0}們!凡過去世任何\twnr{沙門}{29.0}或\twnr{婆羅門}{17.0}曾如實\twnr{現正覺}{75.0},他們全部曾如實現正覺四聖諦。

  比丘們!凡\twnr{未來世}{308.0}任何沙門或婆羅門將如實現正覺,他們全部將如實現正覺四聖諦。

  比丘們!凡現在任何沙門或婆羅門如實現正覺,他們全部如實現正覺四聖諦。

  哪四個?苦聖諦……(中略)導向苦\twnr{滅道跡}{69.0}聖諦。

  比丘們!凡過去世任何沙門或婆羅門曾如實現正覺……(中略)將現正覺……(中略)現正覺……(中略),他們全部如實現正覺這些四聖諦。

  比丘們!因此,在這裡,『這是苦。』努力應該被作……(中略)『這是導向苦滅道跡。』努力應該被作。」



\sutta{6}{6}{沙門婆羅門經第二}{https://agama.buddhason.org/SN/sn.php?keyword=56.6}
  「\twnr{比丘}{31.0}們!凡過去世任何如實\twnr{現正覺}{75.0}的\twnr{沙門}{29.0}或\twnr{婆羅門}{17.0}曾說明,他們全部曾說明如實現正覺的四聖諦。

  比丘們!凡\twnr{未來世}{308.0}任何如實現正覺的沙門或婆羅門將說明,他們全部將說明如實現正覺的四聖諦。

  比丘們!凡現在任何如實現正覺的沙門或婆羅門說明,他們全部說明如實現正覺的四聖諦。

  哪四個?苦聖諦……(中略)導向苦\twnr{滅道跡}{69.0}聖諦。

  比丘們!凡過去世任何如實現正覺的沙門或婆羅門說明……(中略)將說明……(中略)說明,他們全部說明如實現正覺的這些四聖諦。

  比丘們!因此,在這裡,『這是苦。』努力應該被作……(中略)『這是導向苦滅道跡。』努力應該被作。」



\sutta{7}{7}{尋經}{https://agama.buddhason.org/SN/sn.php?keyword=56.7}
  「\twnr{比丘}{31.0}們!你們應該不要尋思諸惡不善尋,即:欲尋、惡意尋、\twnr{加害尋}{376.2},那什麼原因呢?比丘們!這些尋是不\twnr{伴隨利益的}{50.0},非\twnr{梵行基礎的}{446.0},不對\twnr{厭}{15.0}、不對\twnr{離貪}{77.0}、不對\twnr{滅}{68.0}、不對寂靜、不對證智、不對\twnr{正覺}{185.1}、不對涅槃轉起。

  比丘們!而當你們尋思時,應該尋思『這是苦。』應該尋思『這是苦集。』應該尋思『這是苦滅。』應該尋思『這是導向苦\twnr{滅道跡}{69.0}。』那什麼原因呢?比丘們!這些尋是伴隨利益的,這些是梵行基礎的,這些對厭、對離貪、對滅、對寂靜、對證智、對正覺、對涅槃轉起。

  比丘們!因此,在這裡,『這是苦。』努力應該被作……(中略)『這是導向苦滅道跡。』努力應該被作。」



\sutta{8}{8}{思惟經}{https://agama.buddhason.org/SN/sn.php?keyword=56.8}
  「比丘們!你們應該不要思惟惡不善{心}[思惟]:『世界是常恆的』,或『世界是非常恆的』,或『世界是有邊的』,或『世界是無邊的』,或『命即是身體』,或『命是一身體是另一』,或『死後如來存在』,或『死後如來不存在』,或『\twnr{死後如來存在且不存在}{354.0}』,或『死後如來既非存在也非不存在』,那什麼原因呢?比丘們!這個思惟是不\twnr{伴隨利益的}{50.0},非\twnr{梵行基礎的}{446.0},不對\twnr{厭}{15.0}、不對\twnr{離貪}{77.0}、不對\twnr{滅}{68.0}、不對寂靜、不對證智、不對\twnr{正覺}{185.1}、不對涅槃轉起。

  比丘們!而當你們思惟時,應該思惟『這是苦。』應該思惟『這是苦集。』應該思惟『這是苦滅。』應該思惟『這是導向苦\twnr{滅道跡}{69.0}。』那什麼原因呢?比丘們!這些尋是伴隨利益的,這些是梵行基礎的,這些對厭、對離貪、對滅、對寂靜、對證智、對正覺、對涅槃轉起。

  比丘們!因此,在這裡,『這是苦。』努力應該被作……(中略)『這是導向苦滅道跡。』努力應該被作。」



\sutta{9}{9}{諍論經}{https://agama.buddhason.org/SN/sn.php?keyword=56.9}
  「\twnr{比丘}{31.0}們!你們應該不要談論諍論:『你不了知這法、律,我了知這法、律;你了知這法、律什麼!你是邪行者,我是正行者;應該先說的你後說,應該後說的你先說;我的是一致的,你的是不一致的;你長時間熟練的是顛倒的;你已被論破(你的理論已被反駁),請你去救(使脫離)理論;你已被折伏,或請你解開,如果你能夠。』那什麼原因呢?比丘們!這些談論是不\twnr{伴隨利益的}{50.0},非\twnr{梵行基礎的}{446.0},不對\twnr{厭}{15.0}、不對\twnr{離貪}{77.0}、不對\twnr{滅}{68.0}、不對寂靜、不對證智、不對\twnr{正覺}{185.1}、不對涅槃轉起。

  比丘們!而當你們談論時,應該談論『這是苦。』應該談論『這是苦集。』應該談論『這是苦滅。』應該談論『這是導向苦\twnr{滅道跡}{69.0}』……(中略)努力應該被作。」



\sutta{10}{10}{畜生論經}{https://agama.buddhason.org/SN/sn.php?keyword=56.10}
  「\twnr{比丘}{31.0}們!你們應該不要談論各種\twnr{畜生論}{442.0},即:國王論、盜賊論、大臣論、軍隊論、\twnr{怖畏論}{753.2}、戰爭論、食物論、飲料論、衣服論、臥具論、花環論、氣味論、親里論、車乘論、村落論、城鎮論、城市論、國土論、女人論、英雄論、街道論(街道流言)、水井論(井邊流言)、祖靈論、種種論、世界起源論、海洋起源論、\twnr{如是有無}{931.0}論等,那什麼原因呢?比丘們!這些談論是不\twnr{伴隨利益的}{50.0},非\twnr{梵行基礎的}{446.0},不對\twnr{厭}{15.0}、不對\twnr{離貪}{77.0}、不對\twnr{滅}{68.0}、不對寂靜、不對證智、不對\twnr{正覺}{185.1}、不對涅槃轉起。

  比丘們!而當你們談論時,應該談論『這是苦。』應該談論『這是苦集。』應該談論『這是苦滅。』應該談論『這是導向苦\twnr{滅道跡}{69.0}。』那什麼原因呢?比丘們!這些談論是伴隨利益的,這些是梵行基礎的,這些對厭、對離貪、對滅、對寂靜、對證智、對正覺、對涅槃轉起。

  比丘們!因此,在這裡,『這是苦。』努力應該被作……(中略)『這是導向苦滅道跡。』努力應該被作。」

  定品第一,其\twnr{攝頌}{35.0}:

  「定、獨坐,善男子二則在後,

   沙門婆羅門、尋,思惟、諍論、談論。」





\pin{法輪轉起品}{11}{20}
\sutta{11}{11}{法輪轉起經}{https://agama.buddhason.org/SN/sn.php?keyword=56.11}
  \twnr{有一次}{2.0},\twnr{世尊}{12.0}住在波羅奈仙人墜落處的鹿林。

  在那裡,世尊召喚\twnr{五位一群的比丘們}{828.0}:

  「\twnr{比丘}{31.0}們!有兩個邊(極端),不應該被出家人實行,哪兩個呢?凡這在諸欲上欲之享樂的實踐:下劣的、粗俗的、一般人的、非聖者的、伴隨無利益的,以及凡這自我折磨的實踐:苦的、非聖者的、伴隨無利益的。比丘們!不走入這些那些兩個邊後,\twnr{作眼、作智}{505.0}的\twnr{中道}{399.0}被\twnr{如來}{4.0}\twnr{現正覺}{75.0},它轉起寂靜、證智、\twnr{正覺}{185.1}、涅槃。

  比丘們!而哪個是那個作眼、作智的中道被如來現正覺,它轉起寂靜、證智、正覺、涅槃?就是這\twnr{八支聖道}{525.0},即:正見、正志、正語、正業、正命、正精進、正念、正定。

  比丘們!這是那個作眼、作智的中道被如來現正覺,它轉起寂靜、證智、正覺、涅槃。

  比丘們!又,這是苦聖諦:生是苦,老也是苦,病也是苦,死也是苦,與不愛的結合也是苦,與所愛的別離也是苦,凡沒得到想要的,那也是苦。以簡要:\twnr{五取蘊}{36.0}是苦。

  比丘們!又,這是苦集聖諦:凡這個導致再有的、與歡喜及貪俱行的、\twnr{到處歡喜的}{96.2}渴愛,即:欲的渴愛、有的渴愛、\twnr{虛無的渴愛}{244.0}。

  比丘們!又,這是苦滅聖諦:凡正是那個渴愛的\twnr{無餘褪去與滅}{491.0}、捨棄、\twnr{斷念}{211.0}、解脫、無\twnr{阿賴耶}{391.0}。

  比丘們!又,這是導向苦\twnr{滅道跡}{69.0}聖諦:就是這\twnr{八支聖道}{525.0},即:正見……(中略)正定。

  『這是苦聖諦』:比丘們!在以前不曾聽聞的諸法上,我的眼生起,智生起,慧生起,明生起,\twnr{光生起}{511.0}。

  又,那個『這苦聖諦應該被\twnr{遍知}{154.0}』:比丘們!在以前……(中略)生起。

  又,那個『這苦聖諦已被遍知』:比丘們!在以前不曾聽聞的諸法上,我的眼生起,智生起,慧生起,明生起,光生起。

  『這是苦集聖諦』:比丘們!在以前不曾聽聞的諸法上,我的眼生起,智生起,慧生起,明生起,光生起。

  又,那個『這苦集聖諦應該被捨斷』:比丘們!在以前……(中略)生起。

  又,那個『這苦集聖諦已被捨斷』:比丘們!在以前不曾聽聞的諸法上,我的眼生起,智生起,慧生起,明生起,光生起。

  『這是苦滅聖諦』:比丘們!在以前不曾聽聞的諸法上,我的眼生起,智生起,慧生起,明生起,光生起。

  又,那個『這苦滅聖諦應該被作證』:比丘們!在以前……(中略)生起。

  又,那個『這苦滅聖諦已被作證』:比丘們!在以前不曾聽聞的諸法上,我的眼生起,智生起,慧生起,明生起,光生起。

  『這是導向苦滅道跡聖諦』:比丘們!在以前不曾聽聞的諸法上,我的眼生起,智生起,慧生起,明生起,光生起。

  又,那個『這導向苦滅道跡聖諦應該被\twnr{修習}{94.0}』:比丘們!在以前……(中略)生起。

  又,那個『這導向苦滅道跡聖諦已被修習』:比丘們!在以前不曾聽聞的諸法上,我的眼生起,智生起,慧生起,明生起,光生起。

  比丘們!只要我在這四聖諦上\twnr{三轉}{x656}、\twnr{十二行相}{x657}沒有這麼已善清淨的如實\twnr{智見}{433.0},比丘們!我在包括天,在包括魔,在包括梵的世間;在包括沙門婆羅門,在包括天-人的\twnr{世代}{38.0}中,就不自稱『\twnr{已現正覺}{75.0}\twnr{無上遍正覺}{37.0}』。

  比丘們!但當我在這四聖諦上三轉、十二行相有這麼已善清淨的如實智見,比丘們!那時,我在包括天,在包括魔,在包括梵的世間;在包括沙門婆羅門,在包括天-人的世代中,才自稱『已現正覺無上遍正覺』。而且,我的\twnr{智與見}{433.0}生起:『我的解脫是不動搖的,這是最後的出生,現在,沒有\twnr{再有}{21.0}。』」

  世尊說這個,悅意的五位一群的比丘們歡喜世尊的所說。

  還有,\twnr{在當這個解說被說時}{136.0},\twnr{尊者}{200.0}憍陳如的\twnr{遠塵、離垢之法眼}{62.0}生起:

  「凡任何\twnr{集法}{67.1}那個全部是\twnr{滅法}{68.1}。」

  而且,在法輪被世尊轉起時,諸地居天使隨聽到聲音:

  「在波羅奈仙人墜落處的鹿林,這個無上法輪被世尊轉起,不能被沙門,或被婆羅門,或被天,或被魔,或被梵,或被世間中任何者反轉。」

  聽到諸地居天的聲音後,諸四大天王天使隨聽到聲音:

  「在波羅奈仙人墜落處的鹿林,這個無上法輪被世尊轉起,不能被沙門,或被婆羅門,或被天,或被魔,或被梵,或被世間中任何者反轉。」

  聽到諸四大天王天的聲音後,諸三十三天……(中略)諸夜摩天……(中略)諸兜率天……(中略)諸\twnr{化樂天}{371.0}……(中略)諸\twnr{他化自在天}{372.0}……(中略)諸梵眾天使隨聽到聲音:

  「在波羅奈仙人墜落處的鹿林,這個無上法輪被世尊轉起,不能被沙門,或被婆羅門,或被天,或被魔,或被梵,或被世間中任何者反轉。」

  像這樣,在那個剎那,(在那個頃刻,)在那個片刻,聲音傳播直到梵天世界。

  這十千世間界震動、大震動、激烈震動,以及無量廣大的光明在世間出現,超越諸天眾的天威。

  那時,世尊吟出這\twnr{優陀那}{184.0}:

  「\twnr{先生}{202.0}!憍陳如確實知道,先生!憍陳如確實知道。」

  像這樣,尊者憍陳如就有這個「\twnr{阿若憍陳如}{x658}」的名字。



\sutta{12}{12}{如來經}{https://agama.buddhason.org/SN/sn.php?keyword=56.12}
  『這是苦聖諦』:\twnr{比丘}{31.0}們!在以前不曾聽聞的諸法上,如來的眼生起,智生起,慧生起,明生起,光生起。

  又,那個『這苦聖諦應該被\twnr{遍知}{154.0}』:比丘們!在以前不曾聽聞的諸法上,如來的……(中略)生起。

  又,那個『這苦聖諦已被遍知』:比丘們!在以前不曾聽聞的諸法上,如來的眼生起,智生起,慧生起,明生起,光生起。

  『這是苦集聖諦』:比丘們!在以前不曾聽聞的諸法上,如來的眼生起,智生起,慧生起,明生起,光生起。

  又,那個『這苦集聖諦應該被捨斷』:比丘們!在以前不曾聽聞的諸法上,如來的眼生起,智生起,慧生起,明生起,光生起。

  又,那個『這苦集聖諦已被捨斷』:比丘們!在以前不曾聽聞的諸法上,如來的……(中略)生起。

  『這是苦滅聖諦』:比丘們!在以前不曾聽聞的諸法上,如來的眼生起,智生起,慧生起,明生起,光生起。

  又,那個『這苦滅聖諦應該被作證』:比丘們!在以前不曾聽聞的諸法上,如來的……(中略)起。

  又,那個『這苦滅聖諦已被作證』:比丘們!在以前不曾聽聞的諸法上,如來的眼生起,智生起,慧生起,明生起,光生起。

  『這是導向苦滅道跡聖諦』:比丘們!在以前不曾聽聞的諸法上,如來的眼生起,智生起,慧生起,明生起,光生起。

  又,那個『這導向苦滅道跡聖諦應該被\twnr{修習}{94.0}』:比丘們!在以前不曾聽聞的諸法上,如來的……(中略)生起。

  又,那個『這導向苦滅道跡聖諦已被修習』:比丘們!在以前不曾聽聞的諸法上,如來的眼生起,智生起,慧生起,明生起,光生起。」



\sutta{13}{13}{蘊經}{https://agama.buddhason.org/SN/sn.php?keyword=56.13}
  「\twnr{比丘}{31.0}們!有這四聖諦,哪四個?苦聖諦、苦集聖諦、苦滅聖諦、導向苦\twnr{滅道跡}{69.0}聖諦。

  比丘們!而什麼是苦聖諦?『\twnr{五取蘊}{36.0}』應該被回答,即:色取蘊……(中略)識取蘊,比丘們!這被稱為苦聖諦。

  比丘們!而什麼是苦集聖諦?凡這個導致再有的、與歡喜及貪俱行的、\twnr{到處歡喜的}{96.2}渴愛,即:欲的渴愛、有的渴愛、\twnr{虛無的渴愛}{244.0},比丘們!這被稱為苦集聖諦。

  比丘們!而什麼是苦滅聖諦?凡正是那個渴愛的\twnr{無餘褪去與滅}{491.0}、捨棄、\twnr{斷念}{211.0}、解脫、無\twnr{阿賴耶}{391.0},比丘們!這被稱為苦滅聖諦。

  比丘們!而什麼是導向苦\twnr{滅道跡}{69.0}聖諦呢?就是這\twnr{八支聖道}{525.0},即:正見……(中略)正定,比丘們!這被稱為導向苦滅道跡聖諦。

  比丘們!這些是四聖諦。

  比丘們!因此,在這裡,『這是苦。』努力應該被作……(中略)『這是導向苦滅道跡。』努力應該被作。」



\sutta{14}{14}{內處經}{https://agama.buddhason.org/SN/sn.php?keyword=56.14}
  「\twnr{比丘}{31.0}們!有這四聖諦,哪四個?苦聖諦、苦集聖諦、苦滅聖諦、導向苦\twnr{滅道跡}{69.0}聖諦。

  比丘們!而什麼是苦聖諦?『六內處』應該被回答。即:眼處……(中略)意處,比丘們!這被稱為苦聖諦。

  比丘們!而什麼是苦集聖諦?凡這個導致再有的、與歡喜及貪俱行的、\twnr{到處歡喜的}{96.2}渴愛,即:欲的渴愛、有的渴愛、\twnr{虛無的渴愛}{244.0},比丘們!這被稱為苦集聖諦。

  比丘們!而什麼是苦滅聖諦?凡正是那個渴愛的\twnr{無餘褪去與滅}{491.0}、捨棄、\twnr{斷念}{211.0}、解脫、無\twnr{阿賴耶}{391.0},比丘們!這被稱為苦滅聖諦。

  比丘們!而什麼是導向苦\twnr{滅道跡}{69.0}聖諦呢?就是這\twnr{八支聖道}{525.0},即:正見……(中略)正定,比丘們!這被稱為導向苦滅道跡聖諦。

  比丘們!這些是四聖諦。

  比丘們!因此,在這裡,『這是苦。』努力應該被作……(中略)『這是導向苦滅道跡。』努力應該被作。」



\sutta{15}{15}{憶持經第一}{https://agama.buddhason.org/SN/sn.php?keyword=56.15}
  「\twnr{比丘}{31.0}們!你們能\twnr{憶持}{57.0}被我教導的四聖諦嗎?」

  在這麼說時,某位比丘對\twnr{世尊}{12.0}說這個:

  「\twnr{大德}{45.0}!我憶持被世尊教導的四聖諦。」

  「比丘!那麼,如怎樣你憶持被我教導的四聖諦呢?」

  「大德!我憶持:苦是被世尊教導的第一聖諦;大德!我憶持:苦集是被世尊教導的第二聖諦;大德!我憶持:苦滅是被世尊教導的第三聖諦;大德!我憶持:導向苦\twnr{滅道跡}{69.0}是被世尊教導的第四聖諦,大德!我這樣憶持被世尊教導的四聖諦。」

  「比丘!\twnr{好}{44.0}!好!比丘!好!你憶持被我教導的四聖諦:比丘!苦是被我教導的第一聖諦,請你像這樣憶持它;比丘!苦集是被我教導的第二聖諦,請你像這樣憶持它;比丘!苦滅是被我教導的第三聖諦,請你像這樣憶持它;比丘!導向苦滅道跡是被我教導的第四聖諦,請你像這樣憶持它,比丘!請你這樣憶持被我教導的四聖諦。

  比丘們!因此,在這裡,『這是苦。』努力應該被作……(中略)『這是導向苦滅道跡。』努力應該被作。」



\sutta{16}{16}{憶持經第二}{https://agama.buddhason.org/SN/sn.php?keyword=56.16}
  「\twnr{比丘}{31.0}們!你們能\twnr{憶持}{57.0}被我教導的四聖諦嗎?」

  在這麼說時,某位比丘對\twnr{世尊}{12.0}說這個:

  「\twnr{大德}{45.0}!我憶持被世尊教導的四聖諦。」

  「那麼,如怎樣你憶持被我教導的四聖諦呢?」

  「大德!我憶持:苦是被世尊教導的第一聖諦,大德!如果任何\twnr{沙門}{29.0}或\twnr{婆羅門}{17.0}這麼說:『這個苦不是被沙門\twnr{喬達摩}{80.0}教導的第一聖諦,我拒絕這個苦是第一聖諦後,將\twnr{安立}{143.0}另一個苦為第一聖諦。』\twnr{這不存在可能性}{650.0}。

  大德!我憶持:苦集是被世尊教導的第二聖諦……(中略)大德!我憶持:導向苦\twnr{滅道跡}{69.0}是被世尊教導的第四聖諦,大德!如果任何沙門或婆羅門這麼說:『這個導向苦滅道跡不是被沙門喬達摩教導的第四聖諦,我拒絕這個導向苦滅道跡是第四聖諦後,將安立另一個導向苦滅道跡為第四聖諦。』這不存在可能性。大德!我這樣憶持被世尊教導的四聖諦。」

  「比丘!\twnr{好}{44.0}!好!比丘!好!你憶持被我教導的四聖諦:比丘!苦是被我教導的第一聖諦,請你像這樣憶持它,比丘!如果任何沙門或婆羅門這麼說:『這個苦不是被沙門喬達摩教導的第一聖諦,我拒絕這個苦是第一聖諦後,將安立另一個苦為第一聖諦。』這不存在可能性。

  比丘!苦集……(中略)比丘!苦滅……(中略)比丘!導向苦滅道跡是被我教導的第四聖諦,請你像這樣憶持它,比丘!如果任何沙門或婆羅門這麼說:『這個導向苦滅道跡不是被沙門喬達摩教導的第四聖諦,我拒絕這個導向苦滅道跡是第四聖諦後,將安立另一個導向苦滅道跡為第四聖諦。』這不存在可能性。比丘!你要這樣憶持被我教導的四聖諦。

  比丘們!因此,在這裡,『這是苦。』努力應該被作……(中略)『這是導向苦滅道跡。』努力應該被作。」



\sutta{17}{17}{無明經}{https://agama.buddhason.org/SN/sn.php?keyword=56.17}
  在一旁坐下的那位\twnr{比丘}{31.0}對\twnr{世尊}{12.0}說這個:

  「\twnr{大德}{45.0}!被稱為『\twnr{無明}{207.0}、無明』,大德!什麼是無明?而什麼情形是\twnr{進入無明者}{645.0}?」

  「比丘!凡在苦上的無知、在苦集上的無知、在苦滅上的無知、在導向苦\twnr{滅道跡}{69.0}上的無知,比丘!這被稱為無明,而這個情形是進入無明者。

  比丘!因此,在這裡,『這是苦。』努力應該被作……(中略)『這是導向苦滅道跡。』努力應該被作。」



\sutta{18}{18}{明經}{https://agama.buddhason.org/SN/sn.php?keyword=56.18}
  那時,\twnr{某位比丘}{39.0}去見\twnr{世尊}{12.0}。抵達後,向世尊\twnr{問訊}{46.0}後,在一旁坐下。在一旁坐下的那位比丘對世尊說這個:

  「\twnr{大德}{45.0}!被稱為『\twnr{明}{207.0}、明』,大德!什麼是明?而什麼情形是進入明者?」

  「比丘!在苦上的智,在苦集上的智,在苦滅上的智,在導向苦\twnr{滅道跡}{69.0}上的智,比丘!這被稱為明,而這個情形是進入明者。

  比丘!因此,在這裡,『這是苦。』努力應該被作……(中略)『這是導向苦滅道跡。』努力應該被作。」



\sutta{19}{19}{說明經}{https://agama.buddhason.org/SN/sn.php?keyword=56.19}
  「\twnr{比丘}{31.0}們!『這是苦聖諦』被我告知,在那裡,有無量字、無量的辭、無量的說明:『像這樣,這都是苦聖諦。』

  『這是苦集聖諦』……(中略)『這是苦滅聖諦』……(中略)。

  比丘們!『這是導向苦\twnr{滅道跡}{69.0}聖諦』被我告知,在那裡,有無量字、無量的辭、無量的說明:『像這樣,這都是導向苦滅道跡聖諦。』

  比丘們!因此,在這裡,『這是苦。』努力應該被作……(中略)『這是導向苦滅道跡。』努力應該被作。」 



\sutta{20}{20}{真實經}{https://agama.buddhason.org/SN/sn.php?keyword=56.20}
  「\twnr{比丘}{31.0}們!這四者是真實的、無誤的、\twnr{無例外的}{855.2},哪四個?

  比丘們!『這是苦』:這是真實的,這是無誤的,這是無例外的;『這是苦集』:這是真實的,這是無誤的,這是無例外的;『這是苦滅』:這是真實的,這是無誤的,這是無例外的;『這是導向苦\twnr{滅道跡}{69.0}』:這是真實的,這是無誤的,這是無例外的。比丘們!這四者是真實的、無誤的、無例外的。

  比丘們!因此,在這裡,『這是苦。』努力應該被作……(中略)『這是導向苦滅道跡。』努力應該被作。」

  法輪轉起品第二,其\twnr{攝頌}{35.0}:

  「法輪、如來,蘊與入處,

   憶持與無明各二則,明、說明、真實。」





\pin{拘利村品}{21}{30}
\sutta{21}{21}{拘利村經第一}{https://agama.buddhason.org/SN/sn.php?keyword=56.21}
  \twnr{有一次}{2.0},\twnr{世尊}{12.0}住在跋耆族人的拘利村。

  在那裡,世尊召喚\twnr{比丘}{31.0}們:

  「比丘們!以四聖諦的不隨覺、\twnr{不通達}{355.0},這樣,這被我連同你們長時間地流轉、輪迴,哪四個?比丘們!以苦聖諦的不隨覺、不通達,這樣,這被我連同你們長時間地流轉、輪迴;以苦集聖諦……(中略)以苦滅聖諦……(中略)以導向苦\twnr{滅道跡}{69.0}聖諦的不隨覺、不通達,這樣,這被我連同你們長時間地流轉、輪迴。

  比丘們!那個這個苦聖諦已隨覺、已通達,苦集聖諦已隨覺、已通達,苦滅聖諦已隨覺、已通達,導向苦滅道跡聖諦已隨覺、已通達,\twnr{有的渴愛}{244.0}已切斷,\twnr{有之管道}{279.0}已滅盡,現在,沒有\twnr{再有}{21.0}。」

  世尊說這個,說這個後,\twnr{善逝}{8.0}、\twnr{大師}{145.0}又更進一步說這個:

  「以四聖諦的,不如實看見,

   被長時間輪迴:就在個個出生中。

   那些這些已看見,有之管道已根除,

   苦的根已切斷,現在沒有再有。」[\ccchref{DN.16}{https://agama.buddhason.org/DN/dm.php?keyword=16}, 155段]



\sutta{22}{22}{拘利村經第二}{https://agama.buddhason.org/SN/sn.php?keyword=56.22}
  「\twnr{比丘}{31.0}們!凡任何\twnr{沙門}{29.0}或\twnr{婆羅門}{17.0}不如實知道『這是苦』,不如實知道『這是苦\twnr{集}{67.0}』,不如實知道『這是苦\twnr{滅}{68.0}』,不如實知道『這是導向苦\twnr{滅道跡}{69.0}』,比丘們!那些沙門或婆羅門不被我認同為\twnr{沙門中的沙門}{560.0},或婆羅門中的婆羅門,而且,那些\twnr{尊者}{200.0}也不以證智自作證後,在當生中\twnr{進入後住於}{66.0}\twnr{沙門義}{327.0}或婆羅門義。

  比丘們!而凡任何沙門或婆羅門如實知道『這是苦』,如實知道『這是苦集』,如實知道『這是苦滅』,如實知道『這是導向苦滅道跡』,比丘們!那些沙門或婆羅門被我認同為沙門中的沙門,或婆羅門中的婆羅門,而且,那些尊者也以證智自作證後,在當生中進入後住於沙門義或婆羅門義。」

  \twnr{世尊}{12.0}說這個,說這個後,\twnr{善逝}{8.0}、\twnr{大師}{145.0}又更進一步說這個:

  「凡不知道苦,還有苦的生成,

   與一切苦,被破滅無餘之處,

   以及不知道,那個導向苦之寂止道跡。

   他們是\twnr{心解脫}{16.0}的欠缺者,還有在\twnr{慧解脫}{539.0}上,

   他們是作終結的不能夠者,他們確實是進入生-老者。

   凡知道苦,還有苦的生成,

   與一切苦,被破滅無餘之處,

   以及知道,那個導向苦之寂止道跡。

   他們是心解脫的具足者,還有在慧解脫上,

   他們是作終結的{一切}[能夠]者,他們是不進入生-老者。」[\ccchref{It.103}{https://agama.buddhason.org/It/dm.php?keyword=103}]



\sutta{23}{23}{遍正覺者經}{https://agama.buddhason.org/SN/sn.php?keyword=56.23}
  起源於舍衛城。

  「\twnr{比丘}{31.0}們!有這四聖諦,哪四個?苦聖諦……(中略)導向苦\twnr{滅道跡}{69.0}聖諦。比丘們!這些是四聖諦。

  比丘們!\twnr{如來}{4.0}以這些四聖諦的如實\twnr{現正覺}{75.0}的情況被稱為『\twnr{阿羅漢}{5.0}、\twnr{遍正覺者}{6.0}』。

  比丘們!因此,在這裡,『這是苦。』努力應該被作……(中略)『這是導向苦滅道跡。』努力應該被作。」



\sutta{24}{24}{阿羅漢經}{https://agama.buddhason.org/SN/sn.php?keyword=56.24}
  起源於舍衛城。

  「\twnr{比丘}{31.0}們!凡過去世任何\twnr{阿羅漢}{5.0}、\twnr{遍正覺者}{6.0}曾如實\twnr{現正覺}{75.0},他們全部曾如實現正覺四聖諦。

  比丘們!凡\twnr{未來世}{308.0}任何阿羅漢、遍正覺者將如實現正覺,他們全部將如實現正覺四聖諦。

  比丘們!凡現在任何阿羅漢、遍正覺者如實現正覺,他們全部如實現正覺四聖諦。

  哪四個?苦聖諦、苦集聖諦、苦滅聖諦、導向苦\twnr{滅道跡}{69.0}聖諦。

  比丘們!凡過去世任何阿羅漢、遍正覺者曾如實現正覺……(中略)將現正覺……(中略)現正覺,他們全部如實現正覺這些四聖諦。

  比丘們!因此,在這裡,『這是苦。』努力應該被作……(中略)『這是導向苦滅道跡。』努力應該被作。」



\sutta{25}{25}{漏的滅盡經}{https://agama.buddhason.org/SN/sn.php?keyword=56.25}
  「\twnr{比丘}{31.0}們!我說知者、見者有諸\twnr{漏}{188.0}的滅盡,非不知者、不見者。比丘們!而知、見什麼者有諸漏的滅盡?比丘們!知、見『這是苦』者有諸漏的滅盡,知、見『這是苦集』者有諸漏的滅盡,知、見『這是苦滅』者有諸漏的滅盡,知、見『這是導向苦\twnr{滅道跡}{69.0}』者有諸漏的滅盡。比丘們!這樣知者、這樣見者有諸漏的滅盡。

  比丘們!因此,在這裡,『這是苦。』努力應該被作……(中略)『這是導向苦滅道跡。』努力應該被作。」



\sutta{26}{26}{朋友經}{https://agama.buddhason.org/SN/sn.php?keyword=56.26}
  「\twnr{比丘}{31.0}們!凡任何你們會憐愍,以及凡他們會想應該被聽聞的朋友,或同事,或親族,或有血緣者,比丘們!他們應該被你為了四聖諦的如實\twnr{現觀}{53.0}們勸導、應該被使確立、應該被使建立,哪四個?苦聖諦、苦集聖諦、苦滅聖諦、導向苦\twnr{滅道跡}{69.0}聖諦。比丘們!凡任何你們會憐愍,以及凡他們會想應該被聽聞的朋友,或同事,或親族,或有血緣者,比丘們!他們應該被你們為了四聖諦的如實現觀勸導、應該被使確立、應該被使建立。

  比丘們!因此,在這裡,『這是苦。』努力應該被作……(中略)『這是導向苦滅道跡。』努力應該被作。」



\sutta{27}{27}{真實經}{https://agama.buddhason.org/SN/sn.php?keyword=56.27}
  「\twnr{比丘}{31.0}們!有這四聖諦,哪四個?苦聖諦、苦集聖諦、苦滅聖諦、導向苦\twnr{滅道跡}{69.0}聖諦。

  比丘們!這四聖諦是真實的、無誤的、\twnr{無例外的}{855.2},因此被稱為『聖諦』。

  比丘們!因此,在這裡,『這是苦。』努力應該被作……(中略)『這是導向苦滅道跡。』努力應該被作。」



\sutta{28}{28}{世間經}{https://agama.buddhason.org/SN/sn.php?keyword=56.28}
  「\twnr{比丘}{31.0}們!有這四聖諦,哪四個?苦聖諦、苦集聖諦、苦滅聖諦、導向苦\twnr{滅道跡}{69.0}聖諦。比丘們!在包括天,在包括魔,在包括梵的世間;在包括沙門婆羅門,在包括天-人的\twnr{世代}{38.0}中,如來是聖者,因此被稱為『聖諦』。

  比丘們!因此,在這裡,『這是苦。』努力應該被作……(中略)『這是導向苦滅道跡。』努力應該被作。」



\sutta{29}{29}{應該被遍知經}{https://agama.buddhason.org/SN/sn.php?keyword=56.29}
  「\twnr{比丘}{31.0}們!有這四聖諦,哪四個?苦聖諦、苦集聖諦、苦滅聖諦、導向苦\twnr{滅道跡}{69.0}聖諦。比丘們!這些是四聖諦。

  比丘們!這四聖諦有應該被\twnr{遍知}{154.0}的聖諦,有應該被捨斷的聖諦,有應該被作證的聖諦,有應該被修習的聖諦。

  比丘們!而什麼是應該被遍知的聖諦?苦聖諦應該被遍知,苦集聖諦應該被捨斷,苦滅聖諦應該被作證,導向苦滅道跡聖諦應該被\twnr{修習}{94.0}。

  比丘們!因此,在這裡,『這是苦。』努力應該被作……(中略)『這是導向苦滅道跡。』努力應該被作。」



\sutta{30}{30}{牛主經}{https://agama.buddhason.org/SN/sn.php?keyword=56.30}
  \twnr{有一次}{2.0},眾多\twnr{上座}{135.0}\twnr{比丘}{31.0}住在支提的瑟訶者尼葛。

  當時,當眾多上座比丘\twnr{餐後已從施食返回}{512.0},在圓亭棚集會共坐時,這個談論中出現:

  「\twnr{學友}{201.0}們!凡看見苦者,他也看見苦集,也看見苦滅,也看見導向苦\twnr{滅道跡}{69.0}者嗎?」

  在這麼說時,\twnr{尊者}{200.0}牛主對上座比丘們說這個:

  「學友們!這被我從世尊的面前聽聞,從面前領受:『凡看見苦者,他也看見苦集,也看見苦滅,也看見導向苦滅道跡;凡看見苦集者,他也看見苦,也看見苦滅,也看見導向苦滅道跡;凡看見苦滅者,他也看見苦,也看見苦集,也看見導向苦滅道跡;凡看見導向苦滅道跡者,他也看見苦,也看見苦集,也看見苦滅。』」[\suttaref{SN.56.44}]

  拘利村品第三,其\twnr{攝頌}{35.0}:

  「二則跋耆族人、遍正覺者,阿羅漢、漏的滅盡,

   朋友、真實與世間,應該被遍知、牛主。」





\pin{申恕林品}{31}{40}
\sutta{31}{31}{申恕林經}{https://agama.buddhason.org/SN/sn.php?keyword=56.31}
  \twnr{有一次}{2.0},\twnr{世尊}{12.0}住在\twnr{憍賞彌}{140.0}申恕林中。

  那時,世尊以手取微少的申恕樹葉後,召喚比丘們:

  「比丘們!你們怎麼想它,哪個是比較多的呢:凡被我以手取微少的申恕樹葉,或凡這在申恕林上面的?」

  「\twnr{大德}{45.0}!被世尊以手取微少的申恕樹葉是少量的,而這正是比較多的,即:在申恕林上面的。」

  「同樣的,比丘們!這正是比較多的:凡證知後沒被我告知你們的。比丘們!而為何這不被我告知呢?比丘們!因為這是不\twnr{伴隨利益的}{50.0},非\twnr{梵行基礎的}{446.0},不對\twnr{厭}{15.0}、不對\twnr{離貪}{77.0}、不對\twnr{滅}{68.0}、不對寂靜、不對證智、不對\twnr{正覺}{185.1}、不對涅槃轉起,因此,它不被我告知。

  比丘們!而什麼被我告知呢?比丘們!『這是苦』被我告知,『這是苦集』被我告知,『這是苦滅』被我告知,『這是導向苦\twnr{滅道跡}{69.0}』被我告知。

  比丘們!而為何這被我告知呢?比丘們!因為這是伴隨利益的,這是梵行基礎的,這對厭、對離貪、對滅、對寂靜、對證智、對正覺、對涅槃轉起,因此它被我告知。

  比丘們!因此,在這裡,『這是苦。』努力應該被作……(中略)『這是導向苦滅道跡。』努力應該被作。」



\sutta{32}{32}{金合歡樹樹葉經}{https://agama.buddhason.org/SN/sn.php?keyword=56.32}
  「\twnr{比丘}{31.0}們!凡如果這麼說:『我未如實\twnr{現觀}{53.0}苦聖諦後,未如實現觀苦集聖諦後,未如實現觀苦滅聖諦後,未如實現觀導向苦\twnr{滅道跡}{69.0}聖諦後,我將得到苦的完全結束。』\twnr{這不存在可能性}{650.0}。[\suttaref{SN.56.44}]

  比丘們!猶如凡如果這麼說:『我作金合歡樹樹葉或長葉松葉或餘甘子葉的容器後,將運來水或棕櫚{葉}[果(錫蘭本)]。』這不存在可能性。同樣的,比丘們!凡如果這麼說:『我未如實現觀苦聖諦後……(中略)未如實現觀導向苦滅道跡聖諦後,我將得到苦的完全結束。』這不存在可能性。

  比丘們!而凡如果這麼說:『我如實現觀苦聖諦後,如實現觀苦集聖諦後,如實現觀苦滅聖諦後,如實現觀導向苦滅道跡聖諦後,我將得到苦的完全結束。』這存在可能性。

  比丘們!猶如凡如果這麼說:『我作蓮葉或蘇芳樹葉或闊葉藤葉的容器後,將運來水或棕櫚果。』這存在可能性。同樣的,比丘們!凡如果這麼說:『我如實現觀苦聖諦後……(中略)如實現觀導向苦滅道跡聖諦後,我將得到苦的完全結束。』這存在可能性。

  比丘們!因此,在這裡,『這是苦。』努力應該被作,『這是苦集。』努力應該被作,『這是苦滅。』努力應該被作,『這是導向苦滅道跡。』努力應該被作。」



\sutta{33}{33}{棍子經}{https://agama.buddhason.org/SN/sn.php?keyword=56.33}
  「\twnr{比丘}{31.0}們!猶如棍子被向上投擲到空中,有時以底部落下,有時以頂端落下。同樣的,比丘們!\twnr{無明蓋}{158.0}、渴愛結、流轉的、輪迴的眾生有時從這個世界到其它世界,有時從其它世界到這個世界[\suttaref{SN.15.9}],那什麼原因呢?以四聖諦的未看見狀態,哪四個?苦聖諦……(中略)導向苦\twnr{滅道跡}{69.0}聖諦。

  比丘們!因此,在這裡,『這是苦。』努力應該被作……(中略)『這是導向苦滅道跡。』努力應該被作。」



\sutta{34}{34}{衣服經}{https://agama.buddhason.org/SN/sn.php?keyword=56.34}
  「\twnr{比丘}{31.0}們!在衣服或頭被燃燒時,什麼會是應該被做的?」

  「\twnr{大德}{45.0}!在衣服或頭被燃燒時,就為了使他的衣服或頭的熄滅,\twnr{極度的}{510.0}意欲、精進、勇猛、努力、不畏縮、念、正知應該被做。」

  「比丘們!對被燃燒的衣服或頭旁觀後不作意後,為了未\twnr{現觀}{53.0}四聖諦的如實現觀,極度的意欲、精進、勇猛、努力、不畏縮、念、正知應該被做。哪四個?苦聖諦、苦集聖諦、苦滅聖諦、導向苦\twnr{滅道跡}{69.0}聖諦。

  比丘們!因此,在這裡,『這是苦。』努力應該被作……(中略)『這是導向苦滅道跡。』努力應該被作。」



\sutta{35}{35}{百槍經}{https://agama.buddhason.org/SN/sn.php?keyword=56.35}
  「比丘們!猶如男子有百年壽命、百年生命,[有人]對他這麼說:『喂!男子!來!他們午前時將以百槍擊你,中午時將以百槍擊你,傍晚時將以百槍擊你。喂!男子!那個百年壽命、百年生命,天天被三百槍擊的你,經過百年後,你將現觀未\twnr{現觀}{53.0}的四聖諦。』

  比丘們!以通曉道理的\twnr{善男子}{41.0},足以承擔,那是什麼原因?比丘們!這輪迴是無始的,槍的擊打、劍的擊打、箭的擊打、斧的擊打之\twnr{起始點}{639.0}不被知道。

  比丘們!如果這是這樣,然而我不說四聖諦的現觀與苦俱,與憂俱,比丘們!而是我說四聖諦的現觀只與樂俱,只與喜悅俱。哪四個?苦聖諦……(中略)導向苦\twnr{滅道跡}{69.0}聖諦。

  比丘們!因此,在這裡,『這是苦。』努力應該被作……(中略)『這是導向苦滅道跡。』努力應該被作。」



\sutta{36}{36}{生類經}{https://agama.buddhason.org/SN/sn.php?keyword=56.36}
  「\twnr{比丘}{31.0}們!猶如男子切凡在\twnr{閻浮洲中}{x659}的草、木、枝、葉後聚集在一起,聚集在一起後製作矛,製作矛後,凡大海中大的生類在大的矛上刺穿牠們,凡大海中中等的生類在中等的矛上刺穿牠們,凡大海中微小的生類在微小的矛上刺穿牠們,比丘們!但大海中粗大的生類未被遍取,那時,在這閻浮洲中的草、木、枝、葉走到遍盡、遍取(耗盡),比丘們!到這裡,大海中有更多不容易在矛上刺穿的微小生類,那是什麼原因?比丘們!個體的微小狀態。同樣的,比丘們!\twnr{苦界}{109.0}是大的,比丘們!從這麼大的苦界被解脫的、\twnr{見具足的}{575.0}個人如實知道『這是苦』……(中略)如實知道『這是導向苦\twnr{滅道跡}{69.0}』。

  比丘們!因此,在這裡,『這是苦。』努力應該被作……(中略)『這是導向苦滅道跡。』努力應該被作。」



\sutta{37}{37}{太陽經第一}{https://agama.buddhason.org/SN/sn.php?keyword=56.37}
  「\twnr{比丘}{31.0}們!對太陽的上升來說,\twnr{這是先導,這是前相}{x660},即:\twnr{黎明}{x661}[\ccchref{AN.10.121}{https://agama.buddhason.org/AN/an.php?keyword=10.121}]。同樣的,比丘們!對比丘的如實\twnr{現觀}{53.0}四聖諦,這是先導,這是前相,即:正見。

  比丘們!這位比丘的這個能被預期:將如實知道『這是苦』……(中略)將如實知道『這是導向苦\twnr{滅道跡}{69.0}』。

  比丘們!因此,在這裡,『這是苦。』努力應該被作……(中略)『這是導向苦滅道跡。』努力應該被作。」



\sutta{38}{38}{太陽經第二}{https://agama.buddhason.org/SN/sn.php?keyword=56.38}
  「\twnr{比丘}{31.0}們!只要太陽、月亮不在世間出現,就仍然沒有大光明、大光亮的出現,那時是黑暗、黑暗闇黑,日夜仍然還不被知道,月、半月不被知道,季節、年不被知道。

  比丘們!但自從太陽、月亮在世間出現,那時有大光明、大光亮的出現,那時就沒有黑暗、黑暗闇黑,那時日夜被知道,月、半月被知道,季節、年被知道。同樣的,比丘們!只要\twnr{如來}{4.0}、\twnr{阿羅漢}{5.0}、\twnr{遍正覺者}{6.0}不在世間出現,就仍然沒有大光明、大光亮的出現,那時是黑暗、黑暗闇黑,仍然還沒有四聖諦的告知、教導、\twnr{使知}{143.0}、建立、開顯、解析、闡明。比丘們!但自從如來、阿羅漢、遍正覺者在世間出現,那時有大光明、大光亮的出現,那時就沒有黑暗、黑暗闇黑,那時有四聖諦的告知、教導、使知、建立、開顯、解析、闡明。

  哪四個?苦聖諦……(中略)導向苦\twnr{滅道跡}{69.0}聖諦。

  比丘們!因此,在這裡,『這是苦。』努力應該被作……(中略)『這是導向苦滅道跡。』努力應該被作。」



\sutta{39}{39}{因陀羅柱經}{https://agama.buddhason.org/SN/sn.php?keyword=56.39}
  「\twnr{比丘}{31.0}們!凡任何\twnr{沙門}{29.0}或\twnr{婆羅門}{17.0}不如實知道『這是苦』……(中略)不如實知道『這是導向苦\twnr{滅道跡}{69.0}』者,他們\twnr{仰視}{x662}其他沙門或婆羅門的臉:『這位\twnr{尊師}{203.0}確實是知道者,他知道,是看見者,他\twnr{看見}{660.0}。』

  比丘們!猶如輕的、隨風飄的木棉花絨或棉花絨被掉落在平整的地方,隨即,東風帶往西;西風帶往東;北風帶往南;南風帶往北,那是什麼原因?比丘們!以棉花絨輕的狀態。同樣的,比丘們!凡任何沙門或婆羅門不如實知道『這是苦』……(中略)不如實知道『這是導向苦滅道跡』者,他們仰視其他沙門或婆羅門的臉:『這位尊師確實是知道者,他知道,是看見者,他看見。』那是什麼原因?比丘們!以四聖諦未看見的狀態。

  比丘們!凡任何沙門或婆羅門如實知道『這是苦』……(中略)如實知道『這是導向苦滅道跡』者,他們不仰視其他沙門或婆羅門的臉:『這位尊師確實是知道者,他知道,是看見者,他看見。』

  比丘們!猶如深基礎的、善埋的、不動的、不大震動的鐵柱或\twnr{因陀羅柱}{631.0},即使暴風雨從東方到來,既不震動,也不大震動,也不大搖動;即使從西方……(中略)即使從北方……(中略)即使從南方來了暴風雨,既不震動,也不大震動,也不大搖動,那是什麼原因?比丘們!因陀羅柱基礎深的狀態、善埋的狀態。同樣的,比丘們!凡任何沙門或婆羅門如實知道『這是苦』……(中略)如實知道『這是導向苦滅道跡』者,他們不仰視其他沙門或婆羅門的臉:『這位尊師確實是知道者,他知道,是看見者,他看見。』那是什麼原因?比丘們!以四聖諦善看見的狀態。

  哪四個?苦聖諦……(中略)導向苦滅道跡聖諦。

  比丘們!因此,在這裡,『這是苦。』努力應該被作……(中略)『這是導向苦滅道跡。』努力應該被作。」



\sutta{40}{40}{希求辯論經}{https://agama.buddhason.org/SN/sn.php?keyword=56.40}
  「\twnr{比丘}{31.0}們!凡任何比丘如實知道『這是苦』……(中略)如實知道『這是導向苦\twnr{滅道跡}{69.0}』者,即使希求辯論、\twnr{尋求辯論}{x663}的\twnr{沙門}{29.0}或\twnr{婆羅門}{17.0}從東方到來:『我將反駁(使登上)他的理論。』『他將以如法使之震動,或使之大震動,或使之大搖動。』\twnr{這不存在可能性}{650.0}。即使從西方……(中略)即使從北方……(中略)即使希求辯論、尋求辯論的沙門或婆羅門從南方到來:『我將反駁他的理論。』『他將以如法使之震動,或使之大震動,或使之大搖動。』這不存在可能性。

  比丘們!猶如十六肘的石柱,它的八肘是在下面的基礎部分,八肘是在基礎上面,即使暴風雨從東方到來,既不震動,也不大震動,也不大搖動;即使從西方……(中略)即使從北方……(中略)即使從南方來了暴風雨,既不震動,也不大震動,也不大搖動,那是什麼原因?比丘們!石柱基礎深的狀態、善埋的狀態。同樣的,比丘們!凡任何比丘如實知道『這是苦』……(中略)如實知道『這是導向苦滅道跡』者,即使希求辯論、尋求辯論的沙門或婆羅門從東方到來:『我將反駁他的理論。』『他將以如法使之震動,或使之大震動,或使之大搖動。』這不存在可能性。即使從西方……(中略)即使從北方……(中略)即使希求辯論、尋求辯論的沙門或婆羅門從南方到來:『我將反駁他的理論。』『他將以如法使之震動,或使之大震動,或使之大搖動。』這不存在可能性。那是什麼原因?比丘們!以四聖諦善看見的狀態。

  哪四個?苦聖諦……(中略)導向苦滅道跡聖諦。

  比丘們!因此,在這裡,『這是苦。』努力應該被作……(中略)『這是導向苦滅道跡。』努力應該被作。」

  申恕林品第四,其\twnr{攝頌}{35.0}:

  「申恕,金合歡樹,棍子,衣服,與以百槍,

   生類,太陽譬喻二種,因陀羅柱與爭論者。」





\pin{斷崖品}{41}{50}
\sutta{41}{41}{世間的思惟經}{https://agama.buddhason.org/SN/sn.php?keyword=56.41}
  \twnr{有一次}{2.0},\twnr{世尊}{12.0}住在王舍城栗鼠飼養處的竹林中。

  在那裡,世尊召喚\twnr{比丘}{31.0}們:

  「比丘們!從前,某位男子從舍衛城出去後:『\twnr{我將思惟世間的思惟}{x664}。』去須摩揭陀蓮花池。抵達後,在須摩揭陀蓮花池畔坐下,思惟著世間的思惟。

  比丘們!那位男子在須摩揭陀蓮花池畔看見正進入蓮莖的四種軍。看見了後,他這麼想:『我確實是發瘋者,我確實是狂亂者,凡世間中不存在者,那個被我看見。』

  比丘們!那時,那位男子進城後,告訴大群人:『大德!我確實是發瘋者,我確實是狂亂者,凡世間中不存在者,那個被我看見。』

  『喂!男子!那麼,你是怎樣發瘋者,是怎樣狂亂者?而世間中什麼不存在,被你看見呢?』

  『大德!這裡,我從舍衛城出去後:『我將思惟世間的思惟。』去須摩揭陀蓮花池。抵達後,在須摩揭陀蓮花池畔坐下,思惟著世間的思惟。大德!我在須摩揭陀蓮花池畔看見正進入蓮莖的四種軍。大德!我是這樣發瘋者,這樣狂亂者,而世間中這個不存在,被我看見。』

  『喂!男子!你確實是發瘋者,確實是狂亂者,而這個世間中不存在,那個被你看見。』

  比丘們!然而,那位男子就看見那個真實的,非不真實的。

  比丘們!從前,天神、阿修羅的戰鬥已群集。比丘們!又,在那場戰鬥中,天神們勝,阿修羅們敗。比丘們!敗北、驚恐的阿修羅們為了迷惑天神經蓮莖進入阿修羅城。

  比丘們!因此,在這裡,你們不要思惟世間的思惟:『世界是常恆的』,或『\twnr{世界是非常恆的}{170.0}』,或『世界是有邊的』,或『世界是無邊的』,或『命即是身體』,或『\twnr{命是一身體是另一}{169.0}』,或『死後\twnr{如來}{4.0}存在』,或『死後如來不存在』,或『\twnr{死後如來存在且不存在}{354.0}』,或『死後如來既非存在也非不存在』,那什麼原因呢?比丘們!這個思惟是不\twnr{伴隨利益的}{50.0},非\twnr{梵行基礎的}{446.0},不對\twnr{厭}{15.0}、不對\twnr{離貪}{77.0}、不對\twnr{滅}{68.0}、不對寂靜、不對證智、不對\twnr{正覺}{185.1}、不對涅槃轉起。

  比丘們!當思惟時,你們應該思惟『這是苦。』……(中略)你們應該思惟『這是導向苦\twnr{滅道跡}{69.0}』,那什麼原因呢?比丘們!這個思惟是伴隨利益的,這些是梵行基礎的,這些對厭、對離貪、對滅、對寂靜、對證智、對正覺、對涅槃轉起。

  比丘們!因此,在這裡,『這是苦。』努力應該被作……(中略)『這是導向苦滅道跡。』努力應該被作。」



\sutta{42}{42}{斷崖經}{https://agama.buddhason.org/SN/sn.php?keyword=56.42}
  \twnr{有一次}{2.0},\twnr{世尊}{12.0}住在王舍城\twnr{耆闍崛山}{258.0}。

  那時,世尊召喚\twnr{比丘}{31.0}們:

  「比丘們!我們走,\twnr{為了白天的住處}{128.0}我們將去巴低邦那山頂。」

  「是的,\twnr{大德}{45.0}!」那些比丘回答世尊。

  那時,世尊與眾多比丘一起去巴低邦那山頂。

  某位比丘看見在巴低邦那山頂的大斷崖,見了後,對世尊說這個:

  「大德!這個斷崖實在是大的;大德!斷崖是極恐怖的,大德!有其它斷崖比這個斷崖更大與更恐怖的嗎?」

  「比丘!有其它斷崖比這個斷崖更大與更恐怖的。」

  「大德!那麼,哪個其它斷崖比這個斷崖更大與更恐怖的呢?」

  「比丘們!凡任何\twnr{沙門}{29.0}或\twnr{婆羅門}{17.0}不如實知道『這是苦』;不如實知道『這是苦集』;不如實知道『這是苦滅』;不如實知道『這是導向苦\twnr{滅道跡}{69.0}』,他們\twnr{在導致(轉起)出生的諸行上尋歡}{x665};在導致老的諸行上尋歡;在導致死的諸行上尋歡;在導致愁、悲、苦、憂、\twnr{絕望}{342.0}的諸行上尋歡,他們已在導致出生的諸行上尋歡;已在導致老的諸行上尋歡;已在導致死的諸行上尋歡;已在導致愁、悲、苦、憂、絕望的諸行上尋歡,他們\twnr{造作導致出生的諸行}{x666};造作導致老的諸行;造作導致死的諸行;造作導致愁、悲、苦、憂、絕望的諸行,他們造作導致出生的諸行後;造作導致老的諸行後;造作導致死的諸行後;造作導致愁、悲、苦、憂、絕望的諸行後,跌落出生的斷崖,也跌落老的斷崖,也跌落死的斷崖,也跌落愁、悲、苦、憂、絕望的斷崖,他們不從出生、老、死、愁、悲、苦、憂、絕望被釋放,我說:『不從苦被釋放。』

  比丘們!但凡任何沙門或婆羅門如實知道『這是苦』……(中略)如實知道『這是導向苦滅道跡』,他們在導致出生的諸行上不尋歡;在導致老的諸行上不尋歡;在導致死的諸行上不尋歡;在導致愁、悲、苦、憂、絕望的諸行上不尋歡,已在導致出生的諸行上不尋歡;已在導致老的諸行上不尋歡;已在導致死的諸行上不尋歡;已在導致愁、悲、苦、憂、絕望的諸行上不尋歡,他們\twnr{不造作}{751.0}導致出生的諸行;不造作導致老的諸行;不造作導致死的諸行;不造作導致愁、悲、苦、憂、絕望的諸行,他們不造作導致出生的諸行後;不造作導致老的諸行後;不造作導致死的諸行後;不造作導致愁、悲、苦、憂、絕望的諸行後,不跌落出生的斷崖,也不跌落老的斷崖,也不跌落死的斷崖,也不跌落愁、悲、苦、憂、絕望的斷崖,他們從出生、老、死、愁、悲、苦、憂、絕望被釋放,我說:『從苦被釋放。』

  比丘們!因此,在這裡,『這是苦。』努力應該被作……(中略)『這是導向苦滅道跡。』努力應該被作。」



\sutta{43}{43}{大熱惱經}{https://agama.buddhason.org/SN/sn.php?keyword=56.43}
  「\twnr{比丘}{31.0}們!有名叫大熱惱的地獄,在那裡,凡以眼看見任何色,只看見非要想的色、不想要的色;只看見非可愛的色、不可愛的色;只看見非合意的色、不合意的色。

  凡以耳聽到任何聲音……(中略)凡以鼻嗅到任何氣味……(中略)凡以舌嚐到任何味道……(中略)凡以身觸到任何\twnr{所觸}{220.2}……(中略)凡以意識知任何法,只識知非想要的法、不想要的法;只識知非可愛的法、不可愛的法;只識知非合意的法、不合意的法。」

  在這麼說時,某位比丘對\twnr{世尊}{12.0}說這個:

  「\twnr{大德}{45.0}!那確實是大熱惱;大德!那確實是極大熱惱。大德!有其它熱惱比這個大熱惱更大同時也更恐怖的嗎?」

  「比丘!有其它熱惱比這個大熱惱更大同時也更恐怖的。」

  「大德!那麼,哪個其它熱惱比這個大熱惱更大同時也更恐怖的呢?」

  「比丘們!凡任何\twnr{沙門}{29.0}或\twnr{婆羅門}{17.0}不如實知道『這是苦』……(中略)不如實知道『這是導向苦\twnr{滅道跡}{69.0}』,他們在導致(轉起)出生的諸行上尋歡[\suttaref{SN.56.42}]……(中略)已尋歡……(中略)造作……(中略)造作後,被出生的熱惱遍燒,也被老的熱惱遍燒,也被死的熱惱遍燒,也被愁、悲、苦、憂、\twnr{絕望}{342.0}的熱惱燒,他們不從出生、老、死、愁、悲、苦、憂、絕望被釋放,我說:『不從苦被釋放。』

  比丘們!但凡任何沙門或婆羅門如實知道『這是苦』……(中略)如實知道『這是導向苦滅道跡』,他們在導致出生的諸行上不尋歡……(中略)已不尋歡……(中略)\twnr{不造作}{751.0}……(中略)不造作後,不被出生的熱惱遍燒,也不被老的熱惱遍燒,也不被死的熱惱遍燒,也不被愁、悲、苦、憂、絕望的熱惱燒,他們從出生、老、死、愁、悲、苦、憂、絕望被釋放,我說:『從苦被釋放。』

  比丘們!因此,在這裡,『這是苦。』努力應該被作……(中略)『這是導向苦滅道跡。』努力應該被作。」



\sutta{44}{44}{重閣經}{https://agama.buddhason.org/SN/sn.php?keyword=56.44}
  「\twnr{比丘}{31.0}們!凡如果這麼說:『未如實\twnr{現觀}{53.0}苦聖諦後……(中略)未如實現觀導向苦\twnr{滅道跡}{69.0}聖諦後,我將得到苦的完全結束。』\twnr{這不存在可能性}{650.0}。[\suttaref{SN.56.32}]

  比丘們!猶如凡如果這麼說:『不作\twnr{重閣}{213.0}的最下層屋後,我將使最上層屋登上。』這不存在可能性。同樣的,比丘們!凡如果這麼說:『未如實現觀苦聖諦後……(中略)未如實現觀導向苦滅道跡聖諦後,我將得到苦的完全結束。』這不存在可能性。

  比丘們!但凡如果這麼說:『如實現觀苦聖諦』後……(中略)如實現觀導向苦滅道跡聖諦後,我將得到苦的完全結束。』這存在可能性。

  比丘們!猶如凡如果這麼說:『作重閣的最下層屋後,我將使最上層屋登上。』這存在可能性。同樣的,比丘們!凡如果有人說這個:『如實現觀苦聖諦後……(中略)如實現觀導向苦滅道跡聖諦後,我將得到苦的完全結束。』這存在可能性。

  比丘們!因此,在這裡,『這是苦。』努力應該被作……(中略)『這是導向苦滅道跡。』努力應該被作。」[→\suttaref{SN.56.30}]



\sutta{45}{45}{毛經}{https://agama.buddhason.org/SN/sn.php?keyword=56.45}
  \twnr{有一次}{2.0},世尊住在毘舍離大林重閣講堂。

  那時,\twnr{尊者}{200.0}阿難午前時穿衣、拿起衣鉢後,\twnr{為了托鉢}{87.0}進入毘舍離。

  尊者阿難看見眾多正在\twnr{集會所}{761.0}操練(作)弓術:就從遠處經微小的鑰匙孔無失誤地\twnr{一箭接著一箭}{x667}射著箭的離車族少年,看見後想這個:「這些離車族少年確實已學習,這些離車族少年確實已善學習,確實是因為就從遠處經微小的鑰匙孔將無失誤地一箭接著一箭射箭。」

  那時,尊者阿難在毘舍離為了托鉢行走後,\twnr{餐後已從施食返回}{512.0},去見世尊。抵達後,向世尊\twnr{問訊}{46.0}後,在一旁坐下。在一旁坐下的尊者阿難對世尊說這個:

  「\twnr{大德}{45.0}!這裡,我午前時穿衣、拿起衣鉢後,為了托鉢進入毘舍離,我看見眾多正在集會所操練弓術:就從遠處經微小的鑰匙孔無失誤地一箭接著一箭射穿箭的離車族少年,看見後我想這個: 『這些離車族少年確實已學習,這些離車族少年確實已善學習,確實是因為就從遠處經微小的鑰匙孔將無失誤地一箭接著一箭射箭。』」

  「阿難!你怎麼想,哪個是更難作的,或更難達到的:就從遠處經微小的鑰匙孔能無失誤地一箭接著一箭射箭,或\twnr{能經分裂成七分的毛之一端貫通另一端}{x668}?」

  「大德!這正是更難作的,同時也更難達到的:能經分裂成七分的毛之一端貫穿另一端。」

  「阿難!但他們貫通更難貫通的:如實貫通(洞察)『這是苦』……如實貫通『這是導向苦\twnr{滅道跡}{69.0}』者。

  \twnr{比丘}{31.0}們!因此,『這是苦。』努力應該被作……(中略)『這是導向苦滅道跡。』努力應該被作。」



\sutta{46}{46}{黑暗經}{https://agama.buddhason.org/SN/sn.php?keyword=56.46}
  「\twnr{比丘}{31.0}們!有\twnr{世界中間空無防護的}{890.0}黑暗、黑暗闇黑,以這些這麼\twnr{大神通力}{405.0}、這麼大威力日月的光明不經歷之處。」

  在這麼說時,某位比丘對\twnr{世尊}{12.0}說這個:

  「\twnr{大德}{45.0}!那確實是大黑暗,大德!那確實是極大黑暗。大德!有其它黑暗比這個大黑暗更大與更恐怖的嗎?」

  「比丘!有其它黑暗比這個大黑暗更大與更恐怖的。」

  「大德!那麼,哪個其它黑暗比這個大黑暗更大與更恐怖的呢?」

  「比丘!凡任何\twnr{沙門}{29.0}或\twnr{婆羅門}{17.0}不如實知道『這是苦』……(中略)不如實知道『這是導向苦\twnr{滅道跡}{69.0}』,他們在導致(轉起)出生的諸行上尋歡[\suttaref{SN.56.42}]……(中略)已尋歡……(中略)造作……(中略)造作後,跌落出生的黑暗;跌落老的黑暗;跌落死的黑暗;跌落愁、悲、苦、憂、\twnr{絕望}{342.0}的黑暗,他們不從出生、老、死、愁、悲、苦、憂、絕望被釋放,我說:『不從苦被釋放。』

  比丘們!但凡任何沙門或婆羅門如實知道『這是苦』……(中略)如實知道『這是導向苦滅道跡』,他們在導致出生的諸行上不尋歡……(中略)已不尋歡……(中略)\twnr{不造作}{751.0}……(中略)不造作後,不跌落出生的黑暗;不跌落老的黑暗;不跌落死的黑暗;不跌落愁、悲、苦、憂、絕望的黑暗,他們從出生、老、死、愁、悲、苦、憂、絕望被釋放,我說:『從苦被釋放。』

  比丘們!因此,在這裡,『這是苦。』努力應該被作……(中略)『這是導向苦滅道跡。』努力應該被作。」



\sutta{47}{47}{有孔之軛經第一}{https://agama.buddhason.org/SN/sn.php?keyword=56.47}
  「\twnr{比丘}{31.0}們!猶如男子在大海投入單孔軛,在那裡也有隻盲龜,牠每經過一百年浮出一次。比丘們!你們怎麼想它:是否經過一百年浮出一次的盲龜,能使頸部進入那個單孔軛中呢?」

  「\twnr{大德}{45.0}!如果偶爾經過長時間。」

  「比丘們!那隻每一百年浮出一次的盲龜使頭伸入單孔軛中比較快,比丘們!然而,我不說,比愚者落入\twnr{下界}{111.0}一次[再]成為人的狀態,那是什麼原因?比丘們!因為在這裡沒有法行、正行,善的行為、福德行為,比丘們!在這裡輾轉吞食的、弱者吞食的轉起,那是什麼原因?比丘們!以四聖諦未看見的狀態,哪四個?苦聖諦……(中略)導向苦\twnr{滅道跡}{69.0}聖諦。

  比丘們!因此,在這裡,『這是苦。』努力應該被作……(中略)『這是導向苦滅道跡。』努力應該被作。」



\sutta{48}{48}{有孔之軛經第二}{https://agama.buddhason.org/SN/sn.php?keyword=56.48}
  「\twnr{比丘}{31.0}們!猶如這大地成為全部(單一)水,在那裡,男子在大海投入單孔軛,隨即,東風帶往西;西風帶往東;北風帶往南;南風帶往北,在那裡有隻盲龜,牠每經過一百年浮出一次。比丘們!你們怎麼想它:是否經過一百年浮出一次的盲龜,能使頸部進入那個單孔軛中呢?」

  「大德!這是偶然的:凡每一百年浮出一次的那隻盲龜能使頸部進入那個單孔軛中。」

  「比丘們!像這樣,這是偶然的:凡得到人的狀態,比丘們!像這樣,這是偶然的:凡\twnr{如來}{4.0}、\twnr{阿羅漢}{5.0}、\twnr{遍正覺者}{6.0}在世間出現(生起),比丘們!像這樣,這是偶然的:凡如來宣說的法律在世間{自娛}[\twnr{照耀}{x669}]。

  比丘們!這人的狀態已得到,如來、阿羅漢、遍正覺者已在世間出現,以及如來宣說的法律在世間照耀。

  比丘們!因此,在這裡,『這是苦。』努力應該被作……(中略)『這是導向苦\twnr{滅道跡}{69.0}。』努力應該被作。」



\sutta{49}{49}{須彌山山王經第一}{https://agama.buddhason.org/SN/sn.php?keyword=56.49}
  「\twnr{比丘}{31.0}們!猶如男子對\twnr{須彌山山王}{272.0}放置七顆綠豆大小的碎石,比丘們!你們怎麼想它,哪個是比較多的呢:凡被放置的七顆綠豆大小的碎石,或凡須彌山山王?」

  「\twnr{大德}{45.0}!這正是比較多的,即:須彌山山王,被放置的七顆綠豆大小碎石是少量的。被放置的七顆綠豆大小碎石比須彌山山王後,不來到計算,也不來到比較,也不來到十六分之一的部分。」

  「同樣的,比丘們!對\twnr{見具足之人}{575.0}、已\twnr{現觀}{53.0}的\twnr{聖弟子}{24.0},這正是比較多的,即:已遍滅盡、已耗盡(遍取)的苦,殘留的是少量的,即:\twnr{最多七次}{161.0}的狀態,比較先前已遍滅盡、已耗盡的苦蘊後,不來到計算,也不來到比較,也不來到十六分之一的部分,凡如實知道『這是苦』……(中略)如實知道『這是導向苦\twnr{滅道跡}{69.0}』。

  比丘們!因此,『這是苦。』努力應該被作……(中略)『這是導向苦滅道跡。』努力應該被作。」



\sutta{50}{50}{須彌山山王經第二}{https://agama.buddhason.org/SN/sn.php?keyword=56.50}
  「\twnr{比丘}{31.0}們!猶如\twnr{須彌山山王}{272.0}除了七顆綠豆大小的碎石外,走到遍盡、耗盡(遍取),比丘們!你們怎麼想它,哪個是比較多的呢:凡七顆綠豆大小的碎石,或凡已遍盡、已耗盡的須彌山山王?」

  「大德!這正是比較多的,即:已遍盡、已耗盡的須彌山山王,殘餘的七顆綠豆大小的碎石是少量的。殘餘的七顆綠豆大小的小石粒比較已遍盡、已耗盡的須彌山山王後,不來到計算,也不來到比較,也不來到十六分之一的部分。」

  「同樣的,比丘們!對\twnr{見具足之人}{575.0}、已\twnr{現觀}{53.0}的\twnr{聖弟子}{24.0},這正是比較多的,即:已遍滅盡、已耗盡的苦,殘留的是少量的,即:\twnr{最多七次}{161.0}的狀態,比較先前已遍滅盡、已耗盡的苦蘊後,不來到計算,也不來到比較,也不來到十六分之一的部分,凡如實知道『這是苦』……(中略)如實知道『這是導向苦\twnr{滅道跡}{69.0}』。

  比丘們!因此,在這裡,『這是苦。』努力應該被作……(中略)『這是導向苦滅道跡。』努力應該被作。」

  斷崖品第五,其\twnr{攝頌}{35.0}:

  「思惟、斷崖、熱惱,屋頂、毛與黑暗,

   以及以孔為二說,須彌山二則在後。」





\pin{現觀品}{51}{60}
\sutta{51}{51}{指甲尖經}{https://agama.buddhason.org/SN/sn.php?keyword=56.51}
  那時,\twnr{世尊}{12.0}使微少塵土沾在指甲尖後,召喚\twnr{比丘}{31.0}們:

  「比丘們!你們怎麼想它,哪個是比較多的呢:凡這被我沾在指甲尖的微少塵土,或凡這大地?」

  「\twnr{大德}{45.0}!這正是比較多的,即:大地,被世尊沾在指甲尖的微少塵土是少量的。被世尊沾在指甲尖的微少塵土比較大地後,不來到計算,也不來到比較,也不來到十六分之一的部分。」

  「同樣的,比丘們!對\twnr{見具足之人}{575.0}、已\twnr{現觀}{53.0}的\twnr{聖弟子}{24.0},這正是比較多的,即:已遍滅盡、已耗盡(遍取)的苦,殘留的是少量的,即:\twnr{最多七次}{161.0}的狀態,比較先前已遍滅盡、已耗盡的苦蘊後,不來到計算,也不來到比較,也不來到十六分之一的部分,凡如實知道『這是苦』……(中略)如實知道『這是導向苦\twnr{滅道跡}{69.0}』。

  比丘們!因此,在這裡,『這是苦。』努力應該被作……(中略)『這是導向苦滅道跡。』努力應該被作。」



\sutta{52}{52}{蓮花池經}{https://agama.buddhason.org/SN/sn.php?keyword=56.52}
  「\twnr{比丘}{31.0}們!猶如被水充滿的、滿到邊緣的、能被烏鴉喝飲的蓮花池長五十\twnr{由旬}{148.0},寬五十由旬,深五十由旬。男子以茅草尖從那裡取出水,比丘們!你們怎麼想它,哪個是比較多的呢:凡被茅草尖取出的水,或凡蓮花池的水?」

  「\twnr{大德}{45.0}!這正是比較多的,即:蓮花池的水,被茅草尖取出的水是少量的。被茅草尖取出的水比較蓮花池的水後,不來到計算,也不來到比較,也不來到十六分之一的部分。」

  「同樣的,比丘們!……\twnr{聖弟子}{24.0}……(中略)努力應該被作。」



\sutta{53}{53}{合流經第一}{https://agama.buddhason.org/SN/sn.php?keyword=56.53}
  「\twnr{比丘}{31.0}們!猶如在這些大河會合、集合之處,即:恒河、耶牟那河、阿致羅筏底河、薩羅浮河、摩醯河,男子從那裡取出二或三滴水,比丘們!你們怎麼想它,哪個是比較多的呢:凡被取出的二或三滴水,或凡合流的水?」

  「大德!這正是比較多的,即:合流的水,被取出的二或三滴水是少量的。被取出的二或三滴水比較合流的水後,不來到計算,也不來到比較,也不來到十六分之一的部分。」

  「同樣的,比丘們!……\twnr{聖弟子}{24.0}……(中略)努力應該被作。」



\sutta{54}{54}{合流經第二}{https://agama.buddhason.org/SN/sn.php?keyword=56.54}
  「\twnr{比丘}{31.0}們!猶如在這些大河會合、集合之處,即:恒河、耶牟那河、阿致羅筏底河、薩羅浮河、摩醯河,那個水除了二或三滴水外,走到遍盡、耗盡(遍取),比丘們!你們怎麼想它,哪個是比較多的呢:凡已遍盡、已耗盡的合流水,或凡二或三滴殘留的水?」

  「\twnr{大德}{45.0}!這正是比較多的,即:已遍盡、已耗盡的合流水,二或三滴殘留的水是少量的。二或三滴殘留的水比較已遍盡、已耗盡的合流水後,不來到計算,也不來到比較,也不來到十六分之一的部分。」

  「同樣的,比丘們!……\twnr{聖弟子}{24.0}……(中略)努力應該被作。」



\sutta{55}{55}{大地經第一}{https://agama.buddhason.org/SN/sn.php?keyword=56.55}
  「\twnr{比丘}{31.0}們!猶如男子在大地上放置七顆棗子大小土團,比丘們!你們怎麼想它,哪個是比較多的呢:凡被放置的七顆棗子大小土團,或凡這大地?」

  「\twnr{大德}{45.0}!這正是比較多的,即:大地,而被放置的七顆棗子大小土團是少量的。放置的七顆棗子大小土團比較大地後,不來到計算,也不來到比較,也不來到十六分之一的部分。」 

  「同樣的,比丘們!……\twnr{聖弟子}{24.0}……(中略)努力應該被作。」



\sutta{56}{56}{大地經第二}{https://agama.buddhason.org/SN/sn.php?keyword=56.56}
  「\twnr{比丘}{31.0}們!猶如大地除了七顆棗子大小土團外,走到遍盡、耗盡(遍取),比丘們!你們怎麼想它,哪個是比較多的呢:凡已遍盡、已耗盡的大地,或凡殘留的七顆棗子大小土團?」

  「\twnr{大德}{45.0}!這正是比較多的,即:已遍盡、已耗盡的大地,殘留的七顆棗子大小土團是少量的。殘留的七顆棗子大小土團比較已遍盡、已耗盡的大地後,不來到計算,也不來到比較,也不來到十六分之一的部分。」 

  「同樣的,比丘們!……\twnr{聖弟子}{24.0}……(中略)努力應該被作。」



\sutta{57}{57}{大海經第一}{https://agama.buddhason.org/SN/sn.php?keyword=56.57}
  「比丘們!猶如男子從大海取出二或三滴水,比丘們!你們怎麼想它,哪個是比較多的呢:凡被取出的二或三滴水,或凡大海的水?」

  「\twnr{大德}{45.0}!這正是比較多的,即:大海的水,被取出的二或三滴水是少量的。被取出的二或三滴水比較大海的水後,不來到計算,也不來到比較,也不來到十六分之一的部分。」

  「同樣的,比丘們!……\twnr{聖弟子}{24.0}……(中略)努力應該被作。」



\sutta{58}{58}{大海經第二}{https://agama.buddhason.org/SN/sn.php?keyword=56.58}
  「比丘們!猶如大海除了二或三滴水外,走到遍盡、耗盡(遍取),比丘們!你們怎麼想它,哪個是比較多的呢:凡已遍盡、已耗盡的大海水,或凡二或三滴殘留的水?」

  「\twnr{大德}{45.0}!這正是比較多的,即:已遍盡、已耗盡的大海水,二或三滴殘留的水是少量的。二或三滴殘留的水比較已遍盡、已耗盡的大海水後,不來到計算,也不來到比較,也不來到十六分之一的部分。」

  「同樣的,比丘們!……\twnr{聖弟子}{24.0}……(中略)努力應該被作。」



\sutta{59}{59}{如山經第一}{https://agama.buddhason.org/SN/sn.php?keyword=56.59}
  「\twnr{比丘}{31.0}們!猶如男子對喜瑪拉雅山山王放置七顆芥子大小的小石粒,比丘們!你們怎麼想它,哪個是比較多的呢:凡七顆芥子大小的小石粒,或凡這喜瑪拉雅山山王?」

  「\twnr{大德}{45.0}!這正是比較多的,即:喜瑪拉雅山山王,而七顆芥子大小的小石粒是少量的。七顆芥子大小的小石粒比較喜瑪拉雅山山王後,不來到計算,也不來到比較,也不來到十六分之一的部分。」

  「同樣的,比丘們!……\twnr{聖弟子}{24.0}……(中略)努力應該被作。」



\sutta{60}{60}{如山經第二}{https://agama.buddhason.org/SN/sn.php?keyword=56.60}
  「\twnr{比丘}{31.0}們!猶如喜瑪拉雅山山王除了七顆芥子大小的小石粒外,走到遍盡、耗盡(遍取),比丘們!你們怎麼想它,哪個是比較多的呢:凡七顆芥子大小的小石粒,或凡已遍盡、已耗盡的喜瑪拉雅山山王?」

  「\twnr{大德}{45.0}!這正是比較多的,即:已遍盡、已耗盡的喜瑪拉雅山山王,而殘留的七顆芥子大小小石粒是少量的。殘留的七顆芥子大小小石粒比較已遍盡、已耗盡的喜瑪拉雅山山王後,不來到計算,也不來到比較,也不來到十六分之一的部分。」

  「同樣的,比丘們!對\twnr{見具足之人}{575.0}、已\twnr{現觀}{53.0}的\twnr{聖弟子}{24.0},這正是比較多的,即:已遍盡、已耗盡的苦,殘留的是少量的,即:\twnr{最多七次}{161.0}的狀態,比較先前已遍盡、已耗盡的苦蘊後,不來到計算,也不來到比較,也不來到十六分之一的部分,凡如實知道『這是苦』……(中略)如實知道『這是導向苦\twnr{滅道跡}{69.0}』。

  比丘們!因此,在這裡,『這是苦。』努力應該被作……(中略)『這是導向苦滅道跡。』努力應該被作。」

  現觀品第六,其\twnr{攝頌}{35.0}:

  「指甲尖、蓮花池,合流在後二則,

   地二則、海二則,以及二則像山一樣。」





\pin{第一生穀中略品}{61}{70}
\sutta{61}{61}{他處經}{https://agama.buddhason.org/SN/sn.php?keyword=56.61}
  那時,\twnr{世尊}{12.0}使微少塵土沾在指甲尖後,召喚\twnr{比丘}{31.0}們:

  「比丘們!你們怎麼想它,哪個是比較多的呢:凡這被我沾在指甲尖的微少塵土,或凡這大地?」

  「\twnr{大德}{45.0}!這正是比較多的,即:大地,被世尊沾在指甲尖的微少塵土是少量的。被世尊沾在指甲尖的微少塵土比較大地後,不來到計算,也不來到比較,也不來到十六分之一的部分。」

  「同樣的,比丘們!那些眾生是少的:凡再生於人,而這些眾生正是更多的:凡從人間再生於他處,那是什麼原因?比丘們!以四聖諦的未看見狀態,哪四個?苦聖諦……(中略)導向苦\twnr{滅道跡}{69.0}聖諦。

  比丘們!因此,在這裡,『這是苦。』努力應該被作……(中略)『這是導向苦滅道跡。』努力應該被作。」[\suttaref{SN.20.2}]



\sutta{62}{62}{邊地經}{https://agama.buddhason.org/SN/sn.php?keyword=56.62}
  那時,\twnr{世尊}{12.0}使微少塵土沾在指甲尖後,召喚\twnr{比丘}{31.0}們:

  「比丘們!你們怎麼想它,哪個是比較多的呢:凡這被我沾在指甲尖的微少塵土,或凡這大地?」

  「\twnr{大德}{45.0}!這正是比較多的,即:大地,被世尊沾在指甲尖的微少塵土是少量的。被世尊沾在指甲尖的微少塵土比較大地後,不來到計算,也不來到比較,也不來到十六分之一的部分。」

  「同樣的,比丘們!那些眾生是少的:凡再生\twnr{於中國地方}{x670},而這些眾生正是更多的:凡再生於\twnr{邊地地方}{30.1}無知蠻族……(中略)。」



\sutta{63}{63}{慧經}{https://agama.buddhason.org/SN/sn.php?keyword=56.63}
  ……「同樣的,比丘們!那些眾生是少的:凡具備聖慧眼的,而凡這些眾生正是更多的:凡\twnr{進入無明}{645.0}癡昧者……(中略)。」



\sutta{64}{64}{榖酒果酒經}{https://agama.buddhason.org/SN/sn.php?keyword=56.64}
  ……「同樣的,比丘們!那些眾生是少的:凡離榖酒、果酒、\twnr{酒放逸處}{107.0}者,而這些眾生正是更多的:凡不離榖酒、果酒、酒放逸處者……(中略)。」



\sutta{65}{65}{水經}{https://agama.buddhason.org/SN/sn.php?keyword=56.65}
  ……「同樣的,\twnr{比丘}{31.0}們!那些眾生是少的:凡陸生的,而這些眾生正是更多的:凡水中的[AN1.322],那是什麼原因?……(中略)。」



\sutta{66}{66}{尊敬母親經}{https://agama.buddhason.org/SN/sn.php?keyword=56.66}
  ……「同樣的,\twnr{比丘}{31.0}們!那些眾生是少的:凡尊敬母親者,而這些眾生正是更多的:凡不尊敬母親者……(中略)。」



\sutta{67}{67}{尊敬父親經}{https://agama.buddhason.org/SN/sn.php?keyword=56.67}
  ……「同樣的,\twnr{比丘}{31.0}們!那些眾生是少的:凡尊敬父親者,而這些眾生正是更多的:凡不尊敬父親者……(中略)。」



\sutta{68}{68}{尊敬沙門經}{https://agama.buddhason.org/SN/sn.php?keyword=56.68}
  ……「同樣的,\twnr{比丘}{31.0}們!那些眾生是少的:凡\twnr{尊敬沙門}{328.1}者,而這些眾生正是更多的:凡不尊敬沙門者……(中略)。」



\sutta{69}{69}{尊敬婆羅門經}{https://agama.buddhason.org/SN/sn.php?keyword=56.69}
  ……「同樣的,\twnr{比丘}{31.0}們!那些眾生是少的:凡尊敬婆羅門者,而這些眾生正是更多的:凡不尊敬婆羅門者……(中略)。」



\sutta{70}{70}{尊敬經}{https://agama.buddhason.org/SN/sn.php?keyword=56.70}
  ……「同樣的,\twnr{比丘}{31.0}們!那些眾生是少的:凡尊敬家族中長輩者,而這些眾生正是更多的:凡不尊敬家族中長輩者……(中略)。」

  第一生穀中略品第七,其\twnr{攝頌}{35.0}:

  「他處、邊地、慧,榖酒果酒、水,

   尊敬母親而且也尊敬父親,尊敬沙門、婆羅門。」





\pin{第二生穀中略品}{71}{80}
\sutta{71}{71}{殺生經}{https://agama.buddhason.org/SN/sn.php?keyword=56.71}
  ……(中略)「同樣的,\twnr{比丘}{31.0}們!那些眾生是少的:凡離殺生者,而這些眾生正是更多的:凡不離殺生者,那是什麼原因?……(中略)。」



\sutta{72}{72}{未給予而取經}{https://agama.buddhason.org/SN/sn.php?keyword=56.72}
  ……(中略)「同樣的,\twnr{比丘}{31.0}們!那些眾生是少的:凡離\twnr{未給予而取}{104.0}者,而這些眾生正是更多的:凡不離未給予而取者……(中略)。」



\sutta{73}{73}{邪淫經}{https://agama.buddhason.org/SN/sn.php?keyword=56.73}
  ……(中略)「同樣的,\twnr{比丘}{31.0}們!那些眾生是少的:凡離\twnr{邪淫}{105.0}者,而這些眾生正是更多的:凡不離邪淫者……(中略)。」



\sutta{74}{74}{妄語經}{https://agama.buddhason.org/SN/sn.php?keyword=56.74}
  ……(中略)「同樣的,\twnr{比丘}{31.0}們!那些眾生是少的:凡離\twnr{妄語}{106.0}者,而這些眾生正是更多的:凡不離妄語者……(中略)。」



\sutta{75}{75}{離間語經}{https://agama.buddhason.org/SN/sn.php?keyword=56.75}
  ……(中略)「同樣的,\twnr{比丘}{31.0}們!那些眾生是少的:凡離\twnr{離間語}{234.0}者,而這些眾生正是更多的:凡不離離間語者……(中略)。」



\sutta{76}{76}{粗惡語經}{https://agama.buddhason.org/SN/sn.php?keyword=56.76}
  ……(中略)「同樣的,\twnr{比丘}{31.0}們!那些眾生是少的:凡離\twnr{粗惡語}{235.0}者,而這些眾生正是更多的:凡不離粗惡語者……(中略)。」



\sutta{77}{77}{雜穢語經}{https://agama.buddhason.org/SN/sn.php?keyword=56.77}
  ……(中略)「同樣的,\twnr{比丘}{31.0}們!那些眾生是少的:凡離\twnr{雜穢語}{236.0}者,而這些眾生正是更多的:凡不離雜穢語者……(中略)。」



\sutta{78}{78}{種子類經}{https://agama.buddhason.org/SN/sn.php?keyword=56.78}
  ……(中略)「同樣的,\twnr{比丘}{31.0}們!那些眾生是少的:凡離破壞\twnr{種子類}{638.0}、草木類,而這些眾生正是更多的:凡不離破壞種子類、草木類……(中略)。」



\sutta{79}{79}{非時食經}{https://agama.buddhason.org/SN/sn.php?keyword=56.79}
  ……(中略)「同樣的,\twnr{比丘}{31.0}們!那些眾生是少的:凡離\twnr{非時食}{266.2}者,而這些眾生正是更多的:凡不離非時食者……(中略)。」



\sutta{80}{80}{香料塗油經}{https://agama.buddhason.org/SN/sn.php?keyword=56.80}
  ……(中略)「同樣的,\twnr{比丘}{31.0}們!那些眾生是少的:凡離花環、香料、塗油之持有、莊嚴、裝飾狀態者,而這些眾生正是更多的:凡不離花環、香料、塗油之持有、莊嚴、裝飾狀態者……(中略)。」

  第二生穀中略品第八,其攝頌:

  「生命、未給予、在諸欲上,妄語與離間語,

   粗惡語、雜穢語,種子與非時、香料。」





\pin{第三生穀中略品}{81}{90}
\sutta{81}{81}{跳舞歌曲經}{https://agama.buddhason.org/SN/sn.php?keyword=56.81}
  ……(中略)「同樣的,\twnr{比丘}{31.0}們!那些眾生是少的:凡離跳舞、歌曲、音樂、看戲者,而這些眾生正是更多的:凡不離跳舞、歌曲、音樂、看戲者,那是什麼原因?……(中略)。」



\sutta{82}{82}{高床經}{https://agama.buddhason.org/SN/sn.php?keyword=56.82}
  ……(中略)「同樣的,\twnr{比丘}{31.0}們!那些眾生是少的:凡離高床、\twnr{大床}{992.0}者,而這些眾生正是更多的:凡不離高床、大床者……(中略)。」



\sutta{83}{83}{金銀經}{https://agama.buddhason.org/SN/sn.php?keyword=56.83}
  ……(中略)「同樣的,\twnr{比丘}{31.0}們!那些眾生是少的:凡離金銀的領受者,而這些眾生正是更多的:凡不離金銀的領受者……(中略)。」



\sutta{84}{84}{生穀經}{https://agama.buddhason.org/SN/sn.php?keyword=56.84}
  ……(中略)「同樣的,\twnr{比丘}{31.0}們!那些眾生是少的:凡離生穀的領受者,而這些眾生正是更多的:凡不離生穀的領受者……(中略)。」



\sutta{85}{85}{生肉經}{https://agama.buddhason.org/SN/sn.php?keyword=56.85}
  ……(中略)「同樣的,\twnr{比丘}{31.0}們!那些眾生是少的:凡離生肉的領受者,而這些眾生正是更多的:凡不離生肉的領受者……(中略)。」



\sutta{86}{86}{少女經}{https://agama.buddhason.org/SN/sn.php?keyword=56.86}
  ……(中略)「同樣的,\twnr{比丘}{31.0}們!那些眾生是少的:凡離女子、少女的領受者,而這些眾生正是更多的:凡不離女子、少女的領受者……(中略)。」



\sutta{87}{87}{男奴僕女奴經}{https://agama.buddhason.org/SN/sn.php?keyword=56.87}
  ……(中略)「同樣的,\twnr{比丘}{31.0}們!那些眾生是少的:凡離男奴僕、女奴的領受者,而這些眾生正是更多的:凡不離男奴僕、女奴的領受者……(中略)。」



\sutta{88}{88}{山羊與羊經}{https://agama.buddhason.org/SN/sn.php?keyword=56.88}
  ……(中略)「同樣的,\twnr{比丘}{31.0}們!那些眾生是少的:凡離山羊與羊的領受者,而這些眾生正是更多的:凡不離領受山羊與羊的領受者……(中略)。」



\sutta{89}{89}{雞與豬經}{https://agama.buddhason.org/SN/sn.php?keyword=56.89}
  ……(中略)「同樣的,\twnr{比丘}{31.0}們!那些眾生是少的:凡離雞豬的領受者,而這些眾生正是更多的:凡不離雞豬的領受者……(中略)。」



\sutta{90}{90}{象與牛經}{https://agama.buddhason.org/SN/sn.php?keyword=56.90}
  ……(中略)「同樣的,\twnr{比丘}{31.0}們!那些眾生是少的:凡離象、牛、馬、騾馬的領受者,而這些眾生正是更多的:凡不離象、牛、馬、騾馬的領受者……(中略)。」

  第三生穀中略品第九,其\twnr{攝頌}{35.0}:

  「跳舞、床、銀,榖、肉、少女,

   男奴連同山羊與羊,雞、豬、象。」





\pin{第四生穀中略品}{91}{101}
\sutta{91}{91}{田與地經}{https://agama.buddhason.org/SN/sn.php?keyword=56.91}
  ……(中略)「同樣的,\twnr{比丘}{31.0}們!那些眾生是少的:凡離田地宅地的領受者,而這些眾生正是更多的:凡不離田地宅地的領受者……(中略)。」



\sutta{92}{92}{買賣經}{https://agama.buddhason.org/SN/sn.php?keyword=56.92}
  ……(中略)「同樣的,\twnr{比丘}{31.0}們!那些眾生是少的:凡離買賣者,而這些眾生正是更多的:凡不離買賣者……(中略)。」



\sutta{93}{93}{差使經}{https://agama.buddhason.org/SN/sn.php?keyword=56.93}
  ……(中略)「同樣的,\twnr{比丘}{31.0}們!那些眾生是少的:凡離差使、遣使之從事者,而這些眾生正是更多的:凡不離差使、遣使之從事者……(中略)。」



\sutta{94}{94}{在秤重上欺瞞經}{https://agama.buddhason.org/SN/sn.php?keyword=56.94}
  ……(中略)「同樣的,\twnr{比丘}{31.0}們!那些眾生是少的:凡離在秤重上欺瞞、偽造貨幣、度量欺詐者,而這些眾生正是更多的:凡不離在秤重上欺瞞、偽造貨幣、度量欺詐者……(中略)。」



\sutta{95}{95}{賄賂經}{https://agama.buddhason.org/SN/sn.php?keyword=56.95}
  ……(中略)「同樣的,\twnr{比丘}{31.0}們!那些眾生是少的:凡離賄賂、欺瞞、詐欺、不實者,而這些眾生正是更多的:凡不離賄賂、欺瞞、詐欺、不實者……(中略)。」



\sutta{96}{101}{割截等經}{https://agama.buddhason.org/SN/sn.php?keyword=56.96}
  ……(中略)「同樣的,\twnr{比丘}{31.0}們!那些眾生是少的:凡離割截、殺害、捕縛、搶奪、掠奪、暴力者,而這些眾生正是更多的:凡不離割截、殺害、捕縛、搶奪、掠奪、暴力者,那什麼原因呢?以四聖諦的未看見狀態,哪四個?苦聖諦……(中略)導向苦\twnr{滅道跡}{69.0}聖諦。

  比丘們!因此,在這裡,『這是苦。』努力應該被作……(中略)『這是導向苦滅道跡。』努力應該被作。」

  第四生穀中略品第十,其\twnr{攝頌}{35.0}:

  「田地、買、差使,在秤重上欺瞞、賄賂,

   割截、殺害、捕縛,搶奪、掠奪、暴力。」





\pin{五趣中略品}{102}{131}
\sutta{102}{102}{人死地獄經}{https://agama.buddhason.org/SN/sn.php?keyword=56.102}
  那時,\twnr{世尊}{12.0}使微少塵土沾在指甲尖後,召喚\twnr{比丘}{31.0}們:

  「比丘們!你們怎麼想它,哪個是比較多的呢:凡這被我沾在指甲尖的微少塵土,或凡這大地?」

  「\twnr{大德}{45.0}!這正是比較多的,即:大地,被世尊沾在指甲尖的微少塵土是少量的。被世尊沾在指甲尖的微少塵土比較大地後,不來到計算,也不來到比較,也不來到十六分之一的部分。」

  「同樣的,比丘們!那些眾生是少的:凡從人死沒者再生於人,而這些眾生正是更多的:凡從人死沒者再生於地獄……(中略)。」[\suttaref{SN.20.2}]



\sutta{103}{103}{人死畜生經}{https://agama.buddhason.org/SN/sn.php?keyword=56.103}
  「同樣的,\twnr{比丘}{31.0}們!那些眾生是少的:凡從人死後再生於人,而這些眾生正是更多的:凡從人死後再生於畜生界……(中略)。」



\sutta{104}{104}{人死餓鬼界經}{https://agama.buddhason.org/SN/sn.php?keyword=56.104}
  「同樣的,\twnr{比丘}{31.0}們!那些眾生是少的:凡從人死後再生於人,而這些眾生正是更多的:凡從人死後再生於餓鬼界……(中略)。」



\sutta{105}{107}{人死天地獄等經}{https://agama.buddhason.org/SN/sn.php?keyword=56.105}
  「同樣的,\twnr{比丘}{31.0}們!那些眾生是少的:凡從人死後再生於諸天,而這些眾生正是更多的:凡從人死後再生於地獄……(中略)再生與畜生界……(中略)再生於餓鬼界……(中略)。」



\sutta{108}{110}{天死地獄等經}{https://agama.buddhason.org/SN/sn.php?keyword=56.108}
  「同樣的,\twnr{比丘}{31.0}們!那些眾生是少的:凡從天神死後再生於諸天,而這些眾生正是更多的:凡從天神死後再生於地獄……(中略)再生於畜生界……(中略)再生於餓鬼界……(中略)。」



\sutta{111}{113}{天人地獄等經}{https://agama.buddhason.org/SN/sn.php?keyword=56.111}
  「同樣的,\twnr{比丘}{31.0}們!那些眾生是少的:凡從天神死後再生於人,而這些眾生正是更多的:凡從天神死後再生於地獄……(中略)再生於畜生界……(中略)再生於餓鬼界……(中略)。」



\sutta{114}{116}{地獄人地獄等經}{https://agama.buddhason.org/SN/sn.php?keyword=56.114}
  「同樣的,\twnr{比丘}{31.0}們!那些眾生是少的:凡從地獄死後再生於人,而這些眾生正是更多的:凡從地獄死後再生於地獄……(中略)再生於畜生界……(中略)再生於餓鬼界……(中略)。」



\sutta{117}{119}{地獄天地獄等經}{https://agama.buddhason.org/SN/sn.php?keyword=56.117}
  「同樣的,\twnr{比丘}{31.0}們!那些眾生是少的:凡從地獄死後再生於諸天,而這些眾生正是更多的:凡從地獄死後再生於地獄……(中略)再生於畜生界……(中略)再生於餓鬼界……(中略)。」



\sutta{120}{122}{畜生人地獄等經}{https://agama.buddhason.org/SN/sn.php?keyword=56.120}
  「同樣的,\twnr{比丘}{31.0}們!那些眾生是少的:凡從畜生界死後再生於人,而這些眾生正是更多的:凡從畜生界死後再生於地獄……(中略)再生於畜生界……(中略)再生於餓鬼界……(中略)。」



\sutta{123}{125}{畜生天地獄等經}{https://agama.buddhason.org/SN/sn.php?keyword=56.123}
  「同樣的,\twnr{比丘}{31.0}們!那些眾生是少的:凡從畜生界死後再生於諸天,而這些眾生正是更多的:凡從畜生界死後再生於地獄……(中略)再生於畜生界……(中略)再生於餓鬼界……(中略)。」



\sutta{126}{128}{餓鬼人地獄等經}{https://agama.buddhason.org/SN/sn.php?keyword=56.126}
  「同樣的,\twnr{比丘}{31.0}們!那些眾生是少的:凡從\twnr{餓鬼界}{362.0}死後再生於人,而這些眾生正是更多的:凡從餓鬼界死後再生於地獄……(中略)再生於畜生界……(中略)再生於餓鬼界……(中略)。」



\sutta{129}{130}{餓鬼天地獄等經}{https://agama.buddhason.org/SN/sn.php?keyword=56.129}
  「同樣的,\twnr{比丘}{31.0}們!那些眾生是少的:凡從\twnr{餓鬼界}{362.0}死後再生於諸天,而這些眾生正是更多的:凡從餓鬼界死後再生於地獄……(中略)。」

  「同樣的,比丘們!那些眾生是少的:凡從餓鬼界死後再生於諸天,而這些眾生正是更多的:凡從餓鬼界死後再生於畜生界……(中略)。」



\sutta{131}{131}{餓鬼天餓鬼界經}{https://agama.buddhason.org/SN/sn.php?keyword=56.131}
  「同樣的,\twnr{比丘}{31.0}們!那些眾生是少的:凡從\twnr{餓鬼界}{362.0}死後再生於諸天,而這些眾生正是更多的:凡從餓鬼界死後再生於餓鬼界,那什麼原因呢?以四聖諦的未看見狀態,哪四個?苦聖諦、苦集聖諦、苦滅聖諦、導向苦\twnr{滅道跡}{69.0}聖諦。

  比丘們!因此,在這裡,『這是苦。』努力應該被作,『這是苦集。』努力應該被作,『這是苦滅。』努力應該被作,『這是導向苦滅道跡。』努力應該被作。」

  \twnr{世尊}{12.0}說這個,那些悅意的比丘歡喜世尊的所說。

  五趣中略品第十一,其\twnr{攝頌}{35.0}:

  「從人死後六則,從天死後地獄[也六則],

   畜生餓鬼界,趣品有三十則之多。」

  諦相應第十二。

  大篇第五,其攝頌:

  「道、覺支,念,根,正勤,

   力、神足、阿那律,禪、入出息相應,

   入流,以及諦,這被稱為『大篇』。」

  大篇相應經典終了。





\page


\startitemgroup[noteitems]
\item\subnoteref{1.0}\NoteKeywordAgamaHead{「如是我聞(SA/DA);我聞如是(MA);聞如是(AA)」},南傳作\NoteKeywordNikaya{「被我這麼聽聞」}(Evaṃ me sutaṃ),菩提比丘長老英譯為\NoteKeywordBhikkhuBodhi{「這樣我聽到」}(Thus have I heard)。按:「如是我聞」,依今日的習慣,「我」成為主詞,「聞」為動詞,但原文「我」為工具格,「聞」為過去分詞,兩者的表達模式都是古文沒有的。又,「如是我聞……歡喜奉行。」的經文格式,依印順法師的考定,應該是在《增一阿含》或《增支部》成立的時代才形成的(《原始佛教聖典之集成》p.9),南傳《相應部》多數經只簡略地指出發生地點,應該是比較早期的風貌。
\stopitemgroup

\startitemgroup[noteitems]
\item\subnoteref{2.0}\NoteKeywordNikayaHead{「一時」},南傳作\NoteKeywordNikaya{「有一次」}(ekaṃ samayaṃ,逐字譯為「一-時」),菩提比丘長老英譯為\NoteKeywordBhikkhuBodhi{「有一次」}(On one occasion)。
\stopitemgroup

\startitemgroup[noteitems]
\item\subnoteref{3.0}\NoteKeywordNikayaHead{「佛;佛陀;覺者」}(buddho),菩提比丘長老英譯為\NoteKeywordBhikkhuBodhi{「開化者」}(the Enlightened One)。
\stopitemgroup

\startitemgroup[noteitems]
\item\subnoteref{4.0}\NoteKeywordNikayaHead{「如來」}(tathāgato,音譯為「多陀阿伽度;多薩阿竭」),菩提比丘長老英譯照錄不譯,其含意可以指佛陀,也可以是世俗語言中生死流轉的生命主體,參看《如來藏之研究》p.12。
\stopitemgroup

\startitemgroup[noteitems]
\item\subnoteref{5.0}\NoteSubKeyHead{(1)}\NoteKeywordAgamaHead{「阿羅漢/羅漢/阿羅呵/阿羅訶(SA);應真/至真(GA);無所著/無著/至真/至真人/真人/阿羅呵/阿羅訶(MA);阿羅漢/無所著/至真/真人(DA);阿羅漢/至真(AA)」},南傳作\NoteKeywordNikaya{「阿羅漢」}(arahaṃ, arahant,義譯為「應;應供」),智髻比丘長老英譯為「完成者」(Accomplished One),菩提比丘長老英譯照錄不譯。\ccchref{DN.29}{https://agama.buddhason.org/DN/dm.php?keyword=29}又稱之為「第四果」(catutthaṃ phalaṃ)。
\item\subnoteref{5.1}\NoteSubKeyHead{(2)}\NoteKeywordNikayaHead{「阿羅漢狀態」}(arahatta,另譯為「阿羅漢性;阿羅漢果」),菩提比丘長老英譯為\NoteKeywordBhikkhuBodhi{「阿羅漢狀態;阿羅漢身份」}(arahantship)。
\stopitemgroup

\startitemgroup[noteitems]
\item\subnoteref{6.0}\NoteKeywordAgamaHead{「等正覺;平等正覺(SA);正盡覺(MA);正遍知(DA)」},南傳作\NoteKeywordNikaya{「遍正覺者」}(sammāsambuddhaṃ,另譯為「正等覺者;正等正覺者」,音譯為「三藐三佛陀;三耶三佛」),菩提比丘長老英譯為\NoteKeywordBhikkhuBodhi{「已純然無瑕的開化者」}(the Perfectly Enlightened Ones),或「遍正覺」(sammāsambodhiṃ,音譯為「三藐三菩提」),菩提比丘長老英譯為\NoteKeywordBhikkhuBodhi{「純然無瑕的開化」}(perfect enlightenment)。按:「正;等(SA/AA);平等(SA)」(sammā,另譯作「完全地;正確地」)。
\stopitemgroup

\startitemgroup[noteitems]
\item\subnoteref{7.0}\NoteKeywordAgamaHead{「明行足/明行具足(SA);明行成為/明行成/明行作證(MA);明行具足/明行成就者(DA);明行成為(AA)」},南傳作\NoteKeywordNikaya{「明行具足者」}(vijjācaraṇasampanno),菩提比丘長老英譯為\NoteKeywordBhikkhuBodhi{「在真實的理解與行為上已完成者」}(accomplished in true knowledge and conduct)。「明」(①宿命②天眼③漏盡等三明)、「行」(①戒具足②在諸根上守護門③飲食知適量④專修清醒⑤有信⑥有慚⑦有愧⑧多聞⑨活力已被發動⑩有念⑪有慧⑫~⑮四種禪),參看\ccchref{MN.53}{https://agama.buddhason.org/MN/dm.php?keyword=53}、\ccchref{DN.3}{https://agama.buddhason.org/DN/dm.php?keyword=3}。
\stopitemgroup

\startitemgroup[noteitems]
\item\subnoteref{8.0}\NoteKeywordAgamaHead{「善逝;修伽陀(AA)」},南傳作\NoteKeywordNikaya{「善逝」}(sugato,義譯為「已善去者;已達到善者;已到了善的情況者」),菩提比丘長老英譯為\NoteKeywordBhikkhuBodhi{「幸運者」}(the Fortunate One),或「善離」(is well departed, \ccchref{AN.5.200}{https://agama.buddhason.org/AN/an.php?keyword=5.200}),Maurice Walshe先生英譯為「完善者」(Well-Farer),或「善安置」(well set, \ccchref{DN.33}{https://agama.buddhason.org/DN/dm.php?keyword=33})。按:《顯揚真義》以完成者(sammaggatesu,正行者)解說\suttaref{SN.6.9}偈誦中的善逝,在這裡應泛指阿羅漢,而非獨指世尊。
\stopitemgroup

\startitemgroup[noteitems]
\item\subnoteref{9.0}\NoteKeywordNikayaHead{「世間解」},南傳作\NoteKeywordNikaya{「世間知者」}(lokavidū,逐字譯為「世間+知者;世間+賢明者」),菩提比丘長老英譯為\NoteKeywordBhikkhuBodhi{「世界的知道者」}(knower of the world)。
\stopitemgroup

\startitemgroup[noteitems]
\item\subnoteref{10.0}\NoteKeywordAgamaHead{「無上士調御丈夫/無上調御丈夫(SA);無上士調御丈夫/無上調御之師(GA);無上士道法御(MA);無上士道法御(AA)」},南傳作\NoteKeywordNikaya{「應該被調御人的無上調御者」}(anuttaro purisadammasārathi,另譯為「無上者、應該被調御人的調御者」),菩提比丘長老英譯為\NoteKeywordBhikkhuBodhi{「被馴服者的無可凌駕引導者」}(unsurpassed leader of persons to be tamed)。按:「無上士調御丈夫」是一個稱號,詳細請參看拙文《學佛的基本認識》〈\ccchref{佛陀十號的釐清}{https://agama.buddhason.org/book/bb/bb20.htm}〉。
\stopitemgroup

\startitemgroup[noteitems]
\item\subnoteref{11.0}\NoteKeywordNikayaHead{「天人師」},南傳作\NoteKeywordNikaya{「天-人們的大師」}(satthā devamanussānaṃ),菩提比丘長老英譯為\NoteKeywordBhikkhuBodhi{「天與人的老師」}(teacher of devas and humans)。
\stopitemgroup

\startitemgroup[noteitems]
\item\subnoteref{12.0}\NoteKeywordNikayaHead{「世尊;眾祐」},南傳作\NoteKeywordNikaya{「世尊」}(bhagavā,音譯為「婆伽婆;婆伽梵;薄伽梵」,義譯為「有幸者」,古譯為「尊祐」),菩提比丘長老英譯為\NoteKeywordBhikkhuBodhi{「幸福者」}(the Blessed One)。請參看拙文《學佛的基本認識》〈\ccchref{世尊譯詞的探討}{https://agama.buddhason.org/book/bb/bb21.htm}〉。
\stopitemgroup

\startitemgroup[noteitems]
\item\subnoteref{13.0}\NoteKeywordAgamaHead{「異我(SA);色是我所(SA/DA);神有色/色是我有(MA);色是我所有(摩訶僧祇律)」},南傳作\NoteKeywordNikaya{「或我擁有色」}(rūpavantaṃ vā attānaṃ,直譯為「或有色的我」),菩提比丘長老英譯為\NoteKeywordBhikkhuBodhi{「或以自我持有色」}(or self as possessing form)。按:《顯揚真義》說,取無色的為『我』後,如樹有影子般(chāyāvantaṃ rukkhaṃ viya)認為它是有色的。另外,「色是我」則以當油燈燃燒時,火焰是容色,容色是火焰(telappadīpassa jhāyato yā acci, so vaṇṇo. Yo vaṇṇo, sā accīti, \suttaref{SN.22.1})來比喻,《無礙解道》〈\ccchref{2.真我隨見的說明}{https://agama.buddhason.org/Ps/Ps2.htm\letterhash 2}〉有詳細的解說。
\stopitemgroup

\startitemgroup[noteitems]
\item\subnoteref{14.0}\NoteKeywordAgamaHead{「相在/色在我、我在色(SA);見神中有色,見色中有神(MA);色中有我,我中有色(DA/摩訶僧祇律)」},南傳作\NoteKeywordNikaya{「或色在我中,或我在色中」}(attani vā rūpaṃ, rūpasmiṃ vā attānaṃ),菩提比丘長老英譯為\NoteKeywordBhikkhuBodhi{「或以色在自我中,或以自我在色中」}(or form as in self, or self as in form)。按:「是我、異我、相在」即二十種「(有)身見」(薩迦耶見),參看\ccchref{SA.57}{https://agama.buddhason.org/SA/dm.php?keyword=57}、\ccchref{SA.109}{https://agama.buddhason.org/SA/dm.php?keyword=109}。《顯揚真義》說,取無色的為『我』,如氣味在花中般(pupphasmiṃ gandhaṃ viya)認為色在我中;如寶珠在盒子中般(karaṇḍake maṇiṃ viya, \suttaref{SN.22.1})認為我在色中(直譯:對我來說,在色中),《無礙解道》〈\ccchref{2.我隨見的說明}{https://agama.buddhason.org/Ps/Ps2.htm\letterhash 2}〉有詳細的解說。
\stopitemgroup

\startitemgroup[noteitems]
\item\subnoteref{15.0}\NoteSubKeyHead{(1)}\NoteKeywordAgamaHead{「厭;厭離(SA);患厭(MA/DA);厭患(AA)」},南傳作\NoteKeywordNikaya{「厭」}(nibbindati,名詞nibbidā),菩提比丘長老在英譯為「(由沈迷中)清醒」(disenchant, disenchantment, MN/AN),或「厭惡;反感」(revulsion, SN),長老表示,前者的程度不足,後者太過,真正的意思應該介於這兩者之間,但英文沒有恰如其分的字可以對譯(2003年7月/美國新澤西州同淨蘭若)。按:《顯揚真義》等以「厭;不滿意」(ukkaṇṭheyya, \suttaref{SN.12.61}/ukkaṇṭhati, \ccchref{MN.22}{https://agama.buddhason.org/MN/dm.php?keyword=22})解說,《破斥猶豫》並說nibbidā的意趣為「導向毘婆舍那的生起」(vuṭṭhānagāminīvipassanā, \ccchref{MN.22}{https://agama.buddhason.org/MN/dm.php?keyword=22})。
\item\subnoteref{15.1}\NoteSubKeyHead{(2)}\NoteKeywordAgamaHead{「正厭離(SA);平等厭患(AA)」},南傳作\NoteKeywordNikaya{「完全地厭」}(sammā nibbindamāno;另譯為「正厭」),菩提比丘長老英譯為\NoteKeywordBhikkhuBodhi{「完全地清醒」}(is completely disenchanted, AN)。
\stopitemgroup

\startitemgroup[noteitems]
\item\subnoteref{16.0}\NoteKeywordAgamaHead{「心解脫/意解脫(SA/AA);心解脫(MA/DA)」},南傳作\NoteKeywordNikaya{「心解脫」}(cetovimutti),菩提比丘長老英譯為\NoteKeywordBhikkhuBodhi{「心的釋放;心的自由」}(the liberation of the mind)。按:\ccchref{SA.710}{https://agama.buddhason.org/SA/dm.php?keyword=710}說:「離貪欲者心解脫;離無明者慧解脫。」\ccchref{SA.1027}{https://agama.buddhason.org/SA/dm.php?keyword=1027}、\ccchref{AN.2.32}{https://agama.buddhason.org/AN/an.php?keyword=2.32}亦同,《滿足希求》以「這是果定」(Phalasamādhissetaṃ, \ccchref{AN.2.32}{https://agama.buddhason.org/AN/an.php?keyword=2.32})解說。
\stopitemgroup

\startitemgroup[noteitems]
\item\subnoteref{17.0}\NoteSubKeyHead{(1)}\NoteKeywordAgamaHead{「婆羅門(SA/DA/AA);梵志(MA)」},南傳作\NoteKeywordNikaya{「婆羅門」}(brāhmaṇaṃ,另譯為「梵志:以求往生梵天為志者」),為佛陀時代傳統宗教的宗教師,後來成為一個種姓階層,自稱地位高於王族(剎帝利),但\ccchref{MA.154}{https://agama.buddhason.org/MA/dm.php?keyword=154}等說,其地位最初是在王族之下。另外,解脫阿羅漢、佛陀有時也被稱為婆羅門,如\ccchref{SA.1161}{https://agama.buddhason.org/SA/dm.php?keyword=1161}說「羅漢婆羅門」。
\item\subnoteref{17.1}\NoteSubKeyHead{(2)}\NoteKeywordAgamaHead{「婆羅門尼(SA);婆羅門家女(GA)」},即「女婆羅門」(brāhmaṇī)的音譯。
\stopitemgroup

\startitemgroup[noteitems]
\item\subnoteref{18.0}\NoteKeywordAgamaHead{「我生已盡(SA);生已盡(MA);生死已盡(DA/AA)」},南傳作\NoteKeywordNikaya{「出生已盡」}(khīṇā jāti),菩提比丘長老英譯為\NoteKeywordBhikkhuBodhi{「已被破壞的是出生;出生已被破壞」}(destroyed is birth)。
\stopitemgroup

\startitemgroup[noteitems]
\item\subnoteref{19.0}\NoteKeywordNikayaHead{「梵行已立」},南傳作\NoteKeywordNikaya{「梵行已完成」}(vusitaṃ brahmacariyaṃ),菩提比丘長老英譯為\NoteKeywordBhikkhuBodhi{「聖潔的生活已被生活」}(the holy life has been lived)。
\stopitemgroup

\startitemgroup[noteitems]
\item\subnoteref{20.0}\NoteKeywordAgamaHead{「所作已作(SA);所作已辦(GA/MA/AA/DA)」},南傳作\NoteKeywordNikaya{「應該被作的已作的;應該被作的已作者」}(kataṃ karaṇīyaṃ, katakicco, Kataṃ kiccaṃ),菩提比丘長老英譯為\NoteKeywordBhikkhuBodhi{「所有必須作的已經做完」}(what had to be done has been done),或「已經完成他的任務」(has done his task, \ccchref{AN.3.58}{https://agama.buddhason.org/AN/an.php?keyword=3.58}),或「任務已經完成」(Done is the task, \ccchref{AN.4.4}{https://agama.buddhason.org/AN/an.php?keyword=4.4})。
\stopitemgroup

\startitemgroup[noteitems]
\item\subnoteref{21.0}\NoteSubKeyHead{(1)}\NoteKeywordNikayaHead{「後有」},南傳作\NoteKeywordNikaya{「再有」}(ponobbhaviko),菩提比丘長老英譯為\NoteKeywordBhikkhuBodhi{「帶來重新的生命」}(brings renewed being)。「有」(bhava),同十二緣起支中的「有」支。
\item\subnoteref{21.1}\NoteSubKeyHead{(2)}\NoteKeywordAgamaHead{「自知不受後有/不復轉還有(SA);不更受有(MA);不受後有/更不受有(DA);更不復受有/更不復受胎(AA)」},南傳作\NoteKeywordNikaya{「不再有此處[輪迴]的狀態」}(nāparaṃ itthattāyā),菩提比丘長老英譯為\NoteKeywordBhikkhuBodhi{「不再有這生命的狀態」}(there is no more for this state of being),含意相當於「不受後有;不受有」、「沒有再有」(natthi punabbhava)。
\stopitemgroup

\startitemgroup[noteitems]
\item\subnoteref{22.0}\NoteKeywordAgamaHead{「我(SA/MA)」},南傳作\NoteKeywordNikaya{「我作」}(ahaṅkāra, ahaṃkāra),菩提比丘長老英譯為\NoteKeywordBhikkhuBodhi{「我-作;我-造」}(I-making)。按:《顯揚真義》等以「我作為(邪)見」(ahaṃkāradiṭṭhi, \suttaref{SN.18.11}/\ccchref{AN.3.32}{https://agama.buddhason.org/AN/an.php?keyword=3.32}, ahaṃkāro diṭṭhi, \ccchref{MN.72}{https://agama.buddhason.org/MN/dm.php?keyword=72})解說,水野弘元《巴利語辭典》作「我慢,我見」。
\stopitemgroup

\startitemgroup[noteitems]
\item\subnoteref{23.0}\NoteSubKeyHead{(1)}\NoteKeywordAgamaHead{「無我;非我(SA);非神(MA)」},南傳作\NoteKeywordNikaya{「無我」}(anattā,另譯為「非我」),這是「真我」(attā)加上否定前置詞(an),表示「實在、永恆、不變」的「真我」之否定,菩提比丘長老英譯為\NoteKeywordBhikkhuBodhi{「無自我」}(nonself)。「我」,中阿含經常譯為「神」。
\item\subnoteref{23.1}\NoteSubKeyHead{(2)}\NoteKeywordAgamaHead{「一切法無我(SA)」},南傳作\NoteKeywordNikaya{「一切法是無我」}(sabbe dhammā anattā’ti),菩提比丘長老英譯為\NoteKeywordBhikkhuBodhi{「所有現象是無自我」}(all phenomena are nonself, SN)。
\stopitemgroup

\startitemgroup[noteitems]
\item\subnoteref{24.0}\NoteKeywordNikayaHead{「多聞聖弟子」},南傳作\NoteKeywordNikaya{「有聽聞的聖弟子」}(sutavā ariyasāvako),菩提比丘長老英譯為\NoteKeywordBhikkhuBodhi{「已受教導之高潔的弟子」}(the instructed noble disciple)。按:《顯揚真義》以「開始教說;精勤教說」(desanaṃ ārabhi, \suttaref{SN.12.37})解說,所以「多聞」不只是「多聽」而已,也有受教導而實踐的意義。其中之「多聞」不只是「多聽」而已,應該含有受教導而實踐的意義。又,「有聽聞的」(sutavā=sutavant),也譯為「具聞的;博聞的;多聞的」。而「聖」(ariya,梵語ārya),與「雅利安人」(梵語aryans)之「雅利安」顯然同字,「雅利安人」為印歐族白種人,遷居入印度後,以高貴人種自居,也許是這個字的來源。在佛教中,「聖弟子」多指證入初果以上的聖者,但有時也泛指一般佛陀弟子。
\stopitemgroup

\startitemgroup[noteitems]
\item\subnoteref{25.0}\NoteKeywordAgamaHead{「我所見(SA);我所作(MA)」},南傳作\NoteKeywordNikaya{「我所作」}(mamaṅkāra, mamaṃkāra),菩提比丘長老英譯為\NoteKeywordBhikkhuBodhi{「我的-作;我的-造」}(mine-making)。按:《顯揚真義》等以「渴愛」(mamaṃkārataṇhā, \suttaref{SN.18.11}/\ccchref{AN.3.32}{https://agama.buddhason.org/AN/an.php?keyword=3.32}, mamaṃkāro taṇhā, \ccchref{MN.72}{https://agama.buddhason.org/MN/dm.php?keyword=72})解說,後者再加入等勝劣者的我等、我勝、我劣之「九種慢」(navavidhamānato, \ccchref{AN.7.49}{https://agama.buddhason.org/AN/an.php?keyword=7.49})解說,水野弘元《巴利語辭典》作「我所見,我所執」。
\stopitemgroup

\startitemgroup[noteitems]
\item\subnoteref{26.0}\NoteSubKeyHead{(1)}\NoteKeywordAgamaHead{「我慢使繫著/我慢繫著使(SA);慢使(SA/MA/DA);憍慢使(SA/AA)」},南傳作\NoteKeywordNikaya{「慢煩惱潛在趨勢」}(mānānusayā,另譯為「慢隨眠」),Maurice Walshe先生英譯為「自大之潛在傾向」(latent proclivity of conceit)。菩提比丘長老英譯為\NoteKeywordBhikkhuBodhi{「自大之表面下的趨勢」}(underlying tendency to conceit),並解說「慢」(māna)的根源是「我是之慢」(asmimāna),所以也為「我作」負責。按:《破斥猶豫》說,慢煩惱潛在趨勢是慢(mānānusayo māno, \ccchref{MN.72}{https://agama.buddhason.org/MN/dm.php?keyword=72})。
\item\subnoteref{26.1}\NoteSubKeyHead{(2)}\NoteKeywordNikayaHead{「見與慢煩惱潛在趨勢」}(diṭṭhimānānusayaṃ),智髻比丘長老英譯為「見解與自大在表面下的趨勢」(the underlying tendency to the view and conceit)。
\stopitemgroup

\startitemgroup[noteitems]
\item\subnoteref{27.0}\NoteKeywordAgamaHead{「解脫知見(SA/DA);便知解脫(MA);解脫見慧/解脫智(AA)」},南傳作\NoteSubEntryKey{(i)}\NoteKeywordNikaya{「有『[這是]解脫』之智」}(vimuttamiti ñāṇaṃ hoti),菩提比丘長老英譯為\NoteKeywordBhikkhuBodhi{「出現『這是已被釋放』的理解(智)」}(there comes the knowledge:"It's liberated")。按:《破斥猶豫》以「省察智」(paccavekkhaṇañāṇaṃ, \ccchref{MN.4}{https://agama.buddhason.org/MN/dm.php?keyword=4})解說。\NoteSubEntryKey{(ii)}\NoteKeywordNikaya{「解脫智見」}(vimuttiñāṇadassana),菩提比丘長老英譯為\NoteKeywordBhikkhuBodhi{「釋放的理解與眼光」}(the knowledge and vision of liberation, \ccchref{AN.10.1}{https://agama.buddhason.org/AN/an.php?keyword=10.1}),智髻比丘長老英譯為「釋放的理解與眼光」(the knowledge and vision of deliverance, \ccchref{MN.32}{https://agama.buddhason.org/MN/dm.php?keyword=32})。按:《顯揚真義》以「省察智」(paccavekkhaṇañāṇaṃ, \suttaref{SN.46.3})解說。顯然「有『[這是]解脫』之智」即「解脫智見」,是體驗解脫後的下一個心念(心路):知道自己已解脫了,亦即「無學正智」。
\stopitemgroup

\startitemgroup[noteitems]
\item\subnoteref{28.0}\NoteKeywordAgamaHead{「善解脫/正解脫/正善解脫(SA);善解脫(AA)」},南傳作\NoteKeywordNikaya{「善解脫」}(suvimuttaṃ),菩提比丘長老英譯為\NoteKeywordBhikkhuBodhi{「完全地被釋放了」}(well liberated)。按:《顯揚真義》以「阿羅漢果的解脫」(arahattaphalavimuttiyā, \suttaref{SN.22.56})解說。
\stopitemgroup

\startitemgroup[noteitems]
\item\subnoteref{29.0}\NoteSubKeyHead{(1)}\NoteKeywordAgamaHead{「沙門;道士(SA/AA);道人(GA/AA)」}(samaṇa)是婆羅門以外的出家修道者之通稱,「沙門尼」(samaṇī)為女性沙門。
\item\subnoteref{29.1}\NoteSubKeyHead{(2)}\NoteKeywordNikayaHead{「沙門法」}(samaṇadhamme),菩提比丘長老英譯為\NoteKeywordBhikkhuBodhi{「禁欲修道者的特質」}(the character of an ascetic, SN),或「禁欲修道者的責任」(the duty of an ascetic)。
\stopitemgroup

\startitemgroup[noteitems]
\item\subnoteref{30.0}\NoteSubKeyHead{(1)}\NoteKeywordNikayaHead{「邊地臥坐處」}(pantāni senāsanāni; pantañca sayanāsanaṃ),菩提比丘長老英譯為\NoteKeywordBhikkhuBodhi{「偏僻住處」}(remote lodgings, SN/MN/AN),Maurice Walshe先生英譯為「隱遁之住處;獨居生活;住偏僻處」(seclusion of dwelling, solitary life, dwell in remote, DN)。按:《吉祥悅意》以「擺脫接觸的臥坐處」(sayanāsanañca saṅghaṭṭanavirahitanti, \ccchref{DN.14}{https://agama.buddhason.org/DN/dm.php?keyword=14})解說,又,《滿足希求》以「極遠的邊界」(pariyantāni atidūrāni, \ccchref{AN.2.31}{https://agama.buddhason.org/AN/an.php?keyword=2.31})解說「邊地」。
\item\subnoteref{30.1}\NoteSubKeyHead{(2)}\NoteKeywordAgamaHead{「邊地(SA/AA);邊國/彼方土(MA)」},南傳作\NoteKeywordNikaya{「於邊地地方」}(paccantimesu janapadesu),菩提比丘長老英譯為\NoteKeywordBhikkhuBodhi{「在偏僻領域;在邊境地區」}(in the outlying provinces, SN/AN; in the borderlands, AN),Maurice Walshe先生英譯為「在邊境地區」(in the border regions, DN)。按:paccanta=paṭi+anta,逐字譯為「朝向+邊」,與panta=pa+anta,逐字譯為「在前+邊」,兩者皆為「邊地」,從字的構成來看應是同義字。
\stopitemgroup

\startitemgroup[noteitems]
\item\subnoteref{31.0}\NoteKeywordNikayaHead{「比丘」}(bhikkhu,義譯為「乞食者」),女性音譯為「比丘尼」(bhikkhunī),菩提比丘長老英譯照錄不譯。按:「比丘」與「乞食者」(bhikkhaka)在通俗話語中是同義詞,但佛教中的「比丘」有其特定條件與意義,形成「比丘」是「乞食者」,但「乞食者」不一定都是「比丘」。另外,「比丘」的台語發音,與巴利原音幾乎等同,玄奘法師則音譯為「苾芻」。
\stopitemgroup

\startitemgroup[noteitems]
\item\subnoteref{32.0}\NoteSubKeyHead{(1)}\NoteKeywordNikayaHead{「這是我的」}(etaṃ mama),菩提比丘長老英譯為\NoteKeywordBhikkhuBodhi{「這是我的」}(this is mine)。按:《顯揚真義》等以「渴愛之執」(taṇhāgāho/taṇhāggāho, \suttaref{SN.12.61}/\ccchref{MN.22}{https://agama.buddhason.org/MN/dm.php?keyword=22})解說,前者又說「因此百八種渴愛思潮被握持」(tena aṭṭhasatataṇhāvicaritaṃ gahitaṃ hoti)。
\item\subnoteref{32.1}\NoteSubKeyHead{(2)}\NoteKeywordAgamaHead{「彼一切非我有(MA);彼非我有(MA/DA/AA);彼不我有(AA)」},南傳作\NoteKeywordNikaya{「這不是我的」}(netaṃ mama),菩提比丘長老英譯為\NoteKeywordBhikkhuBodhi{「這不是我的」}(this is not mine)。
\stopitemgroup

\startitemgroup[noteitems]
\item\subnoteref{33.0}\NoteSubKeyHead{(1)}\NoteKeywordNikayaHead{「我是這個」}(esohamasmi, ayamahamasmi),菩提比丘長老英譯為\NoteKeywordBhikkhuBodhi{「我是這個」}(this I am)。按:《顯揚真義》等以「慢之執」(mānagāho, \suttaref{SN.12.61}/\ccchref{MN.22}{https://agama.buddhason.org/MN/dm.php?keyword=22})解說,前者又說「因此九種慢被握持」(tena nava mānā gahitā honti)。
\item\subnoteref{33.1}\NoteSubKeyHead{(2)}\NoteKeywordAgamaHead{「我非彼有(MA/DA/AA)」},南傳作\NoteKeywordNikaya{「我不是這個」}(nesohamasmi),菩提比丘長老英譯為\NoteKeywordBhikkhuBodhi{「我不是這個」}(this I am not)。
\stopitemgroup

\startitemgroup[noteitems]
\item\subnoteref{34.0}\NoteSubKeyHead{(1)}\NoteKeywordNikayaHead{「我的真我」}(attā me; me attā),智髻比丘長老英譯為「對我來說的真我」(for me a self, MN),菩提比丘長老英譯為\NoteKeywordBhikkhuBodhi{「我的自我」}(my self, SN)。按:《顯揚真義》等以「(邪)見之執」(diṭṭhigāho, \suttaref{SN.12.61}/\ccchref{MN.22}{https://agama.buddhason.org/MN/dm.php?keyword=22})解說,前者又說「因此六十二見被握持」(tena dvāsaṭṭhi diṭṭhiyo gahitā honti)。
\item\subnoteref{34.1}\NoteSubKeyHead{(2)}\NoteKeywordNikayaHead{「這是我的真我」}(meso attā),菩提比丘長老英譯為\NoteKeywordBhikkhuBodhi{「這是我的自我」}(this is my self)。
\item\subnoteref{34.2}\NoteSubKeyHead{(3)}\NoteKeywordAgamaHead{「亦非神(MA)」},南傳作\NoteKeywordNikaya{「這不是我的真我」}(na meso attā),菩提比丘長老英譯為\NoteKeywordBhikkhuBodhi{「這不是我的自我」}(this is not my self)。
\stopitemgroup

\startitemgroup[noteitems]
\item\subnoteref{35.0}此為「攝頌」(uddānaṃ),《瑜伽師地論》譯作「嗢拕南」,這是以前面或後面(如中阿含)十經的經名或經文關鍵詞依序串集,等同目錄功能,有助於「持經者」的背誦。十經是體例,少於十經者應有所脫落,多於十經者應有後來的增入。以\ccchref{SA.7}{https://agama.buddhason.org/SA/dm.php?keyword=7}為例,「時,諸比丘聞佛所說,歡喜奉行。」後面的「無常及苦、空、非我、正思惟,無知等四種,及於色喜樂。」即為前面十經的「攝頌」,印順法師即依此在其《雜阿含經論會編》中重編經號。
\stopitemgroup

\startitemgroup[noteitems]
\item\subnoteref{36.0}\NoteKeywordAgamaHead{「五受陰(SA/DA);五盛陰(MA/AA);五苦盛陰(AA)」},南傳作\NoteKeywordNikaya{「五取蘊」}(pañcannaṃ upādānakkhandhānaṃ, pañcupādānakkhandhā),菩提比丘長老英譯為\NoteKeywordBhikkhuBodhi{「所執著的五個集合體」}(five aggregates subject to clinging)。按:《顯揚真義》說,僅欲貪不是五蘊(chandarāgamattaṃ pañcakkhandhā na hoti),對蘊從天生或所緣脫離後沒有取(sahajātato vā ārammaṇato vā khandhe muñcitvā upādānaṃ natthi, \suttaref{SN.22.22}),《破斥猶豫》說,從取為行蘊的一部分狀態來說,僅那個取非那些五取蘊,但在五取蘊外也沒有取(nāpi aññatra pañcahi upādānakkhandhehi upādānaṃ, \ccchref{MN.44}{https://agama.buddhason.org/MN/dm.php?keyword=44})。
\stopitemgroup

\startitemgroup[noteitems]
\item\subnoteref{37.0}\NoteKeywordAgamaHead{「阿耨多羅三藐三菩提(SA);無上正盡之覺(MA);無上正真道(MA/DA/AA);最正覺(DA);無上正覺(DA/AA);無上正真等正覺/無上正真之道/無上至真等正覺(AA)」},南傳作\NoteKeywordNikaya{「無上遍正覺」}(anuttaraṃ sammāsambodhiṃ),菩提比丘長老英譯為\NoteKeywordBhikkhuBodhi{「無可凌駕、已純然無瑕的開化」}(unsurpassed perfect enlightenment)。按:「阿耨多羅三藐三菩提」為音譯,義譯為「無上遍正覺;無上等正覺;無上正等覺;無上正等正覺;無上正盡覺」這個詞,經文中只看到用於「佛陀」,不見用於「阿羅漢」或「辟支佛」上。
\stopitemgroup

\startitemgroup[noteitems]
\item\subnoteref{38.0}\NoteKeywordAgamaHead{「世人/眾/諸世間(SA);世人(AA)」},南傳作\NoteKeywordNikaya{「世代」}(pajāya, pajaṃ),菩提比丘長老英譯為\NoteKeywordBhikkhuBodhi{「在這一代」}(in this generation, SN),或「在這一群」(in this population, AN)。按:「世代」(pajā),水野弘元《巴利語辭典》作「人人」。
\stopitemgroup

\startitemgroup[noteitems]
\item\subnoteref{39.0}\NoteKeywordAgamaHead{「異(SA/AA)」},南傳作\NoteKeywordNikaya{「某位」}(aññataro,為「異」(añña)的「比較級」,另譯為「隨一;兩者之一」),菩提比丘長老英譯為\NoteKeywordBhikkhuBodhi{「某位」}(a certain)。如「異比丘」即「某位比丘」;「異婆羅門」即「某位婆羅門」。
\stopitemgroup

\startitemgroup[noteitems]
\item\subnoteref{40.0}\NoteKeywordNikayaHead{「稽首禮足;稽首佛足;稽首禮;稽首作禮;頭面禮足;禮足;接足禮;頂禮佛足;頂禮;頭面作禮;頭面禮世尊足;禮世尊足」},南傳作\NoteSubEntryKey{(i)}\NoteKeywordNikaya{「身體伏在地上之禮」}(nipaccakāraṃ,逐字譯為「倒禮(五體投地)+行為」或「跪拜+行為」),菩提比丘長老英譯為\NoteKeywordBhikkhuBodhi{「尊敬;敬意」}(honour)。\NoteSubEntryKey{(ii)}\NoteKeywordNikaya{「以頭落在世尊的腳上後」}(bhagavato pādesu sirasā nipatitvā),菩提比丘長老英譯為\NoteKeywordBhikkhuBodhi{「以他的頭拜倒在幸福者的腳上」}(prostrated himself with his head at the Blessed One's feet)。\NoteSubEntryKey{(iii)}\NoteKeywordNikaya{「以頭禮敬世尊的足」}(bhagavato pāde sirasā vandati),菩提比丘長老英譯為\NoteKeywordBhikkhuBodhi{「以頭在幸福者的腳上表示敬意」}(pays homage with his head at the Blessed One's feet)。按:梵語作「五體投地」(pañca-maṇḍala-namaskāreṇa van-date)。
\stopitemgroup

\startitemgroup[noteitems]
\item\subnoteref{41.0}\NoteKeywordNikayaHead{「善男子;族姓子;族姓男」},南傳作\NoteKeywordNikaya{「善男子」}(kulaputtassa, kolaputti,另譯為「良家子、族姓子、族姓男」),菩提比丘長老英譯為\NoteKeywordBhikkhuBodhi{「族人」}(clansmen)。按:通常這是指來自大姓人家之男子。
\stopitemgroup

\startitemgroup[noteitems]
\item\subnoteref{42.0}\NoteKeywordNikayaHead{「現法;見法;現世」},南傳作\NoteKeywordNikaya{「在當生中;當生」}(diṭṭheva dhamme,逐字譯為「見-法」,直接意思為「在被看見的法上」),菩提比丘長老英譯為\NoteKeywordBhikkhuBodhi{「就在這一生」}(in this very life)。
\stopitemgroup

\startitemgroup[noteitems]
\item\subnoteref{43.0}\NoteSubKeyHead{(1)}\NoteKeywordNikayaHead{「汝當諦聽;汝等諦聽;諦聽」},南傳作\NoteKeywordNikaya{「你們要聽;請你們聽」}(suṇātha),菩提比丘長老英譯為\NoteKeywordBhikkhuBodhi{「聽!」}(listen)。
\item\subnoteref{43.1} (2)「善思;善思念之」,南傳作\NoteKeywordNikaya{「你們要好好作意;請你們要好好作意」}(sādhukaṃ manasi karotha,逐字譯為「善的(十分的)-意-你們作」),菩提比丘長老英譯為\NoteKeywordBhikkhuBodhi{「仔細地注意」}(attend closely)。按:「作意」(manasikaroti)為「意」與「作」的複合詞,可以是「注意」,也可以有「思惟」的意思。
\stopitemgroup

\startitemgroup[noteitems]
\item\subnoteref{44.0}\NoteKeywordNikayaHead{「善哉」},南傳作\NoteKeywordNikaya{「好;那就好了;那是好的」}(sādhu,感嘆詞),菩提比丘長老英譯為\NoteKeywordBhikkhuBodhi{「好」}(good),或「那就好了!」(it would be good),或「那是好的」(it is good)。
\stopitemgroup

\startitemgroup[noteitems]
\item\subnoteref{45.0}\NoteKeywordNikayaHead{「大德!」}(bhante,呼格),菩提比丘長老英譯為\NoteKeywordBhikkhuBodhi{「值得尊敬的尊長」}(venerable sir)。按:這是對戒臘較高者的稱呼。
\stopitemgroup

\startitemgroup[noteitems]
\item\subnoteref{46.0}\NoteKeywordNikayaHead{「問訊」},南傳作\NoteKeywordNikaya{「問訊,接著……」}(abhivādetvā,連續體,原形abhivādeti,另譯為「敬禮;禮拜」),菩提比丘長老英譯為\NoteKeywordBhikkhuBodhi{「對……表示敬意;行屬臣的禮儀」}(pay homage to)。
\stopitemgroup

\startitemgroup[noteitems]
\item\subnoteref{47.0}\NoteKeywordAgamaHead{「右繞(SA/AA)」},南傳作\NoteKeywordNikaya{「作右繞後」}(padakkhiṇaṃ katvā),菩提比丘長老英譯為\NoteKeywordBhikkhuBodhi{「保持他在他的右邊」}(keeping him on his right)。按:此即佛陀時代印度表示尊敬的「右繞禮」,依順時針方向保持受禮者在自己右邊繞三圈。相反的,「作左繞」(apasabyato karoti, apabyāmato karoti)則表示輕侮。
\stopitemgroup

\startitemgroup[noteitems]
\item\subnoteref{48.0}\NoteKeywordAgamaHead{「信家非家出家/正信非家出家(SA);至信捨家無家(MA);平等信出家(DA);信家非家捨家出家(AA)」},南傳作\NoteKeywordNikaya{「以信從在家出家成為無家者」}(saddhā agārasmā anagāriyaṃ pabbajitā),菩提比丘長老英譯為\NoteKeywordBhikkhuBodhi{「他出於信仰,從有家生活外出(出家)進入無家者」}(who have gone forth from the household life into homelessness out of faith, who has gone forth out of faith from the household life into homelessness)。
\stopitemgroup

\startitemgroup[noteitems]
\item\subnoteref{49.0}\NoteSubKeyHead{(1)}\NoteKeywordNikayaHead{「波旬」}(pāpimato),為魔王的名字,菩提比丘長老英譯為\NoteKeywordBhikkhuBodhi{「魔王」}(the Evil One)。
\item\subnoteref{49.1}\NoteSubKeyHead{(2)}\NoteKeywordNikayaHead{「對他來說被魔之捕網緊綁」}(paṭimukkassa mārapāso),菩提比丘長老英譯為\NoteKeywordBhikkhuBodhi{「魔的陷阱已對他栓緊」}(Māra's snare has been fastened to him)。《顯揚真義》說「以他的脖子被魔的陷阱緊綁、導入」(assa gīvāya mārapāso paṭimukko pavesito, \suttaref{SN.35.114}),《勝義燈》說「……或者,對他緊綁的會是魔的陷阱」(…Atha vā paṭimukko assa bhaveyya mārapāso, \ccchref{It.68}{https://agama.buddhason.org/It/dm.php?keyword=68})。
\stopitemgroup

\startitemgroup[noteitems]
\item\subnoteref{50.0}\NoteKeywordAgamaHead{「以義饒益(SA);得義饒益/利義饒益(MA)」},南傳作\NoteKeywordNikaya{「伴隨利益的」}(atthasaṃhitaṃ),菩提比丘長老英譯為\NoteKeywordBhikkhuBodhi{「有益的」}(beneficial)。「饒益」:令人受益(《漢語詞典》)。
\stopitemgroup

\startitemgroup[noteitems]
\item\subnoteref{51.0}\NoteKeywordNikayaHead{「長夜」},南傳作\NoteKeywordNikaya{「長久」}(dīgharattaṃ,另譯為「長時間」),菩提比丘長老英譯為\NoteKeywordBhikkhuBodhi{「長時間」}(a long time)。按:「ratta」或作「rattā」,有「染色;染著」、「紅的」、「夜間」(一般慣用「ratti」)、「時間」諸多意思。
\stopitemgroup

\startitemgroup[noteitems]
\item\subnoteref{52.0}\NoteKeywordAgamaHead{「斷諸結/轉去諸結/去諸結(SA);諸結已解(MA);諸結便盡/諸結已盡/諸結已除/除此諸結/滅此諸結(AA)」},南傳作\NoteKeywordNikaya{「破壞結」}(vivattayi saṃyojanaṃ),菩提比丘長老英譯為\NoteKeywordBhikkhuBodhi{「切斷拘束」}(severed the fetters)。按:「破壞」(vivattayi,動詞過去式),另譯為「還轉;(輪迴的)還滅」,故北傳有些經文之「斷諸結」也譯作「轉去諸結」。
\stopitemgroup

\startitemgroup[noteitems]
\item\subnoteref{53.0}\NoteKeywordAgamaHead{「無間等(SA);得道跡(AA)」},南傳作\NoteKeywordNikaya{「現觀」}(abhisamayo名詞, abhisameti動詞),菩提比丘長老英譯為\NoteKeywordBhikkhuBodhi{「突破」}(breakthrough, SN),或「穿透;洞察」(penetration/penetrating, MN/AN)。按:「現觀」,經文中都用於證初果或阿羅漢果時的場合。
\stopitemgroup

\startitemgroup[noteitems]
\item\subnoteref{54.0}\NoteKeywordAgamaHead{「究竟苦邊/作苦邊(SA);得苦際/得苦邊(MA);盡苦際(DA/AA);盡苦原際(AA)」},南傳作\NoteKeywordNikaya{「作苦的終結」}(antamakāsi dukkhassā, dukkhassa karonti antaṃ, dukkhassantaṃ karissanti-動詞),菩提比丘長老英譯為\NoteKeywordBhikkhuBodhi{「已結束苦」}(has made an end to suffering),或「苦的作終結者,苦的作終結」(dukkhassantakaro, dukkhassantakiriyaṃ, dukkhassa antakiriyaṃ-名詞),菩提比丘長老英譯為\NoteKeywordBhikkhuBodhi{「結束苦」}(making an end of suffering)。
\stopitemgroup

\startitemgroup[noteitems]
\item\subnoteref{55.0}\NoteKeywordAgamaHead{「一切諸行空寂(SA);諸行寂靜/行盡/諸行悉休息止(AA)」},南傳作\NoteKeywordNikaya{「一切行的止」}(sabbasaṅkhārasamathe),菩提比丘長老英譯為\NoteKeywordBhikkhuBodhi{「所有形成物的靜止」}(the stilling of all formations)。按:《顯揚真義》以「就是一切的熄滅(涅槃)」(sabbaṃ nibbānameva, \suttaref{SN.6.1})解說。「止」(samatha),另音譯為「舍摩他;奢摩他」。
\stopitemgroup

\startitemgroup[noteitems]
\item\subnoteref{56.0}\NoteKeywordAgamaHead{「法眼/導(SA);法之導(GA)」},南傳作\NoteKeywordNikaya{「世尊為導引的」}(bhagavaṃnettikā),菩提比丘長老英譯為\NoteKeywordBhikkhuBodhi{「被幸福者導引」}(guided by the Blessed One)。按:「導引」(nettika),另譯為「作為導引的;當作眼睛的」。
\stopitemgroup

\startitemgroup[noteitems]
\item\subnoteref{57.0}\NoteKeywordNikayaHead{「持;受持」},南傳作\NoteSubEntryKey{(i)}\NoteKeywordNikaya{「憶持」}(dhāreti),菩提比丘長老英譯為\NoteKeywordBhikkhuBodhi{「記得」}(remember)。\NoteSubEntryKey{(ii)}\NoteKeywordNikaya{「受持」}(samādiyati, samādāna, samādāya,另譯為「取;受」),菩提比丘長老英譯為\NoteKeywordBhikkhuBodhi{「承擔」}(undertaken)。按:名詞「dhāraṇa」水野弘元《巴利語辭典》也譯作「受持;憶持」。
\stopitemgroup

\startitemgroup[noteitems]
\item\subnoteref{58.0}\NoteSubKeyHead{(1)}\NoteKeywordAgamaHead{「法、次法向/向法、次法/行隨順法(SA);所行次第不越限度(GA);向法、次法/趣法、向法/順法、次法(MA);法法成就(DA);法法相明(AA)」},南傳作\NoteKeywordNikaya{「法、隨法行的(者)」}(dhammānudhammappaṭipanno,另譯為「法、隨法之行道;法、隨法之實踐」),菩提比丘長老英譯為\NoteKeywordBhikkhuBodhi{「依照法實行者」}(who is practising in accordance with the Dhamma)。按:《破斥猶豫》以「法的隨法能相應的道跡」(dhammassa anudhammaṃ anucchavikaṃ paṭipadaṃ, \ccchref{MN.75}{https://agama.buddhason.org/MN/dm.php?keyword=75})解說「法隨法」。《小義釋》說,四念住……八支聖道被稱為法;完成戒的、在諸根上守護門的、在飲食上知適量的、專修清醒的、念與正知,這些被稱為隨法。
\item\subnoteref{58.1}\NoteSubKeyHead{(2)}\NoteKeywordNikayaHead{「隨順法;次法」},南傳作\NoteKeywordNikaya{「隨法」}(anudhammo,另譯為「如法;依照法;符合法」),菩提比丘長老英譯為\NoteKeywordBhikkhuBodhi{「依照法」}(in accordance with the Dhamma, accords with the Dhamma)。
\stopitemgroup

\startitemgroup[noteitems]
\item\subnoteref{59.0}\NoteSubKeyHead{(1)}\NoteKeywordAgamaHead{「觀察/觀(SA)」},南傳作\NoteSubEntryKey{(i)}\NoteKeywordNikaya{「隨看著」}(anupassī, anupassino,另譯為「觀察」,形容詞,但以如現在分詞的動作形容詞解讀),菩提比丘長老英譯為\NoteKeywordBhikkhuBodhi{「凝視;熟視;注視」}(contemplating)。\NoteSubEntryKey{(ii)}\NoteKeywordNikaya{「隨看」}(anupassanā,名詞),菩提比丘長老英譯為\NoteKeywordBhikkhuBodhi{「凝視;熟視;注視」}(contemplation)。
\item\subnoteref{59.1}\NoteSubKeyHead{(2)}\NoteKeywordAgamaHead{「觀無常/觀無常(SA)」},南傳作\NoteKeywordNikaya{「隨看著無常」}(aniccānupassī),菩提比丘長老英譯為\NoteKeywordBhikkhuBodhi{「凝視無常」}(contemplating impermanence)。按:《清淨道論》說,無常指五蘊,為什麼呢?生起、消散、變異性(Uppādavayaññathattabhāvā),……隨看著無常指具備那種隨看者(tāya anupassanāya samannāgato, 8.236)。
\stopitemgroup

\startitemgroup[noteitems]
\item\subnoteref{60.0}\NoteSubKeyHead{(1)}\NoteKeywordNikayaHead{「長者;居士」},南傳作\NoteSubEntryKey{(i)}\NoteKeywordNikaya{「經營錢莊的屋主」}(seṭṭhi gahapati),菩提比丘長老英譯為\NoteKeywordBhikkhuBodhi{「金融業屋主;財物官屋主」}(a financier householder),並解說「seṭṭhi」是指北印度大城市中有錢的放貸款的人,一開始是公會會長(guild masters),隨時間的演變,形成私人銀行功能,並在政治上扮演決定性角色,「給孤獨(那邠祁)」就被稱為「seṭṭhi」。\NoteSubEntryKey{(ii)}\NoteKeywordNikaya{「屋主」}(gahapati,另譯為「家主、居士」),菩提比丘長老英譯為\NoteKeywordBhikkhuBodhi{「房子的所有權人;屋主;戶長」}(householder)。
\item\subnoteref{60.1}\NoteSubKeyHead{(2)}\NoteKeywordAgamaHead{「長者眾(SA/AA);居士眾(GA/MA/DA/AA)」},南傳作\NoteKeywordNikaya{「屋主眾」}(gahapatiparisā),菩提比丘長老英譯為\NoteKeywordBhikkhuBodhi{「屋主團體」}(assembly of householders)。
\stopitemgroup

\startitemgroup[noteitems]
\item\subnoteref{61.0}\NoteSubKeyHead{(1)}\NoteKeywordNikayaHead{「遊行」}(cārikaṃ,另譯為「旅行;徘徊」),菩提比丘長老英譯為\NoteKeywordBhikkhuBodhi{「在旅程中;在遊歷中」}(on tour)。
\item\subnoteref{61.1}\NoteSubKeyHead{(2)}\NoteKeywordNikayaHead{「次第地進行著遊行」}(anupubbena cārikaṃ caramāno),菩提比丘長老英譯為\NoteKeywordBhikkhuBodhi{「逐步遊走著;遊走在旅程中」}(wandering by stages; wandering on tour)。
\stopitemgroup

\startitemgroup[noteitems]
\item\subnoteref{62.0}\NoteKeywordAgamaHead{「遠塵、離垢得法眼淨(SA);遠塵、離垢諸法法眼生(MA/DA);諸塵垢盡得法眼淨(AA)」},南傳作\NoteKeywordNikaya{「遠塵、離垢之法眼」}(virajaṃ vītamalaṃ dhammacakkhuṃ),菩提比丘長老英譯為\NoteKeywordBhikkhuBodhi{「屬於正法,無塵、無瑕的眼光」}(the dust-free, stainless vision of the Dhamma)。由南傳的經文來看,漢譯「法眼淨」的「淨」,是「遠塵、離垢」的同義詞。「法眼淨」的具體內容,\suttaref{SN.35.74}等均作「凡任何集法都是滅法」,\ccchref{AA.38.7}{https://agama.buddhason.org/AA/dm.php?keyword=38.7}的「諸可習法盡是滅法」與之相當。「法眼淨」在經文中都用於證初果者。
\stopitemgroup

\startitemgroup[noteitems]
\item\subnoteref{63.0}\NoteKeywordAgamaHead{「見法、得法、知法、入法/見法、得法、知法、起法/見法、得法、覺法/見法、得法,入法、解法(SA);見法、證法(GA);見法、得法,覺白淨法(MA);見法、得法/見法(DA);已得法、見法、分別其法/得法、見法(AA)」},南傳作\NoteKeywordNikaya{「已見法、已獲得法、已知法、已深入法」}(diṭṭhadhammo pattadhammo viditadhammo pariyogāḷhadhammo),菩提比丘長老英譯為\NoteKeywordBhikkhuBodhi{「成為已見法、已到達法、已理解法、已徹底明白法者」}(became one who had seen the Dhamma, attained the Dhamma, understood the Dhamma, fathomed the Dhamma, \ccchref{AN.8.12}{https://agama.buddhason.org/AN/an.php?keyword=8.12}),並解說「這一連串分詞」是對證得「法眼」(dhammacakkhu)者的標準形容,而這個法即是證初果者所見的緣起(\suttaref{SN.12.33}),或四聖諦(\ccchref{MN.56}{https://agama.buddhason.org/MN/dm.php?keyword=56})。
\stopitemgroup

\startitemgroup[noteitems]
\item\subnoteref{64.0}\NoteKeywordAgamaHead{「盡壽歸依/盡形壽歸依(SA);終身自歸(MA)」},南傳作\NoteKeywordNikaya{「已終生歸依」}(pāṇupetaṃ saraṇaṃ gatanti),菩提比丘長老英譯為\NoteKeywordBhikkhuBodhi{「前往終生依靠」}(who has gone to refuge for life)。「盡形壽」另作「盡其形壽;盡形」,「終生」的意思。
\stopitemgroup

\startitemgroup[noteitems]
\item\subnoteref{65.0}\NoteKeywordAgamaHead{「歸依僧(SA/DA);自歸於比丘眾(SA);歸比丘眾(MA);歸比丘僧(SA/MA);歸依僧寶/自歸眾/自歸命聖眾(AA)」},南傳作\NoteKeywordNikaya{「歸依比丘僧團」}(saraṇaṃ gacchāmi…bhikkhusaṅghañca),菩提比丘長老英譯為\NoteKeywordBhikkhuBodhi{「我前往依靠僧團」}(I go for refuge to saṅgha)。「僧;眾」實為「僧伽」(saṅgha)的簡略,意譯為「眾;和合眾」,指的是「團體」,而非任何「個人」。
\stopitemgroup

\startitemgroup[noteitems]
\item\subnoteref{66.0}\NoteKeywordAgamaHead{「具足住(SA);成就遊(MA);自娛樂/自遊化(DA/AA);自遊戲/得娛樂/自遊樂(AA)」},南傳作\NoteKeywordNikaya{「進入後住於;進入後而住」}(upasampajja viharanti,逐字譯為「具足住」),菩提比丘長老英譯為\NoteKeywordBhikkhuBodhi{「進入及住在」}(enter and dwell in)。
\stopitemgroup

\startitemgroup[noteitems]
\item\subnoteref{67.0}\NoteSubKeyHead{(1)}\NoteKeywordAgamaHead{「集(SA/DA);習(SA/MA);習/習起(AA)」},南傳作\NoteKeywordNikaya{「集;集起」}(samudayaṃ,動詞samudayati),菩提比丘長老英譯為\NoteKeywordBhikkhuBodhi{「起源,起來;出現」}(origin, origination/動詞originate, arising)。
\item\subnoteref{67.1}\NoteSubKeyHead{(2)}\NoteKeywordAgamaHead{「集法(SA);習法/有習之法(AA)」},南傳作\NoteKeywordNikaya{「集法」}(samudayadhammo),菩提比丘長老英譯為\NoteKeywordBhikkhuBodhi{「屬於有起源者」}(subject to origination),或「起源性質」(the nature of origination, \suttaref{SN.47.40})。這裡的「法」不是指「正法」。
\stopitemgroup

\startitemgroup[noteitems]
\item\subnoteref{68.0}\NoteSubKeyHead{(1)}\NoteKeywordAgamaHead{「滅;盡(SA/DA/AA);滅盡(AA)」},南傳作\NoteKeywordNikaya{「滅」}(nirodhaṃ),菩提比丘長老英譯為\NoteKeywordBhikkhuBodhi{「停止」}(cessation)。
\item\subnoteref{68.1}\NoteSubKeyHead{(2)}\NoteKeywordAgamaHead{「滅法(SA/MA/DA);盡法/磨滅[法](AA)」},南傳作\NoteKeywordNikaya{「滅法」}(nirodhadhammaṃ),菩提比丘長老英譯為\NoteKeywordBhikkhuBodhi{「屬於停止者」}(subject to cessation)。這裡的「法」不是指「正法」。
\stopitemgroup

\startitemgroup[noteitems]
\item\subnoteref{69.0}\NoteSubKeyHead{(1)}\NoteKeywordAgamaHead{「滅道跡(SA);滅道(MA);出要/盡道(DA)」},南傳作\NoteKeywordNikaya{「導向滅道跡」}(nirodhagāminiṃ paṭipadaṃ),菩提比丘長老英譯為\NoteKeywordBhikkhuBodhi{「導向其停止的路」}(the way leading to its cessation)。
\item\subnoteref{69.1}\NoteSubKeyHead{(2)}\NoteKeywordNikayaHead{「道」}(maggaṃ),菩提比丘長老英譯為\NoteKeywordBhikkhuBodhi{「路徑」}(the path)。
\stopitemgroup

\startitemgroup[noteitems]
\item\subnoteref{70.0}\NoteKeywordAgamaHead{「變易法(SA);變易之法(MA)」},南傳作\NoteKeywordNikaya{「變易法/變易法的狀態」}(vipariṇāmadhammaṃ/vipariṇāmadhammatā),菩提比丘長老英譯為\NoteKeywordBhikkhuBodhi{「屬於會改變的」}(subject to change),此處的「法」(dhamma),不是指「正法」。
\stopitemgroup

\startitemgroup[noteitems]
\item\subnoteref{71.0}\NoteKeywordAgamaHead{「自覺涅槃/自得涅槃(SA);獨到涅槃(DA)」},南傳作\NoteKeywordNikaya{「就自己證涅槃」}(paccattaññeva parinibbāyati),菩提比丘長老英譯為\NoteKeywordBhikkhuBodhi{「自己到達涅槃」}(personally attains nibbāna)。按:「就自己」(paccattaññeva),另作「各自的,分開的」,《顯揚真義》等以「就自己」(sayameva attanāva, \suttaref{SN.12.51}/\ccchref{DN.1}{https://agama.buddhason.org/DN/dm.php?keyword=1}),「非以其他的勢力」(na aññassa ānubhāvena, \suttaref{SN.12.51})解說,今準此譯,「證涅槃」(parinibbāyati,動詞),音譯為「般涅槃」,另譯為「完成;圓寂;被遍熄滅(消盡、消失);變遍寂靜(寂滅)」。
\stopitemgroup

\startitemgroup[noteitems]
\item\subnoteref{72.0}\NoteSubKeyHead{(1)}\NoteKeywordAgamaHead{「般涅槃;得涅槃(GA);般泥洹(DA)」},南傳作\NoteKeywordNikaya{「般涅槃」}(parinibbānaṃ,名詞,另譯為「圓寂;完全涅槃;遍涅槃」),菩提比丘長老英譯為\NoteKeywordBhikkhuBodhi{「最後的涅槃」}(final nibbāna)。
\item\subnoteref{72.1}\NoteSubKeyHead{(2)}\NoteKeywordAgamaHead{「般涅槃;入涅槃(GA)」},南傳作\NoteKeywordNikaya{「般涅槃」}(parinibbāyati, parinibbāti,動詞),菩提比丘長老英譯為\NoteKeywordBhikkhuBodhi{「達到最後的涅槃」}(attained final nibbāna)。
\stopitemgroup

\startitemgroup[noteitems]
\item\subnoteref{73.0}\NoteKeywordNikayaHead{「心、意、識」}(cittaṃ itipi, mano itipi, viññāṇaṃ itipi),菩提比丘長老英譯為\NoteKeywordBhikkhuBodhi{「『心』與『心理』與『識』」}(‘mind’ and ‘mentality’ and ‘consciousness’)。按:《顯揚真義》說,這三者「全都名為意處」(sabbaṃ manāyatanasseva nāmaṃ),但長老認為在尼科耶(nikāya)中通常有顯著的不同使用,如「識」用於「眼識」等六識,以及從此生流轉到來生相續的潛流(underlying stream);「意」用於「身口意」三行,以及認識「法」(心理現象,mental phenomena)的意根;心是經驗的中心(the center of personal experience),為心思、意志、情感的主體(as the subject of thought, volition and emotion),是這個心需要被了解、修練與解脫的(it is citta that needs to be understood, trained and liberated)。
\stopitemgroup

\startitemgroup[noteitems]
\item\subnoteref{74.0}\NoteKeywordAgamaHead{「愚癡無聞凡夫/愚夫(SA);凡夫愚人(MA);愚癡凡夫(AA)」},南傳作\NoteKeywordNikaya{「未聽聞的一般人」}(assutavā puthujjano,可逐字譯為「無聞-凡夫」),菩提比丘長老英譯為\NoteKeywordBhikkhuBodhi{「未受教導的俗人」}(the uninstructed worldling)。
\stopitemgroup

\startitemgroup[noteitems]
\item\subnoteref{75.0}\NoteKeywordAgamaHead{「成正覺/成佛(SA);覺(MA);成佛道/成正覺(AA)」},南傳作\NoteKeywordNikaya{「(已)現正覺」}(abhisambuddho),菩提比丘長老英譯為\NoteKeywordBhikkhuBodhi{「完全開化」}(fully enlightened),或「醒悟」(awakened)。按:這是「對;向」(abhi)與「正覺」(sambuddha)的複合詞,表示「正覺的體證」。
\stopitemgroup

\startitemgroup[noteitems]
\item\subnoteref{76.0}\NoteKeywordAgamaHead{「真人法(MA);善知識法(AA)」},南傳作\NoteKeywordNikaya{「善人法」}(sappurisadhamme),菩提比丘長老英譯為\NoteKeywordBhikkhuBodhi{「上等人的法」}(superior persons' Dhamma, \suttaref{SN.22.43}),智髻比丘長老英譯為「真人的品格」(the character of a true man, \ccchref{MN.113}{https://agama.buddhason.org/MN/dm.php?keyword=113})。「善人」(sappurisa),逐字譯為「善-男子」,另譯為「善士;正士」。「非善人」,《破斥猶豫》以「惡人」(pāpapuriso, \ccchref{MN.110}{https://agama.buddhason.org/MN/dm.php?keyword=110})解說。
\stopitemgroup

\startitemgroup[noteitems]
\item\subnoteref{77.0}\NoteSubKeyHead{(1)}\NoteKeywordAgamaHead{「離欲(SA)」},南傳作\NoteKeywordNikaya{「離貪;褪去」}(virāgo),菩提比丘長老英譯為\NoteKeywordBhikkhuBodhi{「冷靜」}(dispassion),或「褪去」(fading away)。按:virāga,vi-為接頭詞,「離;別;異;反」的意思,rāga有兩個意思:一為「貪瞋癡」的「貪」,二為「色彩」,所以virāga可以解讀為「離貪」,也可解讀為如色彩的「褪去」。在「厭者離染」的場合,《破斥猶豫》以「這裡,離貪為[聖]道」(ettha virāgoti maggo)解說;「經由離貪而解脫」則以「由離貪的[聖]道而說解脫果」(ettha virāgena maggena vimuccatīti phalaṃ kathitaṃ, \ccchref{MN.22}{https://agama.buddhason.org/MN/dm.php?keyword=22})解說。
\item\subnoteref{77.1}\NoteSubKeyHead{(2)}\NoteKeywordNikayaHead{「不離貪」}(avirājayaṃ),菩提比丘長老英譯為\NoteKeywordBhikkhuBodhi{「對之不成為冷靜」}(without becoming dispassionate towards it)。按:《顯揚真義》以「不離去」(avigacchāpento, \suttaref{SN.35.26})解說。
\stopitemgroup

\startitemgroup[noteitems]
\item\subnoteref{78.0}\NoteKeywordAgamaHead{「六觸入處(SA);六更觸/六更觸處/六觸處/六更樂處(MA);六細滑更樂入(AA)」},南傳作\NoteKeywordNikaya{「六觸處」}(channaṃ phassāyatanānaṃ),菩提比丘長老英譯為\NoteKeywordBhikkhuBodhi{「六個適於觸的基地」}(the six bases for contact)。按:「處」(āyatana),另譯為「入處」,這裡的「觸處」指的就是(適合)生起「觸」(phassa)的「六處」(saḷāyatana),也就是「眼、耳、鼻、舌、身、意」六根,\ccchref{AN.3.62}{https://agama.buddhason.org/AN/an.php?keyword=3.62}就說是「眼觸處、耳觸處、……意觸處」這六個,《吉祥悅意》亦然(cakkhuphassāyatanaṃ, sotaphassāyatanaṃ, ……manophassāyatananti imāni cha, \ccchref{DN.1}{https://agama.buddhason.org/DN/dm.php?keyword=1})。
\stopitemgroup

\startitemgroup[noteitems]
\item\subnoteref{79.0}\NoteKeywordAgamaHead{「外道出家(SA);異學(MA);外道異眾/外道異學/異眾(DA);外道/外道異學(AA)」},南傳作\NoteKeywordNikaya{「其他外道遊行者」}(aññatitthiyā paribbājakā),菩提比丘長老英譯為\NoteKeywordBhikkhuBodhi{「其它教派流浪者」}(wanderers of other sects)。按:「外道」(titthiyā,另譯為「外學;異學;外教徒」),「遊行者」(paribbājakā,另譯為「遍歷者;普行者;流浪者;梵志」),指的就是「出家人」。
\stopitemgroup

\startitemgroup[noteitems]
\item\subnoteref{80.0}\NoteKeywordNikayaHead{「瞿曇」},南傳作\NoteKeywordNikaya{「喬達摩」}(gotama, go-tama),菩提比丘長老英譯照錄原文。按:「喬達摩」字面意思是「黑(tama)牛(go)」,為釋迦牟尼佛、尊者阿難家族的姓。
\stopitemgroup

\startitemgroup[noteitems]
\item\subnoteref{81.0}\NoteKeywordAgamaHead{「善義(SA);其義深遠/句義微妙滿足(GA);有義(MA/AA);義具足/義清淨/義深奧/義諦誠(DA);義理深邃(AA)」},南傳作\NoteKeywordNikaya{「有意義的」}(sātthaṃ),菩提比丘長老英譯為\NoteKeywordBhikkhuBodhi{「具正確意義」}(with the right meaning)。
\stopitemgroup

\startitemgroup[noteitems]
\item\subnoteref{82.0}\NoteSubKeyHead{(1)}\NoteKeywordAgamaHead{「善味/善句味(SA);其語巧妙(GA);有文(MA);味具足/味清淨/味深奧/味諦誠(DA);有味(AA)」},南傳作\NoteKeywordNikaya{「有文字的」}(sabyañjanaṃ),菩提比丘長老英譯為\NoteKeywordBhikkhuBodhi{「具正確詞句」}(with the right phrasing)。按:「句味/味」即「辭句;文」(byañjana)的另譯。
\item\subnoteref{82.1}\NoteSubKeyHead{(2)}\NoteKeywordAgamaHead{「異句、異味(SA);異文、異句(MA)」},南傳作\NoteKeywordNikaya{「以各種語詞、各種法門」}(aññamaññehi padehi aññamaññehi pariyāyehi),菩提比丘長老英譯為\NoteKeywordBhikkhuBodhi{「以種種詞與以種種方法」}(with various terms and with various methods)。
\stopitemgroup

\startitemgroup[noteitems]
\item\subnoteref{83.0}\NoteKeywordAgamaHead{「苦聚(SA);苦陰(MA/DA);苦盛陰(AA)」},南傳作\NoteKeywordNikaya{「苦蘊」}(dukkhakkhandhassa),菩提比丘長老英譯為\NoteKeywordBhikkhuBodhi{「苦團」}(mass of suffering)。這是「苦」(dukkha)與「蘊」(khandha)的複合詞,「蘊」(khandha)在漢譯的經文中也譯為「陰」(如「五陰、五蘊」),或音譯為「犍度」,菩提比丘長老英譯為\NoteKeywordBhikkhuBodhi{「聚集;集合」}(aggregate),或「一團;龐大」(mass)。
\stopitemgroup

\startitemgroup[noteitems]
\item\subnoteref{84.0}\NoteKeywordAgamaHead{「知時(SA/MA);宜知是時/知是時(GA/DA/AA);今正是時(DA/AA)」},南傳作\NoteKeywordNikaya{「你考量的時間」}(kālaṃ maññasī,古譯為「知是時;宜知是時;知時」),菩提比丘長老英譯為\NoteKeywordBhikkhuBodhi{「你請便;你可以你自己的方便走」}(you may go at your own convenience),並解說這是表示離開的定型句,此處的翻譯,為適合上下文略作變更(I have varied the rendering slightly to fit the context. \suttaref{SN.11.18})。
\stopitemgroup

\startitemgroup[noteitems]
\item\subnoteref{85.0}\NoteSubKeyHead{(1)}\NoteKeywordNikayaHead{「隨喜,感謝」}(anumodati,動詞),菩提比丘長老英譯為\NoteKeywordBhikkhuBodhi{「歡喜」}(rejoiced),或「給祝福」(give the blessing, \ccchref{MN.111}{https://agama.buddhason.org/MN/dm.php?keyword=111})。
\item\subnoteref{85.1}\NoteSubKeyHead{(2)}\NoteKeywordAgamaHead{「咒願(MA)」},南傳作\NoteKeywordNikaya{「感謝」}(anumodana,名詞),智髻比丘長老英譯為「祝福」(the blessing, \ccchref{MN.91}{https://agama.buddhason.org/MN/dm.php?keyword=91}),菩提比丘長解說,這是餐後的簡短談話,以某個法的主題教導施主,並表示希望施主的福德業報將帶來大果報。
\stopitemgroup

\startitemgroup[noteitems]
\item\subnoteref{86.0}\NoteKeywordAgamaHead{「示、教、照、喜(SA);示、教、利、喜(SA/GA/DA/AA)」},南傳作\NoteKeywordNikaya{「開示、勸導、鼓勵、使歡喜」}(sandasseti samādapeti samuttejeti sampahaṃseti),菩提比丘長老英譯為\NoteKeywordBhikkhuBodhi{「教導、鼓勵、啟發、使喜悅」}(instructed, encouraged, inspired, and gladdened. AN)。玄奘法師譯作「示現、教導、讚勵、慶喜」。
\stopitemgroup

\startitemgroup[noteitems]
\item\subnoteref{87.0}\NoteKeywordAgamaHead{「乞食;分衛(AA)」},南傳作\NoteKeywordNikaya{「為了托鉢」}(piṇḍāya,原意為「為了團食(被握成一團的食物-意即施食)」),菩提比丘長老英譯為\NoteKeywordBhikkhuBodhi{「為了施捨;為了捐獻」}(for alms)。
\stopitemgroup

\startitemgroup[noteitems]
\item\subnoteref{88.0}\NoteKeywordAgamaHead{「安陀林(SA);得眼林(GA);晝闇園(AA)」},南傳作\NoteKeywordNikaya{「盲者的樹林」}(andhavanaṃ),菩提比丘長老英譯為\NoteKeywordBhikkhuBodhi{「盲人男子的樹林」}(the Blind Men's Grove)。按:《顯揚真義》說,有位名叫yasodhara的說法聖人(dhammabhāṇakassa ariyapuggalassa)為了修建迦葉遍正覺者的塔廟,勸發財物後到此,眼睛[被五百位盜賊]毀壞。就在那裡,那五百位盜賊眼睛達到破壞(akkhibhedappattehi)而住下來,此後[該]樹林被稱為「盲者的樹林」,它在舍衛城南邊1伽浮他遠(gāvutamatte≈1/4由旬≈3.5公里)處,有國王的守衛保護,想要獨居的比丘、比丘尼前往那裡,因此,這是欲求身遠離者(kāyavivekatthinī)前往的樹林(\suttaref{SN.5.1})。
\stopitemgroup

\startitemgroup[noteitems]
\item\subnoteref{89.0}\NoteKeywordNikayaHead{「無明觸」}(avijjāsamphassa),菩提比丘長老英譯為\NoteKeywordBhikkhuBodhi{「無明-接觸」}(ignorance-contact)。按:這是指受無明支配的觸,反之,就是「明觸」。
\stopitemgroup

\startitemgroup[noteitems]
\item\subnoteref{90.0}\NoteSubKeyHead{(1)}\NoteKeywordAgamaHead{「有為;有為法;壞有(\ccchref{SA.64}{https://agama.buddhason.org/SA/dm.php?keyword=64})」},南傳作\NoteKeywordNikaya{「有為(的)」}(saṅkhatā),菩提比丘長老英譯為\NoteKeywordBhikkhuBodhi{「有條件的;為條件所支配的」}(conditioned)。按:《顯揚真義》以「以緣會合後所作的」(paccayehi samāgantvā kataṃ, \suttaref{SN.12.20})解說。
\item\subnoteref{90.1}\NoteSubKeyHead{(2)}\NoteKeywordNikayaHead{「無為;無為法」},南傳作\NoteKeywordNikaya{「無為(的)」}(asaṅkhataṃ),菩提比丘長老英譯為\NoteKeywordBhikkhuBodhi{「無條件的」}(the unconditioned)。按:「無條件的」顯然指「無產生來生」的條件,亦即涅槃,《顯揚真義》以「不被做的;不造作的」(akataṃ, \suttaref{SN.43.1})解說。
\item\subnoteref{90.2}\NoteSubKeyHead{(3)}\NoteKeywordAgamaHead{「無為處(SA/AA);泥洹(DA/AA);無為之處/無為之境/無為之地/泥曰(AA)」},為「涅槃」(nibbāna)的另譯。另外,「寂滅、寂靜、清涼、無極、無為、滅界、息跡滅度」皆為其同義詞。
\stopitemgroup

\startitemgroup[noteitems]
\item\subnoteref{91.0}\NoteKeywordAgamaHead{「晡時(SA/MA);下晡/晡(MA)」},南傳作\NoteKeywordNikaya{「在傍晚時」}(sāyanhasamayaṃ, sāyaṇhasamayaṃ,另譯為「晡時;黃昏時」),菩提比丘長老英譯為\NoteKeywordBhikkhuBodhi{「在黃昏」}(in the evening)。按:「晡」是指午後三點到五點,即「申時」。
\stopitemgroup

\startitemgroup[noteitems]
\item\subnoteref{92.0}\NoteKeywordAgamaHead{「獨靜禪思/禪思(SA);獨坐/獨坐思惟(GA);宴坐/燕坐(MA);三昧思惟/禪靜(DA);禪思(AA)」},南傳作\NoteKeywordNikaya{「獨坐」}(paṭisallāṇa, 已獨坐-paṭisallīna,另譯為「宴坐、宴默、燕坐、禪思」),菩提比丘長老英譯為\NoteKeywordBhikkhuBodhi{「隱遁」}(seclusion)。按:《顯揚真義》以「隱遁、成為單獨」(nilīnassa ekībhūtassa, \suttaref{SN.3.4}),《破斥猶豫》以「心從色等行境(rūpādigocarato)撤去後退縮(paṭisaṃharitvā līno),因喜樂禪定之實行而(jhānaratisevanavasena)到達獨處(ekībhāvaṃ gato, \ccchref{MN.78}{https://agama.buddhason.org/MN/dm.php?keyword=78})」,《吉祥悅意》以「從種種所緣行(nānārammaṇacārato)返回後隱退、隱遁(paṭikkamma sallīno nilīno),走進獨處後在單一所緣中經驗禪定的喜樂(jhānaratiṃ, \ccchref{DN.6}{https://agama.buddhason.org/DN/dm.php?keyword=6})」,《滿足希求》以「隱遁、獨處」(nilīyanassa ekībhāvassa, \ccchref{AN.7.68}{https://agama.buddhason.org/AN/an.php?keyword=7.68})解說。
\stopitemgroup

\startitemgroup[noteitems]
\item\subnoteref{93.0}\NoteSubKeyHead{(1)}\NoteKeywordAgamaHead{「有身(SA);自己有/自身(MA)」},南傳作\NoteKeywordNikaya{「有身」}(sakkāya),另譯為「己身;常住身」,音譯為「薩迦耶」。
\item\subnoteref{93.1}\NoteSubKeyHead{(2)}\NoteKeywordAgamaHead{「身見(SA/MA/DA);自身見(MA);身邪(AA)」},南傳作\NoteKeywordNikaya{「有身見」}(sakkāyadiṭṭhi,簡為「身見」,音譯為「薩迦耶見」),菩提比丘長老英譯為\NoteKeywordBhikkhuBodhi{「辨識(我)的見解」}(identity view)。「有身見」,\ccchref{SA.57}{https://agama.buddhason.org/SA/dm.php?keyword=57}、\ccchref{SA.109}{https://agama.buddhason.org/SA/dm.php?keyword=109}等列二十種。
\stopitemgroup

\startitemgroup[noteitems]
\item\subnoteref{94.0}\NoteSubKeyHead{(1)}\NoteKeywordNikayaHead{「修習」}(bhāveti,原意為「使有;使存在」,名詞bhāvanā),菩提比丘長老英譯為\NoteKeywordBhikkhuBodhi{「開發;發展」}(develops, 名詞development),或「默想的開發;禪修」(meditative development, \ccchref{AN.8.36}{https://agama.buddhason.org/AN/an.php?keyword=8.36})。按:《顯揚真義》等以「使增大(vaḍḍheti-培育),使自己的心擴大(cittasantāne)一再生起,使生起(abhinibbattetīti, \suttaref{SN.3.18}/\ccchref{MN.2}{https://agama.buddhason.org/MN/dm.php?keyword=2})」,《滿足希求》以「使生起,使增大」(uppādeti vaḍḍheti, \ccchref{AN.1.54}{https://agama.buddhason.org/AN/an.php?keyword=1.54})解說。
\item\subnoteref{94.1}\NoteSubKeyHead{(2)}\NoteKeywordNikayaHead{「修斷」},南傳作\NoteKeywordNikaya{「修習的勤奮」}(bhāvanāppadhānaṃ),菩提比丘長老英譯為\NoteKeywordBhikkhuBodhi{「以開發而努力」}(Striving by development)。
\item\subnoteref{94.2}\NoteSubKeyHead{(3)}\NoteKeywordAgamaHead{「修力(SA)」},南傳作\NoteKeywordNikaya{「修習力」}(bhāvanābalaṃ),菩提比丘長老英譯為\NoteKeywordBhikkhuBodhi{「開發的力量」}(the power of development)。
\stopitemgroup

\startitemgroup[noteitems]
\item\subnoteref{95.0}\NoteKeywordAgamaHead{「多修習(SA/MA);多修習行(DA);廣布(AA)」},南傳作\NoteKeywordNikaya{「多作(bahulīkaroti,名詞bahulīkamma, bahulīkaraṇa, bahulīkāra),菩提比丘長老英譯為「鍛鍊」}(cultivates)。按:《顯揚真義》等以「被一再作的」(punappunaṃ katā, \suttaref{SN.51.10}/\ccchref{DN.16}{https://agama.buddhason.org/DN/dm.php?keyword=16}/\ccchref{AN.1.608}{https://agama.buddhason.org/AN/an.php?keyword=1.608})解說。
\stopitemgroup

\startitemgroup[noteitems]
\item\subnoteref{96.0}\NoteSubKeyHead{(1)}\NoteKeywordAgamaHead{「當來有愛(SA);當來有樂欲/未來有愛/此愛當受未來有(MA)」},南傳作\NoteKeywordNikaya{「導致再有的渴愛」}(taṇhā ponobhavikā),菩提比丘長老英譯為\NoteKeywordBhikkhuBodhi{「渴愛-這導向重新存在的」}(craving which leads to renewed existence)。
\item\subnoteref{96.1}\NoteSubKeyHead{(2)}\NoteKeywordAgamaHead{「貪喜俱(SA);喜欲共俱/與喜欲俱(MA)」},南傳作\NoteKeywordNikaya{「與歡喜及貪俱行的」}(nandirāgasahagatā, nandīrāgasahagatā),菩提比丘長老英譯為\NoteKeywordBhikkhuBodhi{「由歡樂與慾望陪同」}(accompanied by delight and lust)。
\item\subnoteref{96.2}\NoteSubKeyHead{(3)}\NoteKeywordAgamaHead{「彼彼樂著(SA);共俱求彼彼有/愛樂彼彼有起/願彼彼有(MA)」},南傳作\NoteKeywordNikaya{「到處歡喜的」}(tatratatrābhinandinī,逐字譯為「彼+彼+歡喜(尋歡)」),菩提比丘長老英譯為\NoteKeywordBhikkhuBodhi{「到處尋歡樂」}(seeking delight here and there)。按:《顯揚真義》以「再有生起的」(punabbhavanibbattikā)解說「導致再有的」,以「達到一起一致的」(saha ekattameva gatā)解說「與…俱行的」,以「在再生處或色等所緣處」(upapattiṭṭhāne vā rūpādīsu vā ārammaṇesu, \suttaref{SN.22.22}),《吉祥悅意》則以「個人(自體)」(attabhāvo, \ccchref{DN.22}{https://agama.buddhason.org/DN/dm.php?keyword=22})解說「到處」。
\stopitemgroup

\startitemgroup[noteitems]
\item\subnoteref{97.0}\NoteKeywordAgamaHead{「慰勞(SA);慶慰(MA)」},南傳作\NoteKeywordNikaya{「歡迎與寒暄後」}(sammodanīyaṃ kathaṃ sāraṇīyaṃ vītisāretvā,逐字譯為「寒暄-談論-禮貌(熱誠)-交換」),菩提比丘長老英譯為\NoteKeywordBhikkhuBodhi{「當他們結束致意與熱誠的交談後」}(when they had concluded their greetings and cordial talk)。
\stopitemgroup

\startitemgroup[noteitems]
\item\subnoteref{98.0}\NoteKeywordAgamaHead{「優婆塞(SA/MA/DA);優婆塞/清信士(AA)」},南傳作\NoteKeywordNikaya{「優婆塞」}(upāsaka),菩提比丘長老英譯為\NoteKeywordBhikkhuBodhi{「男性俗人信奉者」}(male lay follower),也就是「男性在家佛弟子」。
\stopitemgroup

\startitemgroup[noteitems]
\item\subnoteref{99.0}\NoteKeywordAgamaHead{「優婆夷(SA/DA);優婆夷/優婆私(MA);優婆夷/優婆斯/清信女(AA)」},南傳作\NoteKeywordNikaya{「優婆夷」}(upāsika),菩提比丘長老英譯為\NoteKeywordBhikkhuBodhi{「女性俗人信奉者」}(female lay follower),也就是「女性在家佛弟子」。
\stopitemgroup

\startitemgroup[noteitems]
\item\subnoteref{100.0}\NoteKeywordAgamaHead{「師長/師/阿闍梨/阿梨(SA);阿闍梨(GA);師(MA/AA);師長(DA)」},南傳作\NoteKeywordNikaya{「老師」}(ācariya,另譯為「軌範師」),菩提比丘長老英譯為\NoteKeywordBhikkhuBodhi{「老師」}(teacher)。
\stopitemgroup

\startitemgroup[noteitems]
\item\subnoteref{101.0}\NoteKeywordNikayaHead{「福田」}(puññakkhettaṃ),菩提比丘長老英譯為\NoteKeywordBhikkhuBodhi{「功績田地」}(field of merit)。按:此即指被供養者,如植福的田地,《滿足希求》以「福的生長處」(puññaviruhanaṭṭhānaṃ, \ccchref{AN.3.97}{https://agama.buddhason.org/AN/an.php?keyword=3.97})解說。
\stopitemgroup

\startitemgroup[noteitems]
\item\subnoteref{102.0}\NoteKeywordAgamaHead{「年少婆羅門(SA);摩納(GA/MA/DA/AA);摩納磨(MA)」},南傳作\NoteKeywordNikaya{「學生婆羅門」}(māṇava,另譯為「學童;青年;年輕婆羅門」),菩提比丘長老英譯為\NoteKeywordBhikkhuBodhi{「青年;少年」}(youth)。按:「摩納磨;摩納」顯為māṇava的音譯。
\stopitemgroup

\startitemgroup[noteitems]
\item\subnoteref{103.0}\NoteSubKeyHead{(1)}\NoteKeywordNikayaHead{「居士」},南傳作\NoteKeywordNikaya{「屋主」}(gahapati,另譯為「家主、居士」),菩提比丘長老英譯為\NoteKeywordBhikkhuBodhi{「房子的所有權人;屋主;戶長」}(householder)。
\item\subnoteref{103.1}\NoteSubKeyHead{(2)}\NoteKeywordNikayaHead{「屋主婦」}(gahapatānī),菩提比丘長老英譯為\NoteKeywordBhikkhuBodhi{「主婦」}(housewife)。
\stopitemgroup

\startitemgroup[noteitems]
\item\subnoteref{104.0}\NoteKeywordAgamaHead{「盜;不與取;偷盜(SA);盜竊(DA);盜劫(AA)」},南傳作\NoteKeywordNikaya{「未給予而取」}(adinnādānā,另譯為「不與取;偷盜」),菩提比丘長老英譯為\NoteKeywordBhikkhuBodhi{「偷竊,抄襲」}(stealing),或「拿沒被給者」(taking what is not given)。
\stopitemgroup

\startitemgroup[noteitems]
\item\subnoteref{105.0}\NoteKeywordAgamaHead{「婬/邪婬/邪淫/他婬(SA);邪淫(MA);淫/淫亂/邪淫/淫逸(DA);淫泆(AA)」},南傳作\NoteKeywordNikaya{「邪淫」}(kāmesumicchācārā,逐字譯為「在諸欲上-邪(錯誤)-行」),菩提比丘長老英譯為\NoteKeywordBhikkhuBodhi{「性行為不檢(通姦)」}(sexual misconduct, SN/AN/DN),智髻比丘長老英譯為「在感官快樂上的行為不檢」(misconduct in sensual pleasures, MN)。
\stopitemgroup

\startitemgroup[noteitems]
\item\subnoteref{106.0}\NoteKeywordAgamaHead{「妄語(SA/DA/AA);妄言(MA);虛妄語(AA)」},南傳作\NoteKeywordNikaya{「妄語」}(musāvādā,另譯為「虛誑語;謊言」),菩提比丘長老英譯為\NoteKeywordBhikkhuBodhi{「不誠實的語言」}(false speech)。
\stopitemgroup

\startitemgroup[noteitems]
\item\subnoteref{107.0}\NoteKeywordNikayaHead{「榖酒、果酒、酒放逸處」}(surāmerayamajjapamādaṭṭhānā),菩提比丘長老英譯為\NoteKeywordBhikkhuBodhi{「果酒,酒精飲料與懈怠基礎的致醉品」}(wines, liquors and intoxicants which are a basis for negligence)。
\stopitemgroup

\startitemgroup[noteitems]
\item\subnoteref{108.0}\NoteKeywordAgamaHead{「和尚(SA);和上(GA)」},南傳作\NoteKeywordNikaya{「和尚」}(upajjhā,義譯為「親教師」),菩提比丘長老英譯為\NoteKeywordBhikkhuBodhi{「指導老師」}(preceptor)。
\stopitemgroup

\startitemgroup[noteitems]
\item\subnoteref{109.0}\NoteKeywordNikayaHead{「苦界」}(apāyaṃ),菩提比丘長老英譯為\NoteKeywordBhikkhuBodhi{「不幸之處」}(the plane of misery)。
\stopitemgroup

\startitemgroup[noteitems]
\item\subnoteref{110.0}\NoteSubKeyHead{(1)}\NoteKeywordAgamaHead{「惡趣;惡道;惡處/不善處(MA)」},南傳作\NoteKeywordNikaya{「惡趣」}(duggatiṃ),菩提比丘長老英譯為\NoteKeywordBhikkhuBodhi{「壞的目的地」}(the bad destinations)。按:「趣」(gatiṃ)即「去處;往生處」,地獄、餓鬼、畜生界被稱為「三惡趣;三塗」。
\item\subnoteref{110.1} (2)「墮惡趣(AA)」,南傳作\NoteKeywordNikaya{「落入惡趣」}(gacchanti duggatiṃ, It.)。「落入」(gacchanti),另譯作「去;行;走;走到」。
\stopitemgroup

\startitemgroup[noteitems]
\item\subnoteref{111.0}\NoteKeywordNikayaHead{「下界」}(vinipātaṃ,另譯為「墮處;惡處;險難處;惡趣;地獄;受苦的地方」),菩提比丘長老英譯為\NoteKeywordBhikkhuBodhi{「下面的世界」}(the nether world, the lower world)。按:在七識住的場合,《滿足希求》等說,這是指脫離苦界狀態的四類夜叉:uttaramātā, piyaṅkaramātā, phussamittā, dhammaguttāti,以及其他有宮殿的惡鬼(aññe ca vemānikā petā, \ccchref{DN.15}{https://agama.buddhason.org/DN/dm.php?keyword=15}/\ccchref{AN.7.44}{https://agama.buddhason.org/AN/an.php?keyword=7.44})。
\stopitemgroup

\startitemgroup[noteitems]
\item\subnoteref{112.0}\NoteKeywordNikayaHead{「善趣;善處」},南傳作\NoteKeywordNikaya{「善趣」}(sugati, suggatiṃ),菩提比丘長老英譯為\NoteKeywordBhikkhuBodhi{「好的到達地」}(good destination)。按:人界、天界被稱為「善趣」。
\stopitemgroup

\startitemgroup[noteitems]
\item\subnoteref{113.0}\NoteKeywordAgamaHead{「大果、大福利(SA);大福祐、大果報/大福、大果/大利、大果(MA);大果、大福、大利(AA)」},南傳作\NoteKeywordNikaya{「大果、大效益」}(mahapphalā mahānisaṃsā,另譯為「大果、大功德」),菩提比丘長老英譯為\NoteKeywordBhikkhuBodhi{「大成果與利益」}(great fruit and benefit)。按:「效益」(ānisaṃsa),另譯為「功德;利益;勝利;利潤;功績;好的結果;受益」。
\stopitemgroup

\startitemgroup[noteitems]
\item\subnoteref{114.0}\NoteKeywordAgamaHead{「正思惟/思惟/內正思惟(SA);正思惟/善思惟(MA);內自思惟(MA/DA/AA);專意/思惟(AA)」},南傳作\NoteKeywordNikaya{「如理作意」}(yonisomanasikāra,古德也譯作「如理思惟」),菩提比丘長老英譯為\NoteKeywordBhikkhuBodhi{「周密的注意」}(careful attention)。按:「如理」(yoniso)是「從起源;從根源」的意思,《滿足希求》說,以無常為無常等的引導(nayena),被稱為有方法的作意(upāyamanasikārassa, \ccchref{AN.1.16}{https://agama.buddhason.org/AN/an.php?keyword=1.16}),「思惟;作意」(manasikāra)為「在意(心)上」與「行為;所作」的複合詞,可以是「注意」,也可以有「思惟」的意思。在玄奘法師的譯經中,同一經可以見到如理作意與如理思惟同時出現的情形。而「八正道」中的「正思惟」(sammāsaṅkappa,另譯為「正志」),與此處的「正思惟」原文不同,含意也不同。《集異門足論》說:「云何如理作意?答:於耳所聞、耳識所了無倒法義,耳識所引令心專注,隨攝、等攝,作意、發意,審正思惟,心警覺性,如是名為如理作意。」《品類足論》說:「作意云何?謂:心警覺性。」《薩婆多宗五事論》說:「云何作意?謂:心所轉。」《俱舍論》說:「作意,謂:能令心警覺。」
\stopitemgroup

\startitemgroup[noteitems]
\item\subnoteref{115.0}\NoteKeywordAgamaHead{「如如來說/如說說/如佛所說(SA);說法如法/說如法(MA);為法語(DA)」},南傳作\NoteKeywordNikaya{「會是世尊的所說之說者」}(vuttavādī ceva bhagavato assaṃ, vuttavādino ceva bhagavato assatha),菩提比丘長老英譯為\NoteKeywordBhikkhuBodhi{「我按幸福者所說的陳述」}(I am to state what has been said by the Blessed One, SN)。
\stopitemgroup

\startitemgroup[noteitems]
\item\subnoteref{116.0}\NoteKeywordNikayaHead{「剎利」},南傳作\NoteKeywordNikaya{「剎帝利」}(khattiya,音譯,義譯為「王族」),印度四種階級之一,其它為婆羅門(brāhmaṇa,祭司)、毘舍(vessa,平民)、首陀羅(sudda,奴隸)。
\stopitemgroup

\startitemgroup[noteitems]
\item\subnoteref{117.0}\NoteKeywordNikayaHead{「聲聞」},南傳作\NoteKeywordNikaya{「弟子」}(sāvako,古譯為「聲聞」),菩提比丘長老英譯為\NoteKeywordBhikkhuBodhi{「弟子;門徒」}(disciple)。
\stopitemgroup

\startitemgroup[noteitems]
\item\subnoteref{118.0}\NoteSubKeyHead{(1)}\NoteKeywordAgamaHead{「欲欲(SA);貪欲(SA/DA/AA)」},南傳作\NoteKeywordNikaya{「欲的意欲」}(kāmacchandaṃ),菩提比丘長老英譯為\NoteKeywordBhikkhuBodhi{「感官的想要;肉慾的想要」}(sensual desire)。按:《菩提道次第廣論》卷二「貪慾等蓋」藏文འདོད་པ་ལ་འདུན་པ་ ལ་སོགས་པའི་ སྒྲིབ་པ་的「貪慾འདོད་པ་ལ་འདུན་པ」的對應梵文是kāmacchanda,與《佛光大辭典》作「貪欲蓋」梵語rāga-āvaraṇa(P.4794-2)的rāga不同。
\item\subnoteref{118.1}\NoteSubKeyHead{(2)}\NoteKeywordNikayaHead{「欲迷」}(kāmamucchā),菩提比丘長老英譯為\NoteKeywordBhikkhuBodhi{「感官的沈溺」}(sensual infatuation)。
\stopitemgroup

\startitemgroup[noteitems]
\item\subnoteref{119.0}\NoteKeywordAgamaHead{「斷斷(SA)」},南傳作\NoteKeywordNikaya{「捨斷的勤奮」}(pahānappadhānaṃ),菩提比丘長老英譯為\NoteKeywordBhikkhuBodhi{「以捨棄而努力」}(striving by abandonment)。按:《吉祥悅意》等解說為「捨斷欲尋等所生起的活力」(kāmavitakkādayo pajahantassa uppannavīriyaṃ, \ccchref{DN.34}{https://agama.buddhason.org/DN/dm.php?keyword=34}/\ccchref{AN.4.14}{https://agama.buddhason.org/AN/an.php?keyword=4.14})。
\stopitemgroup

\startitemgroup[noteitems]
\item\subnoteref{120.0}\NoteKeywordAgamaHead{「栴陀羅(SA);旃陀羅(MA/AA);真陀羅(GA/DA)」},南傳作\NoteKeywordNikaya{「旃陀羅」}(caṇḍāla),印順法師《以佛法研究佛法》說,這是地位低的賤民(p.55)。
\stopitemgroup

\startitemgroup[noteitems]
\item\subnoteref{121.0}\NoteKeywordAgamaHead{「為哀愍故(SA);以憐愍故(GA);慈哀愍念/為慈愍故(MA)」},南傳作\NoteKeywordNikaya{「出自憐愍」}(anukampaṃ upādāya),菩提比丘長老英譯為\NoteKeywordBhikkhuBodhi{「由於憐憫;由於同情」}(out of compassion)。
\stopitemgroup

\startitemgroup[noteitems]
\item\subnoteref{122.0}\NoteKeywordNikayaHead{「羅剎」}(rakkhasaṃ),菩提比丘長老英譯為\NoteKeywordBhikkhuBodhi{「魔鬼」}(demons),《中華佛教百科全書》說,這是食人肉的惡鬼(p.5857)。
\stopitemgroup

\startitemgroup[noteitems]
\item\subnoteref{123.0}\NoteKeywordNikayaHead{「甘露」},南傳作\NoteSubEntryKey{(i)}\NoteKeywordNikaya{「不死」}(amata,另譯為「甘露;涅槃」),菩提比丘長老英譯作「不死」(deathless),或「神的食物」(ambrosia)。按:要達到「不死」,只能「不(再)生」,也就是「不生則不死」的解脫。\NoteSubEntryKey{(ii)}\NoteKeywordNikaya{「蜜;蜂蜜」}(madhu),菩提比丘長老英譯作「蜂蜜;甘美物」(Honey)。
\stopitemgroup

\startitemgroup[noteitems]
\item\subnoteref{124.0}\NoteSubKeyHead{(1)}\NoteKeywordAgamaHead{「受具足(SA/MA);受於具足(MA);受具足戒(DA/AA)」},南傳作\NoteKeywordNikaya{「受具足戒;已受具足戒的」}(upasampanna),菩提比丘長老英譯為\NoteKeywordBhikkhuBodhi{「已受戒者」}(who has been ordained)。
\item\subnoteref{124.1}\NoteSubKeyHead{(2)}\NoteKeywordNikayaHead{「得到具足戒」}(alattha upasampadaṃ, labheyyaṃ upasampadan’ti),菩提比丘長老英譯為\NoteKeywordBhikkhuBodhi{「領受更高的授任」}(received the higher ordination)。
\item\subnoteref{124.2}\NoteSubKeyHead{(3)}\NoteKeywordNikayaHead{「使受具足戒」}(upasampādeti,使役動詞),菩提比丘長老英譯為\NoteKeywordBhikkhuBodhi{「給予完全的授任」}(give full ordination)。
\stopitemgroup

\startitemgroup[noteitems]
\item\subnoteref{125.0}\NoteSubKeyHead{(1)}\NoteKeywordAgamaHead{「寂默(SA);牟尼(MA)」},南傳作\NoteKeywordNikaya{「牟那」}(mona,另譯為「牟尼;沈默;智慧」),菩提比丘長老英譯為\NoteKeywordBhikkhuBodhi{「賢能狀態」}(sagehood, SN),智髻比丘長老英譯為「沈默」(the silence, MN)。按:《顯揚真義》說,這是指四道智(catumaggañāṇaṃ),知四諦法(catusaccadhamme jānātīti, \suttaref{SN.1.9})之意,《破斥猶豫》以「智」(ñāṇa, \ccchref{MN.56}{https://agama.buddhason.org/MN/dm.php?keyword=56})解說。
\item\subnoteref{125.1}\NoteSubKeyHead{(2)}\NoteKeywordNikayaHead{「牟尼」}(muni,另譯為「寂然;默者;賢人」),菩提比丘長老英譯為\NoteKeywordBhikkhuBodhi{「賢能者;賢明者」}(a sage)。按:《顯揚真義》以「具備智慧法(monadhammena, \suttaref{SN.7.9})者」,《破斥猶豫》以「具備阿羅漢智之牟尼位(arahattañāṇamoneyyena, \ccchref{MN.91}{https://agama.buddhason.org/MN/dm.php?keyword=91})」,《滿足希求》以「具備牟尼位(moneyyena)的漏被滅盡牟尼(khīṇāsavamuni, \ccchref{AN.3.59}{https://agama.buddhason.org/AN/an.php?keyword=3.59})」解說。
\stopitemgroup

\startitemgroup[noteitems]
\item\subnoteref{126.0}\NoteKeywordAgamaHead{「夜叉/悅叉(SA/DA);鬼天(MA);閱叉(AA)」},南傳作\NoteKeywordNikaya{「夜叉」}「(yakkho),菩提比丘長老英譯為\NoteKeywordBhikkhuBodhi{「幽靈」}(spirit)。
\stopitemgroup

\startitemgroup[noteitems]
\item\subnoteref{127.0}\NoteKeywordAgamaHead{「次第乞食(SA/MA);次行乞食(SA/GA);次第乞食/家家乞食(AA)」},南傳作\NoteKeywordNikaya{「為了托鉢次第地行走著」}(sapadānaṃ piṇḍāya caramāno),菩提比丘長老英譯為\NoteKeywordBhikkhuBodhi{「走在不間斷的施捨路線」}(walking on continous alms round)。按:佛制比丘、比丘尼托鉢的時候,要一家接著一家依序(次第)進行,不能挑貧富,不能依自己的好惡跳過某家。
\stopitemgroup

\startitemgroup[noteitems]
\item\subnoteref{128.0}\NoteKeywordAgamaHead{「住於天住/入晝正受(SA);晝行(MA)」},南傳作\NoteKeywordNikaya{「為了白天的住處」}(divāvihārāya),或「為白天的住處」(divāvihāraṃ,另譯為「晝住」),菩提比丘長老英譯為\NoteKeywordBhikkhuBodhi{「為了白天的滯留」}(for the day's abiding, SN/MN),或「為了渡過白天」(to pass the day, AN)。Maurice Walshe先生英譯為「為了他的中午休息」(for his midday rest)。按:《破斥猶豫》以「為了白天的獨坐」(divā paṭisallānatthāya, \ccchref{MN.18}{https://agama.buddhason.org/MN/dm.php?keyword=18})解說。另:「已進入白天的住處」(divāvihāragato)。
\stopitemgroup

\startitemgroup[noteitems]
\item\subnoteref{129.0}\NoteKeywordNikayaHead{「正受」},南傳作\NoteKeywordNikaya{「等至」}(samāpatti,名詞,另譯為「達到;定;入定」,音譯為「三摩鉢底」),菩提比丘長老英譯為\NoteKeywordBhikkhuBodhi{「達成;成就」}(the attainment),或「進入」(entering upon),或「獲得」(the acquisition)。按:《顯揚真義》以「到達(進入;入定)、自性的獲得、生起之義」(samāpajjanāya sabhāvapaṭilābhāya, uppattiyāti attho, \suttaref{SN.45.1})解說。另外,「九等至」則指四禪+四無色定+想受滅。
\stopitemgroup

\startitemgroup[noteitems]
\item\subnoteref{130.0}\NoteKeywordNikayaHead{「非人」}(Amanussā, amānusaṃ),菩提比丘長老英譯為\NoteKeywordBhikkhuBodhi{「非人類的生命」}(Nonhuman beings)。
\stopitemgroup

\startitemgroup[noteitems]
\item\subnoteref{131.0}\NoteKeywordNikayaHead{「龍象;龍」}(nāga),智髻比丘長老英譯為「偉大的生物」(great beings, MN),菩提比丘長老英譯為\NoteKeywordBhikkhuBodhi{「公象」}(bull elephant, SN/AN),或照錄原文,並解說此字解為na+āguṃ(無+罪惡),指佛陀,如\suttaref{SN.8.8}(\ccchref{AN.6.43}{https://agama.buddhason.org/AN/an.php?keyword=6.43}, Note.1317)。
\stopitemgroup

\startitemgroup[noteitems]
\item\subnoteref{132.0}\NoteSubKeyHead{(1)}\NoteKeywordNikayaHead{「善來」},南傳作\NoteKeywordNikaya{「歡迎」}(svāgataṃ,逐字譯為「善+來」),菩提比丘長老英譯為\NoteKeywordBhikkhuBodhi{「歡迎」}(Welcome)。
\item\subnoteref{132.1}\NoteSubKeyHead{(2)}\NoteKeywordNikayaHead{「請你來;來!」}(ehi,古譯也作「善來」,別譯雜阿含經作「汝善來!」),菩提比丘長老英譯為\NoteKeywordBhikkhuBodhi{「來!」}(Come)。按:「善來!比丘!」的「善來」,即用這個字。
\stopitemgroup

\startitemgroup[noteitems]
\item\subnoteref{133.0}\NoteKeywordAgamaHead{「三明達(SA/DA);三明(GA);三達(MA);三達智/三達明(AA)」},南傳作\NoteKeywordNikaya{「三明」}(tisso vijjā, Tevijjo),菩提比丘長老英譯為\NoteKeywordBhikkhuBodhi{「三項真實理解」}(three true knowledge)。按:佛法說的「三明」即「宿命明、天眼明、漏盡明」;婆羅門說的「三明」即「七代血統清淨、誦諸經典、容色端正」(\ccchref{SA.886}{https://agama.buddhason.org/SA/dm.php?keyword=886})。
\stopitemgroup

\startitemgroup[noteitems]
\item\subnoteref{134.0}\NoteKeywordAgamaHead{「五下分結(SA/MA);五下結(DA);五結(AA)」},南傳作\NoteKeywordNikaya{「五下分結」}(pañcannaṃ orambhāgiyānaṃ saṃyojanānaṃ, pañcorambhāgiyāni saññojanāni),菩提比丘長老英譯為\NoteKeywordBhikkhuBodhi{「五種較低拘束」}(the five lower fetters)。按:「五下分結」也簡為「下分結」,即「有身見、戒禁取、疑」(三結)加上「欲的意欲(貪)、惡意(瞋)」。而「五上分結」(pañcuddhambhāgiyāni saṃyojanānaṃ),\ccchref{AN.10.13}{https://agama.buddhason.org/AN/an.php?keyword=10.13}作「色貪、無色貪、慢、掉舉、無明」。
\stopitemgroup

\startitemgroup[noteitems]
\item\subnoteref{135.0}\NoteSubKeyHead{(1)}\NoteKeywordAgamaHead{「上座(SA);長老(MA/DA/AA);上弟子(MA)」},南傳作\NoteKeywordNikaya{「上座」}(thera),菩提比丘長老英譯為\NoteKeywordBhikkhuBodhi{「前輩」}(elder)。按:律典《毘尼母經》以出家二十年以上者稱為「上座」。
\item\subnoteref{135.1}\NoteSubKeyHead{(2)}\NoteKeywordNikayaHead{「上座位」}(thāvareyya,另譯為「長老地位」),菩提比丘長老英譯為\NoteKeywordBhikkhuBodhi{「資深地位」}(longstanding)。
\item\subnoteref{135.2}\NoteSubKeyHead{(3)}\NoteKeywordAgamaHead{「中年(SA);中(AA)」},南傳作\NoteKeywordNikaya{「中臘」}(Majjhimo),菩提比丘長老英譯為\NoteKeywordBhikkhuBodhi{「中間地位」}(middle standing)。
\item\subnoteref{135.3}\NoteSubKeyHead{(4)}\NoteKeywordAgamaHead{「下座(SA);年少(SA/AA)」},南傳作\NoteKeywordNikaya{「最新人;新;新進」}(sabbanavako; navo),菩提比丘長老英譯為\NoteKeywordBhikkhuBodhi{「最後進的比丘;最後進的」}(the most junior bhikkhu; junior)。
\stopitemgroup

\startitemgroup[noteitems]
\item\subnoteref{136.0}\NoteKeywordAgamaHead{「說此法時/說是法時(SA);說此法時(MA/DA/AA)」},南傳作\NoteKeywordNikaya{「還有,在當這個解說被說時」}(Imasmiñca pana veyyākaraṇasmiṃ bhaññamāne),菩提比丘長老英譯為\NoteKeywordBhikkhuBodhi{「當這個講說被說時」}(while this discourse was being spoken)。按:「當在被說時」(bhaññamāne),用的是被動動詞的「現在分詞」語態的處格,明確表示是在說法進行當中的情況。
\stopitemgroup

\startitemgroup[noteitems]
\item\subnoteref{137.0}\NoteKeywordNikayaHead{「是否能被[你]維持生活」}(kacci yāpanīyaṃ),菩提比丘長老英譯為\NoteKeywordBhikkhuBodhi{「我們希望你好轉」}(we hope that you are getting better),與北傳經文的「身小差」意思相合。「是否」(kacci),也可作「希望」解讀,但尊重巴利語原經文的編輯者在該句句尾標上問號(當然不必然得如此),故採用「是否」的意思。
\stopitemgroup

\startitemgroup[noteitems]
\item\subnoteref{138.0}\NoteKeywordAgamaHead{「同學(SA/AA);諸梵行(MA/DA);諸梵行者(MA/AA);諸梵行人(DA/AA)」},南傳作\NoteKeywordNikaya{「同梵行者」}(Sabrahmacārī,另譯為「同修行者」),菩提比丘長老英譯為\NoteKeywordBhikkhuBodhi{「在聖潔生活中的同伴」}(companions in the holy life)、「在聖潔生活中的兄弟」(brothers in the holy life, \suttaref{SN.22.85})、「僧侶同伴」(fellow monks, \ccchref{AN.6.45}{https://agama.buddhason.org/AN/an.php?keyword=6.45})。
\stopitemgroup

\startitemgroup[noteitems]
\item\subnoteref{139.0}\NoteKeywordAgamaHead{「法主(SA/MA);法所依憑(GA);世間法主(DA);法之真主(AA)」},南傳作\NoteKeywordNikaya{「法主」}(dhammassāmī,另譯為「法的所有者;法的支配者」),菩提比丘長老英譯為\NoteKeywordBhikkhuBodhi{「法主」}(the Lord of the Dhamma)。
\stopitemgroup

\startitemgroup[noteitems]
\item\subnoteref{140.0}\NoteKeywordAgamaHead{「拘舍彌國/憍賞彌國/俱睒彌國(SA);拘舍彌(MA);拘深城/拘深(AA)」},即「憍賞彌」(kosambī)城,為跋蹉(vaṃsa)國都城,約在今日安拉阿巴德(Allahabad)西西南45km處(N25.36096°,E81.402655°),\ccchref{AA.31.2}{https://agama.buddhason.org/AA/dm.php?keyword=31.2}說此為「過去四佛所居之處」,佛陀時代「優填王;優填那王」(rājā udeno)為其君主,城內有「瞿師羅園;瞿師園」(ghositārāma),為瞿師羅長者布施給世尊與僧團住的園林。
\stopitemgroup

\startitemgroup[noteitems]
\item\subnoteref{141.0}\NoteKeywordAgamaHead{「無覺少觀定(MA);無覺有觀三昧(DA)」},南傳作\NoteKeywordNikaya{「無尋唯伺定」}(avitakkampi vicāramattaṃ samādhiṃ; avitakkavicāramatto samādhi; avitakko vicāramatto samādhi),智髻比丘長老英譯為「無應用心思但只持續心思之貫注集中」(concentration without applied thought but with sustained thought only)。按:「唯伺」的「唯」(mattaṃ),另一個意思即「小量的」,所以「唯伺」也可以是「少觀;少伺」之意,《破斥猶豫》以「在五種定式中的第二禪定」(pañcakanaye dutiyajjhānasamādhiṃ)解說,菩提比丘長老解說,從阿毘達摩所說的五禪支(尋、伺、喜、樂、一心)來看,這屬於那些無法同時做到無尋無伺的第二禪者。
\stopitemgroup

\startitemgroup[noteitems]
\item\subnoteref{142.0}\NoteKeywordAgamaHead{「阿練若/阿蘭若(SA/AA);無事處(GA);無事/無事處(MA)」},南傳作\NoteKeywordNikaya{「林野」}(arañña,另音譯為「阿蘭若」),菩提比丘長老英譯為\NoteKeywordBhikkhuBodhi{「山林;林野」}(forest)。按:經律中有時也將住林野者簡稱為「阿練若;阿蘭若」。
\stopitemgroup

\startitemgroup[noteitems]
\item\subnoteref{143.0}\NoteKeywordAgamaHead{「施設(SA/MA/AA);說(AA)」},南傳作\NoteKeywordNikaya{「安立;使知」}(paññāpeti, paññapeti,名詞paññatti, paññāpana),菩提比丘長老英譯為\NoteKeywordBhikkhuBodhi{「描述;使知道」}(describe, make known)。按:漢譯大乘經論也有依梵語prajñapti音譯為「波羅聶提」者,《顯揚真義》以「使知道」(jānāpeti, \suttaref{SN.12.20}),《破斥猶豫》以「使看見、使住立」(dasseti ṭhapeti, \ccchref{MN.56}{https://agama.buddhason.org/MN/dm.php?keyword=56})解說。
\stopitemgroup

\startitemgroup[noteitems]
\item\subnoteref{144.0}\NoteKeywordAgamaHead{「受(SA/DA);覺(MA);痛(AA)」},南傳作\NoteKeywordNikaya{「受;感受」}(vedanā)」,菩提比丘長老英譯為\NoteKeywordBhikkhuBodhi{「感受」}(feeling)。
\stopitemgroup

\startitemgroup[noteitems]
\item\subnoteref{145.0}\NoteKeywordNikayaHead{「大師」}(satthā,原形為satthar),菩提比丘長老英譯為\NoteKeywordBhikkhuBodhi{「你們的老師;我們的老師;老師」}(your teacher,our teacher,The teacher)。按:「大師」一詞,在佛陀時代指的是教派的宗師,佛弟子口中的「大師」,就只能是指佛陀。
\stopitemgroup

\startitemgroup[noteitems]
\item\subnoteref{146.0}\NoteKeywordAgamaHead{「若變、若異(SA);敗壞變易/變易異/變易(MA);敗壞、遷移不停(AA)」},南傳作\NoteKeywordNikaya{「變易成為不同的;變易而成為不同的」}(vipariṇamati aññathā hoti, vipariṇamati, aññathā ca hoti),菩提比丘長老英譯為\NoteKeywordBhikkhuBodhi{「改變與變更」}(change and alters, the change and alteration)。
\stopitemgroup

\startitemgroup[noteitems]
\item\subnoteref{147.0}\NoteSubKeyHead{(1)}\NoteKeywordNikayaHead{「多羅樹」},南傳作\NoteKeywordNikaya{「棕櫚樹」}(tāla,音譯為「多羅」),菩提比丘長老英譯為\NoteKeywordBhikkhuBodhi{「棕櫚樹」}(a palm)。
\item\subnoteref{147.1}\NoteSubKeyHead{(2)}\NoteKeywordAgamaHead{「如截多羅樹頭/如折多羅樹/斷多羅樹頭(SA);譬如多羅樹斬(GA);猶如多羅樹斷其頭(DA)」},南傳作\NoteKeywordNikaya{「[如]已斷根的棕櫚樹」}(tālāvatthukata),菩提比丘長老英譯為\NoteKeywordBhikkhuBodhi{「使之像棕櫚樹的殘株」}(make like a palm stump)。《顯揚真義》等說,如棕櫚樹的基礎已作的(tālavatthu viya katāni),又以不生長義(puna aviruhaṇaṭṭhena)如頭已切斷的棕櫚樹(matthakacchinnatālo viya, \suttaref{SN.12.55}/sīsacchinnatālo, \ccchref{MN.22}{https://agama.buddhason.org/MN/dm.php?keyword=22}),以及如連根被拔起後它的住立處已作的(samūlaṃ tālaṃ uddharitvā tassa patiṭṭhitaṭṭhānaṃ viya ca katānīti),今取連根被拔起之意翻譯。
\stopitemgroup

\startitemgroup[noteitems]
\item\subnoteref{148.0}\NoteKeywordAgamaHead{「由旬(SA/DA/AA);由延(SA);拘婁舍(MA)」},南傳作\NoteKeywordNikaya{「由旬」}(yojana),菩提比丘長老英譯為\NoteKeywordBhikkhuBodhi{「里格」}(leagues),並解說,3由旬約20英里。水野弘元《巴利語辭典》解說為1由旬約14公里。
\stopitemgroup

\startitemgroup[noteitems]
\item\subnoteref{149.0}\NoteKeywordAgamaHead{「欝多羅僧(SA);優哆邏僧/優多羅僧(MA);鬱多羅僧(MA/DA)」},南傳作\NoteKeywordNikaya{「上衣」}(uttarāsaṅga),菩提比丘長老英譯為\NoteKeywordBhikkhuBodhi{「上衣」}(upper robe)。
\stopitemgroup

\startitemgroup[noteitems]
\item\subnoteref{150.0}\NoteKeywordNikayaHead{「經行」}(caṅkamanti),菩提比丘長老英譯為\NoteKeywordBhikkhuBodhi{「來回走」}(walking up and down, walking back and forth)。
\stopitemgroup

\startitemgroup[noteitems]
\item\subnoteref{151.0}\NoteKeywordAgamaHead{「剎利頂生王(SA/MA);頂生剎利王/剎利以水灑頂/王剎利頂(MA);剎利水澆頭種(DA);水灌頭王(AA)」},南傳作\NoteKeywordNikaya{「剎帝利灌頂王」}(rañño khattiyassa muddhāvasittassa, rājā khattiyo muddhābhisitto,另譯為「剎利頂生王」),菩提比丘長老英譯為\NoteKeywordBhikkhuBodhi{「頭上洗禮之高貴王」}(head-anointed noble king)。
\stopitemgroup

\startitemgroup[noteitems]
\item\subnoteref{152.0}\NoteKeywordAgamaHead{「衣毛皆豎/身毛竪(SA);心驚毛竪(豎)(SA/AA);身毛為豎(GA);舉身毛豎(竪)(MA);衣毛為豎(DA)」},南傳作\NoteKeywordNikaya{「身毛豎立;身毛豎立的」}(lomāni haṭṭhāni, lomahaṃsa),菩提比丘長老英譯為\NoteKeywordBhikkhuBodhi{「驚嚇;喪膽」}(terrified, SN/AN),智髻比丘長老英譯為「帶著他的髮端豎立」(with his hair standing on end, MN)。
\stopitemgroup

\startitemgroup[noteitems]
\item\subnoteref{153.0}\NoteSubKeyHead{(1)}\NoteKeywordAgamaHead{「食噉含消(MA);種種甘饍(DA);種種甘饌飲食(AA)」},南傳作\NoteKeywordNikaya{「硬食、軟食」}(khādanīyaṃ bhojanīyaṃ,另譯為「嚼食與噉食」),菩提比丘長老英譯為\NoteKeywordBhikkhuBodhi{「各種食物」}(food of various; various kinds of food)。
\item\subnoteref{153.1}\NoteSubKeyHead{(2)}\NoteKeywordAgamaHead{「含消(MA)」},南傳作\NoteKeywordNikaya{「軟食」}(bhojanīyaṃ),菩提比丘長老英譯為\NoteKeywordBhikkhuBodhi{「餐」}(meal)。
\stopitemgroup

\startitemgroup[noteitems]
\item\subnoteref{154.0}\NoteKeywordAgamaHead{「智/知(SA);斷知(SA/MA)」},南傳作\NoteKeywordNikaya{「遍知」}(parijānaṃ, 現在分詞),菩提比丘長老英譯為\NoteKeywordBhikkhuBodhi{「完全地理解」}(fully understanding)。按:《顯揚真義》以「究竟遍知(最終遍知)、超越(渡過/征服)」(accantapariññaṃ, samatikkamanti, \suttaref{SN.22.23})解說。長老說,在經典中,只有解脫阿羅漢才適合說「遍知(parijānāti)」,而只有初果以上的聖者,才適合說「自證;證知(abhijānāti)」(親身體證的知, \suttaref{SN.22.23})。詞態變化:動詞原形parijānāti,形容詞pariñña(遍知的),過去分詞pariññātaṃ(已遍知),未來被動分詞pariññeyyaṃ(應能遍知/應該被遍知/所知)。
\stopitemgroup

\startitemgroup[noteitems]
\item\subnoteref{155.0}\NoteKeywordAgamaHead{「磨滅法(SA);衰法/離散之法(MA);朽壞法(DA);盡法(AA);摩滅法(摩訶僧祇律)」},南傳作\NoteKeywordNikaya{「消散法;消散法的狀態」}(vayadhammo; vayadhammatā),菩提比丘長老英譯為\NoteKeywordBhikkhuBodhi{「屬於消散者」}(subject to vanishing, SN),或「消散性質」(the nature of vanishing, SN),智髻比丘長老英譯為「屬於消失者」(subject to disappearance, MN),Maurice Walshe先生英譯為「屬於逝去的」(subject to passing away)。
\stopitemgroup

\startitemgroup[noteitems]
\item\subnoteref{156.0}\NoteSubKeyHead{(1)}\NoteKeywordAgamaHead{「安樂住/(若)樂/樂住(SA);住安隱/安樂住止/安隱快樂(MA);得安隱(AA)」},南傳作\NoteKeywordNikaya{「安樂住」}(phāsuvihāro, vihāraphāsu,另譯為「安樂住處」),菩提比丘長老英譯為\NoteKeywordBhikkhuBodhi{「舒適地生活」}(live in comfort, SN),智髻比丘長老英譯為「一個舒適的住處」(a comfortable abiding, MN)。
\item\subnoteref{156.1}\NoteSubKeyHead{(2)}\NoteKeywordNikayaHead{「住於安樂」}(phāsu vihariti),菩提比丘長老英譯為\NoteKeywordBhikkhuBodhi{「住於安樂」}(dwell at ease, AN)。
\stopitemgroup

\startitemgroup[noteitems]
\item\subnoteref{157.0}\NoteKeywordNikayaHead{「慈分」}(Mettaṃ so, Mettaṃso),菩提比丘長老均採Mettaṃ so英譯為「有慈愛者」(Who has lovingkindness, SN/AN)。《顯揚真義》說:他修習慈與慈前分,又或「(部)分」被稱為「部份(一分)」(so mettañceva mettāpubbabhāgañca bhāveti. Atha vā aṃsoti koṭṭhāso vuccati, \suttaref{SN.10.4}),《滿足希求》以「成為部份發出慈心者」(mettāyamānacittakoṭṭhāso hutvā, \ccchref{AN.8.1}{https://agama.buddhason.org/AN/an.php?keyword=8.1})解說,《如是語註釋》以「慈所生心的一分(部份),或以慈部分以不捨棄義為已生成的部分」(mettāmayacittakoṭṭhāso, mettāya vā aṃso avijahanaṭṭhena avayavabhūtoti, \ccchref{It.27}{https://agama.buddhason.org/It/dm.php?keyword=27})解說。
\stopitemgroup

\startitemgroup[noteitems]
\item\subnoteref{158.0}\NoteKeywordAgamaHead{「無明覆/無明所蓋(SA);無明之所覆蓋(GA);無明所覆(MA/AA)」},南傳作\NoteKeywordNikaya{「無明蓋」}(avijjānīvaraṇānaṃ),菩提比丘長老英譯為\NoteKeywordBhikkhuBodhi{「被無知所妨礙」}(hindered by ignorance)。
\stopitemgroup

\startitemgroup[noteitems]
\item\subnoteref{159.0}\NoteKeywordAgamaHead{「必定/決定(SA);於道決定(GA);定(MA);決定(DA)」},南傳作\NoteKeywordNikaya{「決定者」}(niyataṃ),菩提比丘長老英譯為\NoteKeywordBhikkhuBodhi{「未來(命運)已固定」}(fixed in destiny)。
\stopitemgroup

\startitemgroup[noteitems]
\item\subnoteref{160.0}\NoteKeywordAgamaHead{「正趣三菩提/定趣三菩提/向於三菩提/正覺趣/正向於正覺/正向三菩提/向正覺(SA);趣正覺(MA);必得涅槃/必成道果(DA);必至涅槃/必至滅度/必成道果(AA)」},南傳作\NoteKeywordNikaya{「正覺為彼岸者」}(sambodhiparāyana),菩提比丘長老英譯為\NoteKeywordBhikkhuBodhi{「以開化為他的目的地」}(with enlightenment as his destination)。按:「正覺」(sambodhi),音譯為「三菩提」。
\stopitemgroup

\startitemgroup[noteitems]
\item\subnoteref{161.0}\NoteKeywordAgamaHead{「七有天人往生/七有天人往來(SA);人天七返(GA);極受七有(MA);極七往返(DA);七死七生(AA)」},南傳作\NoteKeywordNikaya{「最多七次在天上與人間流轉輪迴後」}(Sattakkhattuparamaṃ deve ca manusse ca sandhāvitvā saṃsaritvā),菩提比丘長老英譯為\NoteKeywordBhikkhuBodhi{「在天與人中最多七次漫遊與流浪後」}(who, after roaming and wandering on among devas and humans seven times at most)。按:「最多七次」(sattakkhattuparamaṃ),另譯為「極七返」,《俱舍論》說:「飲光部經分明別說於人、天處各受七生。」顯然與「天上加人間最多七次」的解讀不同。
\stopitemgroup

\startitemgroup[noteitems]
\item\subnoteref{162.0}\NoteKeywordAgamaHead{「見使(SA/MA/DA);邪見使(AA)」},南傳作\NoteKeywordNikaya{「見煩惱潛在趨勢」}(diṭṭhānusayānaṃ,另譯為「見隨眠;見使」),菩提比丘長老英譯為\NoteKeywordBhikkhuBodhi{「見解之表面下的趨勢」}(the underlying tendency to views)。Maurice Walshe先生英譯為「見解之潛在傾向」(latent proclivity of views)。
\stopitemgroup

\startitemgroup[noteitems]
\item\subnoteref{163.0}\NoteKeywordAgamaHead{「士夫(SA/MA/AA);人(MA);人身(AA)」},南傳作\NoteKeywordNikaya{「人;個人」}(puggalo),菩提比丘長老英譯為\NoteKeywordBhikkhuBodhi{「個人」}(the person)。
\stopitemgroup

\startitemgroup[noteitems]
\item\subnoteref{164.0}\NoteKeywordAgamaHead{「一切神(SA);一切眾生有命之類(DA);一切人/一切眾生(AA)」},南傳作\NoteKeywordNikaya{「一切生命」}(sabbe jīvā),菩提比丘長老英譯為\NoteKeywordBhikkhuBodhi{「一切靈魂」}(all souls)。按:「生命」(jīvā),古譯為「命」,十無記中「命即是身」的「命」就用這個字。
\stopitemgroup

\startitemgroup[noteitems]
\item\subnoteref{165.0}\NoteSubKeyHead{(1)}\NoteKeywordNikayaHead{「須陀洹」},南傳作\NoteKeywordNikaya{「入流者」}(sotāpanno,另譯為「須陀洹支;預流者」),菩提比丘長老英譯為\NoteKeywordBhikkhuBodhi{「入流者」}(the stream-enterer),又稱之為「七有」,\ccchref{DN.29}{https://agama.buddhason.org/DN/dm.php?keyword=29}又稱之為「初果;第一果」(paṭhamaṃ phalaṃ)。
\item\subnoteref{165.1}\NoteSubKeyHead{(2)}\NoteKeywordNikayaHead{「入流果;須陀洹果」}(Sotāpattiphalaṃ),菩提比丘長老英譯為\NoteKeywordBhikkhuBodhi{「入流果」}(The fruit of stream-entry)。
\stopitemgroup

\startitemgroup[noteitems]
\item\subnoteref{166.0}\NoteKeywordAgamaHead{「隨信行/信行(SA);堅信(GA);信行者/信行(MA);持信(AA)」},南傳作\NoteKeywordNikaya{「隨信行者」}(saddhānusārī),菩提比丘長老英譯為\NoteKeywordBhikkhuBodhi{「一位信心之信奉者」}(a faith-follower)。按:《破斥猶豫》等說,該人是極有信根的為了入流果作證之行者(向須陀洹),他以信為導向、信為先導地修習聖道(yassa puggalassa sotāpattiphalasacchikiriyāya paṭipannassa saddhindriyaṃ adhimattaṃ hoti, saddhāvāhiṃ saddhāpubbaṅgamaṃ ariyamaggaṃ bhāveti, \ccchref{MN.70}{https://agama.buddhason.org/MN/dm.php?keyword=70}/\ccchref{DN.28}{https://agama.buddhason.org/DN/dm.php?keyword=28}/\ccchref{AN.7.14}{https://agama.buddhason.org/AN/an.php?keyword=7.14}),而\ccchref{AA.27.10}{https://agama.buddhason.org/AA/dm.php?keyword=27.10}所說較鬆。
\stopitemgroup

\startitemgroup[noteitems]
\item\subnoteref{167.0}\NoteKeywordAgamaHead{「隨法行/法行(SA);堅法(GA);法行者/法行(MA);奉法(AA)」},南傳作\NoteKeywordNikaya{「隨法行者」}(dhammānusārī),菩提比丘長老英譯為\NoteKeywordBhikkhuBodhi{「一位法之信奉者」}(a Dhamma-follower)。按:《破斥猶豫》等說,該人是極有慧根的為了入流果的作證之行者(向須陀洹),他以慧為導向、慧為先導地修習聖道(yassa puggalassa sotāpattiphalasacchikiriyāya paṭipannassa paññindriyaṃ adhimattaṃ hoti, paññāvāhiṃ paññāpubbaṅgamaṃ ariyamaggaṃ bhāveti, \ccchref{MN.70}{https://agama.buddhason.org/MN/dm.php?keyword=70}/\ccchref{DN.28}{https://agama.buddhason.org/DN/dm.php?keyword=28}/\ccchref{AN.7.14}{https://agama.buddhason.org/AN/an.php?keyword=7.14}),而\ccchref{AA.27.10}{https://agama.buddhason.org/AA/dm.php?keyword=27.10}所說較鬆。
\stopitemgroup

\startitemgroup[noteitems]
\item\subnoteref{168.0}\NoteKeywordNikayaHead{「一向的」}(ekanta, ekaṃso,另譯為「單一的;專門的」),菩提比丘長老英譯為\NoteKeywordBhikkhuBodhi{「只限;唯有」}(exclusively),或「絕對地」(categorically),或「片面的;單方面的」(one-sided)。
\stopitemgroup

\startitemgroup[noteitems]
\item\subnoteref{169.0}\NoteKeywordAgamaHead{「命異身異(SA);身我異(GA);非命非身(AA)」},南傳作\NoteKeywordNikaya{「命是一身體是另一」}(aññaṃ jīvaṃ aññaṃ sarīranti,逐字譯為「異命異身」),菩提比丘長老英譯為\NoteKeywordBhikkhuBodhi{「靈魂是一回事,身體是另一回事」}(the soul is one thing, the body is another)。
\stopitemgroup

\startitemgroup[noteitems]
\item\subnoteref{170.0}\NoteKeywordAgamaHead{「世間無常/世間我無常(SA);世無有常(MA);世無常/世間無常(DA);世無常(AA)」},南傳作\NoteKeywordNikaya{「世界是非常恆的」}(asassato loko),菩提比丘長老英譯為\NoteKeywordBhikkhuBodhi{「世間不是永恆的」}(the world is nor eternal)。按:「常的」(nicca)與「常恆的」(sassata),應屬於同義字,「無常」(anicca)與「非常恆的」(asassata)也是,但在經文中,「非常恆的」(asassata)一字都用於指「斷滅論」。
\stopitemgroup

\startitemgroup[noteitems]
\item\subnoteref{171.0}\NoteKeywordAgamaHead{「世有邊(SA/AA);世有底(MA);世間有邊/世有邊(DA)」},南傳作\NoteKeywordNikaya{「世界是有邊的」}(antavā loko),菩提比丘長老英譯為\NoteKeywordBhikkhuBodhi{「世間是有限的」}(the world is finite)。
\stopitemgroup

\startitemgroup[noteitems]
\item\subnoteref{172.0}\NoteSubKeyHead{(1)}\NoteKeywordAgamaHead{「自作(SA)」},南傳作\NoteSubEntryKey{(i)}\NoteKeywordNikaya{「自作者」}(attakāre),Maurice Walshe先生英譯為「自己的力量」(self-power, \ccchref{DN.2}{https://agama.buddhason.org/DN/dm.php?keyword=2}),坦尼沙羅比丘長老英譯為「自己引起的」(self-caused, \ccchref{DN.2}{https://agama.buddhason.org/DN/dm.php?keyword=2}),菩提比丘長老英譯為\NoteKeywordBhikkhuBodhi{「自己發動」}self-initiative, \ccchref{AN.6.38}{https://agama.buddhason.org/AN/an.php?keyword=6.38}),K. Nizamis英譯為「自己行為者」(self-doer, \ccchref{AN.6.38}{https://agama.buddhason.org/AN/an.php?keyword=6.38})。\NoteSubEntryKey{(ii)}\NoteKeywordNikaya{「自己作的」}(sayaṃkataṃ),菩提比丘長老英譯為\NoteKeywordBhikkhuBodhi{「被自己創造」}(created by oneself, \suttaref{SN.12.17})。
\item\subnoteref{172.1}\NoteSubKeyHead{(2)}\NoteKeywordAgamaHead{「自他作/自作他作(SA);自造他造(DA)」},南傳作\NoteKeywordNikaya{「自己作的與其他者作的」}(sayaṃkatañca paraṃkatañca),菩提比丘長老英譯為\NoteKeywordBhikkhuBodhi{「被自己與被另外的創造」}(created both by oneself and by another)。
\stopitemgroup

\startitemgroup[noteitems]
\item\subnoteref{173.0}\NoteKeywordAgamaHead{「無因作(SA);無因而出/忽然而有(DA)」},南傳作\NoteKeywordNikaya{「自然生的」}(adhiccasamuppannaṃ,另譯為「無因生的」),菩提比丘長老英譯為\NoteKeywordBhikkhuBodhi{「偶然地發生」}(has arisen fortuitously)。
\stopitemgroup

\startitemgroup[noteitems]
\item\subnoteref{174.0}\NoteKeywordAgamaHead{「離生喜、樂(SA/MA/DA);念持喜、樂/念持喜安/有猗念樂(AA)」},南傳作\NoteKeywordNikaya{「離而生喜、樂」}(vivekajaṃ pītisukhaṃ),菩提比丘長老英譯為\NoteKeywordBhikkhuBodhi{「具隔離所生的狂喜與快樂」}(with rapture and happiness born of seclusion)。
\stopitemgroup

\startitemgroup[noteitems]
\item\subnoteref{175.0}\NoteSubKeyHead{(1)}\NoteKeywordAgamaHead{「有覺、有觀;與覺、觀俱(DA)」},南傳作\NoteKeywordNikaya{「有尋、有伺」}(savitakkaṃ savicāraṃ),菩提比丘長老英譯為\NoteKeywordBhikkhuBodhi{「被心思與查驗陪同」}(is accompanied by thought and examination)。
\item\subnoteref{175.1}\NoteSubKeyHead{(2)}\NoteKeywordAgamaHead{「無覺、無觀;捨覺、觀(DA)」},南傳作\NoteKeywordNikaya{「無尋、無伺」}(Avitakkomhi avicāro),菩提比丘長老英譯為\NoteKeywordBhikkhuBodhi{「無心思與查驗」}(Without thought and examination),按:《顯揚真義》說,以尋伺為雜染而無尋無伺(kilesavitakkavicārehi avitakkāvicāro, \suttaref{SN.47.10}),長老說,這似乎意味著他到達第二禪。
\stopitemgroup

\startitemgroup[noteitems]
\item\subnoteref{176.0}\NoteKeywordAgamaHead{「身身觀(SA);觀身(GA);觀身如身(MA);身念處(SA/DA)」},南傳作\NoteKeywordNikaya{「在身上隨看著身」}(kāye kāyānupassī,逐字譯為「身-身+隨看」),菩提比丘長老英譯為\NoteKeywordBhikkhuBodhi{「在身體凝視著身體」}(contemplating the body in the body, \suttaref{SN.47.1}),並解說,這是「以將之從其它隔離(如受、心等),決定所緣(身)」,「受、心、法」的情況亦同。按:「隨看著」(anupassī,另譯為「觀察」,形容詞,但以如現在分詞的動作形容詞解讀),這裡,不論是「看」(passī)或「凝視」(contemplating)都不宜理解為眼睛的看或凝視。《顯揚真義》說,「隨看著身」是身的隨看(觀察)習慣(anupassanasīlo),或身的隨看中(anupassamāno, \suttaref{SN.47.1}),並列舉七項隨看:無常(捨斷常想)、苦(捨斷樂想)、非我(捨斷我想)、厭(捨斷歡喜想)、離染(捨斷貪)、滅(捨斷集)、斷念(捨斷執取想)。
\stopitemgroup

\startitemgroup[noteitems]
\item\subnoteref{177.0}\NoteKeywordAgamaHead{「頭衣燒然/火燒頭衣(SA);火燒頭燒衣(MA)」},南傳作\NoteKeywordNikaya{「當衣服或頭已被燒時」}(Āditte……cele vā sīse vā),或「衣服已被燒,或頭已被燒」(ādittacelo vā ādittasīso vā),菩提比丘長老英譯為\NoteKeywordBhikkhuBodhi{「衣服或頭已著火」}(clothes or head had caught fire)。
\stopitemgroup

\startitemgroup[noteitems]
\item\subnoteref{178.0}\NoteSubKeyHead{(1)}\NoteKeywordNikayaHead{「止觀;止與觀」}(samathavipassanā, samatho ca vipassanā ca,音譯為「奢摩他、毘婆舍那」,名詞),菩提比丘長老英譯為\NoteKeywordBhikkhuBodhi{「平靜與洞察」}(serenity and insight)。
\item\subnoteref{178.1}\NoteSubKeyHead{(2)}\NoteKeywordNikayaHead{「作觀」}(vipassati,動詞,另譯為「作毘婆舍那」),菩提比丘長老英譯為\NoteKeywordBhikkhuBodhi{「深深地看/清楚地看」}(deeply see/clearly sees, AN),智髻比丘長老英譯為「得到洞察」(gained insight, MN)。
\stopitemgroup

\startitemgroup[noteitems]
\item\subnoteref{179.0}\NoteSubKeyHead{(1)}\NoteKeywordAgamaHead{「記說/記(SA/MA);記別/記(MA/DA/AA)」},南傳作\NoteKeywordNikaya{「記說」}(byākato,動詞byākaroti),菩提比丘長老英譯為\NoteKeywordBhikkhuBodhi{「聲明;斷言」}(declare)。另一個意思是「解說,解答,回答」,菩提比丘長老英譯為\NoteKeywordBhikkhuBodhi{「解說」}(explained)。
\item\subnoteref{179.1}\NoteSubKeyHead{(2)}\NoteKeywordAgamaHead{「無記/不記說(SA);無記(MA);不記(DA/AA)」},南傳作\NoteKeywordNikaya{「無記;不被記說;不記說;沒記說」}(Abyākataṃ,動詞na byākaroti),菩提比丘長老英譯為\NoteKeywordBhikkhuBodhi{「不聲明;不宣說;不斷言」}(has not declared)。
\stopitemgroup

\startitemgroup[noteitems]
\item\subnoteref{180.0}\NoteKeywordNikayaHead{「緣」},南傳作\NoteKeywordNikaya{「為緣;緣」}(paccayā,另有「資具;必需品」之意),菩提比丘長老英譯為\NoteKeywordBhikkhuBodhi{「為條件」}(as condition, \suttaref{SN.12.1}),或「條件」(condition, \suttaref{SN.22.60}),或「由於」(on account of, \suttaref{SN.12.41}),智髻比丘長老英譯為「由於」(because of, \ccchref{MN.31}{https://agama.buddhason.org/MN/dm.php?keyword=31})。
\stopitemgroup

\startitemgroup[noteitems]
\item\subnoteref{181.0}\NoteKeywordAgamaHead{「正盡苦(SA/MA);平等盡苦(DA);盡苦(AA)」},南傳作\NoteKeywordNikaya{「為了苦的完全滅盡」}(sammā dukkhakkhayāya,逐字譯為「正-苦+盡」),菩提比丘長老英譯為\NoteKeywordBhikkhuBodhi{「為了苦的完全破壞」}(for the complete destruction of suffering)。另外,「盡苦」,南傳也作「苦的滅盡」(dukkhakkhayāya),菩提比丘長老英譯為\NoteKeywordBhikkhuBodhi{「摧毀苦」}(destroying suffering)。
\stopitemgroup

\startitemgroup[noteitems]
\item\subnoteref{182.0}\NoteKeywordNikayaHead{「定」}(samādhi),音譯為「三摩地;三摩提;三昧」,義譯為「等持」),菩提比丘長老英譯為\NoteKeywordBhikkhuBodhi{「集中貫注」}(concentration)。
\stopitemgroup

\startitemgroup[noteitems]
\item\subnoteref{183.0}\NoteKeywordAgamaHead{「尼師壇(SA);尼師檀(SA/MA/AA);坐具(GA)」},南傳作\NoteKeywordNikaya{「坐墊布」}(nisīdana,音譯為「尼師檀」,另譯為「坐具」),菩提比丘長老英譯為\NoteKeywordBhikkhuBodhi{「坐布」}(a sitting cloth)。
\stopitemgroup

\startitemgroup[noteitems]
\item\subnoteref{184.0}\NoteKeywordAgamaHead{「優陀那」(udānaṃ),菩提比丘長老英譯為「有所啟示的話」(inspired utterance)。其動詞「吟出」(udāneti, udānesi),另譯為「發語;自說;發感興語」,古德譯為「歎(\ccchref{SA.64}{https://agama.buddhason.org/SA/dm.php?keyword=64})」}。《原始佛教聖典之集成》說:「優陀那」(udāna),或音譯為鄔陀南、嗢拖南等;義譯為讚歎、自說、自然說等。Ud+van,為氣息的由中而出,發為音聲;本義為由於驚、喜、怖、悲等情感,自然抒發出來的音聲。所以古人的解說,主要為「感興語」、「自然說」二類。
\stopitemgroup

\startitemgroup[noteitems]
\item\subnoteref{185.0}\NoteSubKeyHead{(1)}\NoteKeywordNikayaHead{「覺;菩提」}(bodhiṃ),菩提比丘長老英譯為\NoteKeywordBhikkhuBodhi{「開化」}(the enlightenment)。
\item\subnoteref{185.1}\NoteSubKeyHead{(2)}\NoteKeywordAgamaHead{「正覺(SA/MA);等覺(SA/DA)」},南傳作\NoteKeywordNikaya{「正覺」}(sambodhi,音譯為「三菩提」),菩提比丘長老英譯為\NoteKeywordBhikkhuBodhi{「啟發;開化」}(enlightenment)。按:這是指生死解脫的「證悟」。
\item\subnoteref{185.2}\NoteSubKeyHead{(3)}\NoteKeywordNikayaHead{「正覺者」}(sambuddha,另譯為「三佛陀;三佛」),菩提比丘長老英譯為\NoteKeywordBhikkhuBodhi{「開化者」}(enlightened)。按:這是指「解脫者」。
\stopitemgroup

\startitemgroup[noteitems]
\item\subnoteref{186.0}\NoteSubKeyHead{(1)}\NoteKeywordNikayaHead{「菩薩」}(bodhisatta),為「覺」(bodhi)與「有情」(satta)複合詞的音譯,印順法師依「說一切有部」的經論推考定,應該是西元前200年前後出現的名詞(《初期大乘佛教起源與開展》p.130),最初應該只是釋迦牟尼佛未成佛之前的專屬稱呼。
\item\subnoteref{186.1}\NoteSubKeyHead{(2)}\NoteKeywordNikayaHead{「摩訶薩」}(mahāsatta),為音譯,義譯為「大眾生、大有情」,具有大心、大行的眾生。
\stopitemgroup

\startitemgroup[noteitems]
\item\subnoteref{187.0}\NoteKeywordAgamaHead{「五欲功德(SA/MA/DA);五欲德(SA);五欲(SA/AA)」},南傳作\NoteKeywordNikaya{「五種欲」}(pañca kāmaguṇā),菩提比丘長老英譯為\NoteKeywordBhikkhuBodhi{「五束感官快樂」}(five cords of sensual pleasure)。按:「種類」(guṇa),另譯為「功德」,但與「福德」(puñña,也譯為「功德」)不同,這裡應該只是單純的「種類」的意思。在玄奘法師的譯作中沒發現「五欲德;五欲功德」這樣的譯法,應該譯成「五妙欲」了。
\stopitemgroup

\startitemgroup[noteitems]
\item\subnoteref{188.0}\NoteSubKeyHead{(1)}\NoteKeywordNikayaHead{「漏;流漏」}(āsavaṃ-名詞,āsavati-動詞),菩提比丘長老英譯為\NoteKeywordBhikkhuBodhi{「污染」}(taint)。原意是「流出來;漏出來」,引申為「(生死)煩惱」的異名。
\item\subnoteref{188.1}\NoteSubKeyHead{(2)}\NoteKeywordAgamaHead{「漏失(DA/AA);漏(DA),南傳作「流漏的;流漏者」}(avassuto,另譯為「充滿欲貪的;充滿欲貪者」),智髻比丘長老英譯為「動搖;動心」(moved, MN),菩提比丘長老英譯為\NoteKeywordBhikkhuBodhi{「腐化」}(the corrupted, SN),並解說此字義譯為「流入或洩漏」(flown into, or leaky),暗示心被雜染滲透,形容詞的avassuta與動詞的「流動」(anussavati, anusavati, savati),都是基於字根「流動」(su)。按:大致來說,經文中前者(āsava)多用於生死流轉的情形,如對解脫者說「漏盡」,後者多用於六根對六境時的情形。
\item\subnoteref{188.2}\NoteSubKeyHead{(3)}\NoteKeywordAgamaHead{「漏法(SA)」},南傳作\NoteKeywordNikaya{「流漏法門」}(avassutapariyāyañca),菩提比丘長老英譯為\NoteKeywordBhikkhuBodhi{「腐化的解說」}(an exposition on the corrupted, SN)。
\stopitemgroup

\startitemgroup[noteitems]
\item\subnoteref{189.0}\NoteKeywordAgamaHead{「逮得己利(SA);自得善義/而得善義(MA);自獲己利(DA)」},南傳作\NoteKeywordNikaya{「自己的利益已達成的」}(anuppattasadatthā),菩提比丘長老英譯為\NoteKeywordBhikkhuBodhi{「到達他們自己的目標」}(reach their own goal)。按:「自己的利益」(sadatthā),另譯為「善利;妙利,理想」。
\stopitemgroup

\startitemgroup[noteitems]
\item\subnoteref{190.0}\NoteKeywordAgamaHead{「盡諸有結(SA/GA/DA);斷除有結(SA);有結已解/有結盡(MA);盡諸有結使(DA);盡生死原本(AA)」},南傳作\NoteKeywordNikaya{「有之結已滅盡的」}(parikkhīṇabhavasaṃyojanā),菩提比丘長老英譯為\NoteKeywordBhikkhuBodhi{「存在的拘束被完全破壞」}(utterly destroyed the fetters of existence)。按:「有」(bhava),就是愛、取、有的「有」。
\stopitemgroup

\startitemgroup[noteitems]
\item\subnoteref{191.0}\NoteKeywordAgamaHead{「究竟智(MA)」},南傳作\NoteKeywordNikaya{「究竟智」}(sammadaññā,另譯為「正確地了知」),菩提比丘長老英譯為\NoteKeywordBhikkhuBodhi{「最終的理解」}(final knowledge)。按:「究竟智」與「完全智」(aññā)的意思似乎等同(菩提比丘長老的英譯是相同的),北傳多譯為「究竟智」,《顯揚真義》以「經由正確根據知道後」(Sammā kāraṇehi jānitvā, \suttaref{SN.1.8}/3.3),《破斥猶豫》以「正確的了知(完全的了知)」(sammā aññāya, \ccchref{MN.1}{https://agama.buddhason.org/MN/dm.php?keyword=1})解說。「以究竟智解脫」:《顯揚真義》以「經由道慧(Maggapaññāya)知道四諦法後,以解脫果解脫(phalavimuttiyā vimuttāti, \suttaref{SN.3.3})」,《吉祥悅意》等以「經由證確的原因、理由知道後解脫」(sammā hetunā kāraṇena jānitvā vimutto, \ccchref{DN.27}{https://agama.buddhason.org/DN/dm.php?keyword=27}/\ccchref{AN.6.49}{https://agama.buddhason.org/AN/an.php?keyword=6.49})解說。
\stopitemgroup

\startitemgroup[noteitems]
\item\subnoteref{192.0}\NoteSubKeyHead{(1)}\NoteKeywordNikayaHead{「軛安穩;軛安穩者」}(yogakkhemaṃ-安穩的結合,另譯為「瑜珈安穩-修行得安穩;努力得安穩),菩提比丘長老英譯為\NoteKeywordBhikkhuBodhi{「離束縛之安全」}(the security from bondage)。按:《破斥猶豫》說,就是阿羅漢狀態的意趣(arahattameva adhippetaṃ, \ccchref{MN.1}{https://agama.buddhason.org/MN/dm.php?keyword=1}),《吉祥悅意》說,就是涅槃之名(nibbānasseva nāmaṃ, \ccchref{DN.21}{https://agama.buddhason.org/DN/dm.php?keyword=21})。
\item\subnoteref{192.1}\NoteSubKeyHead{(2)}\NoteKeywordNikayaHead{「想不軛安穩」}(ayogakkhemakāmo),菩提比丘長老英譯為\NoteKeywordBhikkhuBodhi{「想危害他」}(wanted to endanger him, SN),智髻比丘長老英譯為「想要束縛」(desiring bondage, MN)。
\stopitemgroup

\startitemgroup[noteitems]
\item\subnoteref{193.0}\NoteSubKeyHead{(1)}\NoteKeywordAgamaHead{「在學地/學/學人/在學地者(SA);學地(GA);學者(MA);學(DA);學人(AA)」},南傳作\NoteKeywordNikaya{「有學」}(sekkhā, sekhaṃ),菩提比丘長老英譯為\NoteKeywordBhikkhuBodhi{「訓練中者」}(trainees)。按:通常這是指初果到三果的聖者,有時也泛指未解脫前的在學者。
\item\subnoteref{193.1}\NoteSubKeyHead{(2)}\NoteKeywordNikayaHead{「無學, 無學的」}(asekha, asekhiya),菩提比丘長老英譯為\NoteKeywordBhikkhuBodhi{「超越訓練者」}(who is beyond training)。按:「無學」就是指「阿羅漢」。
\stopitemgroup

\startitemgroup[noteitems]
\item\subnoteref{194.0}\NoteKeywordAgamaHead{「戒取(SA/DA);戒受(MA);戒盜(DA/AA)」},南傳作\NoteKeywordNikaya{「戒禁取」}(sīlabbatupādānaṃ, sīlabbataparāmāsaṃ),菩提比丘長老英譯為\NoteKeywordBhikkhuBodhi{「行為與遵奉的錯誤把握」}(wrong grasp of behavior and observances),或「規則與誓約的扭曲把握」(distorted grasp of rules and vows)。按:MA的「受」即「取」的另譯。《顯揚真義》等以「心想:牛戒(習慣)、牛禁戒(Gosīlagovatādīni, \suttaref{SN.12.2}/\ccchref{MN.9}{https://agama.buddhason.org/MN/dm.php?keyword=9})這樣很好,就自己執持、執取」,《吉祥悅意》以「心想:這些是好的而有這樣戒(習慣)、禁戒的執持(sīlavatānaṃ gahaṇaṃ, \ccchref{DN.33}{https://agama.buddhason.org/DN/dm.php?keyword=33}),《滿足希求》以「執取(parāmasitvā)戒(習慣)與禁戒後,握持、執持的程度(gahitaṃ gahaṇamattaṃ, \ccchref{AN.6.55}{https://agama.buddhason.org/AN/an.php?keyword=6.55})」解說。
\stopitemgroup

\startitemgroup[noteitems]
\item\subnoteref{195.0}\NoteKeywordAgamaHead{「三事和合緣觸/三事和合觸/三法和合觸/三事和合生觸(SA);三事共會便有更觸(MA);三事相因便有更樂(AA)」},南傳作\NoteKeywordNikaya{「三者的會合而有觸(而觸存在)」}(tiṇṇaṃ saṅgati phasso),菩提比丘長老英譯為\NoteKeywordBhikkhuBodhi{「三者的會合為接觸」}(the meeting of the three is contact)。按:這裡的「法」(dhammānaṃ,複數),不是指「正法」,而是指「事」,即「根」、「境」、「識」三者。這裡的「觸」(phasso),同十二緣起中的「觸」,與一般的「接觸;觸達」(phuṭṭha)不同。
\stopitemgroup

\startitemgroup[noteitems]
\item\subnoteref{196.0}\NoteKeywordNikayaHead{「施食」}(piṇḍapāta,另譯為「鉢食;托鉢食;乞食(名詞)」),菩提比丘長老英譯為\NoteKeywordBhikkhuBodhi{「施捨的食物」}(almsfood),為「團食」(piṇḍa)與「落;投」(pāta)的複合字,意思是「投入或落入鉢裡的一團團食物」,也就是托鉢時被施捨的食物,今準此譯。
\stopitemgroup

\startitemgroup[noteitems]
\item\subnoteref{197.0}\NoteKeywordAgamaHead{「隨形好(SA);好(AA)」},南傳作\NoteKeywordNikaya{「細相」}(anubyañjanaso,另譯為「隨好;隨相;隨形好;隨標記;隨特徵」),菩提比丘長老英譯為\NoteKeywordBhikkhuBodhi{「特徵」}(features),並解說,「相」(nimitta,sign-形跡)是「整體的」(the composite),「細相」是「個別的」(by seperation)。
\stopitemgroup

\startitemgroup[noteitems]
\item\subnoteref{198.0}\NoteKeywordAgamaHead{「億波提/有餘(SA);受身(GA)」},南傳作\NoteKeywordNikaya{「依著」}(upadhiṃ,另譯為「依戀;再生的基質;執著」),菩提比丘長老英譯為\NoteKeywordBhikkhuBodhi{「獲得;所獲」}(acquisition)。按:《顯揚真義》以「蘊與污染的造作」(khandhakilesābhisaṅkhāresu, \suttaref{SN.8.2}),《破斥猶豫》以「蘊的依著、污染的依著、造作的依著、五種欲的依著」(pañcakāmaguṇūpadhīti, \ccchref{MN.140}{https://agama.buddhason.org/MN/dm.php?keyword=140})解說。《華嚴經》:「捨一切烏波提涅槃法,能生一切菩薩行,修習不斷故。」《華嚴經疏》:「烏波提者此云有苦。」
\stopitemgroup

\startitemgroup[noteitems]
\item\subnoteref{199.0}\NoteKeywordNikayaHead{「寒林」}(sītavane),菩提比丘長老英譯為\NoteKeywordBhikkhuBodhi{「冷的小樹林」}(Cool Grove)。按:「寒林」為王舍城北方的一個棄屍樹林,《大唐西域記》:「寒林者,棄屍之所,俗謂不祥之地,人絕遊往之迹。」《一切經音義》:「屍陀林(正言:尸多婆那,此名寒林,其林幽𨗉而寒,因以名也,在王舍城側。陀者多也,死人多迸其中,今總指棄屍之處名屍陀林者,取彼名)。」
\stopitemgroup

\startitemgroup[noteitems]
\item\subnoteref{200.0}\NoteKeywordNikayaHead{「尊者」}(āyasmā, āyasmant,另譯為「具壽」),菩提比丘長老英譯為\NoteKeywordBhikkhuBodhi{「尊敬的」}(Venerable)。按:《顯揚真義》說,這是愛語、這是敬語(piyavacanametaṃ garuvacanametaṃ, \suttaref{SN.12.60}),《破斥猶豫》等說,這是愛語(piyavacanametaṃ, \ccchref{MN.3}{https://agama.buddhason.org/MN/dm.php?keyword=3}/\ccchref{DN.23}{https://agama.buddhason.org/DN/dm.php?keyword=23}/\ccchref{AN.2.16}{https://agama.buddhason.org/AN/an.php?keyword=2.16})。
\stopitemgroup

\startitemgroup[noteitems]
\item\subnoteref{201.0}\NoteKeywordNikayaHead{「朋友;道友;學友」}(āvuso),菩提比丘長老英譯為\NoteKeywordBhikkhuBodhi{「朋友」}(friend)。按:這是對同輩或較低輩分者的稱呼,為了作出區別,出家眾稱在家人時譯為「朋友」,稱外道時譯為「道友」,稱同道時譯為「學友」。
\stopitemgroup

\startitemgroup[noteitems]
\item\subnoteref{202.0}\NoteKeywordNikayaHead{「先生」}(bho, bhavaṃ),菩提比丘長老英譯為\NoteKeywordBhikkhuBodhi{「先生」}(sir, \suttaref{SN.1.38}),Maurice Walshe先生英譯為「朋友」(friend, \ccchref{DN.2}{https://agama.buddhason.org/DN/dm.php?keyword=2})。
\stopitemgroup

\startitemgroup[noteitems]
\item\subnoteref{203.0}\NoteKeywordNikayaHead{「尊師」}(bhavant, bhavaṃ, bhonto, bho),菩提比丘長老英譯為\NoteKeywordBhikkhuBodhi{「大師」}(Master),Maurice Walshe先生英譯為「先生」(Sir, DN)。
\stopitemgroup

\startitemgroup[noteitems]
\item\subnoteref{204.0}\NoteKeywordNikayaHead{「親愛的先生」}(mārisa,另譯為「我的朋友;我的老師;尊師」),菩提比丘長老英譯為\NoteKeywordBhikkhuBodhi{「賢明的先生;先生」}(good sir; sir, SN/MN),或「可敬的先生」(Respected sir, AN),或「親愛的先生」(dear sir, SN),並解說,天神稱呼世尊、著名比丘,或天神互稱時用此語,有時國王互稱也用(\suttaref{SN.1.1})。
\stopitemgroup

\startitemgroup[noteitems]
\item\subnoteref{205.0}\NoteKeywordAgamaHead{「過人法(SA);人上之法(MA);上人法(GA/DA/AA);上尊之法(AA)」},南傳作\NoteKeywordNikaya{「過人法」}(uttarimanussadhammaṃ),菩提比丘長老英譯為\NoteKeywordBhikkhuBodhi{「超人的」}(superhuman),智髻比丘長老英譯為「超人的狀態」(the superhuman state, MN)。按:《破斥猶豫》說,這是指至少熟練一種通往阿羅漢性的毘婆舍那入口(vipassanāmukhaṃ, \ccchref{MN.69}{https://agama.buddhason.org/MN/dm.php?keyword=69}),長老說包括禪定、證智(神通)、道與果(證果)都是。
\stopitemgroup

\startitemgroup[noteitems]
\item\subnoteref{206.0}\NoteKeywordAgamaHead{「奇哉(SA);未曾有/不可思議(GA);未曾有(DA)」},南傳作\NoteKeywordNikaya{「未曾有」}(abbhuta,另譯為「不可思議」),菩提比丘長老英譯為\NoteKeywordBhikkhuBodhi{「令人驚奇的;不可思議的;太棒了」}(wonderful)。按:「未曾有」(abbhuta),另譯為「不可思議」,而「不可思議」(acchariya),也譯為「希有」,這類讚嘆形式的經文漸集成一類,即為《九分教》的〈未曾有〉,後來編入《中阿含經》、《長阿含》、《增一阿含》中,如《中阿含經》中有〈未曾有法品〉,\ccchref{MN.123}{https://agama.buddhason.org/MN/dm.php?keyword=123}經名即為《不可思議-未曾有經》(acchariya-abbhutasuttaṃ),參看印順法師《原始佛教聖典之集成》p.727。
\stopitemgroup

\startitemgroup[noteitems]
\item\subnoteref{207.0}\NoteSubKeyHead{(1)}\NoteKeywordNikayaHead{「無明」}(avijjā),菩提比丘長老英譯為\NoteKeywordBhikkhuBodhi{「無知」}(ignorance)。
\item\subnoteref{207.1}\NoteSubKeyHead{(2)}\NoteKeywordNikayaHead{「明」}(vijjā),菩提比丘長老英譯為\NoteKeywordBhikkhuBodhi{「真實的理解」}(true knowledge)。按:《破斥猶豫》給了些詮釋,如:道慧(maggapaññā, \ccchref{MN.143}{https://agama.buddhason.org/MN/dm.php?keyword=143})、阿羅漢道的明(arahattamaggavijjaṃ, \ccchref{MN.149}{https://agama.buddhason.org/MN/dm.php?keyword=149})等。
\stopitemgroup

\startitemgroup[noteitems]
\item\subnoteref{208.0}\NoteSubKeyHead{(1)}\NoteKeywordNikayaHead{「斯陀含;一來者」}(sakadāgāmi),菩提比丘長老英譯為\NoteKeywordBhikkhuBodhi{「回來一次」}(once-returning),\ccchref{DN.29}{https://agama.buddhason.org/DN/dm.php?keyword=29}又稱為「第二果」(dutiyaṃ phalaṃ)。
\item\subnoteref{208.1}\NoteSubKeyHead{(2)}\NoteKeywordNikayaHead{「斯陀含位;一來狀態」}(sakadāgāmitā),Maurice Walshe先生英譯為「一位回來一次者」(a Once-Returner)。
\item\subnoteref{208.2}\NoteSubKeyHead{(3)}\NoteKeywordNikayaHead{「斯陀含果;一來果」}(sakadāgāmiphalaṃ),菩提比丘長老英譯為\NoteKeywordBhikkhuBodhi{「回來一次之果」}(the fruit of once-returning)。
\stopitemgroup

\startitemgroup[noteitems]
\item\subnoteref{209.0}\NoteSubKeyHead{(1)}\NoteKeywordNikayaHead{「阿那含;不還者」}(anāgāmi, anāgāmin, anāgāmī),菩提比丘長老英譯為\NoteKeywordBhikkhuBodhi{「不回來者」}(nonreturning, nonreturner),\ccchref{DN.29}{https://agama.buddhason.org/DN/dm.php?keyword=29}又稱之為「第三果」(tatiyaṃ phalaṃ)。
\item\subnoteref{209.1}\NoteSubKeyHead{(2)}\NoteKeywordNikayaHead{「阿那含果;不還果」}(anāgāmiphalaṃ),菩提比丘長老英譯為\NoteKeywordBhikkhuBodhi{「不回來之果」}(the fruit of nonreturning)。
\item\subnoteref{209.2}\NoteSubKeyHead{(3)}\NoteKeywordNikayaHead{「不轉回者」}(anāvattidhamm),菩提比丘長老英譯為\NoteKeywordBhikkhuBodhi{「不回來」}(without returning)。
\stopitemgroup

\startitemgroup[noteitems]
\item\subnoteref{210.0}\NoteKeywordNikayaHead{「新學」}(navakānaṃ),菩提比丘長老英譯為\NoteKeywordBhikkhuBodhi{「新授命者;新出家者」}(newly ordained)。
\stopitemgroup

\startitemgroup[noteitems]
\item\subnoteref{211.0}\NoteSubKeyHead{(1)}\NoteKeywordAgamaHead{「斷念」(paṭinissagga, 動詞paṭinissajjati,另譯為「捨遣;捨離;定棄」),菩提比丘長老英譯為「對其斷念;死心」(relinquishing of it),並解說「斷念」主要用在毘婆舍那階段,對所有有為法經由洞察無常而積極消除雜染,發生在「安那般那念」的第十六階(參看\ccchref{SA.803}{https://agama.buddhason.org/SA/dm.php?keyword=803}),「捨棄」}則用在聖道成熟,可能意味著完全放棄所有執著的最後狀態,因此在意義上與涅槃緊接(\suttaref{SN.45.2}, note7)。
\item\subnoteref{211.1}\NoteSubKeyHead{(2)}\NoteKeywordAgamaHead{「觀察斷/觀斷(SA)」},南傳作\NoteKeywordNikaya{「隨看著斷念」}(paṭinissaggānupassī),菩提比丘長老英譯為\NoteKeywordBhikkhuBodhi{「凝視死心」}(Contemplating relinquishment)。按:「隨看著」(anupassī,另譯為「觀察」,形容詞,但以如現在分詞的動作形容詞解讀),《清淨道論》說,有兩種斷念:遍捨斷念與躍入斷念(pariccāgapaṭinissaggo ca pakkhandanapaṭinissaggo ca),這是毘婆舍那道的同義語(Vipassanāmaggānaṃ etamadhivacanaṃ, 8.236)。
\stopitemgroup

\startitemgroup[noteitems]
\item\subnoteref{212.0}\NoteKeywordAgamaHead{「四衢道頭(SA/AA);四衢道(MA);四道頭/四交道頭(DA)」},南傳作\NoteKeywordNikaya{「在十字路口」}(catumahāpathe,按字逐譯為「四-大-道」),菩提比丘長老英譯為\NoteKeywordBhikkhuBodhi{「在十字路口」}(at a crossroads)。
\stopitemgroup

\startitemgroup[noteitems]
\item\subnoteref{213.0}\NoteKeywordAgamaHead{「宮殿(SA);樓觀(MA);樓閣(DA);高廣之臺(AA)」},南傳作\NoteKeywordNikaya{「重閣」}(kūṭāgāra,另譯為「峰屋(屋頂為尖塔般的二層樓建築物)」),菩提比丘長老英譯為\NoteKeywordBhikkhuBodhi{「尖屋頂的房子」}(a house with a peaked roof, a peaked house),Maurice Walshe先生英譯為「山形[屋頂]房間」(gabled chamber, \ccchref{DN.17}{https://agama.buddhason.org/DN/dm.php?keyword=17})。
\stopitemgroup

\startitemgroup[noteitems]
\item\subnoteref{214.0}\NoteKeywordNikayaHead{「初夜」}(paṭhamo yāmo, rattiyā paṭhamaṃ yāmaṃ),菩提比丘長老英譯為\NoteKeywordBhikkhuBodhi{「初更」}(the first watch)。按:若將夜間分作三等分,分別稱為「初夜、中夜、後夜」。
\stopitemgroup

\startitemgroup[noteitems]
\item\subnoteref{215.0}\NoteKeywordAgamaHead{「大梵王;娑婆世界主梵天王(SA);梵天王(AA)」},南傳作\NoteKeywordNikaya{「梵王娑婆主」}(brahmā sahampati, Sahampatibrahma)。按:這是指「大梵天」,為色界初禪天的領導者,其另一個稱號為「梵王常童子」(brahmā sanaṅkumāro),菩提比丘長老說,當他還是年輕「樂神」(Pañcasikha, 般遮翼;五髻;五頂)時,就修習初禪而往生於梵天世界,所以他們以「童子」(kumāra)認知他,因為他一直保持年輕時的外表,即使年紀很大了也被稱為「常童子」(Sanaṅkumāra, Forever Youthful)。而「娑婆」為saha的音譯,義譯為「請忍耐;要忍耐」,指這個堪忍世間。
\stopitemgroup

\startitemgroup[noteitems]
\item\subnoteref{216.0}\NoteSubKeyHead{(1)}\NoteKeywordAgamaHead{「失念(SA);失正念/無正念(MA);多忘(DA);忘失(AA)」},南傳作\NoteKeywordNikaya{「念已忘失的;失念」}(muṭṭhassati, sati muṭṭhā),菩提比丘長老英譯為\NoteKeywordBhikkhuBodhi{「混亂心的」}(muddle-minded, Muddle-mindedness)。按:《顯揚真義》以「念消失的、無念的,…忘失(失念)」(naṭṭhassatino sativirahitā,…pamussanti, \suttaref{SN.2.25}),《滿足希求》以「念已捨離的」(vissaṭṭhasatino, \ccchref{AN.2.43}{https://agama.buddhason.org/AN/an.php?keyword=2.43})解說。
\item\subnoteref{216.1}\NoteSubKeyHead{(2)}\NoteKeywordNikayaHead{「忘失念」}(muṭṭhasacca),菩提比丘長老英譯為\NoteKeywordBhikkhuBodhi{「混亂心」}(Muddle-mindedness)。
\stopitemgroup

\startitemgroup[noteitems]
\item\subnoteref{217.0}\NoteSubKeyHead{(1)}\NoteKeywordNikayaHead{「律儀」},南傳作\NoteKeywordNikaya{「自制」}(saṃvaraṃ,另譯為「防護;律儀;攝護」),菩提比丘長老英譯為\NoteKeywordBhikkhuBodhi{「自制」}(restraint)。
\item\subnoteref{217.1}\NoteSubKeyHead{(2)}\NoteKeywordNikayaHead{「律儀斷」},南傳作\NoteKeywordNikaya{「自制的勤奮」}(saṃvarappadhānaṃ),菩提比丘長老英譯為\NoteKeywordBhikkhuBodhi{「以自制而努力」}(Striving by restraint)。
\stopitemgroup

\startitemgroup[noteitems]
\item\subnoteref{218.0}\NoteSubKeyHead{(1)}\NoteKeywordNikayaHead{「別理會這個」}(tiṭṭhatetaṃ),菩提比丘長老英譯為\NoteKeywordBhikkhuBodhi{「隨它吧;讓它去吧(別理它)」}(let it be/let that be, SN/MN)。按:《顯揚真義》以「令那個持續吧」(tiṭṭhatu etaṃ)解說。
\item\subnoteref{218.1}\NoteSubKeyHead{(2)}\NoteKeywordNikayaHead{「我確實沒得到你的[理解]」}(Addhā kho tyāhaṃ…na labhāmi, Addhā kho te ahaṃ…na labhāmi),智髻比丘長老英譯為「由於我確實不能說服你」(since I certainly cannot persuade you, MN),菩提比丘長老英譯為\NoteKeywordBhikkhuBodhi{「我確實沒使你瞭解」}(Surely…I am not getting through to you, SN)。
\stopitemgroup

\startitemgroup[noteitems]
\item\subnoteref{219.0}\NoteKeywordAgamaHead{「正見(MA)」},南傳作\NoteKeywordNikaya{「貫通後看見」}(ativijjha passati),智髻比丘長老英譯為「洞察與看見」(penetrates…and sees, MN),菩提比丘長老英譯為\NoteKeywordBhikkhuBodhi{「貫穿了它」}(having pierced through it, AN/SN)。按:「貫通後」(ativijjha),另譯為「通達後;洞察後」,《破斥猶豫》說,以慧洞察污染後,當它(涅槃)成為明瞭的、明顯的時,他看見(paññāya ca kilese nibbijjhitvā tadeva vibhūtaṃ pākaṭaṃ karonto passati),菩提比丘長老以證初果理解(\ccchref{MN.95}{https://agama.buddhason.org/MN/dm.php?keyword=95})。
\stopitemgroup

\startitemgroup[noteitems]
\item\subnoteref{220.0}\NoteSubKeyHead{(1)}\NoteKeywordNikayaHead{「香」},南傳作\NoteKeywordNikaya{「氣味」}(gandha,古譯為「香」),菩提比丘長老英譯為\NoteKeywordBhikkhuBodhi{「氣味、香味、臭味」}(odours)。
\item\subnoteref{220.1}\NoteSubKeyHead{(2)}\NoteKeywordNikayaHead{「味」},南傳作\NoteKeywordNikaya{「味道」}(rasa),菩提比丘長老英譯為\NoteKeywordBhikkhuBodhi{「味、滋味、風味」}(tastes)。
\item\subnoteref{220.2}\NoteSubKeyHead{(3)}\NoteKeywordAgamaHead{「觸(SA/MA/DA);細滑(AA)」},南傳作\NoteKeywordNikaya{「所觸」}(phoṭṭhabba),菩提比丘長老英譯為\NoteKeywordBhikkhuBodhi{「觸覺的對象」}(tactile objects)。
\stopitemgroup

\startitemgroup[noteitems]
\item\subnoteref{221.0}\NoteKeywordAgamaHead{「向於捨/捨於進趣(SA);趣至出要/願至非品/趣非品/趣向非品(MA);捨諸惡法/求出要(AA)」},南傳作\NoteKeywordNikaya{「捨棄的成熟」}(vossaggapariṇāmiṃ),菩提比丘長老英譯為\NoteKeywordBhikkhuBodhi{「在解開上成熟」}(maturing in release, \suttaref{SN.45.2}),並解說「捨棄(解開)」(vossagga)有雙重含義:「永捨之捨棄」(pariccāga vossagga)與「躍進之捨棄」(pakkhandana vossagga),前者是捨斷雜染:從練習毘婆舍那的「彼分」(tadaṅgavasena)到出世間道的「斷」(samucchedavasena),後者是進入涅槃:由練習毘婆舍那時傾向它(tadninnabhāvena),以及在聖道中作為所緣(ārammaṇakaraṇena)。捨斷雜染而進入涅槃,就是「捨棄的成熟」。而「捨棄」(vossagga)與「斷念;定棄」(paṭinissagga),在詞源與含意上是緊密關連的,但用在《尼科耶》中,一個細微的差異似乎將它們分離。「斷念」主要用在毘婆舍那階段,對所有有為法經由洞察無常而積極消除雜染,發生在「安那般那念」的第十六階(參看\ccchref{SA.803}{https://agama.buddhason.org/SA/dm.php?keyword=803}),「捨棄」則用在聖道成熟,可能意味著完全放棄所有執著的最後狀態,因此在意義上與涅槃緊接。「成熟」(pariṇāmiṃ),《顯揚真義》以「成熟的與遍熟的」(paripaccantaṃ paripakkañcāti, \suttaref{SN.3.18})解說。
\stopitemgroup

\startitemgroup[noteitems]
\item\subnoteref{222.0}\NoteKeywordAgamaHead{「法齋日/十五日(SA);十五日(GA);月十五日(MA/DA/AA);十五日齋(DA);齋日(AA)」},南傳作\NoteSubEntryKey{(i)}\NoteKeywordNikaya{「在十四、十五日裡」}(Cātuddasiṃ pañcadasiṃ),菩提比丘長老英譯為\NoteKeywordBhikkhuBodhi{「在第十四、第十五」}(On the fourteenth or fifteenth),或\NoteSubEntryKey{(ii)}\NoteKeywordNikaya{「在那個布薩日」}(tadahuposathe),菩提比丘長老英譯為\NoteKeywordBhikkhuBodhi{「布薩日」}(the Uposatha day),並說這是佛教徒的儀式日(observance day),在陰曆15日或14日晚上比丘聚集在一起背誦「波羅提木叉」(\suttaref{SN.8.7})。佛陀時代印度一年分為寒季、熱季、雨季,每季有四個月,即八個半月(pakkha),第3與第7個半月有14天,其它為15天(摩訶僧祇律卷第二所說亦同),在每個半月的滿月(第14日或第15日)與新月夜晚,以及半月形的夜晚(第8天)被認為特別吉祥,前者為布薩日,該日夜晚比丘們誦戒,而在家信眾則在白天就到寺院守八關齋戒、聞法、修定(\ccchref{MN.4}{https://agama.buddhason.org/MN/dm.php?keyword=4})。後者為「小布薩日;小齋戒日」(minor Uposathaṃ, \suttaref{SN.10.5})。「布薩」(uposatha),安世高(150AD)譯為布薩,支謙(250AD)譯為齋,齋者①戒潔也(東漢-許慎-說文解字100AD)②洗心曰齊(齋),防患曰戒(東晉-韓康伯-易·繫辭註~380AD)。
\stopitemgroup

\startitemgroup[noteitems]
\item\subnoteref{223.0}\NoteKeywordNikayaHead{「疑使」},南傳作\NoteKeywordNikaya{「疑煩惱潛在趨勢」}(vicikicchānusayo,另譯為「疑使」),菩提比丘長老英譯為\NoteKeywordBhikkhuBodhi{「疑惑之表面下的趨勢」}(the underlying tendency to doubt)。Maurice Walshe先生英譯為「疑惑之潛在傾向」(latent proclivity of doubt)。
\stopitemgroup

\startitemgroup[noteitems]
\item\subnoteref{224.0}\NoteSubKeyHead{(1)}\NoteKeywordAgamaHead{「不漏其心(SA)」},南傳作\NoteKeywordNikaya{「(將)不流入」}(nānussavanti, nānvāssavissanti),菩提比丘長老英譯為\NoteKeywordBhikkhuBodhi{「不流進他」}(do not flow in upon him)。
\item\subnoteref{224.1}\NoteSubKeyHead{(2)}\NoteKeywordAgamaHead{「則漏其心(SA)」},南傳作\NoteKeywordNikaya{「會流入」}(anvāssaveyyuṃ),菩提比丘長老英譯為\NoteKeywordBhikkhuBodhi{「可能侵入你」}(might invade you)。
\stopitemgroup

\startitemgroup[noteitems]
\item\subnoteref{225.0}\NoteSubKeyHead{(1)}\NoteKeywordAgamaHead{「緣起法(SA);因緣法(SA/DA/AA);因緣/因緣起(MA)」},南傳作\NoteKeywordNikaya{「緣起」}(paṭiccasamuppādaṃ,另譯為「依之生起的」),菩提比丘長老英譯為\NoteKeywordBhikkhuBodhi{「依之而開始」}(dependent origination)。按:《顯揚真義》以「緣行相(條件的模式)」(paccayākāraṃ)解說,並說,因為緣行相互相緣使伴隨的法生起(Paccayākāro hi aññamaññaṃ paṭicca sahite dhamme uppādeti, \suttaref{SN.12.1}),因此被稱為「緣起」。有關緣起十二支的廣說,參看\ccchref{SA.298}{https://agama.buddhason.org/SA/dm.php?keyword=298}。
\item\subnoteref{225.1}\NoteSubKeyHead{(2)}\NoteKeywordAgamaHead{「緣生法(SA);從因緣起/因緣起所生法(MA)」},南傳作\NoteKeywordNikaya{「緣所生法」}(paṭiccasamuppannā dhammā),菩提比丘長老英譯為\NoteKeywordBhikkhuBodhi{「依之而發生的事象」}(dependently arisen phenomena)。
\stopitemgroup

\startitemgroup[noteitems]
\item\subnoteref{226.0}\NoteSubKeyHead{(1)}\NoteKeywordAgamaHead{「彼岸;度無極(AA)」}(pārimaṃ tīraṃ, pāraṃ),菩提比丘長老英譯為\NoteKeywordBhikkhuBodhi{「遠岸」}(the far shore)。按:此指「解脫涅槃」,依《一切經音義》:「度無極(或言到彼岸,皆一義也,梵言:波羅蜜多是也)。」「度無極」為「波羅蜜多」(pāramitā)的另譯,「波羅蜜多」原意為「完美的;最高的事物」,後來大乘時期轉為指「到彼岸;涅槃」。
\item\subnoteref{226.1}\NoteSubKeyHead{(2)}\NoteKeywordNikayaHead{「此岸」}(orimaṃ tīraṃ, apāraṃ),菩提比丘長老英譯為\NoteKeywordBhikkhuBodhi{「近岸」}(the near shore)。按:此指「生死流轉」。
\item\subnoteref{226.2}\NoteSubKeyHead{(3)}\NoteKeywordNikayaHead{「涅槃為彼岸」}(nibbānaparāyanaṃ),菩提比丘長老英譯為\NoteKeywordBhikkhuBodhi{「涅槃為其到達地」}(nibbana as its destination)。
\stopitemgroup

\startitemgroup[noteitems]
\item\subnoteref{227.0}\NoteKeywordNikayaHead{「在厭逆上」}(paṭikūle),菩提比丘長老英譯為\NoteKeywordBhikkhuBodhi{「厭惡的」}(the repulsive)。按:「厭逆」,《顯揚真義》以「不合意的」(aniṭṭha, \suttaref{SN.46.54}),《吉祥悅意》以「就以看見而帶來嫌惡(厭惡;反感;排斥)」(dassaneneva paṭighāvaho, \ccchref{DN.23}{https://agama.buddhason.org/DN/dm.php?keyword=23})解說。「在厭逆上住於不厭逆想」等的理由,參看\ccchref{AN.5.144}{https://agama.buddhason.org/AN/an.php?keyword=5.144}。《小部/無礙解道》〈22.\ccchref{神通的談論}{https://agama.buddhason.org/Ps/Ps22.htm}〉第17段說,在不合意的對象上以慈佈滿(mettāya vā pharati),或從界聚焦(dhātuto vā upasaṃharati),這樣,在厭逆上住於不厭逆想;在令人想要的對象上以不淨佈滿,或從無常聚焦,這樣,在不厭逆上住於厭逆想;……。也就是說,不管對象如何,以慈佈滿,或從界聚焦住於不厭逆想;以不淨佈滿,或從無常聚焦住於厭逆想。
\stopitemgroup

\startitemgroup[noteitems]
\item\subnoteref{228.0}\NoteSubKeyHead{(1)}\NoteKeywordAgamaHead{「捨(SA/MA);護(DA/AA)」},南傳作\NoteKeywordNikaya{「平靜」}(upekkhā,另譯為「捨;無關心;捨心」),菩提比丘長老英譯為\NoteKeywordBhikkhuBodhi{「平靜」}(equanimity),並解說譯為「平靜」有兩個意義,在有關感受上表示「中性的感受」(neutral feeling),而在心理素質 (mental quality)上表示「心理的中立、不偏,或心的平衡」(mental neutrality, impartiality, or balance of mind),在《阿毘達磨》中用在「行蘊」(saṅkhārakkhandha)上稱為「中捨性;處中性」(tatramajjhattatā),在七覺支中的「捨覺支;平靜覺支」則指「心理的平衡」,在禪定中則是指第三禪與第四禪的境界。
\item\subnoteref{228.1}\NoteSubKeyHead{(2)}\NoteKeywordAgamaHead{「捨根(SA)」},南傳作\NoteKeywordNikaya{「平靜根」}(upekkhindriyaṃ),菩提比丘長老英譯為\NoteKeywordBhikkhuBodhi{「平靜的機能」}(the equanimity faculty)。
\stopitemgroup

\startitemgroup[noteitems]
\item\subnoteref{229.0}\NoteKeywordAgamaHead{「於未來世成不生法(SA);則更不生(GA);則不復生(DA)」},南傳作\NoteKeywordNikaya{「為未來不生之物」}(āyatiṃ anuppādadhammā,直譯為「未來不生法」),菩提比丘長老英譯為\NoteKeywordBhikkhuBodhi{「因此它不再是屬於未來起來(出現)之物」}(so that it is no more subject to future arising)。按:「法」(dhamma),不是指「正法」,而是表示「屬於……之物」(subject to)。
\stopitemgroup

\startitemgroup[noteitems]
\item\subnoteref{230.0}\NoteKeywordAgamaHead{「意行;意所願(\ccchref{MA.143}{https://agama.buddhason.org/MA/dm.php?keyword=143})」},南傳作\NoteKeywordNikaya{「意行」}(manosaṅkhāro),或「心行」(cittasaṅkhāro, cittasaṅkhāraṃ),菩提比丘長老兩者都英譯為「精神的意志形成」(the mental volitional formation, SN),或「心的活動」(mental activities, AN, 在\ccchref{AN.3.61}{https://agama.buddhason.org/AN/an.php?keyword=3.61}中說這屬第二禪以上),Maurice Walshe先生英譯為「心-力」(mind-force, \ccchref{DN.28}{https://agama.buddhason.org/DN/dm.php?keyword=28})。按:《吉祥悅意》等以「心行」(cittasaṅkhārā, \ccchref{DN.28}{https://agama.buddhason.org/DN/dm.php?keyword=28}/\ccchref{AN.3.61}{https://agama.buddhason.org/AN/an.php?keyword=3.61})解說「意行」,\ccchref{MN.44}{https://agama.buddhason.org/MN/dm.php?keyword=44}說,想與受是心行,註疏也以「受想」(vedanāsaññā, \ccchref{DN.28}{https://agama.buddhason.org/DN/dm.php?keyword=28})解說「心行」。
\stopitemgroup

\startitemgroup[noteitems]
\item\subnoteref{231.0}\NoteKeywordAgamaHead{「結夏安居/夏安居(SA);夏坐安居(GA);夏坐(MA);安居/夏安居(DA);夏坐(AA)」},南傳作\NoteKeywordNikaya{「雨季安居」}(vassāvāsaṃ, vassaṃ, vassavāsaṃ,另譯為「雨安居;雨季的住所」,vassaṃvuttho則為「住過雨季安居」),菩提比丘長老英譯為\NoteKeywordBhikkhuBodhi{「雨季駐留」}(the rains residence)。按:佛陀制定在雨季期間,出家弟子應在一處住,不再遊行,避免行走的困難與危險。因雨季落在夏天的三個月,所以也譯為「結夏安居」。
\stopitemgroup

\startitemgroup[noteitems]
\item\subnoteref{232.0}\NoteKeywordNikayaHead{「六識身」},南傳作\NoteKeywordNikaya{「六類識」}(cha viññāṇakāyā,逐字譯為「六識身」),菩提比丘長老英譯為\NoteKeywordBhikkhuBodhi{「六個識的種類;六種識」}(The six classes of consciousness)。按:「身」(kāya),不是指「身體」,而是「種類」的意思。
\stopitemgroup

\startitemgroup[noteitems]
\item\subnoteref{233.0}\NoteKeywordAgamaHead{「不壞淨;無壞信/不壞信(DA)」},南傳作\NoteKeywordNikaya{「不壞淨」}(aveccappasāda,另譯作「不壞信;證淨;絕對的淨信;確知而得的淨信」),菩提比丘長老英譯為\NoteKeywordBhikkhuBodhi{「無瑕的信任」}(perfect confidence)。按:《顯揚真義》等以「不動搖的淨信(明淨)」(acalappasādena, \suttaref{SN.40.10}/\ccchref{MN.7}{https://agama.buddhason.org/MN/dm.php?keyword=7}/\ccchref{DN.18}{https://agama.buddhason.org/DN/dm.php?keyword=18}),《滿足希求》以「知德行不壞(guṇe avecca)後生起的不動搖的淨信(acalappasāde, \ccchref{AN.3.76}{https://agama.buddhason.org/AN/an.php?keyword=3.76})」解說。
\stopitemgroup

\startitemgroup[noteitems]
\item\subnoteref{234.0}\NoteKeywordAgamaHead{「兩舌;鬪亂彼此(AA)」},南傳作\NoteKeywordNikaya{「離間語」}(pisuṇā vācā, pisuṇavāco, pesuñña,另譯為「挑撥的話」),菩提比丘長老英譯為\NoteKeywordBhikkhuBodhi{「有惡意的話」}(malicious speech, MN),或「分化的話」(divisive speech, SN/AN)。
\stopitemgroup

\startitemgroup[noteitems]
\item\subnoteref{235.0}\NoteKeywordAgamaHead{「惡口(SA/AA);麁澀言(GA);麁言(MA);惡罵(DA)」},南傳作\NoteKeywordNikaya{「粗惡語」}(pharusā vācā,另譯為「粗暴語;粗魯苛薄的言語」),菩提比丘長老英譯為\NoteKeywordBhikkhuBodhi{「粗暴的話」}(harsh speech)。
\stopitemgroup

\startitemgroup[noteitems]
\item\subnoteref{236.0}\NoteKeywordNikayaHead{「綺語」},南傳作\NoteKeywordNikaya{「雜穢語」}(samphappalāpo,另譯為「綺語;輕率的廢話」),菩提比丘長老英譯為\NoteKeywordBhikkhuBodhi{「閒聊」}(gossip)。
\stopitemgroup

\startitemgroup[noteitems]
\item\subnoteref{237.0}\NoteKeywordNikayaHead{「意業」}(manokamma),菩提比丘長老英譯為\NoteKeywordBhikkhuBodhi{「由心所作的行為;心理行為」}(action one does by mind, mental acts, mental action, mental kamma)。按:《破斥猶豫》說,在(身、語)兩個門上未達攪動(copanaṃ appatvā)而在意門生起二十九種善惡思(manodvāre uppannā ekūnatiṃsakusalākusalacetanā, \ccchref{MN.56}{https://agama.buddhason.org/MN/dm.php?keyword=56}),名為意業。二十九種思,《清淨道論》說,福行有十三種善思(八種欲界善思及因修習而轉起的五種色界善思),非福行有十二種不善思,不動行有四種無色界善思,以身的思有身行,以語的思有語行,以意的思有心行(manosañcetanā cittasaṅkhāro, 17.592)。
\stopitemgroup

\startitemgroup[noteitems]
\item\subnoteref{238.0}\NoteKeywordNikayaHead{「有漏」}(bhavāsavo),菩提比丘長老英譯為\NoteKeywordBhikkhuBodhi{「存在的污點」}(the taint of being)。按:「有」(bhava),為愛、取、有的「有」,《破斥猶豫》以「對色、無色界的欲貪」(rupārūpabhave chandarāgo, \ccchref{MN.2}{https://agama.buddhason.org/MN/dm.php?keyword=2})解說。
\stopitemgroup

\startitemgroup[noteitems]
\item\subnoteref{239.0}\NoteKeywordNikayaHead{「威儀」},南傳作\NoteKeywordNikaya{「舉止」}(iriyaṃ,另譯為「威儀;行動」),菩提比丘長老英譯為\NoteKeywordBhikkhuBodhi{「行為的方式」}(way of conduct)。按:《顯揚真義》以「行為(慣習)、行、行境、住、行道(實踐)」(vuttiṃ ācāraṃ gocaraṃ vihāraṃ paṭipattiṃ, \suttaref{SN.12.31})解說。
\stopitemgroup

\startitemgroup[noteitems]
\item\subnoteref{240.0}\NoteKeywordNikayaHead{「結跏趺坐;結加趺坐」},南傳作\NoteKeywordNikaya{「盤腿而坐」}(pallaṅkena nisīdati),菩提比丘長老英譯為\NoteKeywordBhikkhuBodhi{「腿交叉而坐」}(seated crosslegged)。Maurice Walshe先生英譯為「腿交叉坐下」(sat down cross-legged)。另外,菩提比丘長老也將「盤腿;跏趺」(pallaṅkaṃ)英譯為「腿交叉而坐」(seated crosslegged)。
\stopitemgroup

\startitemgroup[noteitems]
\item\subnoteref{241.0}\NoteKeywordNikayaHead{「四食」},南傳作\NoteKeywordNikaya{「四種食」}(cattārome āhārā),菩提比丘長老英譯為\NoteKeywordBhikkhuBodhi{「四種滋養物」}(four kinds of nutriment)。參看\ccchref{SA.371}{https://agama.buddhason.org/SA/dm.php?keyword=371}。
\stopitemgroup

\startitemgroup[noteitems]
\item\subnoteref{242.0}\NoteKeywordNikayaHead{「證知」}(abhijānaṃ),菩提比丘長老英譯為\NoteKeywordBhikkhuBodhi{「直接的理解」}(direct knowledge, directly knowing),或「記得」(recall),並說,只有解脫阿羅漢才適合說「遍知(parijānāti, 動詞)」,而只有初果以上的聖者(有學),才適合說「證知(abhijānāti, 動詞)」(\suttaref{SN.22.23}),或「有證智的」(abhiñña, 形容詞)。
\stopitemgroup

\startitemgroup[noteitems]
\item\subnoteref{243.0}\NoteKeywordAgamaHead{「辟支佛;緣覺(SA/AA)」},南傳作\NoteKeywordNikaya{「辟支佛」}(paccekasambuddhaṃ, paccekabuddha,義譯為「獨一的正覺者;獨覺」),菩提比丘長老英譯照錄,並解說辟支佛是獨立於遍正覺者(sammā sambuddha)之外達到正覺(開化)者,但不像遍正覺者,他沒建立教說(sāsana),他們被認為只在沒有佛陀教說存在的世間出現(\suttaref{SN.3.20})。
\stopitemgroup

\startitemgroup[noteitems]
\item\subnoteref{244.0}\NoteKeywordAgamaHead{「欲愛、有愛、無有愛(DA/AA)」},南傳作\NoteKeywordNikaya{「欲的渴愛、有的渴愛、虛無的渴愛」}(kāmataṇhā, bhavataṇhā, vibhavataṇhā),菩提比丘長老英譯為\NoteKeywordBhikkhuBodhi{「感官快樂的渴愛、存在的渴愛、根絕的渴愛」}(craving for sensual pleasures, craving for existence, craving for extermination)。按:此即「三愛;三類渴愛」(tisso taṇhā),為「意行」(manosaṅkhāro),或「心行」(cittasaṅkhāro)的根源。《顯揚真義》等說,色、無色界的貪(rūpārūpabhavarāgo),禪定的欲求,與常見俱行的貪(sassatadiṭṭhisahagato rāgoti),這名為有的渴愛,與斷滅見俱行的貪(ucchedadiṭṭhisahagato rāgo, \suttaref{SN.22.22}/\ccchref{MN.9}{https://agama.buddhason.org/MN/dm.php?keyword=9}/\ccchref{DN.15}{https://agama.buddhason.org/DN/dm.php?keyword=15}),這名為虛無的渴愛。長老認為,有的渴愛更像繼續生存的原始想要,不論有見與否,虛無的渴愛為基於滅絕能帶來真我的潛在傾向之完全結束生存的想要。
\stopitemgroup

\startitemgroup[noteitems]
\item\subnoteref{245.0}\NoteKeywordAgamaHead{「隨護斷(SA)」},南傳作\NoteKeywordNikaya{「隨護的勤奮」}(anurakkhaṇāppadhānaṃ),菩提比丘長老英譯為\NoteKeywordBhikkhuBodhi{「以保護而努力」}(Striving by protection)。
\stopitemgroup

\startitemgroup[noteitems]
\item\subnoteref{246.0}\NoteSubKeyHead{(1)}\NoteKeywordAgamaHead{「受持齋戒(SA);持齋(MA/AA)」},南傳作\NoteKeywordNikaya{「入布薩」}(Uposathaṃ upavasanti, Upavutthassa, uposatho upavuttho bhavati,另譯為「近住布薩」),菩提比丘長老英譯為\NoteKeywordBhikkhuBodhi{「舉行布薩日」}(observe the Uposatha days, SN),或「舉行(遵守)布薩」(observe the uposatha, AN)。
\item\subnoteref{246.1}\NoteSubKeyHead{(2)}\NoteKeywordNikayaHead{「作布薩」}(uposathaṃ karoti),菩提比丘長老英譯為\NoteKeywordBhikkhuBodhi{「進行布薩」}(conduct the uposatha)。
\item\subnoteref{246.2}\NoteSubKeyHead{(3)}\NoteKeywordNikayaHead{「所入的布薩」}(uposatho upavuttho),菩提比丘長老英譯為\NoteKeywordBhikkhuBodhi{「遵守…布薩」}(observed…the uposatha, AN)。
\stopitemgroup

\startitemgroup[noteitems]
\item\subnoteref{247.0}\NoteKeywordAgamaHead{「一坐而食(SA/AA);一座食(MA);一坐一食(AA);一坐食/一食(摩訶僧祇律)」},南傳作\NoteKeywordNikaya{「一座食」}(ekāsanabhojanaṃ),菩提比丘長老英譯為\NoteKeywordBhikkhuBodhi{「在單一期間吃」}(eat at a single session),並解說這是指「日中一食」(eating a single meal in the forenoon only),而這只是建議,並非強制(but not required),因為根據「波羅提木叉」(pātimokkha),比丘只在午後到隔日黎明前被禁止進食。按:\ccchref{MA.194}{https://agama.buddhason.org/MA/dm.php?keyword=194}稱「一座食戒」,\ccchref{MN.65}{https://agama.buddhason.org/MN/dm.php?keyword=65}稱此為「學處」(sikkhāpade),可見有一個時期的「波羅提木叉」應該定有此條,應有強制效力,現存的「波羅提木叉」沒有此條,應該是佛陀立了後又開(解除或修訂)了吧。
\stopitemgroup

\startitemgroup[noteitems]
\item\subnoteref{248.0}\NoteKeywordAgamaHead{「檀越(SA/AA);施主(MA/DA);檀越施主/施主檀越(AA)」},南傳作\NoteKeywordNikaya{「施主(們)」}(dānapatī),菩提比丘長老英譯為\NoteKeywordBhikkhuBodhi{「捐款人;捐贈者」}(the donors)。
\stopitemgroup

\startitemgroup[noteitems]
\item\subnoteref{249.0}\NoteKeywordAgamaHead{「頭陀法/頭陀苦行(SA/AA),行頭陀/頭陀行/頭陀(AA)」},南傳作\NoteKeywordNikaya{「頭陀論者」}(dhutavādā),菩提比丘長老英譯為\NoteKeywordBhikkhuBodhi{「實行苦行的擁護者」}(proponents of the ascetic practices, SN),或「解說實行苦行者」(who expound the ascetic practices, AN)。
\stopitemgroup

\startitemgroup[noteitems]
\item\subnoteref{250.0}\NoteKeywordNikayaHead{「慚(慙)」}(hiri, 動詞hiriyati, harāyati),菩提比丘長老英譯為\NoteKeywordBhikkhuBodhi{「在道德上的羞恥」}(moral shame, AN),或「羞恥感」(a sense of shame, SN),並解說,這被導向內在,從自我尊重生起,引導人們基於自己固有的尊嚴拒絕作壞事(\ccchref{AN.2.8}{https://agama.buddhason.org/AN/an.php?keyword=2.8})。
\stopitemgroup

\startitemgroup[noteitems]
\item\subnoteref{251.0}\NoteKeywordNikayaHead{「愧」}(ottappa, 動詞ottappati),菩提比丘長老英譯為\NoteKeywordBhikkhuBodhi{「在道德上的畏敬」}(moral dread, AN),或「害怕做錯的」(afraid of wrongdoing, SN),並解說,這是向外的,從害怕責備生起,引導人們基於害怕後果而拒絕作壞事(\ccchref{AN.2.8}{https://agama.buddhason.org/AN/an.php?keyword=2.8})。
\stopitemgroup

\startitemgroup[noteitems]
\item\subnoteref{252.0}\NoteKeywordNikayaHead{「緣;緣於」}(paṭicca),菩提比丘長老英譯為\NoteKeywordBhikkhuBodhi{「依於」}(in dependence on)。
\stopitemgroup

\startitemgroup[noteitems]
\item\subnoteref{253.0}\NoteSubKeyHead{(1)}\NoteKeywordAgamaHead{「隨使使/所使(SA);著而著(MA)」},南傳作\NoteKeywordNikaya{「潛伏」}(anuseti,動詞,另譯為「隨眠;潛在」),菩提比丘長老英譯為\NoteKeywordBhikkhuBodhi{「留存;縈繞」}(persists; linger, AN),或「有趨勢於」(has a tendency towards, \suttaref{SN.12.38}),智髻比丘長老英譯為「躺在下面」(underlie, MN)。
\item\subnoteref{253.1}\NoteSubKeyHead{(2)}\NoteKeywordNikayaHead{「使」},南傳作\NoteKeywordNikaya{「煩惱潛在趨勢」}(anusayo,名詞,另譯為「隨眠;煩惱;使」),Maurice Walshe先生英譯為「潛在傾向」(latent proclivity),菩提比丘長老英譯為\NoteKeywordBhikkhuBodhi{「表面下的趨勢」}(the underlying tendency),並說,在論書中污染的發生依序為「煩惱潛在趨勢」(anusaya,心中的潛在性質)、「纏」(pariyuṭṭhāna,纏住與征服了心)、「違犯」(vītikkama,激發不善的身語行)三階段(\ccchref{MN.64}{https://agama.buddhason.org/MN/dm.php?keyword=64}),與《大毘婆沙論》所說「分別論者又說隨眠是纏種子……纏從隨眠生」相合。《眾事分阿毘曇論》說:「微細是使義;使是使義;隨入是使義;隨逐是使義。」同本異譯《品類足論》說:「微細義是隨眠義;隨增義是隨眠義;隨逐義是隨眠義;隨縛義是隨眠義。」
\stopitemgroup

\startitemgroup[noteitems]
\item\subnoteref{254.0}\NoteKeywordAgamaHead{「不墮數/不墮於眾數(SA)」},南傳作\NoteKeywordNikaya{「不來到稱呼」}(saṅkhyaṃ nopeti,另譯為「不來到計算,不來到名稱」),菩提比丘長老英譯為\NoteKeywordBhikkhuBodhi{「不能被推想」}(cannot be reckoned)。按:《顯揚真義》以「不來到『貪染者、瞋怒者、變愚癡者』的安立(名字)」(ratto duṭṭho mūḷhoti paññattiṃ na upeti, \suttaref{SN.36.3})解說。
\stopitemgroup

\startitemgroup[noteitems]
\item\subnoteref{255.0}\NoteKeywordAgamaHead{「一心;心得一(MA);歛心專一(DA);專其一意/專其一心/專精一意(AA)」},南傳作\NoteSubEntryKey{(i)}\NoteKeywordNikaya{「心一境性」}(cittassa ekaggatā, cittekaggatā),菩提比丘長老英譯為\NoteKeywordBhikkhuBodhi{「心的統一」}(unification of mind)。\NoteSubEntryKey{(ii)}\NoteKeywordNikaya{「心的專一性」}(cetaso ekodibhāvaṃ),菩提比丘長老英譯為\NoteKeywordBhikkhuBodhi{「心的統一」}(unification of mind, \suttaref{SN.36.31})。\NoteSubEntryKey{(iii)}\NoteKeywordNikaya{「心一境的」}(ekaggacitto; cittaṃ ekaggaṃ),智髻比丘長老英譯為「心是統一的」(mind is unified, \ccchref{MN.29}{https://agama.buddhason.org/MN/dm.php?keyword=29})。
\stopitemgroup

\startitemgroup[noteitems]
\item\subnoteref{256.0}\NoteKeywordAgamaHead{「內淨(SA);內靜(MA);內信/內信歡悅(DA);內發歡喜/內自歡喜/內有喜心/內有歡喜(AA)」},南傳作\NoteKeywordNikaya{「自身內的明淨」}(ajjhattaṃ sampasādanaṃ,逐字譯為「內-淨」),菩提比丘長老英譯為\NoteKeywordBhikkhuBodhi{「內在的自信」}(internal confidence)。按:《阿毘曇毘婆沙論》說:「第二禪有四枝:內信、喜、樂、一心。」
\stopitemgroup

\startitemgroup[noteitems]
\item\subnoteref{257.0}\NoteKeywordAgamaHead{「信解(MA);意解(AA)」},南傳作\NoteKeywordNikaya{「勝解;信解;志向」}(adhimucca, adhimutti, adhimuccati, adhimutta),菩提比丘長老英譯為\NoteKeywordBhikkhuBodhi{「決心;決定;融入」}(resolve, SN),或「意圖;意向」(intent upon, \suttaref{SN.35.132}),或「傾向」(were disposed, \suttaref{SN.35.245}),或「專注於」(focus on, AN),智髻比丘長老英譯為「決定」(decided, MN),或「性向」(inclination, MN)。
\stopitemgroup

\startitemgroup[noteitems]
\item\subnoteref{258.0}\NoteKeywordNikayaHead{「耆闍崛山(中)」}(gijjhakūṭe pabbate),為古音譯,義譯為「鷲峰山;靈鷲山」。
\stopitemgroup

\startitemgroup[noteitems]
\item\subnoteref{259.0}\NoteKeywordAgamaHead{「菴羅園(SA);菴婆林(DA)」},南傳作\NoteKeywordNikaya{「芒果園」}(ambavana),菩提比丘長老英譯為\NoteKeywordBhikkhuBodhi{「芒果樹林」}(Mango Grove)。
\stopitemgroup

\startitemgroup[noteitems]
\item\subnoteref{260.0}\NoteSubKeyHead{(1)}\NoteKeywordAgamaHead{「心之分齊(SA);他心智(SA/GA/MA/AA);他人心智(DA)」},南傳作\NoteKeywordNikaya{「他心智」}(cetopariyañāṇaṃ),菩提比丘長老英譯為\NoteKeywordBhikkhuBodhi{「圍繞……的心之理解」}(knowledge encompassing the minds of the……)。
\item\subnoteref{260.1}\NoteSubKeyHead{(2)}\NoteKeywordNikayaHead{「他心智」}(pariye ñāṇaṃ,原義為「理解智;看穿智」),Maurice Walshe先生英譯為「其他人心的理解」(knowledge of others' minds)。按:《吉祥悅意》引《分別論》說,對其他眾生、其他個人以心熟知心後知道(parasattānaṃ parapuggalānaṃ cetasā ceto paricca pajānāti, vibha. 796)為「他心智」(paresaṃ cittaparicchede ñāṇaṃ, \ccchref{DN.33}{https://agama.buddhason.org/DN/dm.php?keyword=33})。
\stopitemgroup

\startitemgroup[noteitems]
\item\subnoteref{261.0}\NoteKeywordNikayaHead{「欲有、色有、無色有」}(kāmabhavo, rūpabhavo, arūpabhavo),菩提比丘長老英譯為\NoteKeywordBhikkhuBodhi{「感官-領域的存在,色-領域的存在,非色-領域的存在」}(sense-sphere existence, form-sphere existence, formless-sphere existence)。按:「無色」即「無色界」,「有」(bhavo),即「愛、取、有」的有。
\stopitemgroup

\startitemgroup[noteitems]
\item\subnoteref{262.0}\NoteSubKeyHead{(1)}\NoteKeywordNikayaHead{「釋迦族人」}(sakyā),Maurice Walshe先生夾註原文sāka英譯為「[壯如]柚木」([strong as ]teak)。按:sakyā,另作sakka,sakkoti的grd./a.,意為「能夠的」,《一切經音義》說,釋迦云能仁也,《吉祥悅意》也以「能夠的、有能力的」(samatthā, paṭibalāti, \ccchref{DN.20}{https://agama.buddhason.org/DN/dm.php?keyword=20})解說。
\item\subnoteref{262.1}\NoteSubKeyHead{(2)}\NoteKeywordAgamaHead{「釋種子;釋子(SA/MA/DA)」},南傳作\NoteKeywordNikaya{「釋迦之徒的」}(sakyaputtiyā,另譯為「釋迦子的;佛弟子」),菩提比丘長老英譯為\NoteKeywordBhikkhuBodhi{「追隨釋迦族的兒子」}(following the Sakyan son)。按:「釋迦人之子」(sakyaputta),指釋迦牟尼佛,「釋迦子的;釋迦之徒的」指隨世尊出家者。而\suttaref{SN.20.11}, \suttaref{SN.20.12}中所指的,《顯揚真義》說是提婆達多(Devadatta)。
\stopitemgroup

\startitemgroup[noteitems]
\item\subnoteref{263.0}\NoteKeywordAgamaHead{「天帝釋;帝釋;釋提桓因(SA/DA/AA);天王釋(SA/MA)」},南傳作\NoteKeywordNikaya{「天帝釋」}(sakko devānamindo,音譯為「釋提桓因陀羅」,義譯為「釋-諸天之王」,有時簡為「因陀羅-indo」),菩提比丘長老英譯為\NoteKeywordBhikkhuBodhi{「神的統治者,Sakka」}(Sakka, ruler of gods或Sakka, lord of the devas)。按:天帝釋名字為「拘翼;俱尸迦;憍尸迦」(kosiya,意譯為「梟;貓頭鷹」),別名「千眼、帝釋、舍脂夫、須闍之夫、襪瑟哇、摩伽婆(婆娑婆、娑婆婆)、釋脂之夫摩佉婆」,參看\ccchref{SA.1106}{https://agama.buddhason.org/SA/dm.php?keyword=1106}。
\stopitemgroup

\startitemgroup[noteitems]
\item\subnoteref{264.0}\NoteSubKeyHead{(1)}\NoteKeywordAgamaHead{「優鉢羅(SA/DA);優鉢/憂鉢(AA)」},南傳作\NoteKeywordNikaya{「青蓮花」}(uppala,另譯為「水蓮」),菩提比丘長老英譯為\NoteKeywordBhikkhuBodhi{「青蓮」}(blue lotus)。
\item\subnoteref{264.1}\NoteSubKeyHead{(2)}\NoteKeywordAgamaHead{「鉢曇摩(SA);鉢頭摩(MA/DA/AA)」},南傳作\NoteKeywordNikaya{「紅蓮花」}(paduma,另譯為「蓮、蓮花、赤蓮、波頭摩」),菩提比丘長老英譯為\NoteKeywordBhikkhuBodhi{「紅蓮」}(red lotus)。
\item\subnoteref{264.2}\NoteSubKeyHead{(3)}\NoteKeywordAgamaHead{「拘牟頭(SA/AA);俱物頭(MA/DA);鳩勿頭(DA);拘勿頭/拘牟陀(AA)」},南傳作\NoteKeywordNikaya{「黃蓮花」}(kumuda,另譯為「蓮、白睡蓮」)。
\item\subnoteref{264.3}\NoteSubKeyHead{(4)}\NoteKeywordAgamaHead{「分陀利(SA/DA/AA);芬陀利(SA)」},南傳作\NoteKeywordNikaya{「蓮花」}(puṇḍarīka,另譯為「白蓮」),菩提比丘長老英譯為\NoteKeywordBhikkhuBodhi{「白蓮」}(white lotus)。
\stopitemgroup

\startitemgroup[noteitems]
\item\subnoteref{265.0}\NoteSubKeyHead{(1)}\NoteKeywordAgamaHead{「無想心三昧(SA);無想心定(MA);無想定(DA)」},南傳作\NoteKeywordNikaya{「無相心定」}(animittaṃ cetosamādhiṃ),菩提比丘長老英譯為\NoteKeywordBhikkhuBodhi{「心之無形跡的集中貫注」}(the signless concentration of mind, SN/MN),或「無標記之心的集中貫注」(the markless mental concentration, AN)。按:《顯揚真義》說,這是關於捨斷常相等後,被轉起的毘婆舍那定(vipassanāsamādhiṃyeva, \suttaref{SN.40.9});無相定是一種毘婆舍那定,以常相等的根除(niccanimittādīnaṃ samugghātanena, \suttaref{SN.22.80})而被稱為無相,《破斥猶豫》以「毘婆舍那心定」(vipassanācittasamādhiṃ, \ccchref{MN.121}{https://agama.buddhason.org/MN/dm.php?keyword=121})解說,《滿足希求》說,無相是強力的毘婆舍那定(balavavipassanāsamādhiṃ, \ccchref{AN.7.56}{https://agama.buddhason.org/AN/an.php?keyword=7.56})。
\item\subnoteref{265.1}\NoteSubKeyHead{(2)}\NoteKeywordAgamaHead{「無相心三昧/無相心正受(SA)」},南傳作\NoteKeywordNikaya{「無相心解脫」}(animittā cetovimutti),菩提比丘長老英譯為\NoteKeywordBhikkhuBodhi{「無形跡的心的釋放」}(the signless liberation of mind, SN),或「無標記的心釋放」(the markless liberation of the mind, AN)。按:《顯揚真義》說,這是13法:毘婆舍那(除去常相、樂相、我相)、四無色[定](色相不存在)、四道與四果(以相的作者之雜染(nimittakarānaṃ kilesānaṃ)不存在, \suttaref{SN.41.7})。又,依\suttaref{SN.41.7}、\ccchref{MN.43}{https://agama.buddhason.org/MN/dm.php?keyword=43},進入後住於「無相心定」被稱為「無相心解脫」,依\ccchref{SA.567}{https://agama.buddhason.org/SA/dm.php?keyword=567},「無想心三昧」即「無相三昧」,依\ccchref{SA.272}{https://agama.buddhason.org/SA/dm.php?keyword=272}與\suttaref{SN.22.80}的比對,「無相三昧」即「無相定」的另譯,依\ccchref{MA.211}{https://agama.buddhason.org/MA/dm.php?keyword=211}與\ccchref{MN.43}{https://agama.buddhason.org/MN/dm.php?keyword=43}的比對,「無想定」即「無相心解脫等至」。
\stopitemgroup

\startitemgroup[noteitems]
\item\subnoteref{266.0}\NoteSubKeyHead{(1)}\NoteKeywordNikayaHead{「非時」},南傳作\NoteKeywordNikaya{「不適時的」}(Akālo),菩提比丘長老英譯為\NoteKeywordBhikkhuBodhi{「非時機」}(not the time)。
\item\subnoteref{266.1}\NoteSubKeyHead{(2)}\NoteKeywordAgamaHead{「非時;不待時節(SA)」},南傳作\NoteKeywordNikaya{「即時的」}(akāliko),菩提比丘長老英譯為\NoteKeywordBhikkhuBodhi{「即時的」}(immediate)。
\item\subnoteref{266.2}\NoteSubKeyHead{(3)}\NoteKeywordNikayaHead{「非時食;離時之食」}(vikālabhojanā),菩提比丘長老英譯為\NoteKeywordBhikkhuBodhi{「在不適當時間飲食」}(eating at improper times, SN),智髻比丘長老英譯為「離開適當時間的膳食」(meal outside the proper time, MN)。
\stopitemgroup

\startitemgroup[noteitems]
\item\subnoteref{267.0}\NoteKeywordAgamaHead{「野干(SA/DA);豺/豺狼(MA);野狐/狐(AA)」},南傳作\NoteKeywordNikaya{「狐狼」}(siṅgālaṃ),菩提比丘長老英譯為\NoteKeywordBhikkhuBodhi{「狐狼」}(jackal)。
\stopitemgroup

\startitemgroup[noteitems]
\item\subnoteref{268.0}\NoteKeywordAgamaHead{「淨解脫(MA/DA)」},南傳作\NoteKeywordNikaya{「只志向『清淨的』」}(Subhanteva adhimutto hoti,另譯為「只勝解清淨的」),菩提比丘長老英譯為\NoteKeywordBhikkhuBodhi{「只決心於美的者」}(One is resolved only on 'beautiful.', AN),坦尼沙羅比丘長老英譯為「只熱心於美的者」(One is intent only on the beautiful),Maurice Walshe先生英譯為「想著:『它是美的。』人們專心於它」(Thinking: "It is beautiful", one becomes intent on it),並解說,這是經由專注於純淨明亮顏色的「遍處」(kasiṇa, \ccchref{DN.15}{https://agama.buddhason.org/DN/dm.php?keyword=15})。
\stopitemgroup

\startitemgroup[noteitems]
\item\subnoteref{269.0}\NoteKeywordAgamaHead{「百千歲(SA);千百歲(AA)」},南傳作\NoteKeywordNikaya{「十萬年」}(vassasatasahassāni,逐字譯為「年-百-千」,另譯為「百千歲;百千年」),菩提比丘長老英譯為\NoteKeywordBhikkhuBodhi{「千的百倍年」}(hundreds of thousands of years)。
\stopitemgroup

\startitemgroup[noteitems]
\item\subnoteref{270.0}\NoteKeywordNikayaHead{「僧伽梨」},南傳作\NoteKeywordNikaya{「大衣」}(saṅghāṭi,音譯為「僧伽梨;僧迦梨;僧迦利」,另譯為「重衣」),菩提比丘長老英譯為\NoteKeywordBhikkhuBodhi{「外衣」}(outer robe)。
\stopitemgroup

\startitemgroup[noteitems]
\item\subnoteref{271.0}\NoteKeywordAgamaHead{「內外身身觀/內外身觀(SA);觀內外身如身(MA);內外身觀(DA);內外觀身身(AA)」},南傳作\NoteKeywordNikaya{「在內外的身上隨看著身」}(ajjhattabahiddhā kāye kāyānupassī),菩提比丘長老英譯為\NoteKeywordBhikkhuBodhi{「在內在與外在的身體凝視著身體」}(contemplating the body in the body internally and externally)。按:《破斥猶豫》等說,有時(kālena-在適當時間)在自己的吸氣與呼氣之身(attano vā assāsapassāsakāye)、有時在他人的吸氣與呼氣之身(parassa assāsapassāsakāye),以這些保持熟練業處後(paguṇakammaṭṭhānaṃ aṭṭhapetvā, \ccchref{MN.10}{https://agama.buddhason.org/MN/dm.php?keyword=10}/\ccchref{DN.22}{https://agama.buddhason.org/DN/dm.php?keyword=22}),屢屢適時徘徊而說,但這並非在同一時中得到兩者。
\stopitemgroup

\startitemgroup[noteitems]
\item\subnoteref{272.0}\NoteKeywordNikayaHead{「大雪山」}(Himavant,喜馬拉雅山),南傳作\NoteKeywordNikaya{「須彌山山王」}(sinerupabbatarājā),菩提比丘長老英譯為\NoteKeywordBhikkhuBodhi{「須彌,諸山之王」}(Sineru, the king of mountains)。
\stopitemgroup

\startitemgroup[noteitems]
\item\subnoteref{273.0}\NoteSubKeyHead{(1)}\NoteKeywordNikayaHead{「盡」},南傳作\NoteKeywordNikaya{「滅盡」}(khaya,另譯為「盡;盡滅」),菩提比丘長老英譯為\NoteKeywordBhikkhuBodhi{「破壞」}(destruction)。
\item\subnoteref{273.1}\NoteSubKeyHead{(2)}\NoteKeywordAgamaHead{「滅盡法(SA/MA);斷法(SA);盡法(MA/DA)」},南傳作\NoteKeywordNikaya{「滅盡法/滅盡法的狀態」}(khayadhammo/khayadhammatā),菩提比丘長老英譯為\NoteKeywordBhikkhuBodhi{「屬於破壞者」}(subject to destruction)。
\stopitemgroup

\startitemgroup[noteitems]
\item\subnoteref{274.0}\NoteKeywordNikayaHead{「南無」},南傳作\NoteKeywordNikaya{「禮敬」}(Namo),菩提比丘長老英譯為\NoteKeywordBhikkhuBodhi{「尊敬;臣服禮」}(Homage)。
\stopitemgroup

\startitemgroup[noteitems]
\item\subnoteref{275.0}\NoteKeywordAgamaHead{「邊際究竟/究竟邊際(SA);得至究竟(MA);究竟(DA/AA)」},南傳作\NoteKeywordNikaya{「究竟終結的」}(accantaniṭṭho,另譯為「徹底完成的」),菩提比丘長老英譯為\NoteKeywordBhikkhuBodhi{「到達終極的結尾者」}(one who has reached the ultimate end)。按:《顯揚真義》以「終極被征服的終結;常恆的終結」(antaṃ atikkantaniṭṭho satataniṭṭhā, \suttaref{SN.22.4}),《吉祥悅意》以「終極被稱為破壞,這些的終極、過去為終結」(anto vuccati vināso, antaṃ atītā niṭṭhā etesanti, \ccchref{DN.21}{https://agama.buddhason.org/DN/dm.php?keyword=21}),《滿足希求》以「終極被征服的終結;不動搖的終結(akuppaniṭṭho);常恆的終結(dhuvaniṭṭhoti, \ccchref{AN.3.144}{https://agama.buddhason.org/AN/an.php?keyword=3.144})」解說,《破斥猶豫》以「滅盡、消散、終極、過去」(khayavayasaṅkhātaṃ antaṃ atītāti, \ccchref{MN.37}{https://agama.buddhason.org/MN/dm.php?keyword=37})解說「究竟」。
\stopitemgroup

\startitemgroup[noteitems]
\item\subnoteref{276.0}\NoteKeywordAgamaHead{「不涌、不沒(SA)」},南傳作\NoteKeywordNikaya{「不彎曲、不彎離」}(na cābhinato na cāpanato),菩提比丘長老英譯為\NoteKeywordBhikkhuBodhi{「不前傾,不後傾」}(does not lean forward and does not lean back)。按:《顯揚真義》說,心隨順貪(Rāgānugataṃ)名為彎曲,隨順瞋(dosānugataṃ, \suttaref{SN.1.38})名彎離。
\stopitemgroup

\startitemgroup[noteitems]
\item\subnoteref{277.0}\NoteKeywordAgamaHead{「居家白衣(SA);在家白衣(SA/MA/AA)」},南傳作\NoteKeywordNikaya{「白衣在家人」}(gihī odātavasano),菩提比丘長老英譯為\NoteKeywordBhikkhuBodhi{「穿白衣的俗人」}(a layman clothed in white)。按:「白衣」、「在家人」為同義複詞。
\stopitemgroup

\startitemgroup[noteitems]
\item\subnoteref{278.0}\NoteKeywordAgamaHead{「轉輪王;轉輪聖王(SA/DA/AA)」},南傳作\NoteKeywordNikaya{「轉輪王」}(Rañño…cakkavattissa),菩提比丘長老英譯為\NoteKeywordBhikkhuBodhi{「轉輪君主」}(a wheel-turning monarch),或「全世界的君主」(a universal monarch, \ccchref{AN.4.247}{https://agama.buddhason.org/AN/an.php?keyword=4.247})。
\stopitemgroup

\startitemgroup[noteitems]
\item\subnoteref{279.0}\NoteKeywordAgamaHead{「有流(SA)」},南傳作\NoteSubEntryKey{(i)}\NoteKeywordNikaya{「有的暴流」}(bhavogho),菩提比丘長老英譯為\NoteKeywordBhikkhuBodhi{「存在的洪水」}(the flood of existence),《顯揚真義》以「對色、無色有的欲貪」(rūpārūpabhavesu chandarāgo)解說。\NoteSubEntryKey{(ii)}\NoteKeywordNikaya{「有之管道」}(bhavanetti),菩提比丘長老英譯為\NoteKeywordBhikkhuBodhi{「往存在的導管」}(the conduit to existence),Maurice Walshe先生英譯為「被綁向形成的連結」(the link that bound it to becoming)。按:有(bhava),就是愛、取、有的「有」,《顯揚真義》以有之繩(bhavarajju, \suttaref{SN.23.3})解說,長老說,這是「有的渴愛」(bhavataṇhā)之同義詞。
\stopitemgroup

\startitemgroup[noteitems]
\item\subnoteref{280.0}\NoteKeywordNikayaHead{「三十三天眾」}(tāvatiṃsakāyikā,逐字譯為「三十三天+身體的」),菩提比丘長老英譯為\NoteKeywordBhikkhuBodhi{「一大群三十三天」}(tāvatiṃsa host)。
\stopitemgroup

\startitemgroup[noteitems]
\item\subnoteref{281.0}\NoteKeywordNikayaHead{「以偈頌」}(gāthāya,原形gāthā,音譯為「伽陀」,或簡為「偈」),菩提比丘長老英譯為\NoteKeywordBhikkhuBodhi{「以詩」}(in verse)。按:巴利語之偈頌,對音節數目與音韻有其一定的規則,漢文的詩賦與之相當,古漢譯都以「詩」的形式譯,而比對菩提比丘長老的英譯,有時一些字句的順序與原文不同,作了大幅更動以譯出其真正的含意與韻味,譯者對此無力涉略,只能按原文直譯,不琢磨所譯字數與音韻的對丈。
\stopitemgroup

\startitemgroup[noteitems]
\item\subnoteref{282.0}\NoteKeywordNikayaHead{「天子」}(devaputto,逐字譯為「天+子」),菩提比丘長老英譯為\NoteKeywordBhikkhuBodhi{「年輕的神」}(the young deva)。
\stopitemgroup

\startitemgroup[noteitems]
\item\subnoteref{283.0}\NoteKeywordAgamaHead{「令智慧羸/令慧力羸/能羸智慧/慧力羸(SA);慧羸(MA)」},南傳作\NoteKeywordNikaya{「慧的減弱的」}(paññāya dubbalīkaraṇe),智髻比丘長老英譯為「削弱慧」(weaken wisdom,MN),菩提比丘長老英譯為\NoteKeywordBhikkhuBodhi{「慧的削弱者」}(weakeners of wisdom, SN),或「削弱慧」(weaken of wisdom, AN),Maurice Walshe先生英譯為「削弱理解」(weaken understanding, DN)。
\stopitemgroup

\startitemgroup[noteitems]
\item\subnoteref{284.0}\NoteKeywordAgamaHead{「歸依/自歸(SA/MA);歸依(GA/DA);歸命(MA/DA);自歸命/自歸(AA)」},南傳作\NoteKeywordNikaya{「歸依」}(saraṇaṃ gacchāmi),菩提比丘長老英譯為\NoteKeywordBhikkhuBodhi{「我前往依靠」}(I go for refuge to)。
\stopitemgroup

\startitemgroup[noteitems]
\item\subnoteref{285.0}\NoteKeywordNikayaHead{「給孤獨」}(anāthapiṇḍiko),為義譯,因為常施與孤獨無依者而得的別名,其本名為「須達多」(Sudatta),義譯為「善施」。
\stopitemgroup

\startitemgroup[noteitems]
\item\subnoteref{286.0}\NoteKeywordAgamaHead{「四念處(SA/MA/DA);四意止(AA)」},南傳作\NoteKeywordNikaya{「四念住」}(cattāro satipaṭṭhānā),菩提比丘長老英譯為\NoteKeywordBhikkhuBodhi{「四個深切注意的建立」}(four establishments of mindfulness),並解說「satipaṭṭhāna」一詞,論師們有兩種解讀法,若解讀為「念+現起」(sati+upaṭṭhāna),則為「念住;深切注意的建立」,若解讀為「念+出發點」(sati+paṭṭhāna),則為「念處;深切注意的基礎」,前者強調「建立念的行為」,後者強調「應用念的所緣」,雖然論師多傾向後者,但前者確定是更原始的。按:四念處被稱為「自處父母境界」(\ccchref{SA.617}{https://agama.buddhason.org/SA/dm.php?keyword=617})、「放牧處」(\ccchref{SA.1249}{https://agama.buddhason.org/SA/dm.php?keyword=1249})、「自洲自依」(\ccchref{SA.638}{https://agama.buddhason.org/SA/dm.php?keyword=638})。
\stopitemgroup

\startitemgroup[noteitems]
\item\subnoteref{287.0}\NoteKeywordNikayaHead{「五蓋」}(pañca nīvaraṇā),菩提比丘長老英譯為\NoteKeywordBhikkhuBodhi{「五個障礙」}(the five hindrances)。
\stopitemgroup

\startitemgroup[noteitems]
\item\subnoteref{288.0}\NoteKeywordAgamaHead{「上煩惱/惱/煩惱/覆(SA);穢(MA);垢穢/瑕穢(DA);結(AA)」},南傳作\NoteKeywordNikaya{「隨雜染」}(upakkilesā,另譯為「隨煩惱;小雜染;隨染,染污;垢穢,鏽」),智髻比丘長老英譯為「不完備;缺點」(the imperfections),菩提比丘長老英譯為\NoteKeywordBhikkhuBodhi{「敗壞」}(corruptions),並說此字有時指修定的缺點或不完備,有時指修觀的缺點或不完備,有時指從貪瞋癡三不善根生起的小污穢─三不善的模式或分枝(either as their modes or their offshoots, \ccchref{MN.7}{https://agama.buddhason.org/MN/dm.php?keyword=7})。
\stopitemgroup

\startitemgroup[noteitems]
\item\subnoteref{289.0}\NoteKeywordNikayaHead{「波羅提木叉」}(pātimokkha,另譯為「從解脫;別解脫;別別解脫;戒」)。按:這是佛陀所制訂有益於解脫的一條條(別別)學處(戒條)的編集,也稱為「波羅提木叉經;波羅提木叉修多羅;戒經」,《五分律》以「於諸善法最為初門」作字義解說,《翻譯名義集》說:「此云別解脫,謂:三業七支各各防非,別別解脫故。」
\stopitemgroup

\startitemgroup[noteitems]
\item\subnoteref{290.0}\NoteSubKeyHead{(1)}\NoteKeywordNikayaHead{「沙彌」}(sāmaṇera, samaṇuddesa),菩提比丘長老英譯為\NoteKeywordBhikkhuBodhi{「新加入之僧侶」}(a novice monk)。「舍羅」或即「沙彌」的音譯。
\item\subnoteref{290.1}\NoteSubKeyHead{(2)}\NoteKeywordNikayaHead{「沙彌尼」}(sāmaṇerī),菩提比丘長老英譯為\NoteKeywordBhikkhuBodhi{「見習修道女」}(probationary nun)。「舍羅磨尼離」或即「沙彌尼」的音譯。
\stopitemgroup

\startitemgroup[noteitems]
\item\subnoteref{291.0}\NoteKeywordNikayaHead{「精進根」},南傳作\NoteKeywordNikaya{「活力根」}(vīriyindriyaṃ),菩提比丘長老英譯為\NoteKeywordBhikkhuBodhi{「活力的機能」}(the faculty of energy)。按:漢譯的「精進」,對應的巴利語有vāyāma(如八聖道的「正精進」)、viriya(另譯為「勤」)、padhāna(另譯為「精勤;斷;勤奮」如「四正勤;四正斷」)。
\stopitemgroup

\startitemgroup[noteitems]
\item\subnoteref{292.0}\NoteKeywordAgamaHead{「四正斷(SA);四正勤(SA/MA);四意斷(DA/AA)」},南傳作\NoteKeywordNikaya{「在四正勤上」}(Catūsu sammappadhānesu,另譯為「在四個正確的勤奮上」),菩提比丘長老英譯為\NoteKeywordBhikkhuBodhi{「在四個正確的努力上」}(in the four right strivings)。「勤」即「勤奮」(padhāna, 複數padhānānī),長老英譯為「努力」(strivings)。
\stopitemgroup

\startitemgroup[noteitems]
\item\subnoteref{293.0}\NoteKeywordAgamaHead{「患(SA/MA);過/過失(DA);大患(DA/AA)」},南傳作\NoteKeywordNikaya{「過患」}(ādīnavo,另譯為「患」),菩提比丘長老英譯為\NoteKeywordBhikkhuBodhi{「危難」}(danger)。
\stopitemgroup

\startitemgroup[noteitems]
\item\subnoteref{294.0}\NoteSubKeyHead{(1)}\NoteKeywordAgamaHead{「離/出離/出遠離/出要(SA);出要(MA/DA/AA)」},南傳作\NoteSubEntryKey{(i)}\NoteKeywordNikaya{「出離」}(nissaraṇaṃ,另譯為「離;出要」),菩提比丘長老英譯為\NoteKeywordBhikkhuBodhi{「脫離」}(Emancipating),或\NoteSubEntryKey{(ii)}\NoteKeywordNikaya{「出離的」}(niyyānikaṃ,另譯為「出發的」),菩提比丘長老英譯為\NoteKeywordBhikkhuBodhi{「脫離」}(emancipating)。
\item\subnoteref{294.1}\NoteSubKeyHead{(2)}\NoteKeywordAgamaHead{「出要界/出離界(DA)」},南傳作\NoteKeywordNikaya{「出離界」}(nissāraṇīyā dhātuyo, nissaraṇiyā dhātuyo, nissaraṇīyāsu dhātūsu),菩提比丘長老英譯為\NoteKeywordBhikkhuBodhi{「逃脫的元素」}(elements of escape, AN),Maurice Walshe先生英譯為「為了釋放的元素」(elements making for deliverance, DN)。按:《滿足希求》以「已離縛的、已捨棄的」解說「出離」(nissāraṇīyāti nissaṭā visaññuttā, \ccchref{AN.5.200}{https://agama.buddhason.org/AN/an.php?keyword=5.200})。
\stopitemgroup

\startitemgroup[noteitems]
\item\subnoteref{295.0}\NoteKeywordNikayaHead{「味」},南傳作\NoteKeywordNikaya{「樂味」}(assādo,另譯為「味;愛味;快味」,動詞assādeti),菩提比丘長老英譯為\NoteKeywordBhikkhuBodhi{「滿足;喜悅」}(gratification, 名詞),「品嘗;享受」(relishes, 動詞)。這是會令人「回味黏著;還想再要」的滿足。
\stopitemgroup

\startitemgroup[noteitems]
\item\subnoteref{296.0}\NoteKeywordAgamaHead{「向須陀洹(SA/AA);須陀洹趣(MA);須陀洹向(DA);趣須陀洹(AA)」},南傳作\NoteKeywordNikaya{「是為了入流果的作證之行者」}(sotāpattiphalasacchikiriyāya paṭipanno hoti),菩提比丘長老英譯為\NoteKeywordBhikkhuBodhi{「為了實現入-流之結果而實行者」}(one is practising for the realization of the fruit of stream-entry)。
\stopitemgroup

\startitemgroup[noteitems]
\item\subnoteref{297.0}\NoteKeywordAgamaHead{「中般涅槃(SA/MA/DA);中間般涅槃(DA)」},南傳作\NoteKeywordNikaya{「中般涅槃者」}(antarāparinibbāyī),菩提比丘長老英譯為\NoteKeywordBhikkhuBodhi{「在間隔中達成涅槃者」}(an attainer of Nibbāna in the interval)。按:《顯揚真義》等說,他不超越[生到淨居天]壽命一半證涅槃(yo āyuvemajjhaṃ anatikkamitvā parinibbāyati, \suttaref{SN.46.3}/\ccchref{AN.3.88}{https://agama.buddhason.org/AN/an.php?keyword=3.88}),或污染的般涅槃達阿羅漢狀態(kilesaparinibbānaṃ arahattaṃ patto, \ccchref{DN.33}{https://agama.buddhason.org/DN/dm.php?keyword=33}),但長老認為,如果從[antara]字面上的意思,那應該是在兩個生命之間的間隔證涅槃(attains Nibbāna in the interval between two lives),也許是在細身(subtle body)存在的中間狀態。按:「細身存在的中間狀態」意同「中有」(antarābhava)。「中有」的存在否,部派論師間是有異議的,南傳主流一向是不承認「中有」的。
\stopitemgroup

\startitemgroup[noteitems]
\item\subnoteref{298.0}\NoteKeywordAgamaHead{「生般涅槃(SA/MA/DA);生彼般涅槃(DA)」},南傳作\NoteKeywordNikaya{「生般涅槃者」}(upahaccaparinibbāyī),菩提比丘長老英譯為\NoteKeywordBhikkhuBodhi{「在降臨時達成涅槃者」}(an attainer of Nibbāna upon landing)。
\stopitemgroup

\startitemgroup[noteitems]
\item\subnoteref{299.0}\NoteKeywordNikayaHead{「無行般涅槃」},南傳作\NoteKeywordNikaya{「無行般涅槃者」}(asaṅkhāraparinibbāyī),菩提比丘長老英譯為\NoteKeywordBhikkhuBodhi{「無努力而達成涅槃者」}(an attainer of Nibbāna without exertion)。按:《顯揚真義》等以「[被生往某處而]以無行(無為作)、無努力(無加行)到達[阿羅漢狀態]」(Asaṅkhārena appayogena patto, \suttaref{SN.46.3}/\ccchref{AN.3.88}{https://agama.buddhason.org/AN/an.php?keyword=3.88}),《吉祥悅意》以「以無行、無努力、無疲勞地以樂到達」(Asaṅkhārena appayogena akilamanto sukhena patto, \ccchref{DN.33}{https://agama.buddhason.org/DN/dm.php?keyword=33})解說。
\stopitemgroup

\startitemgroup[noteitems]
\item\subnoteref{300.0}\NoteKeywordAgamaHead{「有行般涅槃(SA/DA);行般涅槃(MA)」},南傳作\NoteKeywordNikaya{「有行般涅槃者」}(sasaṅkhāraparinibbāyī),菩提比丘長老英譯為\NoteKeywordBhikkhuBodhi{「以努力而達成涅槃者」}(an attainer of Nibbāna with exertion)。按:《顯揚真義》等以「被生往某處而以有行(有為作)、有努力(有加行)到達阿羅漢狀態」(Yattha katthaci uppanno pana sasaṅkhārena sappayogena arahattaṃ patto, \suttaref{SN.46.3}/\ccchref{AN.3.88}{https://agama.buddhason.org/AN/an.php?keyword=3.88}),《吉祥悅意》以「以有行、有努力、疲勞地以苦到達」(Sasaṅkhārena sappayogena kilamanto dukkhena patto, \ccchref{DN.33}{https://agama.buddhason.org/DN/dm.php?keyword=33})解說。
\stopitemgroup

\startitemgroup[noteitems]
\item\subnoteref{301.0}\NoteKeywordAgamaHead{「上流般涅槃(SA);上流色究竟/上流阿迦膩吒般涅槃(MA);上流阿迦尼吒/阿迦尼吒(DA)」},南傳作\NoteKeywordNikaya{「上流到阿迦膩吒者」}(uddhaṃsoto……akaniṭṭhagāmī),菩提比丘長老英譯為\NoteKeywordBhikkhuBodhi{「駛往上游,走向阿迦膩吒領域者」}(one who is bound upstream, heading towards the Akaniṭṭha realm)。按:「阿迦膩吒」(akaniṭṭha),另譯為「色究竟天」,《一切經音義》說:「言阿迦者色也,尼瑟吒究竟也……又云,阿無也,迦尼瑟吒小也……此之一天唯大無小故以名也。」《吉祥悅意》說,以一切有德(saguṇehi)的有之到達(bhavasampattiyā)與年長,在這裡沒有年輕的(natthettha kaniṭṭhāti, \ccchref{DN.14}{https://agama.buddhason.org/DN/dm.php?keyword=14}),接著解說其他四天:以自己的等至不衰退、不被捨棄為無煩,無任何執著的煩苦為無熱,美之看見、清淨上等色為善見,看見善或看見這是美的為善現。
\stopitemgroup

\startitemgroup[noteitems]
\item\subnoteref{302.0}\NoteKeywordAgamaHead{「家家(SA/MA/AA)」},南傳作\NoteKeywordNikaya{「良家到良家者」}(kolaṃkolo,逐字譯為「家-家」),菩提比丘長老英譯為\NoteKeywordBhikkhuBodhi{「氏族-到-氏族的達成者」}(a clan-to-clan attainer)。按:依\suttaref{SN.48.24},這屬於「斯陀含向」,《顯揚真義》以「二、三生(dve tayo bhave, \suttaref{SN.48.24})輪迴後得到苦的結束者」,《滿足希求》以「[此為]從良家到良家者(kulā kulaṃ gamanako),這裡良家為生(存在)的同義語(Kulanti cettha bhavo adhippeto, \ccchref{AN.3.88}{https://agama.buddhason.org/AN/an.php?keyword=3.88})」,所以「二、三個良家」為二、三生之意。
\stopitemgroup

\startitemgroup[noteitems]
\item\subnoteref{303.0}\NoteKeywordAgamaHead{「一種子道(SA);一種(SA/MA)」},南傳作\NoteKeywordNikaya{「一種子者」}(ekabījī),菩提比丘長老英譯為\NoteKeywordBhikkhuBodhi{「一-種子的達成者」}(a one-seed attainer)。按:依\suttaref{SN.48.24},這屬於「斯陀含向」,《顯揚真義》說,凡成為須陀洹後,就一個個體出生後達到阿羅漢狀態者,這名為一種子者(\suttaref{SN.48.24}),《滿足希求》則以「對這個來說,有一生的種子」(ekasseva bhavassa bījaṃ etassa atthīti, \ccchref{AN.3.88}{https://agama.buddhason.org/AN/an.php?keyword=3.88})解說。
\stopitemgroup

\startitemgroup[noteitems]
\item\subnoteref{304.0}\NoteKeywordAgamaHead{「數力/計數力(SA)」},南傳作\NoteKeywordNikaya{「思擇力」}(paṭisaṅkhānabalaṃ,另譯為「省察力」),菩提比丘長老英譯為\NoteKeywordBhikkhuBodhi{「深思的力量」}(the power of reflection)。按:《滿足希求》以「省察力」(paccavekkhaṇabalaṃ, \ccchref{AN.2.11}{https://agama.buddhason.org/AN/an.php?keyword=2.11})解說。
\stopitemgroup

\startitemgroup[noteitems]
\item\subnoteref{305.0}\NoteKeywordAgamaHead{「無想天(MA/DA/AA)」},南傳作\NoteSubEntryKey{(i)}\NoteKeywordNikaya{「無想眾生天;無想眾生天神們」}(devā asaññasattā, asaññasattā devā),菩提比丘長老英譯為\NoteKeywordBhikkhuBodhi{「無知覺天」}(the devas that are impercipient, AN)。\NoteSubEntryKey{(ii)}\NoteKeywordNikaya{「無想眾生處」}(Asaññasattāyatanaṃ),Maurice Walshe先生英譯為「無意識生命的領域」(The Realm of Unconscious Beings),坦尼沙羅比丘長老英譯為「無覺知力生命的維度」(The dimension of non-percipient beings)。
\stopitemgroup

\startitemgroup[noteitems]
\item\subnoteref{306.0}\NoteKeywordAgamaHead{「精進力(SA)」},南傳作\NoteKeywordNikaya{「活力之力」}(vīriyabalaṃ),菩提比丘長老英譯為\NoteKeywordBhikkhuBodhi{「活力的力量」}(the power of energy)。按:漢譯的「精進」,對應的巴利語有vāyāma(如八聖道的「正精進」)、viriya(另譯為「勤」)、padhāna(另譯為「精勤;斷;勤奮」如「四正勤;四正斷」)。
\stopitemgroup

\startitemgroup[noteitems]
\item\subnoteref{307.0}\NoteKeywordNikayaHead{「攝力」}(saṅgahabalaṃ),菩提比丘長老英譯為\NoteKeywordBhikkhuBodhi{「吸引隨行人員的力量」}(the power of attracting a retinue)。
\stopitemgroup

\startitemgroup[noteitems]
\item\subnoteref{308.0}\NoteKeywordAgamaHead{「未來世;未來時」(anāgatamaddhānaṃ),菩提比丘長老英譯為「在未來」(in the future)。按:巴利語的「世」(addhāna),水野弘元《巴利語辭典》譯為「時間,世、行路,旅行」,Buddhadatta譯為「長路途;時間;旅程;大道」(a long path, time, or journey; highroad)。這可以指來生,如\suttaref{SN.22.82}的「願我未來世有這樣的色!」也可以指當生中的未來,如\ccchref{SA.75}{https://agama.buddhason.org/SA/dm.php?keyword=75}說如來「於未來世開覺聲聞」}。
\stopitemgroup

\startitemgroup[noteitems]
\item\subnoteref{309.0}\NoteKeywordNikayaHead{「梵輪」}(brahmacakkaṃ),菩提比丘長老英譯為\NoteKeywordBhikkhuBodhi{「梵之輪」}(the brahma wheel/the Wheel of Brahmā, AN/MN)。按:《顯揚真義》以「這是清淨法輪的同義語」(visuddhassa dhammacakkassetaṃ adhivacanaṃ, \suttaref{SN.12.22}),《滿足希求》以「最上輪」(seṭṭhacakkaṃ, \ccchref{AN.5.11}{https://agama.buddhason.org/AN/an.php?keyword=5.11})解說,「梵」則都以「最上的、最高的、殊勝的清淨的」(seṭṭhaṃ uttamaṃ visiṭṭhaṃ visuddhaṃ , \suttaref{SN.12.22}/\ccchref{MN.12}{https://agama.buddhason.org/MN/dm.php?keyword=12}/\ccchref{AN.4.8}{https://agama.buddhason.org/AN/an.php?keyword=4.8})解說,而《破斥猶豫》說,法輪有兩種:洞察智與教說智(paṭivedhañāṇañceva desanāñāṇañca, \ccchref{MN.12}{https://agama.buddhason.org/MN/dm.php?keyword=12})。
\stopitemgroup

\startitemgroup[noteitems]
\item\subnoteref{310.0}\NoteKeywordAgamaHead{「精進覺支/精進覺分(SA/MA);精進覺意(DA/AA)」},南傳作\NoteKeywordNikaya{「活力覺支」}(vīriyasambojjhaṅgo,另譯為「活力等覺分」),菩提比丘長老英譯為\NoteKeywordBhikkhuBodhi{「活力的開化要素」}(the enlightenment factor of energy)。
\stopitemgroup

\startitemgroup[noteitems]
\item\subnoteref{311.0}\NoteKeywordAgamaHead{「擇法覺分(SA);擇法覺支(MA);法覺意(DA/AA)」},南傳作\NoteKeywordNikaya{「擇法覺支」}(dhammavicayasambojjhaṅgo,另譯為「擇法等覺分」),菩提比丘長老英譯為\NoteKeywordBhikkhuBodhi{「狀態之識別的開化要素」}(the enlightenment factor of discrimination of states)。
\stopitemgroup

\startitemgroup[noteitems]
\item\subnoteref{312.0}\NoteKeywordAgamaHead{「喜覺分(SA);喜覺支(MA);喜覺意(DA/AA)」},南傳作\NoteKeywordNikaya{「喜覺支」}(pītisambojjhaṅgo,另譯為「喜等覺分」),菩提比丘長老英譯為\NoteKeywordBhikkhuBodhi{「狂喜的開化要素」}(the enlightenment factor of rapture)。按:「喜」與五禪支的「喜」同字。
\stopitemgroup

\startitemgroup[noteitems]
\item\subnoteref{313.0}\NoteSubKeyHead{(1)}\NoteKeywordAgamaHead{「猗/猗息/休息/止息(SA);息/止(MA);猗(DA/AA)」},南傳作\NoteKeywordNikaya{「寧靜」}(passaddhi,另譯為「輕安」),菩提比丘長老英譯為\NoteKeywordBhikkhuBodhi{「寧靜」}(tranquillity)。按:《顯揚真義》等以「身心寧靜、污染寧靜」(kāyacittapassaddhi, kilesapassaddhi, \suttaref{SN.35.87}/\ccchref{MN.144}{https://agama.buddhason.org/MN/dm.php?keyword=144})解說。
\item\subnoteref{313.1}\NoteSubKeyHead{(2)}\NoteKeywordAgamaHead{「猗覺分/猗覺支/除覺分(SA);息覺支(MA);猗覺意(DA/AA)」},南傳作\NoteKeywordNikaya{「寧靜覺支」}(passaddhisambojjhaṅgo,另譯為「輕安等覺分」),菩提比丘長老英譯為\NoteKeywordBhikkhuBodhi{「寧靜的開化要素」}(the enlightenment factor of tranquillity)。
\stopitemgroup

\startitemgroup[noteitems]
\item\subnoteref{314.0}\NoteKeywordAgamaHead{「捨覺分(SA);捨覺支(SA/MA);護覺意(DA/AA)」},南傳作\NoteKeywordNikaya{「平靜覺支」}(upekkhāsambojjhaṅgo,另譯為「捨等覺分」),菩提比丘長老英譯為\NoteKeywordBhikkhuBodhi{「平靜的開化要素」}(the enlightenment factor of equanimity)。
\stopitemgroup

\startitemgroup[noteitems]
\item\subnoteref{315.0}\NoteSubKeyHead{(1)}\NoteKeywordAgamaHead{「念覺分(SA);念覺支(MA);念覺意(DA/AA)」},南傳作\NoteKeywordNikaya{「念覺支」}(satisambojjhaṅgo),菩提比丘長老英譯為\NoteKeywordBhikkhuBodhi{「深切注意的開化要素」}(the enlightenment factor of mindfulness)。
\item\subnoteref{315.1}\NoteSubKeyHead{(2)}\NoteKeywordAgamaHead{「念覺分處法(SA)」},南傳作\NoteKeywordNikaya{「念覺支處諸法」}(satisambojjhaṅgaṭṭhānīyā dhammā),菩提比丘長老英譯為\NoteKeywordBhikkhuBodhi{「深切注意的開化要素基礎的事」}(things that are the basis for the enlightenment factor of mindfulness)。按:《顯揚真義》說,這是念的所緣法:三十七菩提分與九出世間法(\suttaref{SN.46.2})。 
\stopitemgroup

\startitemgroup[noteitems]
\item\subnoteref{316.0}\NoteKeywordNikayaHead{「作目標後」}(aṭṭhiṃ katvā,另譯為「很關注後」),菩提比丘長老英譯為\NoteKeywordBhikkhuBodhi{「留心它」}(heeds it, AN),或「當成緊要大事」(as a matter of vital concern, SN),智髻比丘長老英譯為「留心它」(heeds it, MN)。Maurice Walshe先生英譯為「經思慮後」(having thought over, \ccchref{DN.18}{https://agama.buddhason.org/DN/dm.php?keyword=18})。按:《顯揚真義》以「作目標的後:『這個目標應該被我們到達。』」(atthikaṃ katvā, ‘‘ayaṃ no adhigantabbo attho’’ti)解說。《破斥猶豫》以「作為目標後(作有利益性後);成為目標後」(atthikabhāvaṃ katvā, atthiko hutvāti)解說。《滿足希求》以「成為目標後」(atthiko hutvāti)解說。
\stopitemgroup

\startitemgroup[noteitems]
\item\subnoteref{317.0}\NoteSubKeyHead{(1)}\NoteKeywordAgamaHead{「安樂住/樂住(SA);住於安樂/安樂而住(GA)」},南傳作\NoteKeywordNikaya{「住於樂」}(sukhaṃ viharati,逐字譯為「樂-住」),菩提比丘長老英譯為\NoteKeywordBhikkhuBodhi{「幸福地住著」}(dwells happily, \suttaref{SN.22.43}/\ccchref{AN.2.184}{https://agama.buddhason.org/AN/an.php?keyword=2.184}),或「住於快樂」(dwells in happiness, \ccchref{AN.1.318}{https://agama.buddhason.org/AN/an.php?keyword=1.318})。
\item\subnoteref{317.1}(2)第三禪的「安樂住/身心受樂/樂住(SA);樂住(MA);樂/樂行(DA);樂/歡樂(AA)」,南傳作\NoteKeywordNikaya{「安樂住者」}(sukhavihārī),菩提比丘長老英譯為\NoteKeywordBhikkhuBodhi{「幸福地住著者」}(one who dwells happily, SN/AN),智髻比丘長老英譯為「他有個快樂的滯留」(He has a pleasant abiding, MN),Maurice Walshe先生英譯為「幸福住著」(Happy dwells, DN)。按:《阿毘曇毘婆沙論》說:「第三禪有五枝:捨、念、慧、樂、一心。」
\stopitemgroup

\startitemgroup[noteitems]
\item\subnoteref{318.0}\NoteKeywordAgamaHead{「身猗息/身止息/身得止息(SA);身獲柔軟(GA);止身/止息身(MA);身心安隱(DA)」},南傳作\NoteKeywordNikaya{「身體的寧靜;身已寧靜」}(kāyapassaddhi, passaddhakāyo,另譯為「身體的輕安」),菩提比丘長老英譯為\NoteKeywordBhikkhuBodhi{「身體的寧靜」}(tranquillity of body)。按:《顯揚真義》說,這是[受想行]三蘊煩苦的寧靜(darathapassaddhi, \suttaref{SN.46.2}),長老說,依據阿毘達磨,身與心成對出現時,「身」都被解說為「心所」(cetasika),也就是執行第二階段認知的功能,但這裡從字面上來看,似乎只是指「身體」(physical body),「視為對一個經驗性質的活躍建樹」,並在2010年8月3日的回函中進一步解說:「簡單來說,這意味著部分身體狀況決定體驗是否為心的寧靜或心的擾動。請記得這是我的解釋。這裡的身,巴利註釋採用為名身:『心理因素的聚集』(除了心之外)」。
\stopitemgroup

\startitemgroup[noteitems]
\item\subnoteref{319.0}\NoteKeywordAgamaHead{「心猗息(SA)」},南傳作\NoteKeywordNikaya{「心的寧靜」}(cittapassaddhi),菩提比丘長老英譯為\NoteKeywordBhikkhuBodhi{「心的寧靜」}( tranquillity of mind)。按:《顯揚真義》說,這是指識蘊(viññāṇakkhandhassa)煩苦的寧靜(darathapassaddhi, \suttaref{SN.46.2})。
\stopitemgroup

\startitemgroup[noteitems]
\item\subnoteref{320.0}\NoteKeywordNikayaHead{「意喜;喜心」}(pīti-mana, pītimano),菩提比丘長老英譯為\NoteKeywordBhikkhuBodhi{「心被狂喜所提升」}(the mind is uplifted by rapture/the mind is elated by rapture, \suttaref{SN.35.97}/\suttaref{SN.42.13}),或「以喜悅的心」(With joyful mind, \suttaref{SN.7.18})。按:《顯揚真義》以「有喜悅心的」(tuṭṭhacitto, \suttaref{SN.7.18}),《破斥猶豫》以「使意被這個喜取悅」(tāya pītiyā pīṇitamanassa, \ccchref{MN.7}{https://agama.buddhason.org/MN/dm.php?keyword=7}),《吉祥悅意》以「喜相應的心」(pītisampayuttacittassa, \ccchref{DN.2}{https://agama.buddhason.org/DN/dm.php?keyword=2}),《勝義光明》以「有喜悅的心」(tuṭṭhacitto hoti, Sn.4.1)解說。
\stopitemgroup

\startitemgroup[noteitems]
\item\subnoteref{321.0}\NoteKeywordAgamaHead{「善知識;善勝丈夫(GA)」},南傳作\NoteKeywordNikaya{「善的朋友之誼」}(kalyāṇamittatā),菩提比丘長老英譯為\NoteKeywordBhikkhuBodhi{「好友誼」}(good friendship),並說「善的朋友之誼、善的同伴之誼、善的親密朋友之誼」三者為同義詞。按:「善的朋友」(kalyāṇamitta),古譯為「善知識」,字尾加「tā」表示「友誼;在……的狀態」。
\stopitemgroup

\startitemgroup[noteitems]
\item\subnoteref{322.0}\NoteKeywordAgamaHead{「依遠離(SA);依離(SA/MA/DA)」},南傳作\NoteKeywordNikaya{「依止遠離」}(vivekanissitaṃ),菩提比丘長老英譯為\NoteKeywordBhikkhuBodhi{「基於隔離」}(is based upon seclusion)。按:《顯揚真義》以vivittatā解說viveka,並舉了五種遠離,即《無礙解道》〈24.\ccchref{遠離的談論}{https://agama.buddhason.org/note/Ps1.htm}〉說的:i.由練習毘婆舍那而達暫時的「彼分遠離」(tadaṅgaviveko)。ii.由入禪定而達暫時的「鎮伏遠離」(vikkhambhana viveka)。iii.由出世間道而達永久的「斷絕遠離」(samuccheda viveka)。iv.由證果位而達永久的「安息遠離;平息遠離」(paṭippassaddh viveka)。v.由證涅槃而達永久的「出離遠離」(nissaraṇavivekoti, \suttaref{SN.45.2})。
\stopitemgroup

\startitemgroup[noteitems]
\item\subnoteref{323.0}\NoteKeywordAgamaHead{「有餘(SA/MA)」},南傳作\NoteKeywordNikaya{「有餘依;有殘餘」}(upādisese, sa-upādisesaṃ,另譯為「有餘的」),菩提比丘長老英譯為\NoteKeywordBhikkhuBodhi{「有執著的殘渣」}(there is a residue of clinging, SN),並解說這裡所譯的「執著」(clinging),只為了表示上的清晰,而不是要以「取;執取」(upādāna)來取代「生命的燃料」(upādi)的意思,而此原慣用語的意思,只是單純「(未被指定的)殘渣」(an (unspecified) residue)。按:《顯揚真義》以「持續存在的持有殘餘、執取(燃料)殘餘」(gahaṇasese upādānasese vijjamānamhi, \suttaref{SN.46.57}),《破斥猶豫》等以「執取(燃料)殘餘、未被遍滅盡的(aparikkhīṇe, \ccchref{MN.10}{https://agama.buddhason.org/MN/dm.php?keyword=10}/\ccchref{DN.22}{https://agama.buddhason.org/DN/dm.php?keyword=22})」解說。
\stopitemgroup

\startitemgroup[noteitems]
\item\subnoteref{324.0}\NoteKeywordAgamaHead{「四無量心(DA);四等心(AA)」}:「慈」(Mettā),菩提比丘長老英譯為\NoteKeywordBhikkhuBodhi{「慈愛」}(lovingkindness)。「悲」(Karuṇā),菩提比丘長老英譯為\NoteKeywordBhikkhuBodhi{「憐憫;同情」}(compassion)。「喜悅」(Muditā,另譯為「喜;滿足」),菩提比丘長老英譯為\NoteKeywordBhikkhuBodhi{「利他的喜悅;愛他的喜悅」}(altruistic joy)。按:為了與「初禪」中的「喜」(pīti)區隔而將慣用的「喜」譯為「喜悅」。「捨;平靜」(upekkhā),菩提比丘長老英譯為\NoteKeywordBhikkhuBodhi{「平靜」}(equanimity)。
\stopitemgroup

\startitemgroup[noteitems]
\item\subnoteref{325.0}\NoteKeywordNikayaHead{「邪智」}(micchāñāṇaṃ),菩提比丘長老英譯為\NoteKeywordBhikkhuBodhi{「錯誤的理解」}(wrong knowledge)。按:《顯揚真義》以「邪識、邪觀察」(micchāviññāṇo micchāpaccavekkhana, \suttaref{SN.45.26}),註疏以某人作了惡的後心想:「啊!我作了善的。」解說。
\stopitemgroup

\startitemgroup[noteitems]
\item\subnoteref{326.0}\NoteKeywordNikayaHead{「邪解脫」}(micchāvimutti),菩提比丘長老英譯為\NoteKeywordBhikkhuBodhi{「錯誤的釋放」}(wrong liberation)。按:《顯揚真義》以「非真實的解脫(ayāthāvavimutti, \suttaref{SN.45.26})、不出離的解脫(aniyyānikavimutti, \suttaref{SN.56.26})」,《破斥猶豫》以「當存在未解脫時,『我們是解脫者(vimuttā maya’’nti)』這麼想者,或於未解脫處有解脫想者(avimuttiyaṃ vā vimuttisaññino, \ccchref{MN.8}{https://agama.buddhason.org/MN/dm.php?keyword=8})」解說。
\stopitemgroup

\startitemgroup[noteitems]
\item\subnoteref{327.0}\NoteKeywordNikayaHead{「沙門義」}(sāmaññatthaṃ,另譯為「沙門性義;沙門性的利益;沙門性的目標」),菩提比丘長老英譯為\NoteKeywordBhikkhuBodhi{「禁欲主義者的目標」}(the goal of asceticism, SN/AN),智髻比丘長老英譯為「禁欲修道身分者的目標」(the goal of recluseship, MN)。按:義(attha),有「利益;目標」的意思,也有「道理,意義」的意思。《顯揚真義》以「聖果(ariyaphalaṃ, \suttaref{SN.12.13})、涅槃(nibbānaṃ, \suttaref{SN.45.36})」解說「沙門義」,就是經文說的貪、瞋、癡的滅盡(\suttaref{SN.45.36})。
\stopitemgroup

\startitemgroup[noteitems]
\item\subnoteref{328.0}\NoteSubKeyHead{(1)}\NoteKeywordAgamaHead{「沙門法(SA/MA);得沙門名(DA);沙門之法(AA)」},南傳作\NoteKeywordNikaya{「沙門性;沙門身分」}(sāmaññaṃ),菩提比丘長老英譯為\NoteKeywordBhikkhuBodhi{「禁欲修道主義」}(asceticism, SN),或「禁欲修道的高度」(ascetic stature, \suttaref{SN.2.30}),或「禁欲修道者的生活」(The ascetic life, SN),智髻比丘長老英譯為「禁欲修道身分;禁欲修道狀態」(the recluseship, MN)。按:\ccchref{SA.796}{https://agama.buddhason.org/SA/dm.php?keyword=796}等說是「八正道」,《顯揚真義》以「沙門法」(samaṇadhammo, \suttaref{SN.1.17})解說。
\item\subnoteref{328.1}\NoteSubKeyHead{(2)}\NoteKeywordAgamaHead{「供養沙門(SA);奉事沙門(GA);尊敬沙門(MA/DA);慈孝沙門(AA)」},南傳作\NoteKeywordNikaya{「尊敬沙門」}(sāmañño, sāmaññatā),菩提比丘長老英譯為\NoteKeywordBhikkhuBodhi{「對禁欲修道者適當地對待」}(behave properly toward ascetics, AN)。
\stopitemgroup

\startitemgroup[noteitems]
\item\subnoteref{329.0}\NoteKeywordAgamaHead{「安那般那念(SA);修息出息入(MA);念安般/安般念/安般(AA);阿那般那念(摩訶僧祇律)」},南傳作\NoteKeywordNikaya{「入出息念」}(ānāpānassati,另譯為「出入息念」),菩提比丘長老英譯為\NoteKeywordBhikkhuBodhi{「呼吸的深切注意」}(Mindfulness of breathing)。按:「入出息」(ānāpāna),音譯為「安那般那」,簡為「安般」,義譯為「呼吸」。
\stopitemgroup

\startitemgroup[noteitems]
\item\subnoteref{330.0}\NoteKeywordAgamaHead{「身心止息(SA)」},南傳作\NoteKeywordNikaya{「身已寧靜成為無激情的」}(passaddho kāyo asāraddho,另譯為「身已輕安冷靜」),菩提比丘長老英譯為\NoteKeywordBhikkhuBodhi{「身體成為安靜的並且無苦惱」}(the body becomes tranquil and untroubled, SN),或「身體將安靜無騷動」(body will be tranquil without disturbance, AN)。按:《破斥猶豫》說,這裡的身指名身、色身(nāmakāyo rūpakāyoti),無激情指「離煩苦」(vigatadarathoti, \ccchref{MN.4}{https://agama.buddhason.org/MN/dm.php?keyword=4}),《滿足希求》則以「名身與所生身的寧靜,煩苦被平靜下來」(nāmakāyo ca karajakāyo ca passaddho vūpasantadaratho, \ccchref{AN.3.40}{https://agama.buddhason.org/AN/an.php?keyword=3.40})解說。
\stopitemgroup

\startitemgroup[noteitems]
\item\subnoteref{331.0}\NoteKeywordAgamaHead{「對礙想(SA);有對想(MA/DA);瞋恚想(DA)」},南傳作\NoteKeywordNikaya{「有對想」}(paṭighasaññā),菩提比丘長老英譯為\NoteKeywordBhikkhuBodhi{「嫌惡的認知」}(perceptions of aversion, \suttaref{SN.54.6}),或「知覺器官衝擊的認知」(perceptions of sensory impingement, SN/AN)。按:「有對」(paṭigha),有兩個意思:i.「障礙;對礙」,指的是「五境」對「五根」的撞擊,ii.「嫌惡;厭惡;反感;排斥;怒」。
\stopitemgroup

\startitemgroup[noteitems]
\item\subnoteref{332.0}\NoteKeywordAgamaHead{「增上意學(SA)」},南傳作\NoteKeywordNikaya{「增上心學」}(adhicittasikkhā),菩提比丘長老英譯為\NoteKeywordBhikkhuBodhi{「在較高之心上的訓練」}(the training in the higher mind)。按:依\ccchref{SA.832}{https://agama.buddhason.org/SA/dm.php?keyword=832}、\ccchref{AN.3.91}{https://agama.buddhason.org/AN/an.php?keyword=3.91},「增上意學;增上心學」顯然是指「禪定」(jhāna)。「(也)學習增上心」(adhicittampi sikkhati),菩提比丘長老英譯為\NoteKeywordBhikkhuBodhi{「在較高之心上訓練」}(He trains in the higher mind)。
\stopitemgroup

\startitemgroup[noteitems]
\item\subnoteref{333.0}\NoteKeywordAgamaHead{「滿足者(SA);具(MA);完具者(AA)」},南傳作\NoteKeywordNikaya{「完全的實行者」}(paripūrakārī),菩提比丘長老英譯為\NoteKeywordBhikkhuBodhi{「完全地激活它們者」}(one who activates them fully, SN),或「完成……的行為者」(fulfill……behavior, AN)。《滿足希求》以「已完成者(已完成的作者)」(samattakārī, \ccchref{AN.3.87}{https://agama.buddhason.org/AN/an.php?keyword=3.87})解說。
\stopitemgroup

\startitemgroup[noteitems]
\item\subnoteref{334.0}\NoteSubKeyHead{(1)}\NoteKeywordAgamaHead{「空無果(SA)」},南傳作\NoteKeywordNikaya{「不孕的;徒然白費的」}(vañjha),菩提比丘長老英譯為\NoteKeywordBhikkhuBodhi{「不生育的;不妊的」}(barren)。
\item\subnoteref{334.1}\NoteSubKeyHead{(2)}\NoteKeywordAgamaHead{「不空無果(SA);不空/不虛(MA);不唐其勞(AA)」},南傳作\NoteKeywordNikaya{「功不唐捐的」}(avañjhā,另譯為「非不孕女的;非石女的」),菩提比丘長老英譯為\NoteKeywordBhikkhuBodhi{「不是不生育的」}(not be barren)。
\stopitemgroup

\startitemgroup[noteitems]
\item\subnoteref{335.0}\NoteKeywordAgamaHead{「過二百五十戒(SA);二百五十戒(DA/AA)」},南傳作\NoteKeywordNikaya{「這超過一百五十條學處」}(Sādhikamidaṃ……diyaḍḍhasikkhāpadasataṃ),菩提比丘長老英譯為\NoteKeywordBhikkhuBodhi{「超過一百五十調馴規定」}(the more than a hundred and fifty training rules)。按:「學處」(sikkhāpada),即「戒條」。關於「二百五十」與「一百五十」差異之考證,印順法師斷定後者為正確,參看《原始佛教聖典之集成》p.172。
\stopitemgroup

\startitemgroup[noteitems]
\item\subnoteref{336.0}\NoteKeywordAgamaHead{「細微戒(SA);小小戒(DA)」},南傳作\NoteKeywordNikaya{「諸小隨小學處」}(khuddānukhuddakāni sikkhāpadāni,另譯為「微細學處;細隨細學處;小小學處」),菩提比丘長老英譯為\NoteKeywordBhikkhuBodhi{「較小的與不重要的調訓規則」}(the lesser and minor trainings)。按:《滿足希求》說,這是除四波羅夷(cattāri pārājikāni)外的其餘學處(AN),但長老認為這樣的說法似乎太寬鬆,而傾向指那些向一名比丘懺悔出罪的波逸提(pācittiya)。編按:當然也包括更輕的婆羅提提舍尼(paṭidesemi)與「眾學法」(sikkhākaraṇīyā),而四波羅夷是指犯「殺人、偷盜(一定價值物品達犯法律程度)、淫(性交)、大妄語(未證稱證)」四種受驅擯處理失去出家人資格的罪。學處(sikkhāpada),即戒條。
\stopitemgroup

\startitemgroup[noteitems]
\item\subnoteref{337.0}\NoteSubKeyHead{(1)}\NoteKeywordNikayaHead{「意界」}(manodhātu),菩提比丘長老英譯為\NoteKeywordBhikkhuBodhi{「心的元素」}(the mind element)。按:南傳論師們的看法,意界指善與不善果報兩種領受心(sampaṭicchanacitta)與五門轉向心(pañcadvārāvajjana-citta)。
\item\subnoteref{337.1}\NoteSubKeyHead{(2)}\NoteKeywordNikayaHead{「意識界」}(manoviññāṇadhātu),菩提比丘長老英譯為\NoteKeywordBhikkhuBodhi{「心-識的元素」}(mind-consciousness element)。按:南傳論師們的看法,意識界指五識與意界外的所有識(心)。
\stopitemgroup

\startitemgroup[noteitems]
\item\subnoteref{338.0}\NoteKeywordAgamaHead{「持後邊/持此後邊身(SA);住於最後身/受最後身/後邊之身(GA)」},南傳作\NoteKeywordNikaya{「持最後身者」}(antimadehadhārī, sarīrantimadhārina),菩提比丘長老英譯為\NoteKeywordBhikkhuBodhi{「他帶著他的最後身體」}(One who bears his final body)。
\stopitemgroup

\startitemgroup[noteitems]
\item\subnoteref{339.0}\NoteKeywordAgamaHead{「三法衣/三衣(SA);三法服(MA);三法衣(DA/AA)」},南傳作\NoteKeywordNikaya{「三衣」}(ticīvara, tecīvara),出家眾遵守只保有三件衣服的規定,即「僧伽梨(saṅghāṭī,另譯作「大衣;重衣」)、欝多羅僧(uttarāsaṅga,另譯作「上衣」)、安陀衣(antaravāsaka,另譯作「內衣」)。
\stopitemgroup

\startitemgroup[noteitems]
\item\subnoteref{340.0}\NoteSubKeyHead{(1)}\NoteKeywordNikayaHead{「淨信的;淨信者;已淨信」}(pasannaṃ,另譯為「明淨的;澄淨的;已喜的」),菩提比丘長老英譯為\NoteKeywordBhikkhuBodhi{「信任」}(confidence)。
\item\subnoteref{340.1}\NoteSubKeyHead{(2)}\NoteKeywordNikayaHead{「極淨信的;極淨信者」}(abhippasanno,另譯為「極明淨的」),菩提比丘長老英譯為\NoteKeywordBhikkhuBodhi{「充分信任」}(full confidence)。
\item\subnoteref{340.2}\NoteSubKeyHead{(3)}\NoteKeywordNikayaHead{「淨信的行為」}(pasannākāraṃ),菩提比丘長老英譯為\NoteKeywordBhikkhuBodhi{「信任」}(confidence)。按:《顯揚真義》說,這是施與衣食需要物(cīvarādayo paccaye, \suttaref{SN.16.3}),或施與四種需要物(\suttaref{SN.20.9})。
\stopitemgroup

\startitemgroup[noteitems]
\item\subnoteref{341.0}\NoteKeywordNikayaHead{「念已現起;念已現起的;念被現起」}(upaṭṭhitā sati, upaṭṭhitassati),菩提比丘長老英譯為\NoteKeywordBhikkhuBodhi{「深切注意被樹立」}(mindfulness is set up)。
\stopitemgroup

\startitemgroup[noteitems]
\item\subnoteref{342.0}\NoteKeywordAgamaHead{「惱(SA/AA/DA);懊惱(MA)」},南傳作\NoteKeywordNikaya{「絕望」}(upāyāsā,另譯為「惱,愁,悶,磨難,苦惱」),菩提比丘長老英譯為\NoteKeywordBhikkhuBodhi{「絕望」}(despair),Maurice Walshe先生英譯為「苦惱;苦難」(distress, DN)。按:《吉祥悅意》說,更強力的悲傷(憂愁)為「絕望」(Balavataraṃ āyāso upāyāso, \ccchref{DN.22}{https://agama.buddhason.org/DN/dm.php?keyword=22})。
\stopitemgroup

\startitemgroup[noteitems]
\item\subnoteref{343.0}\NoteKeywordAgamaHead{「六思身(SA/MA/DA)」},南傳作\NoteKeywordNikaya{「六類思」}(chayime cetanākāyā, Cha sañcetanākāyā,逐字譯為「六-思+身」),菩提比丘長老英譯為\NoteKeywordBhikkhuBodhi{「六類意志力」}(six classes of volition)。按:這裡的「身」(kāya)指「種類」而非「身體」,六受身、六想身、六識身的身亦同。思與行(saṅkhārā)的差別,後者常用複數,長老英譯為「意志的形成」(volition formations),論師們慣於指「受、想、識」以外的心理狀態,分「身、語、意」三行,而對色聲等六外處的對應分類,就用「思」。
\stopitemgroup

\startitemgroup[noteitems]
\item\subnoteref{344.0}\NoteSubKeyHead{(1)}\NoteKeywordNikayaHead{「自恣」}:自在地,如「五欲自恣;飲食自恣;自恣而遊行」等。
\item\subnoteref{344.1}\NoteSubKeyHead{(2)}\NoteKeywordAgamaHead{「懷受/受歲(SA/AA);自恣(SA/GA);請/請請(MA);受自恣(摩訶僧祇律)」},南傳作\NoteKeywordNikaya{「作自恣」}(Pavāreti,動詞,另譯為「邀請」),菩提比丘長老英譯為\NoteKeywordBhikkhuBodhi{「請求」}(asks, MN),或「邀請」(invitation, AN)。
\item\subnoteref{344.2}\NoteSubKeyHead{(3)}\NoteKeywordAgamaHead{「月食受時(SA);自恣時到(GA);相請請時(MA);受歲之日(AA)」},南傳作\NoteKeywordNikaya{「在自恣日」}(pavāraṇāya),菩提比丘長老英譯為\NoteKeywordBhikkhuBodhi{「在...自恣儀式」}(on...of the pavāraṇā ceremony),或「邀請儀式」(the invitation ceremony, AN),並解說這是雨季安居結束的儀式,比丘邀請所有其他比丘告誡其犯戒行為。按:僧團規制,結夏安居的最後一天僧團集會,就所見、所聽聞、所懷疑,「自在地」指出別人的過失,以互相砥礪改正,稱為「自恣」,這一天也稱為「自恣日」。
\stopitemgroup

\startitemgroup[noteitems]
\item\subnoteref{345.0}\NoteKeywordAgamaHead{「地獄;泥黎(SA/AA)」},南傳作\NoteKeywordNikaya{「地獄」}(niraya,音譯為「泥黎;泥梨;泥犁;泥洹夜;捺落迦」),菩提比丘長老英譯為\NoteKeywordBhikkhuBodhi{「地獄」}(hells)。
\stopitemgroup

\startitemgroup[noteitems]
\item\subnoteref{346.0}\NoteKeywordNikayaHead{「化生者;化生(的)」}(opapātikā),菩提比丘長老英譯為\NoteKeywordBhikkhuBodhi{「自生的出生」}(spontaneous birth)。
\stopitemgroup

\startitemgroup[noteitems]
\item\subnoteref{347.0}\NoteKeywordAgamaHead{「四雙八輩;四雙八士(SA);四雙人八輩聖士(MA)」},南傳作\NoteKeywordNikaya{「四雙之人、八輩之士」}(cattāri purisayugāni aṭṭha purisapuggalā),菩提比丘長老英譯為\NoteKeywordBhikkhuBodhi{「四對人、八類個人」}(the four pairs of persons, the eight types of individuals)。按:此即四果與四果向者,合為八種聖者之人。
\stopitemgroup

\startitemgroup[noteitems]
\item\subnoteref{348.0}\NoteKeywordAgamaHead{「解脫施(SA)」},南傳作\NoteKeywordNikaya{「自由施捨者」}(muttacāga),菩提比丘長老英譯為\NoteKeywordBhikkhuBodhi{「爽快慷慨的」}(freely generous)。按:「自由」(mutta),另譯為「釋放;解脫」,所以「解脫施」與「自由施捨」,推斷它們的原文是完全相同的。
\stopitemgroup

\startitemgroup[noteitems]
\item\subnoteref{349.0}\NoteKeywordAgamaHead{「等心行施(SA);等心普施(MA);平等心而以惠施/等心普施/等心施與(AA)」},南傳作\NoteKeywordNikaya{「樂於布施物均分者」}(dānasaṃvibhāgarata),菩提比丘長老英譯為\NoteKeywordBhikkhuBodhi{「樂於給予及共享」}(delighting in giving and sharing)。
\stopitemgroup

\startitemgroup[noteitems]
\item\subnoteref{350.0}\NoteKeywordAgamaHead{「快得善利;汝得善利/汝得大利(SA)」},南傳作\NoteKeywordNikaya{「是你的善得的」}(suladdhaṃ te),菩提比丘長老英譯為\NoteKeywordBhikkhuBodhi{「那是被你好收穫的」}(it is well gained by you)。
\stopitemgroup

\startitemgroup[noteitems]
\item\subnoteref{351.0}\NoteKeywordAgamaHead{「俱解脫(SA/MA/AA)」},南傳作\NoteKeywordNikaya{「俱分解脫者」}(ubhatobhāgavimuttā),菩提比丘長老英譯為\NoteKeywordBhikkhuBodhi{「在兩方面自由了」}(liberated in both ways)。按:\ccchref{AN.9.45}{https://agama.buddhason.org/AN/an.php?keyword=9.45}說,身觸達初禪的解脫者被世尊以方便說為俱分解脫者,唯以身觸達想受滅的解脫者被世尊以非方便說為俱分解脫者。\ccchref{MN.70}{https://agama.buddhason.org/MN/dm.php?keyword=70}說,以身觸無色寂靜解脫的解者為俱分解脫者,《破斥猶豫》以「八解脫」(aṭṭha vimokkhe)解說「無色寂靜解脫」(同\ccchref{MA.195}{https://agama.buddhason.org/MA/dm.php?keyword=195})。《顯揚真義》等說,以兩部分解脫,以無色等至從色身解脫(arūpāvacarasamāpattiyā rūpakāyato vimuttā),以無上道從名身[解脫]」(aggamaggena nāmakāyatoti, \suttaref{SN.8.7}/\ccchref{MN.70}{https://agama.buddhason.org/MN/dm.php?keyword=70}/\ccchref{DN.15}{https://agama.buddhason.org/DN/dm.php?keyword=15}/\ccchref{AN.7.14}{https://agama.buddhason.org/AN/an.php?keyword=7.14})。
\stopitemgroup

\startitemgroup[noteitems]
\item\subnoteref{352.0}\NoteSubKeyHead{(1)}\NoteKeywordAgamaHead{「糞掃衣(SA/MA);糞掃之衣(MA);塚間衣(DA);補納衣/納故衣/納衣(AA)」},南傳作\NoteKeywordNikaya{「糞掃衣」}(paṃsukūlaṃ,另譯為「塵土衣」),菩提比丘長老英譯為\NoteKeywordBhikkhuBodhi{「碎布長袍」}(rag-robe)。按:「塚間衣」為「糞掃衣」的一種。
\item\subnoteref{352.1}\NoteSubKeyHead{(2)}\NoteKeywordNikayaHead{「穿糞掃衣者」}(paṃsukūlikā,另譯為「穿塵堆衣者」),菩提比丘長老英譯為\NoteKeywordBhikkhuBodhi{「碎布長袍之穿用者」}(rag-robe wearers)。
\stopitemgroup

\startitemgroup[noteitems]
\item\subnoteref{353.0}\NoteKeywordAgamaHead{「劫貝衣(SA/MA);劫波衣(DA);劫波育樹衣(AA)」},南傳作\NoteKeywordNikaya{「精緻的木綿衣」}(kappāsikasukhumāni),菩提比丘長老英譯為\NoteKeywordBhikkhuBodhi{「精緻的綿衣」}(fine cotton)。按「劫貝」即「木綿;綿花」(kappāsa)的音譯。
\stopitemgroup

\startitemgroup[noteitems]
\item\subnoteref{354.0}\NoteKeywordAgamaHead{「如來死後有無/如來有後生死/有無後死(SA);如來終不終(MA);如來終不終/有如是無如是他死(DA);有終有不終(AA)」},南傳作\NoteKeywordNikaya{「死後如來存在且不存在」}(hoti ca na ca hoti tathāgato paraṃ maraṇā),菩提比丘長老英譯為\NoteKeywordBhikkhuBodhi{「死後如來既存在也不存在」}(The Tathāgata both exists and does not exist after death)。按:\suttaref{SN.22.86}經文明確地指「如來」為「最高的人、無上的人、已證得無上成就」,顯然就是指佛陀,但《顯揚真義》等以「眾生」解說(tathāgatoti satto, \suttaref{SN.16.12}/\ccchref{MN.65}{https://agama.buddhason.org/MN/dm.php?keyword=65}/\ccchref{DN.1}{https://agama.buddhason.org/DN/dm.php?keyword=1}),而《大智度論》譯為「死後有神去後世,無神去後世,亦有神去亦無神去,死後亦非有神去亦非無神去後世」,其中的「神」指一般認為生死流轉中的「主體我」,也就是「如來」傳統含義之一,另參看\ccchref{SA.104}{https://agama.buddhason.org/SA/dm.php?keyword=104}「異色有如來」。
\stopitemgroup

\startitemgroup[noteitems]
\item\subnoteref{355.0}\NoteKeywordAgamaHead{「不覺、不達(\ccchref{MA.97}{https://agama.buddhason.org/MA/dm.php?keyword=97})」},南傳作\NoteKeywordNikaya{「不隨覺、不通達」}(ananubodhā appaṭivedhā ),菩提比丘長老、Maurice Walshe先生英譯為「不理解與不洞察」(not understanding and not penetrating, SN/DN)。按:《顯揚真義》、《吉祥悅意》都以「因[無]理解的遍知而不隨覺」(ñātapariññāvasena ananubujjhanā),「因[無]衡量、捨斷的遍知而不通達」(tīraṇappahānapariññāvasena appaṭivijjhanā)解說。
\stopitemgroup

\startitemgroup[noteitems]
\item\subnoteref{356.0}\NoteSubKeyHead{(1)}\NoteKeywordAgamaHead{「實(MA);真實之物(AA)」},南傳作\NoteKeywordNikaya{「心材;核心」}(sāre,另譯為「堅實;真實;真髓;核;樹心;心木;堅材」),菩提比丘長老英譯為\NoteKeywordBhikkhuBodhi{「心材」}(heartwood),或「本質;真髓」(essence, \ccchref{AN.5.64}{https://agama.buddhason.org/AN/an.php?keyword=5.64})。
\item\subnoteref{356.1}\NoteSubKeyHead{(2)}\NoteKeywordNikayaHead{「膚材」}(pheggū),菩提比丘長老英譯為\NoteKeywordBhikkhuBodhi{「軟材」}(softwood),智髻丘長老英譯為「邊材」(sapwood, MN)。
\item\subnoteref{356.2}\NoteSubKeyHead{(3)}\NoteKeywordNikayaHead{「內皮、外皮」}(tacapapaṭikā),菩提比丘長老英譯為\NoteKeywordBhikkhuBodhi{「樹皮與嫩枝」}(bark and shoots),智髻丘長老英譯為「其內樹皮與外樹皮」(its inner bark, and its outer bark, MN)。
\stopitemgroup

\startitemgroup[noteitems]
\item\subnoteref{357.0}\NoteSubKeyHead{(1)}\NoteKeywordAgamaHead{「十力;十種力(SA)」},南傳作\NoteKeywordNikaya{「十力」}(dasabala),或「十如來力」(Dasa tathāgatabalāni),其內容,參看\ccchref{SA.684}{https://agama.buddhason.org/SA/dm.php?keyword=684}、\ccchref{AA.46.4}{https://agama.buddhason.org/AA/dm.php?keyword=46.4}等。按:唯如來有此十力,故也以「十力」表示如來。
\item\subnoteref{357.1}\NoteSubKeyHead{(2)}\NoteKeywordNikayaHead{「四無畏;四無所畏」}(Cattāri vesārajjāni),其內容參看\ccchref{AA.27.6}{https://agama.buddhason.org/AA/dm.php?keyword=27.6}、\ccchref{AA.46.4}{https://agama.buddhason.org/AA/dm.php?keyword=46.4}等。
\stopitemgroup

\startitemgroup[noteitems]
\item\subnoteref{358.0}\NoteKeywordAgamaHead{「不忍(SA/MA);不忍受(GA);不能堪忍(AA)」},南傳作\NoteKeywordNikaya{「不接受;不容忍」}(nakkhamati),菩提比丘長老英譯為\NoteKeywordBhikkhuBodhi{「[沒有什麼是]可接受的」}([Nothing is] acceptable)。反義詞「接受;容忍」(khamati),另譯為「忍;忍耐;原諒」。
\stopitemgroup

\startitemgroup[noteitems]
\item\subnoteref{359.0}\NoteKeywordAgamaHead{「有道、有跡/有道、有向(SA);是道、是迹(DA)」},南傳作\NoteKeywordNikaya{「有道、有道跡」}(Atthi…maggo atthi paṭipadā),菩提比丘長老英譯為\NoteKeywordBhikkhuBodhi{「有道、有路」}(There is a path…there is a way)。
\stopitemgroup

\startitemgroup[noteitems]
\item\subnoteref{360.0}\NoteKeywordAgamaHead{「愛欲重者/苦貪者(SA);盛欲(GA);重濁欲(MA);貪欲熾盛(AA)」},南傳作\NoteKeywordNikaya{「重貪者」}(tibbarāgo),菩提比丘長老英譯為\NoteKeywordBhikkhuBodhi{「強烈易於慾望者」}(who is strongly prone to lust)。
\stopitemgroup

\startitemgroup[noteitems]
\item\subnoteref{361.0}\NoteSubKeyHead{(1)}\NoteKeywordNikayaHead{「醍醐」}(maṇḍo),菩提比丘長老英譯為\NoteKeywordBhikkhuBodhi{「精華」}(elite)。
\item\subnoteref{361.1}\NoteSubKeyHead{(2)}\NoteKeywordNikayaHead{「醍醐味」}(maṇḍapeyya),菩提比丘長老英譯為\NoteKeywordBhikkhuBodhi{「乳酪飲料」}(a beverage of cream)。按:《顯揚真義》說,以明淨之意(pasannaṭṭhena)為maṇḍaṃ,以能喝之意(pātabbaṭṭhena)為peyyaṃ,長老說,maṇḍa的原意指最好的乳品或奶油。
\item\subnoteref{361.2}\NoteSubKeyHead{(3)}\NoteKeywordNikayaHead{「熟酥醍醐」}(sappimaṇḍo,另譯為「酥精;醍醐味」),菩提比丘長老英譯為\NoteKeywordBhikkhuBodhi{「酥油的乳酪」}(cream-of-ghee)。
\stopitemgroup

\startitemgroup[noteitems]
\item\subnoteref{362.0}\NoteKeywordAgamaHead{「餓鬼;餓鬼趣(SA/DA/AA)」},南傳作\NoteKeywordNikaya{「在餓鬼界中/餓鬼界」}(pettivisaye/pettivisayo),菩提比丘長老英譯為\NoteKeywordBhikkhuBodhi{「在鬼域中」}(in the domain of ghosts, SN),或「受折磨的幽靈領域」(the sphere of afflicted spirits, AN)。
\stopitemgroup

\startitemgroup[noteitems]
\item\subnoteref{363.0}\NoteKeywordAgamaHead{「信心布施/信施(SA)」},南傳作\NoteKeywordNikaya{「亡者供養會」}(saddhaṃ, saddhāni),菩提比丘長老英譯為\NoteKeywordBhikkhuBodhi{「為死者的追悼儀式」}(the memorial rites for the dead, memory of the dead),並解說,這是婆羅門教傳統對已亡父母、祖父母、曾祖父母的祭祀。按:saddha有兩個意思:➀有信的;能信賴的。➁亡者供養會;供物。這裡北傳採前者譯,顯然不合適。《破斥猶豫》等以「祭拜死人的食物」(matakabhatte, \ccchref{MN.93}{https://agama.buddhason.org/MN/dm.php?keyword=93}/\ccchref{AN.3.60}{https://agama.buddhason.org/AN/an.php?keyword=3.60}),《吉祥悅意》以「對死者關於祭拜(所作)食物」(matake uddissa katabhatte, \ccchref{DN.3}{https://agama.buddhason.org/DN/dm.php?keyword=3})解說。
\stopitemgroup

\startitemgroup[noteitems]
\item\subnoteref{364.0}\NoteKeywordAgamaHead{「自手殺生/手自殺生/自殺生(SA)」},南傳作\NoteKeywordNikaya{「自己是殺生者」}(attanā ca pāṇātipātī hoti),菩提比丘長老英譯為\NoteKeywordBhikkhuBodhi{「某人毀滅生命」}(someone destroys life)。
\stopitemgroup

\startitemgroup[noteitems]
\item\subnoteref{365.0}\NoteKeywordAgamaHead{「法服;法衣;袈裟衣(SA/MA);染色衣(SA)」},南傳作\NoteKeywordNikaya{「袈裟衣」}(kāsāyāni vatthāni,袈裟-衣),菩提比丘長老英譯為\NoteKeywordBhikkhuBodhi{「黃色的長外袍」}(the yellow robe),或「赭土色長袍」(ochre robes, AN)。
\stopitemgroup

\startitemgroup[noteitems]
\item\subnoteref{366.0}\NoteKeywordAgamaHead{「塔所(SA);塔(GA);寺(MA);宗廟(DA);神祠/神寺/神廟/寺(AA)」},南傳作\NoteKeywordNikaya{「塔廟」}(cetiye,音譯為「支提;枝提;制多」),菩提比丘長老英譯為\NoteKeywordBhikkhuBodhi{「廟祠」}(Shrine)。
\stopitemgroup

\startitemgroup[noteitems]
\item\subnoteref{367.0}\NoteKeywordAgamaHead{「右脅而臥(SA/MA);右脇著地(AA);偃右脇如師子/偃右脅而臥(DA)」},南傳作\NoteKeywordNikaya{「以右脅作獅子臥」}(dakkhiṇena passena sīhaseyyaṃ kappeti),菩提比丘長老英譯為\NoteKeywordBhikkhuBodhi{「像獅子在其右側躺下」}(lies down, lion-like, on his right side)。
\stopitemgroup

\startitemgroup[noteitems]
\item\subnoteref{368.0}\NoteKeywordAgamaHead{「行意惡行(SA/MA);成就意惡行(MA/DA);意所作行而不平均,以行非法之行/意所作行而不平等(AA)」},南傳作\NoteKeywordNikaya{「他以意行惡行」}(manasā duccaritaṃ carati),菩提比丘長老英譯為\NoteKeywordBhikkhuBodhi{「他從事心的不正行為」}(He engages in misconduct of mind)。按:依\ccchref{SA.1039}{https://agama.buddhason.org/SA/dm.php?keyword=1039}等經,意惡行的內容為①貪(貪婪)-於他財物而起貪欲,言:此物我有者好②瞋恚(瞋害心)-心思惟言:彼眾生應縛、應鞭、應杖、應殺,欲為生難③邪見-無來世、善惡業報、聖者。
\stopitemgroup

\startitemgroup[noteitems]
\item\subnoteref{369.0}\NoteSubKeyHead{(1)}\NoteKeywordAgamaHead{「諸流/駛流/流淵/浚流/瀑駚流/流(SA);駛流/瀑駛流(GA)」},南傳作\NoteKeywordNikaya{「暴流」}(ogha,另譯作「流;洪水」),菩提比丘長老英譯為\NoteKeywordBhikkhuBodhi{「洪水」}(the flood)。按:\ccchref{SA.490}{https://agama.buddhason.org/SA/dm.php?keyword=490}舉「欲流、有流、見流、無明流」即\ccchref{DN.33}{https://agama.buddhason.org/DN/dm.php?keyword=33}說的「欲的暴流、有的暴流、見的暴流、無明的暴流」四種暴流。
\item\subnoteref{369.1}\NoteSubKeyHead{(2)}\NoteKeywordAgamaHead{「生死海流/有流(SA);有駛流/生死海(GA);有流(DA/AA)」},南傳作\NoteKeywordNikaya{「有的暴流」}(bhavogho),菩提比丘長老英譯為\NoteKeywordBhikkhuBodhi{「存在的洪水」}(the flood of existence)。按:《顯揚真義》以「對色無色有的欲貪」(rūpārūpabhavesu chandarāgo)解說,又以「對五種欲的欲貪」(pañcasu kāmaguṇesu chandarāgo)解說「欲的暴流」,以「六十二見」(dvāsaṭṭhi diṭṭhiyo)解說「見的暴流」,以「對四諦的無智」(catūsu saccesu aññāṇaṃ)解說「無明的暴流」(\suttaref{SN.45.172})。
\stopitemgroup

\startitemgroup[noteitems]
\item\subnoteref{370.0}\NoteKeywordAgamaHead{「須陀洹分/須陀洹道分/入流分(SA);須陀洹支(DA)」},南傳作\NoteKeywordNikaya{「入流支」}(sotāpattiyaṅgaṃ,另譯為「須陀洹支;預流支」),菩提比丘長老英譯為\NoteKeywordBhikkhuBodhi{「為求入流的要素;為獲得入流的要素」}(a factor for stream-entry; factors for attaining stream-entry,用於「善人的結交……」這組, \suttaref{SN.55.5}),或「入流者的要素;入流的要素」(the factors of a stream-enterer; factors of stream-entry,用於「於佛不壞淨……」這組, \suttaref{SN.55.16}),並解說這兩種「入流支」是不同的(distinct),但其巴利語都同,不過在\ccchref{DN.33}{https://agama.buddhason.org/DN/dm.php?keyword=33}中,後者的巴利語作「入流者的支」(sotāpannassa aṅgāni)。按:這樣的說法正好與\ccchref{SA.843}{https://agama.buddhason.org/SA/dm.php?keyword=843}所說相同,也與\ccchref{SA.646}{https://agama.buddhason.org/SA/dm.php?keyword=646}/\suttaref{SN.48.8}的對應相符。
\stopitemgroup

\startitemgroup[noteitems]
\item\subnoteref{371.0}\NoteKeywordAgamaHead{「化樂(SA/MA);化自在天(DA/AA)」},南傳作\NoteKeywordNikaya{「化樂天」}(Nimmānaratino devā),菩提比丘長老英譯為\NoteKeywordBhikkhuBodhi{「以創造為樂的天神」}(Devas who take delight in creating),Maurice Walshe先生英譯為「他們製造」(They've made, DN)。
\stopitemgroup

\startitemgroup[noteitems]
\item\subnoteref{372.0}\NoteKeywordAgamaHead{「他自在(SA);他化自在天(SA/GA/DA/AA);他化樂天(MA)」},南傳作\NoteSubEntryKey{(i)}\NoteKeywordNikaya{「他化自在天」}(Paranimmitavasavattin),菩提比丘長老英譯為\NoteKeywordBhikkhuBodhi{「控制所有被其他人(其他神)創造的天神」}(devas Who Control What is Created by Others, AN)。\NoteSubEntryKey{(ii)}\NoteKeywordNikaya{「自在天」}(devā vasavattino),菩提比丘長老英譯為\NoteKeywordBhikkhuBodhi{「行使(運用)控制的天」}(devas who exercise control, SN)。\NoteSubEntryKey{(iii)}\NoteKeywordNikaya{「他化天」}(paranimmitā),Maurice Walshe先生英譯為「那些奪取別人作品者」(those who seize on others' work, DN)。
\stopitemgroup

\startitemgroup[noteitems]
\item\subnoteref{373.0}\NoteSubKeyHead{(1)}\NoteKeywordAgamaHead{「覺悟之/發悟(SA);使懷恐怖(AA)」},南傳作\NoteKeywordNikaya{「使激起急迫感;使激起宗教心」}(saṃvejeti,另譯為「使驚怖;使悚懼」),菩提比丘長老英譯為\NoteKeywordBhikkhuBodhi{「激起急迫感」}(stir up a sense of urgency)。
\item\subnoteref{373.1}\NoteSubKeyHead{(2)}\NoteKeywordNikayaHead{「得到了急迫感」}(Saṃvegamāpādu, Saṃvegamāpādi,逐字譯為「急迫感-來到」),菩提比丘長老英譯為\NoteKeywordBhikkhuBodhi{「得到了急迫感」}(acquired a sense of urgency),Maurice Walshe先生英譯為「極心痛中」(in sore distress)。
\item\subnoteref{373.2}\NoteSubKeyHead{(3)}\NoteKeywordNikayaHead{「作急迫感」}(saṃvegamakāsi),Maurice Walshe先生英譯為「大恐慌」(dismayed)。
\item\subnoteref{373.3}\NoteSubKeyHead{(4)}\NoteKeywordNikayaHead{「生起急迫感」}(Saṃvegajātassa),Maurice Walshe先生英譯為「恐懼」(fears)。
\stopitemgroup

\startitemgroup[noteitems]
\item\subnoteref{374.0}\NoteKeywordAgamaHead{「戒身(SA/MA/AA);戒聚(MA/DA);戒眾(DA)」},南傳作\NoteKeywordNikaya{「戒蘊」}(sīlakkhandhaṃ,另譯為「戒身;戒犍度;戒聚集」),菩提比丘長老英譯為\NoteKeywordBhikkhuBodhi{「德行的聚集」}(aggregate of virtue)。
\stopitemgroup

\startitemgroup[noteitems]
\item\subnoteref{375.0}\NoteKeywordNikayaHead{「眾;僧」},南傳作\NoteKeywordNikaya{「僧團」}(saṅgha, saṃgha),另譯為「僧伽;和合眾」,或簡略為「僧,眾」。附:「弟子僧團」(sāvakasaṅgha),另譯為「聲聞僧伽」。
\stopitemgroup

\startitemgroup[noteitems]
\item\subnoteref{376.0}\NoteSubKeyHead{(1)}\NoteKeywordAgamaHead{「欲覺/貪覺(SA);欲念(MA);欲想(AA)」},南傳作\NoteKeywordNikaya{「欲尋」}(kāmavitakko),菩提比丘長老英譯為\NoteKeywordBhikkhuBodhi{「肉慾的心思」}(sensual thought)。
\item\subnoteref{376.1}\NoteSubKeyHead{(2)}\NoteKeywordAgamaHead{「恚覺(SA);恚念(MA);瞋恚想(AA)」},南傳作\NoteKeywordNikaya{「惡意尋」}(byāpādavitakko),菩提比丘長老英譯為\NoteKeywordBhikkhuBodhi{「惡意的心思」}(thought of ill will)。
\item\subnoteref{376.2}\NoteSubKeyHead{(3)}\NoteKeywordAgamaHead{「害覺(SA);害念(MA);殺害想(AA)」},南傳作\NoteKeywordNikaya{「加害尋」}(vihiṃsāvitakko),菩提比丘長老英譯為\NoteKeywordBhikkhuBodhi{「傷害的心思」}(thought of harming)。「加害」(vihiṃsā),另譯為「害;害意;惱害;傷害」。
\stopitemgroup

\startitemgroup[noteitems]
\item\subnoteref{377.0}\NoteKeywordAgamaHead{「叉手合掌(SA/DA/AA);叉手(MA/DA/AA);叉十(AA)」},南傳作\NoteKeywordNikaya{「合掌」}(añjaliṃ,另譯為「合十」),菩提比丘長老英譯為\NoteKeywordBhikkhuBodhi{「表示敬意地;謙恭地」}(respectfully)。「合掌的;合掌地」(pañjalika)。 
\stopitemgroup

\startitemgroup[noteitems]
\item\subnoteref{378.0}\NoteKeywordAgamaHead{「無齋(MA)」},南傳作\NoteKeywordNikaya{「無供養」}(natthi yiṭṭhaṃ),菩提比丘長老英譯為\NoteKeywordBhikkhuBodhi{「沒什麼東西供養」}(nothing offered)。按:「齋」,顯然是「供養」(yiṭṭha)的另譯。
\stopitemgroup

\startitemgroup[noteitems]
\item\subnoteref{379.0}\NoteKeywordAgamaHead{「得無所畏;成就無畏(DA)」},南傳作\NoteKeywordNikaya{「到達無畏的;已到達無畏」}(vesārajjappatto),菩提比丘長老英譯為\NoteKeywordBhikkhuBodhi{「已達到自信」}(attained self-confidence)。
\stopitemgroup

\startitemgroup[noteitems]
\item\subnoteref{380.0}\NoteKeywordAgamaHead{「不由他度/不由於他(SA);不隨他教(GA);不復從他/不復由他(MA);不信餘道/不由他信/不向餘道(DA);更不事餘師(AA)」},南傳作\NoteKeywordNikaya{「不緣於他」}(aparappaccayo,另譯為「不需要他人」),菩提比丘長老英譯為\NoteKeywordBhikkhuBodhi{「成為獨立者;成為獨立於他人」}(become independent of others)。按:《顯揚真義》以「不依賴(信賴)其他人(aññassa apattiyāyetvā),就有親身體驗的智(attapaccakkhañāṇamevassa, \suttaref{SN.12.15})」,《破斥猶豫》以「不依賴其他人」(aparappattiyo, \ccchref{MN.35}{https://agama.buddhason.org/MN/dm.php?keyword=35})解說。
\stopitemgroup

\startitemgroup[noteitems]
\item\subnoteref{381.0}\NoteSubKeyHead{(1)}\NoteKeywordNikayaHead{「梵行」}(brahmacariya),菩提比丘長老英譯為\NoteKeywordBhikkhuBodhi{「聖潔的生活」}(the holy life)。按:梵行的原義應指修往生梵天界的行為,演為「修清淨行」,又因梵天屬出欲界的色界,故亦指離淫欲,《集異門足論》說:「離二交會說名梵行。」
\item\subnoteref{381.1}\NoteSubKeyHead{(2)}\NoteKeywordAgamaHead{「梵行求(DA)」},南傳作\NoteKeywordNikaya{「梵行的尋求」}(brahmacariyesanā),Maurice Walshe先生英譯為「為聖潔生活的尋求」(quests for the holy life, DN),菩提比丘長老英譯為\NoteKeywordBhikkhuBodhi{「為聖潔生活的尋求」}(the search for a holy life, SN)。按:《顯揚真義》說,此指邪見考量(micchādiṭṭhisaṅkhātassa, \suttaref{SN.35.161})的梵行尋求,《吉祥悅意》則舉像十無記那樣的顛倒執見(vipariyesaggāho, \ccchref{DN.33}{https://agama.buddhason.org/DN/dm.php?keyword=33})。
\item\subnoteref{381.2}\NoteSubKeyHead{(3)}\NoteKeywordAgamaHead{「離非梵行/斷非梵行(MA)」},南傳作\NoteKeywordNikaya{「捨斷非梵行後」}(Abrahmacariyaṃ pahāya),智髻比丘長老英譯為「戒絕卑俗的性交實行」(abstaining from the vulgar practice of sexual intercourse, MN)。
\stopitemgroup

\startitemgroup[noteitems]
\item\subnoteref{382.0}\NoteKeywordAgamaHead{「受持學戒(SA);受持學要(MA);等學諸戒(DA)」},南傳作\NoteKeywordNikaya{「在諸學處上受持後學習」}(samādāya sikkhati sikkhāpadesu),菩提比丘長老英譯為\NoteKeywordBhikkhuBodhi{「承擔調馴規定後,在它們中訓練」}(Having undertaken the training rules, he trains in them)。或「受持諸學處」(sikkhāpadāni samādiyati),Maurice Walshe先生英譯為「承擔訓誡」(undertakes the precepts, \ccchref{DN.5}{https://agama.buddhason.org/DN/dm.php?keyword=5})。按:「學戒」或「學處」(sikkhāpada),即「戒條」。
\stopitemgroup

\startitemgroup[noteitems]
\item\subnoteref{383.0}\NoteKeywordAgamaHead{「如來、應、等正覺所知所見(SA);世尊知見如來、無所著、等正覺/世尊有知有見如來、無所著、等正覺(MA)」},南傳作\NoteKeywordNikaya{「有知、有見的世尊、阿羅漢、遍正覺者」}(bhagavatā jānatā passatā arahatā sammāsambuddhena),菩提比丘長老英譯為\NoteKeywordBhikkhuBodhi{「知道與看見的幸福者、阿羅漢、已純然無瑕的開化者」}(the Blessed One, the Arahant, the Perfectly Enlightened One, who knows and sees)。
\stopitemgroup

\startitemgroup[noteitems]
\item\subnoteref{384.0}\NoteKeywordAgamaHead{「惡慧(SA/MA);無慧(MA);無智(DA)」},南傳作\NoteKeywordNikaya{「劣慧」}(duppaññā,另譯為「惡慧;愚鈍」),菩提比丘長老英譯為\NoteKeywordBhikkhuBodhi{「缺乏智慧」}(lack wisdom)。「劣慧者;劣慧的」(duppañña, duppañño),另譯為「惡慧的;愚人」。
\stopitemgroup

\startitemgroup[noteitems]
\item\subnoteref{385.0}\NoteSubKeyHead{(1)}\NoteKeywordAgamaHead{「於露地;在露地;露地;於空露地(SA)」},南傳作\NoteKeywordNikaya{「在屋外;在露天處;露地」}(abbhokāse, abbhokāsaṃ),菩提比丘長老英譯為\NoteKeywordBhikkhuBodhi{「在戶外」}(in the open)。
\item\subnoteref{385.1}\NoteSubKeyHead{(2)}\NoteKeywordNikayaHead{「天空下著毛毛雨」}(devo ca ekamekaṃ phusāyati, 而天空一一靜靜地下),菩提比丘長老英譯為\NoteKeywordBhikkhuBodhi{「它下著毛毛雨」}(it was drizzling)。按:《顯揚真義》對此句沒解說,《勝義燈》則以「雨雲使有雨滴的水滴落下」(Devoti megho ekamekaṃ phusitakaṃ udakabinduṃ pāteti, \ccchref{Ud.7}{https://agama.buddhason.org/Ud/dm.php?keyword=7})解說這裡的devo。
\stopitemgroup

\startitemgroup[noteitems]
\item\subnoteref{386.0}\NoteKeywordAgamaHead{「扶床欲起/馮床欲起/憑床欲起/從床欲起(SA)」}(凭=憑),南傳作\NoteKeywordNikaya{「在臥床上移動」}(mañcake samadhosi),菩提比丘長老英譯為\NoteKeywordBhikkhuBodhi{「在他的床上起身」}(stirred on his bed, SN/AN),並解說,即使臥病在床,見長輩來,要有起身的表示,以示尊敬(\suttaref{SN.22.87})。
\stopitemgroup

\startitemgroup[noteitems]
\item\subnoteref{387.0}\NoteKeywordAgamaHead{「麁摶食(SA);摶食麁細(MA);摶食(DA);摶食或大或小/揣食(AA)」},南傳作\NoteKeywordNikaya{「或粗或細的物質食物」}(Kabaḷīkāro āhāro– oḷāriko vā sukhumo vā),菩提比丘長老英譯為\NoteKeywordBhikkhuBodhi{「食用的滋養食物,粗劣的或精緻的」}(The nutriment edible food, gross or subtle)。按:「物質食物」(Kabaḷīkāra āhāra, kabaliṅkāra āhāra, kabaliṅkārāhāra),古譯為「摶食;段食」,「摶」同「團」,應該即是以手抓起或捏成的「飯團」(kabala, kabaḷa)。\ccchref{DA.30}{https://agama.buddhason.org/DA/dm.php?keyword=30}說:自上諸天以禪定喜樂為食,即色界眾生以禪定喜樂為食,不食摶食。
\stopitemgroup

\startitemgroup[noteitems]
\item\subnoteref{388.0}\NoteKeywordAgamaHead{「細觸食(SA);更樂食(MA/AA);細滑食(DA)」},南傳作\NoteKeywordNikaya{「觸[食]」}(phasso [āhāro]),菩提比丘長老英譯為\NoteKeywordBhikkhuBodhi{「接觸[滋養物]」}(contact [nutriment])。按:《顯揚真義》等以「三受」(tisso vedanā, \suttaref{SN.12.11}/\ccchref{MN.9}{https://agama.buddhason.org/MN/dm.php?keyword=9})解說。\ccchref{DA.30}{https://agama.buddhason.org/DA/dm.php?keyword=30}說:「卵生眾生觸食。」
\stopitemgroup

\startitemgroup[noteitems]
\item\subnoteref{389.0}\NoteKeywordAgamaHead{「意思食(SA);意思食/意念[食](MA);念食(DA/AA)」},南傳作\NoteKeywordNikaya{「意思[食]」}(manosañcetanā [āhāro]),菩提比丘長老英譯為\NoteKeywordBhikkhuBodhi{「精神意志[滋養物]」}(mental volition [nutriment])。按:「意思」為「意」(mano)與「思」(sañcetanā)之複合字,《顯揚真義》等以「三有」(tayo bhave, \suttaref{SN.12.11}/\ccchref{MN.9}{https://agama.buddhason.org/MN/dm.php?keyword=9})解說。
\stopitemgroup

\startitemgroup[noteitems]
\item\subnoteref{390.0}\NoteKeywordNikayaHead{「識食」}(viññāṇāhāro),菩提比丘長老英譯為\NoteKeywordBhikkhuBodhi{「識滋養物」}(the nutriment consciousness)。按:《顯揚真義》等以「結生名色」(paṭisandhināmarūpanti, \suttaref{SN.12.11}/\ccchref{MN.9}{https://agama.buddhason.org/MN/dm.php?keyword=9})解說。\ccchref{DA.30}{https://agama.buddhason.org/DA/dm.php?keyword=30}說:地獄眾生及無色天,是名識食。
\stopitemgroup

\startitemgroup[noteitems]
\item\subnoteref{391.0}\NoteSubKeyHead{(1)}\NoteKeywordNikayaHead{「阿賴耶」}(ālaya,另譯為「執著;愛著;所執處;窟宅」),菩提比丘長老英譯為\NoteKeywordBhikkhuBodhi{「黏著」}(adhesion, \suttaref{SN.6.1}),或「窩」(the lair, \suttaref{SN.1.45}),或「依靠它」(reliance on it, \suttaref{SN.22.22}),或「執著」(attachment, \ccchref{AN.4.34}{https://agama.buddhason.org/AN/an.php?keyword=4.34}),智髻比丘長老英譯為「渴愛」(craving, \ccchref{MN.9}{https://agama.buddhason.org/MN/dm.php?keyword=9}),或「放縱」(indulgences, \ccchref{MN.98}{https://agama.buddhason.org/MN/dm.php?keyword=98})。《顯揚真義》等說,眾生在五種欲上黏著(sattā pañcasu kāmaguṇesu allīyanti),或黏著百八種渴愛思潮(Aṭṭhasatataṇhāvicaritāni ālayanti, \suttaref{SN.6.1}/\ccchref{MN.26}{https://agama.buddhason.org/MN/dm.php?keyword=26}/\ccchref{DN.14}{https://agama.buddhason.org/DN/dm.php?keyword=14}),因此,那些被稱為阿賴耶。
\item\subnoteref{391.1}\NoteSubKeyHead{(2)}\NoteKeywordAgamaHead{「不染著法(\ccchref{AA.25.3}{https://agama.buddhason.org/AA/dm.php?keyword=25.3})」},南傳作\NoteKeywordNikaya{「無阿賴耶的法」}(anālaye dhamme),菩提比丘長老英譯為\NoteKeywordBhikkhuBodhi{「關於不執著的法」}(the Dhamma about non-attachment, AN)。
\stopitemgroup

\startitemgroup[noteitems]
\item\subnoteref{392.0}\NoteKeywordAgamaHead{「(制)戒(SA/AA);禁戒(SA/MA);學戒(SA/MA/DA)」},南傳作\NoteKeywordNikaya{「學處」}(sikkhāpadaṃ),菩提比丘長老英譯為\NoteKeywordBhikkhuBodhi{「調馴規定」}(training rule)。按:「學處」(sikkhāpada)即「戒條」,集所有「戒條」稱為「波羅提木叉」(pātimokkha),參看印順法師《原始佛教聖典之集成》p.113,139。而「戒」(sīla),另譯為「習慣;道德」,菩提比丘長老英譯為\NoteKeywordBhikkhuBodhi{「有德的」}(virtuous)。
\stopitemgroup

\startitemgroup[noteitems]
\item\subnoteref{393.0}\NoteKeywordAgamaHead{「知心之所念(SA/AA);觀…心/觀察…心(MA);知…心中所念(DA/AA);觀見…心中所念之事/觀察…心意(AA)」},南傳作\NoteKeywordNikaya{「以心熟知心後」}(cetasā ceto paricca),菩提比丘長老英譯為\NoteKeywordBhikkhuBodhi{「以他自己的心圍繞他們的心」}(encompassing their minds with his own mind)。按:《滿足希求》以「以自己的心分別(確定, paricchinditvā, \ccchref{AN.1.43}{https://agama.buddhason.org/AN/an.php?keyword=1.43})他的心」解說。
\stopitemgroup

\startitemgroup[noteitems]
\item\subnoteref{394.0}\NoteKeywordAgamaHead{「此有故,彼有,此起故,彼起,此無故,彼無,此滅故,彼滅/是事有故(有是故),是事有,是事起故,是事起(SA),因是,有是,因是生故,彼則得生,若因不生,則彼不生(GA),因此有彼,無此無彼,此生彼生,此滅彼滅(MA),緣是,有是,無是,則無/是有,則有,此生,則生,此起,則起,此滅,則滅(AA)」},南傳作\NoteKeywordNikaya{「在這個存在時那個存在,以這個的生起那個生起,在這個不存在時那個不存在,以這個的滅那個被滅」}(imasmiṃ sati idaṃ hoti, imassuppādā idaṃ uppajjati; imasmiṃ asati idaṃ na hoti imassa nirodhā idaṃ nirujjhati),菩提比丘長老英譯為\NoteKeywordBhikkhuBodhi{「當這個存在,則那個生成;以這個的發生,則那個發生;當這個不存在了,則那個不生成;以這個的停止,則那個終止」}(when this exists, that comes to be; with the arising of this, that arises. When this does not exist, that does not come to be; with the cessation of this, that ceases)。
\stopitemgroup

\startitemgroup[noteitems]
\item\subnoteref{395.0}\NoteKeywordAgamaHead{「現法、……緣自覺知;現法、……智慧自覺;現法、……自覺證知」,南傳作「法是被世尊善說的、……應該被智者各自經驗的」,此為有關「法」的定型句,參看\ccchref{SA.215}{https://agama.buddhason.org/SA/dm.php?keyword=215}「附記」}。
\stopitemgroup

\startitemgroup[noteitems]
\item\subnoteref{396.0}\NoteKeywordAgamaHead{「隨憶(求憶)、隨覺、隨觀(SA);意所思念(MA)」},南傳作\NoteKeywordNikaya{「被意所隨行」}(anuvicaritaṃ manasā),智髻比丘長老英譯為「心理的沉思」(mentally pondered, MN),菩提比丘長老英譯為\NoteKeywordBhikkhuBodhi{「被心涉及(衡量)」}(ranged over by the mind, SN)。按:《顯揚真義》等以「被心隨行」(cittena anusañcaritaṃ, \suttaref{SN.24.2}/\ccchref{MN.22}{https://agama.buddhason.org/MN/dm.php?keyword=22}/\ccchref{AN.4.23}{https://agama.buddhason.org/AN/an.php?keyword=4.23})解說。菩提比丘長老說,「所見」是「可見色處」(the visible-form base),「所聞」是「聲處」(the sound base),「所覺」是「所嗅、所嚐、所接觸的對象」(the objects of smell, taste, and touch),「所得、所求、被意所隨行」是第四者(所識)的詳述(elaboration)。「隨所求」(anupariyesitaṃ),菩提比丘長老英譯「尋求」(sought after)。
\stopitemgroup

\startitemgroup[noteitems]
\item\subnoteref{397.0}\NoteKeywordAgamaHead{「欲有漏、有有漏、無明有漏(SA);欲漏、有漏、無明漏(GA/MA/DA/AA)」},南傳作\NoteKeywordNikaya{「欲漏、有漏、無明漏」}(Kāmāsavo, bhavāsavo, avijjāsavo),菩提比丘長老英譯為\NoteKeywordBhikkhuBodhi{「感官快樂的污染、存在的污染、無知的污染」}(the taint of sensuality, the taint of existence, the taint of ignorance)。按:「漏;煩惱」(āsava),原意是「流出來;漏出來」,引申為「(生死)煩惱」的異名,《破斥猶豫》等說,五種欲的貪(pañcakāmaguṇiko rāgo, \ccchref{MN.2}{https://agama.buddhason.org/MN/dm.php?keyword=2}/\ccchref{DN.33}{https://agama.buddhason.org/DN/dm.php?keyword=33})為欲漏,對色有(色界)無色有的欲貪(rupārūpabhave chandarāgo),及欲求禪定者的與常見斷見俱行(jhānanikanti ca sassatucchedadiṭṭhisahagatā, \ccchref{MN.2}{https://agama.buddhason.org/MN/dm.php?keyword=2}),與常見俱行的貪,或因有而有熱望(sassatadiṭṭhisahagato rāgo, bhavavasena vā patthanā, \ccchref{DN.33}{https://agama.buddhason.org/DN/dm.php?keyword=33})為有漏,對四諦的無智(catūsu saccesu aññāṇaṃ, \ccchref{MN.2}{https://agama.buddhason.org/MN/dm.php?keyword=2}),或對苦的無智(Dukkhe aññāṇa, \ccchref{DN.33}{https://agama.buddhason.org/DN/dm.php?keyword=33})等而來到無明,名為無明漏。
\stopitemgroup

\startitemgroup[noteitems]
\item\subnoteref{398.0}\NoteKeywordNikayaHead{「隨學」}(anusikkhati,另譯為「模倣;仿效」),菩提比丘長老英譯為\NoteKeywordBhikkhuBodhi{「練習」}(train),或「熱心模仿;仿學」(emulates, \ccchref{AN.8.54}{https://agama.buddhason.org/AN/an.php?keyword=8.54})。
\stopitemgroup

\startitemgroup[noteitems]
\item\subnoteref{399.0}\NoteKeywordAgamaHead{「中道(SA/MA);處中之道(AA)」},南傳作\NoteKeywordNikaya{「中道」}(majjhimā paṭipadā),菩提比丘長老英譯為\NoteKeywordBhikkhuBodhi{「中間道路」}(the Middle Way)。按:\ccchref{MA.88}{https://agama.buddhason.org/MA/dm.php?keyword=88}說有兩種中道:➀斷念欲,亦斷惡(厭惡)念欲等➁八正道。\ccchref{MA.169}{https://agama.buddhason.org/MA/dm.php?keyword=169}、\ccchref{MA.204}{https://agama.buddhason.org/MA/dm.php?keyword=204}說離「求欲樂與求自身苦行」為中道,內容即八正道,\ccchref{SA.912}{https://agama.buddhason.org/SA/dm.php?keyword=912}則分稱兩者為「初中道、第二中道」。\ccchref{SA.301}{https://agama.buddhason.org/SA/dm.php?keyword=301}等六經說離常見斷見的「有無」二邊為中道,內容即「緣起法」。南傳除\ccchref{AN.3.157}{https://agama.buddhason.org/AN/an.php?keyword=3.157}說是四念住外,均作八正道(\suttaref{SN.42.12},\suttaref{SN.56.11},\ccchref{MN.3}{https://agama.buddhason.org/MN/dm.php?keyword=3},\ccchref{MN.139}{https://agama.buddhason.org/MN/dm.php?keyword=139})。
\stopitemgroup

\startitemgroup[noteitems]
\item\subnoteref{400.0}\NoteKeywordAgamaHead{「我慢(SA/MA/DA);憍慢(AA)」},南傳作\NoteKeywordNikaya{「我是之慢」}(asmimāno,另譯為「我慢」),菩提比丘長老英譯為\NoteKeywordBhikkhuBodhi{「『我是』之自大」}(the conceit 'I am,', AN),Maurice Walshe先生英譯為「自我-自大」(Ego-conceit, DN)。按:《破斥猶豫》以「『我是關於色』之慢…關於受…關於想…關於行…『我是關於識』之慢」(asmīti māno, vedanāya… saññāya… saṅkhāresu… viññāṇe asmīti māno, \ccchref{MN.22}{https://agama.buddhason.org/MN/dm.php?keyword=22})解說。《吉祥悅意》以「阿羅漢道」(arahattamaggo, \ccchref{DN.33}{https://agama.buddhason.org/DN/dm.php?keyword=33})解說「我是之慢的根除」。
\stopitemgroup

\startitemgroup[noteitems]
\item\subnoteref{401.0}\NoteKeywordAgamaHead{「淨信心(SA);無疑心(GA/MA);有歡喜心(MA);信心清淨(SA/DA/AA);心淨(DA);心意歡悅/柔軟和悅/歡喜/發心歡喜(AA)」},南傳作\NoteKeywordNikaya{「淨信心的;淨信心地」}(pasannacitto,另譯為「明淨心的;有喜悅心的」),菩提比丘長老英譯為\NoteKeywordBhikkhuBodhi{「有信心」}(with a mind of confidence, \ccchref{AN.9.20}{https://agama.buddhason.org/AN/an.php?keyword=9.20}, With a confident mind, \ccchref{AN.8.21}{https://agama.buddhason.org/AN/an.php?keyword=8.21}),或「心理平靜的」(mentally placid, \ccchref{AN.1.44}{https://agama.buddhason.org/AN/an.php?keyword=1.44})。
\stopitemgroup

\startitemgroup[noteitems]
\item\subnoteref{402.0}\NoteSubKeyHead{(1)}\NoteKeywordNikayaHead{「凝乳」}(dadhi,另譯為「酪」),菩提比丘長老英譯為\NoteKeywordBhikkhuBodhi{「乳酪」}(cream),或「凝乳」(curd, \ccchref{AN.4.95}{https://agama.buddhason.org/AN/an.php?keyword=4.95})。
\item\subnoteref{402.1}\NoteSubKeyHead{(2)}\NoteKeywordNikayaHead{「生酥」}(navanītaṃ),菩提比丘長老英譯為\NoteKeywordBhikkhuBodhi{「奶油」}(butter)。
\item\subnoteref{402.2}\NoteSubKeyHead{(3)}\NoteKeywordNikayaHead{「熟酥」}(sappi),菩提比丘長老英譯為\NoteKeywordBhikkhuBodhi{「酥油」}(ghee)。
\stopitemgroup

\startitemgroup[noteitems]
\item\subnoteref{403.0}\NoteKeywordAgamaHead{「劫數成壞(SA);世間成敗/成敗/成敗劫(MA);成劫敗劫/劫數成敗(DA);劫成敗/成敗劫(AA)」},南傳作\NoteKeywordNikaya{「壞成劫」}(saṃvaṭṭavivaṭṭakappe),菩提比丘長老英譯為\NoteKeywordBhikkhuBodhi{「世界-分解與發展的無限長的時代」}(eons of world-dissolution and evolution)。「壞成」(saṃvaṭṭavivaṭṭaṃ),Maurice Walshe先生英譯為「收縮與膨脹」(contraction and expansion, \ccchref{DN.1}{https://agama.buddhason.org/DN/dm.php?keyword=1})。
\stopitemgroup

\startitemgroup[noteitems]
\item\subnoteref{404.0}\NoteKeywordAgamaHead{「梵室;梵堂(AA)」},南傳作\NoteSubEntryKey{(i)}\NoteKeywordNikaya{「梵住」}(brahmavihāro),菩提比丘長老英譯為\NoteKeywordBhikkhuBodhi{「神的住處;神聖的住處」}(a divine dwelling, \suttaref{SN.54.11})。\NoteSubEntryKey{(ii)}\NoteKeywordNikaya{「與梵天共住」}(brahmānaṃ sahabyatāya),菩提比丘長老英譯為\NoteKeywordBhikkhuBodhi{「梵天的伙伴」}(the company of Brahma)。按:「四梵室」即修往生梵天界的「慈、悲、喜、捨」無量。
\stopitemgroup

\startitemgroup[noteitems]
\item\subnoteref{405.0}\NoteKeywordAgamaHead{「神通大力/大神通力(SA);大如意足(MA/DA);大神足(DA/AA)」},南傳作\NoteKeywordNikaya{「大神通力的(者);大神通力狀態」}(mahiddhikā;mahiddhikatā),菩提比丘長老英譯為\NoteKeywordBhikkhuBodhi{「很大力量」}(very powerful, AN),或「大超常力量」(great spiritual power, SN)。
\stopitemgroup

\startitemgroup[noteitems]
\item\subnoteref{406.0}\NoteKeywordNikayaHead{「如法法王」}(dhammiko dhammarājā),菩提比丘長老英譯為\NoteKeywordBhikkhuBodhi{「以正法統治的正義之王」}(a righteous king who ruled by the Dhamma),長老認為「法王」一詞,只適於指佛陀,不宜用來稱轉輪王,所以譯為「正義之王」(\suttaref{SN.41.10})。
\stopitemgroup

\startitemgroup[noteitems]
\item\subnoteref{407.0}\NoteSubKeyHead{(1)}\NoteKeywordAgamaHead{「觸(SA/DA);更樂(MA/AA)」},南傳作\NoteKeywordNikaya{「觸」}(samphassa或phassa)。按:這個「觸」即「六入緣觸,觸緣受」的「觸」,個人以外境進入對自己有意義的程度理解。
\item\subnoteref{407.1}\NoteSubKeyHead{(2)}\NoteKeywordNikayaHead{「緣於觸」}(phassaṃ paṭicca),菩提比丘長老英譯為\NoteKeywordBhikkhuBodhi{「依於接觸」}(dependent on contact),並說,即使任何邪見主張,那也沒離開依於觸。
\stopitemgroup

\startitemgroup[noteitems]
\item\subnoteref{408.0}\NoteKeywordNikayaHead{「成為非有」}(anabhāvaṃkata,逐字譯為「非+有+已作的」,「非有」也譯為「虛無;非存在」),菩提比丘長老英譯為\NoteKeywordBhikkhuBodhi{「已被刪除」}(obliterated)。按:《滿足希求》以「作為後續不存在的」(anuabhāvaṃ katānīti, \suttaref{SN.12.35})解說,《滿足希求》說,沒有後面的有(pacchābhāvo na hoti, \ccchref{AN.8.11}{https://agama.buddhason.org/AN/an.php?keyword=8.11})。
\stopitemgroup

\startitemgroup[noteitems]
\item\subnoteref{409.0}\NoteSubKeyHead{(1)}\NoteKeywordAgamaHead{「無量種(MA)」},南傳作\NoteKeywordNikaya{「無等同彼的狀態」}(atammayatā),智髻比丘老英譯為「無辨識(我)」(Non-identification, MN)。按:《破斥猶豫》以無渴愛(nittaṇhāti)解說,與北傳經文下一句「若有計者是謂愛也」相合。
\item\subnoteref{409.1}\NoteSubKeyHead{(2)}\NoteKeywordNikayaHead{「不等同彼的;不等同彼者」}(atammayo),菩提比丘長老英譯為\NoteKeywordBhikkhuBodhi{「完全不辨識(我)」}(identifies with nothing at all, \ccchref{AN.3.40}{https://agama.buddhason.org/AN/an.php?keyword=3.40}),或「無辨識(我)」(without identification, \ccchref{AN.6.104}{https://agama.buddhason.org/AN/an.php?keyword=6.104})。按:「彼;那個」(taṃ),即古婆羅門宗教哲學中的「梵」;個人「真我」所歸趣的「大我」,《破斥猶豫》說,「我以那清淨戒不成為等同彼的(na tammayo)、沒有渴愛的(na sataṇho),僅是清淨戒的狀態,我是離渴愛的(parisuddhasīlattāva nittaṇhohamasmīti, \ccchref{MN.47}{https://agama.buddhason.org/MN/dm.php?keyword=47})」;《滿足希求》以「無渴愛與見」(taṇhādiṭṭhiyo, tāhi rahito, \ccchref{AN.6.104}{https://agama.buddhason.org/AN/an.php?keyword=6.104})解說「不等同彼的」。
\stopitemgroup

\startitemgroup[noteitems]
\item\subnoteref{410.0}\NoteKeywordAgamaHead{「出家空閑(SA);出家學道發露曠大(MA)」},南傳作\NoteKeywordNikaya{「出家是露地」}(abbhokāso pabbajjā),菩提比丘長老英譯為\NoteKeywordBhikkhuBodhi{「出家(外出)的生活是開敞的」}(life gone forth is wide open, AN/MN),或「出家(外出)像是戶外」(The going forth is like the open air, SN)。
\stopitemgroup

\startitemgroup[noteitems]
\item\subnoteref{411.0}\NoteKeywordNikayaHead{「磨亮海螺」}(saṅkhalikhita),菩提比丘長老英譯為\NoteKeywordBhikkhuBodhi{「像拋光的海螺」}(like polished conch, SN),或「如一個拋光的海螺殼」(as a polished conch shell, AN),智髻比丘長老英譯為或「如一個拋光的殼」(as a polished shell, MN)。
\stopitemgroup

\startitemgroup[noteitems]
\item\subnoteref{412.0}\NoteKeywordAgamaHead{「隨事所用/隨意所作(SA)」},南傳作\NoteKeywordNikaya{「適合作業的」}(kammaniyaṃ, kammañña,另譯為「適合工作的;準備好的;堪任的;能維持住事業的」),菩提比丘長老英譯為\NoteKeywordBhikkhuBodhi{「益於使用的;適合的」}(wieldy, SN/AN),智髻比丘長老英譯為「可使用的;可加工的」(workable, MN)。
\stopitemgroup

\startitemgroup[noteitems]
\item\subnoteref{413.0}\NoteKeywordAgamaHead{「尊祐(MA)」},南傳作\NoteKeywordNikaya{「主宰者」}(issara,另譯為「自在者, 自在天」),菩提比丘長老英譯為\NoteKeywordBhikkhuBodhi{「上帝」}(God)。
\stopitemgroup

\startitemgroup[noteitems]
\item\subnoteref{414.0}\NoteKeywordAgamaHead{「愛語、行利、同利(SA);愛言、以利、等利(MA);愛語、利人、等利/善言、利益、同利(DA);愛敬、利人、等利(AA)」},南傳作\NoteKeywordNikaya{「愛語、利行、平等」}(peyyavajjaṃ, atthacariyā, samānattatā),菩提比丘長老英譯為\NoteKeywordBhikkhuBodhi{「可愛的話、仁慈行為、公正無私」}(endearing speech, beneficent conduct, and impartiality)。按:「平等」(samānattatā),原意為「相同的狀態;等同的情況」。
\stopitemgroup

\startitemgroup[noteitems]
\item\subnoteref{415.0}\NoteKeywordAgamaHead{「順諸法說/順法/隨順法說/諸次法說(SA);依法順法作如是答(DA)」},南傳作\NoteKeywordNikaya{「法隨法地解說;法隨法地回答」}(dhammassa cānudhammaṃ byākaronti, dhammassa cānudhammaṃ byākaroti),菩提比丘長老英譯為\NoteKeywordBhikkhuBodhi{「按照法(根據法)解說」}(explain in accordance with the Dhamma),智髻比丘長老英譯為「按照法(根據法)回答」(answers in accordance with the Dhamma, \ccchref{MN.103}{https://agama.buddhason.org/MN/dm.php?keyword=103})。
\stopitemgroup

\startitemgroup[noteitems]
\item\subnoteref{416.0}\NoteKeywordAgamaHead{「滅盡定/滅正受/滅(SA);想知滅定/知滅解脫/想知滅/想知滅盡(MA);滅盡定/想知滅定/想知滅(DA);想知滅/滅盡三昧(AA)」},南傳作\NoteKeywordNikaya{「想受滅;想、被感受的之滅」}(saññāvedayitanirodhaṃ),菩提比丘長老英譯為\NoteKeywordBhikkhuBodhi{「認知與感受的停止」}(the cessation of perception and feeling),並解說「想受滅」也叫「滅等至」(nirodhasamāpatti),這是一種心與心理的功能都停止的定之狀況,只有能入八定的阿羅漢或證不還果(第三果)者能進得去(\suttaref{SN.41.6})。
\stopitemgroup

\startitemgroup[noteitems]
\item\subnoteref{417.0}\NoteKeywordAgamaHead{「阿毘曇、律(SA);阿毘曇、毘尼(GA);律、阿毘曇(MA);律、阿毘曇經/阿毘曇、律(AA)」},南傳作\NoteKeywordNikaya{「阿毘達磨、阿毘毘奈耶」}(abhidhamme abhivinaye),菩提比丘長老英譯為\NoteKeywordBhikkhuBodhi{「更高的法與更高的戒律」}(the higher Dhamma and the higher Discipline)。按:i.《破斥猶豫》以阿毘達磨藏與毘奈耶藏(abhidhammapiṭake ceva vinayapiṭake)解說,但長老認為這顯然不合時宜,因為那是在尼科耶之後的產物,經文所說的阿毘達磨可能是對教理的系統化與分析,為阿毘達磨藏的核心起源(\ccchref{MN.32}{https://agama.buddhason.org/MN/dm.php?keyword=32}),而雖然沒有阿毘毘奈耶的實體文獻,但按字面的意思,可能指研究律的系統化與分析,也許隱在律藏的「經分別」(Suttavibhaṅga)中(\ccchref{MN.69}{https://agama.buddhason.org/MN/dm.php?keyword=69})。ii.印順法師在《原始佛教聖典之集成》中列出十部北傳漢譯現存屬於各部派的律論(阿毘毘奈耶)(p.83)。iii.阿毘達磨(Abhidhamma),另譯為「論,對法,勝法」,因為「abhi」有「對(於)~,向(著)~,勝(過)~,越過~,在~之上」諸多含意,現存的「論藏」(阿毘達磨藏)應源於十二分教中的「論議」(upadeśa,優波提舍),而經中所說的阿毘達磨,可能是「論議」的雛形。
\stopitemgroup

\startitemgroup[noteitems]
\item\subnoteref{418.0}\NoteSubKeyHead{(1)}\NoteKeywordAgamaHead{「行處具足(SA);善攝威儀禮節(MA);成就威儀(DA)」},南傳作\NoteKeywordNikaya{「具足正行行境的」}(ācāragocarasampanno),菩提比丘長老英譯為\NoteKeywordBhikkhuBodhi{「在好的行為與適當的去處上達成」}(accomplished in good conduct and proper resort)。
\item\subnoteref{418.1}\NoteSubKeyHead{(2)}\NoteKeywordNikayaHead{「托鉢的村落(行境村落)」}(gocaragāmo),智髻比丘長老英譯為「施捨處的村落」(a village for alms resort, MN)。
\stopitemgroup

\startitemgroup[noteitems]
\item\subnoteref{419.0}\NoteKeywordAgamaHead{「受(MA/DA/AA)」}為「取;執取」(upādāna)的另譯,菩提比丘長老英譯為\NoteKeywordBhikkhuBodhi{「執著」}(clinging)。
\stopitemgroup

\startitemgroup[noteitems]
\item\subnoteref{420.0}\NoteKeywordAgamaHead{「發露(SA/MA/DA)」},南傳作\NoteKeywordNikaya{「公開」}(vivarati),智髻比丘長老英譯為「顯現它;顯露它」(reveal it)。或「承認作的事」(paṭiññātakaraṇaṃ,古譯為「自言治」),智髻比丘長老英譯為「承認犯罪的實行」(the effecting of acknowledgement of an offence, \ccchref{MN.104}{https://agama.buddhason.org/MN/dm.php?keyword=104})。
\stopitemgroup

\startitemgroup[noteitems]
\item\subnoteref{421.0}\NoteSubKeyHead{(1)}\NoteKeywordAgamaHead{「天使(MA);身使(AA)」},南傳作\NoteKeywordNikaya{「天使」}(devadūtaṃ),菩提比丘長老英譯為\NoteKeywordBhikkhuBodhi{「天使;天的使者;天的信差」}(divine messenger)。
\item\subnoteref{421.1}\NoteSubKeyHead{(2)}\NoteKeywordAgamaHead{「閻王(MA);閻羅王(AA)」},南傳作\NoteKeywordNikaya{「閻摩王」}(yamo rājā),菩提比丘長老英譯為\NoteKeywordBhikkhuBodhi{「閻摩王」}(King Yama),並解說,祂是死神。
\item\subnoteref{421.2}\NoteSubKeyHead{(3)}\NoteKeywordAgamaHead{「獄卒(MA/DA/AA);閻王人(MA)」},南傳作\NoteKeywordNikaya{「獄卒」}(nirayapālā,逐字譯為「地獄+看守人(守護者)」),菩提比丘長老英譯為\NoteKeywordBhikkhuBodhi{「地獄的看守者」}(the wardens of hell)。
\stopitemgroup

\startitemgroup[noteitems]
\item\subnoteref{422.0}\NoteKeywordAgamaHead{「最上慢/大慢/增慢(MA),增上慢(DA/AA)」},南傳作\NoteSubEntryKey{(i)}\NoteKeywordNikaya{「增上慢」}(adhimānaṃ),菩提比丘長老英譯為\NoteKeywordBhikkhuBodhi{「高估自己」}(over-rates oneself/self-overestimation/overestimates himself)。\NoteSubEntryKey{(ii)}\NoteKeywordNikaya{「極慢/極慢者」}(atimāno/atimānī,另譯為「過慢/過慢者」),菩提比丘長老英譯為\NoteKeywordBhikkhuBodhi{「自大;傲慢」}(arrogance/arrogant)。按:《法蘊足論》說:云何慢?謂:於劣謂己勝,或於等謂己等,由此起慢,已慢、當慢、心舉恃、心自取,總名為慢。云何過慢?謂:於等謂己勝,或於勝謂己等,由此起慢,……乃至心自取,總名過慢。云何慢過慢?謂:於勝謂己勝,由此起慢,……乃至心自取,總名慢過慢。《滿足希求》以「征服後思量的」(atikkamitvā maññanamānassa, \ccchref{AN.2.232}{https://agama.buddhason.org/AN/an.php?keyword=2.232})解說。
\stopitemgroup

\startitemgroup[noteitems]
\item\subnoteref{423.0}\NoteKeywordAgamaHead{「行欲(\ccchref{MA.135}{https://agama.buddhason.org/MA/dm.php?keyword=135});欲處/隨欲(\ccchref{DA.17}{https://agama.buddhason.org/DA/dm.php?keyword=17})」},南傳作\NoteKeywordNikaya{「意欲的非去處」}(Chandāgatiṃ),Maurice Walshe先生英譯為「從執著湧出」(springs from attachment, \ccchref{DN.31}{https://agama.buddhason.org/DN/dm.php?keyword=31}),John Kelly, Sue Sawyer, and Victoria Yareham英譯為「經由想要的歧途」(astray through desire, \ccchref{DN.31}{https://agama.buddhason.org/DN/dm.php?keyword=31}),坦尼沙羅比丘長老英譯為「循基於想要的偏途」(to follow a bias based on desire, \ccchref{AN.9.7}{https://agama.buddhason.org/AN/an.php?keyword=9.7}),菩提比丘長老英譯為\NoteKeywordBhikkhuBodhi{「由於想要而錯誤的路程」}(a wrong course on account of desire, \ccchref{AN.2.47}{https://agama.buddhason.org/AN/an.php?keyword=2.47})。按:「非去處」(agati),另譯為「非道」,《吉祥悅意》等說,以欲、以情愛(chandena pemena, \ccchref{DN.31}{https://agama.buddhason.org/DN/dm.php?keyword=31})到非去處做不應該作的(akattabbaṃ karonto/akattabbaṃ karoti, \ccchref{DN.31}{https://agama.buddhason.org/DN/dm.php?keyword=31}/\ccchref{AN.4.17}{https://agama.buddhason.org/AN/an.php?keyword=4.17})。
\stopitemgroup

\startitemgroup[noteitems]
\item\subnoteref{424.0}\NoteKeywordNikayaHead{「明淨、住立」}(pasīdati santiṭṭhati,皆為動詞),菩提比丘長老英譯為\NoteKeywordBhikkhuBodhi{「獲得自信,堅固」}(acquires confidence, steadiness)。
\stopitemgroup

\startitemgroup[noteitems]
\item\subnoteref{425.0}\NoteKeywordAgamaHead{「表現微相/以神通力/以神力/如其正受(SA),如其像定/如其像作如意足(MA);以神力(DA)」},南傳作\NoteKeywordNikaya{「造作像那樣的神通作為」}(tathārūpaṃ iddhābhisaṅkhāraṃ abhisaṅkhāsiṃ),菩提比丘長老英譯為\NoteKeywordBhikkhuBodhi{「實行這樣的一個超常力量的極好技術」}(performed such a feat of spiritual power)。
\stopitemgroup

\startitemgroup[noteitems]
\item\subnoteref{426.0}\NoteKeywordNikayaHead{「隱密行為者」}(paṭicchannakammanto),菩提比丘長老英譯為\NoteKeywordBhikkhuBodhi{「他的行為詭異(隱隱藏藏)」}(secretive in his acts)。按:《滿足希求》以「屬於惡業的」(pāpakammassa, \ccchref{AN.2.27}{https://agama.buddhason.org/AN/an.php?keyword=2.27})解說。
\stopitemgroup

\startitemgroup[noteitems]
\item\subnoteref{427.0}\NoteKeywordAgamaHead{「觀察生滅/諦觀法生滅(SA);作生滅相(GA);觀興衰法(MA);觀法生滅(DA)」},南傳作\NoteSubEntryKey{(i)}\NoteKeywordNikaya{「隨看著生滅」}(udayabbayānupassī),菩提比丘長老英譯為\NoteKeywordBhikkhuBodhi{「凝視起落」}(contemplating rise and fall, SN),或「凝視出現與消失」(contemplating arising and vanishing, AN)。按:這裡的「滅;衰滅」原文為 vaya=baya 而非 nirodha。\NoteSubEntryKey{(ii)}\NoteKeywordNikaya{「隨看著消散」}(vayañcassānupassati),菩提比丘長老英譯為\NoteKeywordBhikkhuBodhi{「觀察其消散」}(he observes its vanishing, AN)。按:《滿足希求》說,他看見他這個心的生起與消散(tassa cesa cittassa uppādampi vayampi passati, \ccchref{AN.6.55}{https://agama.buddhason.org/AN/an.php?keyword=6.55})。\NoteSubEntryKey{(iii)}\NoteKeywordNikaya{「導向生起與滅沒[之慧]」}(udayatthagāminiyā [paññāya]),菩提比丘長老英譯為\NoteKeywordBhikkhuBodhi{「識別出現與消失之智慧」}(the wisdom that discerns arising and passing away, AN)。
\stopitemgroup

\startitemgroup[noteitems]
\item\subnoteref{428.0}\NoteKeywordNikayaHead{「喜」}(Pīti),菩提比丘長老英譯為\NoteKeywordBhikkhuBodhi{「狂喜」}(rapture)。而「喜悅」(somanassa),菩提比丘長老英譯為\NoteKeywordBhikkhuBodhi{「喜悅」}(joy)。
\stopitemgroup

\startitemgroup[noteitems]
\item\subnoteref{429.0}\NoteKeywordAgamaHead{「有愛使(SA/DA);有使(MA);欲世間使(AA)」},南傳作\NoteKeywordNikaya{「有貪煩惱潛在趨勢」}(bhavarāgānusayānaṃ),菩提比丘長老英譯為\NoteKeywordBhikkhuBodhi{「對存在的慾望之表面下的趨勢」}(the underlying tendency to lust for existence, AN),智髻比丘長老英譯為「對存在的想要之在表面下的趨勢」(the underlying tendency to desire for being, MN),Maurice Walshe先生英譯為「成為的渴望之潛在傾向」(latent proclivity of craving for becoming, DN)。
\stopitemgroup

\startitemgroup[noteitems]
\item\subnoteref{430.0}\NoteSubKeyHead{(1)}\NoteKeywordAgamaHead{「金(SA);生色(MA);金/黃金(DA);碎金/金(AA)」},南傳作\NoteKeywordNikaya{「黃金」}(suvaṇṇo),菩提比丘長老英譯為\NoteKeywordBhikkhuBodhi{「黃金」}(gold)。按:《摩訶僧祇律》說︰「生色者是金也,似色者是銀,生色、似色者,錢等市用物。」
\item\subnoteref{430.1}\NoteSubKeyHead{(2)}\NoteKeywordAgamaHead{「像寶(MA)」},南傳作\NoteKeywordNikaya{「珠寶」}(maṇi,另譯為「摩尼;寶珠」),菩提比丘長老英譯為\NoteKeywordBhikkhuBodhi{「珠寶」}(gems)。
\stopitemgroup

\startitemgroup[noteitems]
\item\subnoteref{431.0}\NoteKeywordNikayaHead{「另一邊的」}(parato,另譯為「其他的,對面的」),菩提比丘長老英譯為\NoteKeywordBhikkhuBodhi{「如外人;如外國人」}(as alien)。按:《顯揚真義》等以「非自己的之義」(asakaṭṭhena, \suttaref{SN.22.122}, \ccchref{MN.64}{https://agama.buddhason.org/MN/dm.php?keyword=64}, \ccchref{AN.9.36}{https://agama.buddhason.org/AN/an.php?keyword=9.36}),或「無常」(aniccato, \suttaref{SN.8.4})解說,《滿足希求》以「不應順從的之義」(avidheyyaṭṭhena, \ccchref{AN.4.124}{https://agama.buddhason.org/AN/an.php?keyword=4.124})解說。
\stopitemgroup

\startitemgroup[noteitems]
\item\subnoteref{432.0}\NoteKeywordNikayaHead{「(在)葉屋」}(paṇṇakuṭiyā),Maurice Walshe先生英譯為「在樹葉(蓋)的小屋中」(in a leaf-hut)。
\stopitemgroup

\startitemgroup[noteitems]
\item\subnoteref{433.0}\NoteKeywordNikayaHead{「智見;智與見」}(ñāṇadassanaṃ, Ñāṇañca…dassanaṃ),菩提比丘長老英譯為\NoteKeywordBhikkhuBodhi{「理解與眼光」}(the knowledge and vision)。按:此處的「見」(dassanaṃ)為名詞「看見」(動詞dassati),指義理的看見(認識到;領悟到),《顯揚真義》以「因這三轉十二行相而生起的智見(vasena uppannañāṇasaṅkhātaṃ dassanaṃ, \suttaref{SN.56.1})」,《滿足希求》以「道的智見」(maggañāṇasaṅkhātāya dassanāya, \ccchref{AN.4.196}{https://agama.buddhason.org/AN/an.php?keyword=4.196}),或以「已生的天眼」(dibbacakkhubhūtaṃ ñāṇasaṅkhātaṃ dassanaṃ, \ccchref{AN.8.64}{https://agama.buddhason.org/AN/an.php?keyword=8.64})解說,《破斥猶豫》說,在此經中指提婆達多的五神通,已住立於天眼與五神通的頂端(dibbacakkhu ca pañcannaṃ abhiññānaṃ matthake ṭhitaṃ, \ccchref{MN.29}{https://agama.buddhason.org/MN/dm.php?keyword=29}),而\ccchref{MA.72}{https://agama.buddhason.org/MA/dm.php?keyword=72}則指入禪定。
\stopitemgroup

\startitemgroup[noteitems]
\item\subnoteref{434.0}\NoteKeywordAgamaHead{「如是正心解脫(MA)」},南傳作\NoteKeywordNikaya{「心這樣完全地解脫」}(Evaṃ sammā vimuttacittassa,逐字譯為「如是正心解脫」),菩提比丘長老英譯為\NoteKeywordBhikkhuBodhi{「在心中這樣純然無瑕地釋放」}(thus perfectly liberated in mind)。
\stopitemgroup

\startitemgroup[noteitems]
\item\subnoteref{435.0}\NoteSubKeyHead{(1)}\NoteKeywordAgamaHead{「貪(SA/DA);貪伺/增伺(MA);貪樂(AA)」},南傳作\NoteKeywordNikaya{「貪婪」}(abhijjhā),菩提比丘長老英譯為\NoteKeywordBhikkhuBodhi{「貪婪的;貪心的」}(covetous/covetousness, SN/MN),或「熱望」(longing, AN)。
\item\subnoteref{435.1}\NoteSubKeyHead{(2)}\NoteKeywordNikayaHead{「貪婪者」}(abhijjhālu,另譯為「有貪的;貪欲的;貪求的;貪愛的」),菩提比丘長老英譯為\NoteKeywordBhikkhuBodhi{「貪婪的;貪心的」}(covetous),或「充滿熱望」(full of longing, AN)。按:《顯揚真義》以「對他人的物品有貪求的習慣」(parabhaṇḍe lubbhanasīlāti, \suttaref{SN.14.27})解說。
\stopitemgroup

\startitemgroup[noteitems]
\item\subnoteref{436.0}\NoteKeywordAgamaHead{「著/繫著(SA);疲勞/恐怖(MA);恐怖/恐懼(AA)」},南傳作\NoteKeywordNikaya{「戰慄」}(paritassati, paritassana),Maurice Walshe先生英譯為「被激起」(being excited, DN),菩提比丘長老英譯為\NoteKeywordBhikkhuBodhi{「被擾動」}(agitated, \suttaref{SN.22.43}),並說明「戰慄」(paritassati)的動詞語基「tasati」是害怕(to fear)、戰慄(to tremble),《顯揚真義》以「渴愛的戰慄」(taṇhāparitassanāya, \suttaref{SN.22.43}),《破斥猶豫》以「恐懼的戰慄、渴愛的戰慄」(bhayaparitassanāya taṇhāparitassanāya vā, \ccchref{MN.22}{https://agama.buddhason.org/MN/dm.php?keyword=22}),《吉祥悅意》以「渴愛、[邪]見、慢的戰慄」(taṇhādiṭṭhimānaparitassanāyāpi, \ccchref{DN.15}{https://agama.buddhason.org/DN/dm.php?keyword=15})解說。
\stopitemgroup

\startitemgroup[noteitems]
\item\subnoteref{437.0}\NoteKeywordAgamaHead{「漏盡智(SA/MA/DA)」},南傳作\NoteKeywordNikaya{「諸漏的滅盡之智」}(āsavānaṃ khayañāṇāya),智慧髮髻比丘長老英譯為「污染破壞的理解」(knowledge of the destruction of the taints)。按:《顯揚真義》說,當被稱為諸漏已盡,得到阿羅漢果,存在著省察智(paccavekkhaṇañāṇaṃ),先(paṭhamavāraṃ)有被稱已盡的阿羅漢果生起,再有滅盡智生起的狀態。依此,則與「解脫知見」相當。《破斥猶豫》等說,[此即]阿羅漢道智之意,因為破壞諸漏的阿羅漢道破壞諸漏,在那裡,這個智在那時被完成的狀態(pariyāpannattāti, \ccchref{MN.4}{https://agama.buddhason.org/MN/dm.php?keyword=4}/\ccchref{AN.3.59}{https://agama.buddhason.org/AN/an.php?keyword=3.59})。又,「盡智」應為「滅盡智」的另譯。
\stopitemgroup

\startitemgroup[noteitems]
\item\subnoteref{438.0}\NoteSubKeyHead{(1)}\NoteKeywordAgamaHead{「差降(MA)」},南傳作\NoteKeywordNikaya{「差別;特質」}(viseso,另譯為「殊勝;卓越」),菩提比丘長老英譯為\NoteKeywordBhikkhuBodhi{「區別(特性)」}(distinction)。
\item\subnoteref{438.1}\NoteSubKeyHead{(2)}\NoteKeywordAgamaHead{「差降安樂住止(MA)」},解讀為「安樂的等至住處之特質」(住止≈等至住處-vihārasamāpattīnaṃ)。
\stopitemgroup

\startitemgroup[noteitems]
\item\subnoteref{439.0}\NoteKeywordAgamaHead{「剃頭沙門(SA);剃髮道人(GA);禿頭沙門/禿沙門(MA)」},南傳作\NoteKeywordNikaya{「禿頭假沙門」}(muṇḍakā samaṇakā, muṇḍake samaṇake,另譯為「禿頭的似非沙門;糞屎僧」),菩提比丘長老英譯為\NoteKeywordBhikkhuBodhi{「禁欲修道者」}(ascetics),或「剃光頭髮的禁欲修道者」(shaveling ascetics, \suttaref{SN.7.22}),Maurice Walshe先生英譯為「無用的剃光頭髮的禁欲修道者」(shaveling petty ascetics, \ccchref{DN.27}{https://agama.buddhason.org/DN/dm.php?keyword=27})。
\stopitemgroup

\startitemgroup[noteitems]
\item\subnoteref{440.0}\NoteKeywordNikayaHead{「說次第說」}(anupubbiṃ kathaṃ kathesi,另譯為「說次第論」),菩提比丘長老英譯為\NoteKeywordBhikkhuBodhi{「漸進地講說」}(a progressive discourse)。
\stopitemgroup

\startitemgroup[noteitems]
\item\subnoteref{441.0}\NoteKeywordAgamaHead{「正到、正趣/善到、善向(SA);善去、善向(MA);平等行(DA);等成就(AA)」},南傳作\NoteKeywordNikaya{「正行的、正行道的」}(sammaggatā sammāpaṭipannā,另譯為「正行者;依正確而行者(的)」),菩提比丘長老英譯為\NoteKeywordBhikkhuBodhi{「正確地過活與實行」}(faring and practising rightly)。
\stopitemgroup

\startitemgroup[noteitems]
\item\subnoteref{442.0}\NoteKeywordAgamaHead{「畜生之論/畜生論/鳥論(MA);遮道濁亂之言(DA)」},南傳作\NoteKeywordNikaya{「畜生論」}(tiracchānakathaṃ,意譯為「無意義的談論;無用的談論」),菩提比丘長老英譯為\NoteKeywordBhikkhuBodhi{「無意義的談論」}(pointless talk)。按:《顯揚真義》等說,這是不出離的狀態(aniyyānikattā),成為天界與解脫道路傍行(saggamokkhamaggānaṃ tiracchānabhūtaṃ, \suttaref{SN.56.10}/\ccchref{MN.76}{https://agama.buddhason.org/MN/dm.php?keyword=76}/\ccchref{DN.1}{https://agama.buddhason.org/DN/dm.php?keyword=1}/\ccchref{AN.10.69}{https://agama.buddhason.org/AN/an.php?keyword=10.69})的談論。
\stopitemgroup

\startitemgroup[noteitems]
\item\subnoteref{443.0}\NoteKeywordAgamaHead{「四種增上心法現法樂住/四增心法正受現法安樂住/四增心見法安樂住(SA);四增上心現法樂居(MA)」},南傳作\NoteSubEntryKey{(i)}\NoteKeywordNikaya{「增上心、當生樂住處之四禪的」}(catunnañca jhānānaṃ ābhicetasikānaṃ diṭṭhadhammasukhavihārānaṃ, \ccchref{AN.5.166}{https://agama.buddhason.org/AN/an.php?keyword=5.166}),菩提比丘長老英譯為\NoteKeywordBhikkhuBodhi{「構成較高之心與就在這一生中是快樂住處的四種禪」}(the four jhānas that constitute the higher mind and are pleasant dwellings in this very life)。\NoteSubEntryKey{(ii)}\NoteKeywordNikaya{「四個增上心、當生樂住處的」}(catunnaṃ ābhicetasikānaṃ diṭṭhadhammasukhavihārānaṃ, \ccchref{AN.5.179}{https://agama.buddhason.org/AN/an.php?keyword=5.179}),菩提比丘長老英譯為\NoteKeywordBhikkhuBodhi{「關於較高之心的四個看得見的快樂住處」}(four pleasant visible dwellings that pertain to the higher mind)。按:後者於\ccchref{AN.5.179}{https://agama.buddhason.org/AN/an.php?keyword=5.179}中的內容,為證得初果的「四不壞淨」,而非四禪。
\stopitemgroup

\startitemgroup[noteitems]
\item\subnoteref{444.0}\NoteKeywordAgamaHead{「我取(SA);我受(MA/DA/AA)」},南傳作\NoteKeywordNikaya{「[真]我論取」}(attavādupādānaṃ),菩提比丘長老英譯為\NoteKeywordBhikkhuBodhi{「對自我之教義的執著」}(clinging to a doctrine of self)。按:MA的「受」即「取」(upādā),《顯揚真義》等以「[真]我的理論之執取」(Attano vādupādānaṃ, \suttaref{SN.12.2}/\ccchref{MN.9}{https://agama.buddhason.org/MN/dm.php?keyword=9}),《吉祥悅意》解說為「以這個說『真我』且執取」(Attāti etena vadati ceva upādiyati cāti, \ccchref{DN.33}{https://agama.buddhason.org/DN/dm.php?keyword=33})。
\stopitemgroup

\startitemgroup[noteitems]
\item\subnoteref{445.0}\NoteSubKeyHead{(1)}\NoteKeywordNikayaHead{「命根」}(jīvitindriya),菩提比丘長老英譯為\NoteKeywordBhikkhuBodhi{「生命的器官機能」}(life faculty)。按:《顯揚真義》以「對生命行使主權」(Jīvite indaṭṭhaṃ karotīti, \suttaref{SN.48.22})解說。
\item\subnoteref{445.1}\NoteSubKeyHead{(2)}\NoteKeywordAgamaHead{「命根閉塞(MA)」},南傳作\NoteKeywordNikaya{「命根斷絕」}(jīvitindriyassupacchedo),坦尼沙羅比丘長老英譯為「生命的器官機能的中斷」(interruption in the life faculty),緬甸三藏協會英譯為「生命的器官機能的破壞」(the destruction of the life-faculty),菩提比丘長老英譯缺。
\stopitemgroup

\startitemgroup[noteitems]
\item\subnoteref{446.0}\NoteKeywordAgamaHead{「梵行本(MA);梵行初/梵行初首(DA)」},南傳作\NoteKeywordNikaya{「梵行的基礎」}(ādibrahmacariyakoti, ādibrahmacariyaṃ,逐字譯為「最初+梵+行」),智髻比丘長老英譯為「它屬於聖潔生活的基礎」(it belongs to the fundamentals of the holy life, \ccchref{MN.133}{https://agama.buddhason.org/MN/dm.php?keyword=133}),Maurice Walshe先生英譯為「聖潔生活的完成」(the perfection of the holy life, \ccchref{DN.25}{https://agama.buddhason.org/DN/dm.php?keyword=25})。
\stopitemgroup

\startitemgroup[noteitems]
\item\subnoteref{447.0}\NoteKeywordAgamaHead{「異見(MA);見異(DA)」},南傳作\NoteKeywordNikaya{「(以)不同見解的」}(aññadiṭṭhikena,另譯為「其它的見」),菩提比丘長老英譯為\NoteKeywordBhikkhuBodhi{「屬於另外見解者」}(who are of another view)。
\stopitemgroup

\startitemgroup[noteitems]
\item\subnoteref{448.0}\NoteKeywordAgamaHead{「異忍(MA);忍異(DA)」},南傳作\NoteKeywordNikaya{「(以)不同信仰的」}(aññakhantikena,另譯為「其它的信忍」),菩提比丘長老英譯為\NoteKeywordBhikkhuBodhi{「接受另外的教導者」}(who are accept another teaching)。
\stopitemgroup

\startitemgroup[noteitems]
\item\subnoteref{449.0}\NoteKeywordAgamaHead{「異樂(MA);異受(DA)」},南傳作\NoteKeywordNikaya{「(以)不同喜好的」}(aññarucikena,另譯為「其它的喜好」),菩提比丘長老英譯為\NoteKeywordBhikkhuBodhi{「認可另外的教導者」}(who approve of another teaching)。
\stopitemgroup

\startitemgroup[noteitems]
\item\subnoteref{450.0}\NoteKeywordAgamaHead{「行異/異習(DA)」},南傳作\NoteKeywordNikaya{「(以)在他處修行的」}(aññatrayogena,另譯為「異修行」),菩提比丘長老英譯為\NoteKeywordBhikkhuBodhi{「追求不同訓練者」}(who pursue a differenct training)。
\stopitemgroup

\startitemgroup[noteitems]
\item\subnoteref{451.0}\NoteKeywordAgamaHead{「依異法(DA)」},南傳作\NoteKeywordNikaya{「(以)在他處老師的」}(aññatrācariyakena,另譯為「異師」),菩提比丘長老英譯為\NoteKeywordBhikkhuBodhi{「跟隨別的老師者」}(who follow a differenct teacher)。
\stopitemgroup

\startitemgroup[noteitems]
\item\subnoteref{452.0}\NoteKeywordAgamaHead{「貪使/貪欲使(SA);欲使(MA);欲愛使(DA);貪欲之使/欲愛使/貪欲使(AA)」},南傳作\NoteSubEntryKey{(i)}\NoteKeywordNikaya{「欲貪煩惱潛在趨勢」}(kāmarāgānusayo),菩提比丘長老英譯為\NoteKeywordBhikkhuBodhi{「感官慾望之表面下的趨勢」}(the underlying tendency to sensual lust),或\NoteSubEntryKey{(ii)}\NoteKeywordNikaya{「貪煩惱潛在趨勢」}(rāgānusayānaṃ),菩提比丘長老英譯為\NoteKeywordBhikkhuBodhi{「慾望之表面下的趨勢」}(the underlying tendency to lust),Maurice Walshe先生英譯為「感官貪欲之潛在傾向」(latent proclivity of sensuous greed)。
\stopitemgroup

\startitemgroup[noteitems]
\item\subnoteref{453.0}\NoteKeywordAgamaHead{「瞋恚使(SA/DA/AA);恚使(MA)」},南傳作\NoteSubEntryKey{(i)}\NoteKeywordNikaya{「嫌惡煩惱潛在趨勢」}(paṭighānusayānaṃ),菩提比丘長老英譯為\NoteKeywordBhikkhuBodhi{「反感之表面下的趨勢」}(the underlying tendency to aversion),Maurice Walshe先生英譯為「怨恨之潛在傾向」(latent proclivity of resentment),或\NoteSubEntryKey{(ii)}\NoteKeywordNikaya{「惡意煩惱潛在趨勢」}(byāpādānusayo),菩提比丘長老英譯為\NoteKeywordBhikkhuBodhi{「惡意之表面下的趨勢」}(the underlying tendency to ill will)。
\stopitemgroup

\startitemgroup[noteitems]
\item\subnoteref{454.0}\NoteKeywordAgamaHead{「癡使(SA/AA);無明使(SA/MA/DA)」},南傳作\NoteKeywordNikaya{「無明煩惱潛在趨勢」}(avijjānusayānaṃ),菩提比丘長老英譯為\NoteKeywordBhikkhuBodhi{「無知之表面下的趨勢」}(the underlying tendency to ignorance),Maurice Walshe先生英譯為「無知之潛在傾向」(latent proclivity of ignorance)。
\stopitemgroup

\startitemgroup[noteitems]
\item\subnoteref{455.0}\NoteKeywordAgamaHead{「實眼(SA);是眼(MA);為世間眼(DA)」},南傳作\NoteKeywordNikaya{「眼已生者」}(cakkhubhūto),菩提比丘長老英譯為\NoteKeywordBhikkhuBodhi{「他已成為看得見」}(he has become vision, AN),智髻比丘長老英譯為「他是看得見者」(he is vision, MN)。按:《顯揚真義》等以「以自己看見義為領導者」(Svāyaṃ dassanapariṇāyakaṭṭhena, \suttaref{SN.35.116}/\ccchref{MN.18}{https://agama.buddhason.org/MN/dm.php?keyword=18})解說。
\stopitemgroup

\startitemgroup[noteitems]
\item\subnoteref{456.0}\NoteKeywordAgamaHead{「實智(SA);是智(MA);為世間智(DA)」},南傳作\NoteKeywordNikaya{「智已生者」}(ñāṇabhūto),菩提比丘長老英譯為\NoteKeywordBhikkhuBodhi{「他已成為理解」}(he has become knowledge, AN),智髻比丘長老英譯為「他是理解者」(he is knowledge, MN)。按:《顯揚真義》等以「以知道所作義」(Viditakaraṇaṭṭhena, \suttaref{SN.35.116}/\ccchref{MN.18}{https://agama.buddhason.org/MN/dm.php?keyword=18})解說。
\stopitemgroup

\startitemgroup[noteitems]
\item\subnoteref{457.0}\NoteKeywordAgamaHead{「實法(SA);是法(MA);為世間法(DA)」},南傳作\NoteKeywordNikaya{「法已生者」}(dhammabhūto),菩提比丘長老英譯為\NoteKeywordBhikkhuBodhi{「他已成為法」}(he has become the Dhamma, AN),智髻比丘長老英譯為「他是正法」( he is the Dhamma, MN)。按:《顯揚真義》等以「以不顛倒自性義,從學得法之轉起,或以心思惟後以言語說出法」(Aviparītasabhāvaṭṭhena pariyattidhammapavattanato vā hadayena cintetvā vācāya nicchāritadhammamayoti, \suttaref{SN.35.116}/\ccchref{MN.18}{https://agama.buddhason.org/MN/dm.php?keyword=18})解說。
\stopitemgroup

\startitemgroup[noteitems]
\item\subnoteref{458.0}\NoteKeywordAgamaHead{「梵有(MA);為世間梵(DA)」},南傳作\NoteKeywordNikaya{「梵已生者」}(brahmabhūto),菩提比丘長老英譯為\NoteKeywordBhikkhuBodhi{「他已成為聖潔者」}(he has become the holy one, SN),或「他已成為梵」(he has become Brahmā, AN),智髻比丘長老英譯為「他是聖潔者」(he is the holy one, MN)。按:《顯揚真義》等以「以最上義」(Seṭṭhaṭṭhena, \suttaref{SN.35.116}/\ccchref{MN.18}{https://agama.buddhason.org/MN/dm.php?keyword=18}),《滿足希求》以「最上實相」(seṭṭhasabhāvo, \ccchref{AN.10.115}{https://agama.buddhason.org/AN/an.php?keyword=10.115})解說。
\stopitemgroup

\startitemgroup[noteitems]
\item\subnoteref{459.0}\NoteKeywordAgamaHead{「無蓋心(MA);陰蓋輕微(DA)」},南傳作\NoteKeywordNikaya{「離蓋心」}(vinīvaraṇacittaṃ),菩提比丘長老英譯為\NoteKeywordBhikkhuBodhi{「擺脫障礙的心」}(a mind rid of hindrances)。
\stopitemgroup

\startitemgroup[noteitems]
\item\subnoteref{460.0}\NoteKeywordAgamaHead{「身行(MA)」},南傳作\NoteKeywordNikaya{「身行為;身正行」}(kāyasamācāraṃ,另譯為「身行;身行事;身儀法」),菩提比丘長老英譯為\NoteKeywordBhikkhuBodhi{「身體的行為」}(bodily conduct, bodily behaviour)。
\stopitemgroup

\startitemgroup[noteitems]
\item\subnoteref{461.0}\NoteKeywordNikayaHead{「全見者」}(aññadatthudaso,另譯為「普見者」),菩提比丘長老英譯為\NoteKeywordBhikkhuBodhi{「全見者」}(The Universal Seer, AN/\ccchref{DN.1}{https://agama.buddhason.org/DN/dm.php?keyword=1}),智髻比丘長老英譯為「絕對無誤的眼光者」(of Infallible Vision, MN)。
\stopitemgroup

\startitemgroup[noteitems]
\item\subnoteref{462.0}\NoteKeywordNikayaHead{「自在的;自在者」}(vasavattī),菩提比丘長老英譯為\NoteKeywordBhikkhuBodhi{「權力的支配者」}(the wielder of power)。
\stopitemgroup

\startitemgroup[noteitems]
\item\subnoteref{463.0}\NoteKeywordAgamaHead{「安置丹枕(SA);兩頭安枕(MA)」},南傳作\NoteKeywordNikaya{「兩端有紅色枕墊」}(ubhatolohitakūpadhānāni),菩提比丘長老英譯為\NoteKeywordBhikkhuBodhi{「兩端有紅色墊子」}(red cushions at both ends)。按:「丹」即「紅色」(lohita),「枕」即「枕墊」(kūpadhāna),為「帆柱」(kūpa,另譯作「坑,井,洞」)與「容器」(dhāna)的複合詞。
\stopitemgroup

\startitemgroup[noteitems]
\item\subnoteref{464.0}\NoteKeywordAgamaHead{「拘沾婆衣(SA);加陵伽波惒邏衣(MA)」},南傳作\NoteKeywordNikaya{「精緻的毛衣」}(kambalasukhumāni),菩提比丘長老英譯為\NoteKeywordBhikkhuBodhi{「精緻的毛衣」}(fine wool)。
\stopitemgroup

\startitemgroup[noteitems]
\item\subnoteref{465.0}\NoteKeywordAgamaHead{「迦陵伽(SA);加陵伽波惒邏波遮悉多羅那(MA)」},南傳作\NoteKeywordNikaya{「頂級羚鹿皮覆蓋的」}(kadalimigapavarapaccattharaṇāni),菩提比丘長老英譯為\NoteKeywordBhikkhuBodhi{「以挑選過羚羊皮作的床單」}(with choice spreads made of antelope hides)。按「加陵伽」應為「羚鹿」(kadali)的音譯,「波惒邏」應為「頂級」(pavara)的音譯,「波遮悉多羅那」應為「覆蓋的;鋪墊物;覆蓋布」(paccattharaṇa)的音譯。
\stopitemgroup

\startitemgroup[noteitems]
\item\subnoteref{466.0}\NoteKeywordAgamaHead{「芻摩衣(SA);蒭磨衣/初摩衣(MA/DA)」},南傳作\NoteKeywordNikaya{「精緻的亞麻衣」}(khomasukhumāni),菩提比丘長老英譯為\NoteKeywordBhikkhuBodhi{「精緻的亞麻品」}(fine linen)。按:「初摩;芻摩」即「亞麻;亞麻衣」(khoma)的音譯。
\stopitemgroup

\startitemgroup[noteitems]
\item\subnoteref{467.0}\NoteKeywordAgamaHead{「受持八支齋/受八支齋(SA);撿情守戒/受持八戒(GA);聖八支齋(MA);持八關齋/八關齋法(AA)」},南傳作\NoteSubEntryKey{(i)}\NoteKeywordNikaya{「善具備八支」}(aṭṭhaṅgasusamāgataṃ),或\NoteSubEntryKey{(ii)}\NoteKeywordNikaya{「具備八支」}(aṭṭhaṅgasamannāgatassa, \ccchref{AN.3.71}{https://agama.buddhason.org/AN/an.php?keyword=3.71}),菩提比丘長老英譯為\NoteKeywordBhikkhuBodhi{「完成八要素」}(Complete in eight factors),並說,此即「八戒」:不①殺②盜③淫④妄⑤酒⑥過午不食⑦歌舞、聽音樂、看不適當的秀、化妝戴首飾⑧使用高且豪華的床和座椅。按:西晉-康僧會譯為「八關齋」,隋-那連提黎耶舍譯為「八關齋戒」。宋-釋元照說,言關齋者謂禁閉非逸,靜定身心也。明-智旭說,以八戒及齋關閉情欲。
\stopitemgroup

\startitemgroup[noteitems]
\item\subnoteref{468.0}\NoteKeywordAgamaHead{「關閉根門(SA);收攝諸根(GA);守護諸根/護諸根者(MA);制諸入(DA);諸根不亂(AA)」},南傳作\NoteKeywordNikaya{「在諸根上守護門的(者)」}(indriyesu guttadvāro, indriyesu guttadvārena),菩提比丘長老英譯為\NoteKeywordBhikkhuBodhi{「防護感覺門戶」}(guarding the sense doors, SA),或「防護他感官機能的門」(guard the doors of his sense faculties, MA)。
\stopitemgroup

\startitemgroup[noteitems]
\item\subnoteref{469.0}\NoteKeywordAgamaHead{「於貪伺淨除其心(MA)」},南傳作\NoteKeywordNikaya{「使心從貪婪淨化」}(abhijjhāya cittaṃ parisodheti),菩提比丘長老英譯為\NoteKeywordBhikkhuBodhi{「他從貪婪淨化他的心」}(he purifies his mind from covetousness)。
\stopitemgroup

\startitemgroup[noteitems]
\item\subnoteref{470.0}\NoteSubKeyHead{(1)}\NoteKeywordNikayaHead{「活的生命類」}(pāṇabhūta, pāṇabhūtesu),菩提比丘長老英譯為\NoteKeywordBhikkhuBodhi{「活的生命」}(living beings)。
\item\subnoteref{470.1}\NoteSubKeyHead{(2)}\NoteKeywordNikayaHead{「對活的生命類不同情者」}(adayāpanno pāṇabhūtesu),菩提比丘長老英譯為\NoteKeywordBhikkhuBodhi{「對活的生命無慈悲的」}(merciless to living beings)。
\item\subnoteref{470.2}\NoteSubKeyHead{(3)}\NoteKeywordNikayaHead{「對一切活的生命類有憐愍者」}(sabbapāṇabhūtahitānukampī),菩提比丘長老英譯為\NoteKeywordBhikkhuBodhi{「對所有活的生命慈悲的」}(compassionate towards all living beings, \suttaref{SN.42.7}),智髻比丘長老英譯為「對所有活的生命的幸福慈悲的」(compassionate for the welfare of all living beings, \ccchref{MN.107}{https://agama.buddhason.org/MN/dm.php?keyword=107})。
\stopitemgroup

\startitemgroup[noteitems]
\item\subnoteref{471.0}\NoteKeywordAgamaHead{「火母/火鑽(MA);鑽鑽木(DA)」},南傳作\NoteKeywordNikaya{「取火的上鑽木」}(uttarāraṇiṃ, araṇisahitaṃ,另譯為「上面的鑽木」),Maurice Walshe先生英譯為「火-棒」(fire-sticks),菩提比丘長老英譯為\NoteKeywordBhikkhuBodhi{「上火-棒」}(upper fire-stick)。按:取火的鑽木(araṇi),即鑽木取火時,以手搓動的那根木棒。
\stopitemgroup

\startitemgroup[noteitems]
\item\subnoteref{472.0}\NoteKeywordAgamaHead{「首陀/首陀羅(SA/GA/DA/AA);工師(MA)」},南傳作\NoteKeywordNikaya{「首陀羅」}(suddo,另譯為「奴隸族」),智髻比丘長老英譯為「工人」(a worker, MN),菩提比丘長老英譯照錄原文suddas(SN, AN)。
\stopitemgroup

\startitemgroup[noteitems]
\item\subnoteref{473.0}\NoteKeywordAgamaHead{「究竟知見/勝究竟知見/勝妙知見(SA);差降聖知聖見(MA)」},南傳作\NoteKeywordNikaya{「足以為聖者智見特質」}(alamariyañāṇadassanaviseso),菩提比丘長老英譯為\NoteKeywordBhikkhuBodhi{「在理解與眼光上足以為高潔者之區別(特性)」}(distinction in knowledge and vision worthy of the noble ones)。按:「差降」,相當於「差別;特質」(viseso),「見」(dassanaṃ, 名詞,動詞dassati),指義理的看見(認識到;領悟到)。《顯揚真義》等以「聖性能作的殊勝智(特質之智)」(ariyabhāvakaraṇasamattho ñāṇaviseso, \suttaref{SN.41.9}/\ccchref{MN.31}{https://agama.buddhason.org/MN/dm.php?keyword=31},\ccchref{AN.1.45}{https://agama.buddhason.org/AN/an.php?keyword=1.45}近似)解說。
\stopitemgroup

\startitemgroup[noteitems]
\item\subnoteref{474.0}\NoteKeywordAgamaHead{「得如來義(SA);得義(GA)」},南傳作\NoteKeywordNikaya{「得義的信受」}(labhati atthavedaṃ),菩提比丘長老英譯為\NoteKeywordBhikkhuBodhi{「獲得意義的啟發」}(gains the inspiration of the meaning)。按:「義的信受」,《滿足希求》以「依談論的義理生起喜、欣悅」(aṭṭhakathaṃ nissāya uppannaṃ pītipāmojjaṃ, \ccchref{AN.6.10}{https://agama.buddhason.org/AN/an.php?keyword=6.10}),《破斥猶豫》以「從省察而生起的不壞淨」(aveccappasādaṃ paccavekkhato uppannaṃ, \ccchref{MN.7}{https://agama.buddhason.org/MN/dm.php?keyword=7})解說。「信受」(vedaṃ),另譯為「宗教情操,知;智,吠陀」。
\stopitemgroup

\startitemgroup[noteitems]
\item\subnoteref{475.0}\NoteKeywordAgamaHead{「得如來正法(SA);得法(GA)」},南傳作\NoteKeywordNikaya{「得法的信受」}(labhati dhammavedaṃ),菩提比丘長老英譯為\NoteKeywordBhikkhuBodhi{「獲得法的啟發」}(gains the inspiration of the Dhamma)。按:「法的信受」,《滿足希求》以「依聖典生起喜、欣悅」(pāḷiṃ nissāya uppannaṃ pītipāmojjaṃ, \ccchref{AN.6.10}{https://agama.buddhason.org/AN/an.php?keyword=6.10}),《破斥猶豫》以「從省察污染捨斷而生起不壞淨的特定因」(aveccappasādassa hetuṃ odhiso kilesappahānaṃ paccavekkhato uppannaṃ, \ccchref{MN.7}{https://agama.buddhason.org/MN/dm.php?keyword=7})解說。「信受」(vedaṃ),另譯為「宗教情操,知;智,吠陀」。
\stopitemgroup

\startitemgroup[noteitems]
\item\subnoteref{476.0}\NoteKeywordAgamaHead{「毘舍(SA);居士(GA/MA/DA/AA);鞞舍(SA/摩訶僧祇律)」},南傳作\NoteKeywordNikaya{「毘舍」}(vesso,另譯為「平民;庶民族(農、商等)」),菩提比丘長老英譯為\NoteKeywordBhikkhuBodhi{「商人」}(a merchant, MA),或照錄原文(vessa, SN/AN)。
\stopitemgroup

\startitemgroup[noteitems]
\item\subnoteref{477.0}\NoteKeywordNikayaHead{「恭敬行為」}(sāmīcikammaṃ,另譯為「和敬業,適當行為」),菩提比丘長老英譯為\NoteKeywordBhikkhuBodhi{「有禮貌地服侍」}(polite services),或「有禮貌地舉止」(behave courteously, \ccchref{AN.8.51}{https://agama.buddhason.org/AN/an.php?keyword=8.51})。按:《吉祥悅意》以「適當行為的所作」(Anucchavikakammassa pana karaṇaṃ, \ccchref{MN.142}{https://agama.buddhason.org/MN/dm.php?keyword=142}),《吉祥悅意》以「每一個本分所作等的適當行為」(taṃtaṃvattakaraṇādi anucchavikakammaṃ, \ccchref{DN.27}{https://agama.buddhason.org/DN/dm.php?keyword=27})解說。「恭敬」(sāmīci, sāmīcī),另譯為「如法;方正;和敬;友好待遇」。
\stopitemgroup

\startitemgroup[noteitems]
\item\subnoteref{478.0}\NoteKeywordAgamaHead{「坐高標下/標下(MA)」},南傳作\NoteKeywordNikaya{「斬首台;屠宰場」}(āghātane, āghātanaṃ,另譯為「刑場」),菩提比丘長老英譯為\NoteKeywordBhikkhuBodhi{「屠宰場」}(the slaughterhouse, \ccchref{AN.7.74}{https://agama.buddhason.org/AN/an.php?keyword=7.74})。
\stopitemgroup

\startitemgroup[noteitems]
\item\subnoteref{479.0}\NoteKeywordNikayaHead{「全心注意後」}(sabbaṃ cetaso samannāharitvā),菩提比丘長老英譯為\NoteKeywordBhikkhuBodhi{「將整個心導向它」}(directs his whole mind to it, AN),或「對它運用整個心」(applying their whole minds to it, SN),智髻比丘長老英譯為「以他全部的心從事」(engages it with all his mind, MN)。
\stopitemgroup

\startitemgroup[noteitems]
\item\subnoteref{480.0}\NoteKeywordNikayaHead{「尊師!」}(bhadante, bhaddante,呼格,另譯為「大德!」),菩提比丘長老英譯為\NoteKeywordBhikkhuBodhi{「值得尊敬的尊長」}(Venerable sir)。按:《顯揚真義》等說,這是恭敬語,或對大師(老師)的回答(gāravavacanametaṃ, satthuno paṭivacanadānaṃ vā, \suttaref{SN.14.7}/\ccchref{MN.23}{https://agama.buddhason.org/MN/dm.php?keyword=23})。
\stopitemgroup

\startitemgroup[noteitems]
\item\subnoteref{481.0}\NoteKeywordAgamaHead{「憂、喜先已離/憂、喜先斷/前憂喜已滅(SA);喜、憂本已滅(MA);先除憂、喜/先滅憂、喜/憂、喜先滅(DA);先無愁憂/無復憂、喜/先無有憂慼之患(AA)」},南傳作\NoteKeywordNikaya{「以之前喜悅與憂的滅沒」}(pubbeva somanassadomanassānaṃ atthaṅgamā),菩提比丘長老英譯為\NoteKeywordBhikkhuBodhi{「帶著以前喜悅與不快樂的消失」}(with the previous passing away of joy and displeasure, \suttaref{SN.36.31})。
\stopitemgroup

\startitemgroup[noteitems]
\item\subnoteref{482.0}\NoteSubKeyHead{(1)}\NoteKeywordAgamaHead{「自依(SA);自歸己法(MA);當自歸依(DA);當自修行法(AA)」},南傳作\NoteKeywordNikaya{「以自己為歸依」}(attasaraṇā,逐字譯為「自己+歸依」),菩提比丘長老英譯為\NoteKeywordBhikkhuBodhi{「以你們自己為依靠」}(with yourselves as a refuge)。
\item\subnoteref{482.1}\NoteSubKeyHead{(2)}\NoteKeywordAgamaHead{「法依(SA);歸依於法(DA)」},南傳作\NoteKeywordNikaya{「以法為歸依」}(dhammasaraṇā-法+歸依),菩提比丘長老英譯為\NoteKeywordBhikkhuBodhi{「以法為依靠」}(with the Dhamma as a refuge)。
\item\subnoteref{482.2}\NoteSubKeyHead{(3)}\NoteKeywordAgamaHead{「不異依(SA);莫歸餘法(MA);勿他歸依(DA)」},南傳作\NoteKeywordNikaya{「不以其他為歸依」}(anaññasaraṇā-不異+歸依),菩提比丘長老英譯為\NoteKeywordBhikkhuBodhi{「不以其他為依靠」}(with no other refuge)。
\stopitemgroup

\startitemgroup[noteitems]
\item\subnoteref{483.0}\NoteKeywordAgamaHead{「純一滿淨梵行清白(SA);清白顯發梵行(GA);具足清淨顯現梵行(MA);淨修梵行/梵行具足/梵行清淨/開清淨行(DA);具足清淨得修梵行/具足得修梵行/清淨修行梵行(AA)」},南傳作\NoteKeywordNikaya{「說明唯獨圓滿、遍純淨的梵行」}(kevalaparipuṇṇaṃ parisuddhaṃ brahmacariyaṃ abhivadanti/pakāsetha),菩提比丘長老英譯為\NoteKeywordBhikkhuBodhi{「宣告無瑕地完整和純淨的精神生活」}(which proclaim the perfectly complete and pure spiritual life)。按:「具足清淨顯現梵行」,依照巴利經文研判,「顯發(GA);顯現(MA)」應為「說明」(abhivadanti)的對譯,所以應該解讀作「顯發清白[的]梵行;顯現具足清淨[的]梵行」。
\stopitemgroup

\startitemgroup[noteitems]
\item\subnoteref{484.0}\NoteKeywordAgamaHead{「稱量觀察(SA);意所惟觀(MA);思惟觀察(DA);恆念思惟(AA)」},南傳作\NoteKeywordNikaya{「被心隨觀察」}(manasānupekkhitā,另譯為「以心謹慎考慮」),菩提比丘長老英譯為\NoteKeywordBhikkhuBodhi{「以心調查/檢查」}(investigated/examined them with the mind, \ccchref{AN.4.22}{https://agama.buddhason.org/AN/an.php?keyword=4.22}/\ccchref{AN.4.191}{https://agama.buddhason.org/AN/an.php?keyword=4.191}),或「心理調查」(mentally investigated, \ccchref{AN.5.87}{https://agama.buddhason.org/AN/an.php?keyword=5.87}etc. investigated mentally, \ccchref{AN.10.17}{https://agama.buddhason.org/AN/an.php?keyword=10.17} etc.)。按:《破斥猶豫》等以「被心隨觀察」(cittena anupekkhitā, \ccchref{MN.32}{https://agama.buddhason.org/MN/dm.php?keyword=32}/\ccchref{AN.4.22}{https://agama.buddhason.org/AN/an.php?keyword=4.22})解說,今準此譯manasā。
\stopitemgroup

\startitemgroup[noteitems]
\item\subnoteref{485.0}\NoteKeywordAgamaHead{「明見深達(MA);分別法義(DA)」},南傳作\NoteKeywordNikaya{「被見善貫通」}(diṭṭhiyā suppaṭividdhā),菩提比丘長老英譯為\NoteKeywordBhikkhuBodhi{「以見解徹底地洞察」}(penetrated well by view)。按:《破斥猶豫》等以「從道理與原因以慧善貫通眼前作的」(atthato ca kāraṇato ca paññāya suṭṭhu paṭividdhā paccakkhaṃ katā., \ccchref{MN.32}{https://agama.buddhason.org/MN/dm.php?keyword=32}/\ccchref{AN.4.22}{https://agama.buddhason.org/AN/an.php?keyword=4.22})解說
\stopitemgroup

\startitemgroup[noteitems]
\item\subnoteref{486.0}\NoteKeywordAgamaHead{「翫習至千(MA);守持不忘(DA);總持(AA)」},南傳作\NoteKeywordNikaya{「被憶持、被言語累積」}(dhātā, vacasā paricitā」),菩提比丘長老英譯為\NoteKeywordBhikkhuBodhi{「保持在心中,出聲背誦」}(retained in mind, recited verbally)。
\stopitemgroup

\startitemgroup[noteitems]
\item\subnoteref{487.0}\NoteKeywordAgamaHead{「住止(MA)」},南傳作\NoteKeywordNikaya{「等至住處」}(vihārasamāpattiyā, vihārasamāpattīnaṃ),智髻比丘長老英譯為「住處或達成」(abiding or attainment, \ccchref{MN.32}{https://agama.buddhason.org/MN/dm.php?keyword=32}),菩提比丘長老英譯為\NoteKeywordBhikkhuBodhi{「默想的住處與達成」}(meditative dwellings and attainments, \ccchref{AN.6.60}{https://agama.buddhason.org/AN/an.php?keyword=6.60})。「等至」(samāpatti),另譯為「定;正受;入定」,音譯為「三摩鉢底」。
\stopitemgroup

\startitemgroup[noteitems]
\item\subnoteref{488.0}\NoteKeywordNikayaHead{「解脫」}(vimokkhā,另譯為「背捨」),Maurice Walshe先生英譯為「釋放」(liberations),坦尼沙羅比丘長老英譯為「脫離」(emancipations),菩提比丘長老英譯為\NoteKeywordBhikkhuBodhi{「釋放」}(liberations, emancipations, AN)。按:此字多用於指「定」,如「八解脫」(aṭṭha vimokkhā,另譯為「八背捨」),而非涅槃的「解脫」(vimutti)。《破斥猶豫》以「勝解義」(adhimuccanaṭṭho)解說,即「從障礙法的善解脫與因在所緣上之喜樂而善解脫」(Paccanīkadhammehi ca suṭṭhu muccanaṭṭho, ārammaṇe ca abhirativasena suṭṭhu muccanaṭṭho, \ccchref{MN.77}{https://agama.buddhason.org/MN/dm.php?keyword=77})。
\stopitemgroup

\startitemgroup[noteitems]
\item\subnoteref{489.0}\NoteKeywordAgamaHead{「智(SA);得智/所得智(MA)」},南傳作\NoteKeywordNikaya{「完全智」}(aññā, aññaṃ,另譯為「了知;開悟;已知」),菩提比丘長老英譯為\NoteKeywordBhikkhuBodhi{「最終的理解」}(final knowledge)。按:「完全智」與「究竟智」(sammadaññā)的意思似乎是等同的(長老兩者的英譯相同),北傳多譯為「究竟智」,《顯揚真義》以「知道」(jānitvā, \suttaref{SN.9.10}),或「阿羅漢境界」(arahattassa nāmaṃ, \suttaref{SN.12.70}/\ccchref{MN.10}{https://agama.buddhason.org/MN/dm.php?keyword=10}/\ccchref{DN.22}{https://agama.buddhason.org/DN/dm.php?keyword=22})解說。「以完全智解脫者」,《滿足希求》則以「以阿羅漢果解脫的解脫者」(arahattaphalavimuttiyā vimuttassa, \ccchref{AN.3.86}{https://agama.buddhason.org/AN/an.php?keyword=3.86})解說。
\stopitemgroup

\startitemgroup[noteitems]
\item\subnoteref{490.0}\NoteKeywordAgamaHead{「度一切色想(MA);度色想/越一切色想(DA)」},南傳作\NoteKeywordNikaya{「從一切色想的超越」}(sabbaso rūpasaññānaṃ samatikkamā),菩提比丘長老英譯為\NoteKeywordBhikkhuBodhi{「以色之認知的完全超越」}(with the complete surmounting of perceptions of form, \ccchref{MA.8}{https://agama.buddhason.org/MA/dm.php?keyword=8}, with the complete transcendence of perceptions of forms, \suttaref{SN.36.19})。
\stopitemgroup

\startitemgroup[noteitems]
\item\subnoteref{491.0}\NoteKeywordAgamaHead{「無餘斷(SA/MA);斷無餘(MA);永斷無餘/永盡無餘(AA)」},南傳作\NoteKeywordNikaya{「無餘褪去與滅」}(asesavirāganirodho),菩提比丘長老英譯為\NoteKeywordBhikkhuBodhi{「無殘餘褪去與終止」}(the remainderless fading away and ceasing)。
\stopitemgroup

\startitemgroup[noteitems]
\item\subnoteref{492.0}\NoteKeywordAgamaHead{「無餘知,無餘見(MA)」},南傳作\NoteKeywordNikaya{「無殘留智見者」}(aparisesaṃ ñāṇadassanaṃ),菩提比丘長老英譯為\NoteKeywordBhikkhuBodhi{「有完全的理解與眼光」}(to have complete knowledge and vision),或「有全包括的理解與眼光」(to have all-embracing knowledge and vision, \ccchref{AN.3.75}{https://agama.buddhason.org/AN/an.php?keyword=3.75})。按:這裡的「見」(dassanaṃ,名詞,動詞dassati),指義理的看見(認識到;領悟到),與觀念、見解的「見」(diṭṭhi)不同。
\stopitemgroup

\startitemgroup[noteitems]
\item\subnoteref{493.0}\NoteKeywordNikayaHead{「離人之氛圍的」}(vijanavātāni,逐字譯為「離+人們+風」),菩提比丘長老英譯為\NoteKeywordBhikkhuBodhi{「遠離人們的擾亂」}(far from the flurry of people)。
\stopitemgroup

\startitemgroup[noteitems]
\item\subnoteref{494.0}\NoteKeywordAgamaHead{「捨淨念(SA);捨念清淨(MA);護念清淨(DA/AA)」},南傳作\NoteKeywordNikaya{「平靜、念遍純淨」}(upekkhāsatipārisuddhiṃ,逐字譯為「捨(平靜)+念+清淨(遍純淨)」),菩提比丘長老英譯為\NoteKeywordBhikkhuBodhi{「包括以平靜而潔淨的深切注意」}(includes the purification of mindfulness by equanimity, SN),或「有以平靜而潔淨的深切注意」(which has purification of mindfulness by equanimity, AN),智髻比丘長老英譯為「由平靜而純淨的深切注意」(purity of mindfulness due to equanimity, MN)。按:依《阿毘曇毘婆沙論》「第四禪有四枝:不苦不樂、捨、念、一心。」計,則應標點為「捨、淨念」;「捨、念清淨」;「護、念清淨」。
\stopitemgroup

\startitemgroup[noteitems]
\item\subnoteref{495.0}\NoteKeywordNikayaHead{「捨棄為所緣」}(vossaggārammaṇaṃ, SN; vavassaggārammaṇa, AN),菩提比丘長老英譯為\NoteKeywordBhikkhuBodhi{「釋放[為]對象」}(release the object, SN),或「基於釋放」(based on release, AN)。按:水野弘元《巴利語辭典》譯為「發勤的所緣」,《顯揚真義》說,這是以涅槃為所緣(nibbānārammaṇaṃ katvā, \suttaref{SN.48.9}),《滿足希求》說,「捨棄被稱為涅槃,它做為所緣」(vavassaggo vuccati nibbānaṃ, taṃ ārammaṇaṃ karitvāti, \ccchref{AN.1.333}{https://agama.buddhason.org/AN/an.php?keyword=1.333})。長老說,其原始含意可能只是以對解脫的熱望為動機之定的狀態(AN)。《小部/無礙解道》說,先隨看無常苦無我修毘婆舍那,然後生起那些法的捨棄為所緣狀態(vosaggārammaṇatā),心一境性不散亂,即為「毘婆舍那為先導修習舍摩他」(vipassanāpubbaṅgamaṃ samathaṃ bhāveti, \ccchref{Ps.11}{https://agama.buddhason.org/Ps/Ps11.htm}.4),這是另類觀點,不過其註釋書也說,這裡的捨棄指涅槃(ettha vosaggo nibbānaṃ),因為涅槃是有為的捨棄、遍捨(pariccāgato),以毘婆舍那與其相應的法傾斜、意向、住立、涅槃的意義上[說]涅槃為所緣,而在那裡生起那些法的捨棄為所緣狀態,以依止涅槃的狀態生起心的一境性、近不區分(upacārappanābhedo)、不散亂成為定,此為從毘婆舍那,之後生起洞察分之定(nibbedhabhāgiyo samādhi)的解說,這樣又拉回以涅槃為所緣,這段解說也意味著涅槃可同為止觀的所緣。
\stopitemgroup

\startitemgroup[noteitems]
\item\subnoteref{496.0}\NoteKeywordAgamaHead{「計我、異我、相在(SA);是我,是我所,我是彼所(MA);此是我許,我是彼所(AA)」},南傳作\NoteKeywordNikaya{「『我』、『我的』、『我是』」}(ahanti vā mamanti vā asmī),智髻比丘長老英譯為「『我』、『我的』、『我是』」('I' or 'mine' or ' I am')。按:《破斥猶豫》以「渴愛、慢、見之執」(taṇhāmānadiṭṭhiggāho, \ccchref{MN.28}{https://agama.buddhason.org/MN/dm.php?keyword=28})解說,菩提比丘長老說,三者分別代表「有身見、渴愛、慢」(personality view/identity view, craving, and conceit)之固執。
\stopitemgroup

\startitemgroup[noteitems]
\item\subnoteref{497.0}\NoteKeywordAgamaHead{「當來有(SA);當來有本/當來有/未來有/當生(MA);(AA)」},南傳作\NoteKeywordNikaya{「再有的」}(ponobbhavikā),比丘長老英譯為「導向重新存在」(lead to renewed existence, SN),智髻比丘長老英譯為「帶來生命的再有」(bring renewal of being, MN)。
\stopitemgroup

\startitemgroup[noteitems]
\item\subnoteref{498.0}\NoteKeywordAgamaHead{「生滅智慧(SA);修行智慧觀興衰法(MA)」},南傳作\NoteKeywordNikaya{「導向生起與滅沒…之慧」}(udayatthagāminiyā paññāya…),菩提比丘長老英譯為\NoteKeywordBhikkhuBodhi{「識別出現與消失之智慧」}(the wisdom that discerns arising and passing away, AN),或「指向出現與消失之智慧」(wisdom directed to arising and passing away, SN)。
\stopitemgroup

\startitemgroup[noteitems]
\item\subnoteref{499.0}\NoteKeywordAgamaHead{「施(SA/AA);庶幾(MA);給施(DA)」},南傳作\NoteKeywordNikaya{「施捨」}(cāgo,另譯為「捨;棄捨」),菩提比丘長老英譯為\NoteKeywordBhikkhuBodhi{「慷慨」}(generosity)。(比對\ccchref{MA.150}{https://agama.buddhason.org/MA/dm.php?keyword=150}與\ccchref{MN.96}{https://agama.buddhason.org/MN/dm.php?keyword=96})
\stopitemgroup

\startitemgroup[noteitems]
\item\subnoteref{500.0}\NoteKeywordAgamaHead{「如是行、如是因、如是信/如是行、如是因、如是方(SA);有行有相貌(MA);若干種相(DA)」},南傳作\NoteKeywordNikaya{「有行相的、有境遇的」}(sākāraṃ sa-uddesaṃ),菩提比丘長老英譯為\NoteKeywordBhikkhuBodhi{「帶有他們的樣子/樣式與細節」}(with their aspects/modes and details, AN,MN/SN)。Maurice Walshe先生英譯為「他們的情況與細節」(their conditions and details)。
\stopitemgroup

\startitemgroup[noteitems]
\item\subnoteref{501.0}\NoteSubKeyHead{(1)}\NoteKeywordAgamaHead{「寂靜解脫(SA);息解脫(MA)」},南傳作\NoteKeywordNikaya{「寂靜解脫」}(santā vimokkhā),智髻比丘長老英譯為「平和的解脫」(liberations that are peaceful)。按:《顯揚真義》以「寂靜的無色[界]解脫」(santā āruppavimokkhā, \suttaref{SN.12.70})解說,《破斥猶豫》說,「無色」指八等至(aṭṭhapi samāpattiyo),或至少在一種遍處上對遍作(準備)業處(parikammakammaṭṭhānaṃ, \ccchref{MN.69}{https://agama.buddhason.org/MN/dm.php?keyword=69})熟練。
\item\subnoteref{501.1}\NoteSubKeyHead{(2)}\NoteKeywordAgamaHead{「息心解脫(MA)」},南傳作\NoteKeywordNikaya{「寂靜的心解脫」}(santaṃ cetovimuttiṃ),菩提比丘長老英譯為\NoteKeywordBhikkhuBodhi{「心平和的解脫」}(peaceful liberation of mind)。按:《滿足希求》以「八等至中的第四禪等至」解說,因為與雜染相反的寂靜狀態為寂靜,從雜染心解脫的狀態被稱為心解脫(\ccchref{AN.2.37}{https://agama.buddhason.org/AN/an.php?keyword=2.37})。
\stopitemgroup

\startitemgroup[noteitems]
\item\subnoteref{502.0}\NoteKeywordAgamaHead{「作起覺想/作起想思惟(SA);作於起想/起於念覺/生起想(GA);常欲起想/常念欲起(MA);念當時起(DA)」},南傳作\NoteKeywordNikaya{「作意起來想後」}(uṭṭhānasaññaṃ manasi karitvā),菩提比丘長老英譯為\NoteKeywordBhikkhuBodhi{「在你的心中印記了起來的念頭後」}(after noting in your mind the idea of rising, AN),或「作了起來的想法」(having attended to the idea of rising, SN),智髻比丘長老英譯為「在我們的心中印記起來的時間後」(after noting in our minds the time for rising, MN)。
\stopitemgroup

\startitemgroup[noteitems]
\item\subnoteref{503.0}\NoteSubKeyHead{(1)}\NoteKeywordNikayaHead{「神通」}(iddhi,另譯為「神變」),菩提比丘長老英譯為\NoteKeywordBhikkhuBodhi{「超常力量」}(spiritual power, SN),或「精神力量」(psychic potency, AN)。
\item\subnoteref{503.1}\NoteSubKeyHead{(2)}\NoteKeywordAgamaHead{「四如意足(SA/MA/AA);四神足(AA)」},南傳作\NoteKeywordNikaya{「四神足」}(catunnaṃ iddhipādānaṃ),菩提比丘長老英譯為\NoteKeywordBhikkhuBodhi{「四個心靈潛能的基礎」}(Four Bases for Psychic Potency, AN),或「[四個]超常力量的基礎」([four] basis for spiritual power, SN),Maurice Walshe先生英譯為「四條他能通往力量之路」(four road to power he can)。按:其內容參看\ccchref{AA.29.7}{https://agama.buddhason.org/AA/dm.php?keyword=29.7}等,而\suttaref{SN.51.19}等說明了「神通」(iddhi)與「神足」(iddhipādāna)的差別。
\stopitemgroup

\startitemgroup[noteitems]
\item\subnoteref{504.0}\NoteKeywordAgamaHead{「繫念明想/係念明想/繫念明相/係念明相(SA);繫心在明(GA);光明想/意係明相/心作明想/念光明相(MA);繫想在明(DA);思惟計明之想/向明之想/計意在明/繫意在明(AA)」},南傳作\NoteKeywordNikaya{「光明想」}(ālokasaññaṃ,另譯為「明想;光想」),菩提比丘長老英譯為\NoteKeywordBhikkhuBodhi{「明亮的認知」}( the perception of light, AN)。按:北傳光明想出現在中夜入睡前的準備,但南傳入睡前沒這個準備項目,而是出現在防止打瞌睡(\ccchref{AN.7.61}{https://agama.buddhason.org/AN/an.php?keyword=7.61})與離惛沈睡眠(\ccchref{MN.10}{https://agama.buddhason.org/MN/dm.php?keyword=10}等),或是「導向智見獲得」的修習(\ccchref{AN.4.41}{https://agama.buddhason.org/AN/an.php?keyword=4.41}, \ccchref{AN.6.29}{https://agama.buddhason.org/AN/an.php?keyword=6.29}, \ccchref{DN.33}{https://agama.buddhason.org/DN/dm.php?keyword=33})。《破斥猶豫》等以「日夜具備能夠被看見的光明覺知、離蓋、遍淨想」(rattimpi divāpi diṭṭhaālokasañjānanasamatthāya vigatanīvaraṇāya parisuddhāya saññāya samannāgato, \ccchref{MN.27}{https://agama.buddhason.org/MN/dm.php?keyword=27}/\ccchref{DN.2}{https://agama.buddhason.org/DN/dm.php?keyword=2}/\ccchref{AN.4.198}{https://agama.buddhason.org/AN/an.php?keyword=4.198})解說,《滿足希求》另以「對光明相生起想」(ālokanimitte uppannasaññaṃ, \ccchref{AN.6.29}{https://agama.buddhason.org/AN/an.php?keyword=6.29})解說。
\stopitemgroup

\startitemgroup[noteitems]
\item\subnoteref{505.0}\NoteKeywordAgamaHead{「成眼、成智(MA);眼生、智生(AA)」},南傳作\NoteKeywordNikaya{「作眼、作智」}(cakkhukaraṇī ñāṇakaraṇī),智髻比丘長老英譯為「給予眼光,給予理解」(giving vision, giving knowledge, MN),菩提比丘長老英譯為\NoteKeywordBhikkhuBodhi{「給予眼光立起,給予理解立起」}(gives rise to vision, gives rise to knowledge, SN)。按:「作眼」,《顯揚真義》以「作慧眼」(Paññācakkhuṃ karotīti, \suttaref{SN.56.11}),《破斥猶豫》以「他以諸諦之看見轉起看見為領導者」(Sā hi saccānaṃ dassanāya saṃvattati dassanapariṇāyakaṭṭhenāti, \ccchref{MN.3}{https://agama.buddhason.org/MN/dm.php?keyword=3})解說,「作智」,《破斥猶豫》以「以諸諦之智轉起知道作的事(作知)」(Saccānaṃ ñāṇāya saṃvattati viditakaraṇaṭṭhenāti, \ccchref{MN.3}{https://agama.buddhason.org/MN/dm.php?keyword=3})解說。
\stopitemgroup

\startitemgroup[noteitems]
\item\subnoteref{506.0}\NoteKeywordAgamaHead{「塚間(SA/DA);息道/息止道(MA);塜間(AA)」},南傳作\NoteKeywordNikaya{「在墓地」}(sivathikāya, sīvathikāya),菩提比丘長老英譯為\NoteKeywordBhikkhuBodhi{「在棄屍地」}(in a charnel ground)。(塜=塚)
\stopitemgroup

\startitemgroup[noteitems]
\item\subnoteref{507.0}\NoteKeywordNikayaHead{「淨信;明淨」}(pasādo),菩提比丘長老英譯為\NoteKeywordBhikkhuBodhi{「信任」}(confidence)。
\stopitemgroup

\startitemgroup[noteitems]
\item\subnoteref{508.0}\NoteKeywordAgamaHead{「身諍(MA)」},南傳作\NoteKeywordNikaya{「身體躁動」}(sāraddhakāyo),菩提比丘長老英譯為\NoteKeywordBhikkhuBodhi{「身體被擾動」}(body…agitated)。按:「躁動」(sāraddha),原意為「激情的;暴躁的」。
\stopitemgroup

\startitemgroup[noteitems]
\item\subnoteref{509.0}\NoteKeywordNikayaHead{「意向圓滿的」}(paripuṇṇasaṅkappo),菩提比丘長老英譯為\NoteKeywordBhikkhuBodhi{「他們的願望已實現」}(their wish fulfilled),Maurice Walshe先生英譯為「滿足於已達到他的目標」(satisfied at having attained his end)。按:「意向」(saṅkappo),另譯為「思惟;思念;思;(意)志」,《破斥猶豫》說,想:當我聽聞聖難達葛法的教導時,願就在那個座位上證須陀洹果者,她作證須陀洹;斯陀含果、阿那含果、阿羅漢果者亦然,因而佛陀說:是悅意的,同時也是意向圓滿(\ccchref{MN.146}{https://agama.buddhason.org/MN/dm.php?keyword=146})。《吉祥悅意》以「意向已完成」(pariyositasaṅkappo, \ccchref{DN.25}{https://agama.buddhason.org/DN/dm.php?keyword=25})解說。
\stopitemgroup

\startitemgroup[noteitems]
\item\subnoteref{510.0}\NoteKeywordNikayaHead{「極度的」}(adhimatto,另譯為「增上」),菩提比丘長老英譯為\NoteKeywordBhikkhuBodhi{「格外的」}(extraordinary)。
\stopitemgroup

\startitemgroup[noteitems]
\item\subnoteref{511.0}\NoteKeywordAgamaHead{「生眼、智、明、覺(SA);生智、生眼、生覺、生明、生通、生慧、生證(DA);眼生、智生、明生、覺生、光生、慧生(AA)」},南傳作\NoteKeywordNikaya{「眼生起,智生起,慧生起,明生起,光生起」}(cakkhuṃ udapādi ñāṇaṃ udapādi paññā udapādi vijjā udapādi āloko udapādi),菩提比丘長老英譯為\NoteKeywordBhikkhuBodhi{「出現眼光、理解、智慧、真實的理解、光明」}(arose vision, knowledge, wisdom, true knowledge, and light, SN),經文中都用於證果時的場合。
\stopitemgroup

\startitemgroup[noteitems]
\item\subnoteref{512.0}\NoteKeywordAgamaHead{「食已出(SA)」},南傳作\NoteKeywordNikaya{「餐後已從施食返回」}(pacchābhattaṃ piṇḍapātapaṭikkanto),菩提比丘長老英譯為\NoteKeywordBhikkhuBodhi{「在他的飯後,從他的施捨範圍回去」}(after his meal, returned from his alms round)。
\stopitemgroup

\startitemgroup[noteitems]
\item\subnoteref{513.0}\NoteKeywordAgamaHead{「治生(MA)」},南傳作\NoteKeywordNikaya{「買賣」}(Vaṇijjā,另譯為「商賣;販賣」),智髻比丘長老英譯為「貿易」(Trade, MN)。
\stopitemgroup

\startitemgroup[noteitems]
\item\subnoteref{514.0}\NoteKeywordAgamaHead{「正志出要覺/出要覺/出要志(SA);無欲念(MA);無欲思(DA)」},南傳作\NoteKeywordNikaya{「離欲的意向」}(Nekkhammasaṅkappo),菩提比丘長老英譯為\NoteKeywordBhikkhuBodhi{「放棄的意向」}(intention of renunciation, SN),或「熱中於放棄的心思」(engrossed in thoughts of renunciation, AN)。
\stopitemgroup

\startitemgroup[noteitems]
\item\subnoteref{515.0}\NoteKeywordAgamaHead{「無恚覺/無恚志(SA);無恚念(MA);無恚思(DA)」},南傳作\NoteKeywordNikaya{「無惡意的意向」}(abyāpādasaṅkappo),菩提比丘長老英譯為\NoteKeywordBhikkhuBodhi{「無有害意志的意向」}(the intention of non-ill will)。
\stopitemgroup

\startitemgroup[noteitems]
\item\subnoteref{516.0}\NoteKeywordAgamaHead{「不害覺/不害志(SA);無害念(MA);無害思(DA)」},南傳作\NoteKeywordNikaya{「無加害的意向」}(avihiṃsāsaṅkappo),菩提比丘長老英譯為\NoteKeywordBhikkhuBodhi{「無殘酷的意向」}(the intention of non-cruelty)。
\stopitemgroup

\startitemgroup[noteitems]
\item\subnoteref{517.0}\NoteKeywordAgamaHead{「苦治(SA/MA)」},南傳作\NoteKeywordNikaya{「(他們)一再強制地作懲治」}(pasayha pasayha kāraṇaṃ karonti),菩提比丘長老英譯為\NoteKeywordBhikkhuBodhi{「他們採取行動反覆地告誡他」}(they take action……by repeatedly admonishing him)。
\stopitemgroup

\startitemgroup[noteitems]
\item\subnoteref{518.0}\NoteKeywordAgamaHead{「正向/善向(SA);善趣向/善趣(MA)」},南傳作\NoteKeywordNikaya{「善行者」}(suppaṭipanno),菩提比丘長老英譯為\NoteKeywordBhikkhuBodhi{「實行好的路」}(is practicing the good way, SN/AN),或「善實行者」(they are practising well, \suttaref{SN.38.3})。按:雖然「正向」字面上比較像「正直行者」(ujuppaṭipanno),但\ccchref{SA.931}{https://agama.buddhason.org/SA/dm.php?keyword=931}「正向、直向」分列,再依\ccchref{SA.490}{https://agama.buddhason.org/SA/dm.php?keyword=490}比對\suttaref{SN.38.3}, \ccchref{AN.3.73}{https://agama.buddhason.org/AN/an.php?keyword=3.73},還是列為相當於「善行者」。至於\ccchref{AN.3.73}{https://agama.buddhason.org/AN/an.php?keyword=3.73}的「善作者」(sukatā),其它版本作「善逝者」(sugatā),正好與\ccchref{SA.490}{https://agama.buddhason.org/SA/dm.php?keyword=490}/\suttaref{SN.38.3}相同。
\stopitemgroup

\startitemgroup[noteitems]
\item\subnoteref{519.0}\NoteKeywordAgamaHead{「正向滅盡/滅盡/向滅/滅盡向/滅盡法(SA);學斷/行滅(MA)」},南傳作\NoteKeywordNikaya{「(為了)滅的行者」}(nirodhāya paṭipanno),菩提比丘長老英譯為\NoteKeywordBhikkhuBodhi{「(為了)停止而實行」}(is practising for…cessation)。按:「向」,應為「行;行者」的對譯。
\stopitemgroup

\startitemgroup[noteitems]
\item\subnoteref{520.0}\NoteKeywordAgamaHead{「不自高-無憍慢想(SA)/不為色力(GA)/非以貢高故-不為貢高(MA)/不貢高-不懷慢恣(DA),不放逸-無著樂想(SA)/不貪味(DA),不著色-無摩拭想(SA)/肥鮮(GA)/不為自飾(MA)/不求飾好(DA)/肥白(AA),不著莊嚴-無莊嚴想(SA)/端正(GA)/不為莊嚴(MA)」},南傳作\NoteKeywordNikaya{「既不為了娛樂,也不為了自豪,不為了裝飾,不為了莊嚴」}(neva davāya na madāya na maṇḍanāya na vibhūsanāya),菩提比丘長老英譯為\NoteKeywordBhikkhuBodhi{「既不為了娛樂,不為了陶醉,也不為了身體的美麗與吸引力」}(neither for amusement nor for intoxication nor for the sake of physical beauty and attractiveness, \ccchref{AN.8.9}{https://agama.buddhason.org/AN/an.php?keyword=8.9})。按:《顯揚真義》說,詳細就在清淨道論1.18(visuddhimagge (visuddhi. 1.18) vitthāritāneva, \suttaref{SN.12.63})-「不為了娛樂」:「不像村落小孩等娛樂的目的(davatthaṃ),嬉戲的理由(kīḷānimittanti)。」是所說。「不為了自豪」:「不像拳擊手、摔角手等自豪的目的,力量自豪的理由,男子氣概自豪的理由(porisamadanimittañcāti)。」是所說。「不為了裝飾」:不像後宮者、妓女等裝飾的目的,肢體的豐滿狀態理由(pīṇabhāvanimittanti)。」是所說。「不為了莊嚴」:不像舞蹈者、表演者等莊嚴的目的,明淨的皮膚色澤狀態理由(pasannacchavivaṇṇatānimittanti)。」是所說。
\stopitemgroup

\startitemgroup[noteitems]
\item\subnoteref{521.0}\NoteKeywordAgamaHead{「念身/身念(MA/AA);身身念(MA);自念身(DA)」},南傳作\NoteKeywordNikaya{「身至念」}(kāyagatāsati, Kāyagatāya satiyā, Sati kāyagatā),菩提比丘長老英譯為\NoteKeywordBhikkhuBodhi{「指向身的深切注意」}(mindfulness directed to the body),佛授比丘長老英譯為「深切注意集中在身」(mindfulness centred on the body)。按:《吉祥悅意》說,「身至念」指安那般那、四舉止行為有念正知、三十二行相(身分)四界分別、十不淨、棄屍處碎破作意、在頭髮等上之四色[界]禪(esādīsu cattāri rūpajjhānānīti),在這裡生起念(\ccchref{DN.34}{https://agama.buddhason.org/DN/dm.php?keyword=34}),此即\ccchref{MN.119}{https://agama.buddhason.org/MN/dm.php?keyword=119}的內容。
\stopitemgroup

\startitemgroup[noteitems]
\item\subnoteref{522.0}\NoteKeywordNikayaHead{「一法」}(Ekadhammo),菩提比丘長老英譯為\NoteKeywordBhikkhuBodhi{「一件事」}(one thing)。
\stopitemgroup

\startitemgroup[noteitems]
\item\subnoteref{523.0}\NoteKeywordNikayaHead{「受入」},解讀為:
\item\subnoteref{523.1}①攝入;納入,如《阿毘曇八犍度論》(僧伽提婆共竺佛念譯):「三結為受入三不善根?三不善根為受入三結?答曰:各各不相受入。」該段異譯,《發智論》(玄奘法師譯)作:「三結、三不善根互不相攝。」
\item\subnoteref{523.2}②進入,如《集異門足論》(玄奘法師譯):「彼都無慧,猶如覆器,亦如覆瓶,雖多溉水竟無受入。」《阿含口解十二因緣經》(優婆塞都尉安玄共沙門嚴佛調譯):「如人生天上,含人識受入天識,便忘人間事。」
\item\subnoteref{523.3}③接受;接納。
\stopitemgroup

\startitemgroup[noteitems]
\item\subnoteref{524.0}\NoteKeywordAgamaHead{「七覺分/七覺支/七覺(SA);七覺法(MA);七覺意(DA/AA)」},南傳作\NoteKeywordNikaya{「七覺支」}(satta bojjhaṅgā, atta sambojjhaṅgā),菩提比丘長老英譯為\NoteKeywordBhikkhuBodhi{「七個開化要素」}(the seven factors of enlightenment)。
\stopitemgroup

\startitemgroup[noteitems]
\item\subnoteref{525.0}\NoteKeywordAgamaHead{「八聖道(SA/DA);八道(SA);八正道(SA/DA/AA);八真行/八真直行/賢聖八品道/八種之道(AA);賢聖八品道(SA/AA)」},南傳作\NoteKeywordNikaya{「八支聖道」}(Ariyañcaṭṭhaṅgikaṃ maggaṃ, ariyo aṭṭhaṅgiko maggo),菩提比丘長老英譯為\NoteKeywordBhikkhuBodhi{「八層的高潔之路」}(Noble Eightfold Path)。其內容為「正見、正志、正語、正業、正命、正精進、正念、正定」,而《增壹阿含經》多將「正」(sammā)譯為「等」,如「等見」即「正見」(sammādiṭṭhi),「等治;正治」即「正志;正思惟」(sammāsaṅkappa),「等方便」即「正精進;正勤」(sammāvāyāma)等。
\stopitemgroup

\startitemgroup[noteitems]
\item\subnoteref{526.0}\NoteKeywordNikayaHead{「他只具念地吸氣」}(So satova assasati),菩提比丘長老英譯為\NoteKeywordBhikkhuBodhi{「他只深切注意的吸氣」}(just mindful he breathes in, SN/AN),智髻比丘長老英譯為「他一直注意著而吸氣」(ever mindful he breathes in, MN)。按:「吸氣」(assasati),水野弘元《巴利語辭典》譯作「出息(呼氣)」,反之,「呼氣」(passasati),水野弘元《巴利語辭典》譯作「入息(吸氣)」,並加註「本來是出息(呼氣)」。
\stopitemgroup

\startitemgroup[noteitems]
\item\subnoteref{527.0}\NoteKeywordAgamaHead{「覺知一切身(SA);學一切身(MA);盡觀身體入息(AA);息入遍身(摩訶僧祇律)」},南傳作\NoteKeywordNikaya{「經驗著一切身」}(sabbakāyappaṭisaṃvedī),菩提比丘長老英譯為\NoteKeywordBhikkhuBodhi{「體驗著全身」}(experiencing the whole body)。按:「經驗著」(paṭisaṃvedī,另譯為「感受著」),形容詞,但以如現在分詞的動作形容詞解讀,《清淨道論》說,建立(作)著知道的、建立著顯現的(viditaṃ karonto pākaṭaṃ karonto)全部吸氣身的開端、中間與結束(sakalassa assāsakāyassa ādimajjhapariyosānaṃ, 8.220),全部呼氣身(Sakalassa passāsakāyassa)亦然。
\stopitemgroup

\startitemgroup[noteitems]
\item\subnoteref{528.0}\NoteKeywordAgamaHead{「一切身行息/身行休息/身息(SA);學止身行(MA);身行捨(摩訶僧祇律)」},南傳作\NoteKeywordNikaya{「使身行寧靜著」}(passambhayaṃ kāyasaṅkhāraṃ),菩提比丘長老英譯為\NoteKeywordBhikkhuBodhi{「使身體的形成(身行)寧靜」}(tranquillising the bodily formation, \suttaref{SN.54.1})。按:\suttaref{SN.41.6}、\ccchref{MN.44}{https://agama.buddhason.org/MN/dm.php?keyword=44}說:「入息出息是身行。」「使寧靜著」(passambhayaṃ,現在分詞),《清淨道論》以「使變得寧靜(輕安)、止息、滅、平息」(passambhento paṭippassambhento nirodhento vūpasamento, 8.220)等一連串分詞解說。
\stopitemgroup

\startitemgroup[noteitems]
\item\subnoteref{529.0}\NoteKeywordAgamaHead{「繫念面前(SA);正願(MA);繫念在前(AA)」},南傳作\NoteKeywordNikaya{「建立面前的念後」}(parimukhaṃ satiṃ upaṭṭhapetvā),菩提比丘長老英譯為\NoteKeywordBhikkhuBodhi{「在他面前建立深切注意」}(set up mindfulness in front of him, establishing mindfulness before him)。按:《清淨道論》說,比丘在鼻端(nāsikagge)或入口相上(mukhanimitte,長老英譯為「上唇(the upper lip, \suttaref{SN.54.1}/Note289)」)建立起念後而坐,不在來去的出息入息上(āgate vā gate vā assāsapassāse)作意,但來去的出息入息並非不被知道,勤奮被了知、努力完成,到達卓越(visesamadhigacchati, .8.227)。
\stopitemgroup

\startitemgroup[noteitems]
\item\subnoteref{530.0}\NoteKeywordAgamaHead{「攝受…住/繫著住(SA);所纏(MA)」},南傳作\NoteKeywordNikaya{「持續遍取」}(pariyādāya tiṭṭhanti,逐字譯為「遍取後-住立(存續)」),菩提比丘長老英譯為\NoteKeywordBhikkhuBodhi{「持續纏住」}(remain obsessing)。
\stopitemgroup

\startitemgroup[noteitems]
\item\subnoteref{531.0}\NoteKeywordAgamaHead{「闍維(SA/GA/DA);燒葬(GA);耶維(MA/DA);耶維/虵旬/蛇旬(AA)」},南傳作\NoteKeywordNikaya{「使火葬」}(jhāpeti,動詞,另音譯為「使燃燒」),菩提比丘長老英譯為\NoteKeywordBhikkhuBodhi{「火化」}(cremated, \ccchref{AN.5.50}{https://agama.buddhason.org/AN/an.php?keyword=5.50})。
\stopitemgroup

\startitemgroup[noteitems]
\item\subnoteref{532.0}\NoteKeywordAgamaHead{「臥覺常安樂/安隱臥/安隱眠/安睡眠/安眠(SA);安隱眠/眠快/歡喜眠(MA);善眠/臥安(AA)」},南傳作\NoteKeywordNikaya{「睡得安樂」}(sukhaṃ seti, sukhaṃ supati, sukhamasayittha,逐字譯為「樂臥」,「臥」為動詞),菩提比丘長老英譯為\NoteKeywordBhikkhuBodhi{「睡得安樂」}(sleeps at ease, slept well, \suttaref{SN.3.14}/10.8),「睡得香甜」(sleep soundly, \suttaref{SN.11.21}/7.1/1.71),或「活得快樂」(live happily, \ccchref{AN.3.129}{https://agama.buddhason.org/AN/an.php?keyword=3.129}),或「歡喜」(rejoices, \ccchref{AN.4.70}{https://agama.buddhason.org/AN/an.php?keyword=4.70})。
\stopitemgroup

\startitemgroup[noteitems]
\item\subnoteref{533.0}\NoteKeywordAgamaHead{「無所有入處(SA);不用處(GA);無所有處(MA/AA);不用處/不用定/無想入(DA);無有處/不用處(AA)」},南傳作\NoteKeywordNikaya{「無所有處」}(ākiñcañña āyatana, ākiñcaññāyatan),菩提比丘長老英譯為\NoteKeywordBhikkhuBodhi{「無的基礎」}(the base of nothingness)。
\stopitemgroup

\startitemgroup[noteitems]
\item\subnoteref{534.0}\NoteKeywordAgamaHead{「非想非非想入處(SA);非有想非無想處/無想/無想處(MA);非有想非無想處(DA);有想無想處(DA/MA/AA);尼維先天(AA)」},南傳作\NoteKeywordNikaya{「非想非非想處」}(nevasaññā-nāsaññāyatana),菩提比丘長老英譯為\NoteKeywordBhikkhuBodhi{「既非認知也非非認知的基礎」}(the base of neither-perception-nor-non-perception)。按:入此定時已無「粗想」,所以稱「非想」,但還有「微細想」,而此「微細想」已沒有想的功能,所以稱「非非想」,一如水面上的油花,雖還是油,但沒有油的一般功能一樣,《清淨道論》以鉢塗油的譬喻(Pattamakkhanatelappabhutīhi ca upamāhi, 10.287)解說。
\stopitemgroup

\startitemgroup[noteitems]
\item\subnoteref{535.0}\NoteSubKeyHead{(1)}\NoteKeywordAgamaHead{「四方比丘眾(MA);招提僧(DA/AA)」},南傳作\NoteKeywordNikaya{「四方僧團」}(cātuddisaṃ saṅghaṃ),菩提比丘長老英譯為\NoteKeywordBhikkhuBodhi{「四方的僧團」}(the Saṅgha of the four quarters)。
\item\subnoteref{535.1}\NoteSubKeyHead{(2)}\NoteKeywordNikayaHead{「四方者」}(cātuddiso,音譯為「招提」),菩提比丘長老英譯為\NoteKeywordBhikkhuBodhi{「四方為家者」}(at home in the four quarters)。按:《滿足希求》以「於四方無阻礙的行者」(appaṭihatacāro)解說。
\stopitemgroup

\startitemgroup[noteitems]
\item\subnoteref{536.0}\NoteSubKeyHead{(1)}\NoteKeywordAgamaHead{「故起苦覺/宿諸受(SA);故病/故疹(MA);故苦(DA);故痛(AA)」},南傳作\NoteKeywordNikaya{「之前的感受」}(purāṇañca vedanaṃ),菩提比丘長老英譯為\NoteKeywordBhikkhuBodhi{「舊感受」}(the old feeling)。
\item\subnoteref{536.1}\NoteSubKeyHead{(2)}\NoteKeywordAgamaHead{「未起苦覺令不起故/新諸受不生(SA);不起新病(MA);新苦不生(DA);新者不生(AA)」},南傳作\NoteKeywordNikaya{「不激起新的感受」}(navañca vedanaṃ na uppādessāmi),菩提比丘長老英譯為\NoteKeywordBhikkhuBodhi{「沒引起新的感覺」}(not arousing new feelings)。按:《舍利弗阿毘曇論》說,云何斷故受不生新受?若飢,緣飢故生身心苦受,是名故受。何謂新受?若食過度,緣過度故生身心苦受,是名新受。
\stopitemgroup

\startitemgroup[noteitems]
\item\subnoteref{537.0}\NoteSubKeyHead{(1)}\NoteKeywordAgamaHead{「自洲(SA);當自然法燈(MA);當自熾燃(DA);自熾然(AA)」},南傳作\NoteKeywordNikaya{「以自己為島」}(attadīpā,自-洲/燈),菩提比丘長老英譯為\NoteKeywordBhikkhuBodhi{「以你們自己為島」}(with yourselves as an island)。按:「dīpa」有兩個意思,一是「燈火」,一是「洲島」,後者又引申為「庇護所;依靠」。
\item\subnoteref{537.1}\NoteSubKeyHead{(2)}\NoteKeywordAgamaHead{「法洲(SA);熾燃於法(DA)」},南傳作\NoteKeywordNikaya{「以法為島」}(dhammadīpā,法-洲/燈),菩提比丘長老英譯為\NoteKeywordBhikkhuBodhi{「以法為島」}(with the Dhamma as an island)。
\item\subnoteref{537.2}\NoteSubKeyHead{(3)}\NoteKeywordAgamaHead{「不異洲(SA);莫然餘燈(MA);勿他熾燃(DA)」},南傳經文無,推斷應為「不以其他為島」(anaññattadīpā)。
\stopitemgroup

\startitemgroup[noteitems]
\item\subnoteref{538.0}\NoteSubKeyHead{(1)}\NoteKeywordAgamaHead{「塔廟/塔(DA);鍮婆/偷婆(AA)」},南傳作\NoteKeywordNikaya{「塔」}(thūpa),菩提比丘長老英譯為\NoteKeywordBhikkhuBodhi{「紀念塚」}(a memorial mound, AN)。按:「鍮婆」,另作「偷婆、塔婆、塔廟」,為音譯(「鍮」讀作「偷」),又作「塔婆、窣睹波、窣堵波」,義譯為「方墳、圓塚」,即「供奉舍利(遺骸)的塔」。
\item\subnoteref{538.1}\NoteSubKeyHead{(2)}\NoteKeywordNikayaHead{「破裂塔的」}(bhinnathūpe),智髻比丘長老英譯為「其廟已破」(its shrine broken)。按:《破斥猶豫》以「被破壞依止處的」(bhinnapatiṭṭh, \ccchref{MN.104}{https://agama.buddhason.org/MN/dm.php?keyword=104})解說,並說這裡的塔指尼乾子(nāṭaputtova)。
\stopitemgroup

\startitemgroup[noteitems]
\item\subnoteref{539.0}\NoteKeywordAgamaHead{「慧解脫(SA/MA/DA);智慧解脫(DA/AA)」},南傳作\NoteKeywordNikaya{「慧解脫(者)」}(Paññāvimuttā),菩提比丘長老英譯為\NoteKeywordBhikkhuBodhi{「以智慧被釋放」}(liberated by wisdom)。按:《顯揚真義》說:「學友!我們是無禪定的乾觀者(nijjhānakā sukkhavipassakā, \suttaref{SN.12.70}),只以慧的程度解脫。」與\ccchref{SA.347}{https://agama.buddhason.org/SA/dm.php?keyword=347}「不得正受」之說相合,但長老不同意,認為[南傳]經文只說到缺乏神通與無色界[定],並沒有說到禪定……nijjhānakā也可以理解為審慮(nijjhāna)的名詞化而成為審慮者。《滿足希求》說,對身體觀察不淨……對一切行觀察無常而當生證涅槃者解說為乾觀者(sukkhavipassakā, \ccchref{AN.4.169}{https://agama.buddhason.org/AN/an.php?keyword=4.169}-無禪定者),而「種種剎那定毘婆舍那」(nānākkhaṇikā samādhivipassanā, \ccchref{AN.2.32}{https://agama.buddhason.org/AN/an.php?keyword=2.32})之解說也類於乾觀。又《破斥猶豫》等說,有五類慧解脫者:乾觀者、從四種禪定(catūhi jhānehi)之任一出來後達到阿羅漢狀態者,他們未作證八解脫(Aṭṭha vimokkhe, \ccchref{MN.70}{https://agama.buddhason.org/MN/dm.php?keyword=70}/\ccchref{DN.15}{https://agama.buddhason.org/DN/dm.php?keyword=15}/\ccchref{AN.7.14}{https://agama.buddhason.org/AN/an.php?keyword=7.14})(後者同\ccchref{MA.195}{https://agama.buddhason.org/MA/dm.php?keyword=195})。《阿毗達摩概要精解》:不以禪那為基礎而修習觀禪的禪修者名為「純觀者」(sukkhavipassaka)。當這種人達到道果時,他們的道與果心在相當於初禪水準生起(their path and fruition cittas occur at a level corresponding to the first jhāna, §§30-31)。
\stopitemgroup

\startitemgroup[noteitems]
\item\subnoteref{540.0}\NoteKeywordAgamaHead{「知反復者/知反復/反復(DA/AA)」},南傳作\NoteKeywordNikaya{「是知恩者、感恩者」}(kataññū hoti katavedī),菩提比丘長老英譯為\NoteKeywordBhikkhuBodhi{「感恩的與感激的」}(grateful and thankful)。(反復=返復)
\stopitemgroup

\startitemgroup[noteitems]
\item\subnoteref{541.0}\NoteKeywordNikayaHead{「不被解脫」}(na vimuccati),Maurice Walshe先生英譯為「不能讓他們自由」(not make free with them, \ccchref{DN.33}{https://agama.buddhason.org/DN/dm.php?keyword=33}),菩提比丘長老英譯為\NoteKeywordBhikkhuBodhi{「專注於它」}(focused on it, \ccchref{AN.5.200}{https://agama.buddhason.org/AN/an.php?keyword=5.200})。按:《吉祥悅意》、《滿足希求》以「不勝解」(nādhimuccati, \ccchref{DN.33}{https://agama.buddhason.org/DN/dm.php?keyword=33}/\ccchref{AN.5.200}{https://agama.buddhason.org/AN/an.php?keyword=5.200})解說。又,「不躍入、不明淨、不住立、不勝解(不志向)」這一組動詞是定型句,出現在\suttaref{SN.22.90}、\ccchref{AN.4.178}{https://agama.buddhason.org/AN/an.php?keyword=4.178},「不被解脫」可能是「不勝解(不志向)」之訛。
\stopitemgroup

\startitemgroup[noteitems]
\item\subnoteref{542.0}\NoteKeywordAgamaHead{「念休息(AA)」},南傳作\NoteKeywordNikaya{「寂靜隨念」}(upasamānussati),菩提比丘長老英譯為\NoteKeywordBhikkhuBodhi{「平靜的回憶」}(Recollection of peace)。按:《滿足希求》說,這是關於寂靜被生起的隨念(Upasamaṃ ārabbha uppannā anussati, \ccchref{AN.1.297}{https://agama.buddhason.org/AN/an.php?keyword=1.297})。
\stopitemgroup

\startitemgroup[noteitems]
\item\subnoteref{543.0}\NoteKeywordNikayaHead{「八難」},八種不能聽聞佛法的情況,參看\ccchref{MA.124}{https://agama.buddhason.org/MA/dm.php?keyword=124}、\ccchref{AA.42.1}{https://agama.buddhason.org/AA/dm.php?keyword=42.1}。
\stopitemgroup

\startitemgroup[noteitems]
\item\subnoteref{544.0}\NoteKeywordNikayaHead{「正知的行為者」}(sampajānakārī,另譯為「正知的作者),菩提比丘長老英譯為\NoteKeywordBhikkhuBodhi{「以清楚的理解而行動者」}(who acts with clear comprehension)。按:《破斥猶豫》舉了四個正知的內涵:i.目的正知(sātthakasampajaññaṃ),即清楚自己行為的動機。ii.適當正知(sappāyasampajaññaṃ)。iii.行境正知(gocarasampajaññaṃ),即把握自己所選擇的業處。iv.不迷妄正知(asammohasampajaññaṃ),主要是覺知行為中有我的錯覺(\ccchref{MN.10}{https://agama.buddhason.org/MN/dm.php?keyword=10})。《法蘊足論》說,云何正知?謂:離喜時所起於法簡擇,乃至毘鉢舍那,總名正知。《瑜伽師地論》說,云何正知巧便而臥?謂由正念而寢臥時,若有隨一煩惱現前染惱其心,於此煩惱現生起時,能正覺了令不堅著,速疾棄捨,既通達已,令心轉還。
\stopitemgroup

\startitemgroup[noteitems]
\item\subnoteref{545.0}\NoteSubKeyHead{(1)}\NoteKeywordAgamaHead{「隨死念(SA);死想(MA);念死想(DA);念死/死念(AA)」},南傳作\NoteSubEntryKey{(i)}\NoteKeywordNikaya{「死念」}(maraṇassati),菩提比丘長老英譯為\NoteKeywordBhikkhuBodhi{「死的深切注意」}(Mindfulness of death)。按:《滿足希求》以「這是以命根斷絕為所緣之念的同義語」(jīvitindriyupacchedārammaṇāya satiyā etaṃ adhivacan, \ccchref{AN.1.297}{https://agama.buddhason.org/AN/an.php?keyword=1.297}),或「死念業處」(maraṇassatikammaṭṭhānaṃ, \ccchref{AN.6.19}{https://agama.buddhason.org/AN/an.php?keyword=6.19})解說。\NoteSubEntryKey{(ii)}\NoteKeywordNikaya{「死想」}(Maraṇasaññā),菩提比丘長老英譯為\NoteKeywordBhikkhuBodhi{「死的認知」}(the perception of death)。按:《顯揚真義》說,經常省察「必定要死,我的生命被死束縛」(avassaṃ maritabbaṃ, maraṇapaṭibaddhaṃ me jīvita’’nti, \suttaref{SN.46.68})而生起的想。
\stopitemgroup

\startitemgroup[noteitems]
\item\subnoteref{546.0}\NoteKeywordAgamaHead{「非暫時解脫」(asamayavimokkhaṃ, asamayavimutto,另譯為「非時解脫;不時解脫」),菩提比丘長老英譯為「永久的釋放」(permanent liberation, perpetual liberation)。按:反義詞「暫時的解脫」(sāmāyikampi vimuttiṃ)、「暫時的心解脫」(sāmayikaṃ cetovimuttiṃ,\ccchref{SA.1091}{https://agama.buddhason.org/SA/dm.php?keyword=1091}作「時受意解脫」})。按:《破斥猶豫》說,四聖道、四沙門果、涅槃(Cattāro ca ariyamaggā cattāri ca sāmaññaphalāni, nibbānañca)為「非暫時解脫」(《無礙解道》Ps5, 213段),四禪、四無色界等至(Cattāri ca jhānāni catasso ca arūpāvacarasamāpattiyo, \ccchref{MN.29}{https://agama.buddhason.org/MN/dm.php?keyword=29})為「暫時解脫」。
\stopitemgroup

\startitemgroup[noteitems]
\item\subnoteref{547.0}\NoteKeywordNikayaHead{「法界」}(dhammadhātu),菩提比丘長老英譯為\NoteKeywordBhikkhuBodhi{「心-現象的元素」}(mental-phenomena element)。按:《顯揚真義》以「緣行相的(paccayākārassa)被揭開狀態與道理的看見之聲聞完美智(sāvakapāramīñāṇaṃ, \suttaref{SN.12.32})」解說,《阿毘達摩概要精解》以「心所(受想行三蘊)、微細色(即\ccchref{DN.33}{https://agama.buddhason.org/DN/dm.php?keyword=33}說的不可見無對色)、涅槃」定義。
\stopitemgroup

\startitemgroup[noteitems]
\item\subnoteref{548.0}\NoteKeywordAgamaHead{「聖默然(SA);默然(MA);賢聖默然(DA/AA)」},南傳作\NoteKeywordNikaya{「聖沈默狀態」}(ariyo tuṇhībhāvo, ariyo vā tuṇhībhāvo),菩提比丘長老英譯為\NoteKeywordBhikkhuBodhi{「保持高潔的沈默」}(maintain noble silence)。按:「初禪正受時,言語止息;二禪正受時,覺、觀止息」(\ccchref{SA.474}{https://agama.buddhason.org/SA/dm.php?keyword=474}),「有覺、有觀故,則口語」(\ccchref{SA.568}{https://agama.buddhason.org/SA/dm.php?keyword=568})。《顯揚真義》說,這意味著第二禪,或業處之作意、初禪等(kammaṭṭhānamanasikāropi paṭhamajjhānādīnipi, \suttaref{SN.21.1}),《破斥猶豫》說是第二禪或根本業處(mūlakammaṭṭhānampi, \ccchref{MN.26}{https://agama.buddhason.org/MN/dm.php?keyword=26}),《滿足希求》說是第二禪等至(dutiyajjhānasamāpattiṃ, \ccchref{AN.9.4}{https://agama.buddhason.org/AN/an.php?keyword=9.4}),或說,聖沈默狀態名為第四禪,或轉殘餘業處之作意(ariyatuṇhībhāvo nāma catutthajjhānaṃ, sesakammaṭṭhānamanasikāropi vaṭṭati, \ccchref{AN.8.2}{https://agama.buddhason.org/AN/an.php?keyword=8.2})。
\stopitemgroup

\startitemgroup[noteitems]
\item\subnoteref{549.0}\NoteSubKeyHead{(1)}\NoteKeywordNikayaHead{「姟」},《一切經音義》:「……算經云:十萬曰億;十億曰兆;十兆曰京;十京曰姟,數法名也,古今正字云:大也數也。……」
\item\subnoteref{549.1}\NoteSubKeyHead{(2)}\NoteKeywordAgamaHead{「那由他(SA);那術(MA/AA);那維(DA)」}(nahuta,另音譯為「那由佗;尼由多;那遊哆」),水野弘元《巴利語辭典》解說為「一萬」。
\stopitemgroup

\startitemgroup[noteitems]
\item\subnoteref{550.0}\NoteKeywordAgamaHead{「善除黑白(MA);黑白(DA)」},南傳作\NoteKeywordNikaya{「黑白有對比的」}(kaṇhasukkasappaṭibhāgaṃ, kaṇhasukkasappatibhāgesu),Maurice Walshe先生英譯為「什麼是污穢的、美好的或性質混合的」(what is foul, fair or mixed in quality, \ccchref{DN.18}{https://agama.buddhason.org/DN/dm.php?keyword=18}),或「黑暗與光明對比」(contrasting the dark with the light, \ccchref{DN.28}{https://agama.buddhason.org/DN/dm.php?keyword=28}),智髻比丘長老英譯為「以其黑與白相對」(with its dark and bright counterparts, MN),菩提比丘長老英譯為\NoteKeywordBhikkhuBodhi{「黑白狀態及其相對」}(dark and bright states with their counterparts, SN)。按:「有對比的」(paṭibhāgaṃ),《顯揚真義》以「相類果報部分(sadisavipākakoṭṭhāsāti)、敵對事實的(Paṭipakkhabhūtassa)、以有排斥義的(Sappaṭibāhitaṭṭhena, \suttaref{SN.46.2})」,《破斥猶豫》等以「有敵對的(savipakkhaṃ)、排斥(paṭibāhitvā)後有果報的(savipākaṃ, \ccchref{MN.47}{https://agama.buddhason.org/MN/dm.php?keyword=47}/\ccchref{DN.28}{https://agama.buddhason.org/DN/dm.php?keyword=28})」解說,《滿足希求》說,黑白互相排斥,因反對而有對比的(paṭipakkhavasena sappaṭibhāgāti, \ccchref{AN.3.29}{https://agama.buddhason.org/AN/an.php?keyword=3.29}, \ccchref{AN.5.141}{https://agama.buddhason.org/AN/an.php?keyword=5.141}等也類似)。
\stopitemgroup

\startitemgroup[noteitems]
\item\subnoteref{551.0}\NoteKeywordAgamaHead{「是取/所取(SA)」},南傳作\NoteKeywordNikaya{「與執取有關的」}(upādāniyā,另譯為「順取的;應取的」),菩提比丘長老英譯為\NoteKeywordBhikkhuBodhi{「可被固執者」}(that can be clung to)。
\stopitemgroup

\startitemgroup[noteitems]
\item\subnoteref{552.0}\NoteKeywordAgamaHead{「正住(SA);恒久存(MA);常住不移/常住不變(DA);常存/恒在(AA)」},南傳作\NoteKeywordNikaya{「將就像那樣永久地住立」}(sassatisamaṃ tatheva ṭhassati),菩提比丘長老英譯為\NoteKeywordBhikkhuBodhi{「那將正像永恆本身那樣保持相同」}(that will remain the same just like eternity itself)。按:「永久地」(sassatisamaṃ),《顯揚真義》以「等同須彌山、大地、日月等常恆」(sinerumahāpathavīcandimasūriyādīhi sassatīhi samaṃ, \suttaref{SN.22.96}),《破斥猶豫》以「日月、大海、大地、山岳之世間慣用語的常恆(lokavohārena sassatiyoti, \ccchref{MN.2}{https://agama.buddhason.org/MN/dm.php?keyword=2})」解說。
\stopitemgroup

\startitemgroup[noteitems]
\item\subnoteref{553.0}\NoteSubKeyHead{(1)}\NoteKeywordAgamaHead{「大人;大丈夫(SA)」},南傳作\NoteKeywordNikaya{「大丈夫」}(mahāpuriso,另譯為「大男子;大人」),菩提比丘長老英譯為\NoteKeywordBhikkhuBodhi{「偉大男人」}(Great Man)。按:《吉祥悅意》以「誓願受持悲智等德行的偉大男子」(paṇidhisamādānañāṇakaruṇādiguṇamahato purisassa, \ccchref{DN.3}{https://agama.buddhason.org/DN/dm.php?keyword=3})解說。
\item\subnoteref{553.1}\NoteSubKeyHead{(2)}\NoteKeywordAgamaHead{「大人相/大人之相(MA);大人相(DA);大人之相/大人之相貌(AA)」},南傳作\NoteKeywordNikaya{「大丈夫相」}(mahāpurisalakkhaṇaṃ,另譯為「大人相;偉人的特徵」),智髻比丘長老英譯為「偉大男人的標誌」(marks of a Great Man, \ccchref{MN.91}{https://agama.buddhason.org/MN/dm.php?keyword=91})。
\item\subnoteref{553.2}\NoteSubKeyHead{(3)}\NoteKeywordAgamaHead{「大人之念(MA);大人念(AA);大人覺(DA)」},南傳作\NoteKeywordNikaya{「大丈夫之尋」}(mahāpurisavitakkaṃ),菩提比丘長老英譯為\NoteKeywordBhikkhuBodhi{「了不起的人之心思」}(thoughts of a great person)。
\stopitemgroup

\startitemgroup[noteitems]
\item\subnoteref{554.0}\NoteKeywordAgamaHead{「深水之中星宿(AA)」},南傳作\NoteKeywordNikaya{「水光」}(udakatārakā),智髻比丘長老英譯為「水的閃光;水的微光」(a gleam of water)。按:「星宿」,應為「光」(tārakā,另譯為「星;恆星;輝耀光亮之物」)的對譯。
\stopitemgroup

\startitemgroup[noteitems]
\item\subnoteref{555.0}\NoteKeywordNikayaHead{「以離被限制之心」}(vimariyādīkatena cetasā),菩提比丘長老英譯為\NoteKeywordBhikkhuBodhi{「以擺脫障壁之心」}(with a mind rid of barriers)。按:「離被限制」(vimariyādīkata, vimariyādikata, vi-mariyādā-kata),另譯為「無限制的;已變成自由的;被解放的」。
\stopitemgroup

\startitemgroup[noteitems]
\item\subnoteref{556.0}\NoteKeywordNikayaHead{「法鏡」}(dhammādāsotipi),菩提比丘長老英譯為\NoteKeywordBhikkhuBodhi{「法的鏡子」}(The Mirror of the Dhamma)。按:《顯揚真義》等以「法所成的鏡子」(dhammamayaṃ ādāsaṃ, \suttaref{SN.55.8}/\ccchref{DN.16}{https://agama.buddhason.org/DN/dm.php?keyword=16})解說。
\stopitemgroup

\startitemgroup[noteitems]
\item\subnoteref{557.0}\NoteKeywordAgamaHead{「心緣起法(SA);從因緣生/隨所因緣生(MA)」},南傳作\NoteKeywordNikaya{「緣所生的」}(paṭiccasamuppannaṃ),菩提比丘長老英譯為\NoteKeywordBhikkhuBodhi{「依之而發生者」}(dependently arisen)。
\stopitemgroup

\startitemgroup[noteitems]
\item\subnoteref{558.0}\NoteSubKeyHead{(1)}\NoteKeywordNikayaHead{「離染」}(virajjati, virajjituṃ),菩提比丘長老英譯為\NoteKeywordBhikkhuBodhi{「冷靜;不為情所動」}(become dispassionate, dispassionate)。
\item\subnoteref{558.1}\NoteSubKeyHead{(2)}\NoteKeywordNikayaHead{「使離染」}(virājeti),菩提比丘長老英譯為\NoteKeywordBhikkhuBodhi{「分離」}(detaches, AN)。
\stopitemgroup

\startitemgroup[noteitems]
\item\subnoteref{559.0}\NoteKeywordAgamaHead{「此識身及外境界一切相/內識身及外一切相(SA);內身共有識及外諸相(MA)」},南傳作\NoteKeywordNikaya{「在這個有識之身上與在一切外部諸相上」}(imasmiñca saviññāṇake kāye bahiddhā ca sabbanimittesu),菩提比丘長老英譯為\NoteKeywordBhikkhuBodhi{「關於具有識的身體與所有外部的徵候」}(in regard to this body with consciousness and in regard to all external signs)。按:《顯揚真義》說,前者指自己的識身或自己,後者指其他人的識身或其他人或無識者(aviññāṇakaṃ),或非被根繫縛之色(anindriyabaddharūpaṃ, \suttaref{SN.18.21})。
\stopitemgroup

\startitemgroup[noteitems]
\item\subnoteref{560.0}\NoteKeywordAgamaHead{「沙門數/沙門之沙門數(SA);上尊沙門(MA)」},南傳作\NoteKeywordNikaya{「沙門中的沙門」}(samaṇesu vā samaṇa),菩提比丘長老英譯為\NoteKeywordBhikkhuBodhi{「禁欲修道者中的禁欲修道者」}(ascetics among ascetics)。
\stopitemgroup

\startitemgroup[noteitems]
\item\subnoteref{561.0}\NoteKeywordNikayaHead{「作善惡二重者」}(dvayakārino),菩提比丘長老英譯為\NoteKeywordBhikkhuBodhi{「矛盾地行動;兩種衝突地行動」}(acted ambivalently)。按:《顯揚真義》以「作兩種者、作善惡者」(duvidhakārino, kusalākusalakārinoti, \suttaref{SN.29.3})解說。
\stopitemgroup

\startitemgroup[noteitems]
\item\subnoteref{562.0}\NoteSubKeyHead{(1)}\NoteKeywordNikayaHead{「法門;方便」}(pariyāyaṃ,另譯為「教說、部門、理趣、理由、順序」),菩提比丘長老英譯為\NoteKeywordBhikkhuBodhi{「主題」}(the theme),或「種類」(kinds),或「方法」(a method, \suttaref{SN.35.153}),或「在臨時意義上」(in a provisional sense, \ccchref{AN.9.42}{https://agama.buddhason.org/AN/an.php?keyword=9.42}-61),智髻比丘長老英譯為「講說」(a discourse, MN)。按:「法門」(pariyāya,音譯為「波利耶夜」),參看《原始佛教聖典之集成》p.728。「方便」,《滿足希求》以「依一個因素[到達]」(ekena kāraṇena, \ccchref{AN.9.42}{https://agama.buddhason.org/AN/an.php?keyword=9.42})解說。
\item\subnoteref{562.1}\NoteSubKeyHead{(2)}\NoteKeywordNikayaHead{「法的教說」}(dhammapariyāyaṃ, dhammapariyāyo),菩提比丘長老英譯為\NoteKeywordBhikkhuBodhi{「法的講說;法的解說」}(Dhamma discourse, Dhamma exposition; exposition of the Dhamma)。按:《顯揚真義》以「法的理由」(dhammakāraṇaṃ, \suttaref{SN.12.45})解說,《破斥猶豫》以「法的教說」(dhammadesanāya, \ccchref{MN.5}{https://agama.buddhason.org/MN/dm.php?keyword=5}」解說。
\stopitemgroup

\startitemgroup[noteitems]
\item\subnoteref{563.0}\NoteSubKeyHead{(1)}\NoteKeywordAgamaHead{「下道(SA);不肯著路(GA);邪道/惡道(MA)」},南傳作\NoteKeywordNikaya{「旁道;歧途」}(ummaggaṃ),智髻比丘長老英譯為「旁路」(a bypath),或「錯的路」(a wrong road, MN),菩提比丘長老英譯為\NoteKeywordBhikkhuBodhi{「錯的路」}(a wrong path, AN)。按:《顯揚真義》以「天界或人間或往涅槃的非道(邪道, amaggo, \suttaref{SN.35.246})」解說。
\item\subnoteref{563.1}\NoteSubKeyHead{(2)}\NoteKeywordAgamaHead{「賢道(MA)」},南傳作\NoteKeywordNikaya{「想法」}(ummaṅgo, ummaggo),菩提比丘長老英譯為\NoteKeywordBhikkhuBodhi{「智慧」}(intelligence),或「敏銳;聰明」(acumen, AN)。按:《滿足希求》以「浮現,走到慧」(ummujjanaṃ, paññāgamananti, \ccchref{AN.4.186}{https://agama.buddhason.org/AN/an.php?keyword=4.186}),或「問題想法」(pañhummaggo, \ccchref{AN.4.192}{https://agama.buddhason.org/AN/an.php?keyword=4.192})解說。菩提比丘長老2009/7/20回函告知,ummaṅgo, ummaggaṃ PTS英巴辭典(year 2000)中列有三個意思i.歧途;邪道(a wrong way, a deviant way)。ii.「隧道」(a tunnel)。iii.「出現在表面上的一個主意,一個概念,最初的心思」(emergence on the surface, an idea, a conception, an initial thought)。
\stopitemgroup

\startitemgroup[noteitems]
\item\subnoteref{564.0}\NoteKeywordAgamaHead{「大德神力(SA)」},南傳作\NoteKeywordNikaya{「大通智」}(mahābhiññataṃ),菩提比丘長老英譯為\NoteKeywordBhikkhuBodhi{「巨大之直接的理解」}(greatness of direct knowledge)。按:《顯揚真義》以「六種神通情況」(chaḷabhiññataṃ, \suttaref{SN.21.1}/47.28)解說。
\stopitemgroup

\startitemgroup[noteitems]
\item\subnoteref{565.0}\NoteKeywordAgamaHead{「一乘道(SA);一道(MA/AA);一入道(AA);唯有一道(GA)」},南傳作\NoteKeywordNikaya{「無岔路之道」}(Ekāyano……maggo),Maurice Walshe先生英譯為「一條道路」(one way, DN),智髻比丘長老英譯為「直接的道路」(the direct path, MN),菩提比丘長老英譯為\NoteKeywordBhikkhuBodhi{「單行道」}(the one-way path, SN),並解說,此詞常被譯為「唯一之道」(the only way, the sole way),意味著這是獨家之道(exclusive path),但《顯揚真義》等只說,這不成為歧道(na dvedhāpathabhūto, \suttaref{SN.47.1}/\ccchref{MN.10}{https://agama.buddhason.org/MN/dm.php?keyword=10})。此詞在\ccchref{MN.12}{https://agama.buddhason.org/MN/dm.php?keyword=12}清楚地表示,其意思是「直通目的地之路」(a path leading straight to its destination-相當的\ccchref{AA.50.6}{https://agama.buddhason.org/AA/dm.php?keyword=50.6}譯作「直從一道來」),也許這是比對其它總是無法直通目的地的禪法而說的。又,此詞不應與《妙法蓮華經》(the Saddharma Puṇḍarika Satra)中心主題的「一乘」(ekayāna)混淆(\suttaref{SN.47.1})。按:檢視阿含經中稱「一乘道;一道」的,\ccchref{SA.550}{https://agama.buddhason.org/SA/dm.php?keyword=550}為「六念」、\ccchref{SA.561}{https://agama.buddhason.org/SA/dm.php?keyword=561}為「四如意足」、\ccchref{SA.563}{https://agama.buddhason.org/SA/dm.php?keyword=563}為「戒定慧」、\ccchref{MA.189}{https://agama.buddhason.org/MA/dm.php?keyword=189}為「正定-八正道」,都不是「四念住」,這與菩提比丘長老不贊成將之譯為「唯一之道」的觀點相順。另外\ccchref{SA.962}{https://agama.buddhason.org/SA/dm.php?keyword=962}佛陀沒有「這才正確,其它都錯。」的觀念,也可參考。
\stopitemgroup

\startitemgroup[noteitems]
\item\subnoteref{566.0}\NoteKeywordAgamaHead{「智慧明達/決定智慧/決定智/分別智慧(SA);別智(GA);慧明達/明達慧/明達智慧(MA)」},南傳作\NoteKeywordNikaya{「洞察慧」}(nibbedhikapañño,另譯為「抉擇慧;擇慧;明達慧」),菩提比丘長老英譯為\NoteKeywordBhikkhuBodhi{「有洞察力的智慧」}(penetrative wisdom)。按:《顯揚真義》等以「洞察、破壞以前未洞察、未破壞的貪蘊(anibbiddhapubbaṃ appadālitapubbaṃ lobhakkhandhaṃ nibbijjhati padāletīti);…瞋蘊…癡蘊…憤怒…怨恨…一切導向有的業(bhavagāmikamme, \suttaref{SN.2.29}/\ccchref{MN.111}{https://agama.buddhason.org/MN/dm.php?keyword=111}/\ccchref{DN.30}{https://agama.buddhason.org/DN/dm.php?keyword=30}/\ccchref{AN.1.584}{https://agama.buddhason.org/AN/an.php?keyword=1.584})解說。
\stopitemgroup

\startitemgroup[noteitems]
\item\subnoteref{567.0}\NoteKeywordAgamaHead{「菩提分法(SA)」},南傳作\NoteKeywordNikaya{「覺分法」}(bodhipakkhiyā dhammā,另譯為「菩提分法」),菩提比丘長老英譯為\NoteKeywordBhikkhuBodhi{「有益於開化的狀態」}(the states conducive to enlightenment)。按:「分」(pakkhiyā),另譯為「伴黨(徒)的;部分的」。
\stopitemgroup

\startitemgroup[noteitems]
\item\subnoteref{568.0}\NoteKeywordAgamaHead{「欲定斷行成就如意足(SA);欲定如意足(MA);欲定滅行成就修習神足/欲定精勤不懈滅行成就以修神足(DA);自在三昧行盡神足/自在三昧神力(AA)」},南傳作\NoteKeywordNikaya{「具備意欲定勤奮之行的神足」}(chandasamādhippadhānasaṅkhārasamannāgataṃ iddhipādaṃ),Maurice Walshe先生英譯為「具備努力意願的意志之集中貫注的通往力量之路」(the road to power which is concentration of intention accompanied by effort of will),菩提比丘長老英譯為\NoteKeywordBhikkhuBodhi{「持有基於想要與努力之意志形成的集中貫注之超常力量的基礎」}(the basis for spiritual power that possesses concentration due to desire and volitional formations of striving, \suttaref{SN.51.15}),並解說,神足(iddhipādaṃ)可解讀為「往神通的基礎」(iddhiyā pāda),或「神通的基礎」(iddhibhūtaṃ pādaṃ),依\suttaref{SN.51.13},神足包括三部分:「定、四勤奮之行、產生定的特別因素:欲、活力、心、考察」。按:北傳經文的「斷行;滅行」,應該是「勤奮之行」(padhānasaṅkhāra),內容就是「四正斷;四正勤」(cattāro sammappadhāna)。
\stopitemgroup

\startitemgroup[noteitems]
\item\subnoteref{569.0}\NoteKeywordAgamaHead{「思惟定斷行成就如意足(SA);思惟定如意足(MA);思惟定滅行成就修習神足/思惟定精勤不懈滅行成就以修神足(DA);誡三昧行盡神足/試三昧神力(AA)」},南傳作\NoteKeywordNikaya{「具備考察定勤奮之行的神足」}(vīmaṃsāsamādhippadhānasaṅkhārasamannāgataṃ iddhipādaṃ),Maurice Walshe先生英譯為「具備努力意願的研究調查之集中貫注的通往力量之路」(the road to power which is concentration of investigation accompanied by effort of will),菩提比丘長老英譯為\NoteKeywordBhikkhuBodhi{「持有基於研究調查與努力之意志形成的集中貫注之超常力量的基礎」}(the basis for spiritual power that possesses concentration due to investigation and volitional formations of striving, \suttaref{SN.51.15})。按:《法蘊足論》說,觀三摩地勝行成就神足者,……此中觀者,謂:依出家遠離所生善法所起,於法簡擇、極簡擇、最極簡擇;解了、等了、近了,機黠通達,審察聰叡,覺明慧行,毘鉢舍那,是名觀。
\stopitemgroup

\startitemgroup[noteitems]
\item\subnoteref{570.0}\NoteKeywordAgamaHead{「後前想(MA)」},南傳作\NoteKeywordNikaya{「前後有感知的」}(pacchāpuresaññī),菩提比丘長老英譯為\NoteKeywordBhikkhuBodhi{「什麼是前與什麼是後的知覺」}(percipient of what is behind and what is in front)。
\stopitemgroup

\startitemgroup[noteitems]
\item\subnoteref{571.0}\NoteKeywordAgamaHead{「空、無願、無相(MA);空三昧、無願三昧、無相三昧/空三昧、無相三昧、無作三昧/空三昧、無想三昧、無作三昧(DA);空三昧、無願三昧、無相三昧/空三昧、無願三昧、無想三昧(AA)」},南傳作\NoteKeywordNikaya{「空定、無相定、無願定」}(Suññato samādhi, animitto samādhi, appaṇihito samādhi),菩提比丘長老英譯為\NoteKeywordBhikkhuBodhi{「空的集中貫注、無形跡的集中貫注、無指向/無願望的集中貫注」}(The emptiness concentration, the signless/markless concentration, the undirected/wishless concentration, \suttaref{SN.43.4}/\ccchref{AN.3.184}{https://agama.buddhason.org/AN/an.php?keyword=3.184})。按:《吉祥悅意》說,執著無我、看見無我後從無我出來,他的毘婆舍那名為空,因為非空之作者的雜染不存在(Asuññatattakārakānaṃ kilesānaṃ abhāvā),以毘婆舍那導致的道定(Vipassanāgamanena maggasamādhi)名為空,以道導致的果{定?}名為空的。再者,執著無常……他的毘婆舍那名為無相,因為相之作者的雜染不存在……。再者,執著苦……他的毘婆舍那名為無願,因為願求之作者的雜染不存在……。這是從抵達(āgamanato)說的。又,道定以貪等空的狀態為空的,貪等相不存在為無相,貪願求等不存在為無願,這是從重疊(saguṇato)說的。涅槃以貪等空的狀態、貪等相與願求不存在為空、無相、無願,道定的那個所緣(Tadārammaṇo maggasamādhi)為空、無相、無願,這是從所緣(ārammaṇato, \ccchref{DN.33}{https://agama.buddhason.org/DN/dm.php?keyword=33})說的。
\stopitemgroup

\startitemgroup[noteitems]
\item\subnoteref{572.0}\NoteKeywordNikayaHead{「看見具足者」}(dassanasampanno,另譯為「見具足」),菩提比丘長老英譯為\NoteKeywordBhikkhuBodhi{「眼光已完成者」}(accomplished in vision),或「擁有眼光者」(possessed in vision, \ccchref{AN.3.45}{https://agama.buddhason.org/AN/an.php?keyword=3.45})。按:《顯揚真義》說是見具足的同義語,見具足為具足道之見(maggadiṭṭhiyā sampanno, \suttaref{SN.12.27}),《滿足希求》以「須陀洹」(sotāpanno, \ccchref{AN.5.31}{https://agama.buddhason.org/AN/an.php?keyword=5.31})解說,或說,見具足為看見具足的須陀洹(Diṭṭhisampannanti dassanasampannaṃ sotāpannaṃ, \ccchref{AN.9.20}{https://agama.buddhason.org/AN/an.php?keyword=9.20})。
\stopitemgroup

\startitemgroup[noteitems]
\item\subnoteref{573.0}\NoteKeywordAgamaHead{「光明/光澤(SA);極清淨(MA)」},南傳作\NoteKeywordNikaya{「極光淨的;輝耀的」}(pabhassara,另譯為「放光的;清淨的」),菩提比丘長老英譯為\NoteKeywordBhikkhuBodhi{「發光的;明亮的」}(Luminous),並解說,這個字在\ccchref{AN.3.103}{https://agama.buddhason.org/AN/an.php?keyword=3.103}中是形容入定的心,在\ccchref{AN.5.23}{https://agama.buddhason.org/AN/an.php?keyword=5.23}中是形容離五蓋的心,在\ccchref{MN.140}{https://agama.buddhason.org/MN/dm.php?keyword=140}中是形容第四禪的平靜,並質疑《滿足希求》在\ccchref{AN.1.49}{https://agama.buddhason.org/AN/an.php?keyword=1.49}中解說極光淨的心是指「有分心」(bhavaṅgacitta)。按:這個字在\suttaref{SN.46.33}中也形容離五蓋的心,在\ccchref{MN.50}{https://agama.buddhason.org/MN/dm.php?keyword=50}, \suttaref{SN.6.5}中佛陀被稱為「極光淨者」,在\suttaref{SN.51.22}中形容如來作身的幸福想與輕想時的身體,在\suttaref{SN.46.33}, \ccchref{MN.140}{https://agama.buddhason.org/MN/dm.php?keyword=140}, \ccchref{AN.5.23}{https://agama.buddhason.org/AN/an.php?keyword=5.23}, \ccchref{AN.3.102}{https://agama.buddhason.org/AN/an.php?keyword=3.102}-103中形容冶煉好的黃金,在\ccchref{MN.93}{https://agama.buddhason.org/MN/dm.php?keyword=93}, \ccchref{MN.96}{https://agama.buddhason.org/MN/dm.php?keyword=96}, \ccchref{MN.99}{https://agama.buddhason.org/MN/dm.php?keyword=99}中用來形容火焰。《舍利弗阿毘曇論》:「心性清淨,為客塵染,凡夫未聞故,不能如實知見,亦無修心,聖人聞故,如實知見,亦有修心。心性清淨,離客塵垢,凡夫未聞故,不能如實知見,亦無修心,聖人聞故,能如實知見,亦有修心。」」
\stopitemgroup

\startitemgroup[noteitems]
\item\subnoteref{574.0}\NoteSubKeyHead{(1)}\NoteKeywordAgamaHead{「內法(SA)」},南傳作\NoteKeywordNikaya{「內支」}(Ajjhattikaṃ…aṅganti),菩提比丘長老英譯為\NoteKeywordBhikkhuBodhi{「內部要素」}(internal factors)。
\item\subnoteref{574.1}\NoteSubKeyHead{(2)}\NoteKeywordAgamaHead{「外法(\ccchref{SA.717}{https://agama.buddhason.org/SA/dm.php?keyword=717})」},南傳作\NoteKeywordNikaya{「外支」}(Bāhiraṃ…aṅganti),菩提比丘長老英譯為\NoteKeywordBhikkhuBodhi{「外部要素」}(external factors)。按:「支」(aṅga),另譯為「部分;成分」,也作「肢體,關心,理由」。
\stopitemgroup

\startitemgroup[noteitems]
\item\subnoteref{575.0}\NoteSubKeyHead{(1)}\NoteKeywordAgamaHead{「見諦(SA);見諦人/成就正見[者](MA)」},南傳作\NoteKeywordNikaya{「見具足之人」}(diṭṭhisampanno puggalo,另譯為「見成就的人」),智髻比丘長老英譯為「持有正見的人」(a person possessing right view, MN)。
\item\subnoteref{575.1}\NoteSubKeyHead{(2)}\NoteKeywordNikayaHead{「見具足(者)」}(diṭṭhisampanno),菩提比丘長老英譯為\NoteKeywordBhikkhuBodhi{「見解已完成者」}(who is accomplished in view, SN)。按:《顯揚真義》以「具足道之見」(maggadiṭṭhiyā sampanno, \suttaref{SN.12.27}),《破斥猶豫》以「具足道之見的須陀洹聖弟子」(maggadiṭṭhiyā sampanno sotāpanno ariyasāvako, \ccchref{MN.115}{https://agama.buddhason.org/MN/dm.php?keyword=115})解說,《滿足希求》說,見具足為看見具足的須陀洹(Diṭṭhisampannanti dassanasampannaṃ sotāpannaṃ, \ccchref{AN.9.20}{https://agama.buddhason.org/AN/an.php?keyword=9.20})。
\stopitemgroup

\startitemgroup[noteitems]
\item\subnoteref{576.0}\NoteKeywordAgamaHead{「明分(SA)」},南傳作\NoteKeywordNikaya{「明的一部分」}(vijjābhāgiyā,另譯為「明分;有益於明的;與明連接的」),智髻比丘長老英譯為「參與真實理解」(partake of true knowledge, MN),菩提比丘長老英譯為\NoteKeywordBhikkhuBodhi{「屬於真實的理解」}(pertain to true knowledge, AN)。按:「明」的內容,i.《破斥猶豫》舉「毘婆舍那智[身體無常等]、意生[身]神通(manomayiddhi)、六證智[六通]」之八明[如\ccchref{MN.77}{https://agama.buddhason.org/MN/dm.php?keyword=77}所說]。ii.\ccchref{SA.1034}{https://agama.buddhason.org/SA/dm.php?keyword=1034}舉「無常想、無常苦想、苦無我想、觀食想、一切世間不可樂想、死想」為六明分想。iii.\ccchref{AN.2.32}{https://agama.buddhason.org/AN/an.php?keyword=2.32}指「止觀」。iv.\ccchref{AN.1.571}{https://agama.buddhason.org/AN/an.php?keyword=1.571}指「身至念」。
\stopitemgroup

\startitemgroup[noteitems]
\item\subnoteref{577.0}\NoteKeywordAgamaHead{「身證(SA/MA/AA);身證者(MA)」},南傳作\NoteKeywordNikaya{「身證者」}(kāyasakkhi),菩提比丘長老英譯為\NoteKeywordBhikkhuBodhi{「身體的見證者;親身體證者」}(a body-witness)。按:\ccchref{SA.936}{https://agama.buddhason.org/SA/dm.php?keyword=936}說「八解脫身作證具足住,而不見有漏斷」,\ccchref{MA.195}{https://agama.buddhason.org/MA/dm.php?keyword=195}類同,\ccchref{MN.70}{https://agama.buddhason.org/MN/dm.php?keyword=70}說身觸「無色寂靜解脫(無色界定)……某些漏被滅盡」,《破斥猶豫》等以「作證接觸(觸達)(Phuṭṭhantaṃ sacchikarotīti/Phuṭṭhantaṃ sacchi karotīti, \ccchref{MN.70}{https://agama.buddhason.org/MN/dm.php?keyword=70}/\ccchref{DN.28}{https://agama.buddhason.org/DN/dm.php?keyword=28}/\ccchref{AN.7.14}{https://agama.buddhason.org/AN/an.php?keyword=7.14})解說,並說,[身]先接觸禪定之接觸,後作證滅、涅槃(\ccchref{AN.2.49}{https://agama.buddhason.org/AN/an.php?keyword=2.49}亦同),有六種:起於證須陀洹果到立於阿羅漢道(arahattamaggaṭṭhā-向阿羅漢),而\ccchref{AA.27.10}{https://agama.buddhason.org/AA/dm.php?keyword=27.10}所說較鬆。
\stopitemgroup

\startitemgroup[noteitems]
\item\subnoteref{578.0}\NoteKeywordAgamaHead{「見到(SA/MA/AA);見到者(MA)」},南傳作\NoteKeywordNikaya{「達到見者」}(diṭṭhippatto),菩提比丘長老英譯為\NoteKeywordBhikkhuBodhi{「達到見者」}(one attained-to-view),玄奘法師譯為「見至」。按:\ccchref{SA.936}{https://agama.buddhason.org/SA/dm.php?keyword=936}說「不得八解脫身作證具足住,然,於正法律如實知見」,\ccchref{MA.195}{https://agama.buddhason.org/MA/dm.php?keyword=195}說「隨所聞法,便以慧增上觀、增上忍」,\ccchref{MN.70}{https://agama.buddhason.org/MN/dm.php?keyword=70}說「如來所宣說的法被慧深解、深察」,某些漏被滅盡而未入無色界定,《破斥猶豫》等以「『諸行是苦,滅為樂』被知道、被看見、被發現、被作證、被以慧觸達」(dukkhā saṅkhārā, sukho nirodhoti ñātaṃ hoti diṭṭhaṃ viditaṃ sacchikataṃ phusitaṃ paññāyāti, \ccchref{MN.70}{https://agama.buddhason.org/MN/dm.php?keyword=70}/\ccchref{DN.28}{https://agama.buddhason.org/DN/dm.php?keyword=28}/\ccchref{AN.7.14}{https://agama.buddhason.org/AN/an.php?keyword=7.14})解說,並說有六種如身證者(須陀洹果到向阿羅漢),而\ccchref{AA.27.10}{https://agama.buddhason.org/AA/dm.php?keyword=27.10}所說較鬆。
\stopitemgroup

\startitemgroup[noteitems]
\item\subnoteref{579.0}\NoteKeywordAgamaHead{「信解脫(SA/MA/AA);信解脫者(AA)」},南傳作\NoteKeywordNikaya{「信解脫者」}(saddhāvimutto),菩提比丘長老英譯為\NoteKeywordBhikkhuBodhi{「以信自由者;以信釋放者」}(one liberated-by-faith)。按:\ccchref{SA.936}{https://agama.buddhason.org/SA/dm.php?keyword=936}說「於正法律如實知見,不得見到」,\ccchref{MA.195}{https://agama.buddhason.org/MA/dm.php?keyword=195}說「隨所聞法,以慧觀忍,不如見到」,\ccchref{MN.70}{https://agama.buddhason.org/MN/dm.php?keyword=70}說「以慧看見後,某些漏被滅盡,他對如來的信已住立、已生根、已確立」而未入無色界定,《破斥猶豫》等說也有六種(Sopi vuttanayeneva chabbidho hoti, \ccchref{MN.70}{https://agama.buddhason.org/MN/dm.php?keyword=70}/\ccchref{DN.28}{https://agama.buddhason.org/DN/dm.php?keyword=28}/\ccchref{AN.7.14}{https://agama.buddhason.org/AN/an.php?keyword=7.14}),亦即須陀洹果到向阿羅漢。
\stopitemgroup

\startitemgroup[noteitems]
\item\subnoteref{580.0}\NoteKeywordNikayaHead{「無缺點的;無疑的」}(apaṇṇaka,另譯為「無錯誤的」),菩提比丘長老英譯為\NoteKeywordBhikkhuBodhi{「不被弄錯的」}(The Unmistaken, AN),或「無疑的」(surely, \ccchref{AN.10.46}{https://agama.buddhason.org/AN/an.php?keyword=10.46}),或「無疑的;無可爭議的」(incontrovertible, SN),智髻比丘長老英譯為「確實的;保險的」(assured, \ccchref{MN.82}{https://agama.buddhason.org/MN/dm.php?keyword=82}),或「理清」(cleared up, \ccchref{MN.127}{https://agama.buddhason.org/MN/dm.php?keyword=127}),或「無疑的;無可爭議的」(incontrovertible, \ccchref{MN.60}{https://agama.buddhason.org/MN/dm.php?keyword=60}),坦尼沙羅比丘長老英譯為「安全下注的」(safe-bet)。按:《顯揚真義》說,以無缺點狀態、無罪過狀態轉起(apaṇṇakatāya anaparādhakatāya eva saṃvattatīti, \suttaref{SN.42.13}),《破斥猶豫》說,不失敗的、導向真實的、一向(絕對)可取的(aviraddho advejjhagāmī ekaṁsagāhiko, \ccchref{MN.60}{https://agama.buddhason.org/MN/dm.php?keyword=60}),《滿足希求》說,不失敗的(avirādhitaṃ)、一向地(絕對地-ekaṃsena, \ccchref{AN.10.46}{https://agama.buddhason.org/AN/an.php?keyword=10.46})。
\stopitemgroup

\startitemgroup[noteitems]
\item\subnoteref{581.0}\NoteKeywordNikayaHead{「他的起源已開始」}(yoni cassa āraddhā hoti),菩提比丘長老英譯為\NoteKeywordBhikkhuBodhi{「安置了一個基礎」}(has laid the groundwork-AN, has laid a foundation-SN)。
\stopitemgroup

\startitemgroup[noteitems]
\item\subnoteref{582.0}\NoteSubKeyHead{(1)}\NoteKeywordAgamaHead{「戾語(SA);不善語恭順(MA)」},南傳作\NoteKeywordNikaya{「難順從糾正者」}(dubbaco,另譯為「難說;惡語;惡口」),菩提比丘長老英譯為\NoteKeywordBhikkhuBodhi{「難糾正」}(difficult to correct),或「難告誡」(difficult to admonish)。按:\ccchref{MN.15}{https://agama.buddhason.org/MN/dm.php?keyword=15}列有難順從糾正法16項,\ccchref{MA.89}{https://agama.buddhason.org/MA/dm.php?keyword=89}、\ccchref{SA.1139}{https://agama.buddhason.org/SA/dm.php?keyword=1139}所列也相近。
\item\subnoteref{582.1}\NoteSubKeyHead{(2)}\NoteKeywordAgamaHead{「善語/善語恭順(MA)」},南傳作\NoteKeywordNikaya{「易順從糾正者」}(suvaco,另譯為「善語;易說的;從順的」),菩提比丘長老英譯為\NoteKeywordBhikkhuBodhi{「容易糾正的」}(easy to correct)。
\stopitemgroup

\startitemgroup[noteitems]
\item\subnoteref{583.0}\NoteKeywordAgamaHead{「想(SA/MA);知(SA)」},南傳作\NoteKeywordNikaya{「認知」}(sañjānāti,動詞,另譯為「解了;認出;知覺;想念;意識到,名詞saññā-想」),菩提比丘長老英譯為\NoteKeywordBhikkhuBodhi{「覺知」}(perceive),I.B. Horner英譯為「注意到」(be aware)。
\stopitemgroup

\startitemgroup[noteitems]
\item\subnoteref{584.0}\NoteKeywordAgamaHead{「四道(SA/DA);四斷(MA);四事行跡(AA)」},南傳作\NoteKeywordNikaya{「四行道;四種行道」}(Catasso…paṭipadā,另譯為「四通行」),菩提比丘長老英譯為\NoteKeywordBhikkhuBodhi{「四種實行模式」}(four modes of practice,\ccchref{AN.4.162}{https://agama.buddhason.org/AN/an.php?keyword=4.162})。「行道」(paṭipadā),另譯為「道;行道;道跡」。
\stopitemgroup

\startitemgroup[noteitems]
\item\subnoteref{585.0}\NoteKeywordAgamaHead{「樂非盡道(SA);斷樂遲(MA);樂遲得/樂滅遲得(DA);樂行跡所行愚惑(AA)」},南傳作\NoteKeywordNikaya{「遲緩通達的樂行道」}(sukhā paṭipadā dandhābhiññā),菩提比丘長老英譯為\NoteKeywordBhikkhuBodhi{「快樂而遲鈍的直接理解之實行」}(practice that is pleasant with sluggish direct knowledge)。「行道」(paṭipadā),另譯為「道;行道;道跡」。
\stopitemgroup

\startitemgroup[noteitems]
\item\subnoteref{586.0}\NoteSubKeyHead{(1)}\NoteKeywordAgamaHead{「法利如法得利/法利如法得(MA);得淨利養/以法得養(DA);得法利之具/得法利之養(AA)」},南傳作\NoteKeywordNikaya{「如法所得的如法利得」}(lābhā dhammikā dhammaladdhā),菩提比丘長老英譯為\NoteKeywordBhikkhuBodhi{「正直地得到的正直獲得」}(righteous gains that have been righteously obtained, AN),智髻比丘長老英譯為「任何一種符合法的獲得」(any gain of a kind that accords with the Dhamma, MN)。。
\item\subnoteref{586.1}\NoteSubKeyHead{(2)}\NoteKeywordAgamaHead{「如法而得(SA);如法聚財(GA);應得利(MA)」},南傳作\NoteKeywordNikaya{「如法的如法所得」}(dhammikā dhammaladdhā),菩提比丘長老英譯為\NoteKeywordBhikkhuBodhi{「正當的財富正當地獲得」}(righteous wealth righteously gained)。「如法得(MA);如法所得」(dhammaladdhā)即「正當的獲得」。
\stopitemgroup

\startitemgroup[noteitems]
\item\subnoteref{587.0}\NoteSubKeyHead{(1)}\NoteKeywordNikayaHead{「生法」}(jātidhammaṃ),菩提比丘長老英譯為\NoteKeywordBhikkhuBodhi{「屬於生者」}(subject to birth)。
\item\subnoteref{587.1}\NoteSubKeyHead{(2)}\NoteKeywordAgamaHead{「老法(SA/MA);老之法(AA)」},南傳作\NoteKeywordNikaya{「老法」}(Jarādhammaṃ),菩提比丘長老英譯為\NoteKeywordBhikkhuBodhi{「屬於年老的」}(subject to old age)。
\item\subnoteref{587.2}\NoteSubKeyHead{(3)}\NoteKeywordNikayaHead{「病法」}(byādhidhammo),菩提比丘長老英譯為\NoteKeywordBhikkhuBodhi{「屬於生病的」}(subject to sickness)。
\item\subnoteref{587.3}\NoteSubKeyHead{(4)}\NoteKeywordNikayaHead{「死法」}(maraṇadhammo),菩提比丘長老英譯為\NoteKeywordBhikkhuBodhi{「屬於死亡」}(subject to death)。
\stopitemgroup

\startitemgroup[noteitems]
\item\subnoteref{588.0}\NoteKeywordAgamaHead{「超昇離生(SA)」},南傳作\NoteKeywordNikaya{「正性決定」}(sammattaniyāma, niyāmaṃ…sammattaṃ, sammattaniyato),菩提比丘長老英譯為\NoteKeywordBhikkhuBodhi{「正確性的固定進路」}(the fixed course of rightness)。按:《顯揚真義》以「聖道」(ariyamaggaṃ, \suttaref{SN.25.1})解說,《吉祥悅意》說,這是四聖道的名字(Catunnaṃ ariyamaggānametaṃ nāmaṃ, \ccchref{DN.33}{https://agama.buddhason.org/DN/dm.php?keyword=33})。
\stopitemgroup

\startitemgroup[noteitems]
\item\subnoteref{589.0}\NoteKeywordAgamaHead{「慈心;慈心解脫(MA)」},南傳作\NoteKeywordNikaya{「慈心解脫」}(mettācetovimutti, mettā cetovimutti),菩提比丘長老英譯為\NoteKeywordBhikkhuBodhi{「以慈愛而心釋放」}(the liberation of the mind by loving-kindness)。按:《顯揚真義》等說,當被說為「慈」時,為近行安止(upacāropi appanāpi-近行定),而當被說為「心解脫」時,成為安止(‘‘cetovimuttī’’ti vutte pana appanāva vaṭṭati, \suttaref{SN.42.8}/\suttaref{SN.46.51}/\ccchref{MN.99}{https://agama.buddhason.org/MN/dm.php?keyword=99}/\ccchref{DN.13}{https://agama.buddhason.org/DN/dm.php?keyword=13}),《滿足希求》說,這就是以慈到達安止之意(Idha appanāppattāva mettā adhippetā, \ccchref{AN.1.386}{https://agama.buddhason.org/AN/an.php?keyword=1.386})。
\stopitemgroup

\startitemgroup[noteitems]
\item\subnoteref{590.0}\NoteKeywordNikayaHead{「念與聰敏」}(satinepakkena),智髻比丘長老英譯為「深切注意與技能」(mindfulness and skill, MN),菩提比丘長老英譯為\NoteKeywordBhikkhuBodhi{「深切注意與機警」}(mindfulness and alertness, AN),或「深切注意與判斷」(mindfulness and discretion, SN),並表示,「記憶」為念(sati)的原始含意,可以看作過去式,而「注意力」則為現在式。按:《顯揚真義》說,明智的狀態為聰敏,這是慧之名(paññāyetaṃ nāmaṃ),而以念展現強力情況也被稱為慧,此即強的念,它只有與慧關連才強(\suttaref{SN.48.9}),《滿足希求》說,在這裡聰敏被稱為慧,它以念資助的狀態被拿起(ettha nepakkaṃ vuccati paññā, sā satiyā upakārakabhāvena gahitā, \ccchref{AN.5.14}{https://agama.buddhason.org/AN/an.php?keyword=5.14})。
\stopitemgroup

\startitemgroup[noteitems]
\item\subnoteref{591.0}\NoteKeywordAgamaHead{「有行所持(SA)」},南傳作\NoteKeywordNikaya{「進入被有行折伏後妨礙狀態的」}(sasaṅkhāraniggayhavāritagato, saṅkhāraniggayhavāritagato),菩提比丘長老英譯為\NoteKeywordBhikkhuBodhi{「被有力的抑制[污穢]所勒住與制止」}(is reined in and checked by forcefully suppressing [the defilements]),或依錫蘭本(sasaṅkhāraniggayhavāritavataṃ)英譯為「被有力的抑制所封鎖與制止」(blocked and checked by forceful suppression, \suttaref{SN.1.38})。按:《顯揚真義》等說,被有行、有努力(有加行)的污染折伏後成為妨礙(sasaṅkhārena sappayogena kilese niggahetvā vāritavataṃ, \suttaref{SN.1.38}/\ccchref{AN.3.102}{https://agama.buddhason.org/AN/an.php?keyword=3.102}/\ccchref{AN.9.37}{https://agama.buddhason.org/AN/an.php?keyword=9.37}),《滿足希求》說,如少德有漏定(Appaguṇasāsavasamādhi viya, \ccchref{DN.34}{https://agama.buddhason.org/DN/dm.php?keyword=34})被有行、有努力之心障礙法折伏、污染妨礙。
\stopitemgroup

\startitemgroup[noteitems]
\item\subnoteref{592.0}\NoteSubKeyHead{(1)}\NoteKeywordAgamaHead{「正慢無間等/止慢無間等/正無間等/慢無間等(SA)」},南傳作\NoteKeywordNikaya{「從慢的完全止滅」}(sammā mānābhisamayā,逐字譯為「正-慢+現觀(無間等)」),菩提比丘長老英譯為\NoteKeywordBhikkhuBodhi{「以完全地突破自大」}(by completely breaking through conceit)。按:「止慢無間等;正無間等」皆應為「正慢無間等」,此處的「無間等(現觀)」,水野弘元《巴利語辭典》標作abhisama-ya,解為「止滅;止息」,「現觀」則標作abhi-samaya,兩者不同。《滿足希求》以「依因依理由,九種慢的捨斷止滅」(hetunā kāraṇena navavidhassa mānassa pahānābhisamayā, \ccchref{AN.4.177}{https://agama.buddhason.org/AN/an.php?keyword=4.177})解說,《顯揚真義》等以「依因依理」(hetunā kāraṇena)解說sammā,以「慢的看見止滅與捨斷止滅」(mānassa dassanābhisamayā, pahānābhisamayā ca, \suttaref{SN.36.3}/\ccchref{MN.28}{https://agama.buddhason.org/MN/dm.php?keyword=28})解說mānābhisamayā。
\item\subnoteref{592.1}\NoteSubKeyHead{(2)}\NoteKeywordAgamaHead{「慢無間等(SA)」},南傳作\NoteKeywordNikaya{「以慢的止滅」}(mānābhisamayā),菩提比丘長老英譯為\NoteKeywordBhikkhuBodhi{「以突破自大」}(by breaking through conceit)。
\stopitemgroup

\startitemgroup[noteitems]
\item\subnoteref{593.0}\NoteKeywordAgamaHead{「世間飲食/世間信施食(MA)」},南傳作\NoteKeywordNikaya{「世間物質」}(lokāmisaṃ),智髻比丘長老英譯為「世間的物質物;世俗物質物」(the material things of the world, wordly material things, MN),菩提比丘長老英譯為\NoteKeywordBhikkhuBodhi{「世間的誘惑物」}(the world's bait, the bait of the world, SN)。按:《顯揚真義》說,隱喻的(pariyāyena)指三地(界)流轉的世間物質(tebhūmakavaṭṭaṃ lokāmisaṃ);非隱喻的(nippariyāyena)指四種(衣食住醫藥)必需品(paccayā, \suttaref{SN.1.3}),長老解說,物質-餌(āmisa,另譯為「食物, 味, 餌, 財物, 利益, 貪婪, 慾望」),此字若用在指六境,則比較是指餌鉤,而不是餌本身。
\stopitemgroup

\startitemgroup[noteitems]
\item\subnoteref{594.0}\NoteKeywordNikayaHead{「無流漏」}(anavassavo, anavassavāya),菩提比丘長老英譯為\NoteKeywordBhikkhuBodhi{「不出現」}(does not emergence, AN),智髻比丘長老英譯為「不噴出」(does not erupt, \ccchref{MN.104}{https://agama.buddhason.org/MN/dm.php?keyword=104}),或「無結果;無影響力」(no consequence, \ccchref{MN.14}{https://agama.buddhason.org/MN/dm.php?keyword=14})。
\stopitemgroup

\startitemgroup[noteitems]
\item\subnoteref{595.0}\NoteKeywordAgamaHead{「法無有吾我,亦復無我所,我既非當有,我所何由生(\ccchref{SA.64}{https://agama.buddhason.org/SA/dm.php?keyword=64});我者無我,亦無我所;當來無我,亦無我所(\ccchref{MA.6}{https://agama.buddhason.org/MA/dm.php?keyword=6});無我、無我所,我當不有、我所當不有(\ccchref{MA.75}{https://agama.buddhason.org/MA/dm.php?keyword=75})」},南傳作\NoteKeywordNikaya{「彼不會存在以及我的不會存在,彼將不存在以及我的將不存在」}(No cassa no ca me siyā, Na bhavissati na ca me bhavissati),菩提比丘長老英譯為\NoteKeywordBhikkhuBodhi{「那不會是,那不會是為我;那將不是,那將不是為我」}(It might not be, and it might not be for me; it will not be, [and] it will not be for me),並解說這是「證第三果聖者」不執著的心境,\suttaref{SN.22.55}錫蘭本作no c’ assa no ca me siyā, na bhavissati na me bhavissati。
\stopitemgroup

\startitemgroup[noteitems]
\item\subnoteref{596.0}\NoteKeywordAgamaHead{「攝事(SA/MA);事攝(MA);攝法(DA)」},南傳作\NoteKeywordNikaya{「攝事」}(saṅgahavatthūni),菩提比丘長老英譯為\NoteKeywordBhikkhuBodhi{「維持良好關係的方法」}(means of sustaining a favorable relationship)。按:《吉祥悅意》等以「攝集所作」(saṅgahakāraṇāni, \ccchref{DN.33}{https://agama.buddhason.org/DN/dm.php?keyword=33}/\ccchref{AN.4.32}{https://agama.buddhason.org/AN/an.php?keyword=4.32}/\ccchref{AN.8.24}{https://agama.buddhason.org/AN/an.php?keyword=8.24})解說。
\stopitemgroup

\startitemgroup[noteitems]
\item\subnoteref{597.0}\NoteSubKeyHead{(1)}\NoteKeywordAgamaHead{「觸相(SA);淨相(MA)」},南傳作\NoteKeywordNikaya{「淨相」}(subhanimittaṃ),菩提比丘長老英譯為\NoteKeywordBhikkhuBodhi{「有吸引力的標記」}(the mark of attractive, AN),或「美麗的特徵」(the sign of the beautiful, SN)。按:《破斥猶豫》以「能被貪的、令人想要的所緣」(rāgaṭṭhāniyaṃ iṭṭhārammaṇaṃ, \ccchref{MN.5}{https://agama.buddhason.org/MN/dm.php?keyword=5}),《滿足希求》以「能被貪的所緣」(rāgaṭṭhāniyaṃ ārammaṇaṃ, \ccchref{AN.1.11}{https://agama.buddhason.org/AN/an.php?keyword=1.11})解說,長老說,特別是能引起性慾的(arouses sexual desire)。
\item\subnoteref{597.1}\NoteSubKeyHead{(2)}\NoteKeywordAgamaHead{「惡露不淨(SA/AA);惡露/不淨惡露(MA);惡露(DA);惡露/惡露不淨(AA)」},南傳作\NoteKeywordNikaya{「不淨」}(asubhā),菩提比丘長老英譯為\NoteKeywordBhikkhuBodhi{「無吸引力」}(unattractiveness)。
\item\subnoteref{597.2}\NoteSubKeyHead{(3)}\NoteKeywordNikayaHead{「不淨相」}(asubhanimittaṃ),菩提比丘長老英譯為\NoteKeywordBhikkhuBodhi{「無吸引力的特徵」}(the mark of the unattractive)。按:《顯揚真義》以「腫脹的等十種分解(uddhumātakādibhedā dasa, \suttaref{SN.46.51})不淨所緣法」,《滿足希求》以「在十種不淨上所生起有所緣的初禪(sārammaṇaṃ paṭhamajjhānaṃ, \ccchref{AN.1.16}{https://agama.buddhason.org/AN/an.php?keyword=1.16})」解說。
\stopitemgroup

\startitemgroup[noteitems]
\item\subnoteref{598.0}\NoteKeywordNikayaHead{「受持後轉起」}(samādāya vattantīti),Maurice Walshe先生英譯為「依…進行」(proceed according to),菩提比丘長老英譯為\NoteKeywordBhikkhuBodhi{「採取並且遵循」}(undertake and follow)。
\stopitemgroup

\startitemgroup[noteitems]
\item\subnoteref{599.0}\NoteSubKeyHead{(1)}\NoteKeywordAgamaHead{「穢(MA);弊(AA)」},南傳作\NoteKeywordNikaya{「心荒蕪」}(cetokhilā),菩提比丘長老英譯為\NoteKeywordBhikkhuBodhi{「心理的貧瘠」}(mental barrenness),智髻比丘長老英譯為「心中的荒蕪」(wildernesses in the heart, MN)。按:《破斥猶豫》等以「心的剛愎狀態、塵埃狀態、殘株狀態」(cittassa thaddhabhāvā kacavarabhāvā khāṇukabhāvā, \ccchref{MN.16}{https://agama.buddhason.org/MN/dm.php?keyword=16}/\ccchref{DN.33}{https://agama.buddhason.org/DN/dm.php?keyword=33}/\ccchref{AN.5.205}{https://agama.buddhason.org/AN/an.php?keyword=5.205})解說。
\item\subnoteref{599.1}\NoteSubKeyHead{(2)}\NoteKeywordNikayaHead{「荒蕪」}(khila, khīlaṃ,另譯為「礙;頑固」),菩提比丘長老依錫蘭本(khilaṃ)英譯為「貧瘠」(barrenness, SN/AN),Maurice Walshe先生英譯為「心理妨礙」(mental blockages, DN)。
\stopitemgroup

\startitemgroup[noteitems]
\item\subnoteref{600.0}\NoteKeywordNikayaHead{「不免一死的人」}(macco,另譯為「人;人間」),菩提比丘長老英譯為\NoteKeywordBhikkhuBodhi{「難免一死者;凡人」}(a mortal, mortals)。
\stopitemgroup

\startitemgroup[noteitems]
\item\subnoteref{601.0}\NoteSubKeyHead{(1)}\NoteKeywordNikayaHead{「惡法者」}(pāpadhammo, pāpadhammassa),菩提比丘長老英譯為\NoteKeywordBhikkhuBodhi{「壞德者」}(of bad character),或「惡性者;邪惡品德者」(one of evil character)。按:《顯揚真義》等以「劣法者(劣性者)」(lāmakadhammo, \suttaref{SN.35.241}/\ccchref{AN.2.49}{https://agama.buddhason.org/AN/an.php?keyword=2.49})解說。
\item\subnoteref{601.1}\NoteSubKeyHead{(2)}\NoteKeywordNikayaHead{「善法者」}(kalyāṇadhammo),菩提比丘長老英譯為\NoteKeywordBhikkhuBodhi{「屬於好品格者」}(of good character)。
\stopitemgroup

\startitemgroup[noteitems]
\item\subnoteref{602.0}\NoteKeywordNikayaHead{「不淨的可疑行為者」}(asucisaṅkassarasamācāro),菩提比丘長老英譯為\NoteKeywordBhikkhuBodhi{「屬於不純的與懷疑行為者」}(of impure and suspect behaviour)。按:《顯揚真義》以「懷疑他應該做了什麼的行為者」(saṅkāya saritabbasamācāro),或「懷疑其他行為者[對他]做了什麼」(Saṅkāya vā paresaṃ samācāraṃ saratītipi, \suttaref{SN.35.241})解說「可疑行為者」。
\stopitemgroup

\startitemgroup[noteitems]
\item\subnoteref{603.0}\NoteSubKeyHead{(1)}\NoteKeywordAgamaHead{「樂食受(SA);樂食(MA);食樂痛(AA)」},南傳作\NoteKeywordNikaya{「肉體的樂受」}(sāmisaṃ vā sukhaṃ vedanaṃ),菩提比丘長老英譯為\NoteKeywordBhikkhuBodhi{「肉體的快樂感受」}(carnal pleasant feeling, SN),或「世俗的快樂感受」(worldly pleasant feeling, MN/AN),Maurice Walshe先生英譯為「肉慾的感受」(sensual feeling, DN)。按:「食」即「肉體的」(sāmisa,另譯為「有味的;有食味的;物質的;塗滿食物的;有物質的」),《滿足希求》以「與污染物質(肉體)相應的」(kilesāmisasampayuttā, \ccchref{AN.6.63}{https://agama.buddhason.org/AN/an.php?keyword=6.63}),或「污染、導向輪迴的樂」(saṃkilesaṃ vaṭṭagāmisukhaṃ, \ccchref{AN.2.69}{https://agama.buddhason.org/AN/an.php?keyword=2.69})解說。
\item\subnoteref{603.1}\NoteSubKeyHead{(2)}\NoteKeywordAgamaHead{「樂無食受(SA);樂無食(MA);不食樂痛(AA)」},南傳作\NoteKeywordNikaya{「精神的樂受」}(nirāmisāpi sukhā vedanā),菩提比丘長老英譯為\NoteKeywordBhikkhuBodhi{「精神上的快樂感受」}(spiritual pleasant feeling, SN/AN)。按:「無食;不食」即「精神的」(nirāmisa),另譯為「無食味的;無染污的;離財的;無肉的;無欲望的;無實質性的;離物質的;非物質的」。
\stopitemgroup

\startitemgroup[noteitems]
\item\subnoteref{604.0}\NoteKeywordAgamaHead{「超越、和合、出生/生則生,出則出,成則成(SA);生則生,出則出,成則成(MA)」},南傳作\NoteKeywordNikaya{「出生、進入[胎]、[生起、]生出」}(sañjāti okkanti [nibbatti] abhinibbatti),智髻比丘長老英譯為「他們的來到出生,墜下[進入子宮],產生」(their coming to birth, precipitation [into the womb], generation, MN),菩提比丘長老英譯為\NoteKeywordBhikkhuBodhi{「他們的被生,下降[進入子宮],生產」}(their being born, descent [into the womb], production, SN)。按:《顯揚真義》說生(jāti),那是與未完成處相應的(sā aparipuṇṇāyatanavasena yuttā),出生(sañjāti),那是與完成處相應的(sā paripuṇṇāyatanavasena yuttā),進入[胎](okkanti),那是與卵生、胎生者相應的(sā aṇḍajajalābujavasena yuttā),生出(abhinibbatti-生起;轉生;再生),那是與濕生、化生者相應的(sā saṃsedajaopapātikavasena yuttā, \suttaref{SN.12.2})。
\stopitemgroup

\startitemgroup[noteitems]
\item\subnoteref{605.0}\NoteKeywordNikayaHead{「如是住立的、如是定置的」}(yathāṭhitaṃ yathāpaṇihitaṃ),智髻比丘長老英譯為「無論被安置,無論被處置」(however it is placed, however disposed, MN)。
\stopitemgroup

\startitemgroup[noteitems]
\item\subnoteref{606.0}\NoteKeywordAgamaHead{「故作業(MA)」},南傳作\NoteKeywordNikaya{「故意的業」}(sañcetanikānaṃ kammānaṃ, Sañcetanikaṃ…kammaṃ),菩提比丘長老英譯為\NoteKeywordBhikkhuBodhi{「意志的業」}(volitional kammas, AN),智髻比丘長老英譯為「故意的行為」(an intentional action, MN)。按:《滿足希求》以「意圖、計畫後被作的」(cetetvā pakappetvā katānaṃ, \ccchref{AN.10.217}{https://agama.buddhason.org/AN/an.php?keyword=10.217})解說,長老說,思是業,意志是造業的必要因素,非故意行為仍造業為耆那教立場(\ccchref{AN.6.63}{https://agama.buddhason.org/AN/an.php?keyword=6.63}),認為「故意業」重複了,並將北傳的「不故作業」解讀為「沒作過去業」(one has not done a past kamma, \ccchref{AN.10.217}{https://agama.buddhason.org/AN/an.php?keyword=10.217}),但依北傳的經論顯然不能作如是解,如\ccchref{MA.171}{https://agama.buddhason.org/MA/dm.php?keyword=171}說:「若不故作業,作已成者,我不說必受報也。」《舍利弗阿毘曇論》說:「若業故作受報是名故作業……若業不作不受報是名不故作業……若業有報是名受業……若業無報是名非受業……若業生已滅是名過去業……若業未生未出是名未來業……若業生未滅是名現在業。何等業?思業、思已業;故作業、非故作業;受業、非受業……」「故作業」顯然不是指「過去業」。
\stopitemgroup

\startitemgroup[noteitems]
\item\subnoteref{607.0}\NoteKeywordNikayaHead{「捫摸」}, 南傳作\NoteKeywordNikaya{「執著後」}(abhinivissa),智髻比丘長老英譯為「黏著」(adheres, MN)。按:「捫摸」為重義詞,原意為「摸」,引申為「抓取;執著」。
\stopitemgroup

\startitemgroup[noteitems]
\item\subnoteref{608.0}\NoteKeywordAgamaHead{「彼一切本所作因(SA);悉是過去本所作因(GA);皆因宿命造/皆因本作(MA)」},南傳作\NoteKeywordNikaya{「那全部是過去所作之因」}(sabbaṃ taṃ pubbekatahetu),菩提比丘長老英譯為\NoteKeywordBhikkhuBodhi{「那全是過去所作引起的」}(all that is caused by what was done in the past, SN/MN/AN)。
\stopitemgroup

\startitemgroup[noteitems]
\item\subnoteref{609.0}\NoteKeywordAgamaHead{「信、欲、聞、行覺想/行思惟、見審諦忍/審諦堪忍(SA);信、樂、聞、念、見善觀(MA)」},南傳作\NoteKeywordNikaya{「信、[個人的]愛好、口傳、理由的遍尋思、見解的審慮接受」}(Saddhā, ruci, anussavo, ākāraparivitakko, diṭṭhinijjhānakkhanti),智髻比丘長老英譯為「信,贊成,古口傳的傳統,理論思考,和深思接受的見解」(faith, approval, oral tradition, reasoned cogitation, and reflective acceptance of a view, MN),菩提比丘長老英譯為\NoteKeywordBhikkhuBodhi{「信,個人的偏愛,口傳的傳統,理論深思,和沉思後接受之見解」}(faith, personal preference, oral tradition, reasoned reflection, the acceptance of a view after pondering it, SN)。按:「見解的審慮接受」(見審諦忍/審諦堪忍/見善觀),《顯揚真義》說,這是經由思惟生起惡執見(Kāraṇaṃ cintentassa pāpikā laddhi uppajjati, \suttaref{SN.35.153})。
\stopitemgroup

\startitemgroup[noteitems]
\item\subnoteref{610.0}\NoteKeywordAgamaHead{「不依眼界生貪欲識(SA);不依色而起於識(AA)」},南傳作\NoteKeywordNikaya{「我的識將不是依止眼的」}(na ca me cakkhunissitaṃ viññāṇaṃ bhavissatī’ti),智髻比丘長老英譯為「我的識將不是依憑眼」(and my consciousness will not be dependent on the eye, MN),菩提比丘長老解說,執取眼以欲與貪的方式發生,識是以渴愛與(邪)見依止眼,但由於給孤獨已經是入流者,對他來說,只涉及渴愛,(邪)見已經被入流道根除。
\stopitemgroup

\startitemgroup[noteitems]
\item\subnoteref{611.0}\NoteKeywordNikayaHead{「給與纏頭巾」}(sīsaveṭhanaṃ dadeyya),菩提比丘長老英譯為\NoteKeywordBhikkhuBodhi{「如頭巾般環繞我的頭勒緊」}(were to tighten…around my head as a headband)。
\stopitemgroup

\startitemgroup[noteitems]
\item\subnoteref{612.0}\NoteSubKeyHead{(1)}\NoteKeywordAgamaHead{「大家(GA/MA/DA/AA)」},南傳作\NoteKeywordNikaya{「主人」}(ayyo),菩提比丘長老英譯為\NoteKeywordBhikkhuBodhi{「主人」}(masters)。按:媳婦對公公也稱「大家」,如\ccchref{AA.30.3}{https://agama.buddhason.org/AA/dm.php?keyword=30.3}。
\item\subnoteref{612.1}\NoteSubKeyHead{(2)}\NoteKeywordNikayaHead{「紳士」}(ayyo,一般尊稱語),菩提比丘長老英譯為\NoteKeywordBhikkhuBodhi{「好男人」}(good man, \ccchref{AN.8.12}{https://agama.buddhason.org/AN/an.php?keyword=8.12}),並解說此字的原形為「聖者」(ariyo),但這裡顯然不是這個意思。
\item\subnoteref{612.2}\NoteSubKeyHead{(3)}\NoteKeywordNikayaHead{「聖;聖者」}(ayyo,原形ariya,敬語),菩提比丘長老英譯為\NoteKeywordBhikkhuBodhi{「大師」}(Master),或「值得尊敬的先生」(Venerable sir, \ccchref{MN.86}{https://agama.buddhason.org/MN/dm.php?keyword=86})。
\stopitemgroup

\startitemgroup[noteitems]
\item\subnoteref{613.0}\NoteKeywordNikayaHead{「親族腳子孫」}(bandhupādāpaccā,另譯為「梵天腳子孫」),智髻比丘長老英譯為「同族者的腳之子孫」(offspring of the Kinsman's feet, MN),菩提比丘長老英譯為\NoteKeywordBhikkhuBodhi{「上天的腳之子孫」}(offspring of Load’s feet, SN),Maurice Walshe先生英譯為「梵天的腳所生」(born of Brahmā's foot, DN)。按:《顯揚真義》等說,親族為梵天之意,腳子孫為從梵天的腳底(腳背)所生之意(brahmuno piṭṭhipādato jātāti adhippāyo),他們[婆羅門]傳說這個主張:婆羅門從梵天的口所生出(nikkhantā),剎帝利[王族]從胸,毘舍[平民族]從肚臍,首陀羅[奴隸族]從膝蓋,沙門從腳底(samaṇā piṭṭhipādato, \suttaref{SN.35.132}/\ccchref{MN.50}{https://agama.buddhason.org/MN/dm.php?keyword=50}/\ccchref{DN.3}{https://agama.buddhason.org/DN/dm.php?keyword=3})。
\stopitemgroup

\startitemgroup[noteitems]
\item\subnoteref{614.0}\NoteSubKeyHead{(1)}\NoteKeywordAgamaHead{「所依退減(SA);害(MA)」},南傳作\NoteKeywordNikaya{「近因被破壞的」}(hatūpaniso),菩提比丘長老英譯為\NoteKeywordBhikkhuBodhi{「缺乏其近因」}(lacks its proximate cause)。
\item\subnoteref{614.1}\NoteSubKeyHead{(2)}\NoteKeywordAgamaHead{「根本/所依(SA);有習/說習/因(MA)」},南傳作\NoteKeywordNikaya{「有近因的」}(sa-upaniso,另譯為「有助伴的;有依的」),菩提比丘長老英譯為\NoteKeywordBhikkhuBodhi{「有近因」}(has a proximate cause, \suttaref{SN.12.23}),或「跟隨著其支持」(with its supports, SN/MN),或「有支持條件」(having a supporting condition, AN)。按:《顯揚真義》等以「有理由、有緣」(sakāraṇaṃ sappaccayaṃ, \suttaref{SN.12.23}),或「有緣」(sappaccayaṃ, \suttaref{SN.45.28}/\ccchref{MN.117}{https://agama.buddhason.org/MN/dm.php?keyword=117}),《滿足希求》以「有近依、有緣」(saupanissayo sapaccayo, \ccchref{AN.3.68}{https://agama.buddhason.org/AN/an.php?keyword=3.68})解說。又,《佛說本相猗致經》譯作「有本」(T.1p.819)。
\stopitemgroup

\startitemgroup[noteitems]
\item\subnoteref{615.0}\NoteKeywordAgamaHead{「疾(SA)」},南傳作\NoteKeywordNikaya{「直接的」}(anantarā,另譯為「無中途的;無間的;迅速的」),菩提比丘長老英譯為\NoteKeywordBhikkhuBodhi{「直接的;即刻的」}(immediate),並解說,論師們的看法,「果」是接續「道」必然立即發生的,所以這位比丘真正要問的是如何「即刻與直接地」(swiftly and directly)證阿羅漢果,而不在任何較低階段的證悟耽擱(\suttaref{SN.22.81})。按:《摩訶僧祇律》作「次第」,含意顯然不同。《顯揚真義》說,有近的直接與遠的直接(āsannānantarañca dūrānantarañca)兩種直接(無間),前者指毘婆舍那道(Vipassanā maggassa),後者指果(phalassa, \suttaref{SN.22.55}),「諸漏的直接滅盡」,《顯揚真義》以「直接道(無間道)的阿羅漢果」(maggānantaraṃ arahattaphalaṃ, \suttaref{SN.22.81})解說,《滿足希求》以「直接生起阿羅漢狀態」(anantarāyeva arahattaṃ uppajjati, \ccchref{AN.5.170}{https://agama.buddhason.org/AN/an.php?keyword=5.170})解說。
\stopitemgroup

\startitemgroup[noteitems]
\item\subnoteref{616.0}\NoteKeywordAgamaHead{「我無、我不有(\ccchref{MA.215}{https://agama.buddhason.org/MA/dm.php?keyword=215})」},南傳作\NoteKeywordNikaya{「我不會存在以及我的不會存在,如果我不存在,我的將不存在」}(no cassaṃ no ca me siyā nābhavissaṃ na me bhavissatī),菩提比丘長老英譯為\NoteKeywordBhikkhuBodhi{「我不會是,那不會是為我;我必將不是,那必將不是為我」}(I might not be, and it might not be for me; I will not be, [and] it will not be for me),並解說這句以第一人稱未來式語態述說(assaṃ, \suttaref{SN.22.81}/\suttaref{SN.22.153}/\suttaref{SN.24.4}/\ccchref{AN.10.29}{https://agama.buddhason.org/AN/an.php?keyword=10.29}),是「斷滅論者」的觀點,表示認為「我死後就斷滅而將不存在了」,若以第三人稱述說(assa),則是「證第三果聖者」不執著的心境。
\stopitemgroup

\startitemgroup[noteitems]
\item\subnoteref{617.0}\NoteKeywordAgamaHead{「勝入處(SA);除處(MA);除入(DA)」},南傳作\NoteKeywordNikaya{「勝處」}(abhibhāyatanāni),菩提比丘長老英譯為\NoteKeywordBhikkhuBodhi{「征服(克服)的基礎」}(bases of overcoming, AN),或「掌握基地」(mastered bases, SN),或「超越的基礎」(bases of transcendence, MN)。按:《破斥猶豫》說,勝處是經由征服(abhibhavanakāraṇāni)敵對法(Paccanīkadhammepi)與所緣(ārammaṇānipi),它們以對治性(paṭipakkhabhāvena)征服敵對法,對人來說以智超越諸所緣(ñāṇuttaritāya ārammaṇāni),而「我知道,我看見」是「從等至出來後的思惟想(ābhogasaññā pana samāpattito vuṭṭhitasseva, \ccchref{MN.77}{https://agama.buddhason.org/MN/dm.php?keyword=77})」。
\stopitemgroup

\startitemgroup[noteitems]
\item\subnoteref{618.0}\NoteKeywordAgamaHead{「一切入處(SA);一切處(MA)」},南傳作\NoteKeywordNikaya{「遍處」}(kasiṇāyatanāni),菩提比丘長老英譯為\NoteKeywordBhikkhuBodhi{「kasiṇa基礎」}(kasiṇa bases, \ccchref{AN.10.25}{https://agama.buddhason.org/AN/an.php?keyword=10.25}),並解說,這是用作修定對象之代表元素或顏色的盤子(disks representing elements or colors used as objects of samādhi meditation),例如:地遍處是裝滿棕紅色粘土的盤子;水遍處是裝滿水的碗;色遍處是有顏色的盤子。默想者以實體性(physical)的盤子開始,當他能以心眼(mind's eys)清楚地看見遍處時,他拋棄實體性的盤子而只專注於心中的影像。當默想加深時,另一個叫作似相(paṭibhāganimitta)的影像浮現成為注意的錨(an anchor of attention),《清淨道論》4與5章提供遍處詳細的解說。在《清淨道論》體系中,空遍處(空無邊處的起頭)以限定虛空遍處(paricchinnākāsakasiṇa)替代,識遍處以光明遍處(āloka-kasiṇa)[替代]。
\stopitemgroup

\startitemgroup[noteitems]
\item\subnoteref{619.0}\NoteKeywordNikayaHead{「無二種的」}(advayaṃ),菩提比丘長老英譯為\NoteKeywordBhikkhuBodhi{「非雙重的」}(nondual, SN/AN),智髻比丘長老英譯為「未分開的;連續的」(undivided, MN)。按:此即單一成遍而無差別,《破斥猶豫》等說,四方、四方的中間方無二種(disāanudisāsu advayaṃ, \ccchref{MN.77}{https://agama.buddhason.org/MN/dm.php?keyword=77}),即對一種來說,其它狀態不靠近之義(ekassa aññabhāvānupagamanatthaṃ, \ccchref{MN.77}{https://agama.buddhason.org/MN/dm.php?keyword=77}/\ccchref{DN.33}{https://agama.buddhason.org/DN/dm.php?keyword=33})。
\stopitemgroup

\startitemgroup[noteitems]
\item\subnoteref{620.0}\NoteKeywordNikayaHead{「有神變的法」}(sappāṭihāriyaṃ dhammaṃ),智髻比丘長老英譯為「令人信服方式的法」(the Dhamma in a convincing manner, MN),菩提比丘長老英譯為\NoteKeywordBhikkhuBodhi{「有效法」}(the efficacious Dhamma, SN),或「解毒法;矯正法」(the antidotal Dhamma, \ccchref{AN.3.125}{https://agama.buddhason.org/AN/an.php?keyword=3.125}),並解說「神變」(pāṭihāriyaṃ)的動詞是「反擊」(paṭiharati),使役動詞是「擊退;避免」(paṭihāreti),相信兩者相關:神變反擊心的固定成見,並對不可思議的真實敞開,法反擊扭曲之見與污穢(distorted views and defilements),因此是矯正或解毒(counteractive or antidotal)。按:《顯揚真義》等以「直到作了出離後說法」(yāva niyyānikaṃ katvā dhammaṃ desessanti, \suttaref{SN.51.10}/\ccchref{DN.16}{https://agama.buddhason.org/DN/dm.php?keyword=16}/\ccchref{AN.8.70}{https://agama.buddhason.org/AN/an.php?keyword=8.70})解說,《破斥猶豫》以「這就是最古的同義語,有根據之意」(purimassevetaṃ vevacanaṃ, sakāraṇanti attho, \ccchref{MN.77}{https://agama.buddhason.org/MN/dm.php?keyword=77})
\stopitemgroup

\startitemgroup[noteitems]
\item\subnoteref{621.0}\NoteSubKeyHead{(1)}\NoteKeywordAgamaHead{「神足現化/神通變化(SA);神足變化(DA)」},南傳作\NoteKeywordNikaya{「神通神變」}(iddhipāṭihāriyaṃ),菩提比丘長老英譯為\NoteKeywordBhikkhuBodhi{「超常力量」}(spiritual power, SN)。
\item\subnoteref{621.1}\NoteSubKeyHead{(2)}\NoteKeywordAgamaHead{「神足變化示現、他心示現、教誡示現(SA);如意足示現,占念示現,教訓示現(MA);神足,觀察他心,教誡(DA);神足教化,言教教化,訓誨教化(AA)」},南傳作\NoteKeywordNikaya{「神通神變、記心神變、教誡神變」}(iddhipāṭihāriyaṃ, ādesanāpāṭihāriyaṃ anusāsanīpāṭihāriyaṃ),菩提比丘長老英譯為\NoteKeywordBhikkhuBodhi{「心靈能力的驚奇,讀心的驚奇,教導的驚奇」}(The wonder of psychic potency, the wonder of mind-reading, and the wonder of instruction)。「神變」(pāṭihāriya),另譯為「神通;示導;奇蹟」,\suttaref{SN.51.10}說:以如法善折伏對方已生起的議論,折伏後將教導有神變的法。《滿足希求》以「因對敵對的排除而為諸神變」(Pāṭihāriyānīti paccanīkapaṭiharaṇavasena pāṭihāriyāni, \ccchref{AN.3.61}{https://agama.buddhason.org/AN/an.php?keyword=3.61})解說,亦即令敵對者歸順信服的神奇改變,《清淨道論》說,以成功義、完成義、得到義為神通(ijjhanaṭṭhena iddhi, nipphattiatthena paṭilābhaṭṭhena cāti vuttaṃ hoti),……反擊[敵對者paṭipakkha-註疏]為神變(paṭiharatīti pāṭihāriyaṃ, 12.396)。
\stopitemgroup

\startitemgroup[noteitems]
\item\subnoteref{622.0}\NoteKeywordNikayaHead{「寺男」}(ārāmika),菩提比丘長老英譯為\NoteKeywordBhikkhuBodhi{「僧院隨從/僧院工作者」}(monastery attendants, MN/monastery workers, AN)。
\stopitemgroup

\startitemgroup[noteitems]
\item\subnoteref{623.0}\NoteKeywordNikayaHead{「抗論」}(vādapathaṃ,另譯作「語法;語路」),菩提比丘長老英譯為\NoteKeywordBhikkhuBodhi{「教義路線」}(pathways of doctrine, AN),或「教義進路」(courses of doctrine, MN)。按:《破斥猶豫》以「各個質問」(tassa tassa pañhassa, \ccchref{MN.77}{https://agama.buddhason.org/MN/dm.php?keyword=77}),《滿足希求》以「就是教義(理論)」(vādāyeva, \ccchref{AN.4.8}{https://agama.buddhason.org/AN/an.php?keyword=4.8})解說。
\stopitemgroup

\startitemgroup[noteitems]
\item\subnoteref{624.0}\NoteKeywordAgamaHead{「視岸鷹(MA)」},南傳作\NoteKeywordNikaya{「尋岸鳥」}(tīradassiṃ sakuṇaṃ,逐字譯為「岸+看-鳥」」),菩提比丘長老英譯為\NoteKeywordBhikkhuBodhi{「偵查陸地的鳥」}(a land-spotting bird),Maurice Walshe先生英譯為「察看陸地的鳥」(land-sighting bird)。
\stopitemgroup

\startitemgroup[noteitems]
\item\subnoteref{625.0}\NoteKeywordAgamaHead{「梵世法(MA)」},南傳作\NoteKeywordNikaya{「與梵天界共住的法」}(brahmalokasahabyatāya dhammaṃ),菩提比丘長老英譯為\NoteKeywordBhikkhuBodhi{「與梵天為同伴的法」}(Dhamma…for companionship with the brahmā world)。按:「與梵天界共住」指往生梵天。
\stopitemgroup

\startitemgroup[noteitems]
\item\subnoteref{626.0}\NoteKeywordAgamaHead{「於生地(MA);本所生處/本生處(AA)」},南傳作\NoteKeywordNikaya{「在出生地」}(jātibhūmiyaṃ),菩提比丘長老英譯為\NoteKeywordBhikkhuBodhi{「在他的家鄉地區」}(in his native district, AN),智髻比丘長老英譯為「故鄉」(native land, MN)。按:「在出生地」即「在當地出生的」。
\stopitemgroup

\startitemgroup[noteitems]
\item\subnoteref{627.0}\NoteKeywordAgamaHead{「識無量境界(MA);識無形,無量自有光(DA)」},南傳作\NoteKeywordNikaya{「識是不顯現的、無邊的、全面光明的」}(Viññāṇaṃ anidassanaṃ anantaṃ sabbato pabhaṃ),智髻比丘長老英譯為「識:非-顯現的、無邊的、全面發光的」(Consciousness non-manifesting, Boundless, luminous all-round, MN),菩提比丘長老說,《破斥猶豫》以涅槃解說識,但四部中並不見這樣的說法。依其理解,這裏所說的識並非涅槃本身,而是阿羅漢在定中經驗涅槃的識,而這種定中經驗不顯現世間有為法,所以或許可以真的描述為「非-顯現」(non-manifesting),而「全面光明的」則舉\ccchref{AN.1.49}{https://agama.buddhason.org/AN/an.php?keyword=1.49}「這個心是極光淨的」(Pabhassaramidaṃ cittaṃ)與\ccchref{AN.4.141}{https://agama.buddhason.org/AN/an.php?keyword=4.141}最高的「慧的光明」(paññābhā)作對比。Maurice Walshe先生英譯為「識是無形跡的、無邊的、全面發光的」(consciousness is signless, boundless, all-luminous, DN),也舉了\ccchref{AN.1.49}{https://agama.buddhason.org/AN/an.php?keyword=1.49}解說「識是全面發光的」。
\stopitemgroup

\startitemgroup[noteitems]
\item\subnoteref{628.0}\NoteSubKeyHead{(1)}\NoteKeywordAgamaHead{「一切行無常;一切諸行皆歸無常(AA)」},南傳作\NoteKeywordNikaya{「一切行是無常的」}(Sabbe saṅkhārā aniccā),菩提比丘長老英譯為\NoteKeywordBhikkhuBodhi{「所有條件所成的現象是不持久的」}(all conditioned phenomena are impermanent, AN),或「所有形成物是不持久的」(all formations are impermanent, SN/MN),並解說,這裡的「行」是指所有「所有條件所成;所建造;所組合者」(按:《顯揚真義》說「一切三界的諸行(sabbe tebhūmakasaṅkhārā, \suttaref{SN.22.90})」),與五蘊中的「行蘊」、十二緣起的「行(支)」、安般念中的「身行」、四如意足中的「斷行」含意均有所不同(\ccchref{MN.35}{https://agama.buddhason.org/MN/dm.php?keyword=35})。
\item\subnoteref{628.1}\NoteSubKeyHead{(2)}\NoteKeywordNikayaHead{「諸行無常」},南傳作\NoteKeywordNikaya{「諸行是無常的」}(Aniccā… saṅkhārā),菩提比丘長老英譯為\NoteKeywordBhikkhuBodhi{「諸形成物是不持久的」}(Impermanent……are formations, SN),或「條件所成的現象是不持久的」(conditioned phenomena are impermanent, AN)。
\stopitemgroup

\startitemgroup[noteitems]
\item\subnoteref{629.0}\NoteKeywordNikayaHead{「有眼者」}(cakkhumā, cakkhumatā),菩提比丘長老英譯為\NoteKeywordBhikkhuBodhi{「有眼光者」}(the one with vision),並解說「有眼者」是指佛陀,這樣稱呼,是因為佛陀具備「五眼」:佛眼(buddhacakkhu,包括「根之優劣狀態智-indriyaparopariyattañāṇa」與「所依與煩惱潛在趨勢智-āsayānusayañāṇa」)、一切眼(samantacakkhu)、法眼(dhammacakkhu)、天眼(dibbacakkhu)、肉眼(maṃsacakkhu, \suttaref{SN.6.1})。按:印順法師在《初期大乘佛教之起源與開展》中判定「五眼」的出現與《金剛經》、《中品般若》的集成年代相同,約西元50~150年。
\stopitemgroup

\startitemgroup[noteitems]
\item\subnoteref{630.0}\NoteKeywordAgamaHead{「式叉摩尼(SA)」},南傳作\NoteKeywordNikaya{「式叉摩那」}(sikkhamāna,另譯為「學法女;正學女;學戒女」),菩提比丘長老英譯為\NoteKeywordBhikkhuBodhi{「試用期間的女修道者」}(a probationary nun, SN),或「試用期間者」(a probationer, AN),智髻比丘長老英譯為「女試用期間者」(a female probationer, MN)。按:《破斥猶豫》等說,尊者鳩摩羅迦葉的母親不知懷孕而出家,在僧團中生他而引發議論(\ccchref{MN.23}{https://agama.buddhason.org/MN/dm.php?keyword=23}/\ccchref{DN.23}{https://agama.buddhason.org/DN/dm.php?keyword=23}),而有此項女眾成為比丘尼前須在比丘尼僧團等待二年的規制,期間須守五戒與過午不食。
\stopitemgroup

\startitemgroup[noteitems]
\item\subnoteref{631.0}\NoteKeywordNikayaHead{「因陀羅柱」}(indakhīlo,另譯為「王柱」),菩提比丘長老英譯為\NoteKeywordBhikkhuBodhi{「因陀羅柱」}(an Indra's pillar)。按:「因陀羅」(inda,梵語作Indra,另譯為「王」)即「釋提桓因」,「因陀羅柱」依PTS英巴辭典的解說,這是城門前的標竿、樁或圓柱(the post, stake or column),或屋子門前埋入地裡的大石頭(a large slab of stone)。
\stopitemgroup

\startitemgroup[noteitems]
\item\subnoteref{632.0}\NoteKeywordNikayaHead{「像這樣的;像這樣者;像這樣」}(tādino),菩提比丘長老英譯為\NoteKeywordBhikkhuBodhi{「穩固者」}(the Stable One, \suttaref{SN.7.2}),或「公平的」(impartial, \ccchref{Ni.5}{https://agama.buddhason.org/Ni/dm.php?keyword=5})。《小部/大義釋》說世尊與阿羅漢從五方面為像那樣者(tādī, \ccchref{Ni.5}{https://agama.buddhason.org/Ni/dm.php?keyword=5}/\ccchref{Ni.16}{https://agama.buddhason.org/Ni/dm.php?keyword=16})。
\stopitemgroup

\startitemgroup[noteitems]
\item\subnoteref{633.0}\NoteKeywordNikayaHead{「有阿賴耶的快樂,樂於阿賴耶,喜於阿賴耶」}(Ālayarāmā ālayaratā ālayasammuditā),菩提比丘長老英譯為\NoteKeywordBhikkhuBodhi{「在黏著中歡樂,在黏著中取歡樂,在黏著中歡喜」}(delights in adhesion, takes delight in adhesion, rejoices in adhesion)。《顯揚真義》分別以「以阿賴耶樂(動詞)、對阿賴耶已樂(過去分詞)、對阿賴耶的善喜(名詞)」(Tehi ālayehi ramantīti ālayarāmā. Ālayesu ratāti ālayaratā. Ālayesu suṭṭhu muditāti ālayasammuditā)解說。玄奘法師在《成唯識論》中譯為「愛阿賴耶、樂阿賴耶、欣阿賴耶、憙阿賴耶」。
\stopitemgroup

\startitemgroup[noteitems]
\item\subnoteref{634.0}\NoteSubKeyHead{(1)}\NoteKeywordAgamaHead{「法住(SA);住[法](DA)」},南傳作\NoteKeywordNikaya{「法安住性」}(dhammaṭṭhitatā,逐字譯為「法+住立性」),菩提比丘長老英譯為\NoteKeywordBhikkhuBodhi{「法的安定」}(the stableness of the Dhamma, SN),並解說此字與「法住智」的「法住」(dhammaṭṭhiti)似乎同義。按:《顯揚真義》說,以緣、緣的生起而法住立(Paccayena hi paccayuppannā dhammā tiṭṭhanti, \suttaref{SN.12.20}),因此,緣被稱為「法安住性」,《吉祥悅意》以「關於九出世間法(《法集》1464:四道不繫屬的、四沙門果與涅槃)的住立自性」(navalokuttaradhammesu ṭhitasabhāvaṃ, \ccchref{DN.9}{https://agama.buddhason.org/DN/dm.php?keyword=9}),《滿足希求》以「自性(實相)住立性」(sabhāvaṭṭhitatā, \ccchref{AN.3.137}{https://agama.buddhason.org/AN/an.php?keyword=3.137})解說。
\item\subnoteref{634.1}\NoteSubKeyHead{(2)}\NoteKeywordAgamaHead{「知法住(SA)」},南傳作\NoteKeywordNikaya{「法住智」}(dhammaṭṭhitiñāṇaṃ),菩提比丘長老英譯為\NoteKeywordBhikkhuBodhi{「法的安定性之理解」}(That knowledge of the stability of the Dhamma, SN)。按:《顯揚真義》以「關於緣模式之智」(paccayākāre ñāṇaṃ, \suttaref{SN.12.34}),或「首先生起的毘婆舍那智」(vipassanāñāṇaṃ, taṃ paṭhamataraṃ uppajjati, \suttaref{SN.12.70})解說,注疏以「毘婆舍那智」(vipassanāñāṇa’’nti)解說。長老說,這是「條件性法則之智,因為它是事象持續發生的原因」,即「『以生為緣而有老死』智」等,包含過去世與未來世。
\stopitemgroup

\startitemgroup[noteitems]
\item\subnoteref{635.0}\NoteKeywordAgamaHead{「法空(SA)」},南傳作\NoteKeywordNikaya{「法決定性」}(dhammaniyāmatā),菩提比丘長老英譯為\NoteKeywordBhikkhuBodhi{「法的固定進路」}(the fixed course of the Dhamma)。按:《顯揚真義》說,緣決定諸法(Paccayo dhamme niyameti, \suttaref{SN.12.20}),因此被稱為「法決定性」,《吉祥悅意》以「出世間法法決定性」(lokuttaradhammaniyāmataṃ, \ccchref{DN.9}{https://agama.buddhason.org/DN/dm.php?keyword=9}),《滿足希求》以「自性(實相)決定性」(sabhāvaniyāmatā, \ccchref{AN.3.137}{https://agama.buddhason.org/AN/an.php?keyword=3.137})解說。
\stopitemgroup

\startitemgroup[noteitems]
\item\subnoteref{636.0}\NoteSubKeyHead{(1)}\NoteKeywordAgamaHead{「法爾(SA)」},南傳作\NoteKeywordNikaya{「特定條件性」}(idappaccayatā,逐字譯為「此+緣+性」),菩提比丘長老英譯為\NoteKeywordBhikkhuBodhi{「特定條件性」}(specific conditionality)。《顯揚真義》等說,這些的緣為特定條件,特定條件即特定條件性(imesaṃ paccayā idappaccayā; idappaccayā eva idappaccayatā, \suttaref{SN.6.1}/\ccchref{MN.26}{https://agama.buddhason.org/MN/dm.php?keyword=26}/\ccchref{DN.14}{https://agama.buddhason.org/DN/dm.php?keyword=14}),\suttaref{SN.12.20}則作「老死等這些的緣為特定條件,……」。
\item\subnoteref{636.1}\NoteSubKeyHead{(2)}\NoteKeywordAgamaHead{「所因(MA);緣(MA/DA)」},南傳作\NoteKeywordNikaya{「特定條件(特定緣)」}(idappaccayā),Maurice Walshe先生英譯為「一個條件」(a condition, DN),菩提比丘長老英譯為\NoteKeywordBhikkhuBodhi{「一特定理由」}(a particular reason, SN),或「一特定條件」(a specific condition, AN)。
\stopitemgroup

\startitemgroup[noteitems]
\item\subnoteref{637.0}\NoteKeywordAgamaHead{「空相應(SA)」},南傳作\NoteKeywordNikaya{「空關聯的」}(suññatappaṭisaṃyuttā, suññatāpaṭisaṃyuttā),菩提比丘長老英譯為\NoteKeywordBhikkhuBodhi{「關於空」}(dealing with emptiness, SN/connected with emptiness, AN)。
\stopitemgroup

\startitemgroup[noteitems]
\item\subnoteref{638.0}\NoteKeywordAgamaHead{「種子村(MA);一切種子(GA)」},南傳作\NoteKeywordNikaya{「種子類」}(bījagāma,逐字譯為「種子+村」,A.P. Buddhadatta Mahāthera譯為「種子王國」(seed-kingdom),水野弘元《巴利語辭典》譯作「草」),智髻比丘長老英譯為「諸種子」(seeds, MN)。
\stopitemgroup

\startitemgroup[noteitems]
\item\subnoteref{639.0}\NoteKeywordAgamaHead{「本際(SA/MA)」},南傳作\NoteKeywordNikaya{「起始點」}(Pubbā koṭi,逐字譯為「前+點;最初+點」,另譯為「前際;本際」),菩提比丘長老英譯為\NoteKeywordBhikkhuBodhi{「第一點」}(A first point, SN/AN)。
\stopitemgroup

\startitemgroup[noteitems]
\item\subnoteref{640.0}\NoteKeywordNikayaHead{「善行相的」}(svākāre,另譯為「好性格的;好性情的」),菩提比丘長老英譯為\NoteKeywordBhikkhuBodhi{「具有良好的素質的」}(with good qualities)。
\stopitemgroup

\startitemgroup[noteitems]
\item\subnoteref{641.0}\NoteKeywordAgamaHead{「眾生居(SA/MA/DA);眾生居處(AA)」},南傳作\NoteKeywordNikaya{「眾生住處」}(sattāvāsā),菩提比丘長老英譯為\NoteKeywordBhikkhuBodhi{「生命的居住處」}(abodes of beings, SN/AN),並解說,這是指九眾生居住處。按:此即\ccchref{MA.97}{https://agama.buddhason.org/MA/dm.php?keyword=97}所說的「七識住及二處」,也就是「欲界」+四個「色界天」+四個「無色界天」。其中,第四禪的無想天,列在二處之一而不在七識住中,《滿足希求》等說,無想眾生在那裡不來到識狀態之類,不攝入七眾生住處中(Asaññasattā viññāṇābhāvā ettha saṅgahaṃ na gacchanti, sattāvāsesu gacchanti),[第四禪的]廣果天,就結合在第四識住(Vehapphalāpi catutthaviññāṇaṭṭhitimeva bhajanti)[第三禪的遍淨天],淨居天一切時不住於輪迴側(Suddhāvāsā vivaṭṭapakkhe ṭhitā na sabbakālikā, \ccchref{DN.15}{https://agama.buddhason.org/DN/dm.php?keyword=15}/\ccchref{AN.7.44}{https://agama.buddhason.org/AN/an.php?keyword=7.44}),不入眾生居中。
\stopitemgroup

\startitemgroup[noteitems]
\item\subnoteref{642.0}\NoteKeywordAgamaHead{「色觀色(MA/DA)」},南傳作\NoteKeywordNikaya{「有色者看見諸色」}(Rūpī rūpāni passati),菩提比丘長老英譯為\NoteKeywordBhikkhuBodhi{「持有色者看色」}(One possessing form sees forms AN),Maurice Walshe先生英譯為「持有色,他看色」(Possessing form, one sees forms, \ccchref{DN.15}{https://agama.buddhason.org/DN/dm.php?keyword=15})。按:《破斥猶豫》等以「在自己頭髮等上因青遍而生起色[界]禪定的色」(ajjhattakesādīsu nīlakasiṇādivasena uppāditaṃ rūpajjhānaṃ rūpaṃ, \ccchref{MN.77}{https://agama.buddhason.org/MN/dm.php?keyword=77}/\ccchref{DN.15}{https://agama.buddhason.org/DN/dm.php?keyword=15}/\ccchref{AN.1.435}{https://agama.buddhason.org/AN/an.php?keyword=1.435})解說。
\stopitemgroup

\startitemgroup[noteitems]
\item\subnoteref{643.0}\NoteKeywordAgamaHead{「內無色想外觀色(MA);內無色想觀外色(DA)」},南傳作\NoteKeywordNikaya{「內無色想者看見外諸色」}(Ajjhattaṃ arūpasaññī bahiddhā rūpāni passati),菩提比丘長老英譯為\NoteKeywordBhikkhuBodhi{「不覺知內在的色者看外在的色」}(One not percipient of forms internally sees forms externally, AN),Maurice Walshe先生英譯為「不覺知自己的物質類,他看外部的它們(物質類)」(Not perceiving material forms in oneself, one sees them outside, \ccchref{DN.15}{https://agama.buddhason.org/DN/dm.php?keyword=15})。按:《破斥猶豫》等以「以禪定之眼看見自身外部的青遍等色」(bahiddhāpi nīlakasiṇādīni rūpāni jhānacakkhunā passati., \ccchref{MN.77}{https://agama.buddhason.org/MN/dm.php?keyword=77}/\ccchref{DN.15}{https://agama.buddhason.org/DN/dm.php?keyword=15}/\ccchref{AN.1.435}{https://agama.buddhason.org/AN/an.php?keyword=1.435})解說「看見外諸色」。
\stopitemgroup

\startitemgroup[noteitems]
\item\subnoteref{644.0}\NoteKeywordAgamaHead{「陰馬藏(MA/DA)」},南傳作\NoteKeywordNikaya{「隱藏入鞘的陰部」}(kosohitaṃ vatthaguyhaṃ),智髻比丘長老英譯為「男性器官被包在鞘內」(the male organ being enclosed in a sheath, MN),Maurice Walshe先生英譯為「鞘內的生殖器」(sheathed genitals, DN)。按:《破斥猶豫》等以「被鞘包起來隱藏的」(vatthikosena paṭicchanne)解說kosohitaṃ,以「在男根處」(aṅgajāte)解說vatthaguyhaṃ(\ccchref{MN.91}{https://agama.buddhason.org/MN/dm.php?keyword=91}/\ccchref{DN.3}{https://agama.buddhason.org/DN/dm.php?keyword=3})。
\stopitemgroup

\startitemgroup[noteitems]
\item\subnoteref{645.0}\NoteKeywordAgamaHead{「在無明㲉(SA);無明卵之所裹/無明纏(MA)」},南傳作\NoteKeywordNikaya{「進入無明的」}(avijjāgato,另譯為「達到無明的;落入無明的」),菩提比丘長老英譯為\NoteKeywordBhikkhuBodhi{「無知的;愚昧的」}(ignorant, SN),或「陷入無知」(immersed in ignorance, AN)。
\stopitemgroup

\startitemgroup[noteitems]
\item\subnoteref{646.0}\NoteSubKeyHead{(1)}\NoteKeywordNikayaHead{「四大」}(cātummahābhūta),即「四大之種;四大種」,「大」為「大種」(mahābhūta)的簡略,菩提比丘長老英譯為\NoteKeywordBhikkhuBodhi{「大元素」}(great elements)。按:依\ccchref{MA.30}{https://agama.buddhason.org/MA/dm.php?keyword=30},四大指地(堅)水(潤)火(熱)風(動)。《大毘婆沙論》說,問:云何大義?云何種義?答:能減能增能損能益,體有起盡是為種義;體相形量遍諸方域,成大事業是為大義。
\item\subnoteref{646.1}\NoteSubKeyHead{(2)}\NoteKeywordAgamaHead{「四大造色(SA);四大造/四大造為色(MA);四大所造色(AA)」},南傳作\NoteKeywordNikaya{「四大之所造色」}(catunnañca mahābhūtānaṃ upādāya rūpaṃ),菩提比丘長老英譯為\NoteKeywordBhikkhuBodhi{「出自四大元素的色」}(the form derived from the four great elements)。按:「所造」(upādāya),另譯為「執取後;取著的」。
\stopitemgroup

\startitemgroup[noteitems]
\item\subnoteref{647.0}\NoteKeywordNikayaHead{「慢;慢類」}(vidhā,另譯為「種類」),菩提比丘長老英譯為\NoteKeywordBhikkhuBodhi{「歧視;區別」}(discrimination)。按:《顯揚真義》以「各種慢」(mānakoṭṭhāse, \suttaref{SN.18.22}),《滿足希求》以「等、勝、劣三種慢(tisso vidhā, \ccchref{AN.7.49}{https://agama.buddhason.org/AN/an.php?keyword=7.49})」解說,\suttaref{SN.1.20}, \suttaref{SN.45.162}, \ccchref{DN.33}{https://agama.buddhason.org/DN/dm.php?keyword=33}所述即為後者。
\stopitemgroup

\startitemgroup[noteitems]
\item\subnoteref{648.0}\NoteKeywordAgamaHead{「如法語者/善說法者(SA);如法說法者/法說(MA);法語/法說(DA)」},南傳作\NoteKeywordNikaya{「如法之說者;如法說的」}(dhammavādī),菩提比丘長老英譯為\NoteKeywordBhikkhuBodhi{「法的擁護者」}(a proponent of the Dhamma)。按:《破斥猶豫》等以「依止九出世間法後而說」(Navalokuttaradhammasannissitaṃ katvā vadatīti, \ccchref{MN.27}{https://agama.buddhason.org/MN/dm.php?keyword=27}/\ccchref{DN.1}{https://agama.buddhason.org/DN/dm.php?keyword=1}/\ccchref{AN.4.198}{https://agama.buddhason.org/AN/an.php?keyword=4.198})解說,而非實相之說者(asabhāvaṃ vattā, \ccchref{MN.41}{https://agama.buddhason.org/MN/dm.php?keyword=41});不說正法說非法(dhammaṃ na vadati, adhammaṃ vadati nāma, \ccchref{AN.3.70}{https://agama.buddhason.org/AN/an.php?keyword=3.70}/\ccchref{AN.4.22}{https://agama.buddhason.org/AN/an.php?keyword=4.22})為「不如法之說者」(adhammavādī)。
\stopitemgroup

\startitemgroup[noteitems]
\item\subnoteref{649.0}\NoteKeywordAgamaHead{「三苦:行苦、苦苦、變易苦(DA)」},南傳作\NoteKeywordNikaya{「三種苦性:苦苦性、行苦性、變易苦性」}(Tisso…dukkhatā. Dukkhadukkhatā, saṅkhāradukkhatā vipariṇāmadukkhatā),菩提比丘長老英譯為\NoteKeywordBhikkhuBodhi{「三種苦:由於痛苦的苦,由於形成的苦,由於改變的苦」}(three kinds of suffering: the suffering due to pain, the suffering due to formations, the suffering due to change, \suttaref{SN.38.14}),並解說「苦苦性」指身、心感受的痛苦。「行苦性」指三界中所有的有為法(all conditioned phenomena),因為他們都被生滅所壓迫。「變易苦性」指當樂受結束時帶來的苦。
\stopitemgroup

\startitemgroup[noteitems]
\item\subnoteref{650.0}\NoteKeywordAgamaHead{「無有是處(SA/DA);必無是處(MA);終無是處(MA/AA)」},南傳作\NoteSubEntryKey{(i)}\NoteKeywordNikaya{「這不存在可能性」}(netaṃ ṭhānaṃ vijjati),菩提比丘長老英譯為\NoteKeywordBhikkhuBodhi{「那是不可能的」}(that is impossible)。\NoteSubEntryKey{(ii)}\NoteKeywordNikaya{「這是無可能性、無機會」}(aṭṭhānametaṃ…anavakāso),智髻比丘長老英譯為「它是不可能的,它不可能是」(It is impossible, it cannot be, MN),菩提比丘長老英譯為\NoteKeywordBhikkhuBodhi{「它是不可能的和不可思議的」}(it is impossible and inconceivable, AN)。
\stopitemgroup

\startitemgroup[noteitems]
\item\subnoteref{651.0}\NoteKeywordAgamaHead{「斷關/無有關鍵(SA);無門(MA)」},南傳作\NoteKeywordNikaya{「無門閂者」}(niraggaḷo,另譯為「無門的;無障礙的;無遮的」),智髻比丘長老英譯為「無閂者;無遮欄者;無障礙者」(who has no bar, MN),菩提比丘長老英譯為\NoteKeywordBhikkhuBodhi{「無門閂者」}(a boltless one, AN)。按:此喻斷五下分結,《破斥猶豫》說,這個[下分結(Orambhāgiyānīti)]如百葉窗式城門關閉心後住立,被稱為門閂(Etāni hi kavāṭaṃ viya nagaradvāraṃ cittaṃ pidahitvā ṭhitattā aggaḷāti vuccanti, \ccchref{MN.22}{https://agama.buddhason.org/MN/dm.php?keyword=22}),《滿足希求》以「打開蓋的柵欄後住立」(nīvaraṇakavāṭaṃ ugghāṭetvā ṭhito, \ccchref{AN.5.71}{https://agama.buddhason.org/AN/an.php?keyword=5.71})解說。
\stopitemgroup

\startitemgroup[noteitems]
\item\subnoteref{652.0}\NoteKeywordAgamaHead{「度塹(SA/MA);平治城壍(SA)」},南傳作\NoteKeywordNikaya{「拔起門閂者」}(ukkhittapaligho,另譯為「除去障礙的」),智髻比丘長老英譯為「他的箭桿已拔起」(whose shaft has been lifted, MN),菩提比丘長老英譯為\NoteKeywordBhikkhuBodhi{「移除橫木者」}(who has removed the crossbar, AN)。按:此喻斷無明,《破斥猶豫》說,輪迴的根為無明(vaṭṭamūlikā avijjā),以這個難除去消失(Ayañhi durukkhipanaṭṭhena, \ccchref{MN.22}{https://agama.buddhason.org/MN/dm.php?keyword=22})被稱為障礙,《滿足希求》以「無明障礙拔起後、消除後住立」(avijjāpalighaṃ ukkhipitvā apanetvā ṭhito, \ccchref{AN.5.71}{https://agama.buddhason.org/AN/an.php?keyword=5.71})解說。 
\stopitemgroup

\startitemgroup[noteitems]
\item\subnoteref{653.0}\NoteKeywordAgamaHead{「超越境界/度諸嶮難(SA);破墎(MA)」},南傳作\NoteKeywordNikaya{「填滿溝渠者」}(saṃkiṇṇaparikkho, saṃkiṇṇaparikho, samākiṇṇaparikho),智髻比丘長老英譯為「他的溝渠已填滿」(whose trench has been filled, MN),菩提比丘長老英譯為\NoteKeywordBhikkhuBodhi{「填滿壕溝者」}([who has] filled in the moat, AN)。按:此喻斷生死輪迴,《破斥猶豫》說,那[生之輪迴等(Jātisaṃsārotiādīsu)]一再地生起作的事遍佈後住立(punappunaṃ uppattikaraṇavasena parikkhipitvā ṭhitattā, \ccchref{MN.22}{https://agama.buddhason.org/MN/dm.php?keyword=22})被稱為溝,《滿足希求》以「輪迴的溝鋪撒、使消失後住立」(saṃsāraparikhaṃ saṃkiritvā vināsetvā ṭhito, \ccchref{AN.5.71}{https://agama.buddhason.org/AN/an.php?keyword=5.71})解說。
\stopitemgroup

\startitemgroup[noteitems]
\item\subnoteref{654.0}\NoteKeywordAgamaHead{「脫諸防邏/解脫結縛(SA);過塹(MA)」},南傳作\NoteKeywordNikaya{「拔除柱子者」}(abbūḷhesiko),智髻比丘長老英譯為「他的柱子已根除」(whose pillar has been uprooted, MN),菩提比丘長老英譯為\NoteKeywordBhikkhuBodhi{「拉出標柱者」}([who has] pulled out the pillar, AN)。按:此喻斷渴愛,《破斥猶豫》說,輪迴的根為渴愛(vaṭṭamūlikā taṇhā),以這甚深隨行存續(Ayañhi gambhīrānugataṭṭhena, \ccchref{MN.22}{https://agama.buddhason.org/MN/dm.php?keyword=22})被稱為柱,《滿足希求》以「渴愛之柱拔出、拉出後住立」(taṇhāsaṅkhātaṃ esikāthambhaṃ abbuyha luñcitvā ṭhito, \ccchref{AN.5.71}{https://agama.buddhason.org/AN/an.php?keyword=5.71})解說。
\stopitemgroup

\startitemgroup[noteitems]
\item\subnoteref{655.0}\NoteKeywordAgamaHead{「建聖法幢/建立聖幢(SA);聖智慧鏡(MA)」},南傳作\NoteKeywordNikaya{「降下旗幟的、卸下負擔的、離結縛的聖者」}(ariyo pannaddhajo pannabhāro visaṃyutto),智髻比丘長老英譯為「他的旗幟已降下,負擔已放下,他不被束縛」(whose banner is lowered, whose burden is lowered, who is unfettered, MN),菩提比丘長老英譯為\NoteKeywordBhikkhuBodhi{「旗幟已降下的,負擔已落下的,已分離的聖者」}(a noble one with banner lowered, with burden dropped, detached, AN)。按:此喻斷我慢,《破斥猶豫》以「慢」解說「旗」,以「蘊、污染、造作、五種欲」解說「負擔」(khandhabhārakilesabhāraabhisaṅkhārabhārapañckāmaguṇabhārā, \ccchref{MN.22}{https://agama.buddhason.org/MN/dm.php?keyword=22}),以「已離四種軛與一切污染繫縛」解說「已分離」,《滿足希求》以「使慢之旗(mānaddhajañca)、蘊、造作、污染的負擔khandhābhisaṅkhārakilesabhārañca)落下、降落後住立」,及「已離輪迴」(vaṭṭena visaṃyutto, \ccchref{AN.5.71}{https://agama.buddhason.org/AN/an.php?keyword=5.71})解說。
\stopitemgroup

\startitemgroup[noteitems]
\item\subnoteref{656.0}\NoteKeywordAgamaHead{「邪盛大會/大會/邪盛/邪盛會(SA);大祀/祀/祠(GA);齋(MA);大施/大祀(DA);大祠/祠祀(AA)」},南傳作\NoteKeywordNikaya{「大牲祭;牲祭」}(mahāyañño, yaññaṃ),菩提比丘長老英譯為\NoteKeywordBhikkhuBodhi{「大獻祭;獻祭」}(great sacrifice, sacrifice)。
\stopitemgroup

\startitemgroup[noteitems]
\item\subnoteref{657.0}\NoteKeywordAgamaHead{「特牛(SA/DA);牛王(GA)」},南傳作\NoteKeywordNikaya{「公牛」}(usabha,另譯為「牛王」),菩提比丘長老英譯為\NoteKeywordBhikkhuBodhi{「公牛」}(bulls)。按:「牡」為公牛或雄性牲畜,「牸」為母牛或雌性牲畜,「犢」為幼牛或仔畜。
\stopitemgroup

\startitemgroup[noteitems]
\item\subnoteref{658.0}\NoteSubKeyHead{(1)}\NoteKeywordAgamaHead{「善修身(SA)」},南傳作\NoteKeywordNikaya{「已自我修習」}(bhāvitatto),菩提比丘長老英譯為\NoteKeywordBhikkhuBodhi{「內在已開發」}(inwardly developed, \suttaref{SN.1.50}),或「被開發的心」(of developed mind, \suttaref{SN.2.14})。按:《顯揚真義》以「修習自我後、使增大後已住立」(attānaṃ bhāvetvā vaḍḍhetvā ṭhito, \suttaref{SN.6.3})解說,PTS英巴辭典解為「一個人的自我己修習,即:被良好訓練或組織」(one whose attan (ātman) is bhāvita, i.e. well trained or composed),這裡的「自我」等於「心」(Attan here = citta)。
\item\subnoteref{658.1}\NoteSubKeyHead{(2)}\NoteKeywordNikayaHead{「已自我多作」}(bahulīkatattā),菩提比丘長老英譯為\NoteKeywordBhikkhuBodhi{「鍛鍊」}(cultivated),同「多作(多修習)」(bahulīkata)。
\stopitemgroup

\startitemgroup[noteitems]
\item\subnoteref{659.0}\NoteKeywordAgamaHead{「八法(SA);世八法(SA/DA/AA);世間八法(AA)」},南傳作\NoteKeywordNikaya{「八世間法」}(aṭṭha lokadhammā),菩提比丘長老英譯為\NoteKeywordBhikkhuBodhi{「八種世間情況」}(eight worldly conditions)。按:其內容即「利、衰、毀、譽、稱、譏、苦、樂」,參看\ccchref{AA.43.8}{https://agama.buddhason.org/AA/dm.php?keyword=43.8}、\ccchref{DA.10}{https://agama.buddhason.org/DA/dm.php?keyword=10}。
\stopitemgroup

\startitemgroup[noteitems]
\item\subnoteref{660.0}\NoteKeywordNikayaHead{「世尊是知道者,他知道,是看見者,他看見」}(bhagavā jānaṃ jānāti, passaṃ passati),菩提比丘長老英譯為\NoteKeywordBhikkhuBodhi{「知道的,世尊知道;看得見的,他看見」}(knowing, the Blessed One knows; seeing, he sees),或「確實是真的知道者,真的看見者」(is surely one who really knows, who really sees, \suttaref{SN.56.39}),或「確實知道與看見」(surely knows and sees, \ccchref{AN.8.2}{https://agama.buddhason.org/AN/an.php?keyword=8.2})。按:《顯揚真義》等以「知道應該被知道的(jānitabbakaṃ jānāti);看見應該被看見的(passitabbakaṃ passati, \suttaref{SN.35.116}/\ccchref{MN.18}{https://agama.buddhason.org/MN/dm.php?keyword=18}/\ccchref{AN.8.2}{https://agama.buddhason.org/AN/an.php?keyword=8.2}/\ccchref{AN.10.115}{https://agama.buddhason.org/AN/an.php?keyword=10.115})」解說。
\stopitemgroup

\startitemgroup[noteitems]
\item\subnoteref{661.0}\NoteKeywordAgamaHead{「得(此)身/受身(SA)」},南傳作\NoteKeywordNikaya{「個體的獲得」}(attabhāvapaṭilābho),菩提比丘長老英譯為\NoteKeywordBhikkhuBodhi{「個體存在」}(individual existence)。按:四種個體的獲得,《吉祥悅意》說,第一種以戲樂過失之因(khiḍḍāpadosikavasena),第二種以屠羊者等殺害羊等因,第三種以意亂之因(manopadosikāvasena),第四種關於四大王源於在殘餘天的狀態上之因(cātumahārājike upādāya uparisesadevatāvasena),因為那些天神既非以自己的思也非其他的思死去(Te hi devā neva attasañcetanāya maranti, na parasañcetanāya, \ccchref{DN.33}{https://agama.buddhason.org/DN/dm.php?keyword=33})。
\stopitemgroup

\startitemgroup[noteitems]
\item\subnoteref{662.0}\NoteSubKeyHead{(1)}\NoteKeywordAgamaHead{「日種尊/日種姓尊/日種胤/日種苗胤(SA);如日親友(GA);日親/日之親(MA)」},南傳作\NoteKeywordNikaya{「太陽族人」}(ādiccabandhu,另譯為「日種;日種族」)。
\item\subnoteref{662.1}\NoteSubKeyHead{(2)}\NoteKeywordAgamaHead{「日種姓尊說(SA);日親之所說(MA)」},南傳作\NoteKeywordNikaya{「被太陽族人教導」}(desitādiccabandhunā,逐字譯為「已被說示+日(太陽)+親族」),菩提比丘長老英譯為\NoteKeywordBhikkhuBodhi{「太陽族者這樣解說」}(So explained the Kinsman of the Sun)。
\item\subnoteref{662.2}\NoteSubKeyHead{(3)}\NoteKeywordNikayaHead{「放光者」}(Aṅgīrasaṃ),菩提比丘長老英譯照錄原文(AN),或「放光者」(the Radiant One, SN),並解說這與佛陀屬太陽族(日種姓)的傳說有關。按:《滿足希求》說,世尊的所有肢體散發光線(aṅgamaṅgehi rasmiyo niccharanti, \ccchref{AN.5.195}{https://agama.buddhason.org/AN/an.php?keyword=5.195}),因此被稱為「放光者」。
\stopitemgroup

\startitemgroup[noteitems]
\item\subnoteref{663.0}\NoteKeywordAgamaHead{「棟(SA);高波那寫(GA);椽(MA)」},南傳作\NoteKeywordNikaya{「椽;椽木」}(gopānasī),菩提比丘長老英譯為\NoteKeywordBhikkhuBodhi{「椽」}(rafter)。按:「椽」,為木造屋中作為屋頂支撐屋瓦或茅草等覆蓋物的木條。
\stopitemgroup

\startitemgroup[noteitems]
\item\subnoteref{664.0}\NoteKeywordAgamaHead{「薩羅(SA);池水/淵(GA)」},南傳作\NoteKeywordNikaya{「溪流;池湖」}(sarā),菩提比丘長老英譯為\NoteKeywordBhikkhuBodhi{「溪流」}(the streams)。按:「薩羅」顯然是「溪流;池湖」(sara)的音譯,該字尚有:池、湖;流動的等多種意思。
\stopitemgroup

\startitemgroup[noteitems]
\item\subnoteref{665.0}\NoteKeywordAgamaHead{「白練(MA)」},南傳作\NoteKeywordNikaya{「迦尸衣」}(kāsikavatthaṃ),菩提比丘長老英譯為\NoteKeywordBhikkhuBodhi{「迦尸衣」}(Kāsian cloth)。按:「練」指煮熟變白的絲或帛織品,「白練」即白色絹布。「迦尸衣」指迦尸國所出產的布料與衣服,是佛陀時代質料最好的。
\stopitemgroup

\startitemgroup[noteitems]
\item\subnoteref{666.0}\NoteKeywordAgamaHead{「於結所繫法(SA)」},南傳作\NoteKeywordNikaya{「在會被結縛的諸法上」}(saṃyojaniyesu…dhammesu,另譯為「順結法;助於結之法」),菩提比丘長老英譯為\NoteKeywordBhikkhuBodhi{「能加諸束縛的事」}(things that can fetter)。按:這裡的「法」(dhamma),不是指「正法」,而是指「事情;東西」。
\stopitemgroup

\startitemgroup[noteitems]
\item\subnoteref{667.0}\NoteKeywordNikayaHead{「那個食物、那個燃料」}(tadāhāro tadupādāno),菩提比丘長老英譯為\NoteKeywordBhikkhuBodhi{「以那個材料支撐,以它補給燃料」}(sustained by that material, fuelled by it, \suttaref{SN.12.52}),或「以那油支撐,以它補給燃料」(sustained by that oil, fuelled by it, \suttaref{SN.12.53})。按:《顯揚真義》以「該緣(彼緣)」(taṃpaccayo, \suttaref{SN.12.52})解說。
\stopitemgroup

\startitemgroup[noteitems]
\item\subnoteref{668.0}\NoteKeywordAgamaHead{「何觸/何轉(SA);以何為本/由何而有/由何有(MA);誰為原首(DA)」},南傳作\NoteKeywordNikaya{「什麼為根源」}(kiṃpabhavaṃ),菩提比丘長老英譯為\NoteKeywordBhikkhuBodhi{「它從什麼而生產」}(from what is it produced)。按:在雜阿含經中有3經作「何因?何集?何生?何觸?」有4經作「何因?何集?何生?何轉?」而其對應的南傳經文均為「什麼因?什麼集?什麼生?什麼根源」,推斷「何觸」即「何轉」的另譯。
\stopitemgroup

\startitemgroup[noteitems]
\item\subnoteref{669.0}\NoteKeywordNikayaHead{「惱害想的熟知者」}(Vihiṃsasaññī paguṇaṃ),菩提比丘長老英譯為\NoteKeywordBhikkhuBodhi{「預見麻煩」}(Foreseeing trouble, SN),智髻比丘長老英譯為「認為會是麻煩」(Thinking it would be troublesome, MN),Maurice Walshe先生英譯為「因為怕麻煩」(For fear of trouble, DN)。
\stopitemgroup

\startitemgroup[noteitems]
\item\subnoteref{670.0}\NoteSubKeyHead{(1)}\NoteKeywordAgamaHead{「喜行(SA);喜觀/觀(色)喜住(MA);[喜]察行(DA)」},南傳作\NoteKeywordNikaya{「喜悅近伺察」}(somanassūpavicārā,另譯為「喜悅近行」),智髻比丘長老英譯為為「喜悅的探索」(exploration with joy, MN),Maurice Walshe先生英譯為「令人快樂的調查」(pleasurable investigations, DN)。按:《吉祥悅意》以「喜悅相應的伺察」(somanassasampayuttā vicārā, \ccchref{DN.33}{https://agama.buddhason.org/DN/dm.php?keyword=33})解說。
\item\subnoteref{670.1}\NoteSubKeyHead{(2)}\NoteKeywordAgamaHead{「觀…喜住(MA)」},南傳作\NoteKeywordNikaya{「順喜悅處近伺察…」}(somanassaṭṭhāniyaṃ…upavicarati),菩提比丘長老英譯為\NoteKeywordBhikkhuBodhi{「一個人檢查喜悅基礎的…」}(one examines…that is a basis for joy, AN),Maurice Walshe先生英譯為「一個人調查產生快樂的相關目標」(one investigates a corresponding object productive of pleasure, DN)。按:《吉祥悅意》以「以尋尋思後以伺察決定(分別)」(vitakkena vitakketvā vicārena paricchindati, \ccchref{DN.33}{https://agama.buddhason.org/DN/dm.php?keyword=33})解說「順」。
\stopitemgroup

\startitemgroup[noteitems]
\item\subnoteref{671.0}\NoteSubKeyHead{(1)}\NoteKeywordAgamaHead{「憂行(SA);憂觀/觀(憂)喜住(MA);[憂]察行(DA)」},南傳作\NoteKeywordNikaya{「憂近伺察」}(domanassūpavicārā,另譯為「憂戚近行」),智髻比丘長老英譯為「苦惱的探索」(exploration with grief, MN),Maurice Walshe先生英譯為「令人不快樂的調查」(unpleasurable investigations)。
\item\subnoteref{671.1}\NoteSubKeyHead{(2)}\NoteKeywordAgamaHead{「觀…憂住(MA)」},南傳作\NoteKeywordNikaya{「順憂處近伺察…」}(domanassaṭṭhāniyaṃ…upavicarati),菩提比丘長老英譯為\NoteKeywordBhikkhuBodhi{「一個人檢查憂鬱基礎的…」}(one examines…that is a basis for dejection, AN),Maurice Walshe先生英譯為「一個人調查產生不快樂的相關目標」(one investigates a corresponding object productive of displeasure, DN)。
\stopitemgroup

\startitemgroup[noteitems]
\item\subnoteref{672.0}\NoteSubKeyHead{(1)}\NoteKeywordAgamaHead{「捨行(SA);捨觀/觀(捨)喜住(MA)(MA);[捨]察行(DA)」},南傳作\NoteKeywordNikaya{「平靜近伺察」}(upekkhūpavicārā,另譯為「捨近伺察;平靜近行」),智髻比丘長老英譯為「平靜的探索」(exploration with equanimity, MN),Maurice Walshe先生英譯為「不偏袒的調查」(indifferent investigations, DN)。
\item\subnoteref{672.1}\NoteSubKeyHead{(2)}\NoteKeywordAgamaHead{「觀…捨住(MA)」},南傳作\NoteKeywordNikaya{「順平靜處近伺察…」}(upekkhāṭṭhāniyaṃ…upavicarati),菩提比丘長老英譯為\NoteKeywordBhikkhuBodhi{「一個人檢查平靜基礎的…」}(one examines…that is a basis for equanimity, AN),Maurice Walshe先生英譯為「一個人調查產生不偏袒的相關目標」(one investigates a corresponding object productive of indifferent, DN)。
\stopitemgroup

\startitemgroup[noteitems]
\item\subnoteref{673.0}\NoteKeywordAgamaHead{「入於名色(SA)」},南傳作\NoteKeywordNikaya{「有名色的下生(名色的下生存在)」}(nāmarūpassa avakkanti hoti),菩提比丘長老英譯為\NoteKeywordBhikkhuBodhi{「有名-與-色的下降」}(there is a descent of name-and-form),並指出這句與\suttaref{SN.12.12}的「未來再生的出生」通用(interchangeable, \suttaref{SN.12.39})。按:中阿含作作「便生母胎(\ccchref{MA.13}{https://agama.buddhason.org/MA/dm.php?keyword=13});受胎(\ccchref{MA.151}{https://agama.buddhason.org/MA/dm.php?keyword=151});入於母胎(\ccchref{MA.201}{https://agama.buddhason.org/MA/dm.php?keyword=201})」,中部等作「胎的下生」(\ccchref{MN.38}{https://agama.buddhason.org/MN/dm.php?keyword=38}, \ccchref{MN.93}{https://agama.buddhason.org/MN/dm.php?keyword=93}, \ccchref{AN.3.62}{https://agama.buddhason.org/AN/an.php?keyword=3.62}),《顯揚真義》說,[這是]在識、名色中間有一個連結(viññāṇanāmarūpānaṃ antare eko sandhi, \suttaref{SN.12.39}),長老解說,「連結」指「識表示前生的業積聚識(the kammically generative consciousness of the previous existence)[abhisaṅkhāraviññāṇa],名色為此生的開始」,而在我(長老)看來,作為個人連續的原則,更像識跨著前後兩生(viññāṇa straddles both the past life and the present life)。
\stopitemgroup

\startitemgroup[noteitems]
\item\subnoteref{674.0}\NoteSubKeyHead{(1)}\NoteKeywordAgamaHead{「合[心](MA);無散落心/無散心(AA)」},南傳作\NoteKeywordNikaya{「收斂的心」}(saṃkhittaṃ citta, saṅkhittaṃ citta),菩提比丘長老英譯為\NoteKeywordBhikkhuBodhi{「收縮的心;簡約的心」}(contracted mind)。按:「收斂」(saṃkhittaṃ, saṅkhitta)為動詞saṅkhipati的過去分詞,saṅkhipati有兩層意思:一是「簡略;簡約」(原意),一是「集中;統一」(引申),《破斥猶豫》以「已掉入惛沈睡眠的」(thinamiddhānupatitaṃ, \ccchref{MN.10}{https://agama.buddhason.org/MN/dm.php?keyword=10})解說。
\item\subnoteref{674.1}\NoteSubKeyHead{(2)}\NoteKeywordNikayaHead{「自身內收斂的心」}(ajjhattaṃ…saṃkhittaṃ cittan),菩提比丘長老英譯為\NoteKeywordBhikkhuBodhi{「心內縮」}(mind is constricted internally)。《破斥猶豫》以「已隨順惛沈睡眠的」(nāma thinamiddhānugataṃ, \ccchref{AN.7.38}{https://agama.buddhason.org/AN/an.php?keyword=7.38})解說。
\stopitemgroup

\startitemgroup[noteitems]
\item\subnoteref{675.0}\NoteKeywordAgamaHead{「第一之說(SA)」},南傳作\NoteKeywordNikaya{「如牛王之語」}(āsabhī vācā),菩提比丘長老英譯為\NoteKeywordBhikkhuBodhi{「像公牛一樣吼的話」}(bellowing utterance, SN),Maurice Walshe先生英譯為「公牛的聲音」(a bull's voice, DN)。按:「如牛王」(āsabhī, āsabhin),另譯為「偉大的;莊重的」,在古印度,牛被認為是神聖的,《顯揚真義》等以「等同公牛之語不搖動、不顫抖」(usabhassa vācāsadisī acalā asampavedhī, \suttaref{SN.47.12}/\ccchref{DN.28}{https://agama.buddhason.org/DN/dm.php?keyword=28})解說。
\stopitemgroup

\startitemgroup[noteitems]
\item\subnoteref{676.0}\NoteKeywordAgamaHead{「三妙行(SA/MA)」},南傳作\NoteKeywordNikaya{「三善行」}(tiṇṇannaṃ sucaritānaṃ),菩提比丘長老英譯為\NoteKeywordBhikkhuBodhi{「三種好的行為」}(three kinds of good conduct)。依\ccchref{SA.281}{https://agama.buddhason.org/SA/dm.php?keyword=281},「三妙行」即身、口、意三善行:離殺生、偷盜、邪婬、妄言、惡口、兩舌、綺語、貪、恚、邪見,也就是「十善業跡」。
\stopitemgroup

\startitemgroup[noteitems]
\item\subnoteref{677.0}\NoteKeywordNikayaHead{「業為根源者」}(kammayonī,逐字譯為「業+胎者」),菩提比丘長老英譯為\NoteKeywordBhikkhuBodhi{「業為我/他們的起源」}(kamma as my/their origin, AN),智髻比丘長老英譯為「它們來自他們的行為」(they originate from their actions, MN)。按:《破斥猶豫》等以「業是這些的根源(胎)、原因」(Kammaṃ etesaṃ yoni kāraṇanti, \ccchref{MN.135}{https://agama.buddhason.org/MN/dm.php?keyword=135}/\ccchref{AN.5.57}{https://agama.buddhason.org/AN/an.php?keyword=5.57})解說。
\stopitemgroup

\startitemgroup[noteitems]
\item\subnoteref{678.0}\NoteKeywordNikayaHead{「業的眷屬者」}(kammabandhū,逐字譯為「業+親族」),菩提比丘長老英譯為\NoteKeywordBhikkhuBodhi{「業為我/他們的親戚」}(kamma as my/their relative, AN),智髻比丘長老英譯為「被他們的行為所束縛」(are bound to their actions, MN)。按:《破斥猶豫》等以「業是我的眷屬(親族)」(Kammaṃ mayhaṃ bandhūti, \ccchref{MN.135}{https://agama.buddhason.org/MN/dm.php?keyword=135}/\ccchref{AN.5.57}{https://agama.buddhason.org/AN/an.php?keyword=5.57})解說。
\stopitemgroup

\startitemgroup[noteitems]
\item\subnoteref{679.0}\NoteSubKeyHead{(1)}\NoteKeywordAgamaHead{「所生之趣(SA);趣(GA);生(一)處/生(善)處/往至處/所從去/所往至處(MA);生處(DA)」},南傳作\NoteKeywordNikaya{「趣處」}(gati),菩提比丘長老英譯為\NoteKeywordBhikkhuBodhi{「到達地」}(destination),或「去的」(going, SN),智髻比丘長老英譯為「去的」(the going, MN)。按:這裡的「趣處」指「往生之處」。
\item\subnoteref{679.1}\NoteSubKeyHead{(2)}\NoteKeywordAgamaHead{「(隨趣往)來(SA);所從來處(MA)」},南傳作\NoteKeywordNikaya{「來處」}(āgati,另譯為「來;歸來」,名詞),Maurice Walshe先生英譯為「發生」(the arising, DN),智髻比丘長老英譯為「到來的」(The coming, MN),菩提比丘長老英譯為\NoteKeywordBhikkhuBodhi{「到來的」}(coming, SN)。
\stopitemgroup

\startitemgroup[noteitems]
\item\subnoteref{680.0}\NoteKeywordAgamaHead{「憒亂之處所/憒亂法(SA);在家纏眾務(GA)」},南傳作\NoteKeywordNikaya{「在障礙中(Sambādhe),菩提比丘長老英譯為「在禁閉中」}(in the midst of confinement)。按:《顯揚真義》說,有「蓋、欲」兩種障礙,這裡指前者。「空間的到達」(okāsādhigamo,另譯為「機會的獲得」),菩提比丘長老英譯為\NoteKeywordBhikkhuBodhi{「為了…的開口」}(the opening… for, AN)。空間則指「禪定」(\suttaref{SN.2.7})。《滿足希求》說,障礙指五欲,空間指回憶處(\ccchref{AN.6.26}{https://agama.buddhason.org/AN/an.php?keyword=6.26})。
\stopitemgroup

\startitemgroup[noteitems]
\item\subnoteref{681.0}\NoteKeywordNikayaHead{「生起與消散法的」}(uppādavayadhammino),Maurice Walshe先生英譯為「傾向於生起與瓦解」(prone to rise and fall, DN),菩提比丘長老英譯為\NoteKeywordBhikkhuBodhi{「它們的性質是發生與消散」}(Their nature is to arise and vanish, SN)。
\stopitemgroup

\startitemgroup[noteitems]
\item\subnoteref{682.0}\NoteKeywordNikayaHead{「被作為車輛、被作為基礎、被實行、被累積、被善努力」}(yānīkatā vatthukatā anuṭṭhitā paricitā susamāraddhā),菩提比丘長老英譯為\NoteKeywordBhikkhuBodhi{「使它成為我的車與基礎,已貫徹它,已鞏固它,並被適當地承擔」}(made it my vehicle and basis, carried it out, consolidated it, and properly undertaken it, AN),或「使它成為我們的車,使它成為我們的基礎,使它安定,於其中鍛鍊我們自己,並徹底使它純然無瑕」(make it our vehicle, make it our basis, stabilize it, exercise ourselves in it, and fully perfect it, SN),智髻比丘長老英譯為「用作車輛,用作基礎,已建立,已鞏固,並完全地承擔」(used as a vehicle, used as a basis, established, consolidated, and well undertaken, MN)。按:已累積(paricitā),另譯為「聚集的;增加的;習慣的;熟知的」,《顯揚真義》等以「無論在何處心都善增長」(samantato citā suvaḍḍhitā, \suttaref{SN.20.3}/\ccchref{DN.16}{https://agama.buddhason.org/DN/dm.php?keyword=16}/\ccchref{AN.6.13}{https://agama.buddhason.org/AN/an.php?keyword=6.13})解說,《滿足希求》以「善的累積、善增長」(suṭṭhu upacitaṃ suvaḍḍhitaṃ)解說「善累積」(suparicitaṃ, \ccchref{AN.9.25}{https://agama.buddhason.org/AN/an.php?keyword=9.25})。「被善努力」(susamāraddhā),《無礙解道》〈3.入出息念的談論〉解讀為「被善同一發動」(su-sama-āraddhā, \ccchref{Ni.3}{https://agama.buddhason.org/Ni/dm.php?keyword=3})。
\stopitemgroup

\startitemgroup[noteitems]
\item\subnoteref{683.0}\NoteKeywordAgamaHead{「處、非處(SA/AA);是處、非處(DA)」},南傳作\NoteKeywordNikaya{「可能為可能、不可能為不可能」}(ṭhānañca ṭhānato aṭṭhānañca aṭṭhānato,另譯為「原因是原因、非原因是非原因),菩提比丘長老英譯為\NoteKeywordBhikkhuBodhi{「可能為可能與不可能為不可能」}(the possible as possible and the impossible as impossible)。按:「可能」(ṭhānañca),另譯為「處,地方,住處,狀態,點,理由,原因,道理」,《破斥猶豫》等以「原因(根據)」(kāraṇañca, \ccchref{MN.12}{https://agama.buddhason.org/MN/dm.php?keyword=12}/\ccchref{AN.6.64}{https://agama.buddhason.org/AN/an.php?keyword=6.64}/\ccchref{AN.10.21}{https://agama.buddhason.org/AN/an.php?keyword=10.21})解說。
\stopitemgroup

\startitemgroup[noteitems]
\item\subnoteref{684.0}\NoteKeywordAgamaHead{「最勝處智(SA)」},南傳作\NoteKeywordNikaya{「最上位」}(āsabhaṃ ṭhānaṃ,逐字譯為「牛王-處(狀態)」),智髻比丘長老英譯為「群之王的地位」(the herd-leader's place, MN),菩提比丘長老英譯為\NoteKeywordBhikkhuBodhi{「牛王的地位」}(the place of the chief bull, AN)。按:在古印度,牛被認為是神聖的。
\stopitemgroup

\startitemgroup[noteitems]
\item\subnoteref{685.0}\NoteKeywordNikayaHead{「在遠離上的、極樂於離欲的」}(vivekaṭṭhaṃ nekkhammābhirataṃ),菩提比丘長老依錫蘭本(vavakaṭṭhaṃ nekkhammābhirataṃ)英譯為「它被引退,樂於放棄」(it is withdrawn, delighting in renunciation)。按:《滿足希求》以「使污染被避開或成為遠隔的」(kilesehi vajjitaṃ dūrībhūtaṃ vā)解說「在遠離上的」,以「樂於出家」(pabbajjābhirataṃ)解說「極樂於離欲的」(\ccchref{AN.8.28}{https://agama.buddhason.org/AN/an.php?keyword=8.28}),「離欲」(nekkhamma),另作「出離(放棄世俗生活)」。
\stopitemgroup

\startitemgroup[noteitems]
\item\subnoteref{686.0}\NoteSubKeyHead{(1)}\NoteKeywordAgamaHead{「飽食/多食(SA);飲食不調適(GA)」},南傳作\NoteKeywordNikaya{「餐後的睡意」}(bhattasammado),菩提比丘長老英譯為\NoteKeywordBhikkhuBodhi{「飯後的睡意」}(drowsiness after meals)。按:《顯揚真義》以「餐後明顯的餐後迷糊(昏迷)、疲倦、熱惱、身粗重」(bhuttāvissa bhattamucchā bhattakilamatho bhattapariḷāho kāyaduṭṭhullaṃ, \suttaref{SN.46.2})解說,水野弘元《巴利語辭典》作「食不調;飽食」,《法蘊足論》作「食不調性」,《阿毘曇毘婆沙論》作「食不消化」,《舍利弗阿毘曇論》作「身重」。
\item\subnoteref{686.1}\NoteSubKeyHead{(2)}\NoteKeywordAgamaHead{「心憒鬧/微弱(SA);心下狹劣(GA)」},南傳作\NoteKeywordNikaya{「心的退縮」}(cetaso ca līnattaṃ),菩提比丘長老英譯為\NoteKeywordBhikkhuBodhi{「心的遲鈍」}(sluggishness of mind)。按:《顯揚真義》以「心的不堪任性、不適合作業性」(cittassa akalyatā akammaññatā, \suttaref{SN.46.2})等解說。
\stopitemgroup

\startitemgroup[noteitems]
\item\subnoteref{687.0}\NoteKeywordAgamaHead{「障礙相(SA)」},南傳作\NoteKeywordNikaya{「嫌惡相」}(paṭighanimittaṃ),菩提丘長老英譯為「厭惡的特徵」(the sign of the repulsive, the mark of the repulsive, AN)。按:《滿足希求》以「不合意的相」(aniṭṭhaṃ nimittaṃ/aniṭṭhanimittaṃ, \ccchref{AN.1.12}{https://agama.buddhason.org/AN/an.php?keyword=1.12}/\ccchref{AN.2.125}{https://agama.buddhason.org/AN/an.php?keyword=2.125})解說。
\stopitemgroup

\startitemgroup[noteitems]
\item\subnoteref{688.0}\NoteKeywordNikayaHead{「止相;奢摩他相」}(samathanimittaṃ),菩提丘長老英譯為「平靜的徵候」(the sign of serenity)。
\stopitemgroup

\startitemgroup[noteitems]
\item\subnoteref{689.0}\NoteKeywordAgamaHead{「四未曾有法/人四種如意之德/四種人如意足(MA);四神德(DA);四神足(AA)」},南傳作\NoteKeywordNikaya{「四種成就」}(catūhi ca iddhīhi),Maurice Walshe先生英譯為「四種特性」(four properties, DN),智髻比丘長老英譯為「四種成就」(four kinds of success, MN)。按:「成就」(iddhi),另外的意思是「神通;如意」,但此處顯然不作此解。
\stopitemgroup

\startitemgroup[noteitems]
\item\subnoteref{690.0}\NoteSubKeyHead{(1)}\NoteKeywordNikayaHead{「戒具足」}(sīlasampadā),菩提比丘長老英譯為\NoteKeywordBhikkhuBodhi{「在德行上成就」}(accomplishment in virtue)。按:《顯揚真義》以「四清淨戒」(catupārisuddhisīlaṃ, \suttaref{SN.45.69})解說(守學處、自制、適當使用生活所需、正命)。
\item\subnoteref{690.1}\NoteSubKeyHead{(2)}\NoteKeywordNikayaHead{「意欲具足」}(chandasampadā),菩提比丘長老英譯為\NoteKeywordBhikkhuBodhi{「在想要上成就」}(accomplishment in desire)。按:《顯揚真義》以「想為善的意欲」(kusalakattukamyatāchando, \suttaref{SN.45.69})解說。
\item\subnoteref{690.2}\NoteSubKeyHead{(3)}\NoteKeywordNikayaHead{「自己具足」}(attasampadā),菩提比丘長老英譯為\NoteKeywordBhikkhuBodhi{「在自我上成就」}(accomplishment in self)。按:《顯揚真義》以「心的具足狀態」(sampannacittata, \suttaref{SN.45.69})解說。
\stopitemgroup

\startitemgroup[noteitems]
\item\subnoteref{691.0}\NoteKeywordAgamaHead{「正法(SA);善解、如法(MA)」},南傳作\NoteKeywordNikaya{「真理、善法」}(ñāyaṃ dhammaṃ kusalaṃ),菩提比丘長老英譯為\NoteKeywordBhikkhuBodhi{「方法、有益的方法」}(the method, the Dhamma that is wholesome, SN),或「真實的方法、有益的方法」(the true way, the Dhamma that is wholesome, MN/AN)。按:《滿足希求》以「與毘婆舍那一起的道」(sahavipassanakaṃ maggaṃ, \ccchref{AN.2.41}{https://agama.buddhason.org/AN/an.php?keyword=2.41}),《破斥猶豫》以「有理由、無過失的善法」(kāraṇabhūtaṃ anavajjaṭṭhena kusalaṃ dhammaṃ, \ccchref{MN.76}{https://agama.buddhason.org/MN/dm.php?keyword=76})解說。「真理」(ñāyaṃ),另譯為「正理;理趣」,《破斥猶豫》以「聖道」(ariyamagga, \suttaref{SN.45.24})解說。
\stopitemgroup

\startitemgroup[noteitems]
\item\subnoteref{692.0}\NoteKeywordAgamaHead{「賢聖等三昧(SA);聖正定(MA)」},南傳作\NoteKeywordNikaya{「聖正定」}(ariyo sammāsamādhi),菩提比丘長老英譯為\NoteKeywordBhikkhuBodhi{「高潔的正確集中貫注」}(noble right concentration)。按:「等」為「正」的另譯,《破斥猶豫》說,「聖」是無過失的、出世間的(niddosaṃ lokuttaraṃ, \ccchref{MN.117}{https://agama.buddhason.org/MN/dm.php?keyword=117}),以無過失而被稱為「聖」。
\stopitemgroup

\startitemgroup[noteitems]
\item\subnoteref{693.0}\NoteKeywordAgamaHead{「眾具(SA);有助/有助有具/說助說具(MA)」},南傳作\NoteKeywordNikaya{「有資糧」}(saparikkhāro,另譯為「有資助;有資具」),菩提比丘長老英譯為\NoteKeywordBhikkhuBodhi{「跟隨著其要件」}(with its accessories, with its requisites)。按:《顯揚真義》等以「有隨從」(saparivāraṃ, \suttaref{SN.45.28}/\ccchref{MN.117}{https://agama.buddhason.org/MN/dm.php?keyword=117}/\ccchref{DN.18}{https://agama.buddhason.org/DN/dm.php?keyword=18})解說。
\stopitemgroup

\startitemgroup[noteitems]
\item\subnoteref{694.0}\NoteKeywordAgamaHead{「無所有說無所有(MA);斷滅學者(DA)」},南傳作\NoteKeywordNikaya{「虛無論/虛無論者」}(natthikavādo/natthikavādā,另譯為「非有論者」),菩提比丘長老英譯為\NoteKeywordBhikkhuBodhi{「虛無主義者」}(nihilism, AN),智髻比丘長老英譯為「虛無主義之教義;懷疑論之教義」(the doctrine of nihilism, MN)。
\item\subnoteref{694.1}\NoteSubKeyHead{(2)}\NoteKeywordNikayaHead{「存在論/存在論者」}(atthikavādo),智髻比丘長老英譯為「持有存在主義的教義者」(who holds the doctrine of affirmation, MN)。按:《破斥猶豫》以「有被施與的、被施與的果」(atthi dinnaṃ dinnaphalanti, \ccchref{MN.76}{https://agama.buddhason.org/MN/dm.php?keyword=76})解說。
\stopitemgroup

\startitemgroup[noteitems]
\item\subnoteref{695.0}\NoteSubKeyHead{(1)}\NoteKeywordAgamaHead{「說無作/宗本不可作(MA)」},南傳作\NoteKeywordNikaya{「無作業論者;無作業論的」}(akiriyavādā),菩提比丘長老英譯為\NoteKeywordBhikkhuBodhi{「擁護不活動者」}(who were proponents of inactivity, AN),智髻比丘長老英譯為「無作為之教義」(the doctrine of non-doing, MN)。
\item\subnoteref{695.1}\NoteSubKeyHead{(2)}\NoteKeywordAgamaHead{「無也(AA)」},南傳作\NoteKeywordNikaya{「無作業」}(akiriyaṃ),菩提比丘長老英譯為\NoteKeywordBhikkhuBodhi{「不作」}(non-doing, AN),Maurice Walshe先生英譯為「無行動」(non-action, DN)。
\item\subnoteref{695.2}\NoteSubKeyHead{(3)}\NoteKeywordNikayaHead{「[有]作業論者」}(kiriyavādī),智髻比丘長老英譯為「持論行為的道德有效者」(held the doctrine of the moral efficacy of deeds, MN)。
\stopitemgroup

\startitemgroup[noteitems]
\item\subnoteref{696.0}\NoteKeywordAgamaHead{「福處(MA)」},南傳作\NoteKeywordNikaya{「有福分的;有福分者」}(puññabhāgiyā,另譯為「福德的有分者」),菩提比丘長老英譯為\NoteKeywordBhikkhuBodhi{「參與功績者/參與功績」}(Who partake of merit, SN/partaking of merit, MN)。
\stopitemgroup

\startitemgroup[noteitems]
\item\subnoteref{697.0}\NoteKeywordAgamaHead{「諂諛美辭,現相毀呰,以利求利/從他所得以示於人更求他利(DA)」},南傳作\NoteKeywordNikaya{「詭計、攀談、暗示、譏諷、以利得換取其他利得」}(Kuhanā, lapanā, nemittikatā, nippesikatā, lābhena lābhaṃ nijigīsanatā/Kuhako ca hoti, lapako ca, nemittiko ca, nippesiko ca, lābhena ca lābhaṃ nijigīsitā),智髻比丘長老英譯為「計畫、談話、暗示、輕視、以獲得追求獲得」(Scheming, talking, hinting, belittling, pursuing gain with gain, MN),菩提比丘長老英譯為\NoteKeywordBhikkhuBodhi{「計畫者、諂媚者、暗示者、輕視者、以獲得追求獲得者」}(a schemer, a flatterer, a hinter, a belittler, and one who pursues gain with gain, AN)。按:「設計」(Kuhanā),另譯為「詐欺;詐騙」,《分別論》以「依止利養、恭敬、名聲之惡欲求、隨欲求擺佈者以所謂必需品的受用(paccayapaṭisevanasaṅkhātena, 錫蘭本paccayapaṭisedhanasaṅkhātena-拒絕必需品),或以周邊語(sāmantajappitena-為了使人布施的說謎語言),或舉止行為的做作(aṭṭhapanā ṭhapanā saṇṭhapanā)、皺眉頭(bhākuṭikā bhākuṭiyaṃ)、設計(kuhanā kuhāyanā kuhitattaṃ)」解說,餘項類推。「以利得換取其他利得」(lābhena lābhaṃ nijigīsanatā),另譯為「以利求利」,這是指拿自己不喜歡或剩餘的供養品,去換取其他更好的供養品。
\stopitemgroup

\startitemgroup[noteitems]
\item\subnoteref{698.0}\NoteKeywordAgamaHead{「雄猛觀諸義(MA)」},南傳作\NoteKeywordNikaya{「現觀利益的堅固者」}(Atthābhisamayā dhīro),菩提比丘長老英譯為\NoteKeywordBhikkhuBodhi{「一個依獲得利益而堅決者」}(The steadfast one, by attaining the good, SN)。按:《顯揚真義》等以「利益的獲得」(atthapaṭilābhā, \suttaref{SN.3.17}/\ccchref{AN.4.42}{https://agama.buddhason.org/AN/an.php?keyword=4.42})解說「現觀利益」。「堅固者」(dhīro, 動詞dhārayati, dharati),另一個意思是「明智者」(動詞dīdheti)。
\stopitemgroup

\startitemgroup[noteitems]
\item\subnoteref{699.0}\NoteSubKeyHead{(1)}\NoteKeywordNikayaHead{「聖住、梵住、如來住」}(ariyavihāro/ariyo vihāro, brahmavihāro/brahmā vihāro, tathāgatavihāro),菩提比丘長老英譯為\NoteKeywordBhikkhuBodhi{「高潔的住處,神聖的住處,如來的住處」}(a noble dwelling, a divine dwelling, the Tathāgata's dwelling, SN)。按:\ccchref{SA.502}{https://agama.buddhason.org/SA/dm.php?keyword=502}等說,聖住即無相心定。
\item\subnoteref{699.1}\NoteSubKeyHead{(2)}\NoteKeywordNikayaHead{「學住」},南傳作\NoteKeywordNikaya{「有學住」}(sekho vihāro),菩提比丘長老英譯為\NoteKeywordBhikkhuBodhi{「訓練中者的住處」}(the dwelling of a trainee)。
\item\subnoteref{699.2}\NoteSubKeyHead{(3)}\NoteKeywordNikayaHead{「天住」}(dibbo vihāro),Maurice Walshe先生英譯為「天的住處」(deva-abiding, DN)。按:《集異門足論》說:「天住云何?答:謂四靜慮。」《阿毘曇毘婆沙論》:「天所有故,名天住。能生天故,名天住。」《大智度論》:「六種欲天住法,是為天住。」
\stopitemgroup

\startitemgroup[noteitems]
\item\subnoteref{700.0}\NoteKeywordAgamaHead{「心悅/欣悅心(SA)」},南傳作\NoteKeywordNikaya{「使心喜悅著」}(abhippamodayaṃ cittaṃ),菩提比丘長老英譯為\NoteKeywordBhikkhuBodhi{「使心喜悅著」}(Gladdening the mind, \suttaref{SN.54.1})。按:《清淨道論》說,這裡以兩種行相(方式)而有喜悅:由定與毘婆舍那,前者是到達有喜的二種禪定,後者是出該二種禪定後觸知(sammasati)與禪定聯結的喜之滅盡、消散,在這樣的毘婆舍那剎那(Evaṃ vipassanākkhaṇe),以禪定聯結的喜作所緣後使心歡喜、喜悅(cittaṃ āmodeti pamodeti, 8.235)。
\stopitemgroup

\startitemgroup[noteitems]
\item\subnoteref{701.0}\NoteSubKeyHead{(1)}\NoteKeywordAgamaHead{「心定/定心(SA)」},南傳作\NoteKeywordNikaya{「集中著心」}(samādahaṃ cittaṃ),菩提比丘長老英譯為\NoteKeywordBhikkhuBodhi{「使心集中貫注著」}(Concentrating the mind)。按:「集中著」(samādahaṃ,另譯為「定著」),為現在分詞語態,《清淨道論》說,因初禪等而心集中、住立(samaṃ ādahanto samaṃ ṭhapento)於所緣,或出那些禪定後當看見(sampassato)與禪定聯結的心滅盡、消散之毘婆舍那剎那(vipassanākkhaṇe)時,以特相的洞察(lakkhaṇapaṭivedhena)而生起剎那心一境性(khaṇikacittekaggatā, 8.235)。
\item\subnoteref{701.1}\NoteSubKeyHead{(2)}\NoteKeywordNikayaHead{「心定」}(cittasamādhiṃ),菩提比丘長老英譯為\NoteKeywordBhikkhuBodhi{「心的集中貫注」}(concentration of mind)。按:《顯揚真義》以「與毘婆舍那一起的四[果]道」(saha vipassanāya cattāro maggā, \suttaref{SN.42.13})解說。
\stopitemgroup

\startitemgroup[noteitems]
\item\subnoteref{702.0}\NoteKeywordAgamaHead{「心解脫/解脫心(SA)」},南傳作\NoteKeywordNikaya{「使心解脫著」}(vimocayaṃ cittaṃ),菩提比丘長老英譯為\NoteKeywordBhikkhuBodhi{「使心釋放著」}(Liberating the mind, \suttaref{SN.51.3})。按:《清淨道論》說,以初禪心從[五]蓋、以第二禪從尋伺、以第三禪從喜、以第四禪心從苦樂脫離、解脫(mocento vimocento),或出那些禪定後觸知(sammasati)與禪定聯結的心滅盡、消散,在毘婆舍那剎那(vipassanākkhaṇe),他以無常隨看,心從常想脫離、解脫,以苦隨看,心從樂想脫離、解脫,以無我隨看,心從我想脫離、解脫,從歡喜以厭隨看、從貪以離貪隨看、從集以滅隨看、以斷念隨看,心從執取脫離而吸氣與呼氣(paṭinissaggānupassanāya ādānato cittaṃ mocento assasati ceva passasati ca, 8.235)。
\stopitemgroup

\startitemgroup[noteitems]
\item\subnoteref{703.0}\NoteKeywordAgamaHead{「觀察滅/觀滅(SA)」},南傳作\NoteKeywordNikaya{「隨看著滅」}(nirodhānupassī),菩提比丘長老英譯為\NoteKeywordBhikkhuBodhi{「凝視停止」}(Contemplating cessation)。按:「隨看著」(anupassī,另譯為「觀察」,形容詞,但以如現在分詞的動作形容詞解讀),《清淨道論》說,隨看著離貪(virāgānupassī),這裡有兩種離貪:滅盡離貪與究竟離貪(khayavirāgo ca accantavirāgo ca),前者為行的剎那壞滅(saṅkhārānaṃ khaṇabhaṅgo),後者為涅槃,隨看著離貪指由那兩者的看見滅而毘婆舍那與道被轉起(tadubhayadassanavasena pavattā vipassanā ca maggo ca),成為具備這兩種觀察後,吸氣與呼氣。隨看著滅的足跡也是這樣的意趣(Nirodhānupassīpadepi eseva nayo, 8.236)。
\stopitemgroup

\startitemgroup[noteitems]
\item\subnoteref{704.0}\NoteKeywordAgamaHead{「難陀林(SA);歡喜園(GA);難檀槃那園(AA)」},南傳作\NoteKeywordNikaya{「歡喜園」}(Nandana)。按:依\ccchref{MN.75}{https://agama.buddhason.org/MN/dm.php?keyword=75},此為三十三天的園林名字。
\stopitemgroup

\startitemgroup[noteitems]
\item\subnoteref{705.0}\NoteKeywordAgamaHead{「福德潤澤(SA)」},南傳作\NoteKeywordNikaya{「福德的流出」}(puññābhisandā,另譯為「福德的等流;福德的潤澤」),菩提比丘長老英譯為\NoteKeywordBhikkhuBodhi{「功績之流」}(the streams of merit)。按:《滿足希求》以「福德的之流出,依止福德的」(puññassa abhisandā, puññappattiyoti, \ccchref{AN.4.51}{https://agama.buddhason.org/AN/an.php?keyword=4.51})解說。
\stopitemgroup

\startitemgroup[noteitems]
\item\subnoteref{706.0}\NoteKeywordAgamaHead{「群生之所依(SA)」},南傳作\NoteKeywordNikaya{「被人群群眾使用的」}(naragaṇasaṅghasevitā),菩提比丘長老英譯為\NoteKeywordBhikkhuBodhi{「被一大群人使用」}(used by the hosts of people)。
\stopitemgroup

\startitemgroup[noteitems]
\item\subnoteref{707.0}\NoteSubKeyHead{(1)}\NoteKeywordNikayaHead{「寶物聚集之阿賴耶」}(ratanagaṇānamālayaṃ),菩提比丘長老英譯為\NoteKeywordBhikkhuBodhi{「寶石堆容器」}(receptacle of heaps of gems)。
\item\subnoteref{707.1}\NoteSubKeyHead{(2)}\NoteKeywordNikayaHead{「殊勝寶物之阿賴耶」}(ratanavarānamālayaṃ),菩提比丘長老英譯同上。按:「阿賴耶」(ālaya,另譯為「執著;愛著;所執處;窟宅」),《顯揚真義》等說,眾生在五種欲上黏著(sattā pañcasu kāmaguṇesu allīyanti),或黏著百八種渴愛思潮(Aṭṭhasatataṇhāvicaritāni ālayanti, \suttaref{SN.6.1}/\ccchref{MN.26}{https://agama.buddhason.org/MN/dm.php?keyword=26}/\ccchref{DN.14}{https://agama.buddhason.org/DN/dm.php?keyword=14}),因此那些被稱為阿賴耶。
\stopitemgroup

\startitemgroup[noteitems]
\item\subnoteref{708.0}\NoteSubKeyHead{(1)}\NoteKeywordAgamaHead{「惡戒(SA/MA)」},南傳作\NoteKeywordNikaya{「破戒;無德」}(dussīlyaṃ,逐字譯為「惡+戒」,另譯為「劣戒」),菩提比丘長老英譯為\NoteKeywordBhikkhuBodhi{「不道德;惡行」}(immorality)。按:「戒」(sīla),音譯為「尸羅」,義譯為「戒;道德:品德」。
\item\subnoteref{708.1}\NoteSubKeyHead{(2)}\NoteKeywordNikayaHead{「無德形色者」}(Dussīlarūpā),菩提比丘長老英譯為\NoteKeywordBhikkhuBodhi{「不道德的;品行不良的」}(immoral)。「無德」(Dussīla),另譯為「惡戒;破戒;劣戒」。
\stopitemgroup

\startitemgroup[noteitems]
\item\subnoteref{709.0}\NoteKeywordAgamaHead{「那陀村(DA)」},南傳作\NoteKeywordNikaya{「親戚村」}(ñātike, ñātikā),地名。按:「那陀」顯為「親戚」的音譯,《顯揚真義》說,依一個池塘有兩兄弟兒子的兩個村落,其中一個村落[名為親戚村](ekaṃ taḷākaṃ nissāya dvinnaṃ cūḷapitimahāpitiputtānaṃ dve gāmā, tesu ekasmiṃ gāmake, \suttaref{SN.55.8})。
\stopitemgroup

\startitemgroup[noteitems]
\item\subnoteref{710.0}\NoteKeywordNikayaHead{「以種種住處而住」}(nānāvihārehi viharataṃ),菩提比丘長老英譯為\NoteKeywordBhikkhuBodhi{「以所有我們各種職務」}(with all our various engagements)。按:《滿足希求》說,他問:在家人[心]沒有一個固定的住處(nibaddho eko vihāro),因此,當我們住無固定住處時,應該以哪個固定的住處住呢?(tasmā amhākaṃ anibaddhavihārena viharantānaṃ kena vihārena katarena nibaddhavihārena vihātabbanti, \ccchref{AN.11.11}{https://agama.buddhason.org/AN/an.php?keyword=11.11})
\stopitemgroup

\startitemgroup[noteitems]
\item\subnoteref{711.0}\NoteKeywordNikayaHead{「求出生者」}(sambhavesīnaṃ),菩提比丘長老英譯為\NoteKeywordBhikkhuBodhi{「那些將生成者」}(those about to come to be)。按:《顯揚真義》說,求出生者是那些探求(esanti)、尋求(gavesanti)生起生成、出生者(sambhavaṃ jātiṃ nibbattiṃ, \suttaref{SN.12.11})。《俱舍論》說是「中有」:「又契經說︰『食有四種,能令部多有情安住,及能資益諸求生者,無漏不然,故非食體。』……復說『求生』,為何所目?此目中有,由佛世尊以五種名說中有故。」
\stopitemgroup

\startitemgroup[noteitems]
\item\subnoteref{712.0}\NoteSubKeyHead{(1)}\NoteKeywordAgamaHead{「調馬(SA);不調之馬(GA);惡馬(GA/MA)」},南傳作\NoteKeywordNikaya{「未調馬」}(assakhaḷuṅke),菩提比丘長老英譯為\NoteKeywordBhikkhuBodhi{「野生小馬」}(wild colts)。
\item\subnoteref{712.1}\NoteSubKeyHead{(2)}\NoteKeywordAgamaHead{「良馬(SA/MA);馬良善調順/善乘馬(GA)」},南傳作\NoteKeywordNikaya{「賢駿馬」}(bhadrā assājānīyā),菩提比丘長老英譯為\NoteKeywordBhikkhuBodhi{「優秀高尚的馬」}(excellent thoroughbred horses)。
\stopitemgroup

\startitemgroup[noteitems]
\item\subnoteref{713.0}\NoteKeywordAgamaHead{「色具足(SA);好色(MA)」},南傳作\NoteKeywordNikaya{「容色具足的」}(vaṇṇasampanno),菩提比丘長老英譯為\NoteKeywordBhikkhuBodhi{「擁有美貌」}(possesses beauty, is endowed with beauty, AN),或「有美麗顏色」(having a fine colour, SN)。
\stopitemgroup

\startitemgroup[noteitems]
\item\subnoteref{714.0}\NoteSubKeyHead{(1)}\NoteKeywordAgamaHead{「士夫(SA);丈夫之乘(GA)」},南傳作\NoteKeywordNikaya{「賢駿人」}(bhadro purisājānīyo),菩提比丘長老英譯為\NoteKeywordBhikkhuBodhi{「優秀高尚的人」}(excellent thoroughbred person)。
\item\subnoteref{714.1}\NoteSubKeyHead{(2)}\NoteKeywordAgamaHead{「賢人(\ccchref{MA.183}{https://agama.buddhason.org/MA/dm.php?keyword=183})」},南傳作\NoteKeywordNikaya{「賢面」}(bhadramukha),菩提比丘長老英譯為\NoteKeywordBhikkhuBodhi{「我親愛的」}(my dear, AN),智髻比丘長老英譯為「我親愛的」(my dear),或「親愛的」(dear ones),或「先生」(sir)。
\stopitemgroup

\startitemgroup[noteitems]
\item\subnoteref{715.0}\NoteKeywordAgamaHead{「此是真實,餘則虛妄/唯此真實,異則虛妄/是則真實,餘者虛妄/唯此為實,餘皆妄語(SA);此是真諦,餘者虛妄/此是真諦,餘皆虛妄(MA);此為真諦,餘者虛妄/此實餘虛(DA)」},南傳作\NoteKeywordNikaya{「這才是真實的,其它都是空虛的」}(idameva saccaṃ moghamaññanti),菩提比丘長老英譯為\NoteKeywordBhikkhuBodhi{「這才是真實的,別的是錯誤的」}(only this is true, anything else is wrong)。按:此句前後完整內容參看\ccchref{MN.95}{https://agama.buddhason.org/MN/dm.php?keyword=95}。
\stopitemgroup

\startitemgroup[noteitems]
\item\subnoteref{716.0}\NoteKeywordAgamaHead{「動搖見(SA);見所動(MA)」},南傳作\NoteKeywordNikaya{「動搖之見(令人動搖的惡見)」}(diṭṭhivipphanditaṃ),菩提比丘長老英譯為\NoteKeywordBhikkhuBodhi{「動搖之見」}(the vacillation of views)。
\stopitemgroup

\startitemgroup[noteitems]
\item\subnoteref{717.0}\NoteKeywordNikayaHead{「不適用;不適當;不攀取;達不到」}(na upeti),菩提比丘長老英譯為\NoteKeywordBhikkhuBodhi{「不交合;不銜接;不參與」}(does not become engaged, SN),或「不適用」(does not apply, \suttaref{SN.44.1}),或「不順從」(won't submit to, AN),智髻比丘長老英譯為「不適當」(would not be prope, MN)。按:「適用」(upeti)為廣義動詞,原意譯為「靠近;到達」,常隨使用場合有不同指涉,《破斥猶豫》以「不結合;不會合」(na yujjati, \ccchref{MN.72}{https://agama.buddhason.org/MN/dm.php?keyword=72})解說。
\stopitemgroup

\startitemgroup[noteitems]
\item\subnoteref{718.0}\NoteKeywordAgamaHead{「不可思惟(AA)」},南傳作\NoteKeywordNikaya{「超越推論的」}(atakkāvacaro,另譯為「深奧的」),菩提比丘長老英譯為\NoteKeywordBhikkhuBodhi{「僅以推理不能到達」}(unattainable by mere reasoning)。按:《顯揚真義》等以「不能被推論執行進入;僅被智執行」(takkena avacaritabbo ogāhitabbo na hoti, ñāṇeneva avacaritabbo, \suttaref{SN.6.1}/\ccchref{MN.26}{https://agama.buddhason.org/MN/dm.php?keyword=26}等/\ccchref{DN.14}{https://agama.buddhason.org/DN/dm.php?keyword=14}),《滿足希求》以「不能被推論、推理執見掌握」(takkena nayaggāhena gahetuṃ sakkā hoti, \ccchref{AN.4.192}{https://agama.buddhason.org/AN/an.php?keyword=4.192})解說,《吉祥悅意》說,最高智的境域狀態不能被推論執行(Uttamañāṇavisayattā na takkena avacaritabbāti, \ccchref{DN.1}{https://agama.buddhason.org/DN/dm.php?keyword=1})。
\stopitemgroup

\startitemgroup[noteitems]
\item\subnoteref{719.0}\NoteKeywordNikayaHead{「在處一一存在時」}(sati sati-āyatane),菩提比丘長老英譯為\NoteKeywordBhikkhuBodhi{「有適當的基礎」}(there being a suitable basis),並解說,對五種證智是指第四禪,對阿羅漢果是觀(\ccchref{MN.73}{https://agama.buddhason.org/MN/dm.php?keyword=73})。按:《破斥猶豫》說,在有因素(理由)時(sati satikāraṇe),這裡什麼是因素呢?證智或證智基礎的禪定(Abhiññā vā abhiññāpādakajjhānaṃ vā),或阿羅漢狀態的終結或[證]阿羅漢狀態的毘婆舍那因素(avasāne pana arahattaṃ vā kāraṇaṃ arahattassa vipassanā vāti, \ccchref{MN.73}{https://agama.buddhason.org/MN/dm.php?keyword=73}),《滿足希求》以「在有因素(理由)時:過去因與現在能得到的,以及證智基礎的禪定等種類」(pubbahetusaṅkhāte ceva idāni ca paṭiladdhabbe abhiññāpādakajjhānādibhede ca sati satikāraṇe, \ccchref{AN.3.102}{https://agama.buddhason.org/AN/an.php?keyword=3.102})解說。
\stopitemgroup

\startitemgroup[noteitems]
\item\subnoteref{720.0}\NoteKeywordAgamaHead{「身分齊受所覺/身分齊受覺(SA);受身邊時(GA);生後身覺/受身最後覺(MA)」},南傳作\NoteKeywordNikaya{「當感受身體終了的感受時」}(kāyapariyantikaṃ vedanaṃ vedayamāno),菩提比丘長老英譯為\NoteKeywordBhikkhuBodhi{「當他感受身體終止的感受」}(when he feels a feeling terminating with the body)。按:「身分齊」即「身的界限;身的長度」,就是「身體終了」(kāyapariyantikaṃ,另譯為「身體的制限」),《滿足希求》等以「身體極限的身體限制(劃分):在五門身(五根)轉起到五門的感受落下期間」(kāyantikaṃ kāyaparicchinnaṃ, yāva pañcadvārakāyo pavattati, tāva pavattaṃ pañcadvārikavedananti, \ccchref{AN.4.195}{https://agama.buddhason.org/AN/an.php?keyword=4.195}/\suttaref{SN.12.51}),《破斥猶豫》以「身體的終點(kāyakoṭikaṃ):從身體轉起生起後直到感受不生起期間」(Yāva kāyapavattā uppajjitvā tato paraṃ anuppajjanavedananti, \ccchref{MN.140}{https://agama.buddhason.org/MN/dm.php?keyword=140})解說。
\stopitemgroup

\startitemgroup[noteitems]
\item\subnoteref{721.0}\NoteKeywordAgamaHead{「壽分齊受所覺/命分齊受覺(SA);受命邊時(GA);生後命覺/受命最後覺(MA)」},南傳作\NoteKeywordNikaya{「當感受生命終了的感受時」}(jīvitapariyantikaṃ vedanaṃ vediyamāno),菩提比丘長老英譯為\NoteKeywordBhikkhuBodhi{「當他感受生命終止的感受」}(when he feels a feeling terminating with life)。按:「壽分齊」即「壽命的界限;壽命的長度」(jīvitapariyantikaṃ,另譯為「壽命的制限」),《滿足希求》等以「生命極限的生命限制(劃分):從生命轉起直到意門的感受落下期間」(jīvitantikaṃ jīvitaparicchinnaṃ. Yāva jīvitaṃ pavattati, tāva pavattaṃ manodvārikavedananti, \ccchref{AN.4.195}{https://agama.buddhason.org/AN/an.php?keyword=4.195}/\suttaref{SN.12.51})解說。
\stopitemgroup

\startitemgroup[noteitems]
\item\subnoteref{722.0}\NoteKeywordNikayaHead{「惡見;{惡}見」}(diṭṭhigataṃ),菩提比丘長老英譯為\NoteKeywordBhikkhuBodhi{「投機見解」}(speculative view, SN),或「捲入見解」(an involvement with views, AN),智髻比丘長老英譯為「我的見解」(my point of view, MN),或「其他人的見解」(the views of others, \ccchref{MN.27}{https://agama.buddhason.org/MN/dm.php?keyword=27}),Maurice Walshe先生英譯為「見解;觀點」(view, opinion, DN),或「推理的延伸」(line of reasoning, DN)。按:此字以「{惡}見」表示時單純等同「見」(diṭṭhi),但多數情況指「邪見」(micchādiṭṭhi)。《破斥猶豫》以「邪惡的常見」(lāmakā sassatadiṭṭhi, \ccchref{MN.49}{https://agama.buddhason.org/MN/dm.php?keyword=49})解說「邪惡的惡見」,《滿足希求》以「邪見程度」(micchādiṭṭhimattaka, \ccchref{AN.7.54}{https://agama.buddhason.org/AN/an.php?keyword=7.54})解說。
\stopitemgroup

\startitemgroup[noteitems]
\item\subnoteref{723.0}\NoteKeywordAgamaHead{「三昧善(SA)」},南傳作\NoteKeywordNikaya{「定善巧者」}(samādhikusalo),菩提比丘長老英譯為\NoteKeywordBhikkhuBodhi{「熟練於貫注集中」}(is skilled in concentration)。按:《顯揚真義》說,初禪有五支,二禪有三支,這樣,熟練於決意各個支(aṅgavavatthānakusalo, \suttaref{SN.34.1})。
\stopitemgroup

\startitemgroup[noteitems]
\item\subnoteref{724.0}\NoteKeywordAgamaHead{「正受善(SA)」},南傳作\NoteKeywordNikaya{「等至善巧者」}(samāpattikusalo,另譯為「正受善巧」),菩提比丘長老英譯為\NoteKeywordBhikkhuBodhi{「熟練於到達」}(is skilled in attainment)。按:《顯揚真義》以「進入禪定」(jhānaṃ samāpajjituṃ, \suttaref{SN.34.1}),《滿足希求》以「把握有益的食物、有益的氣候(utusappāyaṃ)後熟練地進入等至(samāpattisamāpajjane, \ccchref{AN.2.164}{https://agama.buddhason.org/AN/an.php?keyword=2.164})」解說,長老認為應該以等至自在(samāpattivasi)解說,如:於何時、何地、入定多久等。
\stopitemgroup

\startitemgroup[noteitems]
\item\subnoteref{725.0}\NoteKeywordAgamaHead{「住三昧善(SA)」},南傳作\NoteKeywordNikaya{「持續善巧者」}(ṭhitikusalo,另譯為「存續善巧;住善巧」),菩提比丘長老英譯為\NoteKeywordBhikkhuBodhi{「熟練於維持」}(is skilled in maintenance)。按:《顯揚真義》以「保持(停留於)禪定(jhānaṃ ṭhapetuṃ)七、八個彈指長(sattaṭṭhaaccharāmattaṃ, \suttaref{SN.34.2})」解說,長老認為這與「攝持自在;在定自在」(adhiṭṭhānavasi)相關(能決定何地何時在禪定中持續多久)。
\stopitemgroup

\startitemgroup[noteitems]
\item\subnoteref{726.0}\NoteKeywordAgamaHead{「三昧起善(SA)」},南傳作\NoteKeywordNikaya{「出定善巧者」}(vuṭṭhānakusalo),菩提比丘長老英譯為\NoteKeywordBhikkhuBodhi{「熟練於脫出」}(is skilled in emergence)。按:《顯揚真義》以「如限定[時間]從禪定出來」(yathāparicche jhānato vuṭṭhātuṃ, \suttaref{SN.34.3}),《滿足希求》以「限定[時間]已達到時熟練地起來」(viyatto hutvā uṭṭhahanto, \ccchref{AN.2.164}{https://agama.buddhason.org/AN/an.php?keyword=2.164})解說,長老說,這是指熟練於在預定的時間出定的出定自在(vuṭṭhānavasi)。
\stopitemgroup

\startitemgroup[noteitems]
\item\subnoteref{727.0}\NoteKeywordAgamaHead{「迎善(SA)」},南傳作\NoteKeywordNikaya{「順意善巧者」}(kallitakusalo),菩提比丘長老英譯為\NoteKeywordBhikkhuBodhi{「熟練於柔順」}(is skilled in pliancy)。按:《顯揚真義》以「心喜後適合做」(cittaṃ hāsetvā kallaṃ kātuṃ, \suttaref{SN.34.4})解說。
\stopitemgroup

\startitemgroup[noteitems]
\item\subnoteref{728.0}\NoteKeywordAgamaHead{「所緣善巧(SA)」},南傳作\NoteKeywordNikaya{「所緣善巧者」}(ārammaṇakusalo),菩提比丘長老英譯為\NoteKeywordBhikkhuBodhi{「熟練於對象」}(is skilled in the object)。按:《顯揚真義》以「在遍處所緣上的熟練」(kasiṇārammaṇesu akusalo, \suttaref{SN.34.5})解說。
\stopitemgroup

\startitemgroup[noteitems]
\item\subnoteref{729.0}\NoteKeywordAgamaHead{「處善(SA)」},南傳作\NoteKeywordNikaya{「行境善巧者」}(gocarakusalo,另譯為「適當範圍之善巧」),菩提比丘長老英譯為\NoteKeywordBhikkhuBodhi{「熟練於範圍」}(is skilled in the range)。按:行境(gocara),另譯為「行處;適當範圍;親近處」,《顯揚真義》以「業處行境與乞食行境」(kammaṭṭhānagocare ceva bhikkhācāragocare, \suttaref{SN.34.6})解說,《滿足希求》說,避開不適當無效的法,從事適當有效的,知道:「這是禪定的相與所緣(samādhinimittārammaṇo),這是特相所緣(lakkhaṇārammaṇo, \ccchref{AN.6.24}{https://agama.buddhason.org/AN/an.php?keyword=6.24})。」長老說「特相所緣」指禪定與毘婆舍那的區別。
\stopitemgroup

\startitemgroup[noteitems]
\item\subnoteref{730.0}\NoteKeywordNikayaHead{「決意善巧者」}(abhinīhārakusalo),菩提比丘長老英譯為\NoteKeywordBhikkhuBodhi{「熟練於決心」}(is skilled in resolution)。按:《顯揚真義》以「抽出(心)指向業處」(kammaṭṭhānaṃ abhinīharituṃ, \suttaref{SN.34.7})解說,長老解說,這個意思是指不能由初禪進入第二禪,由第二禪進入第三禪等。《滿足希求》以「進入更上等至之意」(upariuparisamāpattisamāpajjanatthāya, \ccchref{AN.6.24}{https://agama.buddhason.org/AN/an.php?keyword=6.24})解說:出初禪進入第二禪,出第二禪……出第三禪進入第四禪。
\stopitemgroup

\startitemgroup[noteitems]
\item\subnoteref{731.0}\NoteKeywordNikayaHead{「恭敬作者」}(sakkaccakārī),菩提比丘長老英譯為\NoteKeywordBhikkhuBodhi{「徹底的工作者」}(a thorough worker)。按:《顯揚真義》以「能緊固入禪定」(jhānaṃ appetuṃ, \suttaref{SN.34.8})解說。
\stopitemgroup

\startitemgroup[noteitems]
\item\subnoteref{732.0}\NoteKeywordNikayaHead{「常作者」}(sātaccakārī,另譯為「不斷地行動者;堅忍不拔的作者」),菩提比丘長老英譯為\NoteKeywordBhikkhuBodhi{「持續努力的工作者」}(a persistent worker)。按:《顯揚真義》以「安止禪定的常作者」(jhānappanāya satatakārī, \suttaref{SN.34.9})解說。
\stopitemgroup

\startitemgroup[noteitems]
\item\subnoteref{733.0}\NoteKeywordAgamaHead{「方便善(SA)」},南傳作\NoteKeywordNikaya{「適當作者」}(sappāyakārī),菩提比丘長老英譯為\NoteKeywordBhikkhuBodhi{「做所有適當者」}(one who does what is suitable)。按:《顯揚真義》以「能使禪定適當有效法充足(pūretuṃ, \suttaref{SN.34.10})」解說。
\stopitemgroup

\startitemgroup[noteitems]
\item\subnoteref{734.0}\NoteSubKeyHead{(1)}\NoteKeywordAgamaHead{「如殺/為殺(SA)」}即\ccchref{SA.104}{https://agama.buddhason.org/SA/dm.php?keyword=104}殺手的譬喻。
\item\subnoteref{734.1}\NoteSubKeyHead{(2)}\NoteKeywordAgamaHead{「叉摩經/叉摩比丘經/差摩迦修多羅/差摩修多羅/叉摩修多羅(SA)」}即\ccchref{SA.103}{https://agama.buddhason.org/SA/dm.php?keyword=103}。
\item\subnoteref{734.2}\NoteSubKeyHead{(3)}\NoteKeywordAgamaHead{「達摩提那長者修多羅/達摩提那修多羅(SA)」}即\ccchref{SA.1033}{https://agama.buddhason.org/SA/dm.php?keyword=1033}。
\item\subnoteref{734.3}\NoteSubKeyHead{(4)}\NoteKeywordAgamaHead{「淳陀修多羅(SA)」}即\ccchref{SA.1039}{https://agama.buddhason.org/SA/dm.php?keyword=1039}。
\item\subnoteref{734.4}\NoteSubKeyHead{(5)}\NoteKeywordAgamaHead{「篋毒蛇經/篋毒蛇譬經/篋毒蛇(SA)」}即\ccchref{SA.1172}{https://agama.buddhason.org/SA/dm.php?keyword=1172}。
\stopitemgroup

\startitemgroup[noteitems]
\item\subnoteref{735.0}\NoteKeywordAgamaHead{「六隨念;六念;六思念(DA)」},南傳作\NoteKeywordNikaya{「六隨念處」}(cha anussatiṭṭhānāni),菩提比丘長老英譯為\NoteKeywordBhikkhuBodhi{「回憶的六個主題」}(the six subjects of recollection)。按:「六隨念處」即念「佛、法、僧、戒、施、天」,參看\ccchref{SA.550}{https://agama.buddhason.org/SA/dm.php?keyword=550}、\ccchref{AN.6.26}{https://agama.buddhason.org/AN/an.php?keyword=6.26}等,這些又被稱為「無上的隨念」(anussatānuttariyaṃ),參看\ccchref{AN.6.30}{https://agama.buddhason.org/AN/an.php?keyword=6.30}等。
\stopitemgroup

\startitemgroup[noteitems]
\item\subnoteref{736.0}\NoteSubKeyHead{(1)}\NoteKeywordNikayaHead{「波羅夷」}(pārājika,義譯為「驅擯」),犯此類戒條者失去比丘、比丘尼的身份,逐出僧團。
\item\subnoteref{736.1}\NoteSubKeyHead{(2)}\NoteKeywordNikayaHead{「僧伽婆尸沙」}(saṅghādisesa,義譯為「僧殘」),犯此類戒條者需處以與僧團隔離六夜後,向二十位清淨比丘懺悔出罪重回僧團。
\item\subnoteref{736.2}\NoteSubKeyHead{(3)}\NoteKeywordNikayaHead{「波夜提;波逸提」}(pācittiya,義譯為「懺悔」),犯此類戒條者應在僧團中報告,得僧伽同意,然後到離僧伽不遠(眼見耳不聞)處向一位清淨比丘發露出罪。(參看《原始佛教聖典之集成》p.137)。
\stopitemgroup

\startitemgroup[noteitems]
\item\subnoteref{737.0}\NoteKeywordAgamaHead{「月分八日;月八日(SA/GA/MA/DA);八日(AA)」},南傳作\NoteKeywordNikaya{「半月的第八日」}(pakkhassa aṭṭhamī),菩提比丘長老英譯為\NoteKeywordBhikkhuBodhi{「在兩週的第八日」}(on the eighths of the fortnight),並說明半月形的第8天則稱為「小布薩日;小齋戒日」(minor Uposathaṃ, \suttaref{SN.10.5})。
\stopitemgroup

\startitemgroup[noteitems]
\item\subnoteref{738.0}\NoteKeywordAgamaHead{「神變月/神變之月/神足瑞應月/神足月(SA);神足月/如來神足月(GA)」},南傳作\NoteKeywordNikaya{「以及神變月」}(Pāṭihāriyapakkhañca),菩提比丘長老英譯為\NoteKeywordBhikkhuBodhi{「在特別的兩週間」}(during special fortnights)。按:「神變月」為古印度的一些特別月份,有說是正月,有說是二月,有說是正月、五月、九月。
\stopitemgroup

\startitemgroup[noteitems]
\item\subnoteref{739.0}\NoteSubKeyHead{(1)}\NoteKeywordNikayaHead{「方便」},南傳作\NoteSubEntryKey{(i)}\NoteKeywordNikaya{「努力」}(yoga),菩提比丘長老英譯為\NoteKeywordBhikkhuBodhi{「努力;盡力;致力」}(an exertion)。\NoteSubEntryKey{(ii)}\NoteKeywordNikaya{「精進」}(vāyāmo),菩提比丘長老英譯為\NoteKeywordBhikkhuBodhi{「努力於」}(make an effort in)。\NoteSubEntryKey{(iii)}\NoteKeywordNikaya{「法門;方法」}(pariyāyaṃ),菩提比丘長老英譯為\NoteKeywordBhikkhuBodhi{「方法」}(a method, \suttaref{SN.35.153})。
\item\subnoteref{739.1}\NoteSubKeyHead{(2)}\NoteKeywordAgamaHead{「方便具足(\ccchref{SA.91}{https://agama.buddhason.org/SA/dm.php?keyword=91});精勤(GA)」},南傳作\NoteKeywordNikaya{「奮起具足」}(Uṭṭhānasampadā,另譯為「奮起成就」),菩提比丘長老英譯為\NoteKeywordBhikkhuBodhi{「進取成就(完成/達成)」}(accomplishment in initiative)。
\stopitemgroup

\startitemgroup[noteitems]
\item\subnoteref{740.0}\NoteKeywordAgamaHead{「修多羅……優波提舍(SA);修多羅……本事(GA);正經……說義(MA);貫經……大教經(DA);契經……生經(AA)」}皆為「十二分教」,南傳作\NoteKeywordNikaya{「修多羅……毘富羅」}(suttaṃ geyyaṃ veyyākaraṇaṃ gāthaṃ udānaṃ itivuttakaṃ jātakaṃ abbhutadhammaṃ vedallaṃ),為「九分教」,菩提比丘長老英譯為\NoteKeywordBhikkhuBodhi{「說教、散文與詩混合、解說、詩、有所啟示的話、引用語、出生的故事、不可思議的記事、問與答」}(the discourses, mixed prose and verse, expositions, verses, inspired utterances, quotations, birth stories, marvelous accounts, and questions-and-answers)。有關「九分教」與「十二分教」,參看《原始佛教聖典之集成》p.493-628。
\stopitemgroup

\startitemgroup[noteitems]
\item\subnoteref{741.0}\NoteKeywordAgamaHead{「持母者(MA)」},南傳作\NoteKeywordNikaya{「持本母的;持本母者」}(mātikādharā),菩提比丘長老英譯為\NoteKeywordBhikkhuBodhi{「概要專家」}(experts on the outlines, AN),智髻比丘長老英譯為「維護規則者」(who maintain the Codes, MN)。按:「本母」(mātikā),另譯為「論母;智母;律母;綱目」,音譯為「摩得勒迦;摩夷」,為經典與律典先列舉綱目再加解說的一種格式,出現於「阿毘達摩(阿毘曇)」之前,參看印順法師著《原始佛教聖典之集成》〈第五章 摩得勒伽與犍度〉。
\stopitemgroup

\startitemgroup[noteitems]
\item\subnoteref{742.0}\NoteKeywordNikayaHead{「僧團父」}(saṅghapitaro),菩提比丘長老英譯為\NoteKeywordBhikkhuBodhi{「僧團的父親們」}(the fathers of the Saṅgha)。按:《吉祥悅意》等說,這是僧團中立於父親地位者,他指導僧團,率先做後(pubbaṅgamā hutvā, \ccchref{DN.16}{https://agama.buddhason.org/DN/dm.php?keyword=16}/\ccchref{AN.7.23}{https://agama.buddhason.org/AN/an.php?keyword=7.23})使三學轉起。
\stopitemgroup

\startitemgroup[noteitems]
\item\subnoteref{743.0}\NoteKeywordAgamaHead{「隨所思、隨所念(MA)」},南傳作\NoteKeywordNikaya{「隨尋思、隨伺察」}(anuvitakketi anuvicāreti,另譯為「思惟、思考」),智髻比丘長老英譯為「思考與沉思」(thinks and ponders, MN),菩提比丘長老英譯為\NoteKeywordBhikkhuBodhi{「沉思與檢查」}(ponders, examines, AN)。
\stopitemgroup

\startitemgroup[noteitems]
\item\subnoteref{744.0}\NoteKeywordNikayaHead{「通曉阿含的;通曉阿含者」}(āgatāgamā),菩提比丘長老英譯為\NoteKeywordBhikkhuBodhi{「傳承的繼承人」}(heirs to the heritage)。按:「阿含」(āgama)為譯音,意譯為「傳來;傳承」,指「師承傳來的聖教」,《滿足希求》說,一部(尼柯耶)、一阿含;二部、二阿含;五部、五阿含之謂(pañca nikāyā pañca āgamā nāma),在這些阿含中的一阿含到達熟練轉起者(āgato paguṇo pavattito, \ccchref{AN.3.20}{https://agama.buddhason.org/AN/an.php?keyword=3.20}),他們名為通曉阿含。
\stopitemgroup

\startitemgroup[noteitems]
\item\subnoteref{745.0}\NoteKeywordNikayaHead{「阿僧祇」}(asaṅkheyya, asaṅkha,為音譯,義譯為「無法數的;不可測的;無數的」),為古印度一極大或不可數的數目單位。
\stopitemgroup

\startitemgroup[noteitems]
\item\subnoteref{746.0}\NoteKeywordAgamaHead{「四聖住於果(SA);四果(SA/GA);四果具足成(AA)」},南傳作\NoteKeywordNikaya{「與四種在果位上已住立者」}(cattāro ca phale ṭhitā),菩提比丘長老英譯為\NoteKeywordBhikkhuBodhi{「與四種在果上已建立者」}(And the four established in the fruit)。按:「四果」指「須陀洹、斯陀含、阿那含、阿羅漢」,「正向者有四;四向;四種修行者」指「向須陀洹、向斯陀含、向阿那含、向阿羅漢」,兩者合稱為「四雙八輩」、「八種人」。
\stopitemgroup

\startitemgroup[noteitems]
\item\subnoteref{747.0}\NoteSubKeyHead{(1)}\NoteKeywordAgamaHead{「半梵行者(SA);梵行半體(GA);半梵行之人(AA)」},南傳作\NoteKeywordNikaya{「這是梵行的一半」}(upaḍḍhamidaṃ……brahmacariyaṃ),菩提比丘長老英譯為\NoteKeywordBhikkhuBodhi{「這是聖潔生活的一半」}(this is half of the holy life)。
\item\subnoteref{747.1}\NoteSubKeyHead{(2)}\NoteKeywordAgamaHead{「純一滿淨梵行清白(SA);梵行全體(GA);全梵行(AA)」},南傳作\NoteKeywordNikaya{「這是梵行的全部」}(sakalamidaṃ……brahmacariyaṃ),菩提比丘長老英譯為\NoteKeywordBhikkhuBodhi{「這是全部的聖潔生活」}(this is the entire holy life, \suttaref{SN.45.2}),並解說,尊者阿難認為,一位沙門修學成就,一半是靠善友,一半是靠自己的努力,但世尊說,四道、四果等,全都根植於善友,並舉例,以小孩來說,不可能說「多少來自父親,多少來自母親」。
\stopitemgroup

\startitemgroup[noteitems]
\item\subnoteref{748.0}\NoteKeywordAgamaHead{「𤛓牛乳頃(SA);𤛓牛頃(MA/AA)」},南傳作\NoteKeywordNikaya{「如擠牛奶時拉一次奶頭那樣短的時間」}(gandhohanamattampi, gadduhanamattampi),菩提比丘長老英譯為\NoteKeywordBhikkhuBodhi{「拉一次牛的乳房所花的時間」}(the time it takes to pull a cow's udder, AN)。「𤛓」:擠奶。
\stopitemgroup

\startitemgroup[noteitems]
\item\subnoteref{749.0}\NoteKeywordAgamaHead{「阿毘地獄/無{澤}[擇]大地獄(SA);無間地獄(GA/DA);阿鼻地獄(AA)」},即「無間地獄」(avīci niraya),屬「大熱地獄」(mahāpariḷāha niraya),眾生在這裡六根所觸接是苦,且受苦無間斷(參看\suttaref{SN.56.43}、\ccchref{DA.30}{https://agama.buddhason.org/DA/dm.php?keyword=30}),直到業報盡而往生他處為止,\ccchref{AA.42.2}{https://agama.buddhason.org/AA/dm.php?keyword=42.2}說,犯五逆罪者命終後生阿鼻地獄中。
\stopitemgroup

\startitemgroup[noteitems]
\item\subnoteref{750.0}\NoteSubKeyHead{(1)}\NoteKeywordNikayaHead{「思千種義;一千個道理」}(sahassampi atthānaṃ),菩提比丘長老英譯為\NoteKeywordBhikkhuBodhi{「一千件事」}(a thousand matters)。
\item\subnoteref{750.1}\NoteSubKeyHead{(2)}\NoteKeywordNikayaHead{「千眼」}(sahassakkho),為天帝釋的別名。按:註疏以「一千個慧眼」(sahassaṃ paññāakkhīni, \suttaref{SN.11.12})解說。
\stopitemgroup

\startitemgroup[noteitems]
\item\subnoteref{751.0}\NoteKeywordAgamaHead{「不作行/無所為作/[不]作是行(SA)」},南傳作\NoteKeywordNikaya{「不造作」}(anabhisaṅkha,另譯為「不現行、不為作」),菩提比丘長老英譯為\NoteKeywordBhikkhuBodhi{「不生產的」}(nongenerative)。按:《顯揚真義》說,不造作結生後解脫(paṭisandhiṃ anabhisaṅkharitvā vimuttaṃ, \suttaref{SN.22.53})。
\stopitemgroup

\startitemgroup[noteitems]
\item\subnoteref{752.0}\NoteKeywordAgamaHead{「煩惱/惱(SA);煩惱(GA)」},南傳作\NoteKeywordNikaya{「痛苦」}(aghaṃ,另譯為「罪,禍」),菩提比丘長老英譯為\NoteKeywordBhikkhuBodhi{「不幸;悲慘的境遇;痛苦」}(Misery, \suttaref{SN.1.34}/5.9/22.31),或「不幸的;可憐的」(miserable, \suttaref{SN.2.18}),或「苦惱;麻煩」(trouble, \suttaref{SN.2.18})。按:《顯揚真義》以「苦」(dukkhaṃ, \suttaref{SN.22.31})解說。
\stopitemgroup

\startitemgroup[noteitems]
\item\subnoteref{753.0}\NoteSubKeyHead{(1)}\NoteKeywordAgamaHead{「親覺/親里覺/親屬覺(SA);親里之事(DA)」},南傳作\NoteKeywordNikaya{「親里論」}(ñātikathaṃ,另譯為「親族論;親戚論」),菩提比丘長老英譯為\NoteKeywordBhikkhuBodhi{「談論親戚;談論親族;談論親屬」}(talk about relations)。
\item\subnoteref{753.1}\NoteSubKeyHead{(2)}\NoteKeywordAgamaHead{「國土覺/國土人民覺/人眾覺(SA);國論/論國人民(MA);論國事(DA)」},南傳作\NoteKeywordNikaya{「國土論」}(janapadakathaṃ,另譯為「地方論;當地的故事;當地論」),菩提比丘長老英譯為\NoteKeywordBhikkhuBodhi{「談論國土;談論地方」}(talk about countries)。按:「覺」即「尋;心思」(vitakka)。
\item\subnoteref{753.2}\NoteSubKeyHead{(3)}\NoteKeywordNikayaHead{「怖畏論」}(bhayakathaṃ),談論危險事。「水井論」(kumbhaṭṭhānakathaṃ),井邊的談論;井邊打水時東家長西家短。「祖靈論」(pubbapetakathaṃ),談論過亡者。
\stopitemgroup

\startitemgroup[noteitems]
\item\subnoteref{754.0}\NoteSubKeyHead{(1)}\NoteKeywordNikayaHead{「喜」}(pīti)、「樂」(sukha);「喜、樂」(pītisukhaṃ),菩提比丘長老英譯為\NoteKeywordBhikkhuBodhi{「狂喜與快樂」}(the rapture and pleasure)。按:《破斥猶豫》以「有喜的兩種禪定」(iminā sappītikāni dve jhānāni dasseti, \ccchref{MN.14}{https://agama.buddhason.org/MN/dm.php?keyword=14})解說,即初禪與第二禪。
\item\subnoteref{754.1}\NoteSubKeyHead{(2)}\NoteKeywordAgamaHead{「覺知喜/喜覺知(SA)」},南傳作\NoteKeywordNikaya{「經驗著喜」}(Pītipaṭisaṃvedī),菩提比丘長老英譯為\NoteKeywordBhikkhuBodhi{「體驗著狂喜」}(experiencing rapture)。按:「經驗著」(paṭisaṃvedī,另譯為「感受著」),形容詞,但以如現在分詞的動作形容詞解讀,《清淨道論》說,這裡有兩種行相(方式):從所緣與不迷妄(ārammaṇato ca asammohato ca),前者是到達有喜的二種禪定,後者是出該二種禪定後觸知(sammasati)與禪定聯結的喜之滅盡、消散,他在毘婆舍那剎那,以特相的洞察而感受不迷妄的喜(Tassa vipassanākkhaṇe lakkhaṇapaṭivedhena asammohato pīti paṭisaṃviditā hoti, 8.234)。
\stopitemgroup

\startitemgroup[noteitems]
\item\subnoteref{755.0}\NoteSubKeyHead{(1)}\NoteKeywordAgamaHead{「意生(MA)」},南傳作\NoteKeywordNikaya{「意所生的」}(manomayā,另譯為「意所成的」),Maurice Walshe先生英譯為「心-做的」(mind-made)。按:《破斥猶豫》等以「禪定心所作的/以禪定心所生的」(jhānacittamayā/jhānamanena nibbatto, \ccchref{MN.60}{https://agama.buddhason.org/MN/dm.php?keyword=60}/\ccchref{DN.1}{https://agama.buddhason.org/DN/dm.php?keyword=1}/\ccchref{AN.5.44}{https://agama.buddhason.org/AN/an.php?keyword=5.44}),或「以意所產生的」(manena nibbattaṃ, \ccchref{MN.77}{https://agama.buddhason.org/MN/dm.php?keyword=77}/\ccchref{DN.2}{https://agama.buddhason.org/DN/dm.php?keyword=2}, \ccchref{AN.1.198}{https://agama.buddhason.org/AN/an.php?keyword=1.198})解說。
\item\subnoteref{755.1}\NoteSubKeyHead{(2)}\NoteKeywordNikayaHead{「意生身」}(manomayaṃ kāyaṃ,另譯為「意所生的身體;意所成的身體」),菩提比丘長老英譯為\NoteKeywordBhikkhuBodhi{「心做的身體」}(a mind-made body)。詳細參看\ccchref{MN.77}{https://agama.buddhason.org/MN/dm.php?keyword=77}等。
\item\subnoteref{755.2}\NoteSubKeyHead{(3)}\NoteKeywordAgamaHead{「意生天(MA)」},南傳作\NoteKeywordNikaya{「意所生的[天]眾」}(aññataraṃ manomayaṃ kāyaṃ),菩提比丘長老英譯為\NoteKeywordBhikkhuBodhi{「心做的[諸神]」}(group of mind-made [deities])。按:《滿足希求》以「禪定心所生的某個淨居梵身」(jhānamanena nibbattaṃ aññataraṃ suddhāvāsabrahmakāyaṃ, \ccchref{AN.5.166}{https://agama.buddhason.org/AN/an.php?keyword=5.166})解說。
\stopitemgroup

\startitemgroup[noteitems]
\item\subnoteref{756.0}\NoteKeywordAgamaHead{「本二(SA/GA);本妻(GA);本婦(MA);婦(AA)」},南傳作\NoteKeywordNikaya{「前妻」}(purāṇadutiyikā,逐字譯為「故二;本二」),菩提比丘長老英譯為\NoteKeywordBhikkhuBodhi{「前配偶」}(former consort, SN),智髻比丘長老英譯為「前妻們」(former wives, MN)。
\stopitemgroup

\startitemgroup[noteitems]
\item\subnoteref{757.0}\NoteKeywordNikayaHead{「你的[話]是吉祥的」}(bhaddantavā”ti),菩提比丘長老英譯為\NoteKeywordBhikkhuBodhi{「閣下」}(your lordship, SN),智髻比丘長老英譯為「願[你的話]被認為是神聖的」(May [your words] be held sacred, MN),Maurice Walshe先生英譯為「君主」(Lord),T.W. Rhys Davids先生英譯為「祝你好運」(good luck to you, DN)。按:在四部中,這僅用於天神回答帝釋,《顯揚真義》等以「令你的成為吉祥的」(hotu bhaddaṃ tava iti, \suttaref{SN.11.2}/\ccchref{DN.17}{https://agama.buddhason.org/DN/dm.php?keyword=17}),《破斥猶豫》等以「令你的言語成為吉祥的」(hotu bhaddakaṃ tava vacananti, \ccchref{MN.83}{https://agama.buddhason.org/MN/dm.php?keyword=83}/\ccchref{DN.21}{https://agama.buddhason.org/DN/dm.php?keyword=21}類似)解說。
\stopitemgroup

\startitemgroup[noteitems]
\item\subnoteref{758.0}\NoteKeywordNikayaHead{「脫離正行者」}(muttācāro),菩提比丘長老英譯為\NoteKeywordBhikkhuBodhi{「拒絕習俗」}(rejecting conventions, AN/MN),Maurice Walshe先生英譯為「慣於無禮貌的制約」(uses no polite restraints, DN)。按:《破斥猶豫》等舉「站著大小便吃食等世間善男子沒有的行為(lokiyakulaputtācārena virahito, \ccchref{MN.12}{https://agama.buddhason.org/MN/dm.php?keyword=12}/\ccchref{DN.8}{https://agama.buddhason.org/DN/dm.php?keyword=8}/\ccchref{AN.3.157}{https://agama.buddhason.org/AN/an.php?keyword=3.157})」解說。
\stopitemgroup

\startitemgroup[noteitems]
\item\subnoteref{759.0}\NoteKeywordAgamaHead{「不求來尊/不來尊(MA);不受請食(DA)」},南傳作\NoteKeywordNikaya{「受邀不來者」}(na ehibhaddantiko,逐字譯為「不-來+尊者」,原意為「拒絕『請來!』的尊者」),智髻比丘長老英譯為「當受邀時不來」(not coming when asked, MN),Maurice Walshe先生英譯為「當受邀時不來」(does not come when requested, DN)。
\stopitemgroup

\startitemgroup[noteitems]
\item\subnoteref{760.0}\NoteKeywordAgamaHead{「不住尊(MA)」},南傳作\NoteKeywordNikaya{「受邀不住立者」}(na tiṭṭhabhaddantiko),智髻比丘長老英譯為「當受邀時不停下來」(not stopping when asked, MN),Maurice Walshe先生英譯為「當受邀時不停下來」(does not stand still when requested)。
\stopitemgroup

\startitemgroup[noteitems]
\item\subnoteref{761.0}\NoteKeywordNikayaHead{「集會所」}(sandhāgāraṃ, santhāgāraṃ),Maurice Walshe先生英譯為「會議堂」(meeting-hall, DN),或「住處」(lodging, \ccchref{DN.19}{https://agama.buddhason.org/DN/dm.php?keyword=19}),菩提比丘長老英譯為\NoteKeywordBhikkhuBodhi{「集會所」}(assembly hall, MN/\suttaref{SN.35.243}),「訓練所」(the training hall, \suttaref{SN.56.45}),或「祭祀神殿」(sacrificial temple, AN)。
\stopitemgroup

\startitemgroup[noteitems]
\item\subnoteref{762.0}\NoteKeywordNikayaHead{「不受害的安樂」}(abyāsekasukhaṃ),菩提比丘長老英譯為\NoteKeywordBhikkhuBodhi{「無損傷的幸福;無沾污的幸福」}(unsullied bliss)。按:《破斥猶豫》等以「不被污染灌注的安樂,非雜混的安樂(kilesehi anavasittasukhaṃ, avikiṇṇasukhantipi, \ccchref{MN.27}{https://agama.buddhason.org/MN/dm.php?keyword=27}/\ccchref{AN.4.198}{https://agama.buddhason.org/AN/an.php?keyword=4.198}),《吉祥悅意》以「感受無受污染所害、不受害、無混合、遍淨的增上心的安樂」(kilesabyāsekavirahitattā  abyāsekaṃ asammissaṃ parisuddhaṃ adhicittasukhaṃ paṭisaṃvedetīti, \ccchref{DN.2}{https://agama.buddhason.org/DN/dm.php?keyword=2})解說。
\stopitemgroup

\startitemgroup[noteitems]
\item\subnoteref{763.0}\NoteKeywordAgamaHead{「持法(DA)」},南傳作\NoteKeywordNikaya{「持法的;持法者」}(dhammadharā),菩提比丘長老英譯為\NoteKeywordBhikkhuBodhi{「正法的維持者」}(upholders of the Dhamma)。按:《顯揚真義》等以「學得多聞者與貫通多聞者,學得、貫通法的憶持者」(Atha vā pariyattibahussutā ceva paṭivedhabahussutā ca. Pariyattipaṭivedhadhammānaṃyeva dhāraṇato dhammadharāti, \suttaref{SN.51.10}/\ccchref{AN.8.70}{https://agama.buddhason.org/AN/an.php?keyword=8.70})解說,《滿足希求》以「持經藏者」(suttantapiṭakadharā, \ccchref{AN.3.20}{https://agama.buddhason.org/AN/an.php?keyword=3.20}/\ccchref{AN.6.51}{https://agama.buddhason.org/AN/an.php?keyword=6.51}),或「成為經法的容器」(sutadhammānaṃ ādhārabhūtaṃ, \ccchref{AN.4.7}{https://agama.buddhason.org/AN/an.php?keyword=4.7})解說。
\stopitemgroup

\startitemgroup[noteitems]
\item\subnoteref{764.0}\NoteKeywordAgamaHead{「行法如法(MA)」},南傳作\NoteKeywordNikaya{「方正行的;方正行者」}(sāmīcippaṭipannā,另譯為「如法行的;和敬行的」),菩提比丘長老英譯為\NoteKeywordBhikkhuBodhi{「於適當道路實行的」}(practicing in the proper way)。按:《顯揚真義》等以「能相應道跡的行者」(anucchavikapaṭipadaṃ paṭipannā, \suttaref{SN.51.10}/\ccchref{DN.16}{https://agama.buddhason.org/DN/dm.php?keyword=16}/\ccchref{AN.8.70}{https://agama.buddhason.org/AN/an.php?keyword=8.70})解說。
\stopitemgroup

\startitemgroup[noteitems]
\item\subnoteref{765.0}\NoteKeywordAgamaHead{「向次法/隨順行(SA);隨於佛所行之法(GA);隨順於法/行趣(MA);隨順其行(DA)」},南傳作\NoteKeywordNikaya{「隨法行的;隨法行者」}(anudhammacārino, anudhammacārī),菩提比丘長老英譯為\NoteKeywordBhikkhuBodhi{「自己適當行為的」}(conducting themselves accordingly, AN),或「自己依序地行」(conducts himself accordingly, SN)。
\stopitemgroup

\startitemgroup[noteitems]
\item\subnoteref{766.0}\NoteKeywordAgamaHead{「壽命(SA);壽命/壽行(DA)」},南傳作\NoteKeywordNikaya{「壽行」}(āyusaṅkhāraṃ),菩提比丘長老英譯為\NoteKeywordBhikkhuBodhi{「生命力」}(vital force, AN),或「活力形成」(the vital formation, SN/MN)。按:《顯揚真義》以「色命根」(rūpajīvitindriyaṃ, \suttaref{SN.20.6}),《破斥猶豫》以「就是壽命」(āyumeva)解說,並以「命根」(jīvitindriyaṃ)解說「壽;壽命」(āyu, \ccchref{MN.43}{https://agama.buddhason.org/MN/dm.php?keyword=43})。
\stopitemgroup

\startitemgroup[noteitems]
\item\subnoteref{767.0}\NoteSubKeyHead{(1)}\NoteKeywordNikayaHead{「天鼓」}(devadudrabhi, devadundubhi),Maurice Walshe先生英譯為「空中的閃光」(a blaze in the sky)。按:此即閃電打雷,《吉祥悅意》以「乾雲雷鳴」(sukkhavalāhakagajjanaṃ, \ccchref{DN.1}{https://agama.buddhason.org/DN/dm.php?keyword=1})解說。
\item\subnoteref{767.1}\NoteSubKeyHead{(2)}\NoteKeywordNikayaHead{「並且天鼓破裂」}(devadundubhiyo ca phaliṃsu),菩提比丘長老英譯為\NoteKeywordBhikkhuBodhi{「並且轟隆隆的雷鳴震撼了天空」}(and peals of thunder shook the sky)。
\stopitemgroup

\startitemgroup[noteitems]
\item\subnoteref{768.0}\NoteKeywordAgamaHead{「有無二行中(DA)」},南傳作\NoteKeywordNikaya{「權衡不可比的與存在」}(Tulamatulañca sambhavaṃ,另譯為「等不等生成」),菩提比丘長老英譯為\NoteKeywordBhikkhuBodhi{「比較不可比較的與繼續生存」}(Comparing the incomparable and continued existence, \ccchref{AN.8.70}{https://agama.buddhason.org/AN/an.php?keyword=8.70}),並解說此偈難懂,《滿足希求》有二種解讀:i.將tulaṃ(可比的;個別的)與atulaṃ(不可比的)對比解讀,前者指欲界與色界,後者指無色界,而「有行」(bhavasaṅkhāraṃ)是導致有的業。ii.將tulaṃ解讀為tuleti(衡量;考慮;權衡)的現在分詞tulenta,將atulaṃ(不可比的)與sambhavaṃ(出生)對比解讀,前者指涅槃,後者指生存,而「有行」是得到生存的業(\ccchref{AN.8.70}{https://agama.buddhason.org/AN/an.php?keyword=8.70})。長老認為後者的解讀較適宜。
\stopitemgroup

\startitemgroup[noteitems]
\item\subnoteref{769.0}\NoteSubKeyHead{(1)}\NoteKeywordNikayaHead{「有行」}(bhavasaṅkhāro),菩提比丘長老英譯為\NoteKeywordBhikkhuBodhi{「存在的形成活動」}(the formative activity of existence)。按:《顯揚真義》等以「再有的行」(punabbhavassa saṅkhāraṃ, \suttaref{SN.51.10}/\ccchref{DN.16}{https://agama.buddhason.org/DN/dm.php?keyword=16}/\ccchref{AN.8.70}{https://agama.buddhason.org/AN/an.php?keyword=8.70}),或「導至有(存在)的業」(bhavagāmikammaṃ, \ccchref{AN.8.70}{https://agama.buddhason.org/AN/an.php?keyword=8.70}),或「有之造作業」(bhavasaṅkharaṇakammaṃ, \ccchref{AN.10.49}{https://agama.buddhason.org/AN/an.php?keyword=10.49})解說。
\item\subnoteref{769.1}\NoteSubKeyHead{(2)}\NoteKeywordAgamaHead{「捨有為(DA)」},南傳作\NoteKeywordNikaya{「放棄有行」}(bhavasaṅkhāramavassaji),菩提比丘長老英譯為\NoteKeywordBhikkhuBodhi{「放開生存力」}(let go the force of existence)。
\stopitemgroup

\startitemgroup[noteitems]
\item\subnoteref{770.0}\NoteKeywordAgamaHead{「如鳥出於卵(DA)」},南傳作\NoteKeywordNikaya{「如鎧甲般自己的存在」}(kavacamivattasambhavan”ti),菩提比丘長老英譯為\NoteKeywordBhikkhuBodhi{「像鎧甲外衣般他自己的存在」}(his own existence like a coat of armor, AN),或「像鎧甲外衣般繼續自我存在」(Continued self-existence like a coat of armor, SN),Maurice Walshe先生英譯為「生命之殼」(becoming's shell)。按:《顯揚真義》以「自身內喜樂者入定者破裂」(Ajjhattarato samāhito abhindi, \suttaref{SN.51.10})解說,《吉祥悅意》說,在戰場的先頭如鎧甲般自己存在的大戰士,而成為自身內喜樂者入定者破裂污染(Saṅgāmasīse mahāyodho kavacaṃ viya attasambhavaṃ kilesañca ajjhattarato samāhito hutvā abhindī’’ti, \ccchref{DN.16}{https://agama.buddhason.org/DN/dm.php?keyword=16})。
\stopitemgroup

\startitemgroup[noteitems]
\item\subnoteref{771.0}\NoteKeywordAgamaHead{「舍利(SA/DA/AA)」},南傳作\NoteKeywordNikaya{「遺骨」}(sarīraṃ,另譯為「舍利、遺體、身體」),Maurice Walshe先生英譯為「遺骨」(remains)。
\stopitemgroup

\startitemgroup[noteitems]
\item\subnoteref{772.0}\NoteKeywordNikayaHead{「柔軟沙門」}(samaṇasukhumālo),菩提比丘長老英譯為\NoteKeywordBhikkhuBodhi{「精緻的禁欲主義者」}(the delicate ascetic)。按:《吉祥悅意》說,阿羅漢與一切無污染繫縛情況者(ganthakārakilesānaṃ abhāvā, \ccchref{DN.33}{https://agama.buddhason.org/DN/dm.php?keyword=33})為沙門中的柔軟沙門,《滿足希求》說,斷除一切傲慢性執行的污染(thaddhabhāvakarānaṃ kilesānaṃ, \ccchref{AN.4.88}{https://agama.buddhason.org/AN/an.php?keyword=4.88}),已到達柔軟狀態的諸漏已盡者為柔軟沙門。
\stopitemgroup

\startitemgroup[noteitems]
\item\subnoteref{773.0}\NoteKeywordAgamaHead{「慰勞法(MA);(六)重法(DA);(六)重之法(AA)」},南傳作\NoteKeywordNikaya{「友好法」}(dhammā sāraṇīyā, sāraṇīyā dhammā),菩提比丘長老英譯為\NoteKeywordBhikkhuBodhi{「真誠的原則」}(principles of cordiality, MN/AN),Maurice Walshe先生英譯為「有利於公共生活的事」(things conducive to communal living, DN)。按:「友好的」(sāraṇīya),另譯為「有禮貌的;憶念的」,梵文為「相慶慰的」(saṃrañjanīya)。
\stopitemgroup

\startitemgroup[noteitems]
\item\subnoteref{774.0}\NoteKeywordAgamaHead{「慈身業(MA);身常行慈(DA);身行念慈/慈身業(AA)」},南傳作\NoteKeywordNikaya{「慈身業」}(mettaṃ kāyakammaṃ),菩提比丘長老英譯為\NoteKeywordBhikkhuBodhi{「慈愛的身體行為」}(bodily acts of loving-kindness, MN/AN),Maurice Walshe先生英譯為「慈愛的身體行為」(loving-kindness…in acts of body, DN)。
\stopitemgroup

\startitemgroup[noteitems]
\item\subnoteref{775.0}\NoteKeywordAgamaHead{「令攝(MA)」},南傳作\NoteKeywordNikaya{「轉起凝聚」}(saṅgahāya…saṃvattanti),菩提比丘長老英譯為\NoteKeywordBhikkhuBodhi{「有助於凝聚」}(conduce to cohesion/conduce to cohesiveness, MN/AN)。按:「凝聚」(saṅgahāya),另譯為「攝;攝取;愛護;聚會;結集」。
\stopitemgroup

\startitemgroup[noteitems]
\item\subnoteref{776.0}\NoteKeywordNikayaHead{「戒達到一致的」}(sīlasāmaññagatā,相當於古譯的「戒和同行」),Maurice Walshe先生英譯為「那些行為規則保持一致」(keep consistently…those rules of conduct, DN),菩提比丘長老英譯為\NoteKeywordBhikkhuBodhi{「與…一樣的有德品行」}(in common with…of virtuous behavior, AN)。
\stopitemgroup

\startitemgroup[noteitems]
\item\subnoteref{777.0}\NoteKeywordNikayaHead{「被現起」}(paccupaṭṭhitaṃ hoti, paccupaṭṭhitaṃ),菩提比丘長老英譯為\NoteKeywordBhikkhuBodhi{「保持」}(maintains),Maurice Walshe先生英譯為「展現」(show)。
\stopitemgroup

\startitemgroup[noteitems]
\item\subnoteref{778.0}\NoteKeywordAgamaHead{「不衰法(MA);不退法(DA);不退轉法(AA)」},南傳作\NoteKeywordNikaya{「不衰退法」}(aparihāniyā dhammā),Maurice Walshe先生英譯為「防止衰退的原則」(principles for preventing decline, DN),菩提比丘長老英譯為\NoteKeywordBhikkhuBodhi{「不衰退的原則」}(principles of non-decline, AN)。
\stopitemgroup

\startitemgroup[noteitems]
\item\subnoteref{779.0}\NoteKeywordNikayaHead{「除非以欺騙」}(aññatra upalāpanāya),Maurice Walshe先生英譯為「但只以宣傳手段」(but only by means of propaganda, DN),菩提比丘長老英譯為\NoteKeywordBhikkhuBodhi{「除非經由背叛」}(except through treachery, AN)。
\stopitemgroup

\startitemgroup[noteitems]
\item\subnoteref{780.0}\NoteKeywordNikayaHead{「所聽聞的蓄積者」}(sutasannicayo),菩提比丘長老英譯為\NoteKeywordBhikkhuBodhi{「累積所有他學習的」}(accumulates what he has learned, AN),智髻比丘長老英譯為「整理(結合)所有他學習的」(consolidates what he has learned, MN)。
\stopitemgroup

\startitemgroup[noteitems]
\item\subnoteref{781.0}\NoteKeywordAgamaHead{「相的隨行者」(nimittānusāri),Maurice Walshe先生英譯為「嚮往形跡」(hankers after signs)。菩提比丘長老英譯為「跟隨在記號後」(follows after marks, AN),或「跟隨著形跡」(followed along with signs, SA)。按:《顯揚真義》說,當以這觀定住處(vipassanāsamādhivihārena, \suttaref{SN.40.9})而住時,當觀智銳利英勇地運行(vahamāne)時,……他對於[這樣銳利英勇運行的]觀生起愉快(nikanti),而不能完成觀應該作的,《破斥猶豫》以「憶念、追隨色相」}(rūpanimittaṃ anussarati anudhāvatīti, \ccchref{MN.138}{https://agama.buddhason.org/MN/dm.php?keyword=138}),《吉祥悅意》等以「憶念所說種類之相」(vuttappabhedaṃ nimittaṃ anusaratīti/anusaraṇasabhāvaṃ, \ccchref{DN.33}{https://agama.buddhason.org/DN/dm.php?keyword=33}/\ccchref{AN.6.13}{https://agama.buddhason.org/AN/an.php?keyword=6.13})解說。
\stopitemgroup

\startitemgroup[noteitems]
\item\subnoteref{782.0}\NoteKeywordAgamaHead{「有見及無見(MA);有見、無見(DA/AA)」},南傳作\NoteKeywordNikaya{「有見與無有見」}(bhavadiṭṭhi ca vibhavadiṭṭhi),智髻比丘長老英譯為「存在的見解與非存在的見解」(the view of being and the view of non-being, MN),菩提比丘長老英譯為\NoteKeywordBhikkhuBodhi{「實存的見解與根絕的見解」}(the view of existence and the view of extermination, SN/AN)。按:《顯揚真義》等說,前者是常見(sassatadiṭṭhīti),後者是斷見(ucchedadiṭṭhi, \suttaref{SN.22.1}/\ccchref{MN.11}{https://agama.buddhason.org/MN/dm.php?keyword=11}/\ccchref{DN.33}{https://agama.buddhason.org/DN/dm.php?keyword=33}/\ccchref{AN.2.92}{https://agama.buddhason.org/AN/an.php?keyword=2.92}),《破斥猶豫》說,這兩種見的集起為「蘊、無明、觸、想、尋、不如理作意、惡友、他人的音聲(paratoghosopi)」等八見處,其滅沒為「須陀洹道」(sotāpattimaggo, \ccchref{MN.11}{https://agama.buddhason.org/MN/dm.php?keyword=11})。
\stopitemgroup

\startitemgroup[noteitems]
\item\subnoteref{783.0}\NoteKeywordAgamaHead{「非境界/非其境界(SA);非彼境界(AA)」},南傳作\NoteKeywordNikaya{「不在[感官的]境域中」}(avisayasmin,逐字譯為「非+境域中」),菩提比丘長老英譯為\NoteKeywordBhikkhuBodhi{「那不在他的領域中」}(that would not be within his domain)。
\stopitemgroup

\startitemgroup[noteitems]
\item\subnoteref{784.0}\NoteKeywordAgamaHead{「象齋及馬齋,馬齋不障門(MA)」},南傳作\NoteKeywordNikaya{「馬祭、人祭、擲棍祭、飲酒祭、無遮祭」}(Assamedhaṃ purisamedhaṃ, sammāpāsaṃ vājapeyyaṃ niraggaḷhaṃ),菩提比丘長老英譯為\NoteKeywordBhikkhuBodhi{「馬獻祭,人獻祭,Sammāpāsa, vājapeyya, niraggaḷa」}(The horse sacrifice, human sacrifice, Sammāpāsa, vājapeyya, niraggaḷa, SN/AN)。按:《顯揚真義》說,馬祭、人祭、擲棍祭、無遮祭為古時國王攝世間(lokaṃ saṅgaṇhiṃsu)的四攝事(cattāri saṅgahavatthūni)祭祀,後來歐葛葛王(Okkāka,有謂此王即釋迦族的先祖甘蔗王)時代的婆羅門轉變這四攝事為五種殺人、殺馬畜的牲祭。這裡,家中門上的閂不存在被稱為無遮祭(Idaṃ gharadvāresu aggaḷānaṃ abhāvato niraggaḷanti vuccati, \suttaref{SN.3.9}/\ccchref{AN.8.1}{https://agama.buddhason.org/AN/an.php?keyword=8.1})。\ccchref{MA.156}{https://agama.buddhason.org/MA/dm.php?keyword=156}/Sn.2.7說,這是歐葛葛王被一些要從中獲得財富的婆羅門欺騙做的,堪稱是典型的宗教騙財。
\stopitemgroup

\startitemgroup[noteitems]
\item\subnoteref{785.0}\NoteKeywordNikayaHead{「已到達有的彼岸者」}(bhavassa pāragūti),菩提比丘長老英譯為\NoteKeywordBhikkhuBodhi{「已超越存在」}(has transcended existence, AN),或「存在的超越者」(the transcender of existence, SN)。按:「有(的)」(bhavassa)即愛、取、有的有,有的彼岸即是涅槃。
\stopitemgroup

\startitemgroup[noteitems]
\item\subnoteref{786.0}\NoteSubKeyHead{(1)}\NoteKeywordAgamaHead{「如似大牛王/大士之牛王(SA)」},南傳作\NoteKeywordNikaya{「人牛王」}(nisabho),菩提比丘長老英譯為\NoteKeywordBhikkhuBodhi{「首席公牛」}(chief bull)。
\item\subnoteref{786.1}\NoteSubKeyHead{(2)}\NoteKeywordNikayaHead{「獨處的人牛王」}(paṭilīnanisabho),菩提比丘長老英譯為\NoteKeywordBhikkhuBodhi{「移到旁處(抽離)的首席公牛」}(The withdrawn chief bull)。按:《顯揚真義》以「最上的獨處者」(paṭilīnaseṭṭho, \suttaref{SN.2.7})解說。
\stopitemgroup

\startitemgroup[noteitems]
\item\subnoteref{787.0}\NoteKeywordNikayaHead{「大陣列」}(mahābyūha, mahāviyūha,另譯為「大聚集」),菩提比丘長老英譯為\NoteKeywordBhikkhuBodhi{「大陣列」}(Great Array, SN),Maurice Walshe先生英譯為「大的」(great, DN)。
\stopitemgroup

\startitemgroup[noteitems]
\item\subnoteref{788.0}\NoteKeywordAgamaHead{「傳說(MA);語(DA)」},南傳作\NoteKeywordNikaya{「言語道」}(niruttipathā),菩提比丘長老英譯為\NoteKeywordBhikkhuBodhi{「語言途徑」}(pathways of language, SN),Maurice Walshe先生英譯沒譯,坦尼沙羅比丘長老英譯為「指定的手段」(means of designation, DN),萊斯戴維斯先生英譯為「言語表達過程」(process of verbal expression, DN)。按:言語道、名稱道、安立道,\suttaref{SN.22.62}、\ccchref{DN.15}{https://agama.buddhason.org/DN/dm.php?keyword=15}分別為描述五蘊、生死流轉的語言、名稱、安立(描述)方式。
\stopitemgroup

\startitemgroup[noteitems]
\item\subnoteref{789.0}\NoteKeywordAgamaHead{「增語說傳(MA)」},南傳作\NoteKeywordNikaya{「名稱道」}(adhivacanapathā,另譯為「增語道」),菩提比丘長老英譯為\NoteKeywordBhikkhuBodhi{「名稱途徑」}(pathways of designation, SN),Maurice Walshe先生英譯為「名稱之路」(the way of designation, DN),坦尼沙羅比丘長老英譯為「表達的手段」(means of expression, DN),萊斯戴維斯先生英譯為「表現過程」(process of manifestation, DN)。
\stopitemgroup

\startitemgroup[noteitems]
\item\subnoteref{790.0}\NoteKeywordAgamaHead{「演說(DA)」},南傳作\NoteKeywordNikaya{「安立道」}(paññattipathā),菩提比丘長老英譯為\NoteKeywordBhikkhuBodhi{「描述途徑」}(pathways of description, SN),Maurice Walshe先生英譯為「概念之路」(the way of concepts, DN),坦尼沙羅比丘長老英譯為「描寫的手段」(means of delineation, DN),萊斯戴維斯先生英譯為「說明過程」(process of explanation, DN)。
\stopitemgroup

\startitemgroup[noteitems]
\item\subnoteref{791.0}\NoteKeywordAgamaHead{「作、教作(SA);自作、教作(MA);若自作,若教人作(DA)」},南傳作\NoteKeywordNikaya{「作者、使他作者」}(karoto kārayato),菩提比丘長老英譯為\NoteKeywordBhikkhuBodhi{「當一個人[自己]行動,或使他人行動」}(when one acts or make others act, SN/MN),Maurice Walshe先生英譯為「做者或事情的發起者」(by the doer or instigator of a thing, DN)。按:《顯揚真義》等分別以「親自做(sahatthā karontassa)、命令使[他人]做(āṇattiyā kārentassa, \suttaref{SN.24.6}/\ccchref{MN.60}{https://agama.buddhason.org/MN/dm.php?keyword=60}/\ccchref{DN.2}{https://agama.buddhason.org/DN/dm.php?keyword=2})」解說。
\stopitemgroup

\startitemgroup[noteitems]
\item\subnoteref{792.0}\NoteKeywordAgamaHead{「斷、教斷(SA);自斷、教斷(MA);{研}[斫?]伐殘害(DA)」},南傳作\NoteKeywordNikaya{「切斷者、使他切斷者」}(chindato chedāpayato),菩提比丘長老英譯為\NoteKeywordBhikkhuBodhi{「當一個人[自己]切斷[他人手足],或使他人切斷[他人手足]」}(when one mutilates or make others mutilate, SN/MN),Maurice Walshe先生英譯為「切者或導致被切」(by one who cuts or causes to be cut, DN)。
\stopitemgroup

\startitemgroup[noteitems]
\item\subnoteref{793.0}\NoteKeywordAgamaHead{「煮、教煮(SA);煮、教煮(MA);煮{灸}[炙?]切割(DA)」},南傳作\NoteKeywordNikaya{「折磨者、使他折磨者」}(pacato pācāpayato),菩提比丘長老英譯為\NoteKeywordBhikkhuBodhi{「當一個人拷問,或使他人施加拷問」}(when one tortures or make others inflict torture, SN/MN),Maurice Walshe先生英譯為「燒者或導致被燒」(by one who burns or causes to be burnt, DN)。按:「折磨拷打」(pacato),另譯為「煮;炊;燒;責備責難;煎熬」,而「疲累」(kilamato),菩提比丘長老英譯為\NoteKeywordBhikkhuBodhi{「虐待」}(oppresses),「悸動」(phandato)則英譯為「恐嚇」(intimidates)。
\stopitemgroup

\startitemgroup[noteitems]
\item\subnoteref{794.0}\NoteKeywordAgamaHead{「穿牆斷鎖(SA);穿牆開藏/穿牆發藏(MA);踰牆劫奪(DA)」},南傳作\NoteKeywordNikaya{「入侵人家者」}(sandhiṃ chindato,逐字譯為「連結-切斷者」),菩提比丘長老英譯為\NoteKeywordBhikkhuBodhi{「破壞房屋而進入」}(breaks into houses, SN/MN)。
\stopitemgroup

\startitemgroup[noteitems]
\item\subnoteref{795.0}\NoteKeywordAgamaHead{「輿床第五,四人持死人往塚間(SA)」},南傳作\NoteKeywordNikaya{「[四]人、長椅為第五拿取死者後走去」}(āsandipañcamā purisā mataṃ ādāya gacchanti),菩提比丘長老英譯為\NoteKeywordBhikkhuBodhi{「[四]人與棺架為第五帶走屍體」}([Four] men with the bier as fifth carry away the corpse, SN/MN)。按:南傳原經文沒有直接說是四人,只說「人們」(purisā),但因為將抬屍體的長板(棺)指為第五,間接表示抬的人數是四人,應該是四個角落各一人。又,北傳經文的「塚間」,南傳經文放在下一句,作「墓地」(āḷāhanā,另譯為「火葬場」)。
\stopitemgroup

\startitemgroup[noteitems]
\item\subnoteref{796.0}\NoteKeywordAgamaHead{「燒然已(SA)」},南傳作\NoteKeywordNikaya{「祭品成為落下的」}(bhassantā āhutiyo),菩提比丘長老英譯為\NoteKeywordBhikkhuBodhi{「用梣木燒盡牲禮」}(burnt offerings end with ashes, SN/MN)。按:《顯揚真義》等以「灰燼狀態」(bhasmantā, \suttaref{SN.24.5}/\ccchref{DN.2}{https://agama.buddhason.org/DN/dm.php?keyword=2})解說「落下的」,意即「祭品最終成了灰燼狀態」。
\stopitemgroup

\startitemgroup[noteitems]
\item\subnoteref{797.0}\NoteKeywordAgamaHead{「虛誑妄說(SA)」},南傳作\NoteKeywordNikaya{「虛妄、無價值的話」}(musā vilāpo),菩提比丘長老英譯為\NoteKeywordBhikkhuBodhi{「虛偽的無聊話;虛假空談」}(false prattle, SN/MN)。
\stopitemgroup

\startitemgroup[noteitems]
\item\subnoteref{798.0}\NoteKeywordAgamaHead{「於此十四百千生門(SA)」},南傳作\NoteKeywordNikaya{「又有這一百四十萬六千六百最上首之胎」}(cuddasa kho panimāni yonipamukhasatasahassāni saṭṭhi ca satāni cha ca),菩提比丘長老英譯為\NoteKeywordBhikkhuBodhi{「有十四百千最主要的生殖模式」}(there are fourteen hundred thousand principal modes of generation, and six thousand, and six hundre, SN/MN)。按:在\ccchref{DN.2}{https://agama.buddhason.org/DN/dm.php?keyword=2}中這是六師外道末迦利瞿舍羅(makkhali gosāla)邪命外道派的教義,「七身」說則是六師外道浮陀迦旃延(pakudhakaccāyana)的理論,長老說,也許「最上首之胎」為兩派共說的。
\stopitemgroup

\startitemgroup[noteitems]
\item\subnoteref{799.0}\NoteKeywordAgamaHead{「誓願(AA)」},南傳作\NoteKeywordNikaya{「禁戒」}(vata, bata,另譯為「誓戒」),菩提比丘長老英譯為\NoteKeywordBhikkhuBodhi{「誓約」}(vow, SN),或「遵守」(observances, AN/MN),Maurice Walshe先生英譯為「儀式;慣例」(rites, DN),即發誓願不做(或做)某些事。
\stopitemgroup

\startitemgroup[noteitems]
\item\subnoteref{800.0}\NoteKeywordAgamaHead{「八大士地(SA)」},南傳作\NoteKeywordNikaya{「人之八地」}(aṭṭhapurisabhūmiyo, aṭṭha purisabhūmiyo),菩提比丘長老英譯為\NoteKeywordBhikkhuBodhi{「人生的八階段」}(eight stages in the life of man)。按:《顯揚真義》等解說為:愚鈍地(mandabhūmi)、遊戲地(khiḍḍābhūmi)、思察地(vīmaṃsakabhūmi)、正直行為地(ujugatabhūmi)、有學地(sekhabhūmi)、沙門地(samaṇabhūmi)、識知地(jānanabhūmi)、卸下地(pannabhūmīti, \suttaref{SN.24.8}/\ccchref{MN.76}{https://agama.buddhason.org/MN/dm.php?keyword=76}/\ccchref{DN.2}{https://agama.buddhason.org/DN/dm.php?keyword=2})。「地」(bhūmi),另譯為「階段」。
\stopitemgroup

\startitemgroup[noteitems]
\item\subnoteref{801.0}\NoteKeywordAgamaHead{「法之分齊(SA);法靖(MA);總相法(DA)」},南傳作\NoteKeywordNikaya{「法的類比」}(dhammanvayo,另譯為「法的類推,法的隨行」),菩提比丘長老英譯為\NoteKeywordBhikkhuBodhi{「從法的推論」}(by inference from the Dhamma, SN),智髻比丘長老英譯為「根據法推論」(infer…according to Dhamma, MN),Maurice Walshe先生英譯為「法的趨勢」(the drift of the Dhamma, DN)。按:《顯揚真義》等以「隨行法現量智之實行後(dhammassa paccakkhato ñāṇassa anuyogaṃ anugantvā)生起的推論智、理趣的把握(anumānañāṇaṃ nayaggāho, \suttaref{SN.47.12}/\ccchref{MN.89}{https://agama.buddhason.org/MN/dm.php?keyword=89}/\ccchref{DN.28}{https://agama.buddhason.org/DN/dm.php?keyword=28})」,《破斥猶豫》以「被稱為法現量智之實行的推論、隨覺」(paccakkhañāṇasaṅkhātassa dhammassa anunayo anumānaṃ, anubuddhīti. \ccchref{MN.89}{https://agama.buddhason.org/MN/dm.php?keyword=89}) 解說。
\stopitemgroup

\startitemgroup[noteitems]
\item\subnoteref{802.0}\NoteKeywordAgamaHead{「如狗肚藏(SA);如織機相鎖(MA)」},南傳作\NoteKeywordNikaya{「變成糾纏線軸的」}(tantākulakajātā),菩提比丘長老英譯為\NoteKeywordBhikkhuBodhi{「成為像糾纏的線軸」}(become like a tangled skein, SN/AN),Maurice Walshe先生英譯為「已成為像糾纏的線球」(has become like a tangled ball of string, DN)。
\stopitemgroup

\startitemgroup[noteitems]
\item\subnoteref{803.0}\NoteKeywordAgamaHead{「如亂草蘊(SA);如蘊蔓草(MA)」},南傳作\NoteKeywordNikaya{「成為蘆草與燈心草團的」}(muñjapabbajabhūtā),菩提比丘長老英譯為\NoteKeywordBhikkhuBodhi{「像蓬亂的蘆草與燈心草」}(like matted reeds and rushes, SN),或「一團蘆草與燈心草」(a mass of reeds and rushes, AN),Maurice Walshe先生英譯為「糾纏的粗草」(tangled like coarse grass)。
\stopitemgroup

\startitemgroup[noteitems]
\item\subnoteref{804.0}\NoteKeywordAgamaHead{「當來有增長/未來世生/未來有令相續生(SA);生當來有(MA)」},南傳作\NoteKeywordNikaya{「未來再有的出生」}(āyatiṃ punabbhavābhinibbatti),菩提比丘長老英譯為\NoteKeywordBhikkhuBodhi{「未來重新存在的生產」}(the production of future renewed existence, SN),智髻比丘長老英譯為「未來生命的再有被產生」(renewal of being in the future is generated, MN)。
\stopitemgroup

\startitemgroup[noteitems]
\item\subnoteref{805.0}\NoteKeywordAgamaHead{「無量心三昧(SA);無量心解脫(MA)」},南傳作\NoteKeywordNikaya{「無量心解脫」}(appamāṇā cetovimutti),菩提比丘長老英譯為\NoteKeywordBhikkhuBodhi{「不可測量的心的釋放」}(the measureless liberation of mind)。按:《顯揚真義》等以「四梵住、四道、四果」(Cattāro brahmavihārā, cattāro maggā, cattāri phalānīti, \suttaref{SN.41.7}/\ccchref{MN.43}{https://agama.buddhason.org/MN/dm.php?keyword=43})解說。
\stopitemgroup

\startitemgroup[noteitems]
\item\subnoteref{806.0}\NoteKeywordAgamaHead{「無所有心三昧(SA)」},南傳作\NoteKeywordNikaya{「無所有心解脫」}(ākiñcaññā cetovimutti),菩提比丘長老英譯為\NoteKeywordBhikkhuBodhi{「被什麼也沒有釋放的心」}(the liberation of mind by nothingness)。按:《顯揚真義》等以「九法:無所有處、[四]道、[四]果」(nāma nava dhammā ākiñcaññāyatanaṃ maggaphalāni ca, \suttaref{SN.41.7}/\ccchref{MN.43}{https://agama.buddhason.org/MN/dm.php?keyword=43})解說。
\stopitemgroup

\startitemgroup[noteitems]
\item\subnoteref{807.0}\NoteKeywordAgamaHead{「空心三昧(SA)」},南傳作\NoteKeywordNikaya{「空心解脫」}(suññatā cetovimutti),菩提比丘長老英譯為\NoteKeywordBhikkhuBodhi{「被空釋放的心」}(the liberation of mind by emptiness),並解說,這慣用於表示基於對現象進入無我性質的毘婆舍那之定(concentration based on insight into the selfless nature of phenomena, \suttaref{SN.41.7}),以及聖道與聖果。
\stopitemgroup

\startitemgroup[noteitems]
\item\subnoteref{808.0}\NoteKeywordAgamaHead{「無諍者(SA);不動心解脫(MA);無礙心解脫(DA)」},南傳作\NoteKeywordNikaya{「不動心解脫」}(akuppā cetovimutti),菩提比丘長老英譯為\NoteKeywordBhikkhuBodhi{「不可動搖的心的釋放」}(unshakable liberation of mind)。按:《破斥猶豫》以「阿羅漢果的解脫」(arahattaphalavimutti, \ccchref{MN.29}{https://agama.buddhason.org/MN/dm.php?keyword=29}/43)解說,就與「非暫時解脫」的解說略異。
\stopitemgroup

\startitemgroup[noteitems]
\item\subnoteref{809.0}\NoteKeywordAgamaHead{「聖種(MA);賢聖族(DA)」},南傳作\NoteKeywordNikaya{「聖種姓」}(ariyavaṃsā),菩提比丘長老英譯為\NoteKeywordBhikkhuBodhi{「聖血統」}(noble lineages, \ccchref{AN.4.28}{https://agama.buddhason.org/AN/an.php?keyword=4.28}),Maurice Walshe先生英譯為「聖血統」(Ariyan lineages, \ccchref{DN.33}{https://agama.buddhason.org/DN/dm.php?keyword=33})。
\stopitemgroup

\startitemgroup[noteitems]
\item\subnoteref{810.0}\NoteKeywordAgamaHead{「節節行風/節節風(MA)」},南傳作\NoteKeywordNikaya{「隨行於四肢中的風」}(aṅgamaṅgānusārino vātā),智髻比丘長老英譯為「行經肢(手足)的風」(winds that course through the limbs)。
\stopitemgroup

\startitemgroup[noteitems]
\item\subnoteref{811.0}\NoteKeywordAgamaHead{「息出風、息入風(MA)」},南傳作\NoteKeywordNikaya{「呼吸」}(assāso passāso,另譯為「入息出息、出息入息」),智髻比丘長老英譯為「入-呼吸與出-呼吸」(in-breath and out-breath)。
\stopitemgroup

\startitemgroup[noteitems]
\item\subnoteref{812.0}\NoteKeywordNikayaHead{「金澡罐」}(bhiṅgāraṃ,另譯為「金瓶;水罐/瓶/壺」),菩提比丘長老英譯為\NoteKeywordBhikkhuBodhi{「儀式[用]花瓶」}(the ceremonial vase)。
\stopitemgroup

\startitemgroup[noteitems]
\item\subnoteref{813.0}\NoteSubKeyHead{(1)}\NoteKeywordAgamaHead{「漸損(MA)」},南傳作\NoteKeywordNikaya{「削減」}(sallekhā,另譯為「損減;漸損;儉約;制欲」),Maurice Walshe先生英譯為「自制」(restrained, DN),菩提比丘長老英譯為\NoteKeywordBhikkhuBodhi{「削除」}(effacement, AN/MN),並解說,這個詞的原始含意是指苦行或苦行的實行,佛陀用來指根本(radical)的削除或污穢(defilements)的去除。
\item\subnoteref{813.1}\NoteSubKeyHead{(2)}\NoteKeywordAgamaHead{「說漸損(MA)」},南傳作\NoteKeywordNikaya{「削減的談論」}(kathā abhisallekhikā),智髻比丘長老英譯為「[說]關切樸實生活」([talk] concerned with the austere life)。
\stopitemgroup

\startitemgroup[noteitems]
\item\subnoteref{814.0}\NoteKeywordAgamaHead{「心無蓋(MA)」},南傳作\NoteKeywordNikaya{「心離蓋」}(cetovinīvaraṇa, cetovivaraṇa),菩提比丘長老英譯為\NoteKeywordBhikkhuBodhi{「打開心」}(to opening up the heart, AN),智髻比丘長老英譯為「心的釋放」(the mind's release, MN)。按:《滿足希求》以「相應於止觀」(samathavipassanānaṃ sappāyā, \ccchref{AN.5.90}{https://agama.buddhason.org/AN/an.php?keyword=5.90})解說。
\stopitemgroup

\startitemgroup[noteitems]
\item\subnoteref{815.0}\NoteKeywordAgamaHead{「稱說閑居/說燕坐/燕坐論(MA);教他人在閑靜之處(AA)」},南傳作\NoteKeywordNikaya{「獨居論;獨居的談論」}(pavivekakathā,另譯為「遠離論;閑居論;獨住論」),菩提比丘長老英譯為\NoteKeywordBhikkhuBodhi{「說隔離;說孤獨」}(talk on solitude, AN),智髻比丘長老英譯為「說獨處」(talk on seclusion, MN)。按:「燕坐」即「獨坐」(paṭisallāṇa, paṭisallīna,另譯為「禪思;隔離;隱遁」)。
\stopitemgroup

\startitemgroup[noteitems]
\item\subnoteref{816.0}\NoteKeywordAgamaHead{「說不樂聚會(MA)」},南傳作\NoteKeywordNikaya{「不交際論;不交際談論」}(asaṃsaggakathā,另譯為「不合會說;不眾會說」),菩提比丘長老英譯為\NoteKeywordBhikkhuBodhi{「[說]不[與其他人]有束縛」}([talk] on not getting bond up [with others]),智髻比丘長老英譯為「說遠離交際」(talk on aloofness from society, MN)。
\stopitemgroup

\startitemgroup[noteitems]
\item\subnoteref{817.0}\NoteKeywordAgamaHead{「工浴人(MA);人巧浴(DA)」},南傳作\NoteKeywordNikaya{「熟練的澡堂師傅」}(dakkho nhāpako, dakkho nahāpako),菩提比丘長老英譯為\NoteKeywordBhikkhuBodhi{「熟練的澡堂[工]人」}(skillful bath man)。
\stopitemgroup

\startitemgroup[noteitems]
\item\subnoteref{818.0}\NoteKeywordNikayaHead{「鞭子已放置」}(odhastapatodo),菩提比丘長老英譯為\NoteKeywordBhikkhuBodhi{「鞭子已放好等著」}(waiting with goad lying ready)。
\stopitemgroup

\startitemgroup[noteitems]
\item\subnoteref{819.0}\NoteKeywordAgamaHead{「究竟梵行訖(MA),究竟無餘(DA)」},南傳作\NoteKeywordNikaya{「究竟完結的」}(accantapariyosāno,另譯為「究竟的最後目的」),菩提比丘長老英譯為\NoteKeywordBhikkhuBodhi{「已達到終極目標」}(have reached the ultimate goal, SN),智髻比丘長老英譯為「獲得終極的極致」(gained the ultimate consummation, MN),Maurice Walshe先生英譯為「完美地達到終極目標」(perfectly reached the goal, DN)。
\stopitemgroup

\startitemgroup[noteitems]
\item\subnoteref{820.0}\NoteKeywordAgamaHead{「內止/內行止(MA);止(AA)」},南傳作\NoteKeywordNikaya{「內心止」}(ajjhattaṃ cetosamathassa),菩提比丘長老老英譯為「心內在的平靜」(internal serenity of mind, AN/MN)。按:《滿足希求》以「定的業處」(samādhikammaṭṭhānassa, \ccchref{AN.6.16}{https://agama.buddhason.org/AN/an.php?keyword=6.16})解說,「止」(samatha),音譯為「奢摩他」。
\stopitemgroup

\startitemgroup[noteitems]
\item\subnoteref{821.0}\NoteKeywordAgamaHead{「得最上慧觀法(MA)」},南傳作\NoteKeywordNikaya{「觀法增上慧的得到者」}(lābhī…adhipaññādhammavipassanāya),菩提比丘長老英譯為\NoteKeywordBhikkhuBodhi{「獲得洞察入現象的較高智慧」}(gain the higher wisdom of insight into phenomena, AN/MN)。按:《滿足希求》以「為作、掌握觀智」(saṅkhārapariggahavipassanāñāṇassa, \ccchref{AN.9.4}{https://agama.buddhason.org/AN/an.php?keyword=9.4})解說,「觀」(vipassanāya),音譯為「毘婆舍那」。
\stopitemgroup

\startitemgroup[noteitems]
\item\subnoteref{822.0}\NoteKeywordAgamaHead{「於空靜處(MA)」},南傳作\NoteKeywordNikaya{「增益空屋者」}(brūhetā suññāgārānaṃ),菩提比丘長老英譯為英譯為「依靠空屋;訴諸空屋」(resort to empty huts, AN),智髻比丘長老英譯為「住於空屋」(dwell in empty huts, MN)。按:《破斥猶豫》等以「比丘因止觀而取業處後(samathavipassanāvasena kammaṭṭhānaṃ gahetvā, \ccchref{MN.6}{https://agama.buddhason.org/MN/dm.php?keyword=6}/\ccchref{AN.10.71}{https://agama.buddhason.org/AN/an.php?keyword=10.71})進入空屋後日夜坐著[修習]」解說。
\stopitemgroup

\startitemgroup[noteitems]
\item\subnoteref{823.0}\NoteKeywordAgamaHead{「狎習/所習/習行(MA)」},南傳作\NoteSubEntryKey{(i)}\NoteKeywordNikaya{「應該被結交;應該被追求;應該被實行」}(sevitabbampi, 原形sevati),智髻比丘長老英譯為「要交往」(to be associated with, MN),或「應該鍛鍊」(should be cultivates, \ccchref{MN.137}{https://agama.buddhason.org/MN/dm.php?keyword=137}/175),Maurice Walshe先生英譯為「要被追求」(to be pursued, \ccchref{DN.21}{https://agama.buddhason.org/DN/dm.php?keyword=21}),或「(應該)跟隨」(following, \ccchref{DN.31}{https://agama.buddhason.org/DN/dm.php?keyword=31})。菩提比丘長老解說,在上下文指的是「師長」的情況下,譯為「依止」,「衣服、施食」則譯為「使用」,「村落、城鎮、地方、地區」則譯為「住」(\ccchref{AN.10.54}{https://agama.buddhason.org/AN/an.php?keyword=10.54})。按:「結交;追求;實行」(sevati),為一廣義動詞。
\stopitemgroup

\startitemgroup[noteitems]
\item\subnoteref{824.0}\NoteKeywordAgamaHead{「彼學禪,伺、增伺、數數伺/伺、增伺而重伺(MA)」},南傳作\NoteKeywordNikaya{「思慮、強思慮、以種種思慮、從那裡扔下後思慮」}(jhāyanti pajjhāyanti nijjhāyanti apajjhāyanti),智髻比丘長老英譯為「他們默想、預-默想、出-默想、錯誤-默想」(they meditate, premeditate, outmeditate, and mismeditate, MN),菩提比丘長老英譯為\NoteKeywordBhikkhuBodhi{「他默想、思惟、沉思、深思」}(he meditates, cogitates, ponders, and ruminates, \ccchref{AN.11.9}{https://agama.buddhason.org/AN/an.php?keyword=11.9}),並解說,這些在jhāyati加上字首pa, ni, ava/apa的動詞是「修禪」(jhāyati)的同義詞,難適當地譯成英文,其含有「輕視與嘲諷」的意思。《破斥猶豫》以「思慮」解說jhāyanti(Jhāyantīti cintayanti),以「因災禍而使增大的」解說pajjhāyanti等(Pajjhāyantītiādīni upasaggavasena vaḍḍhitāni, \ccchref{MN.50}{https://agama.buddhason.org/MN/dm.php?keyword=50}),《正法光明》分別以「思慮、強思慮、以種種思慮、從那裡扔下後思慮」(cinteti, bhusaṃ cinteti, anekavidhena cinteti, tato apagantvā cinteti, \ccchref{Ni.7}{https://agama.buddhason.org/Ni/dm.php?keyword=7}.53)解說,今準此譯。
\stopitemgroup

\startitemgroup[noteitems]
\item\subnoteref{825.0}\NoteSubKeyHead{(1)}\NoteKeywordAgamaHead{「依貪著(SA);依著/依家(MA)」},南傳作\NoteKeywordNikaya{「依存於家的」}(gehasitāni, gehanissitāni,另譯為「掛慮家的;執著家的」),菩提比丘長老英譯為\NoteKeywordBhikkhuBodhi{「基於有家生活的」}(based on the household life, SN/MN)。按:《破斥猶豫》等以「依止於五種欲/依止於欲類」(pañcakāmaguṇanissitā/kāmaguṇanissitāni, \ccchref{MN.21}{https://agama.buddhason.org/MN/dm.php?keyword=21}/137),或「依存於在家的五種欲」(pañcakāmaguṇagehanissitaṃ, \suttaref{SN.35.94})解說,而以「渴愛的意欲、嫌惡的意欲」(taṇhāchandāpi paṭighachandāpi, \ccchref{MN.21}{https://agama.buddhason.org/MN/dm.php?keyword=21})解說「欲」。
\item\subnoteref{825.1}\NoteSubKeyHead{(2)}\NoteKeywordAgamaHead{「依離貪著(SA);依無欲(MA)」},南傳作\NoteKeywordNikaya{「依存於離欲」}(nekkhammasitāni),菩提比丘長老英譯為\NoteKeywordBhikkhuBodhi{「基於放棄」}(based on renunciation, SN/MN)。按:《破斥猶豫》以「依止於毘婆舍那者」(vipassanānissitāni, \ccchref{MN.137}{https://agama.buddhason.org/MN/dm.php?keyword=137})解說。「離欲」(nekkhamma),另譯為「出離」(放棄世俗生活)。
\stopitemgroup

\startitemgroup[noteitems]
\item\subnoteref{826.0}\NoteKeywordAgamaHead{「處非處/是處非處(MA)」},南傳作\NoteKeywordNikaya{「可能不可能」}(ṭhānāṭhāna,另譯為「處非處」),智髻比丘長老英譯為「什麼是可能什麼是不可能」(what is possible and what is impossible, MN),菩提比丘長老英譯為\NoteKeywordBhikkhuBodhi{「什麼是可能與不可能」}(wha t is possible and impossible, \ccchref{AN.4.158}{https://agama.buddhason.org/AN/an.php?keyword=4.158}),或「他的立場與相反立場」(his position and the opposing position, \ccchref{AN.3.68}{https://agama.buddhason.org/AN/an.php?keyword=3.68})。按:《滿足希求》以「有理由沒理由」(kāraṇākāraṇe)解說,並舉例:恆常論者以合適的理由(yuttena kāraṇena)駁斥斷滅論者,當斷滅論者被駁斥時想:我為何說斷滅論?因而就恆常論的情況解說(sassatavādibhāvameva dīpeti, \ccchref{AN.3.68}{https://agama.buddhason.org/AN/an.php?keyword=3.68}),不能確立自己的教義,就是這個情形。「處」(ṭhāna),另譯為「場所;住處;狀態;點;理由;原因;道理」。
\stopitemgroup

\startitemgroup[noteitems]
\item\subnoteref{827.0}\NoteKeywordAgamaHead{「本修行願(SA);本所行、本所思(MA)」},南傳作\NoteKeywordNikaya{「被造作的、被思惟的」}(abhisaṅkhato abhisañcetayito),智髻比丘長老英譯為「為條件所支配與意志生產的」(is conditioned and volitionally produced, MN),菩提比丘長老英譯為\NoteKeywordBhikkhuBodhi{「被意志所建構與生產的」}(is constructed and produced by volition, AN),或「被意志產生的與形成的」(generated and fashioned by volition, \suttaref{SN.12.37}),並解說,此段這是指反觀(paṭivipassanā, counter-insight),即對行使觀功能之識的行為應用觀的原理,基於此而到達阿羅漢的狀態。按:《顯揚真義》等以「被緣所作的」(paccayehi katoti, \suttaref{SN.12.37}),或「以緣會合後所作」(paccayehi abhisamāgantvā kataṃ, \suttaref{SN.35.146}),或「被做的、使被生起的」(kataṃ uppāditaṃ, \ccchref{MN.52}{https://agama.buddhason.org/MN/dm.php?keyword=52}/\ccchref{AN.11.16}{https://agama.buddhason.org/AN/an.php?keyword=11.16})解說「被造作的」,以「思為基礎的、思作為根本的」(cetanāvatthuko cetanāmūlakoti, \suttaref{SN.12.37}),或「被思計畫(安排)的」(cetanāya pakappitaṃ, \suttaref{SN.35.146}/\ccchref{MN.52}{https://agama.buddhason.org/MN/dm.php?keyword=52}/\ccchref{AN.11.16}{https://agama.buddhason.org/AN/an.php?keyword=11.16})解說「被思惟的」。
\stopitemgroup

\startitemgroup[noteitems]
\item\subnoteref{828.0}\NoteSubKeyHead{(1)}\NoteKeywordAgamaHead{「五比丘(SA/MA/AA)」},南傳作\NoteKeywordNikaya{「五位一群的比丘們」}(pañcavaggiye bhikkhū,逐字譯為「五+群-比丘」),菩提比丘長老英譯為\NoteKeywordBhikkhuBodhi{「五位成一群的比丘們」}(the bhikkus of the group of five)。
\item\subnoteref{828.1}\NoteSubKeyHead{(2)}\NoteKeywordNikayaHead{「六位一群」}(chabbaggiyā,逐字譯為「六群」),智髻比丘長老英譯為「我們六個」(the six of us)。
\stopitemgroup

\startitemgroup[noteitems]
\item\subnoteref{829.0}\NoteKeywordAgamaHead{「欲法、愛法、樂法、靖法(MA)」},南傳作\NoteKeywordNikaya{「仍以那個法貪、那個法歡喜」}(teneva dhammarāgena tāya dhammanandiyā),智髻比丘長老英譯為「因為法的想要,在法中歡樂」(because of that desire for the Dhamma, that delight in the Dhamma, MN),菩提比丘長老英譯為\NoteKeywordBhikkhuBodhi{「因為法的慾望,因為在法中歡樂」}(because of that lust for the Dhamma, because of that that delight in the Dhamma, AN)。按:《破斥猶豫》等說,兩者都是「在止觀法上的欲貪」(samathavipassanādhamme chandarāgena, \ccchref{MN.52}{https://agama.buddhason.org/MN/dm.php?keyword=52}/\ccchref{AN.9.36}{https://agama.buddhason.org/AN/an.php?keyword=9.36})。
\stopitemgroup

\startitemgroup[noteitems]
\item\subnoteref{830.0}\NoteKeywordAgamaHead{「斷支(MA)」},南傳作\NoteKeywordNikaya{「勤奮支」}(padhāniyaṅgāni),智髻比丘長老英譯為「努力要素」(factors of striving, MN),菩提比丘長老英譯為\NoteKeywordBhikkhuBodhi{「協助努力的要素」}(factors that assist striving, AN),Maurice Walshe先生英譯為「努力因素」(factors of endeavour, DN)。
\stopitemgroup

\startitemgroup[noteitems]
\item\subnoteref{831.0}\NoteKeywordNikayaHead{「承受勤奮的」}(padhānakkhamāya),智髻比丘長老英譯為「能承受努力的壓力」(able to bear the strain of striving, MN),菩提比丘長老英譯為\NoteKeywordBhikkhuBodhi{「適合努力」}(suitable for striving, AN),Maurice Walshe先生英譯為「適合努力」(suitable for exertion, DN)。
\stopitemgroup

\startitemgroup[noteitems]
\item\subnoteref{832.0}\NoteKeywordAgamaHead{「現如真(MA)」},南傳作\NoteKeywordNikaya{「如實不誇大自己」}(yathābhūtaṃ attānaṃ āvikattā),菩提比丘長老英譯為\NoteKeywordBhikkhuBodhi{「如其事實地顯露他自己」}(one who reveals himself as he actually is, AN),智髻比丘長老英譯為「如其事實地展現他自己」(shows himself as he actually is, MN)。
\stopitemgroup

\startitemgroup[noteitems]
\item\subnoteref{833.0}\NoteKeywordAgamaHead{「別知相(SA);識識(MA)」},南傳作\NoteKeywordNikaya{「識知」}(vijānāti,動詞,另譯為「了知;了別」,過去分詞viññāta-已識知,名詞viññāṇa-識),菩提比丘長老英譯為\NoteKeywordBhikkhuBodhi{「認知」}(cognize)。附:動詞「了知」(pajānāti),菩提比丘長老英譯為\NoteKeywordBhikkhuBodhi{「理解」}(understand)。
\stopitemgroup

\startitemgroup[noteitems]
\item\subnoteref{834.0}\NoteKeywordAgamaHead{「愛盡解脫(SA/MA);愛盡得解脫/愛盡正善心解脫(MA);為愛所苦身得滅(DA);斷於愛欲心得解脫/斷欲心得解脫(AA)」},南傳作\NoteKeywordNikaya{「渴愛之滅盡解脫的」}(taṇhāsaṅkhayavimutto),菩提比丘長老英譯為\NoteKeywordBhikkhuBodhi{「在渴愛的消滅上自由的」}(liberated in the extinction of craving, SN/AN),智髻比丘長老/Maurice Walshe先生英譯為「以渴愛的破壞自由的」(are liberated by the destruction of craving, MN/DN),或「渴愛之滅盡的解脫」(taṇhāsaṅkhayavimuttiṃ),智髻比丘長老英譯為「經由渴愛的破壞的自由」(deliverance through the destruction of craving, MN)。
\stopitemgroup

\startitemgroup[noteitems]
\item\subnoteref{835.0}\NoteKeywordAgamaHead{「猶如加羅毘伽/猶如迦羅毘伽(MA);如迦羅頻伽鳥聲(DA)」},南傳作\NoteKeywordNikaya{「美聲鳥誦出者」}(karavīkabhāṇī, karavikabhāṇī,另譯為「迦陵頻伽聲的;美聲的」),智髻比丘長老英譯為「像叫作Karavīka鳥的聲音」(voice like the call of the Karavīka bird, MN),Maurice Walshe先生英譯為「像Karavīka鳥的聲音」(voice like that of the karavīka-bird, DN)。
\stopitemgroup

\startitemgroup[noteitems]
\item\subnoteref{836.0}\NoteKeywordAgamaHead{「樂於業(MA);好多為(DA);著事務(AA)」},南傳作\NoteKeywordNikaya{「喜於工作」}(kammārāmatā,kammārāma-喜於工作者),Maurice Walshe先生英譯為「歡喜於工作」(rejoice…in works),菩提比丘長老英譯為\NoteKeywordBhikkhuBodhi{「樂於工作」}(delight in work)。
\stopitemgroup

\startitemgroup[noteitems]
\item\subnoteref{837.0}\NoteKeywordAgamaHead{「習業者(MA);修世榮(AA)」},南傳作\NoteKeywordNikaya{「致力喜於工作者」}(kammārāmatamanuyuttā),Maurice Walshe先生英譯為「成為專注於工作」(become absorbed in works, DN),菩提比丘長老英譯為\NoteKeywordBhikkhuBodhi{「投入在工作中欣喜」}(are not devoted to delight in work, AN)。
\stopitemgroup

\startitemgroup[noteitems]
\item\subnoteref{838.0}\NoteKeywordAgamaHead{「不為暫爾不為德勝(MA)」},南傳作\NoteKeywordNikaya{「不以低劣量殊勝的到達來到終結的中途」}(na oramattakena visesādhigamena antarāvosānaṃ āpajjissanti),Maurice Walshe先生英譯為「不以部分的成就為滿足而歇息」(do not rest content with partial achievements),菩提比丘長老英譯為\NoteKeywordBhikkhuBodhi{「不因為一些小特質的成就而[在他們的發展中]來到中途的停止」}(do not come to a stop midway [in their development] on account of some minor achievement of distinction)。
\stopitemgroup

\startitemgroup[noteitems]
\item\subnoteref{839.0}\NoteKeywordAgamaHead{「於地作小想(MA);觀地無地想(AA)」},南傳作\NoteKeywordNikaya{「小地想」}(parittā pathavīsaññā),菩提比丘長老英譯為\NoteKeywordBhikkhuBodhi{「地的有限認知」}(a limited perception of earth, AN)。
\stopitemgroup

\startitemgroup[noteitems]
\item\subnoteref{840.0}\NoteSubKeyHead{(1)}\NoteKeywordAgamaHead{「一劫有餘(DA)」},南傳作\NoteKeywordNikaya{「一劫的剩餘」}(kappāvasesaṃ),菩提比丘長老英譯為\NoteKeywordBhikkhuBodhi{「一劫的剩餘」}(the remainder of an eon, AN)。按:當七日出時(\ccchref{MA.8}{https://agama.buddhason.org/MA/dm.php?keyword=8}等),光音天以下的世界全被大火災毀滅(\ccchref{DA.30}{https://agama.buddhason.org/DA/dm.php?keyword=30}〈三災品〉),此間眾生全亡,之後世間轉回(\ccchref{MA.154}{https://agama.buddhason.org/MA/dm.php?keyword=154}等)。
\item\subnoteref{840.1} (2)「劫住的」(kappaṭṭho),「住」(ṭho),為「住立的;在~的;存續的」之意,依上項,「劫住」即「持續到劫末(世界毀滅)為止的」。
\stopitemgroup

\startitemgroup[noteitems]
\item\subnoteref{841.0}\NoteKeywordAgamaHead{「顛倒受解(MA)」},南傳作\NoteKeywordNikaya{「以惡把握的」}(duggahitena),智髻比丘長老英譯為「以你的錯誤把握」(by your wrong grasp, MN),Maurice Walshe先生英譯為「錯誤把握」(grasps at wrong, DN),菩提比丘長老英譯為\NoteKeywordBhikkhuBodhi{「誤解」}(misunderstanding, \ccchref{AN.2.23}{https://agama.buddhason.org/AN/an.php?keyword=2.23}),或「不良獲得」(badly acquired, AN),或「不良學習的」(badly learned, \ccchref{AN.4.180}{https://agama.buddhason.org/AN/an.php?keyword=4.180})。
\stopitemgroup

\startitemgroup[noteitems]
\item\subnoteref{842.0}\NoteSubKeyHead{(1)}\NoteKeywordAgamaHead{「根本火(SA);恭敬火(GA)」},南傳作\NoteKeywordNikaya{「應該被奉獻者之火」}(āhuneyyaggi,另譯為「應被供食者之火;應請火」),菩提比丘長老英譯為\NoteKeywordBhikkhuBodhi{「那些值得贈與者之火」}(the fire of those worthy of gifts)。
\item\subnoteref{842.1}\NoteSubKeyHead{(2)}\NoteKeywordAgamaHead{「居家火(SA);苦樂俱火(GA)」},南傳作\NoteKeywordNikaya{「屋主之火」}(gahapataggi,另譯為「居士火」),菩提比丘長老英譯為\NoteKeywordBhikkhuBodhi{「屋主的火」}(the householder's fire)。
\item\subnoteref{842.2}\NoteSubKeyHead{(3)}\NoteKeywordAgamaHead{「田火(SA);福田火(GA)」},南傳作\NoteKeywordNikaya{「應該被供養者之火」}(dakkhiṇeyyaggi,另譯為「應施火」),菩提比丘長老英譯為\NoteKeywordBhikkhuBodhi{「應該供養者之火」}(the fire of those worthy of offerings)。
\stopitemgroup

\startitemgroup[noteitems]
\item\subnoteref{843.0}\NoteSubKeyHead{(1)}\NoteKeywordAgamaHead{「一向論(MA);決定記論(DA)」},南傳作\NoteKeywordNikaya{「應該被一向回答」}(ekaṃsabyākaraṇīyaṃ),菩提比丘長老英譯為\NoteKeywordBhikkhuBodhi{「應該明確回答」}(should be answered categorically)。
\item\subnoteref{843.1}\NoteSubKeyHead{(2)}\NoteKeywordAgamaHead{「分別論(MA);分別記論(DA)」},南傳作\NoteKeywordNikaya{「分別後應該被回答的」}(vibhajjabyākaraṇīyaṃ),菩提比丘長老依錫蘭版英譯為「應該作出區別後回答」(should be answered after making a distinction)。
\item\subnoteref{843.2}\NoteSubKeyHead{(3)}\NoteKeywordAgamaHead{「詰論(MA);詰問記論(DA)」},南傳作\NoteKeywordNikaya{「應該被以反問回答的」}(paṭipucchābyākaraṇīyaṃ),菩提比丘長老英譯為\NoteKeywordBhikkhuBodhi{「應該以一個反-問題回答」}(should be answered with a counter-question)。
\item\subnoteref{843.3}\NoteSubKeyHead{(4)}\NoteKeywordAgamaHead{「止論(MA);止住記論(DA)」},南傳作\NoteKeywordNikaya{「應該被擱置」}(ṭhapanīyaṃ),菩提比丘長老英譯為\NoteKeywordBhikkhuBodhi{「應該被擱置」}(should be set aside, AN)。
\stopitemgroup

\startitemgroup[noteitems]
\item\subnoteref{844.0}\NoteKeywordAgamaHead{「定力(MA);三昧具/定具(DA)」},南傳作\NoteKeywordNikaya{「定的資助」}(samādhiparikkhārā),智髻比丘長老英譯為「集中貫注的裝備」(the equipment of concentration, MN),Maurice Walshe先生英譯為「集中貫注的要件」(requisites of concentration, DN)。按:《破斥猶豫》說,因四正勤四個應被作的之完成(catukiccasādhanavaseneva)而生起的活力(vīriyaṃ),以立於隨從(parivāraṭṭhena)而為其資助(\ccchref{MN.44}{https://agama.buddhason.org/MN/dm.php?keyword=44}),《吉祥悅意》則以三種資助解說:(i)戒與活力為裝飾資助(alaṅkāro parikkhāro)(ii)正見等七個善資助為隨從資助(parivāro parikkhāro)(iii)生病活命的需要物為資糧資助(sambhāro parikkhāro, \ccchref{DN.18}{https://agama.buddhason.org/DN/dm.php?keyword=18})。
\stopitemgroup

\startitemgroup[noteitems]
\item\subnoteref{845.0}\NoteKeywordAgamaHead{「倚身行(AA)」},南傳作\NoteKeywordNikaya{「身行已寧靜者」}(passaddhakāyasaṅkhāro,逐字譯為「輕安(猗)+身+行」),菩提比丘長老英譯為\NoteKeywordBhikkhuBodhi{「身體活動的寧靜」}(tranquilized the bodily activity, AN),Maurice Walshe先生英譯為「他的情感已寧靜」(has tranquillised his emotions, \ccchref{DN.33}{https://agama.buddhason.org/DN/dm.php?keyword=33}),並解說這裡的「身」是指「名身」(mental body)。
\stopitemgroup

\startitemgroup[noteitems]
\item\subnoteref{846.0}\NoteKeywordAgamaHead{「或有所堪(MA)」},南傳作\NoteKeywordNikaya{「考量後忍受一事」}(saṅkhāyekaṃ adhivāseti),智髻比丘長老英譯為「深思後忍受另一件事」(endures another thing after reflecting, MN),菩提比丘長老英譯為\NoteKeywordBhikkhuBodhi{「深思後耐心忍受其他事」}(having reflected, patiently endures other thing, AN),Maurice Walshe先生英譯為「判斷而忍受一事」(judges that one thing endured, DN)。
\stopitemgroup

\startitemgroup[noteitems]
\item\subnoteref{847.0}\NoteKeywordAgamaHead{「何由(SA);云何而得稱遂其心(GA);何益(DA)」},南傳作\NoteKeywordNikaya{「在這裡,那如何可得」}(taṃ kutettha labbhāti),Maurice Walshe先生英譯為「[窩藏怨恨]有甚麼好處?」(What good would it do [to harbour malice]?, DN),菩提比丘長老英譯為\NoteKeywordBhikkhuBodhi{「能對它做什麼呢」}( What can be done about it, AN)。按:此句片語的含意是:那也不能如何,就接受這個現實吧。
\stopitemgroup

\startitemgroup[noteitems]
\item\subnoteref{848.0}\NoteKeywordAgamaHead{「止諍(MA)」},南傳作\NoteKeywordNikaya{「諍訟的止息」}(adhikaraṇasamathā,另譯為「滅諍」),智髻比丘長老英譯為「訴訟的解決」(settlement of litigation, MA),Maurice Walshe先生英譯為「爭議的解決」(settlement of disputed, DA)。
\stopitemgroup

\startitemgroup[noteitems]
\item\subnoteref{849.0}\NoteSubKeyHead{(1)}\NoteKeywordAgamaHead{「得沙門(MA)」},南傳作\NoteKeywordNikaya{「平等性」}(sāmaññāya, samāna-ya),菩提比丘長老英譯為\NoteKeywordBhikkhuBodhi{「一致;融合」}(accord)。按:此字若解為samaṇa-ya即「沙門性」,《滿足希求》說應依「沙門法之義」(samaṇadhammatthāya, \ccchref{AN.8.2}{https://agama.buddhason.org/AN/an.php?keyword=8.2}),或「沙門法之性」samaṇadhammabhāvāya, \ccchref{AN.10.87}{https://agama.buddhason.org/AN/an.php?keyword=10.87})解,與北傳所譯相應。
\item\subnoteref{849.1}\NoteSubKeyHead{(2)}\NoteKeywordAgamaHead{「得一意/得一心(MA)」},南傳作\NoteKeywordNikaya{「一致性」}(ekībhāvāya,另譯為「一性;一趣」),菩提比丘長老英譯為\NoteKeywordBhikkhuBodhi{「單一;一致;合一」}(unity)。按:《滿足希求》以「無間斷性;不間斷性」(nirantarabhāvāya, \ccchref{AN.10.87}{https://agama.buddhason.org/AN/an.php?keyword=10.87})解說。
\stopitemgroup

\startitemgroup[noteitems]
\item\subnoteref{850.0}\NoteKeywordNikayaHead{「無十之事」}(niddasavatthūni,另譯為「殊妙事」),Maurice Walshe先生英譯為「讚許」(commendation, DN),菩提比丘長老英譯為\NoteKeywordBhikkhuBodhi{「[成為]『十-無』的基礎」}(bases for [being] 'ten-less.', \ccchref{AN.7.20}{https://agama.buddhason.org/AN/an.php?keyword=7.20})。按:《吉祥悅意》說,無十(Niddaso)、無二十(nibbīso)……無五十(nippaññāso)為存在於外道宗派的問題,在外道十年死時的死者,耆那教說明不再有十年,九年……不再有一年,尊者阿難在村中聽說後告訴世尊,世尊告訴阿難說:「在我的教說中這是諸漏已盡的同義語(khīṇāsavassetaṃ adhivacanaṃ, \ccchref{AN.7.20}{https://agama.buddhason.org/AN/an.php?keyword=7.20}),因為在十年死時的諸漏已盡證涅槃者不再有十年,九年……不再有一年,因為不再以結生存在。」
\stopitemgroup

\startitemgroup[noteitems]
\item\subnoteref{851.0}\NoteKeywordAgamaHead{「持[資糧](SA);懷[嫉妒](MA)」},南傳作\NoteKeywordNikaya{「維繫[旅途的資糧];懷[嫉妒/嫌隙怨恨]」}(bandhati,動詞,原意為「繫縛;結合;組成」),Maurice Walshe先生英譯為「[怨恨]被激起」([Malice] is stirred up by, DN),菩提比丘長老英譯為\NoteKeywordBhikkhuBodhi{「保證[資糧]」}(secures[provisions], \suttaref{SN.1.79}),或「窩藏[怨恨]」(harbors[resentment], \ccchref{AN.9.29}{https://agama.buddhason.org/AN/an.php?keyword=9.29})。按:\ccchref{AN.9.29}{https://agama.buddhason.org/AN/an.php?keyword=9.29}的「懷」\ccchref{Ni.15}{https://agama.buddhason.org/Ni/dm.php?keyword=15},174偈就用「被生起」(jāyati)。
\stopitemgroup

\startitemgroup[noteitems]
\item\subnoteref{852.0}\NoteKeywordAgamaHead{「異諦(DA)」},南傳作\NoteKeywordNikaya{「各自真理」}(paccekasacco,逐字譯為「單獨的+諦」),Maurice Walshe先生英譯為「個別的信仰」(individual beliefs)。
\stopitemgroup

\startitemgroup[noteitems]
\item\subnoteref{853.0}\NoteKeywordNikayaHead{「洞察分」}(nibbedhabhāgiyo,另譯為「抉擇分」),菩提比丘長老英譯為\NoteKeywordBhikkhuBodhi{「參與洞察」}(partakes of penetration, SN),或「關於洞察」(pertain to penetration, AN),Maurice Walshe先生英譯為「有利於洞察」(conducive to penetration)。按:「分」(bhāgiyo),另譯為「有分的;部分的;與…連接的;有益於…的」。
\stopitemgroup

\startitemgroup[noteitems]
\item\subnoteref{854.0}\NoteKeywordAgamaHead{「宿責食(SA)」},南傳作\NoteKeywordNikaya{「{有諍}[負債]地吃國家施食」}(saraṇo raṭṭhapiṇḍaṃ bhuñjiṃ),菩提比丘長老依錫蘭本(sāṇo raṭṭhapiṇḍaṃ bhuñjiṃ)英譯為「我像個負債者吃國家(地方)的施捨之食品」(I ate the country's almsfood as a debtor)。按:《顯揚真義》說「有污染成為有負債」(sakileso saiṇo hutvā),國家施食指「有信者的可拿取物」(saddhādeyyaṃ),四類受用,i.盜取受用(theyyaparibhogo):無德者(dussīlassa)在僧團中受用。ii.負債受用(iṇaparibhoga):不省察受用的有德者(Sīlavato)。iii.繼承人受用(dāyajjaparibhoga):七類有學人的受用。iv.主人受用;所有者的受用(sāmiparibhoga):諸漏已滅盡者無負債受用(\suttaref{SN.16.11})。
\stopitemgroup

\startitemgroup[noteitems]
\item\subnoteref{855.0}\NoteSubKeyHead{(1)}\NoteKeywordAgamaHead{「如如/如(SA)」},南傳作\NoteKeywordNikaya{「真實的/真實性」}(tathāni/tathatā),菩提比丘長老英譯為\NoteKeywordBhikkhuBodhi{「真實(的)」}(actual/actuality)。
\item\subnoteref{855.1}\NoteSubKeyHead{(2)}\NoteKeywordAgamaHead{「不離如(SA)」},南傳作\NoteKeywordNikaya{「無誤的/無誤性」}(avitathāni/avitathatā,逐字譯為「不+離+如+性」),菩提比丘長老英譯為\NoteKeywordBhikkhuBodhi{「無誤(的)」}(unerring/inerrancy)。
\item\subnoteref{855.2}\NoteSubKeyHead{(3)}\NoteKeywordAgamaHead{「不異如(SA)」},南傳作\NoteKeywordNikaya{「無例外的/無例外性」}(anaññathāni/anaññathatā,逐字譯為「不+異+如+性」),菩提比丘長老英譯為\NoteKeywordBhikkhuBodhi{「非其它(的)」}(not otherwise/not otherwiseness)。按:《顯揚真義》說,「真實性」等,就是緣所作(paccayākārasseva, \suttaref{SN.12.20})的同義語。
\stopitemgroup

\startitemgroup[noteitems]
\item\subnoteref{856.0}\NoteKeywordNikayaHead{「從能被一切漏住立的法」}(sabbaso āsavaṭṭhāniyehi dhammehi),菩提比丘長老英譯為\NoteKeywordBhikkhuBodhi{「污點之基礎的所有事」}(with all things that are a basis for the taints, AN),Maurice Walshe先生英譯為「諸敗壞」(the corruptions, DN)。按:《滿足希求》以「一切三界法」(sabbehi tebhūmakadhammehīti, \ccchref{AN.8.28}{https://agama.buddhason.org/AN/an.php?keyword=8.28})解說,「能被一切漏住立的」則以「諸漏相應而成為原因,污染法之意」(sampayogavasena āsavānaṃ kāraṇabhūtehi, kilesadhammehīti attho, \ccchref{AN.8.28}{https://agama.buddhason.org/AN/an.php?keyword=8.28})解說。
\stopitemgroup

\startitemgroup[noteitems]
\item\subnoteref{857.0}\NoteKeywordNikayaHead{「大欲求者;大欲求的;大欲求性」}(mahiccho, mahicchassa, mahicchatā),菩提比丘長老英譯為\NoteKeywordBhikkhuBodhi{「帶著強烈想要」}(with strong desires, AN),或「有強烈想要」(has strong desires, \ccchref{AN.6.84}{https://agama.buddhason.org/AN/an.php?keyword=6.84}),Maurice Walshe先生英譯為「許多想要」(many wants, DN)。按:《滿足希求》以「大貪」(mahālobho, \ccchref{AN.1.62}{https://agama.buddhason.org/AN/an.php?keyword=1.62})解說。
\stopitemgroup

\startitemgroup[noteitems]
\item\subnoteref{858.0}\NoteKeywordAgamaHead{「緣利有分(MA);因利有用(DA)」},南傳作\NoteKeywordNikaya{「緣於得到而有判斷」}(lābhaṃ paṭicca vinicchayo),Maurice Walshe先生英譯為「獲得成為決定的條件」(acquisition conditions decision-making, \ccchref{DN.15}{https://agama.buddhason.org/DN/dm.php?keyword=15}),坦尼沙羅比丘長老英譯為「確定依賴於獲得」(ascertainment is dependent on acquisition, \ccchref{DN.15}{https://agama.buddhason.org/DN/dm.php?keyword=15}),菩提比丘長老英譯為\NoteKeywordBhikkhuBodhi{「依獲得而有判斷」}(In dependence on gain there is judgment, AN)。按:《吉祥悅意》等說,有智、渴愛、見、尋(ñāṇataṇhādiṭṭhivitakkavasena)等四種判斷,緣於得到而有判斷只來到尋(vitakkoyeva āgato):得到所獲得的後判定合不合意、美妙不美妙之尋;多少將成為我的色所緣利益(rūpārammaṇatthāya)、多少將成為有信者的色所緣利益;多少將成為我的、多少將成為其他人的;我將受用多少、我將貯蓄多少(\ccchref{DN.15}{https://agama.buddhason.org/DN/dm.php?keyword=15}/\ccchref{AN.9.23}{https://agama.buddhason.org/AN/an.php?keyword=9.23})。
\stopitemgroup

\startitemgroup[noteitems]
\item\subnoteref{859.0}\NoteKeywordNikayaHead{「道理的感受者」}(atthapaṭisaṃvedī),Maurice Walshe先生英譯為「他獲得教導精神的把握」(he gains a grasp of the spirit the teaching, DN),菩提比丘長老英譯為\NoteKeywordBhikkhuBodhi{「他經驗意義裡的靈感」}(he experiences inspiration in the meaning, \ccchref{AN.5.26}{https://agama.buddhason.org/AN/an.php?keyword=5.26}),並解說,paṭisaṃvedī的語基為vedī,這樣atthapaṭisaṃvedī與atthaveda(道理的信受)有明顯地連接,而字根vid與「明」(vijjā)、「感受」(vedanā)相關,所以建議veda理解為使「欣悅、喜」生起的「靈感智」(inspired knowledge),或「靈感」(inspiration)。按:《吉祥悅意》等以「知道經典(巴利)的道理」(pāḷiatthaṃ jānantassa, \ccchref{DN.33}{https://agama.buddhason.org/DN/dm.php?keyword=33}/\ccchref{AN.5.26}{https://agama.buddhason.org/AN/an.php?keyword=5.26})解說。
\stopitemgroup

\startitemgroup[noteitems]
\item\subnoteref{860.0}\NoteKeywordAgamaHead{「最為尊貴/獨一無侶/無能勝者(DA)」},南傳作\NoteKeywordNikaya{「最上的創造者」}(seṭṭho sajitā/sañjitā/sajjitā),智髻比丘長老英譯為「最高的神(上帝)」(Most High Providence, MN),菩提比丘長老英譯為\NoteKeywordBhikkhuBodhi{「注定者;制定者」}(the Ordainer, \ccchref{DN.1}{https://agama.buddhason.org/DN/dm.php?keyword=1}),Maurice Walshe先生英譯為「統治者」(Ruler, DN)。按:《吉祥悅意》等以「最上者與創造者」(uttamo ca sajitā ca, \ccchref{MN.49}{https://agama.buddhason.org/MN/dm.php?keyword=49}/\ccchref{DN.1}{https://agama.buddhason.org/DN/dm.php?keyword=1})解說。
\stopitemgroup

\startitemgroup[noteitems]
\item\subnoteref{861.0}\NoteKeywordNikayaHead{「脫離形色的」}(virattarūpā),智髻比丘長老英譯為「沮喪的」(dismayed, MN),菩提比丘長老英譯為\NoteKeywordBhikkhuBodhi{「變得冷淡」}(become indifferent, AN),Maurice Walshe先生英譯為「不高興的」(displeased, DN)。按:《破斥猶豫》等以「離情愛的」(vigatapemā, \ccchref{MN.104}{https://agama.buddhason.org/MN/dm.php?keyword=104}/\ccchref{DN.29}{https://agama.buddhason.org/DN/dm.php?keyword=29})解說。
\stopitemgroup

\startitemgroup[noteitems]
\item\subnoteref{862.0}\NoteKeywordNikayaHead{「為了無瞋害的最大限度」}(abyābajjhaparamatāya),智髻比丘長老英譯為「為了好的健康之利益」(for the benefit of good health, MN),菩提比丘長老英譯為\NoteKeywordBhikkhuBodhi{「維持他的健康」}(to sustain his health, AN)。
\stopitemgroup

\startitemgroup[noteitems]
\item\subnoteref{863.0}\NoteKeywordNikayaHead{「死後無病」}(arogo paraṃ maraṇā),Maurice Walshe先生英譯為「死後是健康的」(after death is healthy, DN),菩提比丘長老英譯為\NoteKeywordBhikkhuBodhi{「死後是無損傷的」}(is unimpaired after death, SN/MN)。按:《破斥猶豫》等說,無病為常(Arogoti nicco, \ccchref{MN.102}{https://agama.buddhason.org/MN/dm.php?keyword=102}/\ccchref{DN.1}{https://agama.buddhason.org/DN/dm.php?keyword=1}),或說,這是關於出生在遍淨天界(idaṃ so subhakiṇhadevaloke nibbattakkhandhe sandhāya, \ccchref{MN.79}{https://agama.buddhason.org/MN/dm.php?keyword=79})。
\stopitemgroup

\startitemgroup[noteitems]
\item\subnoteref{864.0}\NoteKeywordAgamaHead{「一佉梨(SA);一佉利(GA)」},南傳作\NoteKeywordNikaya{「一佉梨重」}(khāri),菩提比丘長老英譯為\NoteKeywordBhikkhuBodhi{「綑綁著」}(bundles of, \suttaref{SN.3.11}),或「量」(measures, \ccchref{AN.10.89}{https://agama.buddhason.org/AN/an.php?keyword=10.89}),Maurice Walshe先生英譯為「扁擔」(carrying-pole, DN)。按:此為古印度計重量單位,尤其是秤穀物,也指八種沙門的生活必需品。
\stopitemgroup

\startitemgroup[noteitems]
\item\subnoteref{865.0}\NoteKeywordNikayaHead{「對無惡婆羅門身分之人的尊敬者」}(apāpapurekkhāro brahmaññāya pajāya,逐字譯為「無+惡+尊敬者-婆羅門-人們」),智髻比丘長老英譯為「他對婆羅門血統不尋求任何傷害」(he does not seek any harm for the line of brahmins),Maurice Walshe先生英譯為「尊敬無可責難的婆羅門生活方式」(honouring the blameless Brahmin way of life, DN)。
\stopitemgroup

\startitemgroup[noteitems]
\item\subnoteref{866.0}\NoteKeywordAgamaHead{「不了憎惡行(MA)」},南傳作\NoteKeywordNikaya{「增上嫌惡」}(adhijegucche),Maurice Walshe先生英譯為「極嚴格生活」(extreme austerity, DN)。按:《吉祥悅意》以「以活力處於惡的嫌惡狀態」(vīriyena pāpajigucchanabhāve, \ccchref{DN.25}{https://agama.buddhason.org/DN/dm.php?keyword=25})解說。
\stopitemgroup

\startitemgroup[noteitems]
\item\subnoteref{867.0}\NoteKeywordAgamaHead{「不了可憎行(MA)」},南傳作\NoteKeywordNikaya{「苦行與嫌惡」}(tapojigucchā),Maurice Walshe先生英譯為「較高級嚴格生活」(the higher austerities, DN),菩提比丘長老英譯為\NoteKeywordBhikkhuBodhi{「嚴格生活與厭惡」}(austerity and disgust, AN)。按:《吉祥悅意》以「惡的嫌惡、惡的回避之活力」(vīriyena pāpajigucchā pāpavivajjanā, \ccchref{DN.25}{https://agama.buddhason.org/DN/dm.php?keyword=25}),《滿足希求》以「被稱為難行的苦行、惡的嫌惡(之因)」(dukkarakārikasaṅkhātena tapena pāpajigucchanahetu, \ccchref{AN.4.196}{https://agama.buddhason.org/AN/an.php?keyword=4.196})解說。
\stopitemgroup

\startitemgroup[noteitems]
\item\subnoteref{868.0}\NoteKeywordNikayaHead{「尊貴王族戰士」}(uggā rājaputtā,逐字譯為「高級的-王+子」),Maurice Walshe先生英譯為「優勝者與高級軍官」(champions and senior officers, DN),菩提比丘長老英譯為\NoteKeywordBhikkhuBodhi{「ugga種姓戰士」}(ugga-caste warriors, \ccchref{AN.7.67}{https://agama.buddhason.org/AN/an.php?keyword=7.67}),並解說,一般ugga是指父親是剎帝利種姓,母親是śudra(sudda,首陀羅)種姓的混血種姓,具有憤怒氣質(wrathful temperament)。按:《吉祥悅意》等以「經常出入戰場一再晉升的王子」(uggatuggatā saṅgāmāvacarā rājaputtā, \ccchref{DN.2}{https://agama.buddhason.org/DN/dm.php?keyword=2}/\ccchref{AN.7.67}{https://agama.buddhason.org/AN/an.php?keyword=7.67})解說。
\stopitemgroup

\startitemgroup[noteitems]
\item\subnoteref{869.0}\NoteKeywordNikayaHead{「大龍戰士」}(mahānāgā),Maurice Walshe先生英譯為「英雄」(heroes, DN),菩提比丘長老英譯為\NoteKeywordBhikkhuBodhi{「大公牛戰士」}(great-bull warriors, AN)。按:《滿足希求》等以「如大龍,這是面對象等到來也不退轉的戰士之同義語」(viya mahānāgā, hatthiādīsupi abhimukhaṃ āgacchantesu anivattitayodhānametaṃ adhivacanaṃ, \ccchref{DN.2}{https://agama.buddhason.org/DN/dm.php?keyword=2}/\ccchref{AN.7.67}{https://agama.buddhason.org/AN/an.php?keyword=7.67})解說。
\stopitemgroup

\startitemgroup[noteitems]
\item\subnoteref{870.0}\NoteKeywordAgamaHead{「當試四月/乃至四月(SA);當留四月觀察(DA)」},南傳作\NoteKeywordNikaya{「別住四個月」}(cattāro māse parivasati),Maurice Walshe先生英譯為「等四個月」(wait four months, DN)。按:此項規制,如《摩訶僧祇律》所記載:「若外道來欲出家者,當共住試之四月。」但也有世尊即刻同意即出家的,如\ccchref{SA.102}{https://agama.buddhason.org/SA/dm.php?keyword=102}、\ccchref{MA.153}{https://agama.buddhason.org/MA/dm.php?keyword=153}所說。
\stopitemgroup

\startitemgroup[noteitems]
\item\subnoteref{871.0}\NoteSubKeyHead{(1)}\NoteKeywordAgamaHead{「念佛;念如來事(SA);佛念(DA)」},南傳作\NoteKeywordNikaya{「佛隨念」}(Buddhānussati),菩提比丘長老英譯為\NoteKeywordBhikkhuBodhi{「佛的回憶」}(Recollection of the Buddha)。按:anussati為動詞也為名詞,今在動詞的場合譯為「回憶」,在名詞的場合譯為「隨念」。
\item\subnoteref{871.1}\NoteSubKeyHead{(2)}\NoteKeywordAgamaHead{「念僧;念比丘僧(SA/AA);念眾(MA/AA);憶念眾(MA);僧念(DA)」},南傳作\NoteKeywordNikaya{「僧團隨念」}(saṅghānussati),菩提比丘長老英譯為\NoteKeywordBhikkhuBodhi{「僧團的回憶」}(Recollection of the Saṅgha)。
\item\subnoteref{871.2}\NoteSubKeyHead{(3)}\NoteKeywordAgamaHead{「念施;施念(DA)」},南傳作\NoteKeywordNikaya{「施捨隨念」}(cāgānussati),菩提比丘長老英譯為\NoteKeywordBhikkhuBodhi{「慷慨的回憶;回憶慷慨」}(Recollection of the generosity, AN)。
\stopitemgroup

\startitemgroup[noteitems]
\item\subnoteref{872.0}\NoteSubKeyHead{(1)}\NoteKeywordNikayaHead{「色之思」}(Rūpasañcetanā),菩提比丘長老英譯為\NoteKeywordBhikkhuBodhi{「對色的意志力」}(volition regarding forms, SN)。
\item\subnoteref{872.1}\NoteSubKeyHead{(2)}\NoteKeywordNikayaHead{「色之尋」}(Rūpavitakko),Maurice Walshe先生英譯為「景象的『想』」('Thinking' of sights, DN),菩提比丘長老英譯為\NoteKeywordBhikkhuBodhi{「關於色的心思」}(thought about forms, AN)。
\item\subnoteref{872.2}\NoteSubKeyHead{(3)}\NoteKeywordNikayaHead{「色之伺」}(Rūpavicāro),Maurice Walshe先生英譯為「景色的『沉思』」('Pondering' on sights, DN),菩提比丘長老英譯為\NoteKeywordBhikkhuBodhi{「色的檢查」}(examination of forms, AN)。
\stopitemgroup

\startitemgroup[noteitems]
\item\subnoteref{873.0}\NoteKeywordNikayaHead{「已完成、已受持」}(samattaṃ samādinnaṃ),菩提比丘長老英譯為\NoteKeywordBhikkhuBodhi{「慫恿與著手」}(instigates and undertakes, AN),智髻比丘長老英譯為「採取與實踐」(taken up and practised, MN)。按:《破斥猶豫》等以「做完後掌握」(paripuṇṇaṃ katvā gahitaṃ, \ccchref{MN.57}{https://agama.buddhason.org/MN/dm.php?keyword=57}/\ccchref{AN.1.306}{https://agama.buddhason.org/AN/an.php?keyword=1.306})解說。
\stopitemgroup

\startitemgroup[noteitems]
\item\subnoteref{874.0}\NoteKeywordAgamaHead{「苦臥(SA);惱不眠(GA);苦眠(MA);不得眠(AA)」},南傳作\NoteKeywordNikaya{「睡不好」}(dukkhaṃ seti, Dukkhaṃ supati,逐字譯為「苦-臥(動詞)」),菩提比丘長老英譯為\NoteKeywordBhikkhuBodhi{「睡不好」}(sleeps badly, SN/AN),或「垂頭喪氣」(is dejected, \ccchref{AN.4.70}{https://agama.buddhason.org/AN/an.php?keyword=4.70})。
\stopitemgroup

\startitemgroup[noteitems]
\item\subnoteref{875.0}\NoteKeywordAgamaHead{「心中懷憎嫉(MA)」},南傳作\NoteKeywordNikaya{「邪惡意向的」}(paduṭṭhamanasaṅkappo),智髻比丘長老英譯為「憎恨的意向」(intentions of hate, MN),菩提比丘長老英譯為\NoteKeywordBhikkhuBodhi{「帶著被憎恨腐化的意向」}(with intentions corrupted by hate, SN)。按:《顯揚真義》以「憤怒的心如有銳利角的公牛」(tikhiṇasiṅgo viya goṇo duṭṭhacitto, \suttaref{SN.22.80})解說,《破斥猶豫》以「具備對他人產生不利的心之意向」(abhadrakena paresaṃ anatthajanakena cittasaṅkappena samannāgatāti, \ccchref{MN.4}{https://agama.buddhason.org/MN/dm.php?keyword=4}),或「由瞋而有憤怒心之意向」(dosena duṭṭhacittasaṅkappo, \ccchref{MN.41}{https://agama.buddhason.org/MN/dm.php?keyword=41})解說。
\stopitemgroup

\startitemgroup[noteitems]
\item\subnoteref{876.0}\NoteKeywordNikayaHead{「有分者」}(bhāgī, bhāgino),菩提比丘長老英譯為\NoteKeywordBhikkhuBodhi{「分享」}(partakes, AN),或「經驗」(experience, SN),或「獲得」(reap, AN/SN)。
\stopitemgroup

\startitemgroup[noteitems]
\item\subnoteref{877.0}\NoteKeywordNikayaHead{「未被摻雜」}(asaṃkiṇṇā, asaṅkiṇṇā),菩提比丘長老英譯為\NoteKeywordBhikkhuBodhi{「未被摻假」}(unadulterated)。
\stopitemgroup

\startitemgroup[noteitems]
\item\subnoteref{878.0}\NoteKeywordAgamaHead{「貪欲(AA)」},南傳作\NoteKeywordNikaya{「婬欲法」}(methunaṃ dhammaṃ,另譯為「婬法;不淨法」),菩提比丘長老英譯為\NoteKeywordBhikkhuBodhi{「性交」}(has sexual intercourse)。
\stopitemgroup

\startitemgroup[noteitems]
\item\subnoteref{879.0}\NoteKeywordNikayaHead{「以及以那個來到幸福」}(tena ca vittiṃ āpajjati),菩提比丘長老英譯為\NoteKeywordBhikkhuBodhi{「在它之中找到滿足」}(finds satisfaction in it, AN),智髻比丘長老英譯為「經由他找到滿足」(finds satisfaction through him, MN)。
\stopitemgroup

\startitemgroup[noteitems]
\item\subnoteref{880.0}\NoteKeywordAgamaHead{「梵身天(SA/GA/MA/DA)」},南傳作\NoteKeywordNikaya{「梵眾天」}(brahmakāyikānaṃ,另譯為「梵身天」),菩提比丘長老英譯為\NoteKeywordBhikkhuBodhi{「梵天的伙伴之天」}(the devas of Brahmā's company)。
\stopitemgroup

\startitemgroup[noteitems]
\item\subnoteref{881.0}\NoteKeywordNikayaHead{「對作有依著的福德者來說」}(Karotaṃ opadhikaṃ puññaṃ),菩提比丘長老英譯為\NoteKeywordBhikkhuBodhi{「作在獲得[再生]上成熟的功績」}(making merit that ripens in acquisitions, AN),或「實行世俗型態的功績」(Performing merit of the mundane type, SN)。按:《顯揚真義》等以「依著果報的福德」(upadhivipākaṃ puññaṃ, \suttaref{SN.11.16}/\ccchref{AN.8.59}{https://agama.buddhason.org/AN/an.php?keyword=8.59})解說。「依著」(upadhi),另譯為「依戀;再生的基質;執著」,「依著果報的福德」即善的往生業。 
\stopitemgroup

\startitemgroup[noteitems]
\item\subnoteref{882.0}\NoteKeywordAgamaHead{「嶮惡世平等(SA)」},南傳作\NoteKeywordNikaya{「在不平順中平順地行」}(caranti visame samanti),菩提比丘長老英譯為\NoteKeywordBhikkhuBodhi{「在不平坦中平坦地過活」}(Fare evenly amidst the uneven)。按:「不平順」(visama),另譯為「不正的;非理的;險難的;不平等的;不調和的」,《顯揚真義》舉「世間共住的不平順、眾生身的不平順、污染生起的不平順」(visame vā lokasannivāse visame vā sattanikāye visame vā kilesajāte, \suttaref{SN.1.7})解說,「平順」(sama),另譯為「正確的;正當的;平等的;寂靜的」。
\stopitemgroup

\startitemgroup[noteitems]
\item\subnoteref{883.0}\NoteKeywordAgamaHead{「無量處/無量(SA);無量法(GA);不可量(MA)」},南傳作\NoteKeywordNikaya{「不能被測量的」}(Appameyyaṃ),智髻比丘長老英譯為「不能被測量」(cannot be measured, MN),菩提比丘長老英譯為\NoteKeywordBhikkhuBodhi{「一位不可測量者」}(An immeasurable one, SN)。按:《顯揚真義》以「諸漏已盡的人(khīṇāsavapuggalaṃ, \suttaref{SN.6.7})[不]這麼測量『這麼多戒、這麼多定、這麼多慧』」,《破斥猶豫》以「(大地)四方無限」(tiriyaṃ pana aparicchinnā, \ccchref{MN.21}{https://agama.buddhason.org/MN/dm.php?keyword=21}),或「不能掌握量」(pamāṇaṃ gaṇhituṃ na sakkuṇeyyo, \ccchref{MN.72}{https://agama.buddhason.org/MN/dm.php?keyword=72})解說。
\stopitemgroup

\startitemgroup[noteitems]
\item\subnoteref{884.0}\NoteKeywordAgamaHead{「別梵天(SA)」},南傳作\NoteKeywordNikaya{「辟支梵天」}(paccekabrahmā,另譯為「獨一的梵天;單獨的梵天」),菩提比丘長老英譯為\NoteKeywordBhikkhuBodhi{「獨立的梵天」}(independent brahmā)。按:註疏以獨行(ekacārī)、沒群眾隨行(na parisacārī)的梵天解說(\suttaref{SN.6.6})。
\stopitemgroup

\startitemgroup[noteitems]
\item\subnoteref{885.0}\NoteSubKeyHead{(1)}\NoteKeywordNikayaHead{「瘤」}(abbudaṃ),菩提比丘長老英譯為\NoteKeywordBhikkhuBodhi{「瘟疫」}(a plague)。按:《顯揚真義》以「破壞的因素,盜賊在世間破壞的意思」(vināsakāraṇaṃ, corā lokasmiṃ vināsakāti attho, \suttaref{SN.1.77})解說。
\item\subnoteref{885.1}\NoteSubKeyHead{(2)}\NoteKeywordNikayaHead{「胞」}(abbudaṃ),菩提比丘長老英譯照錄原文。按:《顯揚真義》以「為洗肉水的色澤」(maṃsadhovanaudakavaṇṇaṃ, \suttaref{SN.10.1})形容這個時期的胚胎。
\item\subnoteref{885.2}\NoteSubKeyHead{(3)}\NoteKeywordNikayaHead{「阿浮陀」}(abbudā)、「尼羅浮多獄;尼羅浮陀;尼羅浮;尼羅部陀」(nirabbudānaṃ),菩提比丘長老說,這是時間單位:阿浮陀=20x10的42次方年,尼羅部陀=20阿浮陀(koṭi=1千萬=10^7, pakoṭi=koṭi of koṭi=10^14, koṭipakoṭi=koṭi of pakoṭi=10^21, nahuta=koṭi of koṭipakoṭi=10^28, ninnahuta=koṭi of nahuta=10^35, abbuda=koṭi of ninnahuta=10^42)。按:此亦為地獄名稱,參看\ccchref{SA.1278}{https://agama.buddhason.org/SA/dm.php?keyword=1278}。
\stopitemgroup

\startitemgroup[noteitems]
\item\subnoteref{886.0}\NoteKeywordAgamaHead{「第一有(MA)」},南傳作\NoteKeywordNikaya{「有之頂點;有的第一」}(bhavaggaṃ),菩提比丘長老英譯為\NoteKeywordBhikkhuBodhi{「存在的頂點」}(the pinnacle of existence, SN),或「生存狀態中的第一」(the foremost among states of existence, AN)。按:此指「非想非非想處」。
\stopitemgroup

\startitemgroup[noteitems]
\item\subnoteref{887.0}\NoteKeywordAgamaHead{「有我,有此世,有他世(SA);此是神,此是世,此是我(MA)」},南傳作\NoteKeywordNikaya{「彼是我者彼即是世間」}(so attā so loko, so loko so attā),菩提比丘長老英譯為\NoteKeywordBhikkhuBodhi{「那自我者即是世界」}(That which is the self is the world, SN),智髻比丘長老英譯為「這是自我,這是世界」(This is self, this the world, MN)。
\stopitemgroup

\startitemgroup[noteitems]
\item\subnoteref{888.0}\NoteSubKeyHead{(1)}\NoteKeywordNikayaHead{「請你安頓」}(saṇṭhapehi),菩提比丘長老英譯為\NoteKeywordBhikkhuBodhi{「穩固」}(Steady)。按:《顯揚真義》以「使完全地住立」(sammā ṭhapehi, \suttaref{SN.21.1})解說。
\item\subnoteref{888.1}\NoteSubKeyHead{(2)}\NoteKeywordNikayaHead{「我使之平靜」}(sannisādemi),智髻比丘長老英譯為「靜止」(quieted, MN)。
\item\subnoteref{888.2}\NoteSubKeyHead{(3)}\NoteKeywordNikayaHead{「我作專一」}(ekodiṃ karomi),菩提比丘長老英譯為\NoteKeywordBhikkhuBodhi{「統一」}(unify, SN),智髻比丘長老英譯為「帶它到單一」(brought it to singleness, MN)。按:《顯揚真義》以「令作為一點」(ekaggaṃ karohi, \suttaref{SN.21.1})解說。
\item\subnoteref{888.3}\NoteSubKeyHead{(4)}\NoteKeywordNikayaHead{「請你集中」}(samādaha),菩提比丘長老英譯為\NoteKeywordBhikkhuBodhi{「集中」}(concentrate)。
\stopitemgroup

\startitemgroup[noteitems]
\item\subnoteref{889.0}\NoteKeywordNikayaHead{「[生命]旅途的遍知」}(Addhānapariññāya),菩提比丘長老英譯為\NoteKeywordBhikkhuBodhi{「完全理解路程」}(the full understanding of the course)。按:《顯揚真義》以「輪迴為旅途,到達涅槃後為遍知」(saṃsāraddhānaṃ nibbānaṃ patvā pariññātaṃ nāma hoti, \suttaref{SN.45.48})解說。
\stopitemgroup

\startitemgroup[noteitems]
\item\subnoteref{890.0}\NoteKeywordAgamaHead{「大闇地獄(SA);無有障蔽(MA)」},南傳作\NoteKeywordNikaya{「世界中間空無防護的」}(lokantarikā aghā asaṃvutā),Maurice Walshe先生英譯為「位於超越世界的盡頭」(lie beyond the world's end, DN),菩提比丘長老英譯為\NoteKeywordBhikkhuBodhi{「世界空隙、空缺、幽暗的地區」}(world interstices, vacant and abysmal regions , SN),智髻比丘長老英譯為「無底的世界中間空虛處」(abysmal world interspaces of vacancy, MN)。按:《破斥猶豫》說,每三個鐵圍山中間的一個世界間隙,而那個世界間隙有8000由旬大(So pana lokantarikanirayo parimāṇato aṭṭhayojanasahasso hoti, \ccchref{MN.123}{https://agama.buddhason.org/MN/dm.php?keyword=123})。《起世經》〈地獄品〉說:「此鐵圍外復有一重大鐵圍山……於兩山間有八大地獄。」
\stopitemgroup

\startitemgroup[noteitems]
\item\subnoteref{891.0}\NoteKeywordNikayaHead{「已切斷虛妄、已切斷路徑」}(chinnapapañce chinnavaṭume),菩提比丘長老英譯為\NoteKeywordBhikkhuBodhi{「切斷增殖,切斷轍跡」}(cut through proliferation, cut through the rut, SN),智髻比丘長老英譯為「切斷增殖[的糾結],打破循環」(cut [the tangle of] proliferation, broke the cycle, MN),Maurice Walshe先生英譯為「已切斷障礙,已切斷[渴愛之]路」(having cut away the hindrances, cut off the road [of craving])。按:《顯揚真義》以「已切斷渴愛虛妄(taṇhāpapañcassa)、渴愛路徑(taṇhāvaṭumasseva, \suttaref{SN.35.83}),《破斥猶豫》等以「渴愛、慢、見三種污染」(taṇhā māno diṭṭhīti ime tayo kilesā, \ccchref{MN.123}{https://agama.buddhason.org/MN/dm.php?keyword=123}/\ccchref{DN.14}{https://agama.buddhason.org/DN/dm.php?keyword=14})解說。「虛妄」(papañca),另譯為「障礙;戲論;迷執;妄想」。
\stopitemgroup

\startitemgroup[noteitems]
\item\subnoteref{892.0}\NoteKeywordNikayaHead{「很多七寶的」}(pahūtarattaratanānaṃn, pahūtasattaratanānaṃn),菩提比丘長老英譯為\NoteKeywordBhikkhuBodhi{「豐富的珍貴物」}(abounding in the seven precious substances)。
\stopitemgroup

\startitemgroup[noteitems]
\item\subnoteref{893.0}\NoteSubKeyHead{(1)}\NoteKeywordAgamaHead{「冥運持命去(SA);人生壽不定,日日趣死徑(GA)」},南傳作\NoteKeywordNikaya{「生命被帶走」}(Upanīyati jīvitam),菩提比丘長老英譯為\NoteKeywordBhikkhuBodhi{「生命被席捲」}(Life is swept along, SN/AN)。按:「被帶走;該被帶走」(Upanīyati/Upanīya);另譯為「被導引;被導近」,《顯揚真義》以「消盡、滅,或靠近、次第到達死亡」(parikkhīyati nirujjhati, upagacchati vā, anupubbena maraṇaṃ upetīti, \suttaref{SN.1.3})解說。
\item\subnoteref{893.1}\NoteSubKeyHead{(2)}\NoteKeywordAgamaHead{「樂往至樂所(SA);作福得趣樂(GA)」},南傳作\NoteKeywordNikaya{「應該作帶來樂的福德」}(puññāni kayirātha sukhāvahānī),菩提比丘長老英譯為\NoteKeywordBhikkhuBodhi{「人們應該作帶來快樂的功績之行為」}(One should do deeds of merit that bring happiness, SN/AN)。
\item\subnoteref{893.2}\NoteSubKeyHead{(3)}\NoteKeywordAgamaHead{「無餘涅槃樂(SA);欲得寂滅樂(GA)」},南傳作\NoteKeywordNikaya{「期待寂靜者」}(santipekkho),菩提比丘長老英譯為\NoteKeywordBhikkhuBodhi{「平和的尋求者」}(A seeker of peace)。按:寂靜,《顯揚真義》以「究竟寂靜被稱為涅槃(nibbānasaṅkhātaṃ accantasantiṃ, \suttaref{SN.1.3})」解說。
\stopitemgroup

\startitemgroup[noteitems]
\item\subnoteref{894.0}\NoteSubKeyHead{(1)}\NoteKeywordAgamaHead{「我是」(Asmī’ti),菩提比丘長老英譯為「我是」(I am)。按:《顯揚真義》以「不離被稱為渴愛、慢、見的妄想(taṇhāmānadiṭṭhisaṅkhātaṃ papañcattayaṃ, \suttaref{SN.22.47})」解說「我是」;以「常見」(sassatadiṭṭhivasena)解說肯定形態的「我將是;我當有(MA)」};以「斷見」(ucchedadiṭṭhivasena)解說否定形態的「我將不是」。長老認為一般都將此歸於慢,《顯揚真義》的解說,或依\suttaref{SN.22.89}之「『我是』之慢、『我是』之欲、『我是』之煩惱潛在趨勢」而說。
\item\subnoteref{894.1}\NoteSubKeyHead{(2)}\NoteKeywordNikayaHead{「我是這個」}(ayamahamasmī’ti),菩提比丘長老英譯為\NoteKeywordBhikkhuBodhi{「我是這個」}(I am this)。按:《顯揚真義》以「我見」(attadiṭṭhivasena, \suttaref{SN.22.47})解說。
\stopitemgroup

\startitemgroup[noteitems]
\item\subnoteref{895.0}\NoteKeywordNikayaHead{「智語與上座語」}(ñāṇavādañca… theravādañca),智髻比丘長老英譯為「理解與自信」(knowledge and assurance, MN),坦尼沙羅比丘長老英譯為「理解之語,前輩之語」(the words of knowledge, the words of the elders, MN)。
\stopitemgroup

\startitemgroup[noteitems]
\item\subnoteref{896.0}\NoteSubKeyHead{(1)}\NoteKeywordAgamaHead{「齒齒相著(MA)」},南傳作\NoteKeywordNikaya{「緊扣牙齒」}(dantebhi dantamādhāya,逐字譯為「(以)齒+強力+放置」),智髻比丘長老英譯為「牙齒緊咬」(teeth clenched)。
\item\subnoteref{896.1}\NoteSubKeyHead{(2)}\NoteKeywordAgamaHead{「以心修心(MA)」},南傳作\NoteKeywordNikaya{「以心抑止、壓迫、破壞心」}(cetasā cittaṃ abhiniggaṇheyyaṃ abhinippīḷeyyaṃ abhisantāpeyyanti),智髻比丘長老英譯為「我以心打下來、限制和粉碎心」(I beat down, constrain, and crush mind with mind)。按:《破斥猶豫》說,應該以善心抑止不善心(kusalacittena akusalacittaṃ abhiniggaṇhitabbaṃ, \ccchref{MN.20}{https://agama.buddhason.org/MN/dm.php?keyword=20}/36)。
\stopitemgroup

\startitemgroup[noteitems]
\item\subnoteref{897.0}\NoteSubKeyHead{(1)}\NoteKeywordAgamaHead{「種種行處(SA);異行/異境界(MA);性行各異(AA)」},南傳作\NoteKeywordNikaya{「不同行境」}(nānāgocarānaṃ),菩提比丘長老英譯為\NoteKeywordBhikkhuBodhi{「不同的領域」}(different domains, SN),智髻比丘長老英譯為「分開的領域」(a separate domain, MN)。
\item\subnoteref{897.1}\NoteSubKeyHead{(2)}\NoteKeywordAgamaHead{「種種境界(SA);異境界(MA);所行不同(AA)」},南傳作\NoteKeywordNikaya{「不同境域」}(nānāvisayānaṃ),菩提比丘長老英譯為\NoteKeywordBhikkhuBodhi{「不同的牧養地」}(different feeding grounds, \suttaref{SN.35.247}),或「不同的去處」(different resorts, \suttaref{SN.48.42}),智髻比丘長老英譯為「分開的範圍」(a separate field, MN)。
\stopitemgroup

\startitemgroup[noteitems]
\item\subnoteref{898.0}\NoteKeywordAgamaHead{「我唐煩勞/唐煩勞我(MA);更生觸擾(DA)」},南傳作\NoteKeywordNikaya{「那對我是疲勞,那對我是傷害」}(So mamassa kilamatho, sā mamassa vihesā’ti),智髻比丘長老英譯為「對我疲勞與麻煩」(wearying and troublesome for me, MN),菩提比丘長老英譯為\NoteKeywordBhikkhuBodhi{「那對我會是疲倦的,那會是麻煩的」}(that would be wearisome for me, that would be troublesome, SN),Maurice Walshe先生英譯為「那對我會是疲倦與麻煩」(that would be a weariness and a trouble to me, DN)。
\stopitemgroup

\startitemgroup[noteitems]
\item\subnoteref{899.0}\NoteKeywordAgamaHead{「六趣(SA);六生(DA)」},南傳作\NoteKeywordNikaya{「在六等級中;六等級」}(chasvevābhijātīsu, chaḷabhijātīsu, chaḷabhijātiyo),Maurice Walshe先生英譯為「六種再生」(the six kinds of rebirth, DN),菩提比丘長老英譯為\NoteKeywordBhikkhuBodhi{「六等級」}(six classes, SN/AN)。按:\ccchref{AN.6.57}{https://agama.buddhason.org/AN/an.php?keyword=6.57}有「六等級」的說明,《破斥猶豫》(\ccchref{MN.60}{https://agama.buddhason.org/MN/dm.php?keyword=60})亦同。
\stopitemgroup

\startitemgroup[noteitems]
\item\subnoteref{900.0}\NoteKeywordNikayaHead{「七無結胎」}(satta nigaṇṭhigabbhā),智髻比丘長老英譯為「七種無結者的領域」(seven spheres of knotless ones, MN),Maurice Walshe先生英譯為「七種『從束縛自由』的生命」(seven as beings 'freed from bonds', DN)。按:《顯揚真義》等以「以結為生胎的,說關於甘蔗、竹子、蘆葦等」(gaṇṭhimhi jātagabbhā, ucchuveḷunaḷādayo sandhāya vadati, \suttaref{SN.24.8}/\ccchref{MN.76}{https://agama.buddhason.org/MN/dm.php?keyword=76}/\ccchref{DN.2}{https://agama.buddhason.org/DN/dm.php?keyword=2})解說「無結胎」。
\stopitemgroup

\startitemgroup[noteitems]
\item\subnoteref{901.0}\NoteKeywordAgamaHead{「往來經歷(SA)」},南傳作\NoteKeywordNikaya{「流轉輪迴後」}(sandhāvitvā saṃsaritvā),菩提比丘長老英譯為\NoteKeywordBhikkhuBodhi{「漫遊與流浪」}(roaming and wandering, AN),Maurice Walshe先生英譯為「運行與循環繞圈」(run on and circle round, DN)。
\stopitemgroup

\startitemgroup[noteitems]
\item\subnoteref{902.0}\NoteKeywordAgamaHead{「至地自住(SA)」},南傳作\NoteKeywordNikaya{「當被解開時它就逃走」}(nibbeṭhiyamānameva paleti),菩提比丘長老英譯為\NoteKeywordBhikkhuBodhi{「它逃離,不纏繞」}(it run away unwinding, SN),Maurice Walshe先生英譯為「跑到它全解開」(runs till it is all unravelled, DN)。
\stopitemgroup

\startitemgroup[noteitems]
\item\subnoteref{903.0}\NoteKeywordAgamaHead{「諍知此義(MA)」},南傳作\NoteKeywordNikaya{「詰難效益」}(upārambhānisaṃsañca),菩提比丘長老英譯為\NoteKeywordBhikkhuBodhi{「在辯論與譴責[其他理論]中」}(in debate and condemning [the theses of others], SN),智髻比丘長老英譯為「為了批評他人」(for the sake of criticising others, MN)。
\stopitemgroup

\startitemgroup[noteitems]
\item\subnoteref{904.0}\NoteKeywordAgamaHead{「非為趣為人說(MA)」},南傳作\NoteKeywordNikaya{「非為了諂媚人們」}(na janalapanatthaṃ),智髻比丘長老英譯為「非為了諂媚人們之目的」( not for the purpose of flattering people, MN),菩提比丘長老英譯為\NoteKeywordBhikkhuBodhi{「不為了哄騙人」}(not…for the sake of cajoling people, AN)。
\stopitemgroup

\startitemgroup[noteitems]
\item\subnoteref{905.0}\NoteKeywordNikayaHead{「像那樣自由說話效益」}(itivādappamokkhānisaṃsā),菩提比丘長老英譯為\NoteKeywordBhikkhuBodhi{「在辯論中贏的利益」}(the benefit of winning in debates, AN),或「救他們的理論利益」(benefits of rescuing their own theses, SN),智髻比丘長老英譯為「在辯論中贏」(winning in debates, MN)。
\stopitemgroup

\startitemgroup[noteitems]
\item\subnoteref{906.0}\NoteSubKeyHead{(1)}\NoteKeywordAgamaHead{「無為無求(MA);無為(AA)」},南傳作\NoteKeywordNikaya{「不活動的;無為者;無關心的;放心的」}(appossukke),菩提比丘長老英譯為\NoteKeywordBhikkhuBodhi{「放心生活;悠閒生活」}(live at ease, SN/MN),或「保持靜止」(keeping still, \suttaref{SN.35.240}),或「放心」(be at ease, \ccchref{AN.8.70}{https://agama.buddhason.org/AN/an.php?keyword=8.70}),或「不活動」(inactive, MN)。
\item\subnoteref{906.1}\NoteSubKeyHead{(2)}\NoteKeywordAgamaHead{「行無為(MA)」},南傳作\NoteKeywordNikaya{「住於不活動的」}(appossukko…viharassu),智髻比丘長老英譯為「住於不活動」(to abide inactive, MN)。
\stopitemgroup

\startitemgroup[noteitems]
\item\subnoteref{907.0}\NoteKeywordAgamaHead{「澡豆(MA/DA/AA)」},南傳作\NoteKeywordNikaya{「擦背用具與肥皂粉」}(sottisināniṃ,逐字譯為「洗背刷+沐浴粉」),智髻比丘長老英譯為「絲瓜布與洗澡粉」(a loofah and bath powder)。
\stopitemgroup

\startitemgroup[noteitems]
\item\subnoteref{908.0}\NoteKeywordAgamaHead{「異(SA)」},南傳作\NoteKeywordNikaya{「除了」}(aññatra),菩提比丘長老英譯為\NoteKeywordBhikkhuBodhi{「除了……之外」}(apart from)。按:「異」(aññatra),有兩個意思:一是「在其他處;其他的」(如\ccchref{SA.105}{https://agama.buddhason.org/SA/dm.php?keyword=105}的「異見、異忍、異求、異欲」等),一是「除了~(之外);~以外」(等於ṭhapetvā,PTS英巴辭典作leaving out, setting aside, excepting),前者如「異見」,指「其它見解」,即外道的見解,後者如「異信」,指「除了從信外;除了以信外」,《顯揚真義》以「除~(vinā)、排除後(apanetvā, \suttaref{SN.35.153})」解說,若將「異信」譯為「不是信、不依信、不靠信、其他信」均非恰當。
\stopitemgroup

\startitemgroup[noteitems]
\item\subnoteref{909.0}\NoteKeywordAgamaHead{「諍本(MA/DA)」},南傳作\NoteKeywordNikaya{「諍論根」}(vivādamūlāni),智髻比丘長老英譯為「爭論的根」(roots of disputes, MN),Maurice Walshe先生英譯為「爭論的根」(roots of contention, DN)。
\stopitemgroup

\startitemgroup[noteitems]
\item\subnoteref{910.0}\NoteKeywordAgamaHead{「面前止諍律/面前律(MA)」},南傳作\NoteKeywordNikaya{「面前毘尼」}(sammukhāvinayo,另譯為「面前律」),智髻比丘長老英譯為「以面對面除去訴訟」(removal of litigation by confrontation, MN),Maurice Walshe先生英譯為「面對面的訴訟程序」(proceedings face-to-face, DN)。
\stopitemgroup

\startitemgroup[noteitems]
\item\subnoteref{911.0}\NoteKeywordAgamaHead{「不癡止諍律/不癡律(MA)」},南傳作\NoteKeywordNikaya{「不癡毘尼」}(amūḷhavinayo,另譯為「不癡律」),智髻比丘長老英譯為「因過去瘋狂除去訴訟」(removal of litigation on account of past insanity, MN),Maurice Walshe先生英譯為「精神紊亂」(mental derangement, DN)。
\stopitemgroup

\startitemgroup[noteitems]
\item\subnoteref{912.0}\NoteKeywordAgamaHead{「君止諍律/居(君?)/君律(MA)」},南傳作\NoteKeywordNikaya{「多數決」}(yebhuyyasikā,另譯為「多覓毘尼;多人語」),智髻比丘長老英譯為「多數意見」(the opinion of the majority, MN),Maurice Walshe先生英譯為「多數裁決」(majority verdict, DN)。按:「君」有「主宰;統治」義,與「多數決」相順。五分律、根有律作「多人語」,僧祇律、四分律、十誦律作「多覓罪相」。
\stopitemgroup

\startitemgroup[noteitems]
\item\subnoteref{913.0}\NoteKeywordAgamaHead{「展轉止諍律/展轉(MA);本言治(五分律)」},南傳作\NoteKeywordNikaya{「覓罪相」}(tassapāpiyasikā; tassapāpiyyasikā,另譯為「求彼罪」),Maurice Walshe先生英譯為「習慣性的壞品格」(habitual bad character, DN),智髻比丘長老英譯為「對某人壞品格的宣告」(the pronouncement of bad character against someone, MN)。
\stopitemgroup

\startitemgroup[noteitems]
\item\subnoteref{914.0}\NoteKeywordAgamaHead{「如棄糞掃止諍律/如棄糞掃止諍法(MA)」},南傳作\NoteKeywordNikaya{「草覆蓋」}(tiṇavatthārako,另譯為「草伏地止諍律」),智髻比丘長老英譯為「以草覆蓋」(covering over with grass, MN)。按:《破斥猶豫》說,凡落入連續諍訟(adhikaraṇaṃ mūlānumūlaṃ),當解決時,粗惡性、棘手性的不和合轉起,就以此羯磨平息,如以草覆蓋糞便(duggandhatāya bādhati-阻擋惡臭)一樣,有隱藏的平息(paṭicchannaṃ vūpasantaṃ hotīti, \ccchref{MN.104}{https://agama.buddhason.org/MN/dm.php?keyword=104})。
\stopitemgroup

\startitemgroup[noteitems]
\item\subnoteref{915.0}\NoteKeywordAgamaHead{「令愛(MA);執在心懷(AA)」},南傳作\NoteKeywordNikaya{「可愛所做的」}(piyakaraṇā),智髻比丘長老英譯為「創造愛」(creates love, MN),菩提比丘長老英譯為\NoteKeywordBhikkhuBodhi{「創造情愛」}(creates affection, AN)。
\stopitemgroup

\startitemgroup[noteitems]
\item\subnoteref{916.0}\NoteKeywordAgamaHead{「眼空…我所空(SA);空於神、神所有(MA)」},南傳作\NoteKeywordNikaya{「以我或我所是空」}(suññaṃ attena vā attaniyena vā),菩提比丘長老英譯為\NoteKeywordBhikkhuBodhi{「我或所有屬於我的這是空」}(Empty is this of self or of what belongs to self, SN),智髻比丘長老英譯為「我或所有屬於我的這是空」(This is void of a self or of what belongs to a self, MN)。
\stopitemgroup

\startitemgroup[noteitems]
\item\subnoteref{917.0}\NoteKeywordAgamaHead{「不動(SA);不移動/不動(MA);無動/不動(DA)」},南傳作\NoteKeywordNikaya{「不動的」}(āneñjaṃ),智髻比丘長老英譯為「泰然自若」(imperturbable, imperturbability),菩提比丘長老說,這是指第四禪以及其上的四個無色界定,但\ccchref{MN.105}{https://agama.buddhason.org/MN/dm.php?keyword=105}, \ccchref{MA.106}{https://agama.buddhason.org/MA/dm.php?keyword=106}似乎指第四禪與兩個較低的無色界禪[按:即虛空無邊處、識無邊處,亦即\ccchref{MA.75}{https://agama.buddhason.org/MA/dm.php?keyword=75}所說的第二、第三淨不動道],\ccchref{MN.122}{https://agama.buddhason.org/MN/dm.php?keyword=122}指進入無色定。又,《法蘊足論》說:「樂苦等已滅,心堅住不動,得清淨天眼,能廣見眾色,不苦不樂受,淨捨念及定,名第四靜慮,……云何不動行?謂四無色定。」《舍利弗阿毘曇論》說:「無色界天上有不動思共彼思識。……云何不動處?不動謂第四禪。」也都說第四禪、無色定為「不動」。
\stopitemgroup

\startitemgroup[noteitems]
\item\subnoteref{918.0}\NoteKeywordAgamaHead{「不至心施(GA);不至心故行施/不故施/不至心行布施(MA);不用心意/不用心/不至誠用心(AA)」},南傳作\NoteKeywordNikaya{「不誠心地施與」}(acittīkatvā deti,另譯為「不至心施,不尊敬地施與」),菩提比丘長老英譯為\NoteKeywordBhikkhuBodhi{「不尊敬地給」}(gives without reverence, \ccchref{AN.5.47}{https://agama.buddhason.org/AN/an.php?keyword=5.47}),或「輕率地給」(gives inconsiderately, \ccchref{AN.9.20}{https://agama.buddhason.org/AN/an.php?keyword=9.20}),智髻比丘長老英譯為「未表示尊重而給」(gives it without showing respect, MN),Maurice Walshe先生英譯為「不情願地給」(gave ungrudgingly, DN)。
\stopitemgroup

\startitemgroup[noteitems]
\item\subnoteref{919.0}\NoteKeywordAgamaHead{「撩擲而與(GA)」},南傳作\NoteKeywordNikaya{「施與丟棄的」}(apaviddhaṃ deti, apaviṭṭhaṃ…deti),菩提比丘長老英譯為\NoteKeywordBhikkhuBodhi{「給那些要被丟棄的」}(gives what would be discarded, AN/MN),Maurice Walshe先生英譯為「漫不經心地給」(gave not carelessly, DN)。
\stopitemgroup

\startitemgroup[noteitems]
\item\subnoteref{920.0}\NoteKeywordAgamaHead{「不觀業果報施(MA)」},南傳作\NoteKeywordNikaya{「[果報]不到來見解地施與」}(anāgamanadiṭṭhiko deti),智髻比丘長老英譯為「以沒什麼將到來的見解而給」(gives it with the view that nothing will come of it, MN),菩提比丘長老英譯為\NoteKeywordBhikkhuBodhi{「以將來無結果的見解而給」}(gives without a view of future consequences, \ccchref{AN.9.20}{https://agama.buddhason.org/AN/an.php?keyword=9.20}),或「以無給與回報之見解而給」(gives without a view about the returns of giving, \ccchref{AN.5.147}{https://agama.buddhason.org/AN/an.php?keyword=5.147})。
\stopitemgroup

\startitemgroup[noteitems]
\item\subnoteref{921.0}\NoteKeywordAgamaHead{「有名假賃至華鬘(MA)」},南傳作\NoteKeywordNikaya{「被套過花環(已訂婚)者」}(mālāguḷaparikkhittāpi),菩提比丘長老英譯為\NoteKeywordBhikkhuBodhi{「已訂婚者」}(one already engaged, AN),智髻比丘長老英譯為「戴花環象徵已訂婚者」(who are garlanded in token of betrothal, MN)。
\stopitemgroup

\startitemgroup[noteitems]
\item\subnoteref{922.0}\NoteKeywordAgamaHead{「發心/發心念(MA)」},南傳作\NoteKeywordNikaya{「心的生起」}(Cittuppādampi),智髻比丘長老英譯為「心的傾向」(the inclination of mind, MN),菩提比丘長老英譯為\NoteKeywordBhikkhuBodhi{「心思的發生」}(the arising of a thought, AN)。
\stopitemgroup

\startitemgroup[noteitems]
\item\subnoteref{923.0}\NoteKeywordNikayaHead{「識之類的」}(viññāṇagataṃ,逐字譯為「識+到達的/樣貌的」),菩提比丘長老英譯為\NoteKeywordBhikkhuBodhi{「屬於識;關於識」}(pertain to consciousness, AN),或「包括在識內」(included in consciousness, \suttaref{SN.35.121}),或「一種識的涉入」(an involvement with consciousness, \suttaref{SN.44.3}),智髻比丘長老英譯為「任何識」(any consciousness, MN),坦尼沙羅比丘長老英譯為「與意識連接」(connected with consciousness, AN),或「識的模式」(a mode of consciousness, MN),或「沉浸於識中」(immersed in consciousness, SN)。
\stopitemgroup

\startitemgroup[noteitems]
\item\subnoteref{924.0}\NoteKeywordNikayaHead{「非不可思議;不可思議」}(anacchariyaṃ),智髻比丘長老英譯為「自然地」(spontaneously, MN),I.B. Horner英譯為「自然地」(naturally, \ccchref{MN.125}{https://agama.buddhason.org/MN/dm.php?keyword=125}),菩提比丘長老英譯為\NoteKeywordBhikkhuBodhi{「驚人的」}(astounding, \suttaref{SN.6.1}),或「不會令人驚訝」(wouldn't be surprising, AN)。按:《顯揚真義》等在「心中出現偈頌/譬喻」的場合以「隨不可思議(伴隨不可思議)」(Anacchariyāti anuacchariyā, \suttaref{SN.6.1}/\ccchref{MN.12}{https://agama.buddhason.org/MN/dm.php?keyword=12})解說,意即「不可思議的偈頌」,今在該場合準此譯。
\stopitemgroup

\startitemgroup[noteitems]
\item\subnoteref{925.0}\NoteKeywordAgamaHead{「共諍論(SA);各說破壞(MA);迭相罵詈/面相毀罵(DA);面相談說/惡聲相向(AA)」},南傳作\NoteKeywordNikaya{「以舌鋒互刺的」}(aññamaññaṃ mukhasattīhi vitudantā),智髻比丘長老英譯為「以言辭上的劍互刺」(stabbing each other with verbal daggers, MN),菩提比丘長老英譯為\NoteKeywordBhikkhuBodhi{「以刺耳的話互刺」}(stabbing each other with piercing words, AN),Maurice Walshe先生英譯為「以言語交戰互相攻擊」(attacking each other with wordy warfare, DN)。按:「舌鋒」(mukhasattīhi),逐字譯為「口+刀/劍」。
\stopitemgroup

\startitemgroup[noteitems]
\item\subnoteref{926.0}\NoteKeywordAgamaHead{「蹲踞說蹲踞(MA)」},南傳作\NoteKeywordNikaya{「歐卡拉的瓦砂與巴聶」}(ukkalā vassabhaññā, okkalā vassabhaññā),菩提比丘長老英譯為\NoteKeywordBhikkhuBodhi{「優卡拉的瓦砂與巴聶」}(Vassa and Bañña of Ukkalā, SN/AN),智髻比丘長老英譯為「那些來自歐卡拉的瓦砂與巴聶老師們」(those teachers from Okkala, Vassa and Bhañña, MN)。按:《破斥猶豫》等說,歐卡拉為住的地方(janapadavāsino),瓦砂巴聶是兩個人(dve janā, \ccchref{MN.117}{https://agama.buddhason.org/MN/dm.php?keyword=117}/\ccchref{AN.4.30}{https://agama.buddhason.org/AN/an.php?keyword=4.30}),《顯揚真義》說,他們兩位是根本惡見者(Dvepi hi te mūladiṭṭhigatikā, \suttaref{SN.22.62})。
\stopitemgroup

\startitemgroup[noteitems]
\item\subnoteref{927.0}\NoteKeywordAgamaHead{「所說無滯(GA)」},南傳作\NoteKeywordNikaya{「清晰的」}(anelagalāya, aneḷagalāya,原意為「沒滴唾液的」),菩提比丘長老英譯為\NoteKeywordBhikkhuBodhi{「表達清楚的,口齒清楚的」}(articulate)。按:《顯揚真義》以「無口水的、無喉嚨的、無過失的,以及無蹣跚(結巴)的字句」(anelāya agalāya niddosāya ceva akkhalitapadabyañjanāya ca., \suttaref{SN.8.6})解說。
\stopitemgroup

\startitemgroup[noteitems]
\item\subnoteref{928.0}\NoteKeywordNikayaHead{「讓我們住於自己的喜樂」}(sakāya ratiyā vaseyyāmāti,逐字譯為「依(以)自己-喜樂-讓我們住」),菩提比丘長老依錫蘭版(sakāya ratiyā raseyyāmāti)英譯為「陶醉於我們自己那種喜悅」(revel in our own kind of delight, SN)。
\stopitemgroup

\startitemgroup[noteitems]
\item\subnoteref{929.0}\NoteKeywordAgamaHead{「瞋恚為毒根(SA)」},南傳作\NoteKeywordNikaya{「端蜜而根毒之憤怒」}(Kodhassa visamūlassa, madhuraggassa),菩提比丘長老英譯為\NoteKeywordBhikkhuBodhi{「帶著其有毒的根與甜蜜的頂端」}(With its poisoned root and honeyed tip)。按:《顯揚真義》說,「毒根」指苦果,「端蜜」指回報憤怒、回罵非難、回擊攻擊後,樂生起(\suttaref{SN.1.71})。另外,偈誦前段的「切斷什麼後」的「切斷後」(chetvā),錫蘭版均作「殺後」(jhatvā)。
\stopitemgroup

\startitemgroup[noteitems]
\item\subnoteref{930.0}\NoteKeywordNikayaHead{「簡樸」}(idamatthitaṃyeva,逐字譯為「這-有-僅那個」),菩提比丘長老英譯為\NoteKeywordBhikkhuBodhi{「簡單;簡樸」}(simplicity)。按:《滿足希求》以「沒有任何世間財物之意(lokāmisanti attho),在這裡只有關於他人殘餘與明顯的需要(Sesamettha ito paresu ca uttānatthameva, \ccchref{AN.5.181}{https://agama.buddhason.org/AN/an.php?keyword=5.181})。
\stopitemgroup

\startitemgroup[noteitems]
\item\subnoteref{931.0}\NoteKeywordNikayaHead{「如是有無」}(itibhavābhava),菩提比丘長老英譯為\NoteKeywordBhikkhuBodhi{「成為這或那」}(becoming this or that),智髻比丘長老英譯為「某種更好狀態的生命」(some better state of being, MN)。按:在畜生論的場合,《顯揚真義》等以「常恆/斷滅、增長/減損(Bhavoti vuddhi, abhavoti hāni)」解說「有/無」(\suttaref{SN.56.10}/\ccchref{MN.76}{https://agama.buddhason.org/MN/dm.php?keyword=76}/\ccchref{DN.1}{https://agama.buddhason.org/DN/dm.php?keyword=1}/\ccchref{AN.10.69}{https://agama.buddhason.org/AN/an.php?keyword=10.69}),在「為了衣服」等場合,《破斥猶豫》以「為了作福,依此在生存(有)處感受樂(puññakiriyavatthuṃ nissāya tasmiṃ tasmiṃ bhave sukhaṃ vedissāmīti, \ccchref{MN.103}{https://agama.buddhason.org/MN/dm.php?keyword=103})而教導法」解說,《吉祥悅意》等以「在顯現利益上落下,如為了衣服等」(nidassanatthe nipāto. Yathā cīvarādihetu)解說「如是」,以「更勝妙的熟酥生酥等」(paṇītatarāni sappinavanītādīni, \ccchref{DN.33}{https://agama.buddhason.org/DN/dm.php?keyword=33}/\ccchref{AN.4.9}{https://agama.buddhason.org/AN/an.php?keyword=4.9})解說「有無」。
\stopitemgroup

\startitemgroup[noteitems]
\item\subnoteref{932.0}\NoteSubKeyHead{(1)}\NoteKeywordNikayaHead{「退分」}(hānabhāgiyā,另譯為「與退失有關的」),菩提比丘長老英譯為\NoteKeywordBhikkhuBodhi{「參與惡化」}(partake of deterioration)。按:《吉祥悅意》以「導向往苦界的退失」(apāyagāmiparihānāya saṃvattanako, \ccchref{DN.34}{https://agama.buddhason.org/DN/dm.php?keyword=34}),《滿足希求》以「得到初禪者與欲俱行的想與作意被執行(paṭhamassa jhānassa lābhiṃ kāmasahagatā saññāmanasikārā samudācaranti, \ccchref{AN.4.179}{https://agama.buddhason.org/AN/an.php?keyword=4.179})」解說。
\item\subnoteref{932.1}\NoteSubKeyHead{(2)}\NoteKeywordAgamaHead{「退法(DA)」},南傳作\NoteKeywordNikaya{「退分的一法」}(eko dhammo hānabhāgiyo),Maurice Walshe先生英譯為「導致縮減的一件事」(one thing that conduces to diminution)。
\stopitemgroup

\startitemgroup[noteitems]
\item\subnoteref{933.0}\NoteKeywordNikayaHead{「住分」}(ṭhitibhāgiyā,另譯為「順住分;持續的部分;有關於持久的」),菩提比丘長老英譯為\NoteKeywordBhikkhuBodhi{「參與穩定」}(partake of stabilization)。
\stopitemgroup

\startitemgroup[noteitems]
\item\subnoteref{934.0}\NoteKeywordNikayaHead{「跟隨實踐」}(anuyogamanvāya),智髻比丘長老英譯為「獻身;專心」(devotion, MN),Maurice Walshe先生英譯為「應用;勤勉」(application, DN)。
\stopitemgroup

\startitemgroup[noteitems]
\item\subnoteref{935.0}\NoteKeywordNikayaHead{「跟隨正確作意」}(sammāmanasikāramanvāya),智髻比丘長老英譯為「正確注意」(right attention, MN),Maurice Walshe先生英譯為「正當的注意」(due attention, DN)。按:「作意」(manasikāra)為「意」與「作」的複合詞,可以是「注意」,也可以有「思惟」的意思。
\stopitemgroup

\startitemgroup[noteitems]
\item\subnoteref{936.0}\NoteKeywordAgamaHead{「定意三昧/心定三昧(DA)」},南傳作\NoteKeywordNikaya{「心定」}(cetosamādhiṃ),菩提比丘長老英譯為\NoteKeywordBhikkhuBodhi{「心的集中貫注」}(concentration of mind, AN/MN),Maurice Walshe先生英譯為「到達這樣程度的集中貫注」(reaches such a level of concentration, DN)。
\stopitemgroup

\startitemgroup[noteitems]
\item\subnoteref{937.0}\NoteKeywordAgamaHead{「世中(MA)」},南傳作\NoteKeywordNikaya{「在一個世間界中」}(ekissā lokadhātuyā),菩提比丘長老英譯為\NoteKeywordBhikkhuBodhi{「一個世界系統」}(in one world-system, AN/MN),Maurice Walshe先生英譯為「一個相同的世界系統」(in one and the same world-system, DN)。按:《破斥猶豫》等以「十千世間界」(dasasahassilokadhātuyā, \ccchref{MN.115}{https://agama.buddhason.org/MN/dm.php?keyword=115}/\ccchref{DN.28}{https://agama.buddhason.org/DN/dm.php?keyword=28}/\ccchref{AN.1.277}{https://agama.buddhason.org/AN/an.php?keyword=1.277})解說。
\stopitemgroup

\startitemgroup[noteitems]
\item\subnoteref{938.0}\NoteSubKeyHead{(1)}\NoteKeywordAgamaHead{「貪想(SA);欲想(DA/AA)」},南傳作\NoteKeywordNikaya{「欲想」}(kāmasaññā),菩提比丘長老英譯為\NoteKeywordBhikkhuBodhi{「肉慾的認知」}(sensual perceptions, SN/AN),Maurice Walshe先生英譯為「肉慾想要的認知」(perception of sensual desire, DN)。
\item\subnoteref{938.1}\NoteSubKeyHead{(2)}\NoteKeywordAgamaHead{「恚想(SA);瞋想(DA);恚想/瞋恚想(AA)」},南傳作\NoteKeywordNikaya{「惡意想」}(byāpādasaññā),菩提比丘長老英譯為\NoteKeywordBhikkhuBodhi{「惡意的認知」}(perception of ill will, SN/AN),Maurice Walshe先生英譯為「敵意的認知」(perception of enmity, DN)。
\item\subnoteref{938.2}\NoteSubKeyHead{(3)}\NoteKeywordAgamaHead{「害想(SA/DA);疾想/害想/殺害想(AA)」},南傳作\NoteKeywordNikaya{「加害想」}(vihiṃsāsaññā),菩提比丘長老英譯為\NoteKeywordBhikkhuBodhi{「傷害的認知」}(perception of harming, SN/AN),Maurice Walshe先生英譯為「殘酷的認知」(perception of cruelty, DN)。按:「加害」(vihiṃsā),另譯為「害;害意;惱害;傷害」。
\stopitemgroup

\startitemgroup[noteitems]
\item\subnoteref{939.0}\NoteKeywordAgamaHead{「威儀(SA);威儀法(MA)」},南傳作\NoteKeywordNikaya{「增上行儀法」}(ābhisamācārikaṃ dhammaṃ),菩提比丘長老英譯為\NoteKeywordBhikkhuBodhi{「適當行為的要素」}(the factor of proper conduct)。按:「增上行儀」(abhisamācārikaṃ,另譯為「好行為」),《破斥猶豫》以「本分行為程度的」(vattapaṭipattimattampi, \ccchref{MN.69}{https://agama.buddhason.org/MN/dm.php?keyword=69}),《滿足希求》以「成為最高正行的行法,因本分而安立的戒」(uttamasamācārabhūtaṃ vattavasena paññattasīlaṃ, \ccchref{AN.5.21}{https://agama.buddhason.org/AN/an.php?keyword=5.21})解說。
\stopitemgroup

\startitemgroup[noteitems]
\item\subnoteref{940.0}\NoteSubKeyHead{(1)}\NoteKeywordAgamaHead{「學法(MA)」},南傳作\NoteKeywordNikaya{「有學法」}(sekhaṃ dhammaṃ),菩提比丘長老英譯為\NoteKeywordBhikkhuBodhi{「訓練中者的因素」}(the factor of a trainee, AN)。按:《滿足希求》以「有學的安立戒(德行)」(sekhapaṇṇattisīlaṃ, \ccchref{AN.5.21}{https://agama.buddhason.org/AN/an.php?keyword=5.21})解說。
\item\subnoteref{940.1}\NoteSubKeyHead{(2)}\NoteKeywordAgamaHead{「學見跡(SA/MA)」},南傳作\NoteKeywordNikaya{「有學行者」}(sekho……pāṭipado),菩提比丘長老英譯為\NoteKeywordBhikkhuBodhi{「在較高訓練中的弟子;已上道者」}(a disciple in higher training, one who has entered upon the way)。按:「學」指有學,「見跡」即實踐(看見)道跡(pāṭipado,另譯為「行者」,《滿足希求》以「向道者」(paṭipannako)解說,今準此譯)。
\stopitemgroup

\startitemgroup[noteitems]
\item\subnoteref{941.0}\NoteKeywordAgamaHead{「瞋恚(SA)」},南傳作\NoteKeywordNikaya{「內瞋(的);內瞋者」}(dosantaro),智髻比丘長老英譯為「內在的恨」(inner hate, MN),菩提比丘長老英譯為\NoteKeywordBhikkhuBodhi{「懷有恨者」}(who harbors hatred, AN),Maurice Walshe先生英譯為「心中敵意」(enmity in [my] heart, DN)。按:《破斥猶豫》以「瞋心」(dosacitto, \ccchref{MN.21}{https://agama.buddhason.org/MN/dm.php?keyword=21}),《滿足希求》以「於內在跌落瞋」(antare patitadoso, \ccchref{AN.2.23}{https://agama.buddhason.org/AN/an.php?keyword=2.23})解說。
\stopitemgroup

\startitemgroup[noteitems]
\item\subnoteref{942.0}\NoteKeywordAgamaHead{「斷截橋樑(SA)」},南傳作\NoteKeywordNikaya{「橋的破壞」}(setughātaṃ, setughāto,另譯為「惡習的打破」),菩提比丘長老英譯為\NoteKeywordBhikkhuBodhi{「橋的破壞」}(the demolition of the bridge, \ccchref{AN.4.159}{https://agama.buddhason.org/AN/an.php?keyword=4.159}),並說,「食物、渴愛、慢」還可以技巧地運用於證阿羅漢果,但淫慾卻完全不能。按:《滿足希求》以「路的破壞、緣的破壞」(padaghātaṃ paccayaghātaṃ, \ccchref{AN.3.75}{https://agama.buddhason.org/AN/an.php?keyword=3.75})解說。
\stopitemgroup

\startitemgroup[noteitems]
\item\subnoteref{943.0}\NoteKeywordAgamaHead{「自隨意所欲(SA)」},南傳作\NoteKeywordNikaya{「愛惜自己/以愛惜自己」}(attakāmā/attakāmena,逐字譯為「自我+欲(想要)」),菩提比丘長老依錫蘭本(atthakāmena)英譯為「想要他自己的好」(one desiring his own good, \suttaref{SN.6.2}),或「愛他自己」(loves himself, \suttaref{SN.3.8}),或「想要他們自己好」(desiring their own good, \ccchref{AN.3.87}{https://agama.buddhason.org/AN/an.php?keyword=3.87}),或「想要好的」(one desiring the good, AN),智髻比丘長老英譯為「追尋他們自己的好處」(seeking their own good, MN)。按:《滿足希求》以「欲(想要)自己的利益」(attano hitakāmā, \ccchref{AN.3.89}{https://agama.buddhason.org/AN/an.php?keyword=3.89})解說,《破斥猶豫》以「住於有自己利益的欲求自性」(attano hitaṃ kāmayamānasabhāvā hutvā viharanti, \ccchref{MN.31}{https://agama.buddhason.org/MN/dm.php?keyword=31})解說「愛惜自己形色」。
\stopitemgroup

\startitemgroup[noteitems]
\item\subnoteref{944.0}\NoteKeywordNikayaHead{「法足;法句」}(dhammapadāni,另譯為「法跡」),菩提比丘長老英譯為\NoteKeywordBhikkhuBodhi{「法的道路」}(the path of Dhamma, SN),或「法的詩節(詩句)」(Dhamma-stanzas, \suttaref{SN.9.10}),或「法的要素」(Dhamma factors, AN),Maurice Walshe先生英譯為「法的區分」(divisions of dhamma, DN)。按:《顯揚真義》以「從布施而涅槃被察悟」(dānato nibbānasaṅkhātaṃ, \suttaref{SN.1.33}),或「一切佛語」(sabbampi buddhavacanaṃ, \suttaref{SN.9.10}),《滿足希求》等以「法的一部分」(dhammakoṭṭhāsā, \ccchref{DN.33}{https://agama.buddhason.org/DN/dm.php?keyword=33}/\ccchref{AN.4.29}{https://agama.buddhason.org/AN/an.php?keyword=4.29})解說。
\stopitemgroup

\startitemgroup[noteitems]
\item\subnoteref{945.0}\NoteKeywordAgamaHead{「遼落(SA);於理不決不能正答(GA)」},南傳作\NoteKeywordNikaya{「轉向無關的」}(aññena vā aññaṃ paṭicariati, aññenāññaṃ paṭicarati),菩提比丘長老英譯為\NoteKeywordBhikkhuBodhi{「推諉地回答」}(answer evasively),Maurice Walshe先生英譯為「逃避這個議題」(evade the issue, DN)。按:《滿足希求》以「以其它言詞隱藏(paṭicchādeti)另一個,或被詢問一個而談論另一個」(\ccchref{AN.3.68}{https://agama.buddhason.org/AN/an.php?keyword=3.68})解說,《吉祥悅意》的解說也大致相同。
\stopitemgroup

\startitemgroup[noteitems]
\item\subnoteref{946.0}\NoteKeywordAgamaHead{「邊見(MA/DA/AA)」},南傳作\NoteKeywordNikaya{「邊見者」}(antaggāhikāya diṭṭhiyā),Maurice Walshe先生英譯為「極端主義的見解」(extremist opinions),菩提比丘長老英譯為\NoteKeywordBhikkhuBodhi{「極端主義的見解」}(extremist view)。按:《吉祥悅意》以「斷滅邊見(diṭṭhi ucchedantassa, \ccchref{DN.25}{https://agama.buddhason.org/DN/dm.php?keyword=25})的執取情況」,《滿足希求》以「執取十事聚[未詳列]後,為邊見的住立者」(dasavatthukāya antaṃ gahetvā ṭhitadiṭṭhiyā, \ccchref{AN.3.51}{https://agama.buddhason.org/AN/an.php?keyword=3.51})解說。
\stopitemgroup

\startitemgroup[noteitems]
\item\subnoteref{947.0}\NoteKeywordNikayaHead{「美善的;美善者」}(pesalā),菩提比丘長老英譯為\NoteKeywordBhikkhuBodhi{「行為良好的」}(Well-behaved)。按:《顯揚真義》等以「可愛德行」(piyasīlā, \suttaref{SN.6.9}/\ccchref{DN.16}{https://agama.buddhason.org/DN/dm.php?keyword=16}/\ccchref{AN.2.201}{https://agama.buddhason.org/AN/an.php?keyword=2.201})解說。
\stopitemgroup

\startitemgroup[noteitems]
\item\subnoteref{948.0}\NoteKeywordNikayaHead{「正確看見」}(Sammādassanaṃ, sammādassanena,名詞),菩提比丘長老英譯為\NoteKeywordBhikkhuBodhi{「正確的看見」}(seeing rightly, SN),或「正確的透視」(a right perspective, AN),坦尼沙羅比丘長老英譯為「正確的視野」(right vision)。
\stopitemgroup

\startitemgroup[noteitems]
\item\subnoteref{949.0}\NoteKeywordAgamaHead{「捨法/捨行(SA)」},南傳作\NoteKeywordNikaya{「下來之祭典」}(Paccorohaṇī,另譯為「捨法」),菩提比丘長老英譯照錄不譯。按:Paccoroha意為「再下來;下降」,原為婆羅門教的祭典之一,《滿足希求》以「惡的」(pāpassa, \ccchref{AN.10.119}{https://agama.buddhason.org/AN/an.php?keyword=10.119})形容它。
\stopitemgroup

\startitemgroup[noteitems]
\item\subnoteref{950.0}\NoteSubKeyHead{(1)}\NoteKeywordAgamaHead{「如月盡(MA);如月向於晦(DA);如月末之月/如月向盡(AA)」},南傳作\NoteKeywordNikaya{「在黑暗側」}(kāḷapakkhe),菩提比丘長老英譯為\NoteKeywordBhikkhuBodhi{「在黑暗的十四天」}(during the dark fortnight)。按:此即指月亮由圓轉缺的那半個月(農曆每月16日到30日)。
\item\subnoteref{950.1}\NoteSubKeyHead{(2)}\NoteKeywordAgamaHead{「如月初生(SA/GA/MA);如月漸盛滿(MA);如月初(AA)」},南傳作\NoteKeywordNikaya{「如在明亮側的月亮」}(sukkapakkheva candimāti),Maurice Walshe先生英譯為「如在增大時期的月亮」(like moon at waxing-time)。按:此即指月亮由缺轉圓的那半個月(農曆每月1日到15日)。
\stopitemgroup

\startitemgroup[noteitems]
\item\subnoteref{951.0}\NoteKeywordNikayaHead{「在非行境」}(agocare),智髻比丘長老英譯為「不當去處」(unsuitable resorts, MN),菩提比丘長老英譯「不當去處」(improper resort, AN),或「在你們去處之外」(outside your own resort, SN)。按:《破斥猶豫》等以「不適當行境」(ayutto gocaro, \ccchref{MN.2}{https://agama.buddhason.org/MN/dm.php?keyword=2}/\ccchref{AN.6.58}{https://agama.buddhason.org/AN/an.php?keyword=6.58})解說。
\stopitemgroup

\startitemgroup[noteitems]
\item\subnoteref{952.0}\NoteSubKeyHead{(1)}\NoteKeywordAgamaHead{「虛偽/虛誑(SA);思/戲(MA);調戲/調(DA)」},南傳作\NoteKeywordNikaya{「虛妄」}(papañcaṃ,名詞,另譯為「障礙;延誤;迷執;擴張;增殖;妄想;戲論」),菩提比丘長老英譯為\NoteKeywordBhikkhuBodhi{「增殖」}(proliferation),或「心理增殖」(mental proliferation, MN)。按:《滿足希求》以「渴愛、見、慢轉起、持續陶醉、污染之虛妄(kilesapapañco, \ccchref{AN.6.14}{https://agama.buddhason.org/AN/an.php?keyword=6.14})」解說,《吉祥悅意》則列舉「渴愛虛妄(百八種渴愛思潮)、慢虛妄(九種慢)、見虛妄(六十二見)」三類虛妄(\ccchref{DN.21}{https://agama.buddhason.org/DN/dm.php?keyword=21}),《勝義燈》以「貪、瞋、癡、渴愛、見為其特質」(Visesato rāgadosamohataṇhādiṭṭhimānā, \ccchref{Ud.67}{https://agama.buddhason.org/Ud/dm.php?keyword=67})。
\item\subnoteref{952.1}\NoteSubKeyHead{(2)}\NoteKeywordAgamaHead{「念(MA);稱量(AA)」},南傳作\NoteKeywordNikaya{「作虛妄」}(papañceti,為papañca的名動詞),智髻比丘長老英譯為「內心增生」(mentally proliferates, MN)。
\stopitemgroup

\startitemgroup[noteitems]
\item\subnoteref{953.0}\NoteKeywordAgamaHead{「達嚫(SA/AA);供養(DA)」}(梵文dakṣiṇā),南傳作\NoteKeywordNikaya{「供養;施物;供養物」}(dakkhiṇā),菩提比丘長老英譯為\NoteKeywordBhikkhuBodhi{「奉獻物;牲禮」}(an offering)。按:《初期大乘佛教之起源與開展》說:「特欹拏(伽陀),或作鐸欹拏、達䞋、大嚫、檀嚫、達嚫等,都是Dakṣiṇā的音譯,義淨義譯為「清淨」(伽他)。……是受供後的讚歎偈,從布施而讚歎三寶,後代用來代替說法的。」(p.504)
\stopitemgroup

\startitemgroup[noteitems]
\item\subnoteref{954.0}\NoteSubKeyHead{(1)}\NoteKeywordAgamaHead{「有求有行(SA)」},南傳作\NoteKeywordNikaya{「我曾行…之遍求」}(pariyesanaṃ acariṃ),菩提比丘長老英譯為\NoteKeywordBhikkhuBodhi{「出發尋求」}(set out seeking)。「遍求」(pariyesanaṃ)為名詞。
\item\subnoteref{954.1}\NoteSubKeyHead{(2)}\NoteKeywordNikayaHead{「隨順覺」},南傳作\NoteKeywordNikaya{「我曾到達」}(ajjhagamaṃ),菩提比丘長老英譯為\NoteKeywordBhikkhuBodhi{「我發覺」}(that I discovered)。
\stopitemgroup

\startitemgroup[noteitems]
\item\subnoteref{955.0}\NoteSubKeyHead{(1)}\NoteKeywordAgamaHead{「明於事者(SA);安宅符呪(DA)」},南傳作\NoteKeywordNikaya{「宅地明」}(vatthuvijjā),菩提比丘長老英譯為\NoteKeywordBhikkhuBodhi{「風水的庸俗技術」}(the debased art of geomancy),Maurice Walshe先生英譯為「房屋與園林的知識」(house-and garden-lore)。按:《顯揚真義》以「各種瓜類所依結果實因素時節方法知識」(vatthūnaṃ phalasampattikāraṇakālajānanupāyo, \suttaref{SN.28.10}),《吉祥悅意》以「住宅、僧園等宅地的性質、過失、箭刺、惡運之術(guṇadosasallakkhaṇavijjā, \ccchref{DN.1}{https://agama.buddhason.org/DN/dm.php?keyword=1}),看土地差別、誦咒……」解說,前者是農作的知識,後者則是風水術,長老傾向接受後者之說。「宅地」(vatthu),另一個意思是「事;事物」,SA似乎取這個意思譯。
\item\subnoteref{955.1}\NoteSubKeyHead{(2)}\NoteKeywordAgamaHead{「畜生之咒(MA);相畜生(DA)」},南傳作\NoteKeywordNikaya{「(以)畜生明」}(tiracchānavijjāya),菩提比丘長老英譯沒譯,Maurice Walshe先生英譯為「這樣基礎之術」(such base arts),也等於沒譯。按:\ccchref{DN.1}{https://agama.buddhason.org/DN/dm.php?keyword=1}、\ccchref{DN.10}{https://agama.buddhason.org/DN/dm.php?keyword=10}中有明確內容。
\stopitemgroup

\startitemgroup[noteitems]
\item\subnoteref{956.0}\NoteKeywordAgamaHead{「非法欲(MA);非法婬(DA)」},南傳作\NoteKeywordNikaya{「非法貪」}(Adhammarāgo),Maurice Walshe先生英譯為「亂倫」(incest, \ccchref{DN.26}{https://agama.buddhason.org/DN/dm.php?keyword=26}),菩提比丘長老英譯為\NoteKeywordBhikkhuBodhi{「非法的慾望」}(illicit lust, \ccchref{AN.3.57}{https://agama.buddhason.org/AN/an.php?keyword=3.57})。按:《吉祥悅意》以「母親、姨媽、姑姑、舅舅等在不適當處的貪(ayuttaṭṭhāne rāgo, \ccchref{DN.26}{https://agama.buddhason.org/DN/dm.php?keyword=26})」解說。《滿足希求》說,貪一向名為非法,但於自己必需品生起的不是非法貪,於他人必需品(paraparikkhāresu)生起的就是非法貪(\ccchref{AN.3.57}{https://agama.buddhason.org/AN/an.php?keyword=3.57})。《法蘊足論》說,於母女姊妹及餘隨一親眷起貪、等貪,執藏防護,堅著愛染,名非法貪。《舍利弗阿毘曇論》說,若母、師妻等作欲染行,是名非法欲染。《瑜伽師地論》說,於諸惡行深生耽著,名非法貪。《大乘阿毘達磨雜集論》說,非法貪者,謂:隨貪著不善業道。
\stopitemgroup

\startitemgroup[noteitems]
\item\subnoteref{957.0}\NoteKeywordAgamaHead{「邪貪(SA);惡貪(MA);非法貪(DA)」},南傳作\NoteKeywordNikaya{「不正貪」}(visamalobho,另譯為「非理貪;異樣貪;邪貪」,玄奘法師譯為「不平等愛;不平等貪」),Maurice Walshe先生英譯為「過度貪婪」(excessive greed, \ccchref{DN.26}{https://agama.buddhason.org/DN/dm.php?keyword=26}),菩提比丘長老英譯為\NoteKeywordBhikkhuBodhi{「不正的貪婪;邪惡的貪婪」}(unrighteous greed, \ccchref{AN.3.57}{https://agama.buddhason.org/AN/an.php?keyword=3.57})。按:《破斥猶豫》以「對他人財物(parabhaṇḍe, \ccchref{MN.7}{https://agama.buddhason.org/MN/dm.php?keyword=7})[的欲貪]」,《吉祥悅意》以「對不適當受用物處的極有力貪(atibalavalobho, \ccchref{DN.26}{https://agama.buddhason.org/DN/dm.php?keyword=26})」解說,《滿足希求》說,沒有貪的正確適時(lobhassa samakālo nāma natthi),而這些是一向的不正(visamova),但從自己占有的事物生起的名為正貪(samalobho, 也譯為「等貪」),從他人占有的事物生起的名為不正(\ccchref{AN.3.57}{https://agama.buddhason.org/AN/an.php?keyword=3.57})。《法蘊足論》說,於他財物及所攝受,起貪、等貪,執藏、防護、堅著、愛染,是名惡貪。復有惡貪:規他生命,貪皮角等,飲血噉肉,如是二種總名惡貪。《舍利弗阿毘曇論》說,若於他物財賂妻子等,欲令我有貪欲染貪著,是名惡貪。《大乘阿毘達磨雜集論》說,不平等貪者,謂:非法非理貪求境界。《四諦論》說,依道理求覓受用,名平等愛。
\stopitemgroup

\startitemgroup[noteitems]
\item\subnoteref{958.0}\NoteKeywordAgamaHead{「邪法(MA);邪見(DA)」},南傳作\NoteKeywordNikaya{「邪法」}(micchādhammo),Maurice Walshe先生英譯為「異常的實行」(deviant practices),菩提比丘長老英譯為\NoteKeywordBhikkhuBodhi{「錯誤的法」}(wrong Dhamma, \ccchref{AN.3.57}{https://agama.buddhason.org/AN/an.php?keyword=3.57})。按:《吉祥悅意》以「男人對男人(purisānaṃ purisesu, \ccchref{DN.26}{https://agama.buddhason.org/DN/dm.php?keyword=26}),女人對女人的欲貪(chandarāgo)」,《滿足希求》以「具備被稱為無基礎之實行(avatthupaṭisevanasaṅkhātena, \ccchref{AN.3.57}{https://agama.buddhason.org/AN/an.php?keyword=3.57})的邪法」解說,其註疏以「在被世間認可為善的之貪的基礎處(rāgassa vatthuṭṭhānaṃ)外的其它基礎上之實行」解說「被稱為無基礎之實行」。《大薩遮尼乾子所說經》說,於諸外道非義論中起義論想,於無益論生利益想,於非法中生是法想,……是名邪法羅網纏心。《大乘阿毘達磨雜集論》說,邪法者,謂:諸外道惡說法律。上兩種邪法的解說,近於\ccchref{DA.6}{https://agama.buddhason.org/DA/dm.php?keyword=6}所說的邪見。
\stopitemgroup

\startitemgroup[noteitems]
\item\subnoteref{959.0}\NoteKeywordAgamaHead{「物類名字、萬物差品、字類分合、歷世本末此五種記(SA);因、緣、正、文、戲五句說(MA)」},南傳作\NoteKeywordNikaya{「包含字彙儀軌的、包含音韻論語源論的、古傳歷史為第五的」}(sanighaṇḍukeṭubhānaṃ sākkharappabhedānaṃ itihāsapañcamānaṃ),菩提比丘長老英譯為\NoteKeywordBhikkhuBodhi{「字彙,儀式,音韻學與語源學,以及歷史為第五」}(vocabularies, ritual, phonology, and etymology, and the histories as a fifth, AN/MN),Maurice Walshe先生英譯為「規則和儀式,聲音和意義的學問,以及第五古口傳的傳統」(the rules and rituals, the lore of sounds and meanings and, fifthly, oral tradition, DN)。按:「音韻論與語源論」(sākkharappabhedānaṃ,逐字譯為「含+文字+區分),即SA的「字類分合」,為兩項。
\stopitemgroup

\startitemgroup[noteitems]
\item\subnoteref{960.0}\NoteKeywordAgamaHead{「定分、相續、轉變(SA);定在數中(DA)」},南傳作\NoteKeywordNikaya{「被命運、意外、本性變化」}(niyatisaṅgatibhāvapariṇatā),菩提比丘長老英譯為\NoteKeywordBhikkhuBodhi{「被命運、境遇與本性所塑成」}(moulded by destiny, circumstance, and nature)。按:《顯揚真義》等以「被隱藏的種類到處出來」(channaṃ abhijātīnaṃ tattha tattha gamanaṃ, \suttaref{SN.24.7}/\ccchref{DN.2}{https://agama.buddhason.org/DN/dm.php?keyword=2})解說「意外」(saṅgati,另譯為「會合」),以「自性」(sabhāvoyeva)解說「本性」(bhāva)。
\stopitemgroup

\startitemgroup[noteitems]
\item\subnoteref{961.0}\NoteKeywordAgamaHead{「生藏、熟藏(SA);生熟二藏(AA)」},或為梵文āmāśaya, pakvāśaya(巴利āmāsaya, pakkāsaya)的對譯,前者為食物在體內還未消化之處,後者為食物在體內已消化之處,《清淨道論》說,糞便……其處所住立於熟臟(Okāsato pakkāsaye ṭhitaṃ, 智髻長老英譯註記為「直腸(rectum)」),熟臟在肚臍下-脊椎根(尾椎)間(heṭṭhānābhi-piṭṭhikaṇṭakamūlānaṃ antare),為腸的終點,高八指長(ubbedhena aṭṭhaṅgulamatto),等同竹莖(veḷunāḷikasadiso)。……凡任何落入生藏(āmāsaye)的飲食等,以胃火起泡(udaragginā pheṇuddehakaṃ)一再消化(pakkaṃ pakkaṃ-成熟),如被研磨磨碎到達變軟後,經腸孔一再流下摩擦後(omadditvā ogaḷitvā ogaḷitvā omadditvā, 8.201)……。依此段描述比對現代醫學,生藏即是胃,熟藏即是乙狀結腸與直腸,如水野弘元《パ─リ語辞典》所說。
\stopitemgroup

\startitemgroup[noteitems]
\item\subnoteref{962.0}\NoteKeywordNikayaHead{「合意形色」}(sātarūpaṃ,另譯為「合意的樣子;悅色」),菩提比丘長老英譯為\NoteKeywordBhikkhuBodhi{「適意的種類;適意的性質」}(agreeable nature, \suttaref{SN.12.66}/\suttaref{SN.35.244}),或「適意的」(agreeable, \suttaref{SN.4.25}/\ccchref{AN.10.26}{https://agama.buddhason.org/AN/an.php?keyword=10.26}),或「愉快之類的」(what is enjoyable, \ccchref{AN.8.56}{https://agama.buddhason.org/AN/an.php?keyword=8.56}),Maurice Walshe先生英譯為「愉快的」(pleasurable, \ccchref{DN.22}{https://agama.buddhason.org/DN/dm.php?keyword=22}),Burma Piṭaka Association英譯為「愉快的特色」(pleasurable characteristics, \ccchref{DN.22}{https://agama.buddhason.org/DN/dm.php?keyword=22})。按:《顯揚真義》等以「如蜜的自性」(madhurasabhāvañca, \suttaref{SN.12.66}/\ccchref{DN.22}{https://agama.buddhason.org/DN/dm.php?keyword=22}),《滿足希求》以「欲樂」(kāmasukhena, \ccchref{AN.8.56}{https://agama.buddhason.org/AN/an.php?keyword=8.56}),或「在合意之類的對象上…」(sātajātikesu ca vatthūsu…, \ccchref{AN.10.26}{https://agama.buddhason.org/AN/an.php?keyword=10.26})解說。
\stopitemgroup

\startitemgroup[noteitems]
\item\subnoteref{963.0}\NoteKeywordAgamaHead{「計(SA/MA);自舉(MA);起…之想(AA)」},南傳作\NoteKeywordNikaya{「思量;思量者」}(maññati, maññita, maññamāno, maññeyya,另譯為「想;認為;思惟」),菩提比丘長老英譯為\NoteKeywordBhikkhuBodhi{「想像;抱有(思想、意見)」}(conceives; conceived; conceiving)。按:《顯揚真義》以「以渴愛、慢、見之思量而思量」(taṇhāmānadiṭṭhimaññanāhi maññamāno, \suttaref{SN.22.64})解說,《破斥猶豫》說,以渴愛之思量而思量(taṇhāmaññanāya maññati)……以慢之思量而思量(mānamaññanāya maññati)……以邪見之思量而思量(diṭṭhimaññanāya maññati)……祈願(paṇidahati-志向)往生(upapajjeyya)為某類生命,這是以渴愛之思量而思量;比較自己與某類生命的勝劣,這是以慢之思量而思量;生命為常恆不變易法(niccā dhuvā sassatā avipariṇāmadhammā’’ti, \ccchref{MN.1}{https://agama.buddhason.org/MN/dm.php?keyword=1}),這是以邪見之思量而思量,《滿足希求》以「不以渴愛、慢、見思量」(taṇhāmānadiṭṭhīhi na maññati, \ccchref{AN.4.24}{https://agama.buddhason.org/AN/an.php?keyword=4.24})解說「不思量」。
\stopitemgroup

\startitemgroup[noteitems]
\item\subnoteref{964.0}\NoteSubKeyHead{(1)}\NoteKeywordAgamaHead{「計(SA/DA/AA);見(SA/MA/DA);言(SA)」},南傳作\NoteKeywordNikaya{「認為」}(samanupassati,另譯為「看;看作」),菩提比丘長老英譯為\NoteKeywordBhikkhuBodhi{「認為;考察;加以考慮」}(regards)。按:《顯揚真義》以「(邪)見的認為」(diṭṭhisamanupassanā, \suttaref{SN.22.81})解說。
\item\subnoteref{964.1}\NoteSubKeyHead{(2)}\NoteKeywordAgamaHead{「見有我/計言有我(SA);計有我(SA/GA)」},南傳作\NoteKeywordNikaya{「認為種種我的認為」}(anekavihitaṃ attānaṃ samanupassamānā samanupassanti),菩提比丘長老英譯為\NoteKeywordBhikkhuBodhi{「以種種方式認為[任何事]是自我」}(regard [anything as] self in various ways)。
\stopitemgroup

\startitemgroup[noteitems]
\item\subnoteref{965.0}\NoteKeywordAgamaHead{「計/傾動(SA);移動(MA)」},南傳作\NoteKeywordNikaya{「擾動」}(ejā,另譯為「動;動貪;動著」),菩提比丘長老英譯為\NoteKeywordBhikkhuBodhi{「被擾動」}(being stirred, SN),Maurice Walshe先生英譯為「激情」(Passion, DN),坦尼沙羅比丘長老英譯為「戀慕」(Yearning, DN)。按:《顯揚真義》以「渴愛」(taṇhā, \suttaref{SN.35.90})解說,或以「以所謂擾動之渴愛的捨斷成為阿羅漢境界」(ejāsaṅkhātāya taṇhāya pahānabhūtaṃ arahattaṃ, \suttaref{SN.22.76})解說「不擾動」,《吉祥悅意》說,渴愛以持續搖動(calanaṭṭhena, \ccchref{DN.21}{https://agama.buddhason.org/DN/dm.php?keyword=21})而被稱為「擾動」。
\stopitemgroup

\startitemgroup[noteitems]
\item\subnoteref{966.0}\NoteSubKeyHead{(1)}\NoteKeywordNikayaHead{「不從眼思量」}(cakkhuto na maññati),菩提比丘長老英譯為\NoteKeywordBhikkhuBodhi{「不從眼想像」}(does not conceive from the eyes)。按:《顯揚真義》說,不思量我已從眼出去(niggato);我的任何執持(mama kiñcanapalibodho, 附屬物)已從眼出去;其它的已從眼出去;其它的任何執持已從眼出去,任一渴愛、慢、見的思量也不生起(\suttaref{SN.35.30})。
\item\subnoteref{966.1}\NoteSubKeyHead{(2)}\NoteKeywordNikayaHead{「從地思量」}(pathavito maññati),菩提比丘長英譯為「他想像[他自己]從地[分離]」(he conceives [himself apart] from earth)。
\stopitemgroup

\startitemgroup[noteitems]
\item\subnoteref{967.0}\NoteKeywordAgamaHead{「沐浴/澡浴、摩飾(SA);按摩、澡浴(MA);摩捫擁護(DA)」},南傳作\NoteKeywordNikaya{「塗身、按摩」}(ucchādanaparimaddana),菩提比丘長老英譯為\NoteKeywordBhikkhuBodhi{「摩擦與按壓」}(to rubbing and pressing, SN),坦尼沙羅比丘長老英譯為「摩擦、按壓」(rubbing, pressing, MN),智髻比丘長老英譯為「被磨損與磨耗」(to being worn and rubbed away, MN),Maurice Walshe先生英譯為「容易受傷和磨損」(liable to be injured and abraded, DN)。按:「塗身、按摩」,另作「削減/破壞、磨碎/磨耗」,PTS《巴英辭典》說,在anicc˚ -- dhamma的複合詞中應作「腐蝕、朽壞(erosion,decay)、磨損(abrasion)」,但《顯揚真義》等說,為了惡臭的除去目的[塗]以身體香膏為塗身法(duggandhavighātatthāya tanuvilepanena ucchādanadhammo),為了肢體肢節病痛的除去目的以蜂蜜按摩為按摩法(aṅgapaccaṅgābādhavinodanatthāya khuddakasambāhanena parimaddanadhammo, \suttaref{SN.35.103}/\ccchref{DN.2}{https://agama.buddhason.org/DN/dm.php?keyword=2}),與北傳經文相應,今準此譯。
\stopitemgroup

\startitemgroup[noteitems]
\item\subnoteref{968.0}\NoteKeywordAgamaHead{「覺知心/心覺知(SA)」},南傳作\NoteKeywordNikaya{「經驗著心」}(Cittapaṭisaṃvedī),菩提比丘長老英譯為\NoteKeywordBhikkhuBodhi{「體驗著心」}(experiencing the mind)。按:「經驗著」(paṭisaṃvedī,另譯為「感受著」),形容詞,但以如現在分詞的動作形容詞解讀,《清淨道論》說,這是因四種禪的(catunnaṃ jhānānaṃ vasena, 8.235)。
\stopitemgroup

\startitemgroup[noteitems]
\item\subnoteref{969.0}我聽到「這世尊所說的、阿羅漢所說的」(Vuttañhetaṃ bhagavatā, vuttamarahatāti me sutaṃ),原應譯為「這被世尊說、被阿羅漢說」被我聽到,今依我們的語言習慣轉譯。
\stopitemgroup

\startitemgroup[noteitems]
\item\subnoteref{970.0}\NoteKeywordAgamaHead{「為無目/能為無目(SA);無慧眼(GA)」},南傳作\NoteKeywordNikaya{「不作眼」}(acakkhukaraṇā, acakkhukaraṇo,逐字譯為「非+眼+所作」),菩提比丘長老英譯為\NoteKeywordBhikkhuBodhi{「造成缺乏見解」}(causing lack of vision, SN),或「導向失去見解」(leads to loss of vision, AN)。按:《顯揚真義》以「慧眼之不作」(paññācakkhussa akaraṇā, \suttaref{SN.46.40}),《滿足希求》以「不作慧眼」(Paññācakkhuṃ na karotīti, \ccchref{AN.3.72}{https://agama.buddhason.org/AN/an.php?keyword=3.72})解說。
\stopitemgroup

\startitemgroup[noteitems]
\item\subnoteref{971.0}\NoteKeywordAgamaHead{「具足奉事(SA);已報佛恩(GA)」},南傳作\NoteKeywordNikaya{「世尊已被我尊敬」}(pariciṇṇo me bhagavā),智髻比丘長老英譯為「世尊已被我崇拜」(The Blessed One has been worshipped by me)。按:「尊敬」(pariciṇṇo),另譯為「侍奉」,《破斥猶豫》等說,眾生、有學尊敬(paricaranti,現在式)世尊,以諸漏的滅盡世尊已被尊敬(pariciṇṇo hoti,完成式, \ccchref{MN.73}{https://agama.buddhason.org/MN/dm.php?keyword=73}/\ccchref{AN.6.103}{https://agama.buddhason.org/AN/an.php?keyword=6.103})。
\stopitemgroup

\startitemgroup[noteitems]
\item\subnoteref{972.0}\NoteKeywordNikayaHead{「入火界定後」}(tejodhātuṃ samāpajjitvā),Maurice Walshe先生英譯為「進入火元素」(entered into the fire-element, DN),菩提比丘長老英譯為\NoteKeywordBhikkhuBodhi{「進入火元素的默想」}(having entered into meditation on the fire element, SN)。按:《顯揚真義》說,他做了火遍處的準備(tejokasiṇaparikammaṃ)後,出基礎禪(pādakajjhānato),決意「讓火焰從身體出來。」以決意心的威力(adhiṭṭhānacittānubhāvena, \suttaref{SN.6.5}),火焰從他全身出來,這樣名為火界進入者,他在那裡達成。
\stopitemgroup

\startitemgroup[noteitems]
\item\subnoteref{973.0}\NoteKeywordAgamaHead{「如雞一飛(MA);雞鳴相聞(DA)」},南傳作\NoteKeywordNikaya{「雞落在一起的」}(kukkuṭasampātikā, kukkuṭasaṃpātikā),Maurice Walshe先生英譯為「將只是一隻公雞的飛行[距離]」(will be but a cock's flight, \ccchref{DN.26}{https://agama.buddhason.org/DN/dm.php?keyword=26}),菩提比丘長老英譯為\NoteKeywordBhikkhuBodhi{「公雞能在其間飛行」}(cocks could fly between them, \ccchref{AN.3.57}{https://agama.buddhason.org/AN/an.php?keyword=3.57})。按:《吉祥悅意》作kukkuṭasampāto,並以「從一個村落的屋頂飛落到其他村落的屋頂」解說(\ccchref{DN.26}{https://agama.buddhason.org/DN/dm.php?keyword=26}),《滿足希求》作kukkuṭasampādikā,並以「雞從村落中走到另一個村落」解說(\ccchref{AN.3.57}{https://agama.buddhason.org/AN/an.php?keyword=3.57}),皆表示村落間很近。
\stopitemgroup

\startitemgroup[noteitems]
\item\subnoteref{974.0}\NoteKeywordAgamaHead{「意行(MA)」},南傳作\NoteKeywordNikaya{「意的近伺察」}(manopavicārā),智髻比丘長老英譯為「心理的探索」(Mental exploration, MN),菩提比丘長老英譯為\NoteKeywordBhikkhuBodhi{「心理的檢查」}(mental examinations, AN),並解說,以尋產生探索或檢查目標,隨後伺結合尋。按:《破斥猶豫》以「尋伺」(vitakkavicārā, \ccchref{MN.137}{https://agama.buddhason.org/MN/dm.php?keyword=137})解說,《滿足希求》說,以尋伺之足(vitakkavicārapādehi)在十八處之意的近伺察(manassa upavicārā, \ccchref{AN.3.62}{https://agama.buddhason.org/AN/an.php?keyword=3.62})。
\stopitemgroup

\startitemgroup[noteitems]
\item\subnoteref{975.0}\NoteSubKeyHead{(1)}\NoteKeywordAgamaHead{「他走到地獄……」,《滿足希求》說,以應該去地獄的業未斷的狀態後續而非直接地走(aparāparaṃ gacchati, na anantarameva, \ccchref{AN.4.123}{https://agama.buddhason.org/AN/an.php?keyword=4.123})。菩提比丘長老依其《阿毘達磨概要精解5章40節》說,從色界無色界死者不立即往生三惡趣(按:無因心),但\ccchref{MA.114}{https://agama.buddhason.org/MA/dm.php?keyword=114}說優陀羅羅摩子「生非有想非無想天中,彼壽盡已,復來此間,生於狸中」},經說與論說顯然不同。
\item\subnoteref{975.1}\NoteSubKeyHead{(2)}\NoteKeywordNikayaHead{「有趣處、往生」},《滿足希求》說,有學的聖弟子不因再生而降到下面處(paṭisandhivasena heṭṭhā anotaritvā, \ccchref{AN.4.123}{https://agama.buddhason.org/AN/an.php?keyword=4.123}),就在那個色界之上的第二、第三等某個梵天世界般涅槃(按:這與經說「七有天人往來」的情況不同),而一般人則到地獄等,這是差異。
\stopitemgroup

\startitemgroup[noteitems]
\item\subnoteref{976.0}\NoteKeywordNikayaHead{「正智」}(sammāñāṇa),菩提比丘長老英譯為\NoteKeywordBhikkhuBodhi{「正確的理解」}(right knowledge)。按:《顯揚真義》以「正省察智」(sammāpaccavekkhaṇā, \suttaref{SN.14.29}),《破斥猶豫》以「十九省察智」(ekūnavīsatibhedaṃ paccavekkhaṇāñāṇaṃ, \ccchref{MN.18}{https://agama.buddhason.org/MN/dm.php?keyword=18})解說,或說,當以道之正定住立時生起道之省察智的正智,當以果之正定住立時生起果之省察智的正智,當住立於道之省察智的正智時生起道之正解脫,當住立於果之省察智的正智時生起果之正解脫(phalapaccavekkhaṇasammāñāṇe ṭhitassa phalasammāvimutti pahotīti, \ccchref{MN.117}{https://agama.buddhason.org/MN/dm.php?keyword=117}),《滿足希求》以「果智」(phalañāṇena, \ccchref{AN.3.146}{https://agama.buddhason.org/AN/an.php?keyword=3.146})解說,長老說,正智能以其已破壞所有污穢之省察智(reviewing knowledge)被識別,而正解脫能以其從所有污穢釋放的經驗被識別(MN note 1112)。依這些解說,無學「正智」應即「解脫知見」。
\stopitemgroup

\startitemgroup[noteitems]
\item\subnoteref{977.0}\NoteKeywordNikayaHead{「足夠信者、足夠情愛者」}(saddhāmattaṃ pemamattaṃ,另譯為「信程度的、情愛程度的」),菩提比丘長老英譯為\NoteKeywordBhikkhuBodhi{「足夠的信、足夠的摯愛」}(sufficient faith, sufficient devotion, SN),智髻比丘長老英譯為「足夠的信、足夠的愛」sufficient faith, sufficient love, MN)。按:《顯揚真義》以「在隨信行道上的人」(saddhānusārimaggaṭṭhapuggalaṃ, \suttaref{SN.55.24})解說,《破斥猶豫》說,他們是能得到觀的個人之意(Te vipassakapuggalā adhippetā, \ccchref{MN.22}{https://agama.buddhason.org/MN/dm.php?keyword=22})。
\stopitemgroup

\startitemgroup[noteitems]
\item\subnoteref{978.0}\NoteKeywordAgamaHead{「利智慧/明利智慧/深利智慧(SA);利智(SA/GA/AA);利慧(MA)」},南傳作\NoteKeywordNikaya{「利慧」}(tikkhapañño,另譯為「銳利慧;捷疾慧」),菩提比丘長老英譯為\NoteKeywordBhikkhuBodhi{「尖銳的智慧」}(sharp wisdom; sharpness of wisdom)。按:\ccchref{SA.276}{https://agama.buddhason.org/SA/dm.php?keyword=276}以分解牛隻的利刀為「利智慧」(\ccchref{MN.146}{https://agama.buddhason.org/MN/dm.php?keyword=146}作「聖慧(ariyā paññā)」),以其能「斷截一切結、縛、使、漏、上煩惱、纏」,《顯揚真義》等以「急速地切斷諸污染」(Khippaṃ kilese chindatīti, \suttaref{SN.2.29}/\ccchref{MN.111}{https://agama.buddhason.org/MN/dm.php?keyword=111}/\ccchref{DN.30}{https://agama.buddhason.org/DN/dm.php?keyword=30}/\ccchref{AN.1.584}{https://agama.buddhason.org/AN/an.php?keyword=1.584})解說。
\stopitemgroup

\startitemgroup[noteitems]
\item\subnoteref{979.0}\NoteKeywordAgamaHead{「無我處所及事,都無所有(SA);離我、我所,真實無我(GA);我非為他而有所為,亦非自為而有所為/我無父母,非父母有(MA)」},南傳作\NoteKeywordNikaya{「我在任何地方不是任何人的任何事物;在任何地方任何事物也不是我的什麼」}(nāhaṃ kvacani kassaci kiñcanatasmiṃ, na ca mama kvacani kismiñci kiñcanaṃ natthī'ti-MN, nāhaṃ kvacani kassaci kiñcanatasmiṃ na ca mama kvacani katthaci kiñcanatatthīti-AN),智髻比丘長老英譯為「我在任何地方不屬於任何人,也不在任何地方有任何東西屬於我」(I am not anything belonging to anyone anywhere, nor is there anything belonging to me in anyone anywhere, MN),菩提比丘長老英譯為\NoteKeywordBhikkhuBodhi{「我在任何地方不是任何人的所有物,在任何地方也沒有任何東西是我的」}(I am not anywhere the belonging of anyone, nor is there anywhere anything in any place that is mine, AN)。按:《破斥猶豫》等以四邊空(catukoṭikā suññatā)分別解說這四句:任何地方都不見[自己的]真我(kvaci attānaṃ na passati, \ccchref{MN.106}{https://agama.buddhason.org/MN/dm.php?keyword=106}/\ccchref{AN.4.185}{https://agama.buddhason.org/AN/an.php?keyword=4.185})、不見[自己]被當成我的其他任何存在(舉例兄弟、同伴、資助)之真我、不見任何其他之真我、不見我的其他任何存在之真我。《滿足希求》對尼乾陀外道所說則解說為:任何人、任何地方都是障礙,而我沒有障礙(palibodho na homīti)、我是切斷障礙者(chinnapalibodhohamasmīti, \ccchref{AN.3.71}{https://agama.buddhason.org/AN/an.php?keyword=3.71})。
\stopitemgroup

\startitemgroup[noteitems]
\item\subnoteref{980.0}\NoteKeywordAgamaHead{「生聞(SA/MA);生漏(AA)」},南傳作\NoteKeywordNikaya{「若奴索尼」}(jāṇussoṇi),\ccchref{MN.99}{https://agama.buddhason.org/MN/dm.php?keyword=99}說他是憍薩羅五大富有婆羅門之一,《顯揚真義》說,「若奴索尼」以內處而這麼得名,是八億財富的大祭司(jāṇussoṇīti ṭhānantaravasena evaṃladdhanāmo asītikoṭivibhavo mahāpurohito, \suttaref{SN.12.47})」,祭司(purohito),另譯為「輔相;國師」,其註疏說,「若奴索尼」這麼得名,是從國王面前獲得的名字(rañño santikā adhigatanāmo),《破斥猶豫》說,「若奴索尼」這不是他父母給的名字,而是內處受領獲得的(ṭhānantarapaṭilābhaladdhaṃ),若奴索尼處這是祭司處的名字(Jāṇussoṇiṭṭhānaṃ kira nāmetaṃ purohitaṭṭhānaṃ),那是國王給與他的(taṃ tassa raññā dinnaṃ, \ccchref{MN.4}{https://agama.buddhason.org/MN/dm.php?keyword=4}),《滿足希求》說,若奴索尼內處是一個內處的名字(jāṇussoṇiṭhānantaraṃ kira nāmekaṃ ṭhānantaraṃ, \ccchref{AN.2.17}{https://agama.buddhason.org/AN/an.php?keyword=2.17}),其註疏說,「一個內處」即一個祭司處(Ekaṃ ṭhānantaranti ekaṃ purohitaṭṭhānaṃ)。
\stopitemgroup

\startitemgroup[noteitems]
\item\subnoteref{981.0}\NoteKeywordAgamaHead{「如是法(SA/DA)」},南傳作\NoteKeywordNikaya{「這樣法者」}(evaṃdhammā),菩提比丘長老英譯為\NoteKeywordBhikkhuBodhi{「屬於這樣的特質」}(of such qualities, SN),智髻比丘長老英譯為「他們[定]的狀態是這樣的」(their state [of concentration] was thus, MN),Maurice Walshe先生英譯為「這樣是他們的教導」(such was their teaching, \ccchref{DN.28}{https://agama.buddhason.org/DN/dm.php?keyword=28}),T.W. Rhys Davids先生英譯為「他們的教義」(their doctrines, DN)。按:\ccchref{AA.48.4}{https://agama.buddhason.org/AA/dm.php?keyword=48.4}作「三昧」,《顯揚真義》等以「伴隨定的法(狀態)」(samādhipakkhā dhammā)、「以道定(maggasamādhinā)、果定(phalasamādhinā)、世間出世間定這樣的定」(\suttaref{SN.47.12}/\ccchref{MN.123}{https://agama.buddhason.org/MN/dm.php?keyword=123}/\ccchref{DN.14}{https://agama.buddhason.org/DN/dm.php?keyword=14})解說,《破斥猶豫》以「就其相應的定蘊與慧蘊」(taṃsampayuttameva samādhikkhandhaṃ paññākkhandhañca, \ccchref{MN.7}{https://agama.buddhason.org/MN/dm.php?keyword=7})解說「這樣的法,這樣的慧」。
\stopitemgroup

\startitemgroup[noteitems]
\item\subnoteref{982.0}\NoteKeywordAgamaHead{「如是住(SA/DA);如是解脫堂(DA)」},南傳作\NoteKeywordNikaya{「這樣住處者」}(evaṃvihārī),菩提比丘長老英譯為\NoteKeywordBhikkhuBodhi{「屬於這樣的住處」}(of such dwellings, SN),智髻比丘長老英譯為「他們的[入定]住處是這樣的」(their abiding [in attainments] was thus, MN),Maurice Walshe先生英譯為「他們這樣的方式」(such their way, DN),T.W. Rhys Davids先生英譯為「他們的生活方式」(their mode of life, DN)。按:《顯揚真義》等以「對伴隨定的法所掌握狀態之住處」(amādhipakkhānaṃ dhammānaṃ gahitattā vihāro)、「滅等至住處」(nirodhasamāpattivihārino, \suttaref{SN.47.12}/\ccchref{MN.123}{https://agama.buddhason.org/MN/dm.php?keyword=123}/\ccchref{DN.14}{https://agama.buddhason.org/DN/dm.php?keyword=14}),《破斥猶豫》以「以果等至住處有樂住處」(phalasamāpattivihārena phāsuvihāro hoti, \ccchref{MN.68}{https://agama.buddhason.org/MN/dm.php?keyword=68})解說。
\stopitemgroup

\startitemgroup[noteitems]
\item\subnoteref{983.0}\NoteKeywordAgamaHead{「如是解脫(SA/DA);如是解(DA)」},南傳作\NoteKeywordNikaya{「這樣解脫者」}(evaṃvimuttā),菩提比丘長老英譯為\NoteKeywordBhikkhuBodhi{「屬於這樣的釋放」}(of such liberation, SN),智髻比丘長老英譯為「他們的釋放是這樣的」(their deliverance was thus, MN)。按:《顯揚真義》等以「鎮伏解脫(vikkhambhanavimutti-八等至從蓋解脫)、彼分解脫(tadaṅgavimutti-無常隨看等從恆常想等解脫)、斷除解脫(samucchedavimutti-四聖道從污染解脫)、止息/寧靜解脫(paṭippassaddhivimutti-四沙門果)、出離解脫(nissaraṇavimuttīti -涅槃, \suttaref{SN.47.12}/\ccchref{MN.123}{https://agama.buddhason.org/MN/dm.php?keyword=123}/\ccchref{DN.14}{https://agama.buddhason.org/DN/dm.php?keyword=14})」五種解脫解說。
\stopitemgroup

\startitemgroup[noteitems]
\item\subnoteref{984.0}\NoteKeywordNikayaHead{「法智」}(dhamme ñāṇaṃ,在法上的智),菩提比丘長老英譯為\NoteKeywordBhikkhuBodhi{「法則之理解」}(knowledge of the principle, SN),Maurice Walshe先生英譯為「法的理解」(knowledge of Dhamma, DN)。按:《顯揚真義》說,「法」指「四諦法(catusaccadhamma)、道智法(maggañāṇadhamma)」,法智是道智(Dhamme ñāṇanti maggañāṇaṃ),在此經中被說是為了諸漏已盡的有學地(Imasmiṃ sutte khīṇāsavassa sekkhabhūmi kathitā hoti, \suttaref{SN.12.33}),《吉祥悅意》引《分別論》「關於四道、四果之智」(Catūsu maggesu catūsu phalesu paññā dhamme ñāṇaṃ, vibha. 16.796)解說為「以一洞察而在四諦法上之智、在四諦中心的滅諦法上之智」(ekapaṭivedhavasena catusaccadhamme ñāṇaṃ catusaccabbhantare nirodhasacce dhamme ñāṇañca, \ccchref{DN.33}{https://agama.buddhason.org/DN/dm.php?keyword=33})。
\stopitemgroup

\startitemgroup[noteitems]
\item\subnoteref{985.0}\NoteKeywordAgamaHead{「等智(DA)」},南傳作\NoteKeywordNikaya{「類比智」}(anvaye ñāṇaṃ,在類比上的智,古譯為「類智」),菩提比丘長老英譯為\NoteKeywordBhikkhuBodhi{「蘊含的理解」}(knowledge of entailment, SN),Maurice Walshe先生英譯為「所有與之一致的理解」(knowledge of what is consonant with it, DN)。按:《顯揚真義》以「這是省察智」(paccavekkhaṇañāṇassetaṃ, \suttaref{SN.12.33})解說,長老認為,這裡指基於任何一對所給的因素間可操作條件的直接辨別而對過去和未來的推論(is an inference extended to past and future, based on the immediate discernment of the conditionality operative between any given pair of factors)。《吉祥悅意》引《分別論》之「他以已理解、已獲得見、已知、已深入此法而導於過去、未來之推論」(so iminā dhammena ñātena diṭṭhena pattena viditena pariyogāḷhena atītānāgatena nayaṃ neti, vibha. 16.796)解說為「從看見現量的這四諦後,依此,這樣,過去、未來就這五蘊為苦諦,就這渴愛為集諦,就這滅為滅諦,就這道為道諦,這樣,為其智之從屬智(tassa ñāṇassa anugatiyaṃ ñāṇaṃ, \ccchref{DN.33}{https://agama.buddhason.org/DN/dm.php?keyword=33})」。
\stopitemgroup

\startitemgroup[noteitems]
\item\subnoteref{986.0}\NoteKeywordNikayaHead{「無劣等外表看見者」}(akhuddāvakāso dassanāya),智髻比丘長老英譯為「直得注意的可看」(remarkable to behold, MN),Maurice Walshe先生英譯為「沒有拙劣的外表」(of no mean appearance, DN)。按:《破斥猶豫》等以「先生身體看見的外表不是小的,是大的,你的全部肢節是全都能被看見的(英俊的),那些也就是大的」(bhoto sarīre dassanassa okāso na khuddako mahā, sabbāneva te aṅgapaccaṅgāni dassanīyāneva, tāni cāpi mahantānevā’’ti, \ccchref{MN.95}{https://agama.buddhason.org/MN/dm.php?keyword=95}/\ccchref{DN.4}{https://agama.buddhason.org/DN/dm.php?keyword=4})解說。
\stopitemgroup

\startitemgroup[noteitems]
\item\subnoteref{987.0}\NoteKeywordAgamaHead{「諸有量業(SA)」},南傳作\NoteKeywordNikaya{「所作的有量業」}(pamāṇakataṃ kammaṃ,另譯為「作為衡量的業」),菩提比丘長老英譯為\NoteKeywordBhikkhuBodhi{「可衡量的業」}(measurable kamma, AN),或「任何已做的業」(any limited kamma that was done, SN),智髻比丘長老英譯為「有限行為」(limiting action, MN)。按:所作的有量業,《顯揚真義》等以「欲的行境(kāmāvacaraṃ-欲界)」解說,所作的無量業(appamāṇakataṃ kammaṃ)則以「色、無色的行境(rūpārūpāvacaraṃ-色、無色界, \suttaref{SN.42.8}/\ccchref{MN.99}{https://agama.buddhason.org/MN/dm.php?keyword=99}/\ccchref{DN.13}{https://agama.buddhason.org/DN/dm.php?keyword=13}/\ccchref{AN.10.219}{https://agama.buddhason.org/AN/an.php?keyword=10.219})」解說。
\stopitemgroup

\startitemgroup[noteitems]
\item\subnoteref{988.0}\NoteKeywordAgamaHead{「復道害村(SA);至他巷劫/斷截王路(MA);斷道為惡(DA)」},南傳作\NoteKeywordNikaya{「也攔路搶劫」}(paripanthepi tiṭṭhanti,直譯為「在路上-站立」),菩提比丘長老英譯為\NoteKeywordBhikkhuBodhi{「在公路埋伏」}(ambush highways),Maurice Walshe先生英譯為「犯搶劫罪」(commits robbery, DN)。按:《破斥猶豫》以「以五六十之多[的人]包圍後取命、搬運」(paṇṇāsamattāpi saṭṭhimattāpi parivāretvā jīvaggāhaṃ gahetvā āharāpenti, \ccchref{MN.13}{https://agama.buddhason.org/MN/dm.php?keyword=13})解說,「也攔路搶劫」,《顯揚真義》等以「在路上站立搶奪來去者」(āgatāgatānaṃ acchindanatthaṃ magge tiṭṭhato, \suttaref{SN.24.6}/\ccchref{DN.2}{https://agama.buddhason.org/DN/dm.php?keyword=2}/\ccchref{AN.3.51}{https://agama.buddhason.org/AN/an.php?keyword=3.51}),《破斥猶豫》以「作路上掠奪業」(panthadūhanakammaṃ karonti, \ccchref{MN.13}{https://agama.buddhason.org/MN/dm.php?keyword=13})解說。又,「也作盜匪」(ekāgārikampi karonti),《顯揚真義》等以「包圍一家後掠奪」(ekameva gharaṃ parivāretvā vilumpanaṃ, \suttaref{SN.24.6}/\ccchref{DN.2}{https://agama.buddhason.org/DN/dm.php?keyword=2}/\ccchref{AN.3.51}{https://agama.buddhason.org/AN/an.php?keyword=3.51})。
\stopitemgroup

\startitemgroup[noteitems]
\item\subnoteref{989.0}\NoteKeywordAgamaHead{「第七仙(SA/AA);七仙(MA)」},南傳作\NoteKeywordNikaya{「最上仙人」}(Isisattamassa,另譯為「第七仙的」),智髻比丘長老英譯為「最好的先知」(The best of seers)。按:這裡的「仙」指「佛」,《破斥猶豫》說,取毘婆尸[佛]開頭的六仙,[釋迦牟尼佛]為第七仙(vipassiādayo cha isayo upādāya sattamassa, \ccchref{MN.56}{https://agama.buddhason.org/MN/dm.php?keyword=56}),《勝義光明》說,世尊為第七仙(bhagavā isi ca sattamo, Sn.2.12)。菩提比丘長老解說,這裡的sattama更可能是「最上的;最善的」之意思,《顯揚真義》就這樣解說(譯按:sattamako isi, \suttaref{SN.8.8}),今依後句「到達最高的境界(梵)」(brahmapattassa)之對稱採此譯法。
\stopitemgroup

\startitemgroup[noteitems]
\item\subnoteref{990.0}\NoteSubKeyHead{(1)}\NoteKeywordAgamaHead{「坐禪(SA/MA);坐(AA)」},南傳作\NoteKeywordNikaya{「(以)安坐」}(nisajjāya,另譯為「坐下;坐禪」),菩提比丘長老英譯為\NoteKeywordBhikkhuBodhi{「坐著;就坐」}(sitting)。
\item\subnoteref{990.1}\NoteSubKeyHead{(2)}\NoteKeywordAgamaHead{「陰障(SA/GA);諸障礙法(MA);陰蓋(AA)」},南傳作\NoteKeywordNikaya{「從障礙法」}(āvaraṇīyehi dhammehi),菩提比丘長老英譯為\NoteKeywordBhikkhuBodhi{「屬於妨礙的狀態」}(of obstructive states)。按:《滿足希求》以「五蓋法」(pañcahi nīvaraṇehi dhammehi)解說,因為諸蓋障礙心後住立,因此被稱為障礙法(Nīvaraṇāni hi cittaṃ āvaritvā tiṭṭhanti, tasmā āvaraṇīyā dhammāti vuccanti, \ccchref{AN.3.16}{https://agama.buddhason.org/AN/an.php?keyword=3.16})。又,陰;蘊也;聚集義。
\stopitemgroup

\startitemgroup[noteitems]
\item\subnoteref{991.0}\NoteKeywordNikayaHead{「摩竭魚」}(makara, maṅkara,另譯作「大魚,海怪,劍魚(旗魚)」),《一切經音義》說:摩伽羅魚(亦云摩竭魚,正言麼迦羅魚,此云鯨魚也),摩竭魚(此云大體也,謂即此方巨鰲魚,其兩目如日,張口如𡼏谷,吞舟,光出濆流如潮若欱水,如壑高下,如山大者,可長二百里也)……,摩竭魚(䖍𦾨反,大魚名也,從立,經文從木作楬,非也)。
\stopitemgroup

\startitemgroup[noteitems]
\item\subnoteref{992.0}\NoteKeywordAgamaHead{「高廣大床(MA/DA);高廣大牀/高床(DA);高廣之床(AA)」},南傳作\NoteKeywordNikaya{「高床、大床」}(uccāsayanamahāsayanaṃ),菩提比丘長老英譯為\NoteKeywordBhikkhuBodhi{「高且豪華的床」}(high and luxurious beds)。按:\ccchref{DA.21}{https://agama.buddhason.org/DA/dm.php?keyword=21}、\ccchref{DN.1}{https://agama.buddhason.org/DN/dm.php?keyword=1}等有詳說,其中「長椅」(pallaṅko)有說是種有雕飾的床座。《滿足希求》說,這裡超越分量(atikkantappamāṇaṃ)名為高床,廣長的、不適當的物品(āyatavitthataṃ akappiyabhaṇḍaṃ, \ccchref{AN.3.71}{https://agama.buddhason.org/AN/an.php?keyword=3.71})名為大床,並列舉一些裝飾情況解說,《顯揚真義》以「不適當的鋪開之物」(akappiyattharaṇaṃ,  \suttaref{SN.56.82})解說大床,顯然「大(廣)」是指豪華而非寬度,《四分律》說「象牙、雜寶高廣大床,種種文繡被褥,及與雜色諸皮」,《十誦律》說,佛陀斥長老闡那用「高廣好床」而說「若比丘欲作床者,當應量作。量者,足高八指,除入梐。過是作者,波逸提。……高八指者,佛言:用我八指,第三分入梐。」《薩婆多毘尼毘婆沙》說:「高廣床者,以生憍慢故,木床高大悉犯,俗人八戒同是也。八指者,一指二寸也。」
\stopitemgroup

\startitemgroup[noteitems]
\item\subnoteref{993.0}\NoteKeywordAgamaHead{「洄澓(SA);迴轉(AA)」},南傳作\NoteKeywordNikaya{「漩渦」}(āvaṭṭa),菩提比丘長老英譯為\NoteKeywordBhikkhuBodhi{「漩渦」}(whirlpools; eddies)。按:《顯揚真義》以「污染的執見」(Kilesagāhehi, \suttaref{SN.35.228})解說,與\ccchref{AA.43.3}{https://agama.buddhason.org/AA/dm.php?keyword=43.3}用於譬喻「疑邪」相類,\ccchref{SA.1174}{https://agama.buddhason.org/SA/dm.php?keyword=1174}用於譬喻「還戒退轉」(還俗),\suttaref{SN.35.241}、\ccchref{MN.67}{https://agama.buddhason.org/MN/dm.php?keyword=67}、\ccchref{AN.4.122}{https://agama.buddhason.org/AN/an.php?keyword=4.122}均用於譬喻「五種欲」。
\stopitemgroup

\startitemgroup[noteitems]
\item\subnoteref{994.0}\NoteKeywordAgamaHead{「便生母胎/受胎/入於母胎(MA)」},南傳作\NoteKeywordNikaya{「有胎的下生」}(gabbhassa avakkanti hotīti),智髻比丘長老英譯為「下來形成胎」(the descent of an embryo comes about, MN),菩提比丘長老英譯為\NoteKeywordBhikkhuBodhi{「發生[未來]胚胎的下降」}(the descent of a [future] embryo occurs, AN)。按:\ccchref{SA.360}{https://agama.buddhason.org/SA/dm.php?keyword=360}等作「入於名色」,\suttaref{SN.12.39}等作「有名色的下生」,《破斥猶豫》以「生起」(nibbatti, \ccchref{MN.38}{https://agama.buddhason.org/MN/dm.php?keyword=38}),《滿足希求》以「進入、生起、出現」(okkanti nibbatti pātubhāvo, \ccchref{AN.3.62}{https://agama.buddhason.org/AN/an.php?keyword=3.62})解說「下生」,以「母胎」(mātukucchi, \ccchref{MN.38}{https://agama.buddhason.org/MN/dm.php?keyword=38})解說「胎」。
\stopitemgroup

\startitemgroup[noteitems]
\item\subnoteref{995.0}\NoteKeywordNikayaHead{「他是善旁觀者」}(sādhukaṃ ajjhupekkhitā hoti),菩提比丘長老英譯為\NoteKeywordBhikkhuBodhi{「他以平靜仔細看」}(he closely looks on with equanimity)。按:「旁觀者」(ajjhupekkhitā),另譯為「觀察者;注意者;守護者」,《顯揚真義》等說,道之行者/舍摩他行者(pathapaṭipannaṃ/samathapaṭipannaṃ, \suttaref{SN.54.10}/\ccchref{MN.118}{https://agama.buddhason.org/MN/dm.php?keyword=118})旁觀(ajjhupekkhati-動詞)與一起隨侍(ekato upaṭṭhānaṃ)旁觀名為兩種旁觀(dvidhā ajjhupekkhati nāma),在這裡為天生的旁觀、所緣的旁觀(Tattha sahajātānampi ajjhupekkhanā hoti ārammaṇassapi ajjhupekkhanā),此處所緣旁觀是意趣(idha ārammaṇaajjhupekkhanā adhippetā)。
\stopitemgroup

\startitemgroup[noteitems]
\item\subnoteref{996.0}\NoteKeywordAgamaHead{「至冷有,無煩亦無熱(MA)」},南傳作\NoteKeywordNikaya{「已寂滅者、清涼已生者」}(nibbuto sītībhūto),智髻比丘長老英譯為「熄滅與冷卻」(extinguished, and cooled, MN),菩提比丘長老英譯為\NoteKeywordBhikkhuBodhi{「熄冷與冷卻」}(quenched and cooled, AN)。按:「至冷有」顯然為「清涼已生者」(sītībhūto,另譯為「已變成冷的」),《破斥猶豫》等以「內在燒焦污染的不存在為清涼已生者(的)」(Anto tāpanakilesānaṃ abhāvā sītalo jātoti)解說;「無煩亦無熱」看起來與「已寂滅者」(nibbuto,另譯為「已熄滅者;已到達涅槃者」)相應,《破斥猶豫》等以「一切污染的熄滅(寂滅)狀態」(Sabbakilesānaṃ nibbutattā, \ccchref{MN.51}{https://agama.buddhason.org/MN/dm.php?keyword=51}/\ccchref{AN.4.198}{https://agama.buddhason.org/AN/an.php?keyword=4.198})解說。
\stopitemgroup

\startitemgroup[noteitems]
\item\subnoteref{997.0}\NoteKeywordAgamaHead{「六見處(SA/MA)」},南傳作\NoteKeywordNikaya{「在這些地方」}(imesu ca ṭhānesu, SN),但菩提比丘長老依錫蘭本(imesu chasu ṭhānesu)英譯為「在這六種情況」(in these six cases),或「六見處」(Chayimāni……diṭṭhiṭṭhānāni, \ccchref{MN.22}{https://agama.buddhason.org/MN/dm.php?keyword=22}),菩提比丘長老英譯為\NoteKeywordBhikkhuBodhi{「六個見的立足處」}(six standpoints for views)。按:這裡的「見」指「邪見」,《顯揚真義》為見解說,而《破斥猶豫》說,見是見處,以見為所緣(diṭṭhiyā ārammaṇampi)也是見處,以見為緣也是(diṭṭhiyā paccayopi)。依\ccchref{SA.133}{https://agama.buddhason.org/SA/dm.php?keyword=133}與\suttaref{SN.24.2},六見處似乎是指「五蘊」加上「所見、所聞、所覺、所識、所得、所求、被意所隨行」合起來,依\ccchref{MA.200}{https://agama.buddhason.org/MA/dm.php?keyword=200}與\ccchref{MN.22}{https://agama.buddhason.org/MN/dm.php?keyword=22},則指「五蘊」加上「此(彼)」,「此(彼)」指的應該就是「梵」,相當於「大我(歸屬感)」之類的。
\stopitemgroup

\startitemgroup[noteitems]
\item\subnoteref{998.0}\NoteKeywordAgamaHead{「能為無智(SA);滅慧(MA)」},南傳作\NoteKeywordNikaya{「慧的滅者」}(paññānirodhiko),智髻比丘長老英譯為「妨礙慧」(it obstructs wisdom,MN),菩提比丘長老英譯為\NoteKeywordBhikkhuBodhi{「有損於慧」}(detrimental to wisdom, SN),或「缺乏理解」(lack of knowledge, AN)。
\stopitemgroup

\startitemgroup[noteitems]
\item\subnoteref{999.0}\NoteKeywordAgamaHead{「隱覆諸過惡/隱諱覆藏惡(SA)」},南傳作\NoteKeywordNikaya{「藏惡」}(makkho,另譯為「覆;掩蓋惡;偽善」),智髻比丘長老英譯為「輕蔑態度」(contempt attitude, MN),Maurice Walshe先生英譯為「卑賤」(mean, \ccchref{DN.25}{https://agama.buddhason.org/DN/dm.php?keyword=25})或「欺騙」(deceitful, \ccchref{DN.33}{https://agama.buddhason.org/DN/dm.php?keyword=33}),坦尼沙羅比丘長老英譯為「自大」(arrogance, \ccchref{AN.9.62}{https://agama.buddhason.org/AN/an.php?keyword=9.62}),菩提比丘長老英譯為\NoteKeywordBhikkhuBodhi{「詆毀」}(denigration, AN)。按:《破斥猶豫》以「在家人或出家人的所作善行之破壞」(Agāriyassa vā anagāriyassa vā sukata- karaṇavināsano, \ccchref{MN.7}{https://agama.buddhason.org/MN/dm.php?keyword=7})解說,Buddhadatta Mahāthera英巴詞典作「貶他人的價值」(depreciation of another's worth),PTS英巴詞典作「偽善或憤怒」(hypocrisy, anger),水野弘元《巴利語辭典》作「悪の覆蔵; 悪意」,今準「悪の覆蔵」譯。
\stopitemgroup

